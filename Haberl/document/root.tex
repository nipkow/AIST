\documentclass[11pt,a4paper]{article}
\usepackage[T1]{fontenc}
\usepackage{isabelle,isabellesym}

% further packages required for unusual symbols (see also
% isabellesym.sty), use only when needed

\usepackage{amssymb}

%\usepackage[only,bigsqcap,bigparallel,fatsemi,interleave,sslash]{stmaryrd}
  %for \<Sqinter>, \<Parallel>, \<Zsemi>, \<Parallel>, \<sslash>

% this should be the last package used
\usepackage{pdfsetup}

% urls in roman style, theory text in math-similar italics
\urlstyle{rm}
\isabellestyle{literal}

\begin{document}

\title{Greibach Normal Form}
\author{Alexander Haberl and Tobias Nipkow and Akihisa Yamada}
\maketitle

\begin{abstract}
This theory formalizes a method to transform a set of productions into the Greibach Normal Form (GNF) \cite{Greibach}. For purposes of this theory, we only consider one part of the GNF definition, that every production starts with a terminal. This means that the tail of a right-hand side can contain other terminals. To express this, we have defined the formalisation of $GNF\_hd$, which only checks this property.

The main idea behind this method is to bring the productions into a $triangular$ form, where nonterminal $Ai$ does not depend on nonterminals $A_i, …, A_n$. Then every $A_0$ production must already start with a terminal, and we can bring all productions into GNF by expanding the head nonterminals in order, starting with A1 productions.

A drawback of this approach is that the resulting GNF grammar can have exponential size, with respect to the number of productions, which is caused by the successive expansion of the head nonterminals.
\end{abstract}

% sane default for proof documents
\parindent 0pt\parskip 0.5ex

% generated text of all theories
%
\begin{isabellebody}%
\setisabellecontext{Lfun}%
%
\isadelimtheory
%
\endisadelimtheory
%
\isatagtheory
\isakeywordONE{theory}\isamarkupfalse%
\ Lfun\isanewline
\ \ \isakeywordTWO{imports}\ \isanewline
\ \ \ \ {\isachardoublequoteopen}Regular{\isacharminus}{\kern0pt}Sets{\isachardot}{\kern0pt}Regular{\isacharunderscore}{\kern0pt}Set{\isachardoublequoteclose}\isanewline
\ \ \ \ {\isachardoublequoteopen}Regular{\isacharminus}{\kern0pt}Sets{\isachardot}{\kern0pt}Regular{\isacharunderscore}{\kern0pt}Exp{\isachardoublequoteclose}\isanewline
\isakeywordTWO{begin}%
\endisatagtheory
{\isafoldtheory}%
%
\isadelimtheory
%
\endisadelimtheory
%
\isadelimdocument
%
\endisadelimdocument
%
\isatagdocument
%
\isamarkupsection{Definition of regular language expressions%
}
\isamarkuptrue%
%
\endisatagdocument
{\isafolddocument}%
%
\isadelimdocument
%
\endisadelimdocument
%
\begin{isamarkuptext}%
Regular language expressions%
\end{isamarkuptext}\isamarkuptrue%
\isakeywordONE{datatype}\isamarkupfalse%
\ {\isacharprime}{\kern0pt}a\ rlexp\ {\isacharequal}{\kern0pt}\ Var\ nat\ \ \ \ \ \ \ \ \ \ \ \ \ \ \ \ \ \ \ \ \ \ \ \ \ \ \isanewline
\ \ \ \ \ \ \ \ \ \ \ \ \ \ \ \ \ \ {\isacharbar}{\kern0pt}\ Const\ {\isachardoublequoteopen}{\isacharprime}{\kern0pt}a\ lang{\isachardoublequoteclose}\ \ \ \ \ \ \ \ \ \ \ \ \ \ \ \ \ \ \isanewline
\ \ \ \ \ \ \ \ \ \ \ \ \ \ \ \ \ \ {\isacharbar}{\kern0pt}\ Union\ {\isachardoublequoteopen}{\isacharprime}{\kern0pt}a\ rlexp{\isachardoublequoteclose}\ {\isachardoublequoteopen}{\isacharprime}{\kern0pt}a\ rlexp{\isachardoublequoteclose}\isanewline
\ \ \ \ \ \ \ \ \ \ \ \ \ \ \ \ \ \ {\isacharbar}{\kern0pt}\ Concat\ {\isachardoublequoteopen}{\isacharprime}{\kern0pt}a\ rlexp{\isachardoublequoteclose}\ {\isachardoublequoteopen}{\isacharprime}{\kern0pt}a\ rlexp{\isachardoublequoteclose}\ \ \ \ \ \isanewline
\ \ \ \ \ \ \ \ \ \ \ \ \ \ \ \ \ \ {\isacharbar}{\kern0pt}\ Star\ {\isachardoublequoteopen}{\isacharprime}{\kern0pt}a\ rlexp{\isachardoublequoteclose}%
\begin{isamarkuptext}%
instantiate each variable with a language%
\end{isamarkuptext}\isamarkuptrue%
\isakeywordONE{type{\isacharunderscore}{\kern0pt}synonym}\isamarkupfalse%
\ {\isacharprime}{\kern0pt}a\ valuation\ {\isacharequal}{\kern0pt}\ {\isachardoublequoteopen}nat\ {\isasymRightarrow}\ {\isacharprime}{\kern0pt}a\ lang{\isachardoublequoteclose}%
\begin{isamarkuptext}%
evaluate the regular language expression for a given valuation, yielding a language%
\end{isamarkuptext}\isamarkuptrue%
\isakeywordONE{primrec}\isamarkupfalse%
\ eval\ {\isacharcolon}{\kern0pt}{\isacharcolon}{\kern0pt}\ {\isachardoublequoteopen}{\isacharprime}{\kern0pt}a\ rlexp\ {\isasymRightarrow}\ {\isacharprime}{\kern0pt}a\ valuation\ {\isasymRightarrow}\ {\isacharprime}{\kern0pt}a\ lang{\isachardoublequoteclose}\ \isakeywordTWO{where}\isanewline
\ \ {\isachardoublequoteopen}eval\ {\isacharparenleft}{\kern0pt}Var\ n{\isacharparenright}{\kern0pt}\ v\ {\isacharequal}{\kern0pt}\ v\ n{\isachardoublequoteclose}\ {\isacharbar}{\kern0pt}\isanewline
\ \ {\isachardoublequoteopen}eval\ {\isacharparenleft}{\kern0pt}Const\ l{\isacharparenright}{\kern0pt}\ {\isacharunderscore}{\kern0pt}\ {\isacharequal}{\kern0pt}\ l{\isachardoublequoteclose}\ {\isacharbar}{\kern0pt}\isanewline
\ \ {\isachardoublequoteopen}eval\ {\isacharparenleft}{\kern0pt}Union\ f\ g{\isacharparenright}{\kern0pt}\ v\ {\isacharequal}{\kern0pt}\ eval\ f\ v\ {\isasymunion}\ eval\ g\ v{\isachardoublequoteclose}\ {\isacharbar}{\kern0pt}\isanewline
\ \ {\isachardoublequoteopen}eval\ {\isacharparenleft}{\kern0pt}Concat\ f\ g{\isacharparenright}{\kern0pt}\ v\ {\isacharequal}{\kern0pt}\ eval\ f\ v\ {\isacharat}{\kern0pt}{\isacharat}{\kern0pt}\ eval\ g\ v{\isachardoublequoteclose}\ {\isacharbar}{\kern0pt}\isanewline
\ \ {\isachardoublequoteopen}eval\ {\isacharparenleft}{\kern0pt}Star\ f{\isacharparenright}{\kern0pt}\ v\ {\isacharequal}{\kern0pt}\ star\ {\isacharparenleft}{\kern0pt}eval\ f\ v{\isacharparenright}{\kern0pt}{\isachardoublequoteclose}%
\begin{isamarkuptext}%
all variables occurring in the regular language expression%
\end{isamarkuptext}\isamarkuptrue%
\isakeywordONE{primrec}\isamarkupfalse%
\ vars\ {\isacharcolon}{\kern0pt}{\isacharcolon}{\kern0pt}\ {\isachardoublequoteopen}{\isacharprime}{\kern0pt}a\ rlexp\ {\isasymRightarrow}\ nat\ set{\isachardoublequoteclose}\ \isakeywordTWO{where}\isanewline
\ \ {\isachardoublequoteopen}vars\ {\isacharparenleft}{\kern0pt}Var\ n{\isacharparenright}{\kern0pt}\ {\isacharequal}{\kern0pt}\ {\isacharbraceleft}{\kern0pt}n{\isacharbraceright}{\kern0pt}{\isachardoublequoteclose}\ {\isacharbar}{\kern0pt}\isanewline
\ \ {\isachardoublequoteopen}vars\ {\isacharparenleft}{\kern0pt}Const\ {\isacharunderscore}{\kern0pt}{\isacharparenright}{\kern0pt}\ {\isacharequal}{\kern0pt}\ {\isacharbraceleft}{\kern0pt}{\isacharbraceright}{\kern0pt}{\isachardoublequoteclose}\ {\isacharbar}{\kern0pt}\isanewline
\ \ {\isachardoublequoteopen}vars\ {\isacharparenleft}{\kern0pt}Union\ f\ g{\isacharparenright}{\kern0pt}\ {\isacharequal}{\kern0pt}\ vars\ f\ {\isasymunion}\ vars\ g{\isachardoublequoteclose}\ {\isacharbar}{\kern0pt}\isanewline
\ \ {\isachardoublequoteopen}vars\ {\isacharparenleft}{\kern0pt}Concat\ f\ g{\isacharparenright}{\kern0pt}\ {\isacharequal}{\kern0pt}\ vars\ f\ {\isasymunion}\ vars\ g{\isachardoublequoteclose}\ {\isacharbar}{\kern0pt}\isanewline
\ \ {\isachardoublequoteopen}vars\ {\isacharparenleft}{\kern0pt}Star\ f{\isacharparenright}{\kern0pt}\ {\isacharequal}{\kern0pt}\ vars\ f{\isachardoublequoteclose}%
\begin{isamarkuptext}%
substitute each occurrence of a variable \isa{i} by the regular language expression \isa{s\ i}%
\end{isamarkuptext}\isamarkuptrue%
\isakeywordONE{primrec}\isamarkupfalse%
\ subst\ {\isacharcolon}{\kern0pt}{\isacharcolon}{\kern0pt}\ {\isachardoublequoteopen}{\isacharparenleft}{\kern0pt}nat\ {\isasymRightarrow}\ {\isacharprime}{\kern0pt}a\ rlexp{\isacharparenright}{\kern0pt}\ {\isasymRightarrow}\ {\isacharprime}{\kern0pt}a\ rlexp\ {\isasymRightarrow}\ {\isacharprime}{\kern0pt}a\ rlexp{\isachardoublequoteclose}\ \isakeywordTWO{where}\isanewline
\ \ {\isachardoublequoteopen}subst\ s\ {\isacharparenleft}{\kern0pt}Var\ n{\isacharparenright}{\kern0pt}\ {\isacharequal}{\kern0pt}\ s\ n{\isachardoublequoteclose}\ {\isacharbar}{\kern0pt}\isanewline
\ \ {\isachardoublequoteopen}subst\ {\isacharunderscore}{\kern0pt}\ {\isacharparenleft}{\kern0pt}Const\ l{\isacharparenright}{\kern0pt}\ {\isacharequal}{\kern0pt}\ Const\ l{\isachardoublequoteclose}\ {\isacharbar}{\kern0pt}\isanewline
\ \ {\isachardoublequoteopen}subst\ s\ {\isacharparenleft}{\kern0pt}Union\ f\ g{\isacharparenright}{\kern0pt}\ {\isacharequal}{\kern0pt}\ Union\ {\isacharparenleft}{\kern0pt}subst\ s\ f{\isacharparenright}{\kern0pt}\ {\isacharparenleft}{\kern0pt}subst\ s\ g{\isacharparenright}{\kern0pt}{\isachardoublequoteclose}\ {\isacharbar}{\kern0pt}\isanewline
\ \ {\isachardoublequoteopen}subst\ s\ {\isacharparenleft}{\kern0pt}Concat\ f\ g{\isacharparenright}{\kern0pt}\ {\isacharequal}{\kern0pt}\ Concat\ {\isacharparenleft}{\kern0pt}subst\ s\ f{\isacharparenright}{\kern0pt}\ {\isacharparenleft}{\kern0pt}subst\ s\ g{\isacharparenright}{\kern0pt}{\isachardoublequoteclose}\ {\isacharbar}{\kern0pt}\isanewline
\ \ {\isachardoublequoteopen}subst\ s\ {\isacharparenleft}{\kern0pt}Star\ f{\isacharparenright}{\kern0pt}\ {\isacharequal}{\kern0pt}\ Star\ {\isacharparenleft}{\kern0pt}subst\ s\ f{\isacharparenright}{\kern0pt}{\isachardoublequoteclose}%
\isadelimdocument
%
\endisadelimdocument
%
\isatagdocument
%
\isamarkupsection{Some lemmas about regular language expressions%
}
\isamarkuptrue%
%
\endisatagdocument
{\isafolddocument}%
%
\isadelimdocument
%
\endisadelimdocument
\isakeywordONE{lemma}\isamarkupfalse%
\ substitution{\isacharunderscore}{\kern0pt}lemma{\isacharcolon}{\kern0pt}\isanewline
\ \ \isakeywordTWO{assumes}\ {\isachardoublequoteopen}{\isasymforall}i{\isachardot}{\kern0pt}\ v{\isacharprime}{\kern0pt}\ i\ {\isacharequal}{\kern0pt}\ eval\ {\isacharparenleft}{\kern0pt}upd\ i{\isacharparenright}{\kern0pt}\ v{\isachardoublequoteclose}\isanewline
\ \ \isakeywordTWO{shows}\ {\isachardoublequoteopen}eval\ {\isacharparenleft}{\kern0pt}subst\ upd\ f{\isacharparenright}{\kern0pt}\ v\ {\isacharequal}{\kern0pt}\ eval\ f\ v{\isacharprime}{\kern0pt}{\isachardoublequoteclose}\isanewline
%
\isadelimproof
\ \ %
\endisadelimproof
%
\isatagproof
\isakeywordONE{using}\isamarkupfalse%
\ assms\ \isakeywordONE{by}\isamarkupfalse%
\ {\isacharparenleft}{\kern0pt}induction\ rule{\isacharcolon}{\kern0pt}\ rlexp{\isachardot}{\kern0pt}induct{\isacharparenright}{\kern0pt}\ auto%
\endisatagproof
{\isafoldproof}%
%
\isadelimproof
\isanewline
%
\endisadelimproof
\isanewline
\isakeywordONE{lemma}\isamarkupfalse%
\ substitution{\isacharunderscore}{\kern0pt}lemma{\isacharunderscore}{\kern0pt}update{\isacharcolon}{\kern0pt}\isanewline
\ \ {\isachardoublequoteopen}eval\ {\isacharparenleft}{\kern0pt}subst\ {\isacharparenleft}{\kern0pt}Var{\isacharparenleft}{\kern0pt}x\ {\isacharcolon}{\kern0pt}{\isacharequal}{\kern0pt}\ f{\isacharprime}{\kern0pt}{\isacharparenright}{\kern0pt}{\isacharparenright}{\kern0pt}\ f{\isacharparenright}{\kern0pt}\ v\ {\isacharequal}{\kern0pt}\ eval\ f\ {\isacharparenleft}{\kern0pt}v{\isacharparenleft}{\kern0pt}x\ {\isacharcolon}{\kern0pt}{\isacharequal}{\kern0pt}\ eval\ f{\isacharprime}{\kern0pt}\ v{\isacharparenright}{\kern0pt}{\isacharparenright}{\kern0pt}{\isachardoublequoteclose}\isanewline
%
\isadelimproof
\ \ %
\endisadelimproof
%
\isatagproof
\isakeywordONE{using}\isamarkupfalse%
\ substitution{\isacharunderscore}{\kern0pt}lemma{\isacharbrackleft}{\kern0pt}of\ {\isachardoublequoteopen}v{\isacharparenleft}{\kern0pt}x\ {\isacharcolon}{\kern0pt}{\isacharequal}{\kern0pt}\ eval\ f{\isacharprime}{\kern0pt}\ v{\isacharparenright}{\kern0pt}{\isachardoublequoteclose}{\isacharbrackright}{\kern0pt}\ \isakeywordONE{by}\isamarkupfalse%
\ force%
\endisatagproof
{\isafoldproof}%
%
\isadelimproof
\isanewline
%
\endisadelimproof
\isanewline
\isakeywordONE{lemma}\isamarkupfalse%
\ subst{\isacharunderscore}{\kern0pt}id{\isacharcolon}{\kern0pt}\ {\isachardoublequoteopen}eval\ {\isacharparenleft}{\kern0pt}subst\ Var\ f{\isacharparenright}{\kern0pt}\ v\ {\isacharequal}{\kern0pt}\ eval\ f\ v{\isachardoublequoteclose}\isanewline
%
\isadelimproof
\ \ %
\endisadelimproof
%
\isatagproof
\isakeywordONE{using}\isamarkupfalse%
\ substitution{\isacharunderscore}{\kern0pt}lemma{\isacharbrackleft}{\kern0pt}of\ v{\isacharbrackright}{\kern0pt}\ \isakeywordONE{by}\isamarkupfalse%
\ simp%
\endisatagproof
{\isafoldproof}%
%
\isadelimproof
\isanewline
%
\endisadelimproof
\isanewline
\isanewline
\isakeywordONE{lemma}\isamarkupfalse%
\ vars{\isacharunderscore}{\kern0pt}subst{\isacharcolon}{\kern0pt}\ {\isachardoublequoteopen}vars\ {\isacharparenleft}{\kern0pt}subst\ upd\ f{\isacharparenright}{\kern0pt}\ {\isacharequal}{\kern0pt}\ {\isacharparenleft}{\kern0pt}{\isasymUnion}x\ {\isasymin}\ vars\ f{\isachardot}{\kern0pt}\ vars\ {\isacharparenleft}{\kern0pt}upd\ x{\isacharparenright}{\kern0pt}{\isacharparenright}{\kern0pt}{\isachardoublequoteclose}\isanewline
%
\isadelimproof
\ \ %
\endisadelimproof
%
\isatagproof
\isakeywordONE{by}\isamarkupfalse%
\ {\isacharparenleft}{\kern0pt}induction\ f{\isacharparenright}{\kern0pt}\ auto%
\endisatagproof
{\isafoldproof}%
%
\isadelimproof
\isanewline
%
\endisadelimproof
\isanewline
\isakeywordONE{lemma}\isamarkupfalse%
\ vars{\isacharunderscore}{\kern0pt}subst{\isacharunderscore}{\kern0pt}upper{\isacharcolon}{\kern0pt}\ {\isachardoublequoteopen}vars\ {\isacharparenleft}{\kern0pt}subst\ upd\ f{\isacharparenright}{\kern0pt}\ {\isasymsubseteq}\ {\isacharparenleft}{\kern0pt}{\isasymUnion}x{\isachardot}{\kern0pt}\ vars\ {\isacharparenleft}{\kern0pt}upd\ x{\isacharparenright}{\kern0pt}{\isacharparenright}{\kern0pt}{\isachardoublequoteclose}\isanewline
%
\isadelimproof
\ \ %
\endisadelimproof
%
\isatagproof
\isakeywordONE{using}\isamarkupfalse%
\ vars{\isacharunderscore}{\kern0pt}subst\ \isakeywordONE{by}\isamarkupfalse%
\ force%
\endisatagproof
{\isafoldproof}%
%
\isadelimproof
\isanewline
%
\endisadelimproof
\isanewline
\isanewline
\isakeywordONE{lemma}\isamarkupfalse%
\ vars{\isacharunderscore}{\kern0pt}subst{\isacharunderscore}{\kern0pt}upd{\isacharunderscore}{\kern0pt}upper{\isacharcolon}{\kern0pt}\ {\isachardoublequoteopen}vars\ {\isacharparenleft}{\kern0pt}subst\ {\isacharparenleft}{\kern0pt}Var{\isacharparenleft}{\kern0pt}x\ {\isacharcolon}{\kern0pt}{\isacharequal}{\kern0pt}\ fx{\isacharparenright}{\kern0pt}{\isacharparenright}{\kern0pt}\ f{\isacharparenright}{\kern0pt}\ {\isasymsubseteq}\ vars\ f\ {\isacharminus}{\kern0pt}\ {\isacharbraceleft}{\kern0pt}x{\isacharbraceright}{\kern0pt}\ {\isasymunion}\ vars\ fx{\isachardoublequoteclose}\isanewline
%
\isadelimproof
%
\endisadelimproof
%
\isatagproof
\isakeywordONE{proof}\isamarkupfalse%
\isanewline
\ \ \isakeywordTHREE{fix}\isamarkupfalse%
\ y\isanewline
\ \ \isakeywordONE{let}\isamarkupfalse%
\ {\isacharquery}{\kern0pt}upd\ {\isacharequal}{\kern0pt}\ {\isachardoublequoteopen}Var{\isacharparenleft}{\kern0pt}x\ {\isacharcolon}{\kern0pt}{\isacharequal}{\kern0pt}\ fx{\isacharparenright}{\kern0pt}{\isachardoublequoteclose}\isanewline
\ \ \isakeywordTHREE{assume}\isamarkupfalse%
\ {\isachardoublequoteopen}y\ {\isasymin}\ vars\ {\isacharparenleft}{\kern0pt}subst\ {\isacharquery}{\kern0pt}upd\ f{\isacharparenright}{\kern0pt}{\isachardoublequoteclose}\isanewline
\ \ \isakeywordONE{then}\isamarkupfalse%
\ \isakeywordTHREE{obtain}\isamarkupfalse%
\ y{\isacharprime}{\kern0pt}\ \isakeywordTWO{where}\ {\isachardoublequoteopen}y{\isacharprime}{\kern0pt}\ {\isasymin}\ vars\ f\ {\isasymand}\ y\ {\isasymin}\ vars\ {\isacharparenleft}{\kern0pt}{\isacharquery}{\kern0pt}upd\ y{\isacharprime}{\kern0pt}{\isacharparenright}{\kern0pt}{\isachardoublequoteclose}\ \isakeywordONE{using}\isamarkupfalse%
\ vars{\isacharunderscore}{\kern0pt}subst\ \isakeywordONE{by}\isamarkupfalse%
\ blast\isanewline
\ \ \isakeywordONE{then}\isamarkupfalse%
\ \isakeywordTHREE{show}\isamarkupfalse%
\ {\isachardoublequoteopen}y\ {\isasymin}\ vars\ f\ {\isacharminus}{\kern0pt}\ {\isacharbraceleft}{\kern0pt}x{\isacharbraceright}{\kern0pt}\ {\isasymunion}\ vars\ fx{\isachardoublequoteclose}\ \isakeywordONE{by}\isamarkupfalse%
\ {\isacharparenleft}{\kern0pt}cases\ {\isachardoublequoteopen}x\ {\isacharequal}{\kern0pt}\ y{\isacharprime}{\kern0pt}{\isachardoublequoteclose}{\isacharparenright}{\kern0pt}\ auto\isanewline
\isakeywordONE{qed}\isamarkupfalse%
%
\endisatagproof
{\isafoldproof}%
%
\isadelimproof
\isanewline
%
\endisadelimproof
\isanewline
\isakeywordONE{lemma}\isamarkupfalse%
\ vars{\isacharunderscore}{\kern0pt}subst{\isacharunderscore}{\kern0pt}upd{\isacharunderscore}{\kern0pt}aux{\isacharcolon}{\kern0pt}\isanewline
\ \ \isakeywordTWO{assumes}\ {\isachardoublequoteopen}x\ {\isasymin}\ vars\ f{\isachardoublequoteclose}\isanewline
\ \ \isakeywordTWO{shows}\ \ \ {\isachardoublequoteopen}vars\ f\ {\isacharminus}{\kern0pt}\ {\isacharbraceleft}{\kern0pt}x{\isacharbraceright}{\kern0pt}\ {\isasymunion}\ vars\ fx\ {\isasymsubseteq}\ vars\ {\isacharparenleft}{\kern0pt}subst\ {\isacharparenleft}{\kern0pt}Var{\isacharparenleft}{\kern0pt}x\ {\isacharcolon}{\kern0pt}{\isacharequal}{\kern0pt}\ fx{\isacharparenright}{\kern0pt}{\isacharparenright}{\kern0pt}\ f{\isacharparenright}{\kern0pt}{\isachardoublequoteclose}\isanewline
%
\isadelimproof
%
\endisadelimproof
%
\isatagproof
\isakeywordONE{proof}\isamarkupfalse%
\isanewline
\ \ \isakeywordTHREE{fix}\isamarkupfalse%
\ y\isanewline
\ \ \isakeywordONE{let}\isamarkupfalse%
\ {\isacharquery}{\kern0pt}upd\ {\isacharequal}{\kern0pt}\ {\isachardoublequoteopen}Var{\isacharparenleft}{\kern0pt}x\ {\isacharcolon}{\kern0pt}{\isacharequal}{\kern0pt}\ fx{\isacharparenright}{\kern0pt}{\isachardoublequoteclose}\isanewline
\ \ \isakeywordTHREE{assume}\isamarkupfalse%
\ as{\isacharcolon}{\kern0pt}\ {\isachardoublequoteopen}y\ {\isasymin}\ vars\ f\ {\isacharminus}{\kern0pt}\ {\isacharbraceleft}{\kern0pt}x{\isacharbraceright}{\kern0pt}\ {\isasymunion}\ vars\ fx{\isachardoublequoteclose}\isanewline
\ \ \isakeywordONE{then}\isamarkupfalse%
\ \isakeywordTHREE{show}\isamarkupfalse%
\ {\isachardoublequoteopen}y\ {\isasymin}\ vars\ {\isacharparenleft}{\kern0pt}subst\ {\isacharquery}{\kern0pt}upd\ f{\isacharparenright}{\kern0pt}{\isachardoublequoteclose}\isanewline
\ \ \isakeywordONE{proof}\isamarkupfalse%
\ {\isacharparenleft}{\kern0pt}cases\ {\isachardoublequoteopen}y\ {\isasymin}\ vars\ f\ {\isacharminus}{\kern0pt}\ {\isacharbraceleft}{\kern0pt}x{\isacharbraceright}{\kern0pt}{\isachardoublequoteclose}{\isacharparenright}{\kern0pt}\isanewline
\ \ \ \ \isakeywordTHREE{case}\isamarkupfalse%
\ True\isanewline
\ \ \ \ \isakeywordONE{then}\isamarkupfalse%
\ \isakeywordTHREE{show}\isamarkupfalse%
\ {\isacharquery}{\kern0pt}thesis\ \isakeywordONE{using}\isamarkupfalse%
\ vars{\isacharunderscore}{\kern0pt}subst\ \isakeywordONE{by}\isamarkupfalse%
\ fastforce\isanewline
\ \ \isakeywordONE{next}\isamarkupfalse%
\isanewline
\ \ \ \ \isakeywordTHREE{case}\isamarkupfalse%
\ False\isanewline
\ \ \ \ \isakeywordONE{with}\isamarkupfalse%
\ as\ \isakeywordONE{have}\isamarkupfalse%
\ {\isachardoublequoteopen}y\ {\isasymin}\ vars\ fx{\isachardoublequoteclose}\ \isakeywordONE{by}\isamarkupfalse%
\ blast\isanewline
\ \ \ \ \isakeywordONE{with}\isamarkupfalse%
\ assms\ \isakeywordTHREE{show}\isamarkupfalse%
\ {\isacharquery}{\kern0pt}thesis\ \isakeywordONE{using}\isamarkupfalse%
\ vars{\isacharunderscore}{\kern0pt}subst\ \isakeywordONE{by}\isamarkupfalse%
\ fastforce\isanewline
\ \ \isakeywordONE{qed}\isamarkupfalse%
\isanewline
\isakeywordONE{qed}\isamarkupfalse%
%
\endisatagproof
{\isafoldproof}%
%
\isadelimproof
\isanewline
%
\endisadelimproof
\isanewline
\isakeywordONE{lemma}\isamarkupfalse%
\ vars{\isacharunderscore}{\kern0pt}subst{\isacharunderscore}{\kern0pt}upd{\isacharcolon}{\kern0pt}\isanewline
\ \ \isakeywordTWO{assumes}\ {\isachardoublequoteopen}x\ {\isasymin}\ vars\ f{\isachardoublequoteclose}\isanewline
\ \ \isakeywordTWO{shows}\ \ \ {\isachardoublequoteopen}vars\ {\isacharparenleft}{\kern0pt}subst\ {\isacharparenleft}{\kern0pt}Var{\isacharparenleft}{\kern0pt}x\ {\isacharcolon}{\kern0pt}{\isacharequal}{\kern0pt}\ fx{\isacharparenright}{\kern0pt}{\isacharparenright}{\kern0pt}\ f{\isacharparenright}{\kern0pt}\ {\isacharequal}{\kern0pt}\ vars\ f\ {\isacharminus}{\kern0pt}\ {\isacharbraceleft}{\kern0pt}x{\isacharbraceright}{\kern0pt}\ {\isasymunion}\ vars\ fx{\isachardoublequoteclose}\isanewline
%
\isadelimproof
\ \ %
\endisadelimproof
%
\isatagproof
\isakeywordONE{using}\isamarkupfalse%
\ assms\ vars{\isacharunderscore}{\kern0pt}subst{\isacharunderscore}{\kern0pt}upd{\isacharunderscore}{\kern0pt}upper\ vars{\isacharunderscore}{\kern0pt}subst{\isacharunderscore}{\kern0pt}upd{\isacharunderscore}{\kern0pt}aux\ \isakeywordONE{by}\isamarkupfalse%
\ blast%
\endisatagproof
{\isafoldproof}%
%
\isadelimproof
\isanewline
%
\endisadelimproof
\isanewline
\isakeywordONE{lemma}\isamarkupfalse%
\ eval{\isacharunderscore}{\kern0pt}vars{\isacharcolon}{\kern0pt}\isanewline
\ \ \isakeywordTWO{assumes}\ {\isachardoublequoteopen}{\isasymforall}i\ {\isasymin}\ vars\ f{\isachardot}{\kern0pt}\ s\ i\ {\isacharequal}{\kern0pt}\ s{\isacharprime}{\kern0pt}\ i{\isachardoublequoteclose}\isanewline
\ \ \isakeywordTWO{shows}\ {\isachardoublequoteopen}eval\ f\ s\ {\isacharequal}{\kern0pt}\ eval\ f\ s{\isacharprime}{\kern0pt}{\isachardoublequoteclose}\isanewline
%
\isadelimproof
\ \ %
\endisadelimproof
%
\isatagproof
\isakeywordONE{using}\isamarkupfalse%
\ assms\ \isakeywordONE{by}\isamarkupfalse%
\ {\isacharparenleft}{\kern0pt}induction\ f{\isacharparenright}{\kern0pt}\ auto%
\endisatagproof
{\isafoldproof}%
%
\isadelimproof
\isanewline
%
\endisadelimproof
\isanewline
\isakeywordONE{lemma}\isamarkupfalse%
\ eval{\isacharunderscore}{\kern0pt}vars{\isacharunderscore}{\kern0pt}subst{\isacharcolon}{\kern0pt}\isanewline
\ \ \isakeywordTWO{assumes}\ {\isachardoublequoteopen}{\isasymforall}i\ {\isasymin}\ vars\ f{\isachardot}{\kern0pt}\ v\ i\ {\isacharequal}{\kern0pt}\ eval\ {\isacharparenleft}{\kern0pt}upd\ i{\isacharparenright}{\kern0pt}\ v{\isachardoublequoteclose}\isanewline
\ \ \isakeywordTWO{shows}\ {\isachardoublequoteopen}eval\ {\isacharparenleft}{\kern0pt}subst\ upd\ f{\isacharparenright}{\kern0pt}\ v\ {\isacharequal}{\kern0pt}\ eval\ f\ v{\isachardoublequoteclose}\isanewline
%
\isadelimproof
%
\endisadelimproof
%
\isatagproof
\isakeywordONE{proof}\isamarkupfalse%
\ {\isacharminus}{\kern0pt}\isanewline
\ \ \isakeywordONE{let}\isamarkupfalse%
\ {\isacharquery}{\kern0pt}v{\isacharprime}{\kern0pt}\ {\isacharequal}{\kern0pt}\ {\isachardoublequoteopen}{\isasymlambda}i{\isachardot}{\kern0pt}\ if\ i\ {\isasymin}\ vars\ f\ then\ v\ i\ else\ eval\ {\isacharparenleft}{\kern0pt}upd\ i{\isacharparenright}{\kern0pt}\ v{\isachardoublequoteclose}\isanewline
\ \ \isakeywordONE{let}\isamarkupfalse%
\ {\isacharquery}{\kern0pt}v{\isacharprime}{\kern0pt}{\isacharprime}{\kern0pt}\ {\isacharequal}{\kern0pt}\ {\isachardoublequoteopen}{\isasymlambda}i{\isachardot}{\kern0pt}\ eval\ {\isacharparenleft}{\kern0pt}upd\ i{\isacharparenright}{\kern0pt}\ v{\isachardoublequoteclose}\isanewline
\ \ \isakeywordONE{have}\isamarkupfalse%
\ v{\isacharprime}{\kern0pt}{\isacharunderscore}{\kern0pt}v{\isacharprime}{\kern0pt}{\isacharprime}{\kern0pt}{\isacharcolon}{\kern0pt}\ {\isachardoublequoteopen}{\isacharquery}{\kern0pt}v{\isacharprime}{\kern0pt}\ i\ {\isacharequal}{\kern0pt}\ {\isacharquery}{\kern0pt}v{\isacharprime}{\kern0pt}{\isacharprime}{\kern0pt}\ i{\isachardoublequoteclose}\ \isakeywordTWO{for}\ i\ \isakeywordONE{using}\isamarkupfalse%
\ assms\ \isakeywordONE{by}\isamarkupfalse%
\ simp\isanewline
\ \ \isakeywordONE{then}\isamarkupfalse%
\ \isakeywordONE{have}\isamarkupfalse%
\ v{\isacharunderscore}{\kern0pt}v{\isacharprime}{\kern0pt}{\isacharprime}{\kern0pt}{\isacharcolon}{\kern0pt}\ {\isachardoublequoteopen}{\isasymforall}i{\isachardot}{\kern0pt}\ {\isacharquery}{\kern0pt}v{\isacharprime}{\kern0pt}{\isacharprime}{\kern0pt}\ i\ {\isacharequal}{\kern0pt}\ eval\ {\isacharparenleft}{\kern0pt}upd\ i{\isacharparenright}{\kern0pt}\ v{\isachardoublequoteclose}\ \isakeywordONE{by}\isamarkupfalse%
\ simp\isanewline
\isanewline
\ \ \isakeywordONE{from}\isamarkupfalse%
\ assms\ \isakeywordONE{have}\isamarkupfalse%
\ {\isachardoublequoteopen}eval\ f\ v\ {\isacharequal}{\kern0pt}\ eval\ f\ {\isacharquery}{\kern0pt}v{\isacharprime}{\kern0pt}{\isachardoublequoteclose}\ \isakeywordONE{using}\isamarkupfalse%
\ eval{\isacharunderscore}{\kern0pt}vars{\isacharbrackleft}{\kern0pt}of\ f{\isacharbrackright}{\kern0pt}\ \isakeywordONE{by}\isamarkupfalse%
\ simp\isanewline
\ \ \isakeywordONE{also}\isamarkupfalse%
\ \isakeywordONE{have}\isamarkupfalse%
\ {\isachardoublequoteopen}{\isasymdots}\ {\isacharequal}{\kern0pt}\ eval\ {\isacharparenleft}{\kern0pt}subst\ upd\ f{\isacharparenright}{\kern0pt}\ v{\isachardoublequoteclose}\isanewline
\ \ \ \ \isakeywordONE{using}\isamarkupfalse%
\ assms\ substitution{\isacharunderscore}{\kern0pt}lemma{\isacharbrackleft}{\kern0pt}OF\ v{\isacharunderscore}{\kern0pt}v{\isacharprime}{\kern0pt}{\isacharprime}{\kern0pt}{\isacharcomma}{\kern0pt}\ of\ f{\isacharbrackright}{\kern0pt}\ \isakeywordONE{by}\isamarkupfalse%
\ {\isacharparenleft}{\kern0pt}simp\ add{\isacharcolon}{\kern0pt}\ eval{\isacharunderscore}{\kern0pt}vars{\isacharparenright}{\kern0pt}\isanewline
\ \ \isakeywordONE{finally}\isamarkupfalse%
\ \isakeywordTHREE{show}\isamarkupfalse%
\ {\isacharquery}{\kern0pt}thesis\ \isakeywordONE{by}\isamarkupfalse%
\ simp\isanewline
\isakeywordONE{qed}\isamarkupfalse%
%
\endisatagproof
{\isafoldproof}%
%
\isadelimproof
%
\endisadelimproof
%
\isadelimdocument
%
\endisadelimdocument
%
\isatagdocument
%
\isamarkupsection{Monotonicity of regular language expressions%
}
\isamarkuptrue%
%
\endisatagdocument
{\isafolddocument}%
%
\isadelimdocument
%
\endisadelimdocument
\isakeywordONE{lemma}\isamarkupfalse%
\ rlexp{\isacharunderscore}{\kern0pt}mono{\isacharunderscore}{\kern0pt}aux{\isacharcolon}{\kern0pt}\isanewline
\ \ \isakeywordTWO{assumes}\ {\isachardoublequoteopen}{\isasymforall}i\ {\isasymin}\ vars\ f{\isachardot}{\kern0pt}\ v\ i\ {\isasymsubseteq}\ v{\isacharprime}{\kern0pt}\ i{\isachardoublequoteclose}\isanewline
\ \ \isakeywordTWO{shows}\ {\isachardoublequoteopen}eval\ f\ v\ {\isasymsubseteq}\ eval\ f\ v{\isacharprime}{\kern0pt}{\isachardoublequoteclose}\isanewline
%
\isadelimproof
%
\endisadelimproof
%
\isatagproof
\isakeywordONE{using}\isamarkupfalse%
\ assms\ \isakeywordONE{proof}\isamarkupfalse%
\ {\isacharparenleft}{\kern0pt}induction\ rule{\isacharcolon}{\kern0pt}\ rlexp{\isachardot}{\kern0pt}induct{\isacharparenright}{\kern0pt}\isanewline
\ \ \isakeywordTHREE{case}\isamarkupfalse%
\ {\isacharparenleft}{\kern0pt}Star\ x{\isacharparenright}{\kern0pt}\isanewline
\ \ \isakeywordONE{then}\isamarkupfalse%
\ \isakeywordTHREE{show}\isamarkupfalse%
\ {\isacharquery}{\kern0pt}case\isanewline
\ \ \ \ \isakeywordONE{by}\isamarkupfalse%
\ {\isacharparenleft}{\kern0pt}smt\ {\isacharparenleft}{\kern0pt}verit{\isacharcomma}{\kern0pt}\ best{\isacharparenright}{\kern0pt}\ eval{\isachardot}{\kern0pt}simps{\isacharparenleft}{\kern0pt}{\isadigit{5}}{\isacharparenright}{\kern0pt}\ in{\isacharunderscore}{\kern0pt}star{\isacharunderscore}{\kern0pt}iff{\isacharunderscore}{\kern0pt}concat\ order{\isacharunderscore}{\kern0pt}trans\ subsetI\ vars{\isachardot}{\kern0pt}simps{\isacharparenleft}{\kern0pt}{\isadigit{5}}{\isacharparenright}{\kern0pt}{\isacharparenright}{\kern0pt}\isanewline
\isakeywordONE{qed}\isamarkupfalse%
\ fastforce{\isacharplus}{\kern0pt}%
\endisatagproof
{\isafoldproof}%
%
\isadelimproof
%
\endisadelimproof
%
\begin{isamarkuptext}%
The actually monotonicity lemma%
\end{isamarkuptext}\isamarkuptrue%
\isakeywordONE{lemma}\isamarkupfalse%
\ rlexp{\isacharunderscore}{\kern0pt}mono{\isacharcolon}{\kern0pt}\isanewline
\ \ \isakeywordTWO{fixes}\ f\ {\isacharcolon}{\kern0pt}{\isacharcolon}{\kern0pt}\ {\isachardoublequoteopen}{\isacharprime}{\kern0pt}a\ rlexp{\isachardoublequoteclose}\isanewline
\ \ \isakeywordTWO{shows}\ {\isachardoublequoteopen}mono\ {\isacharparenleft}{\kern0pt}eval\ f{\isacharparenright}{\kern0pt}{\isachardoublequoteclose}\isanewline
%
\isadelimproof
\ \ %
\endisadelimproof
%
\isatagproof
\isakeywordONE{using}\isamarkupfalse%
\ rlexp{\isacharunderscore}{\kern0pt}mono{\isacharunderscore}{\kern0pt}aux\ \isakeywordONE{by}\isamarkupfalse%
\ {\isacharparenleft}{\kern0pt}metis\ le{\isacharunderscore}{\kern0pt}funD\ monoI{\isacharparenright}{\kern0pt}%
\endisatagproof
{\isafoldproof}%
%
\isadelimproof
%
\endisadelimproof
%
\isadelimdocument
%
\endisadelimdocument
%
\isatagdocument
%
\isamarkupsection{Continuity of regular language expressions%
}
\isamarkuptrue%
%
\endisatagdocument
{\isafolddocument}%
%
\isadelimdocument
%
\endisadelimdocument
\isakeywordONE{lemma}\isamarkupfalse%
\ langpow{\isacharunderscore}{\kern0pt}mono{\isacharcolon}{\kern0pt}\isanewline
\ \ \isakeywordTWO{fixes}\ A\ {\isacharcolon}{\kern0pt}{\isacharcolon}{\kern0pt}\ {\isachardoublequoteopen}{\isacharprime}{\kern0pt}a\ lang{\isachardoublequoteclose}\isanewline
\ \ \isakeywordTWO{assumes}\ {\isachardoublequoteopen}A\ {\isasymsubseteq}\ B{\isachardoublequoteclose}\isanewline
\ \ \isakeywordTWO{shows}\ {\isachardoublequoteopen}A\ {\isacharcircum}{\kern0pt}{\isacharcircum}{\kern0pt}\ n\ {\isasymsubseteq}\ B\ {\isacharcircum}{\kern0pt}{\isacharcircum}{\kern0pt}\ n{\isachardoublequoteclose}\isanewline
%
\isadelimproof
%
\endisadelimproof
%
\isatagproof
\isakeywordONE{using}\isamarkupfalse%
\ assms\ conc{\isacharunderscore}{\kern0pt}mono{\isacharbrackleft}{\kern0pt}of\ A\ B{\isacharbrackright}{\kern0pt}\ \isakeywordONE{by}\isamarkupfalse%
\ {\isacharparenleft}{\kern0pt}induction\ n{\isacharparenright}{\kern0pt}\ auto%
\endisatagproof
{\isafoldproof}%
%
\isadelimproof
%
\endisadelimproof
%
\begin{isamarkuptext}%
The one direction:%
\end{isamarkuptext}\isamarkuptrue%
\isakeywordONE{lemma}\isamarkupfalse%
\ rlexp{\isacharunderscore}{\kern0pt}cont{\isacharunderscore}{\kern0pt}aux{\isadigit{1}}{\isacharcolon}{\kern0pt}\isanewline
\ \ \isakeywordTWO{assumes}\ {\isachardoublequoteopen}{\isasymforall}i{\isachardot}{\kern0pt}\ v\ i\ {\isasymle}\ v\ {\isacharparenleft}{\kern0pt}Suc\ i{\isacharparenright}{\kern0pt}{\isachardoublequoteclose}\isanewline
\ \ \ \ \ \ \isakeywordTWO{and}\ {\isachardoublequoteopen}w\ {\isasymin}\ {\isacharparenleft}{\kern0pt}{\isasymUnion}i{\isachardot}{\kern0pt}\ eval\ f\ {\isacharparenleft}{\kern0pt}v\ i{\isacharparenright}{\kern0pt}{\isacharparenright}{\kern0pt}{\isachardoublequoteclose}\isanewline
\ \ \ \ \isakeywordTWO{shows}\ {\isachardoublequoteopen}w\ {\isasymin}\ eval\ f\ {\isacharparenleft}{\kern0pt}{\isasymlambda}x{\isachardot}{\kern0pt}\ {\isasymUnion}i{\isachardot}{\kern0pt}\ v\ i\ x{\isacharparenright}{\kern0pt}{\isachardoublequoteclose}\isanewline
%
\isadelimproof
%
\endisadelimproof
%
\isatagproof
\isakeywordONE{proof}\isamarkupfalse%
\ {\isacharminus}{\kern0pt}\isanewline
\ \ \isakeywordONE{from}\isamarkupfalse%
\ assms{\isacharparenleft}{\kern0pt}{\isadigit{2}}{\isacharparenright}{\kern0pt}\ \isakeywordTHREE{obtain}\isamarkupfalse%
\ n\ \isakeywordTWO{where}\ n{\isacharunderscore}{\kern0pt}intro{\isacharcolon}{\kern0pt}\ {\isachardoublequoteopen}w\ {\isasymin}\ eval\ f\ {\isacharparenleft}{\kern0pt}v\ n{\isacharparenright}{\kern0pt}{\isachardoublequoteclose}\ \isakeywordONE{by}\isamarkupfalse%
\ auto\isanewline
\ \ \isakeywordONE{have}\isamarkupfalse%
\ {\isachardoublequoteopen}v\ n\ x\ {\isasymsubseteq}\ {\isacharparenleft}{\kern0pt}{\isasymUnion}i{\isachardot}{\kern0pt}\ v\ i\ x{\isacharparenright}{\kern0pt}{\isachardoublequoteclose}\ \isakeywordTWO{for}\ x\ \isakeywordONE{by}\isamarkupfalse%
\ auto\isanewline
\ \ \isakeywordONE{with}\isamarkupfalse%
\ n{\isacharunderscore}{\kern0pt}intro\ \isakeywordTHREE{show}\isamarkupfalse%
\ {\isachardoublequoteopen}{\isacharquery}{\kern0pt}thesis{\isachardoublequoteclose}\isanewline
\ \ \ \ \isakeywordONE{using}\isamarkupfalse%
\ rlexp{\isacharunderscore}{\kern0pt}mono{\isacharunderscore}{\kern0pt}aux{\isacharbrackleft}{\kern0pt}\isakeywordTWO{where}\ v{\isacharequal}{\kern0pt}{\isachardoublequoteopen}v\ n{\isachardoublequoteclose}\ \isakeywordTWO{and}\ v{\isacharprime}{\kern0pt}{\isacharequal}{\kern0pt}{\isachardoublequoteopen}{\isasymlambda}x{\isachardot}{\kern0pt}\ {\isasymUnion}i{\isachardot}{\kern0pt}\ v\ i\ x{\isachardoublequoteclose}{\isacharbrackright}{\kern0pt}\ \isakeywordONE{by}\isamarkupfalse%
\ auto\isanewline
\isakeywordONE{qed}\isamarkupfalse%
%
\endisatagproof
{\isafoldproof}%
%
\isadelimproof
\isanewline
%
\endisadelimproof
\isanewline
\isakeywordONE{lemma}\isamarkupfalse%
\ langpow{\isacharunderscore}{\kern0pt}Union{\isacharunderscore}{\kern0pt}eval{\isacharcolon}{\kern0pt}\isanewline
\ \ \isakeywordTWO{assumes}\ {\isachardoublequoteopen}{\isasymforall}i{\isachardot}{\kern0pt}\ v\ i\ {\isasymle}\ v\ {\isacharparenleft}{\kern0pt}Suc\ i{\isacharparenright}{\kern0pt}{\isachardoublequoteclose}\isanewline
\ \ \ \ \ \ \isakeywordTWO{and}\ {\isachardoublequoteopen}w\ {\isasymin}\ {\isacharparenleft}{\kern0pt}{\isasymUnion}i{\isachardot}{\kern0pt}\ eval\ f\ {\isacharparenleft}{\kern0pt}v\ i{\isacharparenright}{\kern0pt}{\isacharparenright}{\kern0pt}\ {\isacharcircum}{\kern0pt}{\isacharcircum}{\kern0pt}\ n{\isachardoublequoteclose}\isanewline
\ \ \ \ \isakeywordTWO{shows}\ {\isachardoublequoteopen}w\ {\isasymin}\ {\isacharparenleft}{\kern0pt}{\isasymUnion}i{\isachardot}{\kern0pt}\ eval\ f\ {\isacharparenleft}{\kern0pt}v\ i{\isacharparenright}{\kern0pt}\ {\isacharcircum}{\kern0pt}{\isacharcircum}{\kern0pt}\ n{\isacharparenright}{\kern0pt}{\isachardoublequoteclose}\isanewline
%
\isadelimproof
%
\endisadelimproof
%
\isatagproof
\isakeywordONE{using}\isamarkupfalse%
\ assms\ \isakeywordONE{proof}\isamarkupfalse%
\ {\isacharparenleft}{\kern0pt}induction\ n\ arbitrary{\isacharcolon}{\kern0pt}\ w{\isacharparenright}{\kern0pt}\isanewline
\ \ \isakeywordTHREE{case}\isamarkupfalse%
\ {\isadigit{0}}\isanewline
\ \ \isakeywordONE{then}\isamarkupfalse%
\ \isakeywordTHREE{show}\isamarkupfalse%
\ {\isacharquery}{\kern0pt}case\ \isakeywordONE{by}\isamarkupfalse%
\ simp\isanewline
\isakeywordONE{next}\isamarkupfalse%
\isanewline
\ \ \isakeywordTHREE{case}\isamarkupfalse%
\ {\isacharparenleft}{\kern0pt}Suc\ n{\isacharparenright}{\kern0pt}\isanewline
\ \ \isakeywordONE{then}\isamarkupfalse%
\ \isakeywordTHREE{obtain}\isamarkupfalse%
\ u\ u{\isacharprime}{\kern0pt}\ \isakeywordTWO{where}\ w{\isacharunderscore}{\kern0pt}decomp{\isacharcolon}{\kern0pt}\ {\isachardoublequoteopen}w\ {\isacharequal}{\kern0pt}\ u{\isacharat}{\kern0pt}u{\isacharprime}{\kern0pt}{\isachardoublequoteclose}\ \isakeywordTWO{and}\isanewline
\ \ \ \ {\isachardoublequoteopen}u\ {\isasymin}\ {\isacharparenleft}{\kern0pt}{\isasymUnion}i{\isachardot}{\kern0pt}\ eval\ f\ {\isacharparenleft}{\kern0pt}v\ i{\isacharparenright}{\kern0pt}{\isacharparenright}{\kern0pt}\ {\isasymand}\ u{\isacharprime}{\kern0pt}\ {\isasymin}\ {\isacharparenleft}{\kern0pt}{\isasymUnion}i{\isachardot}{\kern0pt}\ eval\ f\ {\isacharparenleft}{\kern0pt}v\ i{\isacharparenright}{\kern0pt}{\isacharparenright}{\kern0pt}\ {\isacharcircum}{\kern0pt}{\isacharcircum}{\kern0pt}\ n{\isachardoublequoteclose}\ \isakeywordONE{by}\isamarkupfalse%
\ fastforce\isanewline
\ \ \isakeywordONE{with}\isamarkupfalse%
\ Suc\ \isakeywordONE{have}\isamarkupfalse%
\ {\isachardoublequoteopen}u\ {\isasymin}\ {\isacharparenleft}{\kern0pt}{\isasymUnion}i{\isachardot}{\kern0pt}\ eval\ f\ {\isacharparenleft}{\kern0pt}v\ i{\isacharparenright}{\kern0pt}{\isacharparenright}{\kern0pt}\ {\isasymand}\ u{\isacharprime}{\kern0pt}\ {\isasymin}\ {\isacharparenleft}{\kern0pt}{\isasymUnion}i{\isachardot}{\kern0pt}\ eval\ f\ {\isacharparenleft}{\kern0pt}v\ i{\isacharparenright}{\kern0pt}\ {\isacharcircum}{\kern0pt}{\isacharcircum}{\kern0pt}\ n{\isacharparenright}{\kern0pt}{\isachardoublequoteclose}\ \isakeywordONE{by}\isamarkupfalse%
\ auto\isanewline
\ \ \isakeywordONE{then}\isamarkupfalse%
\ \isakeywordTHREE{obtain}\isamarkupfalse%
\ i\ j\ \isakeywordTWO{where}\ i{\isacharunderscore}{\kern0pt}intro{\isacharcolon}{\kern0pt}\ {\isachardoublequoteopen}u\ {\isasymin}\ eval\ f\ {\isacharparenleft}{\kern0pt}v\ i{\isacharparenright}{\kern0pt}{\isachardoublequoteclose}\ \isakeywordTWO{and}\ j{\isacharunderscore}{\kern0pt}intro{\isacharcolon}{\kern0pt}\ {\isachardoublequoteopen}u{\isacharprime}{\kern0pt}\ {\isasymin}\ eval\ f\ {\isacharparenleft}{\kern0pt}v\ j{\isacharparenright}{\kern0pt}\ {\isacharcircum}{\kern0pt}{\isacharcircum}{\kern0pt}\ n{\isachardoublequoteclose}\ \isakeywordONE{by}\isamarkupfalse%
\ blast\isanewline
\ \ \isakeywordONE{let}\isamarkupfalse%
\ {\isacharquery}{\kern0pt}m\ {\isacharequal}{\kern0pt}\ {\isachardoublequoteopen}max\ i\ j{\isachardoublequoteclose}\isanewline
\ \ \isakeywordONE{from}\isamarkupfalse%
\ i{\isacharunderscore}{\kern0pt}intro\ Suc{\isachardot}{\kern0pt}prems{\isacharparenleft}{\kern0pt}{\isadigit{1}}{\isacharparenright}{\kern0pt}\ rlexp{\isacharunderscore}{\kern0pt}mono{\isacharunderscore}{\kern0pt}aux\ \isakeywordONE{have}\isamarkupfalse%
\ {\isadigit{1}}{\isacharcolon}{\kern0pt}\ {\isachardoublequoteopen}u\ {\isasymin}\ eval\ f\ {\isacharparenleft}{\kern0pt}v\ {\isacharquery}{\kern0pt}m{\isacharparenright}{\kern0pt}{\isachardoublequoteclose}\isanewline
\ \ \ \ \isakeywordONE{by}\isamarkupfalse%
\ {\isacharparenleft}{\kern0pt}metis\ le{\isacharunderscore}{\kern0pt}fun{\isacharunderscore}{\kern0pt}def\ lift{\isacharunderscore}{\kern0pt}Suc{\isacharunderscore}{\kern0pt}mono{\isacharunderscore}{\kern0pt}le\ max{\isachardot}{\kern0pt}cobounded{\isadigit{1}}\ subset{\isacharunderscore}{\kern0pt}eq{\isacharparenright}{\kern0pt}\isanewline
\ \ \isakeywordONE{from}\isamarkupfalse%
\ Suc{\isachardot}{\kern0pt}prems{\isacharparenleft}{\kern0pt}{\isadigit{1}}{\isacharparenright}{\kern0pt}\ rlexp{\isacharunderscore}{\kern0pt}mono{\isacharunderscore}{\kern0pt}aux\ \isakeywordONE{have}\isamarkupfalse%
\ {\isachardoublequoteopen}eval\ f\ {\isacharparenleft}{\kern0pt}v\ j{\isacharparenright}{\kern0pt}\ {\isasymsubseteq}\ eval\ f\ {\isacharparenleft}{\kern0pt}v\ {\isacharquery}{\kern0pt}m{\isacharparenright}{\kern0pt}{\isachardoublequoteclose}\isanewline
\ \ \ \ \isakeywordONE{by}\isamarkupfalse%
\ {\isacharparenleft}{\kern0pt}metis\ le{\isacharunderscore}{\kern0pt}fun{\isacharunderscore}{\kern0pt}def\ lift{\isacharunderscore}{\kern0pt}Suc{\isacharunderscore}{\kern0pt}mono{\isacharunderscore}{\kern0pt}le\ max{\isachardot}{\kern0pt}cobounded{\isadigit{2}}{\isacharparenright}{\kern0pt}\isanewline
\ \ \isakeywordONE{with}\isamarkupfalse%
\ j{\isacharunderscore}{\kern0pt}intro\ langpow{\isacharunderscore}{\kern0pt}mono\ \isakeywordONE{have}\isamarkupfalse%
\ {\isadigit{2}}{\isacharcolon}{\kern0pt}\ {\isachardoublequoteopen}u{\isacharprime}{\kern0pt}\ {\isasymin}\ eval\ f\ {\isacharparenleft}{\kern0pt}v\ {\isacharquery}{\kern0pt}m{\isacharparenright}{\kern0pt}\ {\isacharcircum}{\kern0pt}{\isacharcircum}{\kern0pt}\ n{\isachardoublequoteclose}\ \isakeywordONE{by}\isamarkupfalse%
\ auto\isanewline
\ \ \isakeywordONE{from}\isamarkupfalse%
\ {\isadigit{1}}\ {\isadigit{2}}\ \isakeywordTHREE{show}\isamarkupfalse%
\ {\isacharquery}{\kern0pt}case\ \isakeywordONE{using}\isamarkupfalse%
\ w{\isacharunderscore}{\kern0pt}decomp\ \isakeywordONE{by}\isamarkupfalse%
\ auto\isanewline
\isakeywordONE{qed}\isamarkupfalse%
%
\endisatagproof
{\isafoldproof}%
%
\isadelimproof
%
\endisadelimproof
%
\begin{isamarkuptext}%
The other direction:%
\end{isamarkuptext}\isamarkuptrue%
\isakeywordONE{lemma}\isamarkupfalse%
\ rlexp{\isacharunderscore}{\kern0pt}cont{\isacharunderscore}{\kern0pt}aux{\isadigit{2}}{\isacharcolon}{\kern0pt}\isanewline
\ \ \isakeywordTWO{assumes}\ {\isachardoublequoteopen}{\isasymforall}i{\isachardot}{\kern0pt}\ v\ i\ {\isasymle}\ v\ {\isacharparenleft}{\kern0pt}Suc\ i{\isacharparenright}{\kern0pt}{\isachardoublequoteclose}\isanewline
\ \ \ \ \ \ \isakeywordTWO{and}\ {\isachardoublequoteopen}w\ {\isasymin}\ eval\ f\ {\isacharparenleft}{\kern0pt}{\isasymlambda}x{\isachardot}{\kern0pt}\ {\isasymUnion}i{\isachardot}{\kern0pt}\ v\ i\ x{\isacharparenright}{\kern0pt}{\isachardoublequoteclose}\isanewline
\ \ \ \ \isakeywordTWO{shows}\ {\isachardoublequoteopen}w\ {\isasymin}\ {\isacharparenleft}{\kern0pt}{\isasymUnion}i{\isachardot}{\kern0pt}\ eval\ f\ {\isacharparenleft}{\kern0pt}v\ i{\isacharparenright}{\kern0pt}{\isacharparenright}{\kern0pt}{\isachardoublequoteclose}\isanewline
%
\isadelimproof
%
\endisadelimproof
%
\isatagproof
\isakeywordONE{using}\isamarkupfalse%
\ assms\ \isakeywordONE{proof}\isamarkupfalse%
\ {\isacharparenleft}{\kern0pt}induction\ arbitrary{\isacharcolon}{\kern0pt}\ w\ rule{\isacharcolon}{\kern0pt}\ rlexp{\isachardot}{\kern0pt}induct{\isacharparenright}{\kern0pt}\isanewline
\ \ \isakeywordTHREE{case}\isamarkupfalse%
\ {\isacharparenleft}{\kern0pt}Concat\ f\ g{\isacharparenright}{\kern0pt}\isanewline
\ \ \isakeywordONE{then}\isamarkupfalse%
\ \isakeywordTHREE{obtain}\isamarkupfalse%
\ u\ u{\isacharprime}{\kern0pt}\ \isakeywordTWO{where}\ w{\isacharunderscore}{\kern0pt}decomp{\isacharcolon}{\kern0pt}\ {\isachardoublequoteopen}w\ {\isacharequal}{\kern0pt}\ u{\isacharat}{\kern0pt}u{\isacharprime}{\kern0pt}{\isachardoublequoteclose}\isanewline
\ \ \ \ \isakeywordTWO{and}\ {\isachardoublequoteopen}u\ {\isasymin}\ eval\ f\ {\isacharparenleft}{\kern0pt}{\isasymlambda}x{\isachardot}{\kern0pt}\ {\isasymUnion}i{\isachardot}{\kern0pt}\ v\ i\ x{\isacharparenright}{\kern0pt}\ {\isasymand}\ u{\isacharprime}{\kern0pt}\ {\isasymin}\ eval\ g\ {\isacharparenleft}{\kern0pt}{\isasymlambda}x{\isachardot}{\kern0pt}\ {\isasymUnion}i{\isachardot}{\kern0pt}\ v\ i\ x{\isacharparenright}{\kern0pt}{\isachardoublequoteclose}\ \isakeywordONE{by}\isamarkupfalse%
\ auto\isanewline
\ \ \isakeywordONE{with}\isamarkupfalse%
\ Concat\ \isakeywordONE{have}\isamarkupfalse%
\ {\isachardoublequoteopen}u\ {\isasymin}\ {\isacharparenleft}{\kern0pt}{\isasymUnion}i{\isachardot}{\kern0pt}\ eval\ f\ {\isacharparenleft}{\kern0pt}v\ i{\isacharparenright}{\kern0pt}{\isacharparenright}{\kern0pt}\ {\isasymand}\ u{\isacharprime}{\kern0pt}\ {\isasymin}\ {\isacharparenleft}{\kern0pt}{\isasymUnion}i{\isachardot}{\kern0pt}\ eval\ g\ {\isacharparenleft}{\kern0pt}v\ i{\isacharparenright}{\kern0pt}{\isacharparenright}{\kern0pt}{\isachardoublequoteclose}\ \isakeywordONE{by}\isamarkupfalse%
\ auto\isanewline
\ \ \isakeywordONE{then}\isamarkupfalse%
\ \isakeywordTHREE{obtain}\isamarkupfalse%
\ i\ j\ \isakeywordTWO{where}\ i{\isacharunderscore}{\kern0pt}intro{\isacharcolon}{\kern0pt}\ {\isachardoublequoteopen}u\ {\isasymin}\ eval\ f\ {\isacharparenleft}{\kern0pt}v\ i{\isacharparenright}{\kern0pt}{\isachardoublequoteclose}\ \isakeywordTWO{and}\ j{\isacharunderscore}{\kern0pt}intro{\isacharcolon}{\kern0pt}\ {\isachardoublequoteopen}u{\isacharprime}{\kern0pt}\ {\isasymin}\ eval\ g\ {\isacharparenleft}{\kern0pt}v\ j{\isacharparenright}{\kern0pt}{\isachardoublequoteclose}\ \isakeywordONE{by}\isamarkupfalse%
\ blast\isanewline
\ \ \isakeywordONE{let}\isamarkupfalse%
\ {\isacharquery}{\kern0pt}m\ {\isacharequal}{\kern0pt}\ {\isachardoublequoteopen}max\ i\ j{\isachardoublequoteclose}\isanewline
\ \ \isakeywordONE{from}\isamarkupfalse%
\ i{\isacharunderscore}{\kern0pt}intro\ Concat{\isachardot}{\kern0pt}prems{\isacharparenleft}{\kern0pt}{\isadigit{1}}{\isacharparenright}{\kern0pt}\ rlexp{\isacharunderscore}{\kern0pt}mono{\isacharunderscore}{\kern0pt}aux\ \isakeywordONE{have}\isamarkupfalse%
\ {\isachardoublequoteopen}u\ {\isasymin}\ eval\ f\ {\isacharparenleft}{\kern0pt}v\ {\isacharquery}{\kern0pt}m{\isacharparenright}{\kern0pt}{\isachardoublequoteclose}\isanewline
\ \ \ \ \isakeywordONE{by}\isamarkupfalse%
\ {\isacharparenleft}{\kern0pt}metis\ le{\isacharunderscore}{\kern0pt}fun{\isacharunderscore}{\kern0pt}def\ lift{\isacharunderscore}{\kern0pt}Suc{\isacharunderscore}{\kern0pt}mono{\isacharunderscore}{\kern0pt}le\ max{\isachardot}{\kern0pt}cobounded{\isadigit{1}}\ subset{\isacharunderscore}{\kern0pt}eq{\isacharparenright}{\kern0pt}\isanewline
\ \ \isakeywordONE{moreover}\isamarkupfalse%
\ \isakeywordONE{from}\isamarkupfalse%
\ j{\isacharunderscore}{\kern0pt}intro\ Concat{\isachardot}{\kern0pt}prems{\isacharparenleft}{\kern0pt}{\isadigit{1}}{\isacharparenright}{\kern0pt}\ rlexp{\isacharunderscore}{\kern0pt}mono{\isacharunderscore}{\kern0pt}aux\ \isakeywordONE{have}\isamarkupfalse%
\ {\isachardoublequoteopen}u{\isacharprime}{\kern0pt}\ {\isasymin}\ eval\ g\ {\isacharparenleft}{\kern0pt}v\ {\isacharquery}{\kern0pt}m{\isacharparenright}{\kern0pt}{\isachardoublequoteclose}\isanewline
\ \ \ \ \isakeywordONE{by}\isamarkupfalse%
\ {\isacharparenleft}{\kern0pt}metis\ le{\isacharunderscore}{\kern0pt}fun{\isacharunderscore}{\kern0pt}def\ lift{\isacharunderscore}{\kern0pt}Suc{\isacharunderscore}{\kern0pt}mono{\isacharunderscore}{\kern0pt}le\ max{\isachardot}{\kern0pt}cobounded{\isadigit{2}}\ subset{\isacharunderscore}{\kern0pt}eq{\isacharparenright}{\kern0pt}\isanewline
\ \ \isakeywordONE{ultimately}\isamarkupfalse%
\ \isakeywordTHREE{show}\isamarkupfalse%
\ {\isacharquery}{\kern0pt}case\ \isakeywordONE{using}\isamarkupfalse%
\ w{\isacharunderscore}{\kern0pt}decomp\ \isakeywordONE{by}\isamarkupfalse%
\ auto\isanewline
\isakeywordONE{next}\isamarkupfalse%
\isanewline
\ \ \isakeywordTHREE{case}\isamarkupfalse%
\ {\isacharparenleft}{\kern0pt}Star\ f{\isacharparenright}{\kern0pt}\isanewline
\ \ \isakeywordONE{then}\isamarkupfalse%
\ \isakeywordTHREE{obtain}\isamarkupfalse%
\ n\ \isakeywordTWO{where}\ n{\isacharunderscore}{\kern0pt}intro{\isacharcolon}{\kern0pt}\ {\isachardoublequoteopen}w\ {\isasymin}\ {\isacharparenleft}{\kern0pt}eval\ f\ {\isacharparenleft}{\kern0pt}{\isasymlambda}x{\isachardot}{\kern0pt}\ {\isasymUnion}i{\isachardot}{\kern0pt}\ v\ i\ x{\isacharparenright}{\kern0pt}{\isacharparenright}{\kern0pt}\ {\isacharcircum}{\kern0pt}{\isacharcircum}{\kern0pt}\ n{\isachardoublequoteclose}\isanewline
\ \ \ \ \isakeywordONE{using}\isamarkupfalse%
\ eval{\isachardot}{\kern0pt}simps{\isacharparenleft}{\kern0pt}{\isadigit{5}}{\isacharparenright}{\kern0pt}\ star{\isacharunderscore}{\kern0pt}pow\ \isakeywordONE{by}\isamarkupfalse%
\ blast\isanewline
\ \ \isakeywordONE{with}\isamarkupfalse%
\ Star\ \isakeywordONE{have}\isamarkupfalse%
\ {\isachardoublequoteopen}w\ {\isasymin}\ {\isacharparenleft}{\kern0pt}{\isasymUnion}i{\isachardot}{\kern0pt}\ eval\ f\ {\isacharparenleft}{\kern0pt}v\ i{\isacharparenright}{\kern0pt}{\isacharparenright}{\kern0pt}\ {\isacharcircum}{\kern0pt}{\isacharcircum}{\kern0pt}\ n{\isachardoublequoteclose}\ \isakeywordONE{using}\isamarkupfalse%
\ langpow{\isacharunderscore}{\kern0pt}mono\ \isakeywordONE{by}\isamarkupfalse%
\ blast\isanewline
\ \ \isakeywordONE{with}\isamarkupfalse%
\ Star{\isachardot}{\kern0pt}prems\ \isakeywordONE{have}\isamarkupfalse%
\ {\isachardoublequoteopen}w\ {\isasymin}\ {\isacharparenleft}{\kern0pt}{\isasymUnion}i{\isachardot}{\kern0pt}\ eval\ f\ {\isacharparenleft}{\kern0pt}v\ i{\isacharparenright}{\kern0pt}\ {\isacharcircum}{\kern0pt}{\isacharcircum}{\kern0pt}\ n{\isacharparenright}{\kern0pt}{\isachardoublequoteclose}\ \isakeywordONE{using}\isamarkupfalse%
\ langpow{\isacharunderscore}{\kern0pt}Union{\isacharunderscore}{\kern0pt}eval\ \isakeywordONE{by}\isamarkupfalse%
\ auto\isanewline
\ \ \isakeywordONE{then}\isamarkupfalse%
\ \isakeywordTHREE{show}\isamarkupfalse%
\ {\isacharquery}{\kern0pt}case\ \isakeywordONE{by}\isamarkupfalse%
\ {\isacharparenleft}{\kern0pt}auto\ simp\ add{\isacharcolon}{\kern0pt}\ star{\isacharunderscore}{\kern0pt}def{\isacharparenright}{\kern0pt}\isanewline
\isakeywordONE{qed}\isamarkupfalse%
\ fastforce{\isacharplus}{\kern0pt}%
\endisatagproof
{\isafoldproof}%
%
\isadelimproof
%
\endisadelimproof
%
\begin{isamarkuptext}%
The actual continuity lemma:%
\end{isamarkuptext}\isamarkuptrue%
\isakeywordONE{lemma}\isamarkupfalse%
\ rlexp{\isacharunderscore}{\kern0pt}cont{\isacharcolon}{\kern0pt}\isanewline
\ \ \isakeywordTWO{assumes}\ {\isachardoublequoteopen}{\isasymforall}i{\isachardot}{\kern0pt}\ v\ i\ {\isasymle}\ v\ {\isacharparenleft}{\kern0pt}Suc\ i{\isacharparenright}{\kern0pt}{\isachardoublequoteclose}\isanewline
\ \ \isakeywordTWO{shows}\ {\isachardoublequoteopen}eval\ f\ {\isacharparenleft}{\kern0pt}{\isasymlambda}x{\isachardot}{\kern0pt}\ {\isasymUnion}i{\isachardot}{\kern0pt}\ v\ i\ x{\isacharparenright}{\kern0pt}\ {\isacharequal}{\kern0pt}\ {\isacharparenleft}{\kern0pt}{\isasymUnion}i{\isachardot}{\kern0pt}\ eval\ f\ {\isacharparenleft}{\kern0pt}v\ i{\isacharparenright}{\kern0pt}{\isacharparenright}{\kern0pt}{\isachardoublequoteclose}\isanewline
%
\isadelimproof
%
\endisadelimproof
%
\isatagproof
\isakeywordONE{proof}\isamarkupfalse%
\isanewline
\ \ \isakeywordONE{from}\isamarkupfalse%
\ assms\ \isakeywordTHREE{show}\isamarkupfalse%
\ {\isachardoublequoteopen}eval\ f\ {\isacharparenleft}{\kern0pt}{\isasymlambda}x{\isachardot}{\kern0pt}\ {\isasymUnion}i{\isachardot}{\kern0pt}\ v\ i\ x{\isacharparenright}{\kern0pt}\ {\isasymsubseteq}\ {\isacharparenleft}{\kern0pt}{\isasymUnion}i{\isachardot}{\kern0pt}\ eval\ f\ {\isacharparenleft}{\kern0pt}v\ i{\isacharparenright}{\kern0pt}{\isacharparenright}{\kern0pt}{\isachardoublequoteclose}\ \isakeywordONE{using}\isamarkupfalse%
\ rlexp{\isacharunderscore}{\kern0pt}cont{\isacharunderscore}{\kern0pt}aux{\isadigit{2}}\ \isakeywordONE{by}\isamarkupfalse%
\ auto\isanewline
\ \ \isakeywordONE{from}\isamarkupfalse%
\ assms\ \isakeywordTHREE{show}\isamarkupfalse%
\ {\isachardoublequoteopen}{\isacharparenleft}{\kern0pt}{\isasymUnion}i{\isachardot}{\kern0pt}\ eval\ f\ {\isacharparenleft}{\kern0pt}v\ i{\isacharparenright}{\kern0pt}{\isacharparenright}{\kern0pt}\ {\isasymsubseteq}\ eval\ f\ {\isacharparenleft}{\kern0pt}{\isasymlambda}x{\isachardot}{\kern0pt}\ {\isasymUnion}i{\isachardot}{\kern0pt}\ v\ i\ x{\isacharparenright}{\kern0pt}{\isachardoublequoteclose}\ \isakeywordONE{using}\isamarkupfalse%
\ rlexp{\isacharunderscore}{\kern0pt}cont{\isacharunderscore}{\kern0pt}aux{\isadigit{1}}\ \isakeywordONE{by}\isamarkupfalse%
\ blast\isanewline
\isakeywordONE{qed}\isamarkupfalse%
%
\endisatagproof
{\isafoldproof}%
%
\isadelimproof
%
\endisadelimproof
%
\isadelimdocument
%
\endisadelimdocument
%
\isatagdocument
%
\isamarkupsection{Regular language expressions which evaluate to regular languages%
}
\isamarkuptrue%
%
\endisatagdocument
{\isafolddocument}%
%
\isadelimdocument
%
\endisadelimdocument
%
\begin{isamarkuptext}%
Evaluating regular language expressions can yield non-regular languages even if
the valuation maps each variable to a regular language. This is because \isa{Const} may introduce
non-regular languages.
We therefore introduce the predicate \isa{reg{\isacharunderscore}{\kern0pt}eval} which guarantees that a regular language expression
\isa{f} yields a regular language if the valuation maps all variables occurring in \isa{f} to some regular
language. This is achieved by only allowing regular languages as constants.
However, note that \isa{reg{\isacharunderscore}{\kern0pt}eval} is an under-approximation, i.e. there exist regular language
expressions which do not satisfy this predicate but evaluate to regular languages anyway.%
\end{isamarkuptext}\isamarkuptrue%
\isakeywordONE{fun}\isamarkupfalse%
\ reg{\isacharunderscore}{\kern0pt}eval\ {\isacharcolon}{\kern0pt}{\isacharcolon}{\kern0pt}\ {\isachardoublequoteopen}{\isacharprime}{\kern0pt}a\ rlexp\ {\isasymRightarrow}\ bool{\isachardoublequoteclose}\ \isakeywordTWO{where}\isanewline
\ \ {\isachardoublequoteopen}reg{\isacharunderscore}{\kern0pt}eval\ {\isacharparenleft}{\kern0pt}Var\ {\isacharunderscore}{\kern0pt}{\isacharparenright}{\kern0pt}\ {\isasymlongleftrightarrow}\ True{\isachardoublequoteclose}\ {\isacharbar}{\kern0pt}\isanewline
\ \ {\isachardoublequoteopen}reg{\isacharunderscore}{\kern0pt}eval\ {\isacharparenleft}{\kern0pt}Const\ l{\isacharparenright}{\kern0pt}\ {\isasymlongleftrightarrow}\ regular{\isacharunderscore}{\kern0pt}lang\ l{\isachardoublequoteclose}\ {\isacharbar}{\kern0pt}\isanewline
\ \ {\isachardoublequoteopen}reg{\isacharunderscore}{\kern0pt}eval\ {\isacharparenleft}{\kern0pt}Union\ f\ g{\isacharparenright}{\kern0pt}\ {\isasymlongleftrightarrow}\ reg{\isacharunderscore}{\kern0pt}eval\ f\ {\isasymand}\ reg{\isacharunderscore}{\kern0pt}eval\ g{\isachardoublequoteclose}\ {\isacharbar}{\kern0pt}\isanewline
\ \ {\isachardoublequoteopen}reg{\isacharunderscore}{\kern0pt}eval\ {\isacharparenleft}{\kern0pt}Concat\ f\ g{\isacharparenright}{\kern0pt}\ {\isasymlongleftrightarrow}\ reg{\isacharunderscore}{\kern0pt}eval\ f\ {\isasymand}\ reg{\isacharunderscore}{\kern0pt}eval\ g{\isachardoublequoteclose}\ {\isacharbar}{\kern0pt}\isanewline
\ \ {\isachardoublequoteopen}reg{\isacharunderscore}{\kern0pt}eval\ {\isacharparenleft}{\kern0pt}Star\ f{\isacharparenright}{\kern0pt}\ {\isasymlongleftrightarrow}\ reg{\isacharunderscore}{\kern0pt}eval\ f{\isachardoublequoteclose}\isanewline
\isanewline
\isanewline
\isakeywordONE{lemma}\isamarkupfalse%
\ emptyset{\isacharunderscore}{\kern0pt}regular{\isacharcolon}{\kern0pt}\ {\isachardoublequoteopen}reg{\isacharunderscore}{\kern0pt}eval\ {\isacharparenleft}{\kern0pt}Const\ {\isacharbraceleft}{\kern0pt}{\isacharbraceright}{\kern0pt}{\isacharparenright}{\kern0pt}{\isachardoublequoteclose}\isanewline
%
\isadelimproof
\ \ %
\endisadelimproof
%
\isatagproof
\isakeywordONE{using}\isamarkupfalse%
\ lang{\isachardot}{\kern0pt}simps{\isacharparenleft}{\kern0pt}{\isadigit{1}}{\isacharparenright}{\kern0pt}\ reg{\isacharunderscore}{\kern0pt}eval{\isachardot}{\kern0pt}simps{\isacharparenleft}{\kern0pt}{\isadigit{2}}{\isacharparenright}{\kern0pt}\ \isakeywordONE{by}\isamarkupfalse%
\ blast%
\endisatagproof
{\isafoldproof}%
%
\isadelimproof
\isanewline
%
\endisadelimproof
\isanewline
\isakeywordONE{lemma}\isamarkupfalse%
\ epsilon{\isacharunderscore}{\kern0pt}regular{\isacharcolon}{\kern0pt}\ {\isachardoublequoteopen}reg{\isacharunderscore}{\kern0pt}eval\ {\isacharparenleft}{\kern0pt}Const\ {\isacharbraceleft}{\kern0pt}{\isacharbrackleft}{\kern0pt}{\isacharbrackright}{\kern0pt}{\isacharbraceright}{\kern0pt}{\isacharparenright}{\kern0pt}{\isachardoublequoteclose}\isanewline
%
\isadelimproof
\ \ %
\endisadelimproof
%
\isatagproof
\isakeywordONE{using}\isamarkupfalse%
\ lang{\isachardot}{\kern0pt}simps{\isacharparenleft}{\kern0pt}{\isadigit{2}}{\isacharparenright}{\kern0pt}\ reg{\isacharunderscore}{\kern0pt}eval{\isachardot}{\kern0pt}simps{\isacharparenleft}{\kern0pt}{\isadigit{2}}{\isacharparenright}{\kern0pt}\ \isakeywordONE{by}\isamarkupfalse%
\ blast%
\endisatagproof
{\isafoldproof}%
%
\isadelimproof
%
\endisadelimproof
%
\begin{isamarkuptext}%
If the valuation \isa{v} maps all variables occurring in the regular language function \isa{f} to
some regular language, then evaluating \isa{f} again yields a regular language.%
\end{isamarkuptext}\isamarkuptrue%
\isakeywordONE{lemma}\isamarkupfalse%
\ reg{\isacharunderscore}{\kern0pt}eval{\isacharunderscore}{\kern0pt}regular{\isacharcolon}{\kern0pt}\isanewline
\ \ \isakeywordTWO{assumes}\ {\isachardoublequoteopen}reg{\isacharunderscore}{\kern0pt}eval\ f{\isachardoublequoteclose}\isanewline
\ \ \ \ \ \ \isakeywordTWO{and}\ {\isachardoublequoteopen}{\isasymAnd}n{\isachardot}{\kern0pt}\ n\ {\isasymin}\ vars\ f\ {\isasymLongrightarrow}\ regular{\isacharunderscore}{\kern0pt}lang\ {\isacharparenleft}{\kern0pt}v\ n{\isacharparenright}{\kern0pt}{\isachardoublequoteclose}\isanewline
\ \ \ \ \isakeywordTWO{shows}\ {\isachardoublequoteopen}regular{\isacharunderscore}{\kern0pt}lang\ {\isacharparenleft}{\kern0pt}eval\ f\ v{\isacharparenright}{\kern0pt}{\isachardoublequoteclose}\isanewline
%
\isadelimproof
%
\endisadelimproof
%
\isatagproof
\isakeywordONE{using}\isamarkupfalse%
\ assms\ \isakeywordONE{proof}\isamarkupfalse%
\ {\isacharparenleft}{\kern0pt}induction\ rule{\isacharcolon}{\kern0pt}\ reg{\isacharunderscore}{\kern0pt}eval{\isachardot}{\kern0pt}induct{\isacharparenright}{\kern0pt}\isanewline
\ \ \isakeywordTHREE{case}\isamarkupfalse%
\ {\isacharparenleft}{\kern0pt}{\isadigit{3}}\ f\ g{\isacharparenright}{\kern0pt}\isanewline
\ \ \isakeywordONE{then}\isamarkupfalse%
\ \isakeywordTHREE{obtain}\isamarkupfalse%
\ r{\isadigit{1}}\ r{\isadigit{2}}\ \isakeywordTWO{where}\ {\isachardoublequoteopen}Regular{\isacharunderscore}{\kern0pt}Exp{\isachardot}{\kern0pt}lang\ r{\isadigit{1}}\ {\isacharequal}{\kern0pt}\ eval\ f\ v\ {\isasymand}\ Regular{\isacharunderscore}{\kern0pt}Exp{\isachardot}{\kern0pt}lang\ r{\isadigit{2}}\ {\isacharequal}{\kern0pt}\ eval\ g\ v{\isachardoublequoteclose}\ \isakeywordONE{by}\isamarkupfalse%
\ auto\isanewline
\ \ \isakeywordONE{then}\isamarkupfalse%
\ \isakeywordONE{have}\isamarkupfalse%
\ {\isachardoublequoteopen}Regular{\isacharunderscore}{\kern0pt}Exp{\isachardot}{\kern0pt}lang\ {\isacharparenleft}{\kern0pt}Plus\ r{\isadigit{1}}\ r{\isadigit{2}}{\isacharparenright}{\kern0pt}\ {\isacharequal}{\kern0pt}\ eval\ {\isacharparenleft}{\kern0pt}Union\ f\ g{\isacharparenright}{\kern0pt}\ v{\isachardoublequoteclose}\ \isakeywordONE{by}\isamarkupfalse%
\ simp\isanewline
\ \ \isakeywordONE{then}\isamarkupfalse%
\ \isakeywordTHREE{show}\isamarkupfalse%
\ {\isacharquery}{\kern0pt}case\ \isakeywordONE{by}\isamarkupfalse%
\ blast\isanewline
\isakeywordONE{next}\isamarkupfalse%
\isanewline
\ \ \isakeywordTHREE{case}\isamarkupfalse%
\ {\isacharparenleft}{\kern0pt}{\isadigit{4}}\ f\ g{\isacharparenright}{\kern0pt}\isanewline
\ \ \isakeywordONE{then}\isamarkupfalse%
\ \isakeywordTHREE{obtain}\isamarkupfalse%
\ r{\isadigit{1}}\ r{\isadigit{2}}\ \isakeywordTWO{where}\ {\isachardoublequoteopen}Regular{\isacharunderscore}{\kern0pt}Exp{\isachardot}{\kern0pt}lang\ r{\isadigit{1}}\ {\isacharequal}{\kern0pt}\ eval\ f\ v\ {\isasymand}\ Regular{\isacharunderscore}{\kern0pt}Exp{\isachardot}{\kern0pt}lang\ r{\isadigit{2}}\ {\isacharequal}{\kern0pt}\ eval\ g\ v{\isachardoublequoteclose}\ \isakeywordONE{by}\isamarkupfalse%
\ auto\isanewline
\ \ \isakeywordONE{then}\isamarkupfalse%
\ \isakeywordONE{have}\isamarkupfalse%
\ {\isachardoublequoteopen}Regular{\isacharunderscore}{\kern0pt}Exp{\isachardot}{\kern0pt}lang\ {\isacharparenleft}{\kern0pt}Times\ r{\isadigit{1}}\ r{\isadigit{2}}{\isacharparenright}{\kern0pt}\ {\isacharequal}{\kern0pt}\ eval\ {\isacharparenleft}{\kern0pt}Concat\ f\ g{\isacharparenright}{\kern0pt}\ v{\isachardoublequoteclose}\ \isakeywordONE{by}\isamarkupfalse%
\ simp\isanewline
\ \ \isakeywordONE{then}\isamarkupfalse%
\ \isakeywordTHREE{show}\isamarkupfalse%
\ {\isacharquery}{\kern0pt}case\ \isakeywordONE{by}\isamarkupfalse%
\ blast\isanewline
\isakeywordONE{next}\isamarkupfalse%
\isanewline
\ \ \isakeywordTHREE{case}\isamarkupfalse%
\ {\isacharparenleft}{\kern0pt}{\isadigit{5}}\ f{\isacharparenright}{\kern0pt}\isanewline
\ \ \isakeywordONE{then}\isamarkupfalse%
\ \isakeywordTHREE{obtain}\isamarkupfalse%
\ r\ \ \isakeywordTWO{where}\ {\isachardoublequoteopen}Regular{\isacharunderscore}{\kern0pt}Exp{\isachardot}{\kern0pt}lang\ r\ {\isacharequal}{\kern0pt}\ eval\ f\ v{\isachardoublequoteclose}\ \isakeywordONE{by}\isamarkupfalse%
\ auto\isanewline
\ \ \isakeywordONE{then}\isamarkupfalse%
\ \isakeywordONE{have}\isamarkupfalse%
\ {\isachardoublequoteopen}Regular{\isacharunderscore}{\kern0pt}Exp{\isachardot}{\kern0pt}lang\ {\isacharparenleft}{\kern0pt}Regular{\isacharunderscore}{\kern0pt}Exp{\isachardot}{\kern0pt}Star\ r{\isacharparenright}{\kern0pt}\ {\isacharequal}{\kern0pt}\ eval\ {\isacharparenleft}{\kern0pt}Star\ f{\isacharparenright}{\kern0pt}\ v{\isachardoublequoteclose}\ \isakeywordONE{by}\isamarkupfalse%
\ simp\isanewline
\ \ \isakeywordONE{then}\isamarkupfalse%
\ \isakeywordTHREE{show}\isamarkupfalse%
\ {\isacharquery}{\kern0pt}case\ \isakeywordONE{by}\isamarkupfalse%
\ blast\isanewline
\isakeywordONE{qed}\isamarkupfalse%
\ simp{\isacharunderscore}{\kern0pt}all%
\endisatagproof
{\isafoldproof}%
%
\isadelimproof
%
\endisadelimproof
%
\begin{isamarkuptext}%
A \isa{reg{\isacharunderscore}{\kern0pt}eval} regular language expression stays \isa{reg{\isacharunderscore}{\kern0pt}eval} if all variables are substituted
by \isa{reg{\isacharunderscore}{\kern0pt}eval} regular language expressions%
\end{isamarkuptext}\isamarkuptrue%
\isakeywordONE{lemma}\isamarkupfalse%
\ subst{\isacharunderscore}{\kern0pt}reg{\isacharunderscore}{\kern0pt}eval{\isacharcolon}{\kern0pt}\isanewline
\ \ \isakeywordTWO{assumes}\ {\isachardoublequoteopen}reg{\isacharunderscore}{\kern0pt}eval\ f{\isachardoublequoteclose}\isanewline
\ \ \ \ \ \ \isakeywordTWO{and}\ {\isachardoublequoteopen}{\isasymforall}x\ {\isasymin}\ vars\ f{\isachardot}{\kern0pt}\ reg{\isacharunderscore}{\kern0pt}eval\ {\isacharparenleft}{\kern0pt}upd\ x{\isacharparenright}{\kern0pt}{\isachardoublequoteclose}\isanewline
\ \ \ \ \isakeywordTWO{shows}\ {\isachardoublequoteopen}reg{\isacharunderscore}{\kern0pt}eval\ {\isacharparenleft}{\kern0pt}subst\ upd\ f{\isacharparenright}{\kern0pt}{\isachardoublequoteclose}\isanewline
%
\isadelimproof
\ \ %
\endisadelimproof
%
\isatagproof
\isakeywordONE{using}\isamarkupfalse%
\ assms\ \isakeywordONE{by}\isamarkupfalse%
\ {\isacharparenleft}{\kern0pt}induction\ f\ rule{\isacharcolon}{\kern0pt}\ reg{\isacharunderscore}{\kern0pt}eval{\isachardot}{\kern0pt}induct{\isacharparenright}{\kern0pt}\ simp{\isacharunderscore}{\kern0pt}all%
\endisatagproof
{\isafoldproof}%
%
\isadelimproof
\isanewline
%
\endisadelimproof
\isanewline
\isanewline
\isakeywordONE{lemma}\isamarkupfalse%
\ subst{\isacharunderscore}{\kern0pt}reg{\isacharunderscore}{\kern0pt}eval{\isacharunderscore}{\kern0pt}update{\isacharcolon}{\kern0pt}\isanewline
\ \ \isakeywordTWO{assumes}\ {\isachardoublequoteopen}reg{\isacharunderscore}{\kern0pt}eval\ f{\isachardoublequoteclose}\isanewline
\ \ \ \ \ \ \isakeywordTWO{and}\ {\isachardoublequoteopen}reg{\isacharunderscore}{\kern0pt}eval\ g{\isachardoublequoteclose}\isanewline
\ \ \ \ \isakeywordTWO{shows}\ {\isachardoublequoteopen}reg{\isacharunderscore}{\kern0pt}eval\ {\isacharparenleft}{\kern0pt}subst\ {\isacharparenleft}{\kern0pt}Var{\isacharparenleft}{\kern0pt}x\ {\isacharcolon}{\kern0pt}{\isacharequal}{\kern0pt}\ g{\isacharparenright}{\kern0pt}{\isacharparenright}{\kern0pt}\ f{\isacharparenright}{\kern0pt}{\isachardoublequoteclose}\isanewline
%
\isadelimproof
\ \ %
\endisadelimproof
%
\isatagproof
\isakeywordONE{using}\isamarkupfalse%
\ assms\ subst{\isacharunderscore}{\kern0pt}reg{\isacharunderscore}{\kern0pt}eval\ fun{\isacharunderscore}{\kern0pt}upd{\isacharunderscore}{\kern0pt}def\ \isakeywordONE{by}\isamarkupfalse%
\ {\isacharparenleft}{\kern0pt}metis\ reg{\isacharunderscore}{\kern0pt}eval{\isachardot}{\kern0pt}simps{\isacharparenleft}{\kern0pt}{\isadigit{1}}{\isacharparenright}{\kern0pt}{\isacharparenright}{\kern0pt}%
\endisatagproof
{\isafoldproof}%
%
\isadelimproof
%
\endisadelimproof
%
\begin{isamarkuptext}%
For any finite union of \isa{reg{\isacharunderscore}{\kern0pt}eval} regular language expressions exists a \isa{reg{\isacharunderscore}{\kern0pt}eval} regular
language expression%
\end{isamarkuptext}\isamarkuptrue%
\isakeywordONE{lemma}\isamarkupfalse%
\ finite{\isacharunderscore}{\kern0pt}Union{\isacharunderscore}{\kern0pt}regular{\isacharunderscore}{\kern0pt}aux{\isacharcolon}{\kern0pt}\isanewline
\ \ {\isachardoublequoteopen}{\isasymforall}f\ {\isasymin}\ set\ fs{\isachardot}{\kern0pt}\ reg{\isacharunderscore}{\kern0pt}eval\ f\ {\isasymLongrightarrow}\ {\isasymexists}g{\isachardot}{\kern0pt}\ reg{\isacharunderscore}{\kern0pt}eval\ g\ {\isasymand}\ {\isasymUnion}{\isacharparenleft}{\kern0pt}vars\ {\isacharbackquote}{\kern0pt}\ set\ fs{\isacharparenright}{\kern0pt}\ {\isacharequal}{\kern0pt}\ vars\ g\isanewline
\ \ \ \ \ \ \ \ \ \ \ \ \ \ \ \ \ \ \ \ \ \ \ \ \ \ \ \ \ \ \ \ \ \ \ \ \ \ {\isasymand}\ {\isacharparenleft}{\kern0pt}{\isasymforall}v{\isachardot}{\kern0pt}\ {\isacharparenleft}{\kern0pt}{\isasymUnion}f\ {\isasymin}\ set\ fs{\isachardot}{\kern0pt}\ eval\ f\ v{\isacharparenright}{\kern0pt}\ {\isacharequal}{\kern0pt}\ eval\ g\ v{\isacharparenright}{\kern0pt}{\isachardoublequoteclose}\isanewline
%
\isadelimproof
%
\endisadelimproof
%
\isatagproof
\isakeywordONE{proof}\isamarkupfalse%
\ {\isacharparenleft}{\kern0pt}induction\ fs{\isacharparenright}{\kern0pt}\isanewline
\ \ \isakeywordTHREE{case}\isamarkupfalse%
\ Nil\isanewline
\ \ \isakeywordONE{then}\isamarkupfalse%
\ \isakeywordTHREE{show}\isamarkupfalse%
\ {\isacharquery}{\kern0pt}case\ \isakeywordONE{using}\isamarkupfalse%
\ emptyset{\isacharunderscore}{\kern0pt}regular\ \isakeywordONE{by}\isamarkupfalse%
\ fastforce\isanewline
\isakeywordONE{next}\isamarkupfalse%
\isanewline
\ \ \isakeywordTHREE{case}\isamarkupfalse%
\ {\isacharparenleft}{\kern0pt}Cons\ f{\isadigit{1}}\ fs{\isacharparenright}{\kern0pt}\isanewline
\ \ \isakeywordONE{then}\isamarkupfalse%
\ \isakeywordTHREE{obtain}\isamarkupfalse%
\ g\ \isakeywordTWO{where}\ {\isacharasterisk}{\kern0pt}{\isacharcolon}{\kern0pt}\ {\isachardoublequoteopen}reg{\isacharunderscore}{\kern0pt}eval\ g\ {\isasymand}\ {\isasymUnion}{\isacharparenleft}{\kern0pt}vars\ {\isacharbackquote}{\kern0pt}\ set\ fs{\isacharparenright}{\kern0pt}\ {\isacharequal}{\kern0pt}\ vars\ g\isanewline
\ \ \ \ \ \ \ \ \ \ \ \ \ \ \ \ \ \ \ \ \ \ \ \ \ \ {\isasymand}\ {\isacharparenleft}{\kern0pt}{\isasymforall}v{\isachardot}{\kern0pt}\ {\isacharparenleft}{\kern0pt}{\isasymUnion}f{\isasymin}set\ fs{\isachardot}{\kern0pt}\ eval\ f\ v{\isacharparenright}{\kern0pt}\ {\isacharequal}{\kern0pt}\ eval\ g\ v{\isacharparenright}{\kern0pt}{\isachardoublequoteclose}\ \isakeywordONE{by}\isamarkupfalse%
\ auto\isanewline
\ \ \isakeywordONE{let}\isamarkupfalse%
\ {\isacharquery}{\kern0pt}g{\isacharprime}{\kern0pt}\ {\isacharequal}{\kern0pt}\ {\isachardoublequoteopen}Union\ f{\isadigit{1}}\ g{\isachardoublequoteclose}\isanewline
\ \ \isakeywordONE{from}\isamarkupfalse%
\ Cons{\isachardot}{\kern0pt}prems\ {\isacharasterisk}{\kern0pt}\ \isakeywordONE{have}\isamarkupfalse%
\ {\isachardoublequoteopen}reg{\isacharunderscore}{\kern0pt}eval\ {\isacharquery}{\kern0pt}g{\isacharprime}{\kern0pt}\ {\isasymand}\ {\isasymUnion}\ {\isacharparenleft}{\kern0pt}vars\ {\isacharbackquote}{\kern0pt}\ set\ {\isacharparenleft}{\kern0pt}f{\isadigit{1}}\ {\isacharhash}{\kern0pt}\ fs{\isacharparenright}{\kern0pt}{\isacharparenright}{\kern0pt}\ {\isacharequal}{\kern0pt}\ vars\ {\isacharquery}{\kern0pt}g{\isacharprime}{\kern0pt}\isanewline
\ \ \ \ \ \ {\isasymand}\ {\isacharparenleft}{\kern0pt}{\isasymforall}v{\isachardot}{\kern0pt}\ {\isacharparenleft}{\kern0pt}{\isasymUnion}f{\isasymin}set\ {\isacharparenleft}{\kern0pt}f{\isadigit{1}}\ {\isacharhash}{\kern0pt}\ fs{\isacharparenright}{\kern0pt}{\isachardot}{\kern0pt}\ eval\ f\ v{\isacharparenright}{\kern0pt}\ {\isacharequal}{\kern0pt}\ eval\ {\isacharquery}{\kern0pt}g{\isacharprime}{\kern0pt}\ v{\isacharparenright}{\kern0pt}{\isachardoublequoteclose}\ \isakeywordONE{by}\isamarkupfalse%
\ simp\isanewline
\ \ \isakeywordONE{then}\isamarkupfalse%
\ \isakeywordTHREE{show}\isamarkupfalse%
\ {\isacharquery}{\kern0pt}case\ \isakeywordONE{by}\isamarkupfalse%
\ blast\isanewline
\isakeywordONE{qed}\isamarkupfalse%
%
\endisatagproof
{\isafoldproof}%
%
\isadelimproof
\isanewline
%
\endisadelimproof
\isanewline
\isakeywordONE{lemma}\isamarkupfalse%
\ finite{\isacharunderscore}{\kern0pt}Union{\isacharunderscore}{\kern0pt}regular{\isacharcolon}{\kern0pt}\isanewline
\ \ \isakeywordTWO{assumes}\ {\isachardoublequoteopen}finite\ F{\isachardoublequoteclose}\isanewline
\ \ \ \ \ \ \isakeywordTWO{and}\ {\isachardoublequoteopen}{\isasymforall}f\ {\isasymin}\ F{\isachardot}{\kern0pt}\ reg{\isacharunderscore}{\kern0pt}eval\ f{\isachardoublequoteclose}\isanewline
\ \ \ \ \isakeywordTWO{shows}\ {\isachardoublequoteopen}{\isasymexists}g{\isachardot}{\kern0pt}\ reg{\isacharunderscore}{\kern0pt}eval\ g\ {\isasymand}\ {\isasymUnion}{\isacharparenleft}{\kern0pt}vars\ {\isacharbackquote}{\kern0pt}\ F{\isacharparenright}{\kern0pt}\ {\isacharequal}{\kern0pt}\ vars\ g\ {\isasymand}\ {\isacharparenleft}{\kern0pt}{\isasymforall}v{\isachardot}{\kern0pt}\ {\isacharparenleft}{\kern0pt}{\isasymUnion}f{\isasymin}F{\isachardot}{\kern0pt}\ eval\ f\ v{\isacharparenright}{\kern0pt}\ {\isacharequal}{\kern0pt}\ eval\ g\ v{\isacharparenright}{\kern0pt}{\isachardoublequoteclose}\isanewline
%
\isadelimproof
\ \ %
\endisadelimproof
%
\isatagproof
\isakeywordONE{using}\isamarkupfalse%
\ assms\ finite{\isacharunderscore}{\kern0pt}Union{\isacharunderscore}{\kern0pt}regular{\isacharunderscore}{\kern0pt}aux\ finite{\isacharunderscore}{\kern0pt}list\ \isakeywordONE{by}\isamarkupfalse%
\ metis%
\endisatagproof
{\isafoldproof}%
%
\isadelimproof
%
\endisadelimproof
%
\isadelimdocument
%
\endisadelimdocument
%
\isatagdocument
%
\isamarkupsection{Constant regular language functions%
}
\isamarkuptrue%
%
\endisatagdocument
{\isafolddocument}%
%
\isadelimdocument
%
\endisadelimdocument
%
\begin{isamarkuptext}%
A regular language expression is constant iff it contains no variables%
\end{isamarkuptext}\isamarkuptrue%
\isakeywordONE{abbreviation}\isamarkupfalse%
\ const{\isacharunderscore}{\kern0pt}rlexp\ {\isacharcolon}{\kern0pt}{\isacharcolon}{\kern0pt}\ {\isachardoublequoteopen}{\isacharprime}{\kern0pt}a\ rlexp\ {\isasymRightarrow}\ bool{\isachardoublequoteclose}\ \isakeywordTWO{where}\isanewline
\ \ {\isachardoublequoteopen}const{\isacharunderscore}{\kern0pt}rlexp\ f\ {\isasymequiv}\ vars\ f\ {\isacharequal}{\kern0pt}\ {\isacharbraceleft}{\kern0pt}{\isacharbraceright}{\kern0pt}{\isachardoublequoteclose}%
\begin{isamarkuptext}%
A constant regular language expression always evaluates to the same language, independent on
the valuation%
\end{isamarkuptext}\isamarkuptrue%
\isakeywordONE{lemma}\isamarkupfalse%
\ const{\isacharunderscore}{\kern0pt}rlexp{\isacharunderscore}{\kern0pt}lang{\isacharcolon}{\kern0pt}\ {\isachardoublequoteopen}const{\isacharunderscore}{\kern0pt}rlexp\ f\ {\isasymLongrightarrow}\ {\isasymexists}l{\isachardot}{\kern0pt}\ {\isasymforall}v{\isachardot}{\kern0pt}\ eval\ f\ v\ {\isacharequal}{\kern0pt}\ l{\isachardoublequoteclose}\isanewline
%
\isadelimproof
\ \ %
\endisadelimproof
%
\isatagproof
\isakeywordONE{by}\isamarkupfalse%
\ {\isacharparenleft}{\kern0pt}induction\ f{\isacharparenright}{\kern0pt}\ auto%
\endisatagproof
{\isafoldproof}%
%
\isadelimproof
%
\endisadelimproof
%
\begin{isamarkuptext}%
A regular language expression which is constant and \isa{reg{\isacharunderscore}{\kern0pt}eval}, evaluates to some regular
language, independent on the valuation%
\end{isamarkuptext}\isamarkuptrue%
\isakeywordONE{lemma}\isamarkupfalse%
\ const{\isacharunderscore}{\kern0pt}rlexp{\isacharunderscore}{\kern0pt}regular{\isacharunderscore}{\kern0pt}lang{\isacharcolon}{\kern0pt}\isanewline
\ \ \isakeywordTWO{assumes}\ {\isachardoublequoteopen}const{\isacharunderscore}{\kern0pt}rlexp\ f{\isachardoublequoteclose}\isanewline
\ \ \ \ \ \ \isakeywordTWO{and}\ {\isachardoublequoteopen}reg{\isacharunderscore}{\kern0pt}eval\ f{\isachardoublequoteclose}\isanewline
\ \ \ \ \isakeywordTWO{shows}\ {\isachardoublequoteopen}{\isasymexists}l{\isachardot}{\kern0pt}\ regular{\isacharunderscore}{\kern0pt}lang\ l\ {\isasymand}\ {\isacharparenleft}{\kern0pt}{\isasymforall}v{\isachardot}{\kern0pt}\ eval\ f\ v\ {\isacharequal}{\kern0pt}\ l{\isacharparenright}{\kern0pt}{\isachardoublequoteclose}\isanewline
%
\isadelimproof
\ \ %
\endisadelimproof
%
\isatagproof
\isakeywordONE{using}\isamarkupfalse%
\ assms\ const{\isacharunderscore}{\kern0pt}rlexp{\isacharunderscore}{\kern0pt}lang\ reg{\isacharunderscore}{\kern0pt}eval{\isacharunderscore}{\kern0pt}regular\ \isakeywordONE{by}\isamarkupfalse%
\ fastforce%
\endisatagproof
{\isafoldproof}%
%
\isadelimproof
\isanewline
%
\endisadelimproof
%
\isadelimtheory
\isanewline
%
\endisadelimtheory
%
\isatagtheory
\isakeywordTWO{end}\isamarkupfalse%
%
\endisatagtheory
{\isafoldtheory}%
%
\isadelimtheory
%
\endisadelimtheory
%
\end{isabellebody}%
\endinput
%:%file=~/studium/semester_7/semantik/homeworks/AIST/Parikh/Lfun.thy%:%
%:%10=1%:%
%:%11=1%:%
%:%12=2%:%
%:%13=3%:%
%:%14=4%:%
%:%15=5%:%
%:%29=8%:%
%:%41=10%:%
%:%43=11%:%
%:%44=11%:%
%:%45=12%:%
%:%46=13%:%
%:%47=14%:%
%:%48=15%:%
%:%50=18%:%
%:%52=19%:%
%:%53=19%:%
%:%55=21%:%
%:%57=22%:%
%:%58=22%:%
%:%59=23%:%
%:%60=24%:%
%:%61=25%:%
%:%62=26%:%
%:%63=27%:%
%:%65=29%:%
%:%67=30%:%
%:%68=30%:%
%:%69=31%:%
%:%70=32%:%
%:%71=33%:%
%:%72=34%:%
%:%73=35%:%
%:%75=37%:%
%:%77=38%:%
%:%78=38%:%
%:%79=39%:%
%:%80=40%:%
%:%81=41%:%
%:%82=42%:%
%:%83=43%:%
%:%90=47%:%
%:%100=49%:%
%:%101=49%:%
%:%102=50%:%
%:%103=51%:%
%:%106=52%:%
%:%110=52%:%
%:%111=52%:%
%:%112=52%:%
%:%117=52%:%
%:%120=53%:%
%:%121=54%:%
%:%122=54%:%
%:%123=55%:%
%:%126=56%:%
%:%130=56%:%
%:%131=56%:%
%:%132=56%:%
%:%137=56%:%
%:%140=57%:%
%:%141=58%:%
%:%142=58%:%
%:%145=59%:%
%:%149=59%:%
%:%150=59%:%
%:%151=59%:%
%:%156=59%:%
%:%159=60%:%
%:%160=61%:%
%:%161=62%:%
%:%162=62%:%
%:%165=63%:%
%:%169=63%:%
%:%170=63%:%
%:%175=63%:%
%:%178=64%:%
%:%179=65%:%
%:%180=65%:%
%:%183=66%:%
%:%187=66%:%
%:%188=66%:%
%:%189=66%:%
%:%194=66%:%
%:%197=67%:%
%:%198=68%:%
%:%199=69%:%
%:%200=69%:%
%:%207=70%:%
%:%208=70%:%
%:%209=71%:%
%:%210=71%:%
%:%211=72%:%
%:%212=72%:%
%:%213=73%:%
%:%214=73%:%
%:%215=74%:%
%:%216=74%:%
%:%217=74%:%
%:%218=74%:%
%:%219=74%:%
%:%220=75%:%
%:%221=75%:%
%:%222=75%:%
%:%223=75%:%
%:%224=76%:%
%:%230=76%:%
%:%233=77%:%
%:%234=78%:%
%:%235=78%:%
%:%236=79%:%
%:%237=80%:%
%:%244=81%:%
%:%245=81%:%
%:%246=82%:%
%:%247=82%:%
%:%248=83%:%
%:%249=83%:%
%:%250=84%:%
%:%251=84%:%
%:%252=85%:%
%:%253=85%:%
%:%254=85%:%
%:%255=86%:%
%:%256=86%:%
%:%257=87%:%
%:%258=87%:%
%:%259=88%:%
%:%260=88%:%
%:%261=88%:%
%:%262=88%:%
%:%263=88%:%
%:%264=89%:%
%:%265=89%:%
%:%266=90%:%
%:%267=90%:%
%:%268=91%:%
%:%269=91%:%
%:%270=91%:%
%:%271=91%:%
%:%272=92%:%
%:%273=92%:%
%:%274=92%:%
%:%275=92%:%
%:%276=92%:%
%:%277=93%:%
%:%278=93%:%
%:%279=94%:%
%:%285=94%:%
%:%288=95%:%
%:%289=96%:%
%:%290=96%:%
%:%291=97%:%
%:%292=98%:%
%:%295=99%:%
%:%299=99%:%
%:%300=99%:%
%:%301=99%:%
%:%306=99%:%
%:%309=100%:%
%:%310=101%:%
%:%311=101%:%
%:%312=102%:%
%:%313=103%:%
%:%316=104%:%
%:%320=104%:%
%:%321=104%:%
%:%322=104%:%
%:%327=104%:%
%:%330=105%:%
%:%331=106%:%
%:%332=106%:%
%:%333=107%:%
%:%334=108%:%
%:%341=109%:%
%:%342=109%:%
%:%343=110%:%
%:%344=110%:%
%:%345=111%:%
%:%346=111%:%
%:%347=112%:%
%:%348=112%:%
%:%349=112%:%
%:%350=112%:%
%:%351=113%:%
%:%352=113%:%
%:%353=113%:%
%:%354=113%:%
%:%355=114%:%
%:%356=115%:%
%:%357=115%:%
%:%358=115%:%
%:%359=115%:%
%:%360=115%:%
%:%361=116%:%
%:%362=116%:%
%:%363=116%:%
%:%364=117%:%
%:%365=117%:%
%:%366=117%:%
%:%367=118%:%
%:%368=118%:%
%:%369=118%:%
%:%370=118%:%
%:%371=119%:%
%:%386=123%:%
%:%396=125%:%
%:%397=125%:%
%:%398=126%:%
%:%399=127%:%
%:%406=128%:%
%:%407=128%:%
%:%408=128%:%
%:%409=129%:%
%:%410=129%:%
%:%411=130%:%
%:%412=130%:%
%:%413=130%:%
%:%414=131%:%
%:%415=131%:%
%:%416=132%:%
%:%417=132%:%
%:%426=134%:%
%:%428=135%:%
%:%429=135%:%
%:%430=136%:%
%:%431=137%:%
%:%434=138%:%
%:%438=138%:%
%:%439=138%:%
%:%440=138%:%
%:%454=142%:%
%:%464=144%:%
%:%465=144%:%
%:%466=145%:%
%:%467=146%:%
%:%468=147%:%
%:%475=148%:%
%:%476=148%:%
%:%477=148%:%
%:%486=150%:%
%:%488=151%:%
%:%489=151%:%
%:%490=152%:%
%:%491=153%:%
%:%492=154%:%
%:%499=155%:%
%:%500=155%:%
%:%501=156%:%
%:%502=156%:%
%:%503=156%:%
%:%504=156%:%
%:%505=157%:%
%:%506=157%:%
%:%507=157%:%
%:%508=158%:%
%:%509=158%:%
%:%510=158%:%
%:%511=159%:%
%:%512=159%:%
%:%513=159%:%
%:%514=160%:%
%:%520=160%:%
%:%523=161%:%
%:%524=162%:%
%:%525=162%:%
%:%526=163%:%
%:%527=164%:%
%:%528=165%:%
%:%535=166%:%
%:%536=166%:%
%:%537=166%:%
%:%538=167%:%
%:%539=167%:%
%:%540=168%:%
%:%541=168%:%
%:%542=168%:%
%:%543=168%:%
%:%544=169%:%
%:%545=169%:%
%:%546=170%:%
%:%547=170%:%
%:%548=171%:%
%:%549=171%:%
%:%550=171%:%
%:%551=172%:%
%:%552=172%:%
%:%553=173%:%
%:%554=173%:%
%:%555=173%:%
%:%556=173%:%
%:%557=174%:%
%:%558=174%:%
%:%559=174%:%
%:%560=174%:%
%:%561=175%:%
%:%562=175%:%
%:%563=176%:%
%:%564=176%:%
%:%565=176%:%
%:%566=177%:%
%:%567=177%:%
%:%568=178%:%
%:%569=178%:%
%:%570=178%:%
%:%571=179%:%
%:%572=179%:%
%:%573=180%:%
%:%574=180%:%
%:%575=180%:%
%:%576=180%:%
%:%577=181%:%
%:%578=181%:%
%:%579=181%:%
%:%580=181%:%
%:%581=181%:%
%:%582=182%:%
%:%592=184%:%
%:%594=185%:%
%:%595=185%:%
%:%596=186%:%
%:%597=187%:%
%:%598=188%:%
%:%605=189%:%
%:%606=189%:%
%:%607=189%:%
%:%608=190%:%
%:%609=190%:%
%:%610=191%:%
%:%611=191%:%
%:%612=191%:%
%:%613=192%:%
%:%614=192%:%
%:%615=193%:%
%:%616=193%:%
%:%617=193%:%
%:%618=193%:%
%:%619=194%:%
%:%620=194%:%
%:%621=194%:%
%:%622=194%:%
%:%623=195%:%
%:%624=195%:%
%:%625=196%:%
%:%626=196%:%
%:%627=196%:%
%:%628=197%:%
%:%629=197%:%
%:%630=198%:%
%:%631=198%:%
%:%632=198%:%
%:%633=198%:%
%:%634=199%:%
%:%635=199%:%
%:%636=200%:%
%:%637=200%:%
%:%638=200%:%
%:%639=200%:%
%:%640=200%:%
%:%641=201%:%
%:%642=201%:%
%:%643=202%:%
%:%644=202%:%
%:%645=203%:%
%:%646=203%:%
%:%647=203%:%
%:%648=204%:%
%:%649=204%:%
%:%650=204%:%
%:%651=205%:%
%:%652=205%:%
%:%653=205%:%
%:%654=205%:%
%:%655=205%:%
%:%656=206%:%
%:%657=206%:%
%:%658=206%:%
%:%659=206%:%
%:%660=206%:%
%:%661=207%:%
%:%662=207%:%
%:%663=207%:%
%:%664=207%:%
%:%665=208%:%
%:%666=208%:%
%:%675=211%:%
%:%677=212%:%
%:%678=212%:%
%:%679=213%:%
%:%680=214%:%
%:%687=215%:%
%:%688=215%:%
%:%689=216%:%
%:%690=216%:%
%:%691=216%:%
%:%692=216%:%
%:%693=216%:%
%:%694=217%:%
%:%695=217%:%
%:%696=217%:%
%:%697=217%:%
%:%698=217%:%
%:%699=218%:%
%:%714=222%:%
%:%726=224%:%
%:%727=225%:%
%:%728=226%:%
%:%729=227%:%
%:%730=228%:%
%:%731=229%:%
%:%732=230%:%
%:%733=231%:%
%:%735=233%:%
%:%736=233%:%
%:%737=234%:%
%:%738=235%:%
%:%739=236%:%
%:%740=237%:%
%:%741=238%:%
%:%742=239%:%
%:%743=240%:%
%:%744=241%:%
%:%745=241%:%
%:%748=242%:%
%:%752=242%:%
%:%753=242%:%
%:%754=242%:%
%:%759=242%:%
%:%762=243%:%
%:%763=244%:%
%:%764=244%:%
%:%767=245%:%
%:%771=245%:%
%:%772=245%:%
%:%773=245%:%
%:%782=248%:%
%:%783=249%:%
%:%785=250%:%
%:%786=250%:%
%:%787=251%:%
%:%788=252%:%
%:%789=253%:%
%:%796=254%:%
%:%797=254%:%
%:%798=254%:%
%:%799=255%:%
%:%800=255%:%
%:%801=256%:%
%:%802=256%:%
%:%803=256%:%
%:%804=256%:%
%:%805=257%:%
%:%806=257%:%
%:%807=257%:%
%:%808=257%:%
%:%809=258%:%
%:%810=258%:%
%:%811=258%:%
%:%812=258%:%
%:%813=259%:%
%:%814=259%:%
%:%815=260%:%
%:%816=260%:%
%:%817=261%:%
%:%818=261%:%
%:%819=261%:%
%:%820=261%:%
%:%821=262%:%
%:%822=262%:%
%:%823=262%:%
%:%824=262%:%
%:%825=263%:%
%:%826=263%:%
%:%827=263%:%
%:%828=263%:%
%:%829=264%:%
%:%830=264%:%
%:%831=265%:%
%:%832=265%:%
%:%833=266%:%
%:%834=266%:%
%:%835=266%:%
%:%836=266%:%
%:%837=267%:%
%:%838=267%:%
%:%839=267%:%
%:%840=267%:%
%:%841=268%:%
%:%842=268%:%
%:%843=268%:%
%:%844=268%:%
%:%845=269%:%
%:%846=269%:%
%:%855=272%:%
%:%856=273%:%
%:%858=274%:%
%:%859=274%:%
%:%860=275%:%
%:%861=276%:%
%:%862=277%:%
%:%865=278%:%
%:%869=278%:%
%:%870=278%:%
%:%871=278%:%
%:%876=278%:%
%:%879=279%:%
%:%880=280%:%
%:%881=281%:%
%:%882=281%:%
%:%883=282%:%
%:%884=283%:%
%:%885=284%:%
%:%888=285%:%
%:%892=285%:%
%:%893=285%:%
%:%894=285%:%
%:%903=288%:%
%:%904=289%:%
%:%906=291%:%
%:%907=291%:%
%:%908=292%:%
%:%909=293%:%
%:%916=294%:%
%:%917=294%:%
%:%918=295%:%
%:%919=295%:%
%:%920=296%:%
%:%921=296%:%
%:%922=296%:%
%:%923=296%:%
%:%924=296%:%
%:%925=297%:%
%:%926=297%:%
%:%927=298%:%
%:%928=298%:%
%:%929=299%:%
%:%930=299%:%
%:%931=299%:%
%:%932=300%:%
%:%933=300%:%
%:%934=301%:%
%:%935=301%:%
%:%936=302%:%
%:%937=302%:%
%:%938=302%:%
%:%939=303%:%
%:%940=303%:%
%:%941=304%:%
%:%942=304%:%
%:%943=304%:%
%:%944=304%:%
%:%945=305%:%
%:%951=305%:%
%:%954=306%:%
%:%955=307%:%
%:%956=307%:%
%:%957=308%:%
%:%958=309%:%
%:%959=310%:%
%:%962=311%:%
%:%966=311%:%
%:%967=311%:%
%:%968=311%:%
%:%982=315%:%
%:%994=317%:%
%:%996=318%:%
%:%997=318%:%
%:%998=319%:%
%:%1000=321%:%
%:%1001=322%:%
%:%1003=323%:%
%:%1004=323%:%
%:%1007=324%:%
%:%1011=324%:%
%:%1012=324%:%
%:%1021=326%:%
%:%1022=327%:%
%:%1024=328%:%
%:%1025=328%:%
%:%1026=329%:%
%:%1027=330%:%
%:%1028=331%:%
%:%1031=332%:%
%:%1035=332%:%
%:%1036=332%:%
%:%1037=332%:%
%:%1042=332%:%
%:%1047=333%:%
%:%1052=334%:%

%
\begin{isabellebody}%
\setisabellecontext{Parikh{\isacharunderscore}{\kern0pt}Img}%
%
\isadelimdocument
%
\endisadelimdocument
%
\isatagdocument
%
\isamarkupsection{Parikh images%
}
\isamarkuptrue%
%
\endisatagdocument
{\isafolddocument}%
%
\isadelimdocument
%
\endisadelimdocument
%
\isadelimtheory
%
\endisadelimtheory
%
\isatagtheory
\isakeywordONE{theory}\isamarkupfalse%
\ Parikh{\isacharunderscore}{\kern0pt}Img\isanewline
\ \ \isakeywordTWO{imports}\ \isanewline
\ \ \ \ {\isachardoublequoteopen}Reg{\isacharunderscore}{\kern0pt}Lang{\isacharunderscore}{\kern0pt}Exp{\isachardoublequoteclose}\isanewline
\ \ \ \ {\isachardoublequoteopen}HOL{\isacharminus}{\kern0pt}Library{\isachardot}{\kern0pt}Multiset{\isachardoublequoteclose}\isanewline
\isakeywordTWO{begin}%
\endisatagtheory
{\isafoldtheory}%
%
\isadelimtheory
%
\endisadelimtheory
%
\isadelimdocument
%
\endisadelimdocument
%
\isatagdocument
%
\isamarkupsubsection{Definition and basic lemmas%
}
\isamarkuptrue%
%
\endisatagdocument
{\isafolddocument}%
%
\isadelimdocument
%
\endisadelimdocument
%
\begin{isamarkuptext}%
The Parikh vector of a finite word describes how often each symbol of the alphabet occurs in the word.
We represent parikh vectors by multisets. The Parikh image of a language \isa{L}, denoted by \isa{{\isasymPsi}\ L},
is then the set of Parikh vectors of all words in the language.%
\end{isamarkuptext}\isamarkuptrue%
\isakeywordONE{abbreviation}\isamarkupfalse%
\ parikh{\isacharunderscore}{\kern0pt}vec\ \isakeywordTWO{where}\isanewline
\ \ {\isachardoublequoteopen}parikh{\isacharunderscore}{\kern0pt}vec\ {\isasymequiv}\ mset{\isachardoublequoteclose}\isanewline
\isanewline
\isakeywordONE{definition}\isamarkupfalse%
\ parikh{\isacharunderscore}{\kern0pt}img\ {\isacharcolon}{\kern0pt}{\isacharcolon}{\kern0pt}\ {\isachardoublequoteopen}{\isacharprime}{\kern0pt}a\ lang\ {\isasymRightarrow}\ {\isacharprime}{\kern0pt}a\ multiset\ set{\isachardoublequoteclose}\ {\isacharparenleft}{\kern0pt}{\isachardoublequoteopen}{\isasymPsi}{\isachardoublequoteclose}{\isacharparenright}{\kern0pt}\ \isakeywordTWO{where}\isanewline
\ \ {\isachardoublequoteopen}{\isasymPsi}\ L\ {\isasymequiv}\ parikh{\isacharunderscore}{\kern0pt}vec\ {\isacharbackquote}{\kern0pt}\ L{\isachardoublequoteclose}\isanewline
\isanewline
\isakeywordONE{lemma}\isamarkupfalse%
\ parikh{\isacharunderscore}{\kern0pt}img{\isacharunderscore}{\kern0pt}Un\ {\isacharbrackleft}{\kern0pt}simp{\isacharbrackright}{\kern0pt}{\isacharcolon}{\kern0pt}\ {\isachardoublequoteopen}{\isasymPsi}\ {\isacharparenleft}{\kern0pt}L{\isadigit{1}}\ {\isasymunion}\ L{\isadigit{2}}{\isacharparenright}{\kern0pt}\ {\isacharequal}{\kern0pt}\ {\isasymPsi}\ L{\isadigit{1}}\ {\isasymunion}\ {\isasymPsi}\ L{\isadigit{2}}{\isachardoublequoteclose}\isanewline
%
\isadelimproof
\ \ %
\endisadelimproof
%
\isatagproof
\isakeywordONE{by}\isamarkupfalse%
\ {\isacharparenleft}{\kern0pt}auto\ simp\ add{\isacharcolon}{\kern0pt}\ parikh{\isacharunderscore}{\kern0pt}img{\isacharunderscore}{\kern0pt}def{\isacharparenright}{\kern0pt}%
\endisatagproof
{\isafoldproof}%
%
\isadelimproof
\isanewline
%
\endisadelimproof
\isanewline
\isakeywordONE{lemma}\isamarkupfalse%
\ parikh{\isacharunderscore}{\kern0pt}img{\isacharunderscore}{\kern0pt}UNION{\isacharcolon}{\kern0pt}\ {\isachardoublequoteopen}{\isasymPsi}\ {\isacharparenleft}{\kern0pt}{\isasymUnion}{\isacharparenleft}{\kern0pt}L\ {\isacharbackquote}{\kern0pt}\ I{\isacharparenright}{\kern0pt}{\isacharparenright}{\kern0pt}\ {\isacharequal}{\kern0pt}\ {\isasymUnion}\ {\isacharparenleft}{\kern0pt}{\isacharparenleft}{\kern0pt}{\isasymlambda}i{\isachardot}{\kern0pt}\ {\isasymPsi}\ {\isacharparenleft}{\kern0pt}L\ i{\isacharparenright}{\kern0pt}{\isacharparenright}{\kern0pt}\ {\isacharbackquote}{\kern0pt}\ I{\isacharparenright}{\kern0pt}{\isachardoublequoteclose}\isanewline
%
\isadelimproof
\ \ %
\endisadelimproof
%
\isatagproof
\isakeywordONE{by}\isamarkupfalse%
\ {\isacharparenleft}{\kern0pt}auto\ simp\ add{\isacharcolon}{\kern0pt}\ parikh{\isacharunderscore}{\kern0pt}img{\isacharunderscore}{\kern0pt}def{\isacharparenright}{\kern0pt}%
\endisatagproof
{\isafoldproof}%
%
\isadelimproof
\isanewline
%
\endisadelimproof
\isanewline
\isakeywordONE{lemma}\isamarkupfalse%
\ parikh{\isacharunderscore}{\kern0pt}img{\isacharunderscore}{\kern0pt}conc{\isacharcolon}{\kern0pt}\ {\isachardoublequoteopen}{\isasymPsi}\ {\isacharparenleft}{\kern0pt}L{\isadigit{1}}\ {\isacharat}{\kern0pt}{\isacharat}{\kern0pt}\ L{\isadigit{2}}{\isacharparenright}{\kern0pt}\ {\isacharequal}{\kern0pt}\ {\isacharbraceleft}{\kern0pt}\ m{\isadigit{1}}\ {\isacharplus}{\kern0pt}\ m{\isadigit{2}}\ {\isacharbar}{\kern0pt}\ m{\isadigit{1}}\ m{\isadigit{2}}{\isachardot}{\kern0pt}\ m{\isadigit{1}}\ {\isasymin}\ {\isasymPsi}\ L{\isadigit{1}}\ {\isasymand}\ m{\isadigit{2}}\ {\isasymin}\ {\isasymPsi}\ L{\isadigit{2}}\ {\isacharbraceright}{\kern0pt}{\isachardoublequoteclose}\isanewline
%
\isadelimproof
\ \ %
\endisadelimproof
%
\isatagproof
\isakeywordONE{unfolding}\isamarkupfalse%
\ parikh{\isacharunderscore}{\kern0pt}img{\isacharunderscore}{\kern0pt}def\ \isakeywordONE{by}\isamarkupfalse%
\ force%
\endisatagproof
{\isafoldproof}%
%
\isadelimproof
\isanewline
%
\endisadelimproof
\isanewline
\isakeywordONE{lemma}\isamarkupfalse%
\ parikh{\isacharunderscore}{\kern0pt}img{\isacharunderscore}{\kern0pt}commut{\isacharcolon}{\kern0pt}\ {\isachardoublequoteopen}{\isasymPsi}\ {\isacharparenleft}{\kern0pt}L{\isadigit{1}}\ {\isacharat}{\kern0pt}{\isacharat}{\kern0pt}\ L{\isadigit{2}}{\isacharparenright}{\kern0pt}\ {\isacharequal}{\kern0pt}\ {\isasymPsi}\ {\isacharparenleft}{\kern0pt}L{\isadigit{2}}\ {\isacharat}{\kern0pt}{\isacharat}{\kern0pt}\ L{\isadigit{1}}{\isacharparenright}{\kern0pt}{\isachardoublequoteclose}\isanewline
%
\isadelimproof
%
\endisadelimproof
%
\isatagproof
\isakeywordONE{proof}\isamarkupfalse%
\ {\isacharminus}{\kern0pt}\isanewline
\ \ \isakeywordONE{have}\isamarkupfalse%
\ {\isachardoublequoteopen}{\isacharbraceleft}{\kern0pt}\ m{\isadigit{1}}\ {\isacharplus}{\kern0pt}\ m{\isadigit{2}}\ {\isacharbar}{\kern0pt}\ m{\isadigit{1}}\ m{\isadigit{2}}{\isachardot}{\kern0pt}\ m{\isadigit{1}}\ {\isasymin}\ {\isasymPsi}\ L{\isadigit{1}}\ {\isasymand}\ m{\isadigit{2}}\ {\isasymin}\ {\isasymPsi}\ L{\isadigit{2}}\ {\isacharbraceright}{\kern0pt}\ {\isacharequal}{\kern0pt}\ \isanewline
\ \ \ \ \ \ \ \ {\isacharbraceleft}{\kern0pt}\ m{\isadigit{2}}\ {\isacharplus}{\kern0pt}\ m{\isadigit{1}}\ {\isacharbar}{\kern0pt}\ m{\isadigit{1}}\ m{\isadigit{2}}{\isachardot}{\kern0pt}\ m{\isadigit{1}}\ {\isasymin}\ {\isasymPsi}\ L{\isadigit{1}}\ {\isasymand}\ m{\isadigit{2}}\ {\isasymin}\ {\isasymPsi}\ L{\isadigit{2}}\ {\isacharbraceright}{\kern0pt}{\isachardoublequoteclose}\isanewline
\ \ \ \ \isakeywordONE{using}\isamarkupfalse%
\ add{\isachardot}{\kern0pt}commute\ \isakeywordONE{by}\isamarkupfalse%
\ blast\isanewline
\ \ \isakeywordONE{then}\isamarkupfalse%
\ \isakeywordTHREE{show}\isamarkupfalse%
\ {\isacharquery}{\kern0pt}thesis\isanewline
\ \ \ \ \isakeywordONE{using}\isamarkupfalse%
\ parikh{\isacharunderscore}{\kern0pt}img{\isacharunderscore}{\kern0pt}conc{\isacharbrackleft}{\kern0pt}of\ L{\isadigit{1}}{\isacharbrackright}{\kern0pt}\ parikh{\isacharunderscore}{\kern0pt}img{\isacharunderscore}{\kern0pt}conc{\isacharbrackleft}{\kern0pt}of\ L{\isadigit{2}}{\isacharbrackright}{\kern0pt}\ \isakeywordONE{by}\isamarkupfalse%
\ auto\isanewline
\isakeywordONE{qed}\isamarkupfalse%
%
\endisatagproof
{\isafoldproof}%
%
\isadelimproof
%
\endisadelimproof
%
\isadelimdocument
%
\endisadelimdocument
%
\isatagdocument
%
\isamarkupsubsection{Monotonicity properties%
}
\isamarkuptrue%
%
\endisatagdocument
{\isafolddocument}%
%
\isadelimdocument
%
\endisadelimdocument
\isakeywordONE{lemma}\isamarkupfalse%
\ parikh{\isacharunderscore}{\kern0pt}img{\isacharunderscore}{\kern0pt}mono{\isacharcolon}{\kern0pt}\ {\isachardoublequoteopen}A\ {\isasymsubseteq}\ B\ {\isasymLongrightarrow}\ {\isasymPsi}\ A\ {\isasymsubseteq}\ {\isasymPsi}\ B{\isachardoublequoteclose}\isanewline
%
\isadelimproof
\ \ %
\endisadelimproof
%
\isatagproof
\isakeywordONE{unfolding}\isamarkupfalse%
\ parikh{\isacharunderscore}{\kern0pt}img{\isacharunderscore}{\kern0pt}def\ \isakeywordONE{by}\isamarkupfalse%
\ fast%
\endisatagproof
{\isafoldproof}%
%
\isadelimproof
\isanewline
%
\endisadelimproof
\isanewline
\isakeywordONE{lemma}\isamarkupfalse%
\ parikh{\isacharunderscore}{\kern0pt}conc{\isacharunderscore}{\kern0pt}right{\isacharunderscore}{\kern0pt}subset{\isacharcolon}{\kern0pt}\ {\isachardoublequoteopen}{\isasymPsi}\ A\ {\isasymsubseteq}\ {\isasymPsi}\ B\ {\isasymLongrightarrow}\ {\isasymPsi}\ {\isacharparenleft}{\kern0pt}A\ {\isacharat}{\kern0pt}{\isacharat}{\kern0pt}\ C{\isacharparenright}{\kern0pt}\ {\isasymsubseteq}\ {\isasymPsi}\ {\isacharparenleft}{\kern0pt}B\ {\isacharat}{\kern0pt}{\isacharat}{\kern0pt}\ C{\isacharparenright}{\kern0pt}{\isachardoublequoteclose}\isanewline
%
\isadelimproof
\ \ %
\endisadelimproof
%
\isatagproof
\isakeywordONE{by}\isamarkupfalse%
\ {\isacharparenleft}{\kern0pt}auto\ simp\ add{\isacharcolon}{\kern0pt}\ parikh{\isacharunderscore}{\kern0pt}img{\isacharunderscore}{\kern0pt}conc{\isacharparenright}{\kern0pt}%
\endisatagproof
{\isafoldproof}%
%
\isadelimproof
\isanewline
%
\endisadelimproof
\isanewline
\isakeywordONE{lemma}\isamarkupfalse%
\ parikh{\isacharunderscore}{\kern0pt}conc{\isacharunderscore}{\kern0pt}left{\isacharunderscore}{\kern0pt}subset{\isacharcolon}{\kern0pt}\ {\isachardoublequoteopen}{\isasymPsi}\ A\ {\isasymsubseteq}\ {\isasymPsi}\ B\ {\isasymLongrightarrow}\ {\isasymPsi}\ {\isacharparenleft}{\kern0pt}C\ {\isacharat}{\kern0pt}{\isacharat}{\kern0pt}\ A{\isacharparenright}{\kern0pt}\ {\isasymsubseteq}\ {\isasymPsi}\ {\isacharparenleft}{\kern0pt}C\ {\isacharat}{\kern0pt}{\isacharat}{\kern0pt}\ B{\isacharparenright}{\kern0pt}{\isachardoublequoteclose}\isanewline
%
\isadelimproof
\ \ %
\endisadelimproof
%
\isatagproof
\isakeywordONE{by}\isamarkupfalse%
\ {\isacharparenleft}{\kern0pt}auto\ simp\ add{\isacharcolon}{\kern0pt}\ parikh{\isacharunderscore}{\kern0pt}img{\isacharunderscore}{\kern0pt}conc{\isacharparenright}{\kern0pt}%
\endisatagproof
{\isafoldproof}%
%
\isadelimproof
\isanewline
%
\endisadelimproof
\isanewline
\isakeywordONE{lemma}\isamarkupfalse%
\ parikh{\isacharunderscore}{\kern0pt}conc{\isacharunderscore}{\kern0pt}subset{\isacharcolon}{\kern0pt}\isanewline
\ \ \isakeywordTWO{assumes}\ {\isachardoublequoteopen}{\isasymPsi}\ A\ {\isasymsubseteq}\ {\isasymPsi}\ C{\isachardoublequoteclose}\isanewline
\ \ \ \ \ \ \isakeywordTWO{and}\ {\isachardoublequoteopen}{\isasymPsi}\ B\ {\isasymsubseteq}\ {\isasymPsi}\ D{\isachardoublequoteclose}\isanewline
\ \ \ \ \isakeywordTWO{shows}\ {\isachardoublequoteopen}{\isasymPsi}\ {\isacharparenleft}{\kern0pt}A\ {\isacharat}{\kern0pt}{\isacharat}{\kern0pt}\ B{\isacharparenright}{\kern0pt}\ {\isasymsubseteq}\ {\isasymPsi}\ {\isacharparenleft}{\kern0pt}C\ {\isacharat}{\kern0pt}{\isacharat}{\kern0pt}\ D{\isacharparenright}{\kern0pt}{\isachardoublequoteclose}\isanewline
%
\isadelimproof
\ \ %
\endisadelimproof
%
\isatagproof
\isakeywordONE{using}\isamarkupfalse%
\ assms\ parikh{\isacharunderscore}{\kern0pt}conc{\isacharunderscore}{\kern0pt}right{\isacharunderscore}{\kern0pt}subset\ parikh{\isacharunderscore}{\kern0pt}conc{\isacharunderscore}{\kern0pt}left{\isacharunderscore}{\kern0pt}subset\ \isakeywordONE{by}\isamarkupfalse%
\ blast%
\endisatagproof
{\isafoldproof}%
%
\isadelimproof
\isanewline
%
\endisadelimproof
\isanewline
\isakeywordONE{lemma}\isamarkupfalse%
\ parikh{\isacharunderscore}{\kern0pt}conc{\isacharunderscore}{\kern0pt}right{\isacharcolon}{\kern0pt}\ {\isachardoublequoteopen}{\isasymPsi}\ A\ {\isacharequal}{\kern0pt}\ {\isasymPsi}\ B\ {\isasymLongrightarrow}\ {\isasymPsi}\ {\isacharparenleft}{\kern0pt}A\ {\isacharat}{\kern0pt}{\isacharat}{\kern0pt}\ C{\isacharparenright}{\kern0pt}\ {\isacharequal}{\kern0pt}\ {\isasymPsi}\ {\isacharparenleft}{\kern0pt}B\ {\isacharat}{\kern0pt}{\isacharat}{\kern0pt}\ C{\isacharparenright}{\kern0pt}{\isachardoublequoteclose}\isanewline
%
\isadelimproof
\ \ %
\endisadelimproof
%
\isatagproof
\isakeywordONE{by}\isamarkupfalse%
\ {\isacharparenleft}{\kern0pt}auto\ simp\ add{\isacharcolon}{\kern0pt}\ parikh{\isacharunderscore}{\kern0pt}img{\isacharunderscore}{\kern0pt}conc{\isacharparenright}{\kern0pt}%
\endisatagproof
{\isafoldproof}%
%
\isadelimproof
\isanewline
%
\endisadelimproof
\isanewline
\isakeywordONE{lemma}\isamarkupfalse%
\ parikh{\isacharunderscore}{\kern0pt}conc{\isacharunderscore}{\kern0pt}left{\isacharcolon}{\kern0pt}\ {\isachardoublequoteopen}{\isasymPsi}\ A\ {\isacharequal}{\kern0pt}\ {\isasymPsi}\ B\ {\isasymLongrightarrow}\ {\isasymPsi}\ {\isacharparenleft}{\kern0pt}C\ {\isacharat}{\kern0pt}{\isacharat}{\kern0pt}\ A{\isacharparenright}{\kern0pt}\ {\isacharequal}{\kern0pt}\ {\isasymPsi}\ {\isacharparenleft}{\kern0pt}C\ {\isacharat}{\kern0pt}{\isacharat}{\kern0pt}\ B{\isacharparenright}{\kern0pt}{\isachardoublequoteclose}\isanewline
%
\isadelimproof
\ \ %
\endisadelimproof
%
\isatagproof
\isakeywordONE{by}\isamarkupfalse%
\ {\isacharparenleft}{\kern0pt}auto\ simp\ add{\isacharcolon}{\kern0pt}\ parikh{\isacharunderscore}{\kern0pt}img{\isacharunderscore}{\kern0pt}conc{\isacharparenright}{\kern0pt}%
\endisatagproof
{\isafoldproof}%
%
\isadelimproof
\isanewline
%
\endisadelimproof
\isanewline
\isakeywordONE{lemma}\isamarkupfalse%
\ parikh{\isacharunderscore}{\kern0pt}pow{\isacharunderscore}{\kern0pt}mono{\isacharcolon}{\kern0pt}\ {\isachardoublequoteopen}{\isasymPsi}\ A\ {\isasymsubseteq}\ {\isasymPsi}\ B\ {\isasymLongrightarrow}\ {\isasymPsi}\ {\isacharparenleft}{\kern0pt}A\ {\isacharcircum}{\kern0pt}{\isacharcircum}{\kern0pt}\ n{\isacharparenright}{\kern0pt}\ {\isasymsubseteq}\ {\isasymPsi}\ {\isacharparenleft}{\kern0pt}B\ {\isacharcircum}{\kern0pt}{\isacharcircum}{\kern0pt}\ n{\isacharparenright}{\kern0pt}{\isachardoublequoteclose}\isanewline
%
\isadelimproof
\ \ %
\endisadelimproof
%
\isatagproof
\isakeywordONE{by}\isamarkupfalse%
\ {\isacharparenleft}{\kern0pt}induction\ n{\isacharparenright}{\kern0pt}\ {\isacharparenleft}{\kern0pt}auto\ simp\ add{\isacharcolon}{\kern0pt}\ parikh{\isacharunderscore}{\kern0pt}img{\isacharunderscore}{\kern0pt}conc{\isacharparenright}{\kern0pt}%
\endisatagproof
{\isafoldproof}%
%
\isadelimproof
\isanewline
%
\endisadelimproof
\isanewline
\isanewline
\isakeywordONE{lemma}\isamarkupfalse%
\ parikh{\isacharunderscore}{\kern0pt}star{\isacharunderscore}{\kern0pt}mono{\isacharcolon}{\kern0pt}\isanewline
\ \ \isakeywordTWO{assumes}\ {\isachardoublequoteopen}{\isasymPsi}\ A\ {\isasymsubseteq}\ {\isasymPsi}\ B{\isachardoublequoteclose}\isanewline
\ \ \isakeywordTWO{shows}\ {\isachardoublequoteopen}{\isasymPsi}\ {\isacharparenleft}{\kern0pt}star\ A{\isacharparenright}{\kern0pt}\ {\isasymsubseteq}\ {\isasymPsi}\ {\isacharparenleft}{\kern0pt}star\ B{\isacharparenright}{\kern0pt}{\isachardoublequoteclose}\isanewline
%
\isadelimproof
%
\endisadelimproof
%
\isatagproof
\isakeywordONE{proof}\isamarkupfalse%
\isanewline
\ \ \isakeywordTHREE{fix}\isamarkupfalse%
\ v\isanewline
\ \ \isakeywordTHREE{assume}\isamarkupfalse%
\ {\isachardoublequoteopen}v\ {\isasymin}\ {\isasymPsi}\ {\isacharparenleft}{\kern0pt}star\ A{\isacharparenright}{\kern0pt}{\isachardoublequoteclose}\isanewline
\ \ \isakeywordONE{then}\isamarkupfalse%
\ \isakeywordTHREE{obtain}\isamarkupfalse%
\ w\ \isakeywordTWO{where}\ w{\isacharunderscore}{\kern0pt}intro{\isacharcolon}{\kern0pt}\ {\isachardoublequoteopen}parikh{\isacharunderscore}{\kern0pt}vec\ w\ {\isacharequal}{\kern0pt}\ v\ {\isasymand}\ w\ {\isasymin}\ star\ A{\isachardoublequoteclose}\ \isakeywordONE{unfolding}\isamarkupfalse%
\ parikh{\isacharunderscore}{\kern0pt}img{\isacharunderscore}{\kern0pt}def\ \isakeywordONE{by}\isamarkupfalse%
\ blast\isanewline
\ \ \isakeywordONE{then}\isamarkupfalse%
\ \isakeywordTHREE{obtain}\isamarkupfalse%
\ n\ \isakeywordTWO{where}\ {\isachardoublequoteopen}w\ {\isasymin}\ A\ {\isacharcircum}{\kern0pt}{\isacharcircum}{\kern0pt}\ n{\isachardoublequoteclose}\ \isakeywordONE{unfolding}\isamarkupfalse%
\ star{\isacharunderscore}{\kern0pt}def\ \isakeywordONE{by}\isamarkupfalse%
\ blast\isanewline
\ \ \isakeywordONE{then}\isamarkupfalse%
\ \isakeywordONE{have}\isamarkupfalse%
\ {\isachardoublequoteopen}v\ {\isasymin}\ {\isasymPsi}\ {\isacharparenleft}{\kern0pt}A\ {\isacharcircum}{\kern0pt}{\isacharcircum}{\kern0pt}\ n{\isacharparenright}{\kern0pt}{\isachardoublequoteclose}\ \isakeywordONE{using}\isamarkupfalse%
\ w{\isacharunderscore}{\kern0pt}intro\ \isakeywordONE{unfolding}\isamarkupfalse%
\ parikh{\isacharunderscore}{\kern0pt}img{\isacharunderscore}{\kern0pt}def\ \isakeywordONE{by}\isamarkupfalse%
\ blast\isanewline
\ \ \isakeywordONE{with}\isamarkupfalse%
\ assms\ \isakeywordONE{have}\isamarkupfalse%
\ {\isachardoublequoteopen}v\ {\isasymin}\ {\isasymPsi}\ {\isacharparenleft}{\kern0pt}B\ {\isacharcircum}{\kern0pt}{\isacharcircum}{\kern0pt}\ n{\isacharparenright}{\kern0pt}{\isachardoublequoteclose}\ \isakeywordONE{using}\isamarkupfalse%
\ parikh{\isacharunderscore}{\kern0pt}pow{\isacharunderscore}{\kern0pt}mono\ \isakeywordONE{by}\isamarkupfalse%
\ blast\isanewline
\ \ \isakeywordONE{then}\isamarkupfalse%
\ \isakeywordTHREE{show}\isamarkupfalse%
\ {\isachardoublequoteopen}v\ {\isasymin}\ {\isasymPsi}\ {\isacharparenleft}{\kern0pt}star\ B{\isacharparenright}{\kern0pt}{\isachardoublequoteclose}\ \isakeywordONE{unfolding}\isamarkupfalse%
\ star{\isacharunderscore}{\kern0pt}def\ \isakeywordONE{using}\isamarkupfalse%
\ parikh{\isacharunderscore}{\kern0pt}img{\isacharunderscore}{\kern0pt}UNION\ \isakeywordONE{by}\isamarkupfalse%
\ fastforce\isanewline
\isakeywordONE{qed}\isamarkupfalse%
%
\endisatagproof
{\isafoldproof}%
%
\isadelimproof
\isanewline
%
\endisadelimproof
\isanewline
\isakeywordONE{lemma}\isamarkupfalse%
\ parikh{\isacharunderscore}{\kern0pt}star{\isacharunderscore}{\kern0pt}mono{\isacharunderscore}{\kern0pt}eq{\isacharcolon}{\kern0pt}\isanewline
\ \ \isakeywordTWO{assumes}\ {\isachardoublequoteopen}{\isasymPsi}\ A\ {\isacharequal}{\kern0pt}\ {\isasymPsi}\ B{\isachardoublequoteclose}\isanewline
\ \ \isakeywordTWO{shows}\ {\isachardoublequoteopen}{\isasymPsi}\ {\isacharparenleft}{\kern0pt}star\ A{\isacharparenright}{\kern0pt}\ {\isacharequal}{\kern0pt}\ {\isasymPsi}\ {\isacharparenleft}{\kern0pt}star\ B{\isacharparenright}{\kern0pt}{\isachardoublequoteclose}\isanewline
%
\isadelimproof
\ \ %
\endisadelimproof
%
\isatagproof
\isakeywordONE{using}\isamarkupfalse%
\ parikh{\isacharunderscore}{\kern0pt}star{\isacharunderscore}{\kern0pt}mono\ \isakeywordONE{by}\isamarkupfalse%
\ {\isacharparenleft}{\kern0pt}metis\ Orderings{\isachardot}{\kern0pt}order{\isacharunderscore}{\kern0pt}eq{\isacharunderscore}{\kern0pt}iff\ assms{\isacharparenright}{\kern0pt}%
\endisatagproof
{\isafoldproof}%
%
\isadelimproof
\isanewline
%
\endisadelimproof
\isanewline
\isanewline
\isakeywordONE{lemma}\isamarkupfalse%
\ parikh{\isacharunderscore}{\kern0pt}img{\isacharunderscore}{\kern0pt}subst{\isacharunderscore}{\kern0pt}mono{\isacharcolon}{\kern0pt}\isanewline
\ \ \isakeywordTWO{assumes}\ {\isachardoublequoteopen}{\isasymforall}i{\isachardot}{\kern0pt}\ {\isasymPsi}\ {\isacharparenleft}{\kern0pt}eval\ {\isacharparenleft}{\kern0pt}A\ i{\isacharparenright}{\kern0pt}\ v{\isacharparenright}{\kern0pt}\ {\isasymsubseteq}\ {\isasymPsi}\ {\isacharparenleft}{\kern0pt}eval\ {\isacharparenleft}{\kern0pt}B\ i{\isacharparenright}{\kern0pt}\ v{\isacharparenright}{\kern0pt}{\isachardoublequoteclose}\isanewline
\ \ \isakeywordTWO{shows}\ {\isachardoublequoteopen}{\isasymPsi}\ {\isacharparenleft}{\kern0pt}eval\ {\isacharparenleft}{\kern0pt}subst\ A\ f{\isacharparenright}{\kern0pt}\ v{\isacharparenright}{\kern0pt}\ {\isasymsubseteq}\ {\isasymPsi}\ {\isacharparenleft}{\kern0pt}eval\ {\isacharparenleft}{\kern0pt}subst\ B\ f{\isacharparenright}{\kern0pt}\ v{\isacharparenright}{\kern0pt}{\isachardoublequoteclose}\isanewline
%
\isadelimproof
%
\endisadelimproof
%
\isatagproof
\isakeywordONE{proof}\isamarkupfalse%
\ {\isacharparenleft}{\kern0pt}induction\ f{\isacharparenright}{\kern0pt}\isanewline
\ \ \isakeywordTHREE{case}\isamarkupfalse%
\ {\isacharparenleft}{\kern0pt}Concat\ f{\isadigit{1}}\ f{\isadigit{2}}{\isacharparenright}{\kern0pt}\isanewline
\ \ \isakeywordONE{then}\isamarkupfalse%
\ \isakeywordONE{have}\isamarkupfalse%
\ {\isachardoublequoteopen}{\isasymPsi}\ {\isacharparenleft}{\kern0pt}eval\ {\isacharparenleft}{\kern0pt}subst\ A\ f{\isadigit{1}}{\isacharparenright}{\kern0pt}\ v\ {\isacharat}{\kern0pt}{\isacharat}{\kern0pt}\ eval\ {\isacharparenleft}{\kern0pt}subst\ A\ f{\isadigit{2}}{\isacharparenright}{\kern0pt}\ v{\isacharparenright}{\kern0pt}\isanewline
\ \ \ \ \ \ \ \ \ \ \ \ \ \ {\isasymsubseteq}\ {\isasymPsi}\ {\isacharparenleft}{\kern0pt}eval\ {\isacharparenleft}{\kern0pt}subst\ B\ f{\isadigit{1}}{\isacharparenright}{\kern0pt}\ v\ {\isacharat}{\kern0pt}{\isacharat}{\kern0pt}\ eval\ {\isacharparenleft}{\kern0pt}subst\ B\ f{\isadigit{2}}{\isacharparenright}{\kern0pt}\ v{\isacharparenright}{\kern0pt}{\isachardoublequoteclose}\isanewline
\ \ \ \ \isakeywordONE{using}\isamarkupfalse%
\ parikh{\isacharunderscore}{\kern0pt}conc{\isacharunderscore}{\kern0pt}subset\ \isakeywordONE{by}\isamarkupfalse%
\ blast\isanewline
\ \ \isakeywordONE{then}\isamarkupfalse%
\ \isakeywordTHREE{show}\isamarkupfalse%
\ {\isacharquery}{\kern0pt}case\ \isakeywordONE{by}\isamarkupfalse%
\ simp\isanewline
\isakeywordONE{next}\isamarkupfalse%
\isanewline
\ \ \isakeywordTHREE{case}\isamarkupfalse%
\ {\isacharparenleft}{\kern0pt}Star\ f{\isacharparenright}{\kern0pt}\isanewline
\ \ \isakeywordONE{then}\isamarkupfalse%
\ \isakeywordONE{have}\isamarkupfalse%
\ {\isachardoublequoteopen}{\isasymPsi}\ {\isacharparenleft}{\kern0pt}star\ {\isacharparenleft}{\kern0pt}eval\ {\isacharparenleft}{\kern0pt}subst\ A\ f{\isacharparenright}{\kern0pt}\ v{\isacharparenright}{\kern0pt}{\isacharparenright}{\kern0pt}\ {\isasymsubseteq}\ {\isasymPsi}\ {\isacharparenleft}{\kern0pt}star\ {\isacharparenleft}{\kern0pt}eval\ {\isacharparenleft}{\kern0pt}subst\ B\ f{\isacharparenright}{\kern0pt}\ v{\isacharparenright}{\kern0pt}{\isacharparenright}{\kern0pt}{\isachardoublequoteclose}\isanewline
\ \ \ \ \isakeywordONE{using}\isamarkupfalse%
\ parikh{\isacharunderscore}{\kern0pt}star{\isacharunderscore}{\kern0pt}mono\ \isakeywordONE{by}\isamarkupfalse%
\ blast\isanewline
\ \ \isakeywordONE{then}\isamarkupfalse%
\ \isakeywordTHREE{show}\isamarkupfalse%
\ {\isacharquery}{\kern0pt}case\ \isakeywordONE{by}\isamarkupfalse%
\ simp\isanewline
\isakeywordONE{qed}\isamarkupfalse%
\ {\isacharparenleft}{\kern0pt}use\ assms{\isacharparenleft}{\kern0pt}{\isadigit{1}}{\isacharparenright}{\kern0pt}\ \isakeywordTWO{in}\ auto{\isacharparenright}{\kern0pt}%
\endisatagproof
{\isafoldproof}%
%
\isadelimproof
\isanewline
%
\endisadelimproof
\isanewline
\isakeywordONE{lemma}\isamarkupfalse%
\ parikh{\isacharunderscore}{\kern0pt}img{\isacharunderscore}{\kern0pt}subst{\isacharunderscore}{\kern0pt}mono{\isacharunderscore}{\kern0pt}upd{\isacharcolon}{\kern0pt}\isanewline
\ \ \isakeywordTWO{assumes}\ {\isachardoublequoteopen}{\isasymPsi}\ {\isacharparenleft}{\kern0pt}eval\ A\ v{\isacharparenright}{\kern0pt}\ {\isasymsubseteq}\ {\isasymPsi}\ {\isacharparenleft}{\kern0pt}eval\ B\ v{\isacharparenright}{\kern0pt}{\isachardoublequoteclose}\isanewline
\ \ \isakeywordTWO{shows}\ {\isachardoublequoteopen}{\isasymPsi}\ {\isacharparenleft}{\kern0pt}eval\ {\isacharparenleft}{\kern0pt}subst\ {\isacharparenleft}{\kern0pt}Var{\isacharparenleft}{\kern0pt}x\ {\isacharcolon}{\kern0pt}{\isacharequal}{\kern0pt}\ A{\isacharparenright}{\kern0pt}{\isacharparenright}{\kern0pt}\ f{\isacharparenright}{\kern0pt}\ v{\isacharparenright}{\kern0pt}\ {\isasymsubseteq}\ {\isasymPsi}\ {\isacharparenleft}{\kern0pt}eval\ {\isacharparenleft}{\kern0pt}subst\ {\isacharparenleft}{\kern0pt}Var{\isacharparenleft}{\kern0pt}x\ {\isacharcolon}{\kern0pt}{\isacharequal}{\kern0pt}\ B{\isacharparenright}{\kern0pt}{\isacharparenright}{\kern0pt}\ f{\isacharparenright}{\kern0pt}\ v{\isacharparenright}{\kern0pt}{\isachardoublequoteclose}\isanewline
%
\isadelimproof
\ \ %
\endisadelimproof
%
\isatagproof
\isakeywordONE{using}\isamarkupfalse%
\ parikh{\isacharunderscore}{\kern0pt}img{\isacharunderscore}{\kern0pt}subst{\isacharunderscore}{\kern0pt}mono{\isacharbrackleft}{\kern0pt}of\ {\isachardoublequoteopen}Var{\isacharparenleft}{\kern0pt}x\ {\isacharcolon}{\kern0pt}{\isacharequal}{\kern0pt}\ A{\isacharparenright}{\kern0pt}{\isachardoublequoteclose}\ v\ {\isachardoublequoteopen}Var{\isacharparenleft}{\kern0pt}x\ {\isacharcolon}{\kern0pt}{\isacharequal}{\kern0pt}\ B{\isacharparenright}{\kern0pt}{\isachardoublequoteclose}{\isacharbrackright}{\kern0pt}\ assms\ \isakeywordONE{by}\isamarkupfalse%
\ auto%
\endisatagproof
{\isafoldproof}%
%
\isadelimproof
\isanewline
%
\endisadelimproof
\isanewline
\isakeywordONE{lemma}\isamarkupfalse%
\ rlexp{\isacharunderscore}{\kern0pt}mono{\isacharunderscore}{\kern0pt}parikh{\isacharcolon}{\kern0pt}\isanewline
\ \ \isakeywordTWO{assumes}\ {\isachardoublequoteopen}{\isasymforall}i\ {\isasymin}\ vars\ f{\isachardot}{\kern0pt}\ {\isasymPsi}\ {\isacharparenleft}{\kern0pt}v\ i{\isacharparenright}{\kern0pt}\ {\isasymsubseteq}\ {\isasymPsi}\ {\isacharparenleft}{\kern0pt}v{\isacharprime}{\kern0pt}\ i{\isacharparenright}{\kern0pt}{\isachardoublequoteclose}\isanewline
\ \ \isakeywordTWO{shows}\ {\isachardoublequoteopen}{\isasymPsi}\ {\isacharparenleft}{\kern0pt}eval\ f\ v{\isacharparenright}{\kern0pt}\ {\isasymsubseteq}\ {\isasymPsi}\ {\isacharparenleft}{\kern0pt}eval\ f\ v{\isacharprime}{\kern0pt}{\isacharparenright}{\kern0pt}{\isachardoublequoteclose}\isanewline
%
\isadelimproof
%
\endisadelimproof
%
\isatagproof
\isakeywordONE{using}\isamarkupfalse%
\ assms\ \isakeywordONE{proof}\isamarkupfalse%
\ {\isacharparenleft}{\kern0pt}induction\ f\ rule{\isacharcolon}{\kern0pt}\ rlexp{\isachardot}{\kern0pt}induct{\isacharparenright}{\kern0pt}\isanewline
\isakeywordTHREE{case}\isamarkupfalse%
\ {\isacharparenleft}{\kern0pt}Concat\ f{\isadigit{1}}\ f{\isadigit{2}}{\isacharparenright}{\kern0pt}\isanewline
\ \ \isakeywordONE{then}\isamarkupfalse%
\ \isakeywordONE{have}\isamarkupfalse%
\ {\isachardoublequoteopen}{\isasymPsi}\ {\isacharparenleft}{\kern0pt}eval\ f{\isadigit{1}}\ v\ {\isacharat}{\kern0pt}{\isacharat}{\kern0pt}\ eval\ f{\isadigit{2}}\ v{\isacharparenright}{\kern0pt}\ {\isasymsubseteq}\ {\isasymPsi}\ {\isacharparenleft}{\kern0pt}eval\ f{\isadigit{1}}\ v{\isacharprime}{\kern0pt}\ {\isacharat}{\kern0pt}{\isacharat}{\kern0pt}\ eval\ f{\isadigit{2}}\ v{\isacharprime}{\kern0pt}{\isacharparenright}{\kern0pt}{\isachardoublequoteclose}\isanewline
\ \ \ \ \isakeywordONE{using}\isamarkupfalse%
\ parikh{\isacharunderscore}{\kern0pt}conc{\isacharunderscore}{\kern0pt}subset\ \isakeywordONE{by}\isamarkupfalse%
\ {\isacharparenleft}{\kern0pt}metis\ UnCI\ vars{\isachardot}{\kern0pt}simps{\isacharparenleft}{\kern0pt}{\isadigit{4}}{\isacharparenright}{\kern0pt}{\isacharparenright}{\kern0pt}\isanewline
\ \ \isakeywordONE{then}\isamarkupfalse%
\ \isakeywordTHREE{show}\isamarkupfalse%
\ {\isacharquery}{\kern0pt}case\ \isakeywordONE{by}\isamarkupfalse%
\ simp\isanewline
\isakeywordONE{qed}\isamarkupfalse%
\ {\isacharparenleft}{\kern0pt}auto\ simp\ add{\isacharcolon}{\kern0pt}\ SUP{\isacharunderscore}{\kern0pt}mono{\isacharprime}{\kern0pt}\ parikh{\isacharunderscore}{\kern0pt}img{\isacharunderscore}{\kern0pt}UNION\ parikh{\isacharunderscore}{\kern0pt}star{\isacharunderscore}{\kern0pt}mono{\isacharparenright}{\kern0pt}%
\endisatagproof
{\isafoldproof}%
%
\isadelimproof
\isanewline
%
\endisadelimproof
\isanewline
\isakeywordONE{lemma}\isamarkupfalse%
\ rlexp{\isacharunderscore}{\kern0pt}mono{\isacharunderscore}{\kern0pt}parikh{\isacharunderscore}{\kern0pt}eq{\isacharcolon}{\kern0pt}\isanewline
\ \ \isakeywordTWO{assumes}\ {\isachardoublequoteopen}{\isasymforall}i\ {\isasymin}\ vars\ f{\isachardot}{\kern0pt}\ {\isasymPsi}\ {\isacharparenleft}{\kern0pt}v\ i{\isacharparenright}{\kern0pt}\ {\isacharequal}{\kern0pt}\ {\isasymPsi}\ {\isacharparenleft}{\kern0pt}v{\isacharprime}{\kern0pt}\ i{\isacharparenright}{\kern0pt}{\isachardoublequoteclose}\isanewline
\ \ \isakeywordTWO{shows}\ {\isachardoublequoteopen}{\isasymPsi}\ {\isacharparenleft}{\kern0pt}eval\ f\ v{\isacharparenright}{\kern0pt}\ {\isacharequal}{\kern0pt}\ {\isasymPsi}\ {\isacharparenleft}{\kern0pt}eval\ f\ v{\isacharprime}{\kern0pt}{\isacharparenright}{\kern0pt}{\isachardoublequoteclose}\isanewline
%
\isadelimproof
\ \ %
\endisadelimproof
%
\isatagproof
\isakeywordONE{using}\isamarkupfalse%
\ assms\ rlexp{\isacharunderscore}{\kern0pt}mono{\isacharunderscore}{\kern0pt}parikh\ \isakeywordONE{by}\isamarkupfalse%
\ blast%
\endisatagproof
{\isafoldproof}%
%
\isadelimproof
%
\endisadelimproof
%
\isadelimdocument
%
\endisadelimdocument
%
\isatagdocument
%
\isamarkupsubsection{$\Psi \; (A \cup B)^* = \Psi \; A^* B^*$%
}
\isamarkuptrue%
%
\endisatagdocument
{\isafolddocument}%
%
\isadelimdocument
%
\endisadelimdocument
%
\begin{isamarkuptext}%
\label{sec:parikh_img_star}%
\end{isamarkuptext}\isamarkuptrue%
%
\begin{isamarkuptext}%
This property is claimed by Pilling in \cite{Pilling} and will be needed later.%
\end{isamarkuptext}\isamarkuptrue%
\isakeywordONE{lemma}\isamarkupfalse%
\ parikh{\isacharunderscore}{\kern0pt}img{\isacharunderscore}{\kern0pt}union{\isacharunderscore}{\kern0pt}pow{\isacharunderscore}{\kern0pt}aux{\isadigit{1}}{\isacharcolon}{\kern0pt}\isanewline
\ \ \isakeywordTWO{assumes}\ {\isachardoublequoteopen}v\ {\isasymin}\ {\isasymPsi}\ {\isacharparenleft}{\kern0pt}{\isacharparenleft}{\kern0pt}A\ {\isasymunion}\ B{\isacharparenright}{\kern0pt}\ {\isacharcircum}{\kern0pt}{\isacharcircum}{\kern0pt}\ n{\isacharparenright}{\kern0pt}{\isachardoublequoteclose}\isanewline
\ \ \ \ \isakeywordTWO{shows}\ {\isachardoublequoteopen}v\ {\isasymin}\ {\isasymPsi}\ {\isacharparenleft}{\kern0pt}{\isasymUnion}i\ {\isasymle}\ n{\isachardot}{\kern0pt}\ A\ {\isacharcircum}{\kern0pt}{\isacharcircum}{\kern0pt}\ i\ {\isacharat}{\kern0pt}{\isacharat}{\kern0pt}\ B\ {\isacharcircum}{\kern0pt}{\isacharcircum}{\kern0pt}\ {\isacharparenleft}{\kern0pt}n{\isacharminus}{\kern0pt}i{\isacharparenright}{\kern0pt}{\isacharparenright}{\kern0pt}{\isachardoublequoteclose}\isanewline
%
\isadelimproof
%
\endisadelimproof
%
\isatagproof
\isakeywordONE{using}\isamarkupfalse%
\ assms\ \isakeywordONE{proof}\isamarkupfalse%
\ {\isacharparenleft}{\kern0pt}induction\ n\ arbitrary{\isacharcolon}{\kern0pt}\ v{\isacharparenright}{\kern0pt}\isanewline
\ \ \isakeywordTHREE{case}\isamarkupfalse%
\ {\isadigit{0}}\isanewline
\ \ \isakeywordONE{then}\isamarkupfalse%
\ \isakeywordTHREE{show}\isamarkupfalse%
\ {\isacharquery}{\kern0pt}case\ \isakeywordONE{by}\isamarkupfalse%
\ simp\isanewline
\isakeywordONE{next}\isamarkupfalse%
\isanewline
\ \ \isakeywordTHREE{case}\isamarkupfalse%
\ {\isacharparenleft}{\kern0pt}Suc\ n{\isacharparenright}{\kern0pt}\isanewline
\ \ \isakeywordONE{then}\isamarkupfalse%
\ \isakeywordTHREE{obtain}\isamarkupfalse%
\ w\ \isakeywordTWO{where}\ w{\isacharunderscore}{\kern0pt}intro{\isacharcolon}{\kern0pt}\ {\isachardoublequoteopen}w\ {\isasymin}\ {\isacharparenleft}{\kern0pt}A\ {\isasymunion}\ B{\isacharparenright}{\kern0pt}\ {\isacharcircum}{\kern0pt}{\isacharcircum}{\kern0pt}\ {\isacharparenleft}{\kern0pt}Suc\ n{\isacharparenright}{\kern0pt}\ {\isasymand}\ parikh{\isacharunderscore}{\kern0pt}vec\ w\ {\isacharequal}{\kern0pt}\ v{\isachardoublequoteclose}\isanewline
\ \ \ \ \isakeywordONE{unfolding}\isamarkupfalse%
\ parikh{\isacharunderscore}{\kern0pt}img{\isacharunderscore}{\kern0pt}def\ \isakeywordONE{by}\isamarkupfalse%
\ auto\isanewline
\ \ \isakeywordONE{then}\isamarkupfalse%
\ \isakeywordTHREE{obtain}\isamarkupfalse%
\ w{\isadigit{1}}\ w{\isadigit{2}}\ \isakeywordTWO{where}\ w{\isadigit{1}}{\isacharunderscore}{\kern0pt}w{\isadigit{2}}{\isacharunderscore}{\kern0pt}intro{\isacharcolon}{\kern0pt}\ {\isachardoublequoteopen}w\ {\isacharequal}{\kern0pt}\ w{\isadigit{1}}{\isacharat}{\kern0pt}w{\isadigit{2}}\ {\isasymand}\ w{\isadigit{1}}\ {\isasymin}\ A\ {\isasymunion}\ B\ {\isasymand}\ w{\isadigit{2}}\ {\isasymin}\ {\isacharparenleft}{\kern0pt}A\ {\isasymunion}\ B{\isacharparenright}{\kern0pt}\ {\isacharcircum}{\kern0pt}{\isacharcircum}{\kern0pt}\ n{\isachardoublequoteclose}\ \isakeywordONE{by}\isamarkupfalse%
\ fastforce\isanewline
\ \ \isakeywordONE{let}\isamarkupfalse%
\ {\isacharquery}{\kern0pt}v{\isadigit{1}}\ {\isacharequal}{\kern0pt}\ {\isachardoublequoteopen}parikh{\isacharunderscore}{\kern0pt}vec\ w{\isadigit{1}}{\isachardoublequoteclose}\ \isakeywordTWO{and}\ {\isacharquery}{\kern0pt}v{\isadigit{2}}\ {\isacharequal}{\kern0pt}\ {\isachardoublequoteopen}parikh{\isacharunderscore}{\kern0pt}vec\ w{\isadigit{2}}{\isachardoublequoteclose}\isanewline
\ \ \isakeywordONE{from}\isamarkupfalse%
\ w{\isadigit{1}}{\isacharunderscore}{\kern0pt}w{\isadigit{2}}{\isacharunderscore}{\kern0pt}intro\ \isakeywordONE{have}\isamarkupfalse%
\ {\isachardoublequoteopen}{\isacharquery}{\kern0pt}v{\isadigit{2}}\ {\isasymin}\ {\isasymPsi}\ {\isacharparenleft}{\kern0pt}{\isacharparenleft}{\kern0pt}A\ {\isasymunion}\ B{\isacharparenright}{\kern0pt}\ {\isacharcircum}{\kern0pt}{\isacharcircum}{\kern0pt}\ n{\isacharparenright}{\kern0pt}{\isachardoublequoteclose}\ \isakeywordONE{unfolding}\isamarkupfalse%
\ parikh{\isacharunderscore}{\kern0pt}img{\isacharunderscore}{\kern0pt}def\ \isakeywordONE{by}\isamarkupfalse%
\ blast\isanewline
\ \ \isakeywordONE{with}\isamarkupfalse%
\ Suc{\isachardot}{\kern0pt}IH\ \isakeywordONE{have}\isamarkupfalse%
\ {\isachardoublequoteopen}{\isacharquery}{\kern0pt}v{\isadigit{2}}\ {\isasymin}\ {\isasymPsi}\ {\isacharparenleft}{\kern0pt}{\isasymUnion}i\ {\isasymle}\ n{\isachardot}{\kern0pt}\ A\ {\isacharcircum}{\kern0pt}{\isacharcircum}{\kern0pt}\ i\ {\isacharat}{\kern0pt}{\isacharat}{\kern0pt}\ B\ {\isacharcircum}{\kern0pt}{\isacharcircum}{\kern0pt}\ {\isacharparenleft}{\kern0pt}n{\isacharminus}{\kern0pt}i{\isacharparenright}{\kern0pt}{\isacharparenright}{\kern0pt}{\isachardoublequoteclose}\ \isakeywordONE{by}\isamarkupfalse%
\ auto\isanewline
\ \ \isakeywordONE{then}\isamarkupfalse%
\ \isakeywordTHREE{obtain}\isamarkupfalse%
\ w{\isadigit{2}}{\isacharprime}{\kern0pt}\ \isakeywordTWO{where}\ w{\isadigit{2}}{\isacharprime}{\kern0pt}{\isacharunderscore}{\kern0pt}intro{\isacharcolon}{\kern0pt}\ {\isachardoublequoteopen}parikh{\isacharunderscore}{\kern0pt}vec\ w{\isadigit{2}}{\isacharprime}{\kern0pt}\ {\isacharequal}{\kern0pt}\ parikh{\isacharunderscore}{\kern0pt}vec\ w{\isadigit{2}}\ {\isasymand}\isanewline
\ \ \ \ \ \ w{\isadigit{2}}{\isacharprime}{\kern0pt}\ {\isasymin}\ {\isacharparenleft}{\kern0pt}{\isasymUnion}i\ {\isasymle}\ n{\isachardot}{\kern0pt}\ A\ {\isacharcircum}{\kern0pt}{\isacharcircum}{\kern0pt}\ i\ {\isacharat}{\kern0pt}{\isacharat}{\kern0pt}\ B\ {\isacharcircum}{\kern0pt}{\isacharcircum}{\kern0pt}\ {\isacharparenleft}{\kern0pt}n{\isacharminus}{\kern0pt}i{\isacharparenright}{\kern0pt}{\isacharparenright}{\kern0pt}{\isachardoublequoteclose}\ \isakeywordONE{unfolding}\isamarkupfalse%
\ parikh{\isacharunderscore}{\kern0pt}img{\isacharunderscore}{\kern0pt}def\ \isakeywordONE{by}\isamarkupfalse%
\ fastforce\isanewline
\ \ \isakeywordONE{then}\isamarkupfalse%
\ \isakeywordTHREE{obtain}\isamarkupfalse%
\ i\ \isakeywordTWO{where}\ i{\isacharunderscore}{\kern0pt}intro{\isacharcolon}{\kern0pt}\ {\isachardoublequoteopen}i\ {\isasymle}\ n\ {\isasymand}\ w{\isadigit{2}}{\isacharprime}{\kern0pt}\ {\isasymin}\ A\ {\isacharcircum}{\kern0pt}{\isacharcircum}{\kern0pt}\ i\ {\isacharat}{\kern0pt}{\isacharat}{\kern0pt}\ B\ {\isacharcircum}{\kern0pt}{\isacharcircum}{\kern0pt}\ {\isacharparenleft}{\kern0pt}n{\isacharminus}{\kern0pt}i{\isacharparenright}{\kern0pt}{\isachardoublequoteclose}\ \isakeywordONE{by}\isamarkupfalse%
\ blast\isanewline
\ \ \isakeywordONE{from}\isamarkupfalse%
\ w{\isadigit{1}}{\isacharunderscore}{\kern0pt}w{\isadigit{2}}{\isacharunderscore}{\kern0pt}intro\ w{\isadigit{2}}{\isacharprime}{\kern0pt}{\isacharunderscore}{\kern0pt}intro\ \isakeywordONE{have}\isamarkupfalse%
\ {\isachardoublequoteopen}parikh{\isacharunderscore}{\kern0pt}vec\ w\ {\isacharequal}{\kern0pt}\ parikh{\isacharunderscore}{\kern0pt}vec\ {\isacharparenleft}{\kern0pt}w{\isadigit{1}}{\isacharat}{\kern0pt}w{\isadigit{2}}{\isacharprime}{\kern0pt}{\isacharparenright}{\kern0pt}{\isachardoublequoteclose}\isanewline
\ \ \ \ \isakeywordONE{by}\isamarkupfalse%
\ simp\isanewline
\ \ \isakeywordONE{moreover}\isamarkupfalse%
\ \isakeywordONE{have}\isamarkupfalse%
\ {\isachardoublequoteopen}parikh{\isacharunderscore}{\kern0pt}vec\ {\isacharparenleft}{\kern0pt}w{\isadigit{1}}{\isacharat}{\kern0pt}w{\isadigit{2}}{\isacharprime}{\kern0pt}{\isacharparenright}{\kern0pt}\ {\isasymin}\ {\isasymPsi}\ {\isacharparenleft}{\kern0pt}{\isasymUnion}i\ {\isasymle}\ Suc\ n{\isachardot}{\kern0pt}\ A\ {\isacharcircum}{\kern0pt}{\isacharcircum}{\kern0pt}\ i\ {\isacharat}{\kern0pt}{\isacharat}{\kern0pt}\ B\ {\isacharcircum}{\kern0pt}{\isacharcircum}{\kern0pt}\ {\isacharparenleft}{\kern0pt}Suc\ n{\isacharminus}{\kern0pt}i{\isacharparenright}{\kern0pt}{\isacharparenright}{\kern0pt}{\isachardoublequoteclose}\isanewline
\ \ \isakeywordONE{proof}\isamarkupfalse%
\ {\isacharparenleft}{\kern0pt}cases\ {\isachardoublequoteopen}w{\isadigit{1}}\ {\isasymin}\ A{\isachardoublequoteclose}{\isacharparenright}{\kern0pt}\isanewline
\ \ \ \ \isakeywordTHREE{case}\isamarkupfalse%
\ True\isanewline
\ \ \ \ \isakeywordONE{with}\isamarkupfalse%
\ i{\isacharunderscore}{\kern0pt}intro\ \isakeywordONE{have}\isamarkupfalse%
\ Suc{\isacharunderscore}{\kern0pt}i{\isacharunderscore}{\kern0pt}valid{\isacharcolon}{\kern0pt}\ {\isachardoublequoteopen}Suc\ i\ {\isasymle}\ Suc\ n{\isachardoublequoteclose}\ \isakeywordTWO{and}\ {\isachardoublequoteopen}w{\isadigit{1}}{\isacharat}{\kern0pt}w{\isadigit{2}}{\isacharprime}{\kern0pt}\ {\isasymin}\ A\ {\isacharcircum}{\kern0pt}{\isacharcircum}{\kern0pt}\ {\isacharparenleft}{\kern0pt}Suc\ i{\isacharparenright}{\kern0pt}\ {\isacharat}{\kern0pt}{\isacharat}{\kern0pt}\ B\ {\isacharcircum}{\kern0pt}{\isacharcircum}{\kern0pt}\ {\isacharparenleft}{\kern0pt}Suc\ n\ {\isacharminus}{\kern0pt}\ Suc\ i{\isacharparenright}{\kern0pt}{\isachardoublequoteclose}\isanewline
\ \ \ \ \ \ \isakeywordONE{by}\isamarkupfalse%
\ {\isacharparenleft}{\kern0pt}auto\ simp\ add{\isacharcolon}{\kern0pt}\ conc{\isacharunderscore}{\kern0pt}assoc{\isacharparenright}{\kern0pt}\isanewline
\ \ \ \ \isakeywordONE{then}\isamarkupfalse%
\ \isakeywordONE{have}\isamarkupfalse%
\ {\isachardoublequoteopen}parikh{\isacharunderscore}{\kern0pt}vec\ {\isacharparenleft}{\kern0pt}w{\isadigit{1}}{\isacharat}{\kern0pt}w{\isadigit{2}}{\isacharprime}{\kern0pt}{\isacharparenright}{\kern0pt}\ {\isasymin}\ {\isasymPsi}\ {\isacharparenleft}{\kern0pt}A\ {\isacharcircum}{\kern0pt}{\isacharcircum}{\kern0pt}\ {\isacharparenleft}{\kern0pt}Suc\ i{\isacharparenright}{\kern0pt}\ {\isacharat}{\kern0pt}{\isacharat}{\kern0pt}\ B\ {\isacharcircum}{\kern0pt}{\isacharcircum}{\kern0pt}\ {\isacharparenleft}{\kern0pt}Suc\ n\ {\isacharminus}{\kern0pt}\ Suc\ i{\isacharparenright}{\kern0pt}{\isacharparenright}{\kern0pt}{\isachardoublequoteclose}\isanewline
\ \ \ \ \ \ \isakeywordONE{unfolding}\isamarkupfalse%
\ parikh{\isacharunderscore}{\kern0pt}img{\isacharunderscore}{\kern0pt}def\ \isakeywordONE{by}\isamarkupfalse%
\ blast\isanewline
\ \ \ \ \isakeywordONE{with}\isamarkupfalse%
\ Suc{\isacharunderscore}{\kern0pt}i{\isacharunderscore}{\kern0pt}valid\ parikh{\isacharunderscore}{\kern0pt}img{\isacharunderscore}{\kern0pt}UNION\ \isakeywordTHREE{show}\isamarkupfalse%
\ {\isacharquery}{\kern0pt}thesis\ \isakeywordONE{by}\isamarkupfalse%
\ fast\isanewline
\ \ \isakeywordONE{next}\isamarkupfalse%
\isanewline
\ \ \ \ \isakeywordTHREE{case}\isamarkupfalse%
\ False\isanewline
\ \ \ \ \isakeywordONE{with}\isamarkupfalse%
\ w{\isadigit{1}}{\isacharunderscore}{\kern0pt}w{\isadigit{2}}{\isacharunderscore}{\kern0pt}intro\ \isakeywordONE{have}\isamarkupfalse%
\ {\isachardoublequoteopen}w{\isadigit{1}}\ {\isasymin}\ B{\isachardoublequoteclose}\ \isakeywordONE{by}\isamarkupfalse%
\ blast\isanewline
\ \ \ \ \isakeywordONE{with}\isamarkupfalse%
\ i{\isacharunderscore}{\kern0pt}intro\ \isakeywordONE{have}\isamarkupfalse%
\ {\isachardoublequoteopen}parikh{\isacharunderscore}{\kern0pt}vec\ {\isacharparenleft}{\kern0pt}w{\isadigit{1}}{\isacharat}{\kern0pt}w{\isadigit{2}}{\isacharprime}{\kern0pt}{\isacharparenright}{\kern0pt}\ {\isasymin}\ {\isasymPsi}\ {\isacharparenleft}{\kern0pt}B\ {\isacharat}{\kern0pt}{\isacharat}{\kern0pt}\ A\ {\isacharcircum}{\kern0pt}{\isacharcircum}{\kern0pt}\ i\ {\isacharat}{\kern0pt}{\isacharat}{\kern0pt}\ B\ {\isacharcircum}{\kern0pt}{\isacharcircum}{\kern0pt}\ {\isacharparenleft}{\kern0pt}n{\isacharminus}{\kern0pt}i{\isacharparenright}{\kern0pt}{\isacharparenright}{\kern0pt}{\isachardoublequoteclose}\isanewline
\ \ \ \ \ \ \isakeywordONE{unfolding}\isamarkupfalse%
\ parikh{\isacharunderscore}{\kern0pt}img{\isacharunderscore}{\kern0pt}def\ \isakeywordONE{by}\isamarkupfalse%
\ blast\isanewline
\ \ \ \ \isakeywordONE{then}\isamarkupfalse%
\ \isakeywordONE{have}\isamarkupfalse%
\ {\isachardoublequoteopen}parikh{\isacharunderscore}{\kern0pt}vec\ {\isacharparenleft}{\kern0pt}w{\isadigit{1}}{\isacharat}{\kern0pt}w{\isadigit{2}}{\isacharprime}{\kern0pt}{\isacharparenright}{\kern0pt}\ {\isasymin}\ {\isasymPsi}\ {\isacharparenleft}{\kern0pt}A\ {\isacharcircum}{\kern0pt}{\isacharcircum}{\kern0pt}\ i\ {\isacharat}{\kern0pt}{\isacharat}{\kern0pt}\ B\ {\isacharcircum}{\kern0pt}{\isacharcircum}{\kern0pt}\ {\isacharparenleft}{\kern0pt}Suc\ n{\isacharminus}{\kern0pt}i{\isacharparenright}{\kern0pt}{\isacharparenright}{\kern0pt}{\isachardoublequoteclose}\isanewline
\ \ \ \ \ \ \isakeywordONE{using}\isamarkupfalse%
\ parikh{\isacharunderscore}{\kern0pt}img{\isacharunderscore}{\kern0pt}commut\ conc{\isacharunderscore}{\kern0pt}assoc\isanewline
\ \ \ \ \ \ \isakeywordONE{by}\isamarkupfalse%
\ {\isacharparenleft}{\kern0pt}metis\ Suc{\isacharunderscore}{\kern0pt}diff{\isacharunderscore}{\kern0pt}le\ conc{\isacharunderscore}{\kern0pt}pow{\isacharunderscore}{\kern0pt}comm\ i{\isacharunderscore}{\kern0pt}intro\ lang{\isacharunderscore}{\kern0pt}pow{\isachardot}{\kern0pt}simps{\isacharparenleft}{\kern0pt}{\isadigit{2}}{\isacharparenright}{\kern0pt}{\isacharparenright}{\kern0pt}\isanewline
\ \ \ \ \isakeywordONE{with}\isamarkupfalse%
\ i{\isacharunderscore}{\kern0pt}intro\ parikh{\isacharunderscore}{\kern0pt}img{\isacharunderscore}{\kern0pt}UNION\ \isakeywordTHREE{show}\isamarkupfalse%
\ {\isacharquery}{\kern0pt}thesis\ \isakeywordONE{by}\isamarkupfalse%
\ fastforce\isanewline
\ \ \isakeywordONE{qed}\isamarkupfalse%
\isanewline
\ \ \isakeywordONE{ultimately}\isamarkupfalse%
\ \isakeywordTHREE{show}\isamarkupfalse%
\ {\isacharquery}{\kern0pt}case\ \isakeywordONE{using}\isamarkupfalse%
\ w{\isacharunderscore}{\kern0pt}intro\ \isakeywordONE{by}\isamarkupfalse%
\ auto\isanewline
\isakeywordONE{qed}\isamarkupfalse%
%
\endisatagproof
{\isafoldproof}%
%
\isadelimproof
\isanewline
%
\endisadelimproof
\isanewline
\isakeywordONE{lemma}\isamarkupfalse%
\ parikh{\isacharunderscore}{\kern0pt}img{\isacharunderscore}{\kern0pt}star{\isacharunderscore}{\kern0pt}aux{\isadigit{1}}{\isacharcolon}{\kern0pt}\isanewline
\ \ \isakeywordTWO{assumes}\ {\isachardoublequoteopen}v\ {\isasymin}\ {\isasymPsi}\ {\isacharparenleft}{\kern0pt}star\ {\isacharparenleft}{\kern0pt}A\ {\isasymunion}\ B{\isacharparenright}{\kern0pt}{\isacharparenright}{\kern0pt}{\isachardoublequoteclose}\isanewline
\ \ \isakeywordTWO{shows}\ {\isachardoublequoteopen}v\ {\isasymin}\ {\isasymPsi}\ {\isacharparenleft}{\kern0pt}star\ A\ {\isacharat}{\kern0pt}{\isacharat}{\kern0pt}\ star\ B{\isacharparenright}{\kern0pt}{\isachardoublequoteclose}\isanewline
%
\isadelimproof
%
\endisadelimproof
%
\isatagproof
\isakeywordONE{proof}\isamarkupfalse%
\ {\isacharminus}{\kern0pt}\isanewline
\ \ \isakeywordONE{from}\isamarkupfalse%
\ assms\ \isakeywordONE{have}\isamarkupfalse%
\ {\isachardoublequoteopen}v\ {\isasymin}\ {\isacharparenleft}{\kern0pt}{\isasymUnion}n{\isachardot}{\kern0pt}\ {\isasymPsi}\ {\isacharparenleft}{\kern0pt}{\isacharparenleft}{\kern0pt}A\ {\isasymunion}\ B{\isacharparenright}{\kern0pt}\ {\isacharcircum}{\kern0pt}{\isacharcircum}{\kern0pt}\ n{\isacharparenright}{\kern0pt}{\isacharparenright}{\kern0pt}{\isachardoublequoteclose}\isanewline
\ \ \ \ \isakeywordONE{unfolding}\isamarkupfalse%
\ star{\isacharunderscore}{\kern0pt}def\ \isakeywordONE{using}\isamarkupfalse%
\ parikh{\isacharunderscore}{\kern0pt}img{\isacharunderscore}{\kern0pt}UNION\ \isakeywordONE{by}\isamarkupfalse%
\ metis\isanewline
\ \ \isakeywordONE{then}\isamarkupfalse%
\ \isakeywordTHREE{obtain}\isamarkupfalse%
\ n\ \isakeywordTWO{where}\ {\isachardoublequoteopen}v\ {\isasymin}\ {\isasymPsi}\ {\isacharparenleft}{\kern0pt}{\isacharparenleft}{\kern0pt}A\ {\isasymunion}\ B{\isacharparenright}{\kern0pt}\ {\isacharcircum}{\kern0pt}{\isacharcircum}{\kern0pt}\ n{\isacharparenright}{\kern0pt}{\isachardoublequoteclose}\ \isakeywordONE{by}\isamarkupfalse%
\ blast\isanewline
\ \ \isakeywordONE{then}\isamarkupfalse%
\ \isakeywordONE{have}\isamarkupfalse%
\ {\isachardoublequoteopen}v\ {\isasymin}\ {\isasymPsi}\ {\isacharparenleft}{\kern0pt}{\isasymUnion}i\ {\isasymle}\ n{\isachardot}{\kern0pt}\ A\ {\isacharcircum}{\kern0pt}{\isacharcircum}{\kern0pt}\ i\ {\isacharat}{\kern0pt}{\isacharat}{\kern0pt}\ B\ {\isacharcircum}{\kern0pt}{\isacharcircum}{\kern0pt}\ {\isacharparenleft}{\kern0pt}n{\isacharminus}{\kern0pt}i{\isacharparenright}{\kern0pt}{\isacharparenright}{\kern0pt}{\isachardoublequoteclose}\isanewline
\ \ \ \ \isakeywordONE{using}\isamarkupfalse%
\ parikh{\isacharunderscore}{\kern0pt}img{\isacharunderscore}{\kern0pt}union{\isacharunderscore}{\kern0pt}pow{\isacharunderscore}{\kern0pt}aux{\isadigit{1}}\ \isakeywordONE{by}\isamarkupfalse%
\ auto\isanewline
\ \ \isakeywordONE{then}\isamarkupfalse%
\ \isakeywordONE{have}\isamarkupfalse%
\ {\isachardoublequoteopen}v\ {\isasymin}\ {\isacharparenleft}{\kern0pt}{\isasymUnion}i{\isasymle}n{\isachardot}{\kern0pt}\ {\isasymPsi}\ {\isacharparenleft}{\kern0pt}A\ {\isacharcircum}{\kern0pt}{\isacharcircum}{\kern0pt}\ i\ {\isacharat}{\kern0pt}{\isacharat}{\kern0pt}\ B\ {\isacharcircum}{\kern0pt}{\isacharcircum}{\kern0pt}\ {\isacharparenleft}{\kern0pt}n{\isacharminus}{\kern0pt}i{\isacharparenright}{\kern0pt}{\isacharparenright}{\kern0pt}{\isacharparenright}{\kern0pt}{\isachardoublequoteclose}\ \isakeywordONE{using}\isamarkupfalse%
\ parikh{\isacharunderscore}{\kern0pt}img{\isacharunderscore}{\kern0pt}UNION\ \isakeywordONE{by}\isamarkupfalse%
\ metis\isanewline
\ \ \isakeywordONE{then}\isamarkupfalse%
\ \isakeywordTHREE{obtain}\isamarkupfalse%
\ i\ \isakeywordTWO{where}\ {\isachardoublequoteopen}i{\isasymle}n\ {\isasymand}\ v\ {\isasymin}\ {\isasymPsi}\ {\isacharparenleft}{\kern0pt}A\ {\isacharcircum}{\kern0pt}{\isacharcircum}{\kern0pt}\ i\ {\isacharat}{\kern0pt}{\isacharat}{\kern0pt}\ B\ {\isacharcircum}{\kern0pt}{\isacharcircum}{\kern0pt}\ {\isacharparenleft}{\kern0pt}n{\isacharminus}{\kern0pt}i{\isacharparenright}{\kern0pt}{\isacharparenright}{\kern0pt}{\isachardoublequoteclose}\ \isakeywordONE{by}\isamarkupfalse%
\ blast\isanewline
\ \ \isakeywordONE{then}\isamarkupfalse%
\ \isakeywordTHREE{obtain}\isamarkupfalse%
\ w\ \isakeywordTWO{where}\ w{\isacharunderscore}{\kern0pt}intro{\isacharcolon}{\kern0pt}\ {\isachardoublequoteopen}parikh{\isacharunderscore}{\kern0pt}vec\ w\ {\isacharequal}{\kern0pt}\ v\ {\isasymand}\ w\ {\isasymin}\ A\ {\isacharcircum}{\kern0pt}{\isacharcircum}{\kern0pt}\ i\ {\isacharat}{\kern0pt}{\isacharat}{\kern0pt}\ B\ {\isacharcircum}{\kern0pt}{\isacharcircum}{\kern0pt}\ {\isacharparenleft}{\kern0pt}n{\isacharminus}{\kern0pt}i{\isacharparenright}{\kern0pt}{\isachardoublequoteclose}\isanewline
\ \ \ \ \isakeywordONE{unfolding}\isamarkupfalse%
\ parikh{\isacharunderscore}{\kern0pt}img{\isacharunderscore}{\kern0pt}def\ \isakeywordONE{by}\isamarkupfalse%
\ blast\isanewline
\ \ \isakeywordONE{then}\isamarkupfalse%
\ \isakeywordTHREE{obtain}\isamarkupfalse%
\ w{\isadigit{1}}\ w{\isadigit{2}}\ \isakeywordTWO{where}\ w{\isacharunderscore}{\kern0pt}decomp{\isacharcolon}{\kern0pt}\ {\isachardoublequoteopen}w{\isacharequal}{\kern0pt}w{\isadigit{1}}{\isacharat}{\kern0pt}w{\isadigit{2}}\ {\isasymand}\ w{\isadigit{1}}\ {\isasymin}\ A\ {\isacharcircum}{\kern0pt}{\isacharcircum}{\kern0pt}\ i\ {\isasymand}\ w{\isadigit{2}}\ {\isasymin}\ B\ {\isacharcircum}{\kern0pt}{\isacharcircum}{\kern0pt}\ {\isacharparenleft}{\kern0pt}n{\isacharminus}{\kern0pt}i{\isacharparenright}{\kern0pt}{\isachardoublequoteclose}\ \isakeywordONE{by}\isamarkupfalse%
\ blast\isanewline
\ \ \isakeywordONE{then}\isamarkupfalse%
\ \isakeywordONE{have}\isamarkupfalse%
\ {\isachardoublequoteopen}w{\isadigit{1}}\ {\isasymin}\ star\ A{\isachardoublequoteclose}\ \isakeywordTWO{and}\ {\isachardoublequoteopen}w{\isadigit{2}}\ {\isasymin}\ star\ B{\isachardoublequoteclose}\ \isakeywordONE{by}\isamarkupfalse%
\ auto\isanewline
\ \ \isakeywordONE{with}\isamarkupfalse%
\ w{\isacharunderscore}{\kern0pt}decomp\ \isakeywordONE{have}\isamarkupfalse%
\ {\isachardoublequoteopen}w\ {\isasymin}\ star\ A\ {\isacharat}{\kern0pt}{\isacharat}{\kern0pt}\ star\ B{\isachardoublequoteclose}\ \isakeywordONE{by}\isamarkupfalse%
\ auto\isanewline
\ \ \isakeywordONE{with}\isamarkupfalse%
\ w{\isacharunderscore}{\kern0pt}intro\ \isakeywordTHREE{show}\isamarkupfalse%
\ {\isacharquery}{\kern0pt}thesis\ \isakeywordONE{unfolding}\isamarkupfalse%
\ parikh{\isacharunderscore}{\kern0pt}img{\isacharunderscore}{\kern0pt}def\ \isakeywordONE{by}\isamarkupfalse%
\ blast\isanewline
\isakeywordONE{qed}\isamarkupfalse%
%
\endisatagproof
{\isafoldproof}%
%
\isadelimproof
\isanewline
%
\endisadelimproof
\isanewline
\isakeywordONE{lemma}\isamarkupfalse%
\ parikh{\isacharunderscore}{\kern0pt}img{\isacharunderscore}{\kern0pt}star{\isacharunderscore}{\kern0pt}aux{\isadigit{2}}{\isacharcolon}{\kern0pt}\isanewline
\ \ \isakeywordTWO{assumes}\ {\isachardoublequoteopen}v\ {\isasymin}\ {\isasymPsi}\ {\isacharparenleft}{\kern0pt}star\ A\ {\isacharat}{\kern0pt}{\isacharat}{\kern0pt}\ star\ B{\isacharparenright}{\kern0pt}{\isachardoublequoteclose}\isanewline
\ \ \isakeywordTWO{shows}\ {\isachardoublequoteopen}v\ {\isasymin}\ {\isasymPsi}\ {\isacharparenleft}{\kern0pt}star\ {\isacharparenleft}{\kern0pt}A\ {\isasymunion}\ B{\isacharparenright}{\kern0pt}{\isacharparenright}{\kern0pt}{\isachardoublequoteclose}\isanewline
%
\isadelimproof
%
\endisadelimproof
%
\isatagproof
\isakeywordONE{proof}\isamarkupfalse%
\ {\isacharminus}{\kern0pt}\isanewline
\ \ \isakeywordONE{from}\isamarkupfalse%
\ assms\ \isakeywordTHREE{obtain}\isamarkupfalse%
\ w\ \isakeywordTWO{where}\ w{\isacharunderscore}{\kern0pt}intro{\isacharcolon}{\kern0pt}\ {\isachardoublequoteopen}parikh{\isacharunderscore}{\kern0pt}vec\ w\ {\isacharequal}{\kern0pt}\ v\ {\isasymand}\ w\ {\isasymin}\ star\ A\ {\isacharat}{\kern0pt}{\isacharat}{\kern0pt}\ star\ B{\isachardoublequoteclose}\isanewline
\ \ \ \ \isakeywordONE{unfolding}\isamarkupfalse%
\ parikh{\isacharunderscore}{\kern0pt}img{\isacharunderscore}{\kern0pt}def\ \isakeywordONE{by}\isamarkupfalse%
\ blast\isanewline
\ \ \isakeywordONE{then}\isamarkupfalse%
\ \isakeywordTHREE{obtain}\isamarkupfalse%
\ w{\isadigit{1}}\ w{\isadigit{2}}\ \isakeywordTWO{where}\ w{\isacharunderscore}{\kern0pt}decomp{\isacharcolon}{\kern0pt}\ {\isachardoublequoteopen}w{\isacharequal}{\kern0pt}w{\isadigit{1}}{\isacharat}{\kern0pt}w{\isadigit{2}}\ {\isasymand}\ w{\isadigit{1}}\ {\isasymin}\ star\ A\ {\isasymand}\ w{\isadigit{2}}\ {\isasymin}\ star\ B{\isachardoublequoteclose}\ \isakeywordONE{by}\isamarkupfalse%
\ blast\isanewline
\ \ \isakeywordONE{then}\isamarkupfalse%
\ \isakeywordTHREE{obtain}\isamarkupfalse%
\ i\ j\ \isakeywordTWO{where}\ {\isachardoublequoteopen}w{\isadigit{1}}\ {\isasymin}\ A\ {\isacharcircum}{\kern0pt}{\isacharcircum}{\kern0pt}\ i{\isachardoublequoteclose}\ \isakeywordTWO{and}\ w{\isadigit{2}}{\isacharunderscore}{\kern0pt}intro{\isacharcolon}{\kern0pt}\ {\isachardoublequoteopen}w{\isadigit{2}}\ {\isasymin}\ B\ {\isacharcircum}{\kern0pt}{\isacharcircum}{\kern0pt}\ j{\isachardoublequoteclose}\ \isakeywordONE{unfolding}\isamarkupfalse%
\ star{\isacharunderscore}{\kern0pt}def\ \isakeywordONE{by}\isamarkupfalse%
\ blast\isanewline
\ \ \isakeywordONE{then}\isamarkupfalse%
\ \isakeywordONE{have}\isamarkupfalse%
\ w{\isadigit{1}}{\isacharunderscore}{\kern0pt}in{\isacharunderscore}{\kern0pt}union{\isacharcolon}{\kern0pt}\ {\isachardoublequoteopen}w{\isadigit{1}}\ {\isasymin}\ {\isacharparenleft}{\kern0pt}A\ {\isasymunion}\ B{\isacharparenright}{\kern0pt}\ {\isacharcircum}{\kern0pt}{\isacharcircum}{\kern0pt}\ i{\isachardoublequoteclose}\ \isakeywordONE{using}\isamarkupfalse%
\ langpow{\isacharunderscore}{\kern0pt}mono\ \isakeywordONE{by}\isamarkupfalse%
\ blast\isanewline
\ \ \isakeywordONE{from}\isamarkupfalse%
\ w{\isadigit{2}}{\isacharunderscore}{\kern0pt}intro\ \isakeywordONE{have}\isamarkupfalse%
\ {\isachardoublequoteopen}w{\isadigit{2}}\ {\isasymin}\ {\isacharparenleft}{\kern0pt}A\ {\isasymunion}\ B{\isacharparenright}{\kern0pt}\ {\isacharcircum}{\kern0pt}{\isacharcircum}{\kern0pt}\ j{\isachardoublequoteclose}\ \isakeywordONE{using}\isamarkupfalse%
\ langpow{\isacharunderscore}{\kern0pt}mono\ \isakeywordONE{by}\isamarkupfalse%
\ blast\isanewline
\ \ \isakeywordONE{with}\isamarkupfalse%
\ w{\isadigit{1}}{\isacharunderscore}{\kern0pt}in{\isacharunderscore}{\kern0pt}union\ w{\isacharunderscore}{\kern0pt}decomp\ \isakeywordONE{have}\isamarkupfalse%
\ {\isachardoublequoteopen}w\ {\isasymin}\ {\isacharparenleft}{\kern0pt}A\ {\isasymunion}\ B{\isacharparenright}{\kern0pt}\ {\isacharcircum}{\kern0pt}{\isacharcircum}{\kern0pt}\ {\isacharparenleft}{\kern0pt}i{\isacharplus}{\kern0pt}j{\isacharparenright}{\kern0pt}{\isachardoublequoteclose}\ \isakeywordONE{using}\isamarkupfalse%
\ lang{\isacharunderscore}{\kern0pt}pow{\isacharunderscore}{\kern0pt}add\ \isakeywordONE{by}\isamarkupfalse%
\ fast\isanewline
\ \ \isakeywordONE{with}\isamarkupfalse%
\ w{\isacharunderscore}{\kern0pt}intro\ \isakeywordTHREE{show}\isamarkupfalse%
\ {\isacharquery}{\kern0pt}thesis\ \isakeywordONE{unfolding}\isamarkupfalse%
\ parikh{\isacharunderscore}{\kern0pt}img{\isacharunderscore}{\kern0pt}def\ \isakeywordONE{by}\isamarkupfalse%
\ auto\isanewline
\isakeywordONE{qed}\isamarkupfalse%
%
\endisatagproof
{\isafoldproof}%
%
\isadelimproof
\isanewline
%
\endisadelimproof
\isanewline
\isakeywordONE{lemma}\isamarkupfalse%
\ parikh{\isacharunderscore}{\kern0pt}img{\isacharunderscore}{\kern0pt}star{\isacharcolon}{\kern0pt}\ {\isachardoublequoteopen}{\isasymPsi}\ {\isacharparenleft}{\kern0pt}star\ {\isacharparenleft}{\kern0pt}A\ {\isasymunion}\ B{\isacharparenright}{\kern0pt}{\isacharparenright}{\kern0pt}\ {\isacharequal}{\kern0pt}\ {\isasymPsi}\ {\isacharparenleft}{\kern0pt}star\ A\ {\isacharat}{\kern0pt}{\isacharat}{\kern0pt}\ star\ B{\isacharparenright}{\kern0pt}{\isachardoublequoteclose}\isanewline
%
\isadelimproof
%
\endisadelimproof
%
\isatagproof
\isakeywordONE{proof}\isamarkupfalse%
\isanewline
\ \ \isakeywordTHREE{show}\isamarkupfalse%
\ {\isachardoublequoteopen}{\isasymPsi}\ {\isacharparenleft}{\kern0pt}star\ {\isacharparenleft}{\kern0pt}A\ {\isasymunion}\ B{\isacharparenright}{\kern0pt}{\isacharparenright}{\kern0pt}\ {\isasymsubseteq}\ {\isasymPsi}\ {\isacharparenleft}{\kern0pt}star\ A\ {\isacharat}{\kern0pt}{\isacharat}{\kern0pt}\ star\ B{\isacharparenright}{\kern0pt}{\isachardoublequoteclose}\ \isakeywordONE{using}\isamarkupfalse%
\ parikh{\isacharunderscore}{\kern0pt}img{\isacharunderscore}{\kern0pt}star{\isacharunderscore}{\kern0pt}aux{\isadigit{1}}\ \isakeywordONE{by}\isamarkupfalse%
\ auto\isanewline
\ \ \isakeywordTHREE{show}\isamarkupfalse%
\ {\isachardoublequoteopen}{\isasymPsi}\ {\isacharparenleft}{\kern0pt}star\ A\ {\isacharat}{\kern0pt}{\isacharat}{\kern0pt}\ star\ B{\isacharparenright}{\kern0pt}\ {\isasymsubseteq}\ {\isasymPsi}\ {\isacharparenleft}{\kern0pt}star\ {\isacharparenleft}{\kern0pt}A\ {\isasymunion}\ B{\isacharparenright}{\kern0pt}{\isacharparenright}{\kern0pt}{\isachardoublequoteclose}\ \isakeywordONE{using}\isamarkupfalse%
\ parikh{\isacharunderscore}{\kern0pt}img{\isacharunderscore}{\kern0pt}star{\isacharunderscore}{\kern0pt}aux{\isadigit{2}}\ \isakeywordONE{by}\isamarkupfalse%
\ auto\isanewline
\isakeywordONE{qed}\isamarkupfalse%
%
\endisatagproof
{\isafoldproof}%
%
\isadelimproof
%
\endisadelimproof
%
\isadelimdocument
%
\endisadelimdocument
%
\isatagdocument
%
\isamarkupsubsection{$\Psi \; (E^* F)^* = \Psi \; \left(\{\varepsilon\} \cup E^* F^* F\right)$%
}
\isamarkuptrue%
%
\endisatagdocument
{\isafolddocument}%
%
\isadelimdocument
%
\endisadelimdocument
%
\begin{isamarkuptext}%
\label{sec:parikh_img_star2}%
\end{isamarkuptext}\isamarkuptrue%
%
\begin{isamarkuptext}%
This property (where $\varepsilon$ denotes the empty word) is claimed by
Pilling as well \cite{Pilling}; we will use it later.%
\end{isamarkuptext}\isamarkuptrue%
\isakeywordONE{lemma}\isamarkupfalse%
\ parikh{\isacharunderscore}{\kern0pt}img{\isacharunderscore}{\kern0pt}conc{\isacharunderscore}{\kern0pt}pow{\isacharcolon}{\kern0pt}\ {\isachardoublequoteopen}{\isasymPsi}\ {\isacharparenleft}{\kern0pt}{\isacharparenleft}{\kern0pt}A\ {\isacharat}{\kern0pt}{\isacharat}{\kern0pt}\ B{\isacharparenright}{\kern0pt}\ {\isacharcircum}{\kern0pt}{\isacharcircum}{\kern0pt}\ n{\isacharparenright}{\kern0pt}\ {\isasymsubseteq}\ {\isasymPsi}\ {\isacharparenleft}{\kern0pt}A\ {\isacharcircum}{\kern0pt}{\isacharcircum}{\kern0pt}\ n\ {\isacharat}{\kern0pt}{\isacharat}{\kern0pt}\ B\ {\isacharcircum}{\kern0pt}{\isacharcircum}{\kern0pt}\ n{\isacharparenright}{\kern0pt}{\isachardoublequoteclose}\isanewline
%
\isadelimproof
%
\endisadelimproof
%
\isatagproof
\isakeywordONE{proof}\isamarkupfalse%
\ {\isacharparenleft}{\kern0pt}induction\ n{\isacharparenright}{\kern0pt}\isanewline
\ \ \isakeywordTHREE{case}\isamarkupfalse%
\ {\isacharparenleft}{\kern0pt}Suc\ n{\isacharparenright}{\kern0pt}\isanewline
\ \ \isakeywordONE{then}\isamarkupfalse%
\ \isakeywordONE{have}\isamarkupfalse%
\ {\isachardoublequoteopen}{\isasymPsi}\ {\isacharparenleft}{\kern0pt}{\isacharparenleft}{\kern0pt}A\ {\isacharat}{\kern0pt}{\isacharat}{\kern0pt}\ B{\isacharparenright}{\kern0pt}\ {\isacharcircum}{\kern0pt}{\isacharcircum}{\kern0pt}\ n\ {\isacharat}{\kern0pt}{\isacharat}{\kern0pt}\ A\ {\isacharat}{\kern0pt}{\isacharat}{\kern0pt}\ B{\isacharparenright}{\kern0pt}\ {\isasymsubseteq}\ {\isasymPsi}\ {\isacharparenleft}{\kern0pt}A\ {\isacharcircum}{\kern0pt}{\isacharcircum}{\kern0pt}\ n\ {\isacharat}{\kern0pt}{\isacharat}{\kern0pt}\ B\ {\isacharcircum}{\kern0pt}{\isacharcircum}{\kern0pt}\ n\ {\isacharat}{\kern0pt}{\isacharat}{\kern0pt}\ A\ {\isacharat}{\kern0pt}{\isacharat}{\kern0pt}\ B{\isacharparenright}{\kern0pt}{\isachardoublequoteclose}\isanewline
\ \ \ \ \isakeywordONE{using}\isamarkupfalse%
\ parikh{\isacharunderscore}{\kern0pt}conc{\isacharunderscore}{\kern0pt}right{\isacharunderscore}{\kern0pt}subset\ conc{\isacharunderscore}{\kern0pt}assoc\ \isakeywordONE{by}\isamarkupfalse%
\ metis\isanewline
\ \ \isakeywordONE{also}\isamarkupfalse%
\ \isakeywordONE{have}\isamarkupfalse%
\ {\isachardoublequoteopen}{\isasymdots}\ {\isacharequal}{\kern0pt}\ {\isasymPsi}\ {\isacharparenleft}{\kern0pt}A\ {\isacharcircum}{\kern0pt}{\isacharcircum}{\kern0pt}\ n\ {\isacharat}{\kern0pt}{\isacharat}{\kern0pt}\ A\ {\isacharat}{\kern0pt}{\isacharat}{\kern0pt}\ B\ {\isacharcircum}{\kern0pt}{\isacharcircum}{\kern0pt}\ n\ {\isacharat}{\kern0pt}{\isacharat}{\kern0pt}\ B{\isacharparenright}{\kern0pt}{\isachardoublequoteclose}\isanewline
\ \ \ \ \isakeywordONE{by}\isamarkupfalse%
\ {\isacharparenleft}{\kern0pt}metis\ parikh{\isacharunderscore}{\kern0pt}img{\isacharunderscore}{\kern0pt}commut\ conc{\isacharunderscore}{\kern0pt}assoc\ parikh{\isacharunderscore}{\kern0pt}conc{\isacharunderscore}{\kern0pt}left{\isacharparenright}{\kern0pt}\isanewline
\ \ \isakeywordONE{finally}\isamarkupfalse%
\ \isakeywordTHREE{show}\isamarkupfalse%
\ {\isacharquery}{\kern0pt}case\ \isakeywordONE{by}\isamarkupfalse%
\ {\isacharparenleft}{\kern0pt}simp\ add{\isacharcolon}{\kern0pt}\ conc{\isacharunderscore}{\kern0pt}assoc\ conc{\isacharunderscore}{\kern0pt}pow{\isacharunderscore}{\kern0pt}comm{\isacharparenright}{\kern0pt}\isanewline
\isakeywordONE{qed}\isamarkupfalse%
\ simp%
\endisatagproof
{\isafoldproof}%
%
\isadelimproof
\isanewline
%
\endisadelimproof
\isanewline
\isakeywordONE{lemma}\isamarkupfalse%
\ parikh{\isacharunderscore}{\kern0pt}img{\isacharunderscore}{\kern0pt}conc{\isacharunderscore}{\kern0pt}star{\isacharcolon}{\kern0pt}\ {\isachardoublequoteopen}{\isasymPsi}\ {\isacharparenleft}{\kern0pt}star\ {\isacharparenleft}{\kern0pt}A\ {\isacharat}{\kern0pt}{\isacharat}{\kern0pt}\ B{\isacharparenright}{\kern0pt}{\isacharparenright}{\kern0pt}\ {\isasymsubseteq}\ {\isasymPsi}\ {\isacharparenleft}{\kern0pt}star\ A\ {\isacharat}{\kern0pt}{\isacharat}{\kern0pt}\ star\ B{\isacharparenright}{\kern0pt}{\isachardoublequoteclose}\isanewline
%
\isadelimproof
%
\endisadelimproof
%
\isatagproof
\isakeywordONE{proof}\isamarkupfalse%
\isanewline
\ \ \isakeywordTHREE{fix}\isamarkupfalse%
\ v\isanewline
\ \ \isakeywordTHREE{assume}\isamarkupfalse%
\ {\isachardoublequoteopen}v\ {\isasymin}\ {\isasymPsi}\ {\isacharparenleft}{\kern0pt}star\ {\isacharparenleft}{\kern0pt}A\ {\isacharat}{\kern0pt}{\isacharat}{\kern0pt}\ B{\isacharparenright}{\kern0pt}{\isacharparenright}{\kern0pt}{\isachardoublequoteclose}\isanewline
\ \ \isakeywordONE{then}\isamarkupfalse%
\ \isakeywordONE{have}\isamarkupfalse%
\ {\isachardoublequoteopen}{\isasymexists}n{\isachardot}{\kern0pt}\ v\ {\isasymin}\ {\isasymPsi}\ {\isacharparenleft}{\kern0pt}{\isacharparenleft}{\kern0pt}A\ {\isacharat}{\kern0pt}{\isacharat}{\kern0pt}\ B{\isacharparenright}{\kern0pt}\ {\isacharcircum}{\kern0pt}{\isacharcircum}{\kern0pt}\ n{\isacharparenright}{\kern0pt}{\isachardoublequoteclose}\ \isakeywordONE{unfolding}\isamarkupfalse%
\ star{\isacharunderscore}{\kern0pt}def\ \isakeywordONE{by}\isamarkupfalse%
\ {\isacharparenleft}{\kern0pt}simp\ add{\isacharcolon}{\kern0pt}\ parikh{\isacharunderscore}{\kern0pt}img{\isacharunderscore}{\kern0pt}UNION{\isacharparenright}{\kern0pt}\isanewline
\ \ \isakeywordONE{then}\isamarkupfalse%
\ \isakeywordTHREE{obtain}\isamarkupfalse%
\ n\ \isakeywordTWO{where}\ {\isachardoublequoteopen}v\ {\isasymin}\ {\isasymPsi}\ {\isacharparenleft}{\kern0pt}{\isacharparenleft}{\kern0pt}A\ {\isacharat}{\kern0pt}{\isacharat}{\kern0pt}\ B{\isacharparenright}{\kern0pt}\ {\isacharcircum}{\kern0pt}{\isacharcircum}{\kern0pt}\ n{\isacharparenright}{\kern0pt}{\isachardoublequoteclose}\ \isakeywordONE{by}\isamarkupfalse%
\ blast\isanewline
\ \ \isakeywordONE{with}\isamarkupfalse%
\ parikh{\isacharunderscore}{\kern0pt}img{\isacharunderscore}{\kern0pt}conc{\isacharunderscore}{\kern0pt}pow\ \isakeywordONE{have}\isamarkupfalse%
\ {\isachardoublequoteopen}v\ {\isasymin}\ {\isasymPsi}\ {\isacharparenleft}{\kern0pt}A\ {\isacharcircum}{\kern0pt}{\isacharcircum}{\kern0pt}\ n\ {\isacharat}{\kern0pt}{\isacharat}{\kern0pt}\ B\ {\isacharcircum}{\kern0pt}{\isacharcircum}{\kern0pt}\ n{\isacharparenright}{\kern0pt}{\isachardoublequoteclose}\ \isakeywordONE{by}\isamarkupfalse%
\ fast\isanewline
\ \ \isakeywordONE{then}\isamarkupfalse%
\ \isakeywordONE{have}\isamarkupfalse%
\ {\isachardoublequoteopen}v\ {\isasymin}\ {\isasymPsi}\ {\isacharparenleft}{\kern0pt}A\ {\isacharcircum}{\kern0pt}{\isacharcircum}{\kern0pt}\ n\ {\isacharat}{\kern0pt}{\isacharat}{\kern0pt}\ star\ B{\isacharparenright}{\kern0pt}{\isachardoublequoteclose}\isanewline
\ \ \ \ \isakeywordONE{unfolding}\isamarkupfalse%
\ star{\isacharunderscore}{\kern0pt}def\ \isakeywordONE{using}\isamarkupfalse%
\ parikh{\isacharunderscore}{\kern0pt}conc{\isacharunderscore}{\kern0pt}left{\isacharunderscore}{\kern0pt}subset\isanewline
\ \ \ \ \isakeywordONE{by}\isamarkupfalse%
\ {\isacharparenleft}{\kern0pt}metis\ {\isacharparenleft}{\kern0pt}no{\isacharunderscore}{\kern0pt}types{\isacharcomma}{\kern0pt}\ lifting{\isacharparenright}{\kern0pt}\ Sup{\isacharunderscore}{\kern0pt}upper\ parikh{\isacharunderscore}{\kern0pt}img{\isacharunderscore}{\kern0pt}mono\ rangeI\ subset{\isacharunderscore}{\kern0pt}eq{\isacharparenright}{\kern0pt}\isanewline
\ \ \isakeywordONE{then}\isamarkupfalse%
\ \isakeywordTHREE{show}\isamarkupfalse%
\ {\isachardoublequoteopen}v\ {\isasymin}\ {\isasymPsi}\ {\isacharparenleft}{\kern0pt}star\ A\ {\isacharat}{\kern0pt}{\isacharat}{\kern0pt}\ star\ B{\isacharparenright}{\kern0pt}{\isachardoublequoteclose}\isanewline
\ \ \ \ \isakeywordONE{unfolding}\isamarkupfalse%
\ star{\isacharunderscore}{\kern0pt}def\ \isakeywordONE{using}\isamarkupfalse%
\ parikh{\isacharunderscore}{\kern0pt}conc{\isacharunderscore}{\kern0pt}right{\isacharunderscore}{\kern0pt}subset\isanewline
\ \ \ \ \isakeywordONE{by}\isamarkupfalse%
\ {\isacharparenleft}{\kern0pt}metis\ {\isacharparenleft}{\kern0pt}no{\isacharunderscore}{\kern0pt}types{\isacharcomma}{\kern0pt}\ lifting{\isacharparenright}{\kern0pt}\ Sup{\isacharunderscore}{\kern0pt}upper\ parikh{\isacharunderscore}{\kern0pt}img{\isacharunderscore}{\kern0pt}mono\ rangeI\ subset{\isacharunderscore}{\kern0pt}eq{\isacharparenright}{\kern0pt}\isanewline
\isakeywordONE{qed}\isamarkupfalse%
%
\endisatagproof
{\isafoldproof}%
%
\isadelimproof
\isanewline
%
\endisadelimproof
\isanewline
\isakeywordONE{lemma}\isamarkupfalse%
\ parikh{\isacharunderscore}{\kern0pt}img{\isacharunderscore}{\kern0pt}conc{\isacharunderscore}{\kern0pt}pow{\isadigit{2}}{\isacharcolon}{\kern0pt}\ {\isachardoublequoteopen}{\isasymPsi}\ {\isacharparenleft}{\kern0pt}{\isacharparenleft}{\kern0pt}A\ {\isacharat}{\kern0pt}{\isacharat}{\kern0pt}\ B{\isacharparenright}{\kern0pt}\ {\isacharcircum}{\kern0pt}{\isacharcircum}{\kern0pt}\ Suc\ n{\isacharparenright}{\kern0pt}\ {\isasymsubseteq}\ {\isasymPsi}\ {\isacharparenleft}{\kern0pt}star\ A\ {\isacharat}{\kern0pt}{\isacharat}{\kern0pt}\ star\ B\ {\isacharat}{\kern0pt}{\isacharat}{\kern0pt}\ B{\isacharparenright}{\kern0pt}{\isachardoublequoteclose}\isanewline
%
\isadelimproof
%
\endisadelimproof
%
\isatagproof
\isakeywordONE{proof}\isamarkupfalse%
\isanewline
\ \ \isakeywordTHREE{fix}\isamarkupfalse%
\ v\isanewline
\ \ \isakeywordTHREE{assume}\isamarkupfalse%
\ {\isachardoublequoteopen}v\ {\isasymin}\ {\isasymPsi}\ {\isacharparenleft}{\kern0pt}{\isacharparenleft}{\kern0pt}A\ {\isacharat}{\kern0pt}{\isacharat}{\kern0pt}\ B{\isacharparenright}{\kern0pt}\ {\isacharcircum}{\kern0pt}{\isacharcircum}{\kern0pt}\ Suc\ n{\isacharparenright}{\kern0pt}{\isachardoublequoteclose}\isanewline
\ \ \isakeywordONE{with}\isamarkupfalse%
\ parikh{\isacharunderscore}{\kern0pt}img{\isacharunderscore}{\kern0pt}conc{\isacharunderscore}{\kern0pt}pow\ \isakeywordONE{have}\isamarkupfalse%
\ {\isachardoublequoteopen}v\ {\isasymin}\ {\isasymPsi}\ {\isacharparenleft}{\kern0pt}A\ {\isacharcircum}{\kern0pt}{\isacharcircum}{\kern0pt}\ Suc\ n\ {\isacharat}{\kern0pt}{\isacharat}{\kern0pt}\ B\ {\isacharcircum}{\kern0pt}{\isacharcircum}{\kern0pt}\ n\ {\isacharat}{\kern0pt}{\isacharat}{\kern0pt}\ B{\isacharparenright}{\kern0pt}{\isachardoublequoteclose}\isanewline
\ \ \ \ \isakeywordONE{by}\isamarkupfalse%
\ {\isacharparenleft}{\kern0pt}metis\ conc{\isacharunderscore}{\kern0pt}pow{\isacharunderscore}{\kern0pt}comm\ lang{\isacharunderscore}{\kern0pt}pow{\isachardot}{\kern0pt}simps{\isacharparenleft}{\kern0pt}{\isadigit{2}}{\isacharparenright}{\kern0pt}\ subsetD{\isacharparenright}{\kern0pt}\isanewline
\ \ \isakeywordONE{then}\isamarkupfalse%
\ \isakeywordONE{have}\isamarkupfalse%
\ {\isachardoublequoteopen}v\ {\isasymin}\ {\isasymPsi}\ {\isacharparenleft}{\kern0pt}star\ A\ {\isacharat}{\kern0pt}{\isacharat}{\kern0pt}\ B\ {\isacharcircum}{\kern0pt}{\isacharcircum}{\kern0pt}\ n\ {\isacharat}{\kern0pt}{\isacharat}{\kern0pt}\ B{\isacharparenright}{\kern0pt}{\isachardoublequoteclose}\isanewline
\ \ \ \ \isakeywordONE{unfolding}\isamarkupfalse%
\ star{\isacharunderscore}{\kern0pt}def\ \isakeywordONE{using}\isamarkupfalse%
\ parikh{\isacharunderscore}{\kern0pt}conc{\isacharunderscore}{\kern0pt}right{\isacharunderscore}{\kern0pt}subset\isanewline
\ \ \ \ \isakeywordONE{by}\isamarkupfalse%
\ {\isacharparenleft}{\kern0pt}metis\ {\isacharparenleft}{\kern0pt}no{\isacharunderscore}{\kern0pt}types{\isacharcomma}{\kern0pt}\ lifting{\isacharparenright}{\kern0pt}\ Sup{\isacharunderscore}{\kern0pt}upper\ parikh{\isacharunderscore}{\kern0pt}img{\isacharunderscore}{\kern0pt}mono\ rangeI\ subset{\isacharunderscore}{\kern0pt}eq{\isacharparenright}{\kern0pt}\isanewline
\ \ \isakeywordONE{then}\isamarkupfalse%
\ \isakeywordTHREE{show}\isamarkupfalse%
\ {\isachardoublequoteopen}v\ {\isasymin}\ {\isasymPsi}\ {\isacharparenleft}{\kern0pt}star\ A\ {\isacharat}{\kern0pt}{\isacharat}{\kern0pt}\ star\ B\ {\isacharat}{\kern0pt}{\isacharat}{\kern0pt}\ B{\isacharparenright}{\kern0pt}{\isachardoublequoteclose}\isanewline
\ \ \ \ \isakeywordONE{unfolding}\isamarkupfalse%
\ star{\isacharunderscore}{\kern0pt}def\ \isakeywordONE{using}\isamarkupfalse%
\ parikh{\isacharunderscore}{\kern0pt}conc{\isacharunderscore}{\kern0pt}right{\isacharunderscore}{\kern0pt}subset\ parikh{\isacharunderscore}{\kern0pt}conc{\isacharunderscore}{\kern0pt}left{\isacharunderscore}{\kern0pt}subset\isanewline
\ \ \ \ \isakeywordONE{by}\isamarkupfalse%
\ {\isacharparenleft}{\kern0pt}metis\ {\isacharparenleft}{\kern0pt}no{\isacharunderscore}{\kern0pt}types{\isacharcomma}{\kern0pt}\ lifting{\isacharparenright}{\kern0pt}\ Sup{\isacharunderscore}{\kern0pt}upper\ parikh{\isacharunderscore}{\kern0pt}img{\isacharunderscore}{\kern0pt}mono\ rangeI\ subset{\isacharunderscore}{\kern0pt}eq{\isacharparenright}{\kern0pt}\isanewline
\isakeywordONE{qed}\isamarkupfalse%
%
\endisatagproof
{\isafoldproof}%
%
\isadelimproof
\isanewline
%
\endisadelimproof
\isanewline
\isakeywordONE{lemma}\isamarkupfalse%
\ parikh{\isacharunderscore}{\kern0pt}img{\isacharunderscore}{\kern0pt}star{\isadigit{2}}{\isacharunderscore}{\kern0pt}aux{\isadigit{1}}{\isacharcolon}{\kern0pt}\isanewline
\ \ {\isachardoublequoteopen}{\isasymPsi}\ {\isacharparenleft}{\kern0pt}star\ {\isacharparenleft}{\kern0pt}star\ E\ {\isacharat}{\kern0pt}{\isacharat}{\kern0pt}\ F{\isacharparenright}{\kern0pt}{\isacharparenright}{\kern0pt}\ {\isasymsubseteq}\ {\isasymPsi}\ {\isacharparenleft}{\kern0pt}{\isacharbraceleft}{\kern0pt}{\isacharbrackleft}{\kern0pt}{\isacharbrackright}{\kern0pt}{\isacharbraceright}{\kern0pt}\ {\isasymunion}\ star\ E\ {\isacharat}{\kern0pt}{\isacharat}{\kern0pt}\ star\ F\ {\isacharat}{\kern0pt}{\isacharat}{\kern0pt}\ F{\isacharparenright}{\kern0pt}{\isachardoublequoteclose}\isanewline
%
\isadelimproof
%
\endisadelimproof
%
\isatagproof
\isakeywordONE{proof}\isamarkupfalse%
\isanewline
\ \ \isakeywordTHREE{fix}\isamarkupfalse%
\ v\isanewline
\ \ \isakeywordTHREE{assume}\isamarkupfalse%
\ {\isachardoublequoteopen}v\ {\isasymin}\ {\isasymPsi}\ {\isacharparenleft}{\kern0pt}star\ {\isacharparenleft}{\kern0pt}star\ E\ {\isacharat}{\kern0pt}{\isacharat}{\kern0pt}\ F{\isacharparenright}{\kern0pt}{\isacharparenright}{\kern0pt}{\isachardoublequoteclose}\isanewline
\ \ \isakeywordONE{then}\isamarkupfalse%
\ \isakeywordONE{have}\isamarkupfalse%
\ {\isachardoublequoteopen}{\isasymexists}n{\isachardot}{\kern0pt}\ v\ {\isasymin}\ {\isasymPsi}\ {\isacharparenleft}{\kern0pt}{\isacharparenleft}{\kern0pt}star\ E\ {\isacharat}{\kern0pt}{\isacharat}{\kern0pt}\ F{\isacharparenright}{\kern0pt}\ {\isacharcircum}{\kern0pt}{\isacharcircum}{\kern0pt}\ n{\isacharparenright}{\kern0pt}{\isachardoublequoteclose}\isanewline
\ \ \ \ \isakeywordONE{unfolding}\isamarkupfalse%
\ star{\isacharunderscore}{\kern0pt}def\ \isakeywordONE{by}\isamarkupfalse%
\ {\isacharparenleft}{\kern0pt}simp\ add{\isacharcolon}{\kern0pt}\ parikh{\isacharunderscore}{\kern0pt}img{\isacharunderscore}{\kern0pt}UNION{\isacharparenright}{\kern0pt}\isanewline
\ \ \isakeywordONE{then}\isamarkupfalse%
\ \isakeywordTHREE{obtain}\isamarkupfalse%
\ n\ \isakeywordTWO{where}\ v{\isacharunderscore}{\kern0pt}in{\isacharunderscore}{\kern0pt}pow{\isacharunderscore}{\kern0pt}n{\isacharcolon}{\kern0pt}\ {\isachardoublequoteopen}v\ {\isasymin}\ {\isasymPsi}\ {\isacharparenleft}{\kern0pt}{\isacharparenleft}{\kern0pt}star\ E\ {\isacharat}{\kern0pt}{\isacharat}{\kern0pt}\ F{\isacharparenright}{\kern0pt}\ {\isacharcircum}{\kern0pt}{\isacharcircum}{\kern0pt}\ n{\isacharparenright}{\kern0pt}{\isachardoublequoteclose}\ \isakeywordONE{by}\isamarkupfalse%
\ blast\isanewline
\ \ \isakeywordTHREE{show}\isamarkupfalse%
\ {\isachardoublequoteopen}v\ {\isasymin}\ {\isasymPsi}\ {\isacharparenleft}{\kern0pt}{\isacharbraceleft}{\kern0pt}{\isacharbrackleft}{\kern0pt}{\isacharbrackright}{\kern0pt}{\isacharbraceright}{\kern0pt}\ {\isasymunion}\ star\ E\ {\isacharat}{\kern0pt}{\isacharat}{\kern0pt}\ star\ F\ {\isacharat}{\kern0pt}{\isacharat}{\kern0pt}\ F{\isacharparenright}{\kern0pt}{\isachardoublequoteclose}\isanewline
\ \ \isakeywordONE{proof}\isamarkupfalse%
\ {\isacharparenleft}{\kern0pt}cases\ n{\isacharparenright}{\kern0pt}\isanewline
\ \ \ \ \isakeywordTHREE{case}\isamarkupfalse%
\ {\isadigit{0}}\isanewline
\ \ \ \ \isakeywordONE{with}\isamarkupfalse%
\ v{\isacharunderscore}{\kern0pt}in{\isacharunderscore}{\kern0pt}pow{\isacharunderscore}{\kern0pt}n\ \isakeywordONE{have}\isamarkupfalse%
\ {\isachardoublequoteopen}v\ {\isacharequal}{\kern0pt}\ parikh{\isacharunderscore}{\kern0pt}vec\ {\isacharbrackleft}{\kern0pt}{\isacharbrackright}{\kern0pt}{\isachardoublequoteclose}\ \isakeywordONE{unfolding}\isamarkupfalse%
\ parikh{\isacharunderscore}{\kern0pt}img{\isacharunderscore}{\kern0pt}def\ \isakeywordONE{by}\isamarkupfalse%
\ simp\isanewline
\ \ \ \ \isakeywordONE{then}\isamarkupfalse%
\ \isakeywordTHREE{show}\isamarkupfalse%
\ {\isacharquery}{\kern0pt}thesis\ \isakeywordONE{unfolding}\isamarkupfalse%
\ parikh{\isacharunderscore}{\kern0pt}img{\isacharunderscore}{\kern0pt}def\ \isakeywordONE{by}\isamarkupfalse%
\ blast\isanewline
\ \ \isakeywordONE{next}\isamarkupfalse%
\isanewline
\ \ \ \ \isakeywordTHREE{case}\isamarkupfalse%
\ {\isacharparenleft}{\kern0pt}Suc\ m{\isacharparenright}{\kern0pt}\isanewline
\ \ \ \ \isakeywordONE{with}\isamarkupfalse%
\ parikh{\isacharunderscore}{\kern0pt}img{\isacharunderscore}{\kern0pt}conc{\isacharunderscore}{\kern0pt}pow{\isadigit{2}}\ v{\isacharunderscore}{\kern0pt}in{\isacharunderscore}{\kern0pt}pow{\isacharunderscore}{\kern0pt}n\ \isakeywordONE{have}\isamarkupfalse%
\ {\isachardoublequoteopen}v\ {\isasymin}\ {\isasymPsi}\ {\isacharparenleft}{\kern0pt}star\ {\isacharparenleft}{\kern0pt}star\ E{\isacharparenright}{\kern0pt}\ {\isacharat}{\kern0pt}{\isacharat}{\kern0pt}\ star\ F\ {\isacharat}{\kern0pt}{\isacharat}{\kern0pt}\ F{\isacharparenright}{\kern0pt}{\isachardoublequoteclose}\ \isakeywordONE{by}\isamarkupfalse%
\ blast\isanewline
\ \ \ \ \isakeywordONE{then}\isamarkupfalse%
\ \isakeywordTHREE{show}\isamarkupfalse%
\ {\isacharquery}{\kern0pt}thesis\ \isakeywordONE{by}\isamarkupfalse%
\ {\isacharparenleft}{\kern0pt}metis\ UnCI\ parikh{\isacharunderscore}{\kern0pt}img{\isacharunderscore}{\kern0pt}Un\ star{\isacharunderscore}{\kern0pt}idemp{\isacharparenright}{\kern0pt}\isanewline
\ \ \isakeywordONE{qed}\isamarkupfalse%
\isanewline
\isakeywordONE{qed}\isamarkupfalse%
%
\endisatagproof
{\isafoldproof}%
%
\isadelimproof
\isanewline
%
\endisadelimproof
\isanewline
\isakeywordONE{lemma}\isamarkupfalse%
\ parikh{\isacharunderscore}{\kern0pt}img{\isacharunderscore}{\kern0pt}star{\isadigit{2}}{\isacharunderscore}{\kern0pt}aux{\isadigit{2}}{\isacharcolon}{\kern0pt}\ {\isachardoublequoteopen}{\isasymPsi}\ {\isacharparenleft}{\kern0pt}star\ E\ {\isacharat}{\kern0pt}{\isacharat}{\kern0pt}\ star\ F\ {\isacharat}{\kern0pt}{\isacharat}{\kern0pt}\ F{\isacharparenright}{\kern0pt}\ {\isasymsubseteq}\ {\isasymPsi}\ {\isacharparenleft}{\kern0pt}star\ {\isacharparenleft}{\kern0pt}star\ E\ {\isacharat}{\kern0pt}{\isacharat}{\kern0pt}\ F{\isacharparenright}{\kern0pt}{\isacharparenright}{\kern0pt}{\isachardoublequoteclose}\isanewline
%
\isadelimproof
%
\endisadelimproof
%
\isatagproof
\isakeywordONE{proof}\isamarkupfalse%
\ {\isacharminus}{\kern0pt}\isanewline
\ \ \isakeywordONE{have}\isamarkupfalse%
\ {\isachardoublequoteopen}F\ {\isasymsubseteq}\ star\ E\ {\isacharat}{\kern0pt}{\isacharat}{\kern0pt}\ F{\isachardoublequoteclose}\ \isakeywordONE{unfolding}\isamarkupfalse%
\ star{\isacharunderscore}{\kern0pt}def\ \isakeywordONE{using}\isamarkupfalse%
\ Nil{\isacharunderscore}{\kern0pt}in{\isacharunderscore}{\kern0pt}star\isanewline
\ \ \ \ \isakeywordONE{by}\isamarkupfalse%
\ {\isacharparenleft}{\kern0pt}metis\ concI{\isacharunderscore}{\kern0pt}if{\isacharunderscore}{\kern0pt}Nil{\isadigit{1}}\ star{\isacharunderscore}{\kern0pt}def\ subsetI{\isacharparenright}{\kern0pt}\isanewline
\ \ \isakeywordONE{then}\isamarkupfalse%
\ \isakeywordONE{have}\isamarkupfalse%
\ {\isachardoublequoteopen}{\isasymPsi}\ {\isacharparenleft}{\kern0pt}star\ E\ {\isacharat}{\kern0pt}{\isacharat}{\kern0pt}\ F\ {\isacharat}{\kern0pt}{\isacharat}{\kern0pt}\ star\ F{\isacharparenright}{\kern0pt}\ {\isasymsubseteq}\ {\isasymPsi}\ {\isacharparenleft}{\kern0pt}star\ E\ {\isacharat}{\kern0pt}{\isacharat}{\kern0pt}\ F\ {\isacharat}{\kern0pt}{\isacharat}{\kern0pt}\ star\ {\isacharparenleft}{\kern0pt}star\ E\ {\isacharat}{\kern0pt}{\isacharat}{\kern0pt}\ F{\isacharparenright}{\kern0pt}{\isacharparenright}{\kern0pt}{\isachardoublequoteclose}\isanewline
\ \ \ \ \isakeywordONE{using}\isamarkupfalse%
\ parikh{\isacharunderscore}{\kern0pt}conc{\isacharunderscore}{\kern0pt}left{\isacharunderscore}{\kern0pt}subset\ parikh{\isacharunderscore}{\kern0pt}img{\isacharunderscore}{\kern0pt}mono\ parikh{\isacharunderscore}{\kern0pt}star{\isacharunderscore}{\kern0pt}mono\ \isakeywordONE{by}\isamarkupfalse%
\ meson\isanewline
\ \ \isakeywordONE{also}\isamarkupfalse%
\ \isakeywordONE{have}\isamarkupfalse%
\ {\isachardoublequoteopen}{\isasymdots}\ {\isasymsubseteq}\ {\isasymPsi}\ {\isacharparenleft}{\kern0pt}star\ {\isacharparenleft}{\kern0pt}star\ E\ {\isacharat}{\kern0pt}{\isacharat}{\kern0pt}\ F{\isacharparenright}{\kern0pt}{\isacharparenright}{\kern0pt}{\isachardoublequoteclose}\isanewline
\ \ \ \ \isakeywordONE{by}\isamarkupfalse%
\ {\isacharparenleft}{\kern0pt}metis\ conc{\isacharunderscore}{\kern0pt}assoc\ inf{\isacharunderscore}{\kern0pt}sup{\isacharunderscore}{\kern0pt}ord{\isacharparenleft}{\kern0pt}{\isadigit{3}}{\isacharparenright}{\kern0pt}\ parikh{\isacharunderscore}{\kern0pt}img{\isacharunderscore}{\kern0pt}mono\ star{\isacharunderscore}{\kern0pt}unfold{\isacharunderscore}{\kern0pt}left{\isacharparenright}{\kern0pt}\isanewline
\ \ \isakeywordONE{finally}\isamarkupfalse%
\ \isakeywordTHREE{show}\isamarkupfalse%
\ {\isacharquery}{\kern0pt}thesis\ \isakeywordONE{using}\isamarkupfalse%
\ conc{\isacharunderscore}{\kern0pt}star{\isacharunderscore}{\kern0pt}comm\ \isakeywordONE{by}\isamarkupfalse%
\ metis\isanewline
\isakeywordONE{qed}\isamarkupfalse%
%
\endisatagproof
{\isafoldproof}%
%
\isadelimproof
\isanewline
%
\endisadelimproof
\isanewline
\isakeywordONE{lemma}\isamarkupfalse%
\ parikh{\isacharunderscore}{\kern0pt}img{\isacharunderscore}{\kern0pt}star{\isadigit{2}}{\isacharcolon}{\kern0pt}\ {\isachardoublequoteopen}{\isasymPsi}\ {\isacharparenleft}{\kern0pt}star\ {\isacharparenleft}{\kern0pt}star\ E\ {\isacharat}{\kern0pt}{\isacharat}{\kern0pt}\ F{\isacharparenright}{\kern0pt}{\isacharparenright}{\kern0pt}\ {\isacharequal}{\kern0pt}\ {\isasymPsi}\ {\isacharparenleft}{\kern0pt}{\isacharbraceleft}{\kern0pt}{\isacharbrackleft}{\kern0pt}{\isacharbrackright}{\kern0pt}{\isacharbraceright}{\kern0pt}\ {\isasymunion}\ star\ E\ {\isacharat}{\kern0pt}{\isacharat}{\kern0pt}\ star\ F\ {\isacharat}{\kern0pt}{\isacharat}{\kern0pt}\ F{\isacharparenright}{\kern0pt}{\isachardoublequoteclose}\isanewline
%
\isadelimproof
%
\endisadelimproof
%
\isatagproof
\isakeywordONE{proof}\isamarkupfalse%
\isanewline
\ \ \isakeywordONE{from}\isamarkupfalse%
\ parikh{\isacharunderscore}{\kern0pt}img{\isacharunderscore}{\kern0pt}star{\isadigit{2}}{\isacharunderscore}{\kern0pt}aux{\isadigit{1}}\isanewline
\ \ \ \ \isakeywordTHREE{show}\isamarkupfalse%
\ {\isachardoublequoteopen}{\isasymPsi}\ {\isacharparenleft}{\kern0pt}star\ {\isacharparenleft}{\kern0pt}star\ E\ {\isacharat}{\kern0pt}{\isacharat}{\kern0pt}\ F{\isacharparenright}{\kern0pt}{\isacharparenright}{\kern0pt}\ {\isasymsubseteq}\ {\isasymPsi}\ {\isacharparenleft}{\kern0pt}{\isacharbraceleft}{\kern0pt}{\isacharbrackleft}{\kern0pt}{\isacharbrackright}{\kern0pt}{\isacharbraceright}{\kern0pt}\ {\isasymunion}\ star\ E\ {\isacharat}{\kern0pt}{\isacharat}{\kern0pt}\ star\ F\ {\isacharat}{\kern0pt}{\isacharat}{\kern0pt}\ F{\isacharparenright}{\kern0pt}{\isachardoublequoteclose}\ \isakeywordONE{{\isachardot}{\kern0pt}}\isamarkupfalse%
\isanewline
\ \ \isakeywordONE{from}\isamarkupfalse%
\ parikh{\isacharunderscore}{\kern0pt}img{\isacharunderscore}{\kern0pt}star{\isadigit{2}}{\isacharunderscore}{\kern0pt}aux{\isadigit{2}}\isanewline
\ \ \ \ \isakeywordTHREE{show}\isamarkupfalse%
\ {\isachardoublequoteopen}{\isasymPsi}\ {\isacharparenleft}{\kern0pt}{\isacharbraceleft}{\kern0pt}{\isacharbrackleft}{\kern0pt}{\isacharbrackright}{\kern0pt}{\isacharbraceright}{\kern0pt}\ {\isasymunion}\ star\ E\ {\isacharat}{\kern0pt}{\isacharat}{\kern0pt}\ star\ F\ {\isacharat}{\kern0pt}{\isacharat}{\kern0pt}\ F{\isacharparenright}{\kern0pt}\ {\isasymsubseteq}\ {\isasymPsi}\ {\isacharparenleft}{\kern0pt}star\ {\isacharparenleft}{\kern0pt}star\ E\ {\isacharat}{\kern0pt}{\isacharat}{\kern0pt}\ F{\isacharparenright}{\kern0pt}{\isacharparenright}{\kern0pt}{\isachardoublequoteclose}\isanewline
\ \ \ \ \isakeywordONE{by}\isamarkupfalse%
\ {\isacharparenleft}{\kern0pt}metis\ le{\isacharunderscore}{\kern0pt}sup{\isacharunderscore}{\kern0pt}iff\ parikh{\isacharunderscore}{\kern0pt}img{\isacharunderscore}{\kern0pt}Un\ star{\isacharunderscore}{\kern0pt}unfold{\isacharunderscore}{\kern0pt}left\ sup{\isachardot}{\kern0pt}cobounded{\isadigit{2}}{\isacharparenright}{\kern0pt}\isanewline
\isakeywordONE{qed}\isamarkupfalse%
%
\endisatagproof
{\isafoldproof}%
%
\isadelimproof
%
\endisadelimproof
%
\isadelimdocument
%
\endisadelimdocument
%
\isatagdocument
%
\isamarkupsubsection{A homogeneous-like property for regular language expressions%
}
\isamarkuptrue%
%
\endisatagdocument
{\isafolddocument}%
%
\isadelimdocument
%
\endisadelimdocument
%
\begin{isamarkuptext}%
\label{sec:rlexp_homogeneous}%
\end{isamarkuptext}\isamarkuptrue%
\isakeywordONE{lemma}\isamarkupfalse%
\ rlexp{\isacharunderscore}{\kern0pt}homogeneous{\isacharunderscore}{\kern0pt}aux{\isacharcolon}{\kern0pt}\isanewline
\ \ \isakeywordTWO{assumes}\ {\isachardoublequoteopen}v\ x\ {\isacharequal}{\kern0pt}\ star\ Y\ {\isacharat}{\kern0pt}{\isacharat}{\kern0pt}\ Z{\isachardoublequoteclose}\isanewline
\ \ \ \ \isakeywordTWO{shows}\ {\isachardoublequoteopen}{\isasymPsi}\ {\isacharparenleft}{\kern0pt}eval\ f\ v{\isacharparenright}{\kern0pt}\ {\isasymsubseteq}\ {\isasymPsi}\ {\isacharparenleft}{\kern0pt}star\ Y\ {\isacharat}{\kern0pt}{\isacharat}{\kern0pt}\ eval\ f\ {\isacharparenleft}{\kern0pt}v{\isacharparenleft}{\kern0pt}x\ {\isacharcolon}{\kern0pt}{\isacharequal}{\kern0pt}\ Z{\isacharparenright}{\kern0pt}{\isacharparenright}{\kern0pt}{\isacharparenright}{\kern0pt}{\isachardoublequoteclose}\isanewline
%
\isadelimproof
%
\endisadelimproof
%
\isatagproof
\isakeywordONE{proof}\isamarkupfalse%
\ {\isacharparenleft}{\kern0pt}induction\ f{\isacharparenright}{\kern0pt}\isanewline
\ \ \isakeywordTHREE{case}\isamarkupfalse%
\ {\isacharparenleft}{\kern0pt}Var\ y{\isacharparenright}{\kern0pt}\isanewline
\ \ \isakeywordTHREE{show}\isamarkupfalse%
\ {\isacharquery}{\kern0pt}case\isanewline
\ \ \isakeywordONE{proof}\isamarkupfalse%
\ {\isacharparenleft}{\kern0pt}cases\ {\isachardoublequoteopen}x\ {\isacharequal}{\kern0pt}\ y{\isachardoublequoteclose}{\isacharparenright}{\kern0pt}\isanewline
\ \ \ \ \isakeywordTHREE{case}\isamarkupfalse%
\ True\isanewline
\ \ \ \ \isakeywordONE{with}\isamarkupfalse%
\ Var\ assms\ \isakeywordTHREE{show}\isamarkupfalse%
\ {\isacharquery}{\kern0pt}thesis\ \isakeywordONE{by}\isamarkupfalse%
\ simp\isanewline
\ \ \isakeywordONE{next}\isamarkupfalse%
\isanewline
\ \ \ \ \isakeywordTHREE{case}\isamarkupfalse%
\ False\isanewline
\ \ \ \ \isakeywordONE{have}\isamarkupfalse%
\ {\isachardoublequoteopen}eval\ {\isacharparenleft}{\kern0pt}Var\ y{\isacharparenright}{\kern0pt}\ v\ {\isasymsubseteq}\ star\ Y\ {\isacharat}{\kern0pt}{\isacharat}{\kern0pt}\ eval\ {\isacharparenleft}{\kern0pt}Var\ y{\isacharparenright}{\kern0pt}\ v{\isachardoublequoteclose}\ \isakeywordONE{by}\isamarkupfalse%
\ {\isacharparenleft}{\kern0pt}metis\ Nil{\isacharunderscore}{\kern0pt}in{\isacharunderscore}{\kern0pt}star\ concI{\isacharunderscore}{\kern0pt}if{\isacharunderscore}{\kern0pt}Nil{\isadigit{1}}\ subsetI{\isacharparenright}{\kern0pt}\isanewline
\ \ \ \ \isakeywordONE{with}\isamarkupfalse%
\ False\ parikh{\isacharunderscore}{\kern0pt}img{\isacharunderscore}{\kern0pt}mono\ \isakeywordTHREE{show}\isamarkupfalse%
\ {\isacharquery}{\kern0pt}thesis\ \isakeywordONE{by}\isamarkupfalse%
\ auto\isanewline
\ \ \isakeywordONE{qed}\isamarkupfalse%
\isanewline
\isakeywordONE{next}\isamarkupfalse%
\isanewline
\ \ \isakeywordTHREE{case}\isamarkupfalse%
\ {\isacharparenleft}{\kern0pt}Const\ l{\isacharparenright}{\kern0pt}\isanewline
\ \ \isakeywordONE{have}\isamarkupfalse%
\ {\isachardoublequoteopen}eval\ {\isacharparenleft}{\kern0pt}Const\ l{\isacharparenright}{\kern0pt}\ v\ {\isasymsubseteq}\ star\ Y\ {\isacharat}{\kern0pt}{\isacharat}{\kern0pt}\ eval\ {\isacharparenleft}{\kern0pt}Const\ l{\isacharparenright}{\kern0pt}\ v{\isachardoublequoteclose}\ \isakeywordONE{using}\isamarkupfalse%
\ concI{\isacharunderscore}{\kern0pt}if{\isacharunderscore}{\kern0pt}Nil{\isadigit{1}}\ \isakeywordONE{by}\isamarkupfalse%
\ blast\isanewline
\ \ \isakeywordONE{then}\isamarkupfalse%
\ \isakeywordTHREE{show}\isamarkupfalse%
\ {\isacharquery}{\kern0pt}case\ \isakeywordONE{by}\isamarkupfalse%
\ {\isacharparenleft}{\kern0pt}simp\ add{\isacharcolon}{\kern0pt}\ parikh{\isacharunderscore}{\kern0pt}img{\isacharunderscore}{\kern0pt}mono{\isacharparenright}{\kern0pt}\isanewline
\isakeywordONE{next}\isamarkupfalse%
\isanewline
\ \ \isakeywordTHREE{case}\isamarkupfalse%
\ {\isacharparenleft}{\kern0pt}Union\ f\ g{\isacharparenright}{\kern0pt}\isanewline
\ \ \isakeywordONE{then}\isamarkupfalse%
\ \isakeywordONE{have}\isamarkupfalse%
\ {\isachardoublequoteopen}{\isasymPsi}\ {\isacharparenleft}{\kern0pt}eval\ {\isacharparenleft}{\kern0pt}Union\ f\ g{\isacharparenright}{\kern0pt}\ v{\isacharparenright}{\kern0pt}\ {\isasymsubseteq}\ {\isasymPsi}\ {\isacharparenleft}{\kern0pt}star\ Y\ {\isacharat}{\kern0pt}{\isacharat}{\kern0pt}\ eval\ f\ {\isacharparenleft}{\kern0pt}v{\isacharparenleft}{\kern0pt}x\ {\isacharcolon}{\kern0pt}{\isacharequal}{\kern0pt}\ Z{\isacharparenright}{\kern0pt}{\isacharparenright}{\kern0pt}\ {\isasymunion}\isanewline
\ \ \ \ \ \ \ \ \ \ \ \ \ \ \ \ \ \ \ \ \ \ \ \ \ \ \ \ \ \ \ \ \ \ \ \ \ \ \ \ \ \ \ \ \ \ \ \ \ \ \ \ \ \ \ \ \ \ \ \ star\ Y\ {\isacharat}{\kern0pt}{\isacharat}{\kern0pt}\ eval\ g\ {\isacharparenleft}{\kern0pt}v{\isacharparenleft}{\kern0pt}x\ {\isacharcolon}{\kern0pt}{\isacharequal}{\kern0pt}\ Z{\isacharparenright}{\kern0pt}{\isacharparenright}{\kern0pt}{\isacharparenright}{\kern0pt}{\isachardoublequoteclose}\isanewline
\ \ \ \ \isakeywordONE{by}\isamarkupfalse%
\ {\isacharparenleft}{\kern0pt}metis\ eval{\isachardot}{\kern0pt}simps{\isacharparenleft}{\kern0pt}{\isadigit{3}}{\isacharparenright}{\kern0pt}\ parikh{\isacharunderscore}{\kern0pt}img{\isacharunderscore}{\kern0pt}Un\ sup{\isachardot}{\kern0pt}mono{\isacharparenright}{\kern0pt}\isanewline
\ \ \isakeywordONE{then}\isamarkupfalse%
\ \isakeywordTHREE{show}\isamarkupfalse%
\ {\isacharquery}{\kern0pt}case\ \isakeywordONE{by}\isamarkupfalse%
\ {\isacharparenleft}{\kern0pt}metis\ conc{\isacharunderscore}{\kern0pt}Un{\isacharunderscore}{\kern0pt}distrib{\isacharparenleft}{\kern0pt}{\isadigit{1}}{\isacharparenright}{\kern0pt}\ eval{\isachardot}{\kern0pt}simps{\isacharparenleft}{\kern0pt}{\isadigit{3}}{\isacharparenright}{\kern0pt}{\isacharparenright}{\kern0pt}\isanewline
\isakeywordONE{next}\isamarkupfalse%
\isanewline
\ \ \isakeywordTHREE{case}\isamarkupfalse%
\ {\isacharparenleft}{\kern0pt}Concat\ f\ g{\isacharparenright}{\kern0pt}\isanewline
\ \ \isakeywordONE{then}\isamarkupfalse%
\ \isakeywordONE{have}\isamarkupfalse%
\ {\isachardoublequoteopen}{\isasymPsi}\ {\isacharparenleft}{\kern0pt}eval\ {\isacharparenleft}{\kern0pt}Concat\ f\ g{\isacharparenright}{\kern0pt}\ v{\isacharparenright}{\kern0pt}\ {\isasymsubseteq}\ {\isasymPsi}\ {\isacharparenleft}{\kern0pt}{\isacharparenleft}{\kern0pt}star\ Y\ {\isacharat}{\kern0pt}{\isacharat}{\kern0pt}\ eval\ f\ {\isacharparenleft}{\kern0pt}v{\isacharparenleft}{\kern0pt}x\ {\isacharcolon}{\kern0pt}{\isacharequal}{\kern0pt}\ Z{\isacharparenright}{\kern0pt}{\isacharparenright}{\kern0pt}{\isacharparenright}{\kern0pt}\isanewline
\ \ \ \ \ \ \ \ \ \ \ \ \ \ \ \ \ \ \ \ \ \ \ \ \ \ \ \ \ \ \ \ \ \ \ \ \ \ \ \ \ \ \ \ \ \ \ \ \ \ \ \ \ \ \ \ \ \ {\isacharat}{\kern0pt}{\isacharat}{\kern0pt}\ star\ Y\ {\isacharat}{\kern0pt}{\isacharat}{\kern0pt}\ eval\ g\ {\isacharparenleft}{\kern0pt}v{\isacharparenleft}{\kern0pt}x\ {\isacharcolon}{\kern0pt}{\isacharequal}{\kern0pt}\ Z{\isacharparenright}{\kern0pt}{\isacharparenright}{\kern0pt}{\isacharparenright}{\kern0pt}{\isachardoublequoteclose}\isanewline
\ \ \ \ \isakeywordONE{by}\isamarkupfalse%
\ {\isacharparenleft}{\kern0pt}metis\ eval{\isachardot}{\kern0pt}simps{\isacharparenleft}{\kern0pt}{\isadigit{4}}{\isacharparenright}{\kern0pt}\ parikh{\isacharunderscore}{\kern0pt}conc{\isacharunderscore}{\kern0pt}subset{\isacharparenright}{\kern0pt}\isanewline
\ \ \isakeywordONE{also}\isamarkupfalse%
\ \isakeywordONE{have}\isamarkupfalse%
\ {\isachardoublequoteopen}{\isasymdots}\ {\isacharequal}{\kern0pt}\ {\isasymPsi}\ {\isacharparenleft}{\kern0pt}star\ Y\ {\isacharat}{\kern0pt}{\isacharat}{\kern0pt}\ star\ Y\ {\isacharat}{\kern0pt}{\isacharat}{\kern0pt}\ eval\ f\ {\isacharparenleft}{\kern0pt}v{\isacharparenleft}{\kern0pt}x\ {\isacharcolon}{\kern0pt}{\isacharequal}{\kern0pt}\ Z{\isacharparenright}{\kern0pt}{\isacharparenright}{\kern0pt}\ {\isacharat}{\kern0pt}{\isacharat}{\kern0pt}\ eval\ g\ {\isacharparenleft}{\kern0pt}v{\isacharparenleft}{\kern0pt}x\ {\isacharcolon}{\kern0pt}{\isacharequal}{\kern0pt}\ Z{\isacharparenright}{\kern0pt}{\isacharparenright}{\kern0pt}{\isacharparenright}{\kern0pt}{\isachardoublequoteclose}\isanewline
\ \ \ \ \isakeywordONE{by}\isamarkupfalse%
\ {\isacharparenleft}{\kern0pt}metis\ conc{\isacharunderscore}{\kern0pt}assoc\ parikh{\isacharunderscore}{\kern0pt}conc{\isacharunderscore}{\kern0pt}right\ parikh{\isacharunderscore}{\kern0pt}img{\isacharunderscore}{\kern0pt}commut{\isacharparenright}{\kern0pt}\isanewline
\ \ \isakeywordONE{also}\isamarkupfalse%
\ \isakeywordONE{have}\isamarkupfalse%
\ {\isachardoublequoteopen}{\isasymdots}\ {\isacharequal}{\kern0pt}\ {\isasymPsi}\ {\isacharparenleft}{\kern0pt}star\ Y\ {\isacharat}{\kern0pt}{\isacharat}{\kern0pt}\ eval\ f\ {\isacharparenleft}{\kern0pt}v{\isacharparenleft}{\kern0pt}x\ {\isacharcolon}{\kern0pt}{\isacharequal}{\kern0pt}\ Z{\isacharparenright}{\kern0pt}{\isacharparenright}{\kern0pt}\ {\isacharat}{\kern0pt}{\isacharat}{\kern0pt}\ eval\ g\ {\isacharparenleft}{\kern0pt}v{\isacharparenleft}{\kern0pt}x\ {\isacharcolon}{\kern0pt}{\isacharequal}{\kern0pt}\ Z{\isacharparenright}{\kern0pt}{\isacharparenright}{\kern0pt}{\isacharparenright}{\kern0pt}{\isachardoublequoteclose}\isanewline
\ \ \ \ \isakeywordONE{by}\isamarkupfalse%
\ {\isacharparenleft}{\kern0pt}metis\ conc{\isacharunderscore}{\kern0pt}assoc\ conc{\isacharunderscore}{\kern0pt}star{\isacharunderscore}{\kern0pt}star{\isacharparenright}{\kern0pt}\isanewline
\ \ \isakeywordONE{finally}\isamarkupfalse%
\ \isakeywordTHREE{show}\isamarkupfalse%
\ {\isacharquery}{\kern0pt}case\ \isakeywordONE{by}\isamarkupfalse%
\ {\isacharparenleft}{\kern0pt}metis\ eval{\isachardot}{\kern0pt}simps{\isacharparenleft}{\kern0pt}{\isadigit{4}}{\isacharparenright}{\kern0pt}{\isacharparenright}{\kern0pt}\isanewline
\isakeywordONE{next}\isamarkupfalse%
\isanewline
\ \ \isakeywordTHREE{case}\isamarkupfalse%
\ {\isacharparenleft}{\kern0pt}Star\ f{\isacharparenright}{\kern0pt}\isanewline
\ \ \isakeywordONE{then}\isamarkupfalse%
\ \isakeywordONE{have}\isamarkupfalse%
\ {\isachardoublequoteopen}{\isasymPsi}\ {\isacharparenleft}{\kern0pt}star\ {\isacharparenleft}{\kern0pt}eval\ f\ v{\isacharparenright}{\kern0pt}{\isacharparenright}{\kern0pt}\ {\isasymsubseteq}\ {\isasymPsi}\ {\isacharparenleft}{\kern0pt}star\ {\isacharparenleft}{\kern0pt}star\ Y\ {\isacharat}{\kern0pt}{\isacharat}{\kern0pt}\ eval\ f\ {\isacharparenleft}{\kern0pt}v{\isacharparenleft}{\kern0pt}x\ {\isacharcolon}{\kern0pt}{\isacharequal}{\kern0pt}\ Z{\isacharparenright}{\kern0pt}{\isacharparenright}{\kern0pt}{\isacharparenright}{\kern0pt}{\isacharparenright}{\kern0pt}{\isachardoublequoteclose}\isanewline
\ \ \ \ \isakeywordONE{using}\isamarkupfalse%
\ parikh{\isacharunderscore}{\kern0pt}star{\isacharunderscore}{\kern0pt}mono\ \isakeywordONE{by}\isamarkupfalse%
\ metis\isanewline
\ \ \isakeywordONE{also}\isamarkupfalse%
\ \isakeywordONE{from}\isamarkupfalse%
\ parikh{\isacharunderscore}{\kern0pt}img{\isacharunderscore}{\kern0pt}conc{\isacharunderscore}{\kern0pt}star\ \isakeywordONE{have}\isamarkupfalse%
\ {\isachardoublequoteopen}{\isasymdots}\ {\isasymsubseteq}\ {\isasymPsi}\ {\isacharparenleft}{\kern0pt}star\ Y\ {\isacharat}{\kern0pt}{\isacharat}{\kern0pt}\ star\ {\isacharparenleft}{\kern0pt}eval\ f\ {\isacharparenleft}{\kern0pt}v{\isacharparenleft}{\kern0pt}x\ {\isacharcolon}{\kern0pt}{\isacharequal}{\kern0pt}\ Z{\isacharparenright}{\kern0pt}{\isacharparenright}{\kern0pt}{\isacharparenright}{\kern0pt}{\isacharparenright}{\kern0pt}{\isachardoublequoteclose}\isanewline
\ \ \ \ \isakeywordONE{by}\isamarkupfalse%
\ fastforce\isanewline
\ \ \isakeywordONE{finally}\isamarkupfalse%
\ \isakeywordTHREE{show}\isamarkupfalse%
\ {\isacharquery}{\kern0pt}case\ \isakeywordONE{by}\isamarkupfalse%
\ {\isacharparenleft}{\kern0pt}metis\ eval{\isachardot}{\kern0pt}simps{\isacharparenleft}{\kern0pt}{\isadigit{5}}{\isacharparenright}{\kern0pt}{\isacharparenright}{\kern0pt}\isanewline
\isakeywordONE{qed}\isamarkupfalse%
%
\endisatagproof
{\isafoldproof}%
%
\isadelimproof
%
\endisadelimproof
%
\begin{isamarkuptext}%
Now we can prove the desired homogeneous-like property which will become useful later.
Notably this property slightly differs from the property claimed in \cite{Pilling}. However, our
property is easier to prove formally and it suffices for the rest of the proof.%
\end{isamarkuptext}\isamarkuptrue%
\isakeywordONE{lemma}\isamarkupfalse%
\ rlexp{\isacharunderscore}{\kern0pt}homogeneous{\isacharcolon}{\kern0pt}\ \ {\isachardoublequoteopen}{\isasymPsi}\ {\isacharparenleft}{\kern0pt}eval\ {\isacharparenleft}{\kern0pt}subst\ {\isacharparenleft}{\kern0pt}Var{\isacharparenleft}{\kern0pt}x\ {\isacharcolon}{\kern0pt}{\isacharequal}{\kern0pt}\ Concat\ {\isacharparenleft}{\kern0pt}Star\ y{\isacharparenright}{\kern0pt}\ z{\isacharparenright}{\kern0pt}{\isacharparenright}{\kern0pt}\ f{\isacharparenright}{\kern0pt}\ v{\isacharparenright}{\kern0pt}\isanewline
\ \ \ \ \ \ \ \ \ \ \ \ \ \ \ \ \ \ \ \ \ \ \ \ \ \ {\isasymsubseteq}\ {\isasymPsi}\ {\isacharparenleft}{\kern0pt}eval\ {\isacharparenleft}{\kern0pt}Concat\ {\isacharparenleft}{\kern0pt}Star\ y{\isacharparenright}{\kern0pt}\ {\isacharparenleft}{\kern0pt}subst\ {\isacharparenleft}{\kern0pt}Var{\isacharparenleft}{\kern0pt}x\ {\isacharcolon}{\kern0pt}{\isacharequal}{\kern0pt}\ z{\isacharparenright}{\kern0pt}{\isacharparenright}{\kern0pt}\ f{\isacharparenright}{\kern0pt}{\isacharparenright}{\kern0pt}\ v{\isacharparenright}{\kern0pt}{\isachardoublequoteclose}\isanewline
\ \ \ \ \ \ \ \ \ \ \ \ \ \ \ \ \ \ \ \ \ \ \ \ \ \ {\isacharparenleft}{\kern0pt}\isakeywordTWO{is}\ {\isachardoublequoteopen}{\isasymPsi}\ {\isacharquery}{\kern0pt}L\ {\isasymsubseteq}\ {\isasymPsi}\ {\isacharquery}{\kern0pt}R{\isachardoublequoteclose}{\isacharparenright}{\kern0pt}\isanewline
%
\isadelimproof
%
\endisadelimproof
%
\isatagproof
\isakeywordONE{proof}\isamarkupfalse%
\ {\isacharminus}{\kern0pt}\isanewline
\ \ \isakeywordONE{let}\isamarkupfalse%
\ {\isacharquery}{\kern0pt}v{\isacharprime}{\kern0pt}\ {\isacharequal}{\kern0pt}\ {\isachardoublequoteopen}v{\isacharparenleft}{\kern0pt}x\ {\isacharcolon}{\kern0pt}{\isacharequal}{\kern0pt}\ star\ {\isacharparenleft}{\kern0pt}eval\ y\ v{\isacharparenright}{\kern0pt}\ {\isacharat}{\kern0pt}{\isacharat}{\kern0pt}\ eval\ z\ v{\isacharparenright}{\kern0pt}{\isachardoublequoteclose}\isanewline
\ \ \isakeywordONE{have}\isamarkupfalse%
\ {\isachardoublequoteopen}{\isasymPsi}\ {\isacharquery}{\kern0pt}L\ {\isacharequal}{\kern0pt}\ {\isasymPsi}\ {\isacharparenleft}{\kern0pt}eval\ f\ {\isacharquery}{\kern0pt}v{\isacharprime}{\kern0pt}{\isacharparenright}{\kern0pt}{\isachardoublequoteclose}\ \isakeywordONE{using}\isamarkupfalse%
\ substitution{\isacharunderscore}{\kern0pt}lemma{\isacharunderscore}{\kern0pt}upd{\isacharbrackleft}{\kern0pt}\isakeywordTWO{where}\ f{\isacharequal}{\kern0pt}f{\isacharbrackright}{\kern0pt}\ \isakeywordONE{by}\isamarkupfalse%
\ simp\isanewline
\ \ \isakeywordONE{also}\isamarkupfalse%
\ \isakeywordONE{have}\isamarkupfalse%
\ {\isachardoublequoteopen}{\isasymdots}\ {\isasymsubseteq}\ {\isasymPsi}\ {\isacharparenleft}{\kern0pt}star\ {\isacharparenleft}{\kern0pt}eval\ y\ v{\isacharparenright}{\kern0pt}\ {\isacharat}{\kern0pt}{\isacharat}{\kern0pt}\ eval\ f\ {\isacharparenleft}{\kern0pt}{\isacharquery}{\kern0pt}v{\isacharprime}{\kern0pt}{\isacharparenleft}{\kern0pt}x\ {\isacharcolon}{\kern0pt}{\isacharequal}{\kern0pt}\ eval\ z\ v{\isacharparenright}{\kern0pt}{\isacharparenright}{\kern0pt}{\isacharparenright}{\kern0pt}{\isachardoublequoteclose}\isanewline
\ \ \ \ \isakeywordONE{using}\isamarkupfalse%
\ rlexp{\isacharunderscore}{\kern0pt}homogeneous{\isacharunderscore}{\kern0pt}aux{\isacharbrackleft}{\kern0pt}of\ {\isacharquery}{\kern0pt}v{\isacharprime}{\kern0pt}{\isacharbrackright}{\kern0pt}\ \isakeywordONE{unfolding}\isamarkupfalse%
\ fun{\isacharunderscore}{\kern0pt}upd{\isacharunderscore}{\kern0pt}def\ \isakeywordONE{by}\isamarkupfalse%
\ auto\isanewline
\ \ \isakeywordONE{also}\isamarkupfalse%
\ \isakeywordONE{have}\isamarkupfalse%
\ {\isachardoublequoteopen}{\isasymdots}\ {\isacharequal}{\kern0pt}\ {\isasymPsi}\ {\isacharquery}{\kern0pt}R{\isachardoublequoteclose}\ \isakeywordONE{using}\isamarkupfalse%
\ substitution{\isacharunderscore}{\kern0pt}lemma{\isacharbrackleft}{\kern0pt}of\ {\isachardoublequoteopen}v{\isacharparenleft}{\kern0pt}x\ {\isacharcolon}{\kern0pt}{\isacharequal}{\kern0pt}\ eval\ z\ v{\isacharparenright}{\kern0pt}{\isachardoublequoteclose}{\isacharbrackright}{\kern0pt}\ \isakeywordONE{by}\isamarkupfalse%
\ simp\isanewline
\ \ \isakeywordONE{finally}\isamarkupfalse%
\ \isakeywordTHREE{show}\isamarkupfalse%
\ {\isacharquery}{\kern0pt}thesis\ \isakeywordONE{{\isachardot}{\kern0pt}}\isamarkupfalse%
\isanewline
\isakeywordONE{qed}\isamarkupfalse%
%
\endisatagproof
{\isafoldproof}%
%
\isadelimproof
%
\endisadelimproof
%
\isadelimdocument
%
\endisadelimdocument
%
\isatagdocument
%
\isamarkupsubsection{Extension of Arden's lemma to Parikh images%
}
\isamarkuptrue%
%
\endisatagdocument
{\isafolddocument}%
%
\isadelimdocument
%
\endisadelimdocument
%
\begin{isamarkuptext}%
\label{sec:parikh_arden}%
\end{isamarkuptext}\isamarkuptrue%
\isakeywordONE{lemma}\isamarkupfalse%
\ parikh{\isacharunderscore}{\kern0pt}img{\isacharunderscore}{\kern0pt}arden{\isacharunderscore}{\kern0pt}aux{\isacharcolon}{\kern0pt}\isanewline
\ \ \isakeywordTWO{assumes}\ {\isachardoublequoteopen}{\isasymPsi}\ {\isacharparenleft}{\kern0pt}A\ {\isacharat}{\kern0pt}{\isacharat}{\kern0pt}\ X\ {\isasymunion}\ B{\isacharparenright}{\kern0pt}\ {\isasymsubseteq}\ {\isasymPsi}\ X{\isachardoublequoteclose}\isanewline
\ \ \isakeywordTWO{shows}\ {\isachardoublequoteopen}{\isasymPsi}\ {\isacharparenleft}{\kern0pt}A\ {\isacharcircum}{\kern0pt}{\isacharcircum}{\kern0pt}\ n\ {\isacharat}{\kern0pt}{\isacharat}{\kern0pt}\ B{\isacharparenright}{\kern0pt}\ {\isasymsubseteq}\ {\isasymPsi}\ X{\isachardoublequoteclose}\isanewline
%
\isadelimproof
%
\endisadelimproof
%
\isatagproof
\isakeywordONE{proof}\isamarkupfalse%
\ {\isacharparenleft}{\kern0pt}induction\ n{\isacharparenright}{\kern0pt}\isanewline
\ \ \isakeywordTHREE{case}\isamarkupfalse%
\ {\isadigit{0}}\isanewline
\ \ \isakeywordONE{with}\isamarkupfalse%
\ assms\ \isakeywordTHREE{show}\isamarkupfalse%
\ {\isacharquery}{\kern0pt}case\ \isakeywordONE{by}\isamarkupfalse%
\ auto\isanewline
\isakeywordONE{next}\isamarkupfalse%
\isanewline
\ \ \isakeywordTHREE{case}\isamarkupfalse%
\ {\isacharparenleft}{\kern0pt}Suc\ n{\isacharparenright}{\kern0pt}\isanewline
\ \ \isakeywordONE{then}\isamarkupfalse%
\ \isakeywordONE{have}\isamarkupfalse%
\ {\isachardoublequoteopen}{\isasymPsi}\ {\isacharparenleft}{\kern0pt}A\ {\isacharcircum}{\kern0pt}{\isacharcircum}{\kern0pt}\ {\isacharparenleft}{\kern0pt}Suc\ n{\isacharparenright}{\kern0pt}\ {\isacharat}{\kern0pt}{\isacharat}{\kern0pt}\ B{\isacharparenright}{\kern0pt}\ {\isasymsubseteq}\ {\isasymPsi}\ {\isacharparenleft}{\kern0pt}A\ {\isacharat}{\kern0pt}{\isacharat}{\kern0pt}\ A\ {\isacharcircum}{\kern0pt}{\isacharcircum}{\kern0pt}\ n\ {\isacharat}{\kern0pt}{\isacharat}{\kern0pt}B{\isacharparenright}{\kern0pt}{\isachardoublequoteclose}\isanewline
\ \ \ \ \isakeywordONE{by}\isamarkupfalse%
\ {\isacharparenleft}{\kern0pt}simp\ add{\isacharcolon}{\kern0pt}\ conc{\isacharunderscore}{\kern0pt}assoc{\isacharparenright}{\kern0pt}\isanewline
\ \ \isakeywordONE{moreover}\isamarkupfalse%
\ \isakeywordONE{from}\isamarkupfalse%
\ Suc\ parikh{\isacharunderscore}{\kern0pt}conc{\isacharunderscore}{\kern0pt}left\ \isakeywordONE{have}\isamarkupfalse%
\ {\isachardoublequoteopen}{\isasymdots}\ {\isasymsubseteq}\ {\isasymPsi}\ {\isacharparenleft}{\kern0pt}A\ {\isacharat}{\kern0pt}{\isacharat}{\kern0pt}\ X{\isacharparenright}{\kern0pt}{\isachardoublequoteclose}\isanewline
\ \ \ \ \isakeywordONE{by}\isamarkupfalse%
\ {\isacharparenleft}{\kern0pt}metis\ conc{\isacharunderscore}{\kern0pt}Un{\isacharunderscore}{\kern0pt}distrib{\isacharparenleft}{\kern0pt}{\isadigit{1}}{\isacharparenright}{\kern0pt}\ parikh{\isacharunderscore}{\kern0pt}img{\isacharunderscore}{\kern0pt}Un\ sup{\isachardot}{\kern0pt}orderE\ sup{\isachardot}{\kern0pt}orderI{\isacharparenright}{\kern0pt}\isanewline
\ \ \isakeywordONE{moreover}\isamarkupfalse%
\ \isakeywordONE{from}\isamarkupfalse%
\ Suc{\isachardot}{\kern0pt}prems\ assms\ \isakeywordONE{have}\isamarkupfalse%
\ {\isachardoublequoteopen}{\isasymdots}\ {\isasymsubseteq}\ {\isasymPsi}\ X{\isachardoublequoteclose}\ \isakeywordONE{by}\isamarkupfalse%
\ auto\isanewline
\ \ \isakeywordONE{ultimately}\isamarkupfalse%
\ \isakeywordTHREE{show}\isamarkupfalse%
\ {\isacharquery}{\kern0pt}case\ \isakeywordONE{by}\isamarkupfalse%
\ fast\isanewline
\isakeywordONE{qed}\isamarkupfalse%
%
\endisatagproof
{\isafoldproof}%
%
\isadelimproof
\isanewline
%
\endisadelimproof
\isanewline
\isakeywordONE{lemma}\isamarkupfalse%
\ parikh{\isacharunderscore}{\kern0pt}img{\isacharunderscore}{\kern0pt}arden{\isacharcolon}{\kern0pt}\isanewline
\ \ \isakeywordTWO{assumes}\ {\isachardoublequoteopen}{\isasymPsi}\ {\isacharparenleft}{\kern0pt}A\ {\isacharat}{\kern0pt}{\isacharat}{\kern0pt}\ X\ {\isasymunion}\ B{\isacharparenright}{\kern0pt}\ {\isasymsubseteq}\ {\isasymPsi}\ X{\isachardoublequoteclose}\isanewline
\ \ \isakeywordTWO{shows}\ {\isachardoublequoteopen}{\isasymPsi}\ {\isacharparenleft}{\kern0pt}star\ A\ {\isacharat}{\kern0pt}{\isacharat}{\kern0pt}\ B{\isacharparenright}{\kern0pt}\ {\isasymsubseteq}\ {\isasymPsi}\ X{\isachardoublequoteclose}\isanewline
%
\isadelimproof
%
\endisadelimproof
%
\isatagproof
\isakeywordONE{proof}\isamarkupfalse%
\isanewline
\ \ \isakeywordTHREE{fix}\isamarkupfalse%
\ x\isanewline
\ \ \isakeywordTHREE{assume}\isamarkupfalse%
\ {\isachardoublequoteopen}x\ {\isasymin}\ {\isasymPsi}\ {\isacharparenleft}{\kern0pt}star\ A\ {\isacharat}{\kern0pt}{\isacharat}{\kern0pt}\ B{\isacharparenright}{\kern0pt}{\isachardoublequoteclose}\isanewline
\ \ \isakeywordONE{then}\isamarkupfalse%
\ \isakeywordONE{have}\isamarkupfalse%
\ {\isachardoublequoteopen}{\isasymexists}n{\isachardot}{\kern0pt}\ x\ {\isasymin}\ {\isasymPsi}\ {\isacharparenleft}{\kern0pt}A\ {\isacharcircum}{\kern0pt}{\isacharcircum}{\kern0pt}\ n\ {\isacharat}{\kern0pt}{\isacharat}{\kern0pt}\ B{\isacharparenright}{\kern0pt}{\isachardoublequoteclose}\isanewline
\ \ \ \ \isakeywordONE{unfolding}\isamarkupfalse%
\ star{\isacharunderscore}{\kern0pt}def\ \isakeywordONE{by}\isamarkupfalse%
\ {\isacharparenleft}{\kern0pt}simp\ add{\isacharcolon}{\kern0pt}\ conc{\isacharunderscore}{\kern0pt}UNION{\isacharunderscore}{\kern0pt}distrib{\isacharparenleft}{\kern0pt}{\isadigit{2}}{\isacharparenright}{\kern0pt}\ parikh{\isacharunderscore}{\kern0pt}img{\isacharunderscore}{\kern0pt}UNION{\isacharparenright}{\kern0pt}\isanewline
\ \ \isakeywordONE{then}\isamarkupfalse%
\ \isakeywordTHREE{obtain}\isamarkupfalse%
\ n\ \isakeywordTWO{where}\ {\isachardoublequoteopen}x\ {\isasymin}\ {\isasymPsi}\ {\isacharparenleft}{\kern0pt}A\ {\isacharcircum}{\kern0pt}{\isacharcircum}{\kern0pt}\ n\ {\isacharat}{\kern0pt}{\isacharat}{\kern0pt}\ B{\isacharparenright}{\kern0pt}{\isachardoublequoteclose}\ \isakeywordONE{by}\isamarkupfalse%
\ blast\isanewline
\ \ \isakeywordONE{then}\isamarkupfalse%
\ \isakeywordTHREE{show}\isamarkupfalse%
\ {\isachardoublequoteopen}x\ {\isasymin}\ {\isasymPsi}\ X{\isachardoublequoteclose}\ \isakeywordONE{using}\isamarkupfalse%
\ parikh{\isacharunderscore}{\kern0pt}img{\isacharunderscore}{\kern0pt}arden{\isacharunderscore}{\kern0pt}aux{\isacharbrackleft}{\kern0pt}OF\ assms{\isacharbrackright}{\kern0pt}\ \isakeywordONE{by}\isamarkupfalse%
\ fast\isanewline
\isakeywordONE{qed}\isamarkupfalse%
%
\endisatagproof
{\isafoldproof}%
%
\isadelimproof
%
\endisadelimproof
%
\isadelimdocument
%
\endisadelimdocument
%
\isatagdocument
%
\isamarkupsubsection{Equivalence class of languages with identical Parikh image%
}
\isamarkuptrue%
%
\endisatagdocument
{\isafolddocument}%
%
\isadelimdocument
%
\endisadelimdocument
%
\begin{isamarkuptext}%
\label{sec:parikh_eq_class}%
\end{isamarkuptext}\isamarkuptrue%
%
\begin{isamarkuptext}%
For a given language \isa{L}, we define the equivalence class of all languages with identical Parikh
image:%
\end{isamarkuptext}\isamarkuptrue%
\isakeywordONE{definition}\isamarkupfalse%
\ parikh{\isacharunderscore}{\kern0pt}img{\isacharunderscore}{\kern0pt}eq{\isacharunderscore}{\kern0pt}class\ {\isacharcolon}{\kern0pt}{\isacharcolon}{\kern0pt}\ {\isachardoublequoteopen}{\isacharprime}{\kern0pt}a\ lang\ {\isasymRightarrow}\ {\isacharprime}{\kern0pt}a\ lang\ set{\isachardoublequoteclose}\ \isakeywordTWO{where}\isanewline
\ \ {\isachardoublequoteopen}parikh{\isacharunderscore}{\kern0pt}img{\isacharunderscore}{\kern0pt}eq{\isacharunderscore}{\kern0pt}class\ L\ {\isasymequiv}\ {\isacharbraceleft}{\kern0pt}L{\isacharprime}{\kern0pt}{\isachardot}{\kern0pt}\ {\isasymPsi}\ L{\isacharprime}{\kern0pt}\ {\isacharequal}{\kern0pt}\ {\isasymPsi}\ L{\isacharbraceright}{\kern0pt}{\isachardoublequoteclose}\isanewline
\isanewline
\isakeywordONE{lemma}\isamarkupfalse%
\ parikh{\isacharunderscore}{\kern0pt}img{\isacharunderscore}{\kern0pt}Union{\isacharunderscore}{\kern0pt}class{\isacharcolon}{\kern0pt}\ {\isachardoublequoteopen}{\isasymPsi}\ A\ {\isacharequal}{\kern0pt}\ {\isasymPsi}\ {\isacharparenleft}{\kern0pt}{\isasymUnion}{\isacharparenleft}{\kern0pt}parikh{\isacharunderscore}{\kern0pt}img{\isacharunderscore}{\kern0pt}eq{\isacharunderscore}{\kern0pt}class\ A{\isacharparenright}{\kern0pt}{\isacharparenright}{\kern0pt}{\isachardoublequoteclose}\isanewline
%
\isadelimproof
%
\endisadelimproof
%
\isatagproof
\isakeywordONE{proof}\isamarkupfalse%
\isanewline
\ \ \isakeywordONE{let}\isamarkupfalse%
\ {\isacharquery}{\kern0pt}A{\isacharprime}{\kern0pt}\ {\isacharequal}{\kern0pt}\ {\isachardoublequoteopen}{\isasymUnion}{\isacharparenleft}{\kern0pt}parikh{\isacharunderscore}{\kern0pt}img{\isacharunderscore}{\kern0pt}eq{\isacharunderscore}{\kern0pt}class\ A{\isacharparenright}{\kern0pt}{\isachardoublequoteclose}\isanewline
\ \ \isakeywordTHREE{show}\isamarkupfalse%
\ {\isachardoublequoteopen}{\isasymPsi}\ A\ {\isasymsubseteq}\ {\isasymPsi}\ {\isacharquery}{\kern0pt}A{\isacharprime}{\kern0pt}{\isachardoublequoteclose}\isanewline
\ \ \ \ \isakeywordONE{unfolding}\isamarkupfalse%
\ parikh{\isacharunderscore}{\kern0pt}img{\isacharunderscore}{\kern0pt}eq{\isacharunderscore}{\kern0pt}class{\isacharunderscore}{\kern0pt}def\ \isakeywordONE{by}\isamarkupfalse%
\ {\isacharparenleft}{\kern0pt}simp\ add{\isacharcolon}{\kern0pt}\ Union{\isacharunderscore}{\kern0pt}upper\ parikh{\isacharunderscore}{\kern0pt}img{\isacharunderscore}{\kern0pt}mono{\isacharparenright}{\kern0pt}\isanewline
\ \ \isakeywordTHREE{show}\isamarkupfalse%
\ {\isachardoublequoteopen}{\isasymPsi}\ {\isacharquery}{\kern0pt}A{\isacharprime}{\kern0pt}\ {\isasymsubseteq}\ {\isasymPsi}\ A{\isachardoublequoteclose}\isanewline
\ \ \isakeywordONE{proof}\isamarkupfalse%
\isanewline
\ \ \ \ \isakeywordTHREE{fix}\isamarkupfalse%
\ v\isanewline
\ \ \ \ \isakeywordTHREE{assume}\isamarkupfalse%
\ {\isachardoublequoteopen}v\ {\isasymin}\ {\isasymPsi}\ {\isacharquery}{\kern0pt}A{\isacharprime}{\kern0pt}{\isachardoublequoteclose}\isanewline
\ \ \ \ \isakeywordONE{then}\isamarkupfalse%
\ \isakeywordTHREE{obtain}\isamarkupfalse%
\ a\ \isakeywordTWO{where}\ a{\isacharunderscore}{\kern0pt}intro{\isacharcolon}{\kern0pt}\ {\isachardoublequoteopen}parikh{\isacharunderscore}{\kern0pt}vec\ a\ {\isacharequal}{\kern0pt}\ v\ {\isasymand}\ a\ {\isasymin}\ {\isacharquery}{\kern0pt}A{\isacharprime}{\kern0pt}{\isachardoublequoteclose}\isanewline
\ \ \ \ \ \ \isakeywordONE{unfolding}\isamarkupfalse%
\ parikh{\isacharunderscore}{\kern0pt}img{\isacharunderscore}{\kern0pt}def\ \isakeywordONE{by}\isamarkupfalse%
\ blast\isanewline
\ \ \ \ \isakeywordONE{then}\isamarkupfalse%
\ \isakeywordTHREE{obtain}\isamarkupfalse%
\ L\ \isakeywordTWO{where}\ L{\isacharunderscore}{\kern0pt}intro{\isacharcolon}{\kern0pt}\ {\isachardoublequoteopen}a\ {\isasymin}\ L\ {\isasymand}\ L\ {\isasymin}\ parikh{\isacharunderscore}{\kern0pt}img{\isacharunderscore}{\kern0pt}eq{\isacharunderscore}{\kern0pt}class\ A{\isachardoublequoteclose}\isanewline
\ \ \ \ \ \ \isakeywordONE{unfolding}\isamarkupfalse%
\ parikh{\isacharunderscore}{\kern0pt}img{\isacharunderscore}{\kern0pt}eq{\isacharunderscore}{\kern0pt}class{\isacharunderscore}{\kern0pt}def\ \isakeywordONE{by}\isamarkupfalse%
\ blast\isanewline
\ \ \ \ \isakeywordONE{then}\isamarkupfalse%
\ \isakeywordONE{have}\isamarkupfalse%
\ {\isachardoublequoteopen}{\isasymPsi}\ L\ {\isacharequal}{\kern0pt}\ {\isasymPsi}\ A{\isachardoublequoteclose}\ \isakeywordONE{unfolding}\isamarkupfalse%
\ parikh{\isacharunderscore}{\kern0pt}img{\isacharunderscore}{\kern0pt}eq{\isacharunderscore}{\kern0pt}class{\isacharunderscore}{\kern0pt}def\ \isakeywordONE{by}\isamarkupfalse%
\ fastforce\isanewline
\ \ \ \ \isakeywordONE{with}\isamarkupfalse%
\ a{\isacharunderscore}{\kern0pt}intro\ L{\isacharunderscore}{\kern0pt}intro\ \isakeywordTHREE{show}\isamarkupfalse%
\ {\isachardoublequoteopen}v\ {\isasymin}\ {\isasymPsi}\ A{\isachardoublequoteclose}\ \isakeywordONE{unfolding}\isamarkupfalse%
\ parikh{\isacharunderscore}{\kern0pt}img{\isacharunderscore}{\kern0pt}def\ \isakeywordONE{by}\isamarkupfalse%
\ blast\isanewline
\ \ \isakeywordONE{qed}\isamarkupfalse%
\isanewline
\isakeywordONE{qed}\isamarkupfalse%
%
\endisatagproof
{\isafoldproof}%
%
\isadelimproof
\isanewline
%
\endisadelimproof
\isanewline
\isakeywordONE{lemma}\isamarkupfalse%
\ subseteq{\isacharunderscore}{\kern0pt}comm{\isacharunderscore}{\kern0pt}subseteq{\isacharcolon}{\kern0pt}\isanewline
\ \ \isakeywordTWO{assumes}\ {\isachardoublequoteopen}{\isasymPsi}\ A\ {\isasymsubseteq}\ {\isasymPsi}\ B{\isachardoublequoteclose}\isanewline
\ \ \isakeywordTWO{shows}\ {\isachardoublequoteopen}A\ {\isasymsubseteq}\ {\isasymUnion}{\isacharparenleft}{\kern0pt}parikh{\isacharunderscore}{\kern0pt}img{\isacharunderscore}{\kern0pt}eq{\isacharunderscore}{\kern0pt}class\ B{\isacharparenright}{\kern0pt}{\isachardoublequoteclose}\ {\isacharparenleft}{\kern0pt}\isakeywordTWO{is}\ {\isachardoublequoteopen}A\ {\isasymsubseteq}\ {\isacharquery}{\kern0pt}B{\isacharprime}{\kern0pt}{\isachardoublequoteclose}{\isacharparenright}{\kern0pt}\isanewline
%
\isadelimproof
%
\endisadelimproof
%
\isatagproof
\isakeywordONE{proof}\isamarkupfalse%
\isanewline
\ \ \isakeywordTHREE{fix}\isamarkupfalse%
\ a\isanewline
\ \ \isakeywordTHREE{assume}\isamarkupfalse%
\ a{\isacharunderscore}{\kern0pt}in{\isacharunderscore}{\kern0pt}A{\isacharcolon}{\kern0pt}\ {\isachardoublequoteopen}a\ {\isasymin}\ A{\isachardoublequoteclose}\isanewline
\ \ \isakeywordONE{from}\isamarkupfalse%
\ assms\ \isakeywordONE{have}\isamarkupfalse%
\ {\isachardoublequoteopen}{\isasymPsi}\ A\ {\isasymsubseteq}\ {\isasymPsi}\ {\isacharquery}{\kern0pt}B{\isacharprime}{\kern0pt}{\isachardoublequoteclose}\isanewline
\ \ \ \ \isakeywordONE{using}\isamarkupfalse%
\ parikh{\isacharunderscore}{\kern0pt}img{\isacharunderscore}{\kern0pt}Union{\isacharunderscore}{\kern0pt}class\ \isakeywordONE{by}\isamarkupfalse%
\ blast\isanewline
\ \ \isakeywordONE{with}\isamarkupfalse%
\ a{\isacharunderscore}{\kern0pt}in{\isacharunderscore}{\kern0pt}A\ \isakeywordONE{have}\isamarkupfalse%
\ vec{\isacharunderscore}{\kern0pt}a{\isacharunderscore}{\kern0pt}in{\isacharunderscore}{\kern0pt}B{\isacharprime}{\kern0pt}{\isacharcolon}{\kern0pt}\ {\isachardoublequoteopen}parikh{\isacharunderscore}{\kern0pt}vec\ a\ {\isasymin}\ {\isasymPsi}\ {\isacharquery}{\kern0pt}B{\isacharprime}{\kern0pt}{\isachardoublequoteclose}\ \isakeywordONE{unfolding}\isamarkupfalse%
\ parikh{\isacharunderscore}{\kern0pt}img{\isacharunderscore}{\kern0pt}def\ \isakeywordONE{by}\isamarkupfalse%
\ fast\isanewline
\ \ \isakeywordONE{then}\isamarkupfalse%
\ \isakeywordONE{have}\isamarkupfalse%
\ {\isachardoublequoteopen}{\isasymexists}b{\isachardot}{\kern0pt}\ parikh{\isacharunderscore}{\kern0pt}vec\ b\ {\isacharequal}{\kern0pt}\ parikh{\isacharunderscore}{\kern0pt}vec\ a\ {\isasymand}\ b\ {\isasymin}\ {\isacharquery}{\kern0pt}B{\isacharprime}{\kern0pt}{\isachardoublequoteclose}\isanewline
\ \ \ \ \isakeywordONE{unfolding}\isamarkupfalse%
\ parikh{\isacharunderscore}{\kern0pt}img{\isacharunderscore}{\kern0pt}def\ \isakeywordONE{by}\isamarkupfalse%
\ fastforce\isanewline
\ \ \isakeywordONE{then}\isamarkupfalse%
\ \isakeywordTHREE{obtain}\isamarkupfalse%
\ b\ \isakeywordTWO{where}\ b{\isacharunderscore}{\kern0pt}intro{\isacharcolon}{\kern0pt}\ {\isachardoublequoteopen}parikh{\isacharunderscore}{\kern0pt}vec\ b\ {\isacharequal}{\kern0pt}\ parikh{\isacharunderscore}{\kern0pt}vec\ a\ {\isasymand}\ b\ {\isasymin}\ {\isacharquery}{\kern0pt}B{\isacharprime}{\kern0pt}{\isachardoublequoteclose}\ \isakeywordONE{by}\isamarkupfalse%
\ blast\isanewline
\ \ \isakeywordONE{with}\isamarkupfalse%
\ vec{\isacharunderscore}{\kern0pt}a{\isacharunderscore}{\kern0pt}in{\isacharunderscore}{\kern0pt}B{\isacharprime}{\kern0pt}\ \isakeywordONE{have}\isamarkupfalse%
\ {\isachardoublequoteopen}{\isasymPsi}\ {\isacharparenleft}{\kern0pt}{\isacharquery}{\kern0pt}B{\isacharprime}{\kern0pt}\ {\isasymunion}\ {\isacharbraceleft}{\kern0pt}a{\isacharbraceright}{\kern0pt}{\isacharparenright}{\kern0pt}\ {\isacharequal}{\kern0pt}\ {\isasymPsi}\ {\isacharquery}{\kern0pt}B{\isacharprime}{\kern0pt}{\isachardoublequoteclose}\isakeywordONE{unfolding}\isamarkupfalse%
\ parikh{\isacharunderscore}{\kern0pt}img{\isacharunderscore}{\kern0pt}def\ \isakeywordONE{by}\isamarkupfalse%
\ blast\isanewline
\ \ \isakeywordONE{with}\isamarkupfalse%
\ parikh{\isacharunderscore}{\kern0pt}img{\isacharunderscore}{\kern0pt}Union{\isacharunderscore}{\kern0pt}class\ \isakeywordONE{have}\isamarkupfalse%
\ {\isachardoublequoteopen}{\isasymPsi}\ {\isacharparenleft}{\kern0pt}{\isacharquery}{\kern0pt}B{\isacharprime}{\kern0pt}\ {\isasymunion}\ {\isacharbraceleft}{\kern0pt}a{\isacharbraceright}{\kern0pt}{\isacharparenright}{\kern0pt}\ {\isacharequal}{\kern0pt}\ {\isasymPsi}\ B{\isachardoublequoteclose}\ \isakeywordONE{by}\isamarkupfalse%
\ blast\isanewline
\ \ \isakeywordONE{then}\isamarkupfalse%
\ \isakeywordTHREE{show}\isamarkupfalse%
\ {\isachardoublequoteopen}a\ {\isasymin}\ {\isacharquery}{\kern0pt}B{\isacharprime}{\kern0pt}{\isachardoublequoteclose}\ \isakeywordONE{unfolding}\isamarkupfalse%
\ parikh{\isacharunderscore}{\kern0pt}img{\isacharunderscore}{\kern0pt}eq{\isacharunderscore}{\kern0pt}class{\isacharunderscore}{\kern0pt}def\ \isakeywordONE{by}\isamarkupfalse%
\ blast\isanewline
\isakeywordONE{qed}\isamarkupfalse%
%
\endisatagproof
{\isafoldproof}%
%
\isadelimproof
\isanewline
%
\endisadelimproof
%
\isadelimtheory
\isanewline
%
\endisadelimtheory
%
\isatagtheory
\isakeywordTWO{end}\isamarkupfalse%
%
\endisatagtheory
{\isafoldtheory}%
%
\isadelimtheory
%
\endisadelimtheory
%
\end{isabellebody}%
\endinput
%:%file=~/studium/semester_7/semantik/homeworks/AIST/Parikh/Parikh_Img.thy%:%
%:%11=3%:%
%:%27=5%:%
%:%28=5%:%
%:%29=6%:%
%:%30=7%:%
%:%31=8%:%
%:%32=9%:%
%:%46=12%:%
%:%58=14%:%
%:%59=15%:%
%:%60=16%:%
%:%62=18%:%
%:%63=18%:%
%:%64=19%:%
%:%65=20%:%
%:%66=21%:%
%:%67=21%:%
%:%68=22%:%
%:%69=23%:%
%:%70=24%:%
%:%71=24%:%
%:%74=25%:%
%:%78=25%:%
%:%79=25%:%
%:%84=25%:%
%:%87=26%:%
%:%88=27%:%
%:%89=27%:%
%:%92=28%:%
%:%96=28%:%
%:%97=28%:%
%:%102=28%:%
%:%105=29%:%
%:%106=30%:%
%:%107=30%:%
%:%110=31%:%
%:%114=31%:%
%:%115=31%:%
%:%116=31%:%
%:%121=31%:%
%:%124=32%:%
%:%125=33%:%
%:%126=33%:%
%:%133=34%:%
%:%134=34%:%
%:%135=35%:%
%:%136=35%:%
%:%137=36%:%
%:%138=37%:%
%:%139=37%:%
%:%140=37%:%
%:%141=38%:%
%:%142=38%:%
%:%143=38%:%
%:%144=39%:%
%:%145=39%:%
%:%146=39%:%
%:%147=40%:%
%:%162=43%:%
%:%172=45%:%
%:%173=45%:%
%:%176=46%:%
%:%180=46%:%
%:%181=46%:%
%:%182=46%:%
%:%187=46%:%
%:%190=47%:%
%:%191=48%:%
%:%192=48%:%
%:%195=49%:%
%:%199=49%:%
%:%200=49%:%
%:%205=49%:%
%:%208=50%:%
%:%209=51%:%
%:%210=51%:%
%:%213=52%:%
%:%217=52%:%
%:%218=52%:%
%:%223=52%:%
%:%226=53%:%
%:%227=54%:%
%:%228=54%:%
%:%229=55%:%
%:%230=56%:%
%:%231=57%:%
%:%234=58%:%
%:%238=58%:%
%:%239=58%:%
%:%240=58%:%
%:%245=58%:%
%:%248=59%:%
%:%249=60%:%
%:%250=60%:%
%:%253=61%:%
%:%257=61%:%
%:%258=61%:%
%:%263=61%:%
%:%266=62%:%
%:%267=63%:%
%:%268=63%:%
%:%271=64%:%
%:%275=64%:%
%:%276=64%:%
%:%281=64%:%
%:%284=65%:%
%:%285=66%:%
%:%286=66%:%
%:%289=67%:%
%:%293=67%:%
%:%294=67%:%
%:%299=67%:%
%:%302=68%:%
%:%303=69%:%
%:%304=70%:%
%:%305=70%:%
%:%306=71%:%
%:%307=72%:%
%:%314=73%:%
%:%315=73%:%
%:%316=74%:%
%:%317=74%:%
%:%318=75%:%
%:%319=75%:%
%:%320=76%:%
%:%321=76%:%
%:%322=76%:%
%:%323=76%:%
%:%324=76%:%
%:%325=77%:%
%:%326=77%:%
%:%327=77%:%
%:%328=77%:%
%:%329=77%:%
%:%330=78%:%
%:%331=78%:%
%:%332=78%:%
%:%333=78%:%
%:%334=78%:%
%:%335=78%:%
%:%336=79%:%
%:%337=79%:%
%:%338=79%:%
%:%339=79%:%
%:%340=79%:%
%:%341=80%:%
%:%342=80%:%
%:%343=80%:%
%:%344=80%:%
%:%345=80%:%
%:%346=80%:%
%:%347=81%:%
%:%353=81%:%
%:%356=82%:%
%:%357=83%:%
%:%358=83%:%
%:%359=84%:%
%:%360=85%:%
%:%363=86%:%
%:%367=86%:%
%:%368=86%:%
%:%369=86%:%
%:%374=86%:%
%:%377=87%:%
%:%378=88%:%
%:%379=89%:%
%:%380=89%:%
%:%381=90%:%
%:%382=91%:%
%:%389=92%:%
%:%390=92%:%
%:%391=93%:%
%:%392=93%:%
%:%393=94%:%
%:%394=94%:%
%:%395=94%:%
%:%396=95%:%
%:%397=96%:%
%:%398=96%:%
%:%399=96%:%
%:%400=97%:%
%:%401=97%:%
%:%402=97%:%
%:%403=97%:%
%:%404=98%:%
%:%405=98%:%
%:%406=99%:%
%:%407=99%:%
%:%408=100%:%
%:%409=100%:%
%:%410=100%:%
%:%411=101%:%
%:%412=101%:%
%:%413=101%:%
%:%414=102%:%
%:%415=102%:%
%:%416=102%:%
%:%417=102%:%
%:%418=103%:%
%:%419=103%:%
%:%424=103%:%
%:%427=104%:%
%:%428=105%:%
%:%429=105%:%
%:%430=106%:%
%:%431=107%:%
%:%434=108%:%
%:%438=108%:%
%:%439=108%:%
%:%440=108%:%
%:%445=108%:%
%:%448=109%:%
%:%449=110%:%
%:%450=110%:%
%:%451=111%:%
%:%452=112%:%
%:%459=113%:%
%:%460=113%:%
%:%461=113%:%
%:%462=114%:%
%:%463=114%:%
%:%464=115%:%
%:%465=115%:%
%:%466=115%:%
%:%467=116%:%
%:%468=116%:%
%:%469=116%:%
%:%470=117%:%
%:%471=117%:%
%:%472=117%:%
%:%473=117%:%
%:%474=118%:%
%:%475=118%:%
%:%480=118%:%
%:%483=119%:%
%:%484=120%:%
%:%485=120%:%
%:%486=121%:%
%:%487=122%:%
%:%490=123%:%
%:%494=123%:%
%:%495=123%:%
%:%496=123%:%
%:%510=127%:%
%:%522=129%:%
%:%526=130%:%
%:%528=132%:%
%:%529=132%:%
%:%530=133%:%
%:%531=134%:%
%:%538=135%:%
%:%539=135%:%
%:%540=135%:%
%:%541=136%:%
%:%542=136%:%
%:%543=137%:%
%:%544=137%:%
%:%545=137%:%
%:%546=137%:%
%:%547=138%:%
%:%548=138%:%
%:%549=139%:%
%:%550=139%:%
%:%551=140%:%
%:%552=140%:%
%:%553=140%:%
%:%554=141%:%
%:%555=141%:%
%:%556=141%:%
%:%557=142%:%
%:%558=142%:%
%:%559=142%:%
%:%560=142%:%
%:%561=143%:%
%:%562=143%:%
%:%563=144%:%
%:%564=144%:%
%:%565=144%:%
%:%566=144%:%
%:%567=144%:%
%:%568=145%:%
%:%569=145%:%
%:%570=145%:%
%:%571=145%:%
%:%572=146%:%
%:%573=146%:%
%:%574=146%:%
%:%575=147%:%
%:%576=147%:%
%:%577=147%:%
%:%578=148%:%
%:%579=148%:%
%:%580=148%:%
%:%581=148%:%
%:%582=149%:%
%:%583=149%:%
%:%584=149%:%
%:%585=150%:%
%:%586=150%:%
%:%587=151%:%
%:%588=151%:%
%:%589=151%:%
%:%590=152%:%
%:%591=152%:%
%:%592=153%:%
%:%593=153%:%
%:%594=154%:%
%:%595=154%:%
%:%596=154%:%
%:%597=155%:%
%:%598=155%:%
%:%599=156%:%
%:%600=156%:%
%:%601=156%:%
%:%602=157%:%
%:%603=157%:%
%:%604=157%:%
%:%605=158%:%
%:%606=158%:%
%:%607=158%:%
%:%608=158%:%
%:%609=159%:%
%:%610=159%:%
%:%611=160%:%
%:%612=160%:%
%:%613=161%:%
%:%614=161%:%
%:%615=161%:%
%:%616=161%:%
%:%617=162%:%
%:%618=162%:%
%:%619=162%:%
%:%620=163%:%
%:%621=163%:%
%:%622=163%:%
%:%623=164%:%
%:%624=164%:%
%:%625=164%:%
%:%626=165%:%
%:%627=165%:%
%:%628=166%:%
%:%629=166%:%
%:%630=167%:%
%:%631=167%:%
%:%632=167%:%
%:%633=167%:%
%:%634=168%:%
%:%635=168%:%
%:%636=169%:%
%:%637=169%:%
%:%638=169%:%
%:%639=169%:%
%:%640=169%:%
%:%641=170%:%
%:%647=170%:%
%:%650=171%:%
%:%651=172%:%
%:%652=172%:%
%:%653=173%:%
%:%654=174%:%
%:%661=175%:%
%:%662=175%:%
%:%663=176%:%
%:%664=176%:%
%:%665=176%:%
%:%666=177%:%
%:%667=177%:%
%:%668=177%:%
%:%669=177%:%
%:%670=178%:%
%:%671=178%:%
%:%672=178%:%
%:%673=178%:%
%:%674=179%:%
%:%675=179%:%
%:%676=179%:%
%:%677=180%:%
%:%678=180%:%
%:%679=180%:%
%:%680=181%:%
%:%681=181%:%
%:%682=181%:%
%:%683=181%:%
%:%684=181%:%
%:%685=182%:%
%:%686=182%:%
%:%687=182%:%
%:%688=182%:%
%:%689=183%:%
%:%690=183%:%
%:%691=183%:%
%:%692=184%:%
%:%693=184%:%
%:%694=184%:%
%:%695=185%:%
%:%696=185%:%
%:%697=185%:%
%:%698=185%:%
%:%699=186%:%
%:%700=186%:%
%:%701=186%:%
%:%702=186%:%
%:%703=187%:%
%:%704=187%:%
%:%705=187%:%
%:%706=187%:%
%:%707=188%:%
%:%708=188%:%
%:%709=188%:%
%:%710=188%:%
%:%711=188%:%
%:%712=189%:%
%:%718=189%:%
%:%721=190%:%
%:%722=191%:%
%:%723=191%:%
%:%724=192%:%
%:%725=193%:%
%:%732=194%:%
%:%733=194%:%
%:%734=195%:%
%:%735=195%:%
%:%736=195%:%
%:%737=196%:%
%:%738=196%:%
%:%739=196%:%
%:%740=197%:%
%:%741=197%:%
%:%742=197%:%
%:%743=197%:%
%:%744=198%:%
%:%745=198%:%
%:%746=198%:%
%:%747=198%:%
%:%748=198%:%
%:%749=199%:%
%:%750=199%:%
%:%751=199%:%
%:%752=199%:%
%:%753=199%:%
%:%754=200%:%
%:%755=200%:%
%:%756=200%:%
%:%757=200%:%
%:%758=200%:%
%:%759=201%:%
%:%760=201%:%
%:%761=201%:%
%:%762=201%:%
%:%763=201%:%
%:%764=202%:%
%:%765=202%:%
%:%766=202%:%
%:%767=202%:%
%:%768=202%:%
%:%769=203%:%
%:%775=203%:%
%:%778=204%:%
%:%779=205%:%
%:%780=205%:%
%:%787=206%:%
%:%788=206%:%
%:%789=207%:%
%:%790=207%:%
%:%791=207%:%
%:%792=207%:%
%:%793=208%:%
%:%794=208%:%
%:%795=208%:%
%:%796=208%:%
%:%797=209%:%
%:%812=213%:%
%:%824=215%:%
%:%828=216%:%
%:%829=217%:%
%:%831=220%:%
%:%832=220%:%
%:%839=221%:%
%:%840=221%:%
%:%841=222%:%
%:%842=222%:%
%:%843=223%:%
%:%844=223%:%
%:%845=223%:%
%:%846=224%:%
%:%847=224%:%
%:%848=224%:%
%:%849=225%:%
%:%850=225%:%
%:%851=225%:%
%:%852=226%:%
%:%853=226%:%
%:%854=227%:%
%:%855=227%:%
%:%856=227%:%
%:%857=227%:%
%:%858=228%:%
%:%859=228%:%
%:%864=228%:%
%:%867=229%:%
%:%868=230%:%
%:%869=230%:%
%:%876=231%:%
%:%877=231%:%
%:%878=232%:%
%:%879=232%:%
%:%880=233%:%
%:%881=233%:%
%:%882=234%:%
%:%883=234%:%
%:%884=234%:%
%:%885=234%:%
%:%886=234%:%
%:%887=235%:%
%:%888=235%:%
%:%889=235%:%
%:%890=235%:%
%:%891=236%:%
%:%892=236%:%
%:%893=236%:%
%:%894=236%:%
%:%895=237%:%
%:%896=237%:%
%:%897=237%:%
%:%898=238%:%
%:%899=238%:%
%:%900=238%:%
%:%901=239%:%
%:%902=239%:%
%:%903=240%:%
%:%904=240%:%
%:%905=240%:%
%:%906=241%:%
%:%907=241%:%
%:%908=241%:%
%:%909=242%:%
%:%910=242%:%
%:%911=243%:%
%:%917=243%:%
%:%920=244%:%
%:%921=245%:%
%:%922=245%:%
%:%929=246%:%
%:%930=246%:%
%:%931=247%:%
%:%932=247%:%
%:%933=248%:%
%:%934=248%:%
%:%935=249%:%
%:%936=249%:%
%:%937=249%:%
%:%938=250%:%
%:%939=250%:%
%:%940=251%:%
%:%941=251%:%
%:%942=251%:%
%:%943=252%:%
%:%944=252%:%
%:%945=252%:%
%:%946=253%:%
%:%947=253%:%
%:%948=254%:%
%:%949=254%:%
%:%950=254%:%
%:%951=255%:%
%:%952=255%:%
%:%953=255%:%
%:%954=256%:%
%:%955=256%:%
%:%956=257%:%
%:%962=257%:%
%:%965=258%:%
%:%966=259%:%
%:%967=259%:%
%:%968=260%:%
%:%975=261%:%
%:%976=261%:%
%:%977=262%:%
%:%978=262%:%
%:%979=263%:%
%:%980=263%:%
%:%981=264%:%
%:%982=264%:%
%:%983=264%:%
%:%984=265%:%
%:%985=265%:%
%:%986=265%:%
%:%987=266%:%
%:%988=266%:%
%:%989=266%:%
%:%990=266%:%
%:%991=267%:%
%:%992=267%:%
%:%993=268%:%
%:%994=268%:%
%:%995=269%:%
%:%996=269%:%
%:%997=270%:%
%:%998=270%:%
%:%999=270%:%
%:%1000=270%:%
%:%1001=270%:%
%:%1002=271%:%
%:%1003=271%:%
%:%1004=271%:%
%:%1005=271%:%
%:%1006=271%:%
%:%1007=272%:%
%:%1008=272%:%
%:%1009=273%:%
%:%1010=273%:%
%:%1011=274%:%
%:%1012=274%:%
%:%1013=274%:%
%:%1014=274%:%
%:%1015=275%:%
%:%1016=275%:%
%:%1017=275%:%
%:%1018=275%:%
%:%1019=276%:%
%:%1020=276%:%
%:%1021=277%:%
%:%1027=277%:%
%:%1030=278%:%
%:%1031=279%:%
%:%1032=279%:%
%:%1039=280%:%
%:%1040=280%:%
%:%1041=281%:%
%:%1042=281%:%
%:%1043=281%:%
%:%1044=281%:%
%:%1045=282%:%
%:%1046=282%:%
%:%1047=283%:%
%:%1048=283%:%
%:%1049=283%:%
%:%1050=284%:%
%:%1051=284%:%
%:%1052=284%:%
%:%1053=285%:%
%:%1054=285%:%
%:%1055=285%:%
%:%1056=286%:%
%:%1057=286%:%
%:%1058=287%:%
%:%1059=287%:%
%:%1060=287%:%
%:%1061=287%:%
%:%1062=287%:%
%:%1063=288%:%
%:%1069=288%:%
%:%1072=289%:%
%:%1073=290%:%
%:%1074=290%:%
%:%1081=291%:%
%:%1082=291%:%
%:%1083=292%:%
%:%1084=292%:%
%:%1085=293%:%
%:%1086=293%:%
%:%1087=293%:%
%:%1088=294%:%
%:%1089=294%:%
%:%1090=295%:%
%:%1091=295%:%
%:%1092=296%:%
%:%1093=296%:%
%:%1094=297%:%
%:%1109=301%:%
%:%1121=303%:%
%:%1123=305%:%
%:%1124=305%:%
%:%1125=306%:%
%:%1126=307%:%
%:%1133=308%:%
%:%1134=308%:%
%:%1135=309%:%
%:%1136=309%:%
%:%1137=310%:%
%:%1138=310%:%
%:%1139=311%:%
%:%1140=311%:%
%:%1141=312%:%
%:%1142=312%:%
%:%1143=313%:%
%:%1144=313%:%
%:%1145=313%:%
%:%1146=313%:%
%:%1147=314%:%
%:%1148=314%:%
%:%1149=315%:%
%:%1150=315%:%
%:%1151=316%:%
%:%1152=316%:%
%:%1153=316%:%
%:%1154=317%:%
%:%1155=317%:%
%:%1156=317%:%
%:%1157=317%:%
%:%1158=318%:%
%:%1159=318%:%
%:%1160=319%:%
%:%1161=319%:%
%:%1162=320%:%
%:%1163=320%:%
%:%1164=321%:%
%:%1165=321%:%
%:%1166=321%:%
%:%1167=321%:%
%:%1168=322%:%
%:%1169=322%:%
%:%1170=322%:%
%:%1171=322%:%
%:%1172=323%:%
%:%1173=323%:%
%:%1174=324%:%
%:%1175=324%:%
%:%1176=325%:%
%:%1177=325%:%
%:%1178=325%:%
%:%1179=326%:%
%:%1180=327%:%
%:%1181=327%:%
%:%1182=328%:%
%:%1183=328%:%
%:%1184=328%:%
%:%1185=328%:%
%:%1186=329%:%
%:%1187=329%:%
%:%1188=330%:%
%:%1189=330%:%
%:%1190=331%:%
%:%1191=331%:%
%:%1192=331%:%
%:%1193=332%:%
%:%1194=333%:%
%:%1195=333%:%
%:%1196=334%:%
%:%1197=334%:%
%:%1198=334%:%
%:%1199=335%:%
%:%1200=335%:%
%:%1201=336%:%
%:%1202=336%:%
%:%1203=336%:%
%:%1204=337%:%
%:%1205=337%:%
%:%1206=338%:%
%:%1207=338%:%
%:%1208=338%:%
%:%1209=338%:%
%:%1210=339%:%
%:%1211=339%:%
%:%1212=340%:%
%:%1213=340%:%
%:%1214=341%:%
%:%1215=341%:%
%:%1216=341%:%
%:%1217=342%:%
%:%1218=342%:%
%:%1219=342%:%
%:%1220=343%:%
%:%1221=343%:%
%:%1222=343%:%
%:%1223=343%:%
%:%1224=344%:%
%:%1225=344%:%
%:%1226=345%:%
%:%1227=345%:%
%:%1228=345%:%
%:%1229=345%:%
%:%1230=346%:%
%:%1240=348%:%
%:%1241=349%:%
%:%1242=350%:%
%:%1244=351%:%
%:%1245=351%:%
%:%1246=352%:%
%:%1247=353%:%
%:%1254=354%:%
%:%1255=354%:%
%:%1256=355%:%
%:%1257=355%:%
%:%1258=356%:%
%:%1259=356%:%
%:%1260=356%:%
%:%1261=356%:%
%:%1262=357%:%
%:%1263=357%:%
%:%1264=357%:%
%:%1265=358%:%
%:%1266=358%:%
%:%1267=358%:%
%:%1268=358%:%
%:%1269=359%:%
%:%1270=359%:%
%:%1271=359%:%
%:%1272=359%:%
%:%1273=359%:%
%:%1274=360%:%
%:%1275=360%:%
%:%1276=360%:%
%:%1277=360%:%
%:%1278=361%:%
%:%1293=364%:%
%:%1305=366%:%
%:%1307=368%:%
%:%1308=368%:%
%:%1309=369%:%
%:%1310=370%:%
%:%1317=371%:%
%:%1318=371%:%
%:%1319=372%:%
%:%1320=372%:%
%:%1321=373%:%
%:%1322=373%:%
%:%1323=373%:%
%:%1324=373%:%
%:%1325=374%:%
%:%1326=374%:%
%:%1327=375%:%
%:%1328=375%:%
%:%1329=376%:%
%:%1330=376%:%
%:%1331=376%:%
%:%1332=377%:%
%:%1333=377%:%
%:%1334=378%:%
%:%1335=378%:%
%:%1336=378%:%
%:%1337=378%:%
%:%1338=379%:%
%:%1339=379%:%
%:%1340=380%:%
%:%1341=380%:%
%:%1342=380%:%
%:%1343=380%:%
%:%1344=380%:%
%:%1345=381%:%
%:%1346=381%:%
%:%1347=381%:%
%:%1348=381%:%
%:%1349=382%:%
%:%1355=382%:%
%:%1358=383%:%
%:%1359=384%:%
%:%1360=384%:%
%:%1361=385%:%
%:%1362=386%:%
%:%1369=387%:%
%:%1370=387%:%
%:%1371=388%:%
%:%1372=388%:%
%:%1373=389%:%
%:%1374=389%:%
%:%1375=390%:%
%:%1376=390%:%
%:%1377=390%:%
%:%1378=391%:%
%:%1379=391%:%
%:%1380=391%:%
%:%1381=392%:%
%:%1382=392%:%
%:%1383=392%:%
%:%1384=392%:%
%:%1385=393%:%
%:%1386=393%:%
%:%1387=393%:%
%:%1388=393%:%
%:%1389=393%:%
%:%1390=394%:%
%:%1405=397%:%
%:%1417=399%:%
%:%1421=400%:%
%:%1422=401%:%
%:%1424=403%:%
%:%1425=403%:%
%:%1426=404%:%
%:%1427=405%:%
%:%1428=406%:%
%:%1429=406%:%
%:%1436=407%:%
%:%1437=407%:%
%:%1438=408%:%
%:%1439=408%:%
%:%1440=409%:%
%:%1441=409%:%
%:%1442=410%:%
%:%1443=410%:%
%:%1444=410%:%
%:%1445=411%:%
%:%1446=411%:%
%:%1447=412%:%
%:%1448=412%:%
%:%1449=413%:%
%:%1450=413%:%
%:%1451=414%:%
%:%1452=414%:%
%:%1453=415%:%
%:%1454=415%:%
%:%1455=415%:%
%:%1456=416%:%
%:%1457=416%:%
%:%1458=416%:%
%:%1459=417%:%
%:%1460=417%:%
%:%1461=417%:%
%:%1462=418%:%
%:%1463=418%:%
%:%1464=418%:%
%:%1465=419%:%
%:%1466=419%:%
%:%1467=419%:%
%:%1468=419%:%
%:%1469=419%:%
%:%1470=420%:%
%:%1471=420%:%
%:%1472=420%:%
%:%1473=420%:%
%:%1474=420%:%
%:%1475=421%:%
%:%1476=421%:%
%:%1477=422%:%
%:%1483=422%:%
%:%1486=423%:%
%:%1487=424%:%
%:%1488=424%:%
%:%1489=425%:%
%:%1490=426%:%
%:%1497=427%:%
%:%1498=427%:%
%:%1499=428%:%
%:%1500=428%:%
%:%1501=429%:%
%:%1502=429%:%
%:%1503=430%:%
%:%1504=430%:%
%:%1505=430%:%
%:%1506=431%:%
%:%1507=431%:%
%:%1508=431%:%
%:%1509=432%:%
%:%1510=432%:%
%:%1511=432%:%
%:%1512=432%:%
%:%1513=432%:%
%:%1514=433%:%
%:%1515=433%:%
%:%1516=433%:%
%:%1517=434%:%
%:%1518=434%:%
%:%1519=434%:%
%:%1520=435%:%
%:%1521=435%:%
%:%1522=435%:%
%:%1523=435%:%
%:%1524=436%:%
%:%1525=436%:%
%:%1526=436%:%
%:%1527=436%:%
%:%1528=436%:%
%:%1529=437%:%
%:%1530=437%:%
%:%1531=437%:%
%:%1532=437%:%
%:%1533=438%:%
%:%1534=438%:%
%:%1535=438%:%
%:%1536=438%:%
%:%1537=438%:%
%:%1538=439%:%
%:%1544=439%:%
%:%1549=440%:%
%:%1554=441%:%

%
\begin{isabellebody}%
\setisabellecontext{Eq{\isacharunderscore}{\kern0pt}Sys}%
%
\isadelimdocument
%
\endisadelimdocument
%
\isatagdocument
%
\isamarkupsection{Context free grammars and systems of equations%
}
\isamarkuptrue%
%
\endisatagdocument
{\isafolddocument}%
%
\isadelimdocument
%
\endisadelimdocument
%
\isadelimtheory
%
\endisadelimtheory
%
\isatagtheory
\isakeywordONE{theory}\isamarkupfalse%
\ Eq{\isacharunderscore}{\kern0pt}Sys\isanewline
\ \ \isakeywordTWO{imports}\isanewline
\ \ \ \ {\isachardoublequoteopen}Parikh{\isacharunderscore}{\kern0pt}Img{\isachardoublequoteclose}\isanewline
\ \ \ \ {\isachardoublequoteopen}Context{\isacharunderscore}{\kern0pt}Free{\isacharunderscore}{\kern0pt}Grammar{\isachardot}{\kern0pt}Context{\isacharunderscore}{\kern0pt}Free{\isacharunderscore}{\kern0pt}Language{\isachardoublequoteclose}\isanewline
\isakeywordTWO{begin}%
\endisatagtheory
{\isafoldtheory}%
%
\isadelimtheory
%
\endisadelimtheory
%
\begin{isamarkuptext}%
In this section, we will first introduce two types of systems of
equations. Then we will show that to each CFG correspond two systems of equations - one for both
of the types - and that the language defined by the CFG is a minimal solution of both systems.%
\end{isamarkuptext}\isamarkuptrue%
%
\isadelimdocument
%
\endisadelimdocument
%
\isatagdocument
%
\isamarkupsubsection{Introduction of systems of equations%
}
\isamarkuptrue%
%
\endisatagdocument
{\isafolddocument}%
%
\isadelimdocument
%
\endisadelimdocument
%
\begin{isamarkuptext}%
For the first type of systems, each equation is of the form
$$X_i \supseteq r_i$$
For the second type of systems, each equation is of the form
$$\Psi \; X_i \supseteq \Psi \; r_i$$
i.e. the Parikh image is applied on both sides of each equation.
In both cases, we represent the whole system by a list of regular language expression where each
of the variables \isa{X\isactrlsub {\isadigit{0}}{\isacharcomma}{\kern0pt}\ X\isactrlsub {\isadigit{1}}{\isacharcomma}{\kern0pt}\ {\isasymdots}} is identified by its integer, i.e. \isa{\isaconst{Var}\ \isafree{i}} denotes the variable
\isa{X\isactrlsub i}. The \isa{i}-th item of the list then represents the right-hand side \isa{r\isactrlsub i} of the \isa{i}-th equation:%
\end{isamarkuptext}\isamarkuptrue%
\isakeywordONE{type{\isacharunderscore}{\kern0pt}synonym}\isamarkupfalse%
\ {\isacharprime}{\kern0pt}a\ eq{\isacharunderscore}{\kern0pt}sys\ {\isacharequal}{\kern0pt}\ {\isachardoublequoteopen}{\isacharprime}{\kern0pt}a\ rlexp\ list{\isachardoublequoteclose}%
\begin{isamarkuptext}%
Now we can define what it means for a valuation \isa{v} to solve a system of equations of the
first type, i.e. a system without Parikh images. Afterwards we characterize minimal solutions of
such a system.%
\end{isamarkuptext}\isamarkuptrue%
\isakeywordONE{definition}\isamarkupfalse%
\ solves{\isacharunderscore}{\kern0pt}ineq{\isacharunderscore}{\kern0pt}sys\ {\isacharcolon}{\kern0pt}{\isacharcolon}{\kern0pt}\ {\isachardoublequoteopen}{\isacharprime}{\kern0pt}a\ eq{\isacharunderscore}{\kern0pt}sys\ {\isasymRightarrow}\ {\isacharprime}{\kern0pt}a\ valuation\ {\isasymRightarrow}\ bool{\isachardoublequoteclose}\ \isakeywordTWO{where}\isanewline
\ \ {\isachardoublequoteopen}solves{\isacharunderscore}{\kern0pt}ineq{\isacharunderscore}{\kern0pt}sys\ sys\ v\ {\isasymequiv}\ {\isasymforall}i\ {\isacharless}{\kern0pt}\ length\ sys{\isachardot}{\kern0pt}\ eval\ {\isacharparenleft}{\kern0pt}sys\ {\isacharbang}{\kern0pt}\ i{\isacharparenright}{\kern0pt}\ v\ {\isasymsubseteq}\ v\ i{\isachardoublequoteclose}\isanewline
\isanewline
\isakeywordONE{definition}\isamarkupfalse%
\ min{\isacharunderscore}{\kern0pt}sol{\isacharunderscore}{\kern0pt}ineq{\isacharunderscore}{\kern0pt}sys\ {\isacharcolon}{\kern0pt}{\isacharcolon}{\kern0pt}\ {\isachardoublequoteopen}{\isacharprime}{\kern0pt}a\ eq{\isacharunderscore}{\kern0pt}sys\ {\isasymRightarrow}\ {\isacharprime}{\kern0pt}a\ valuation\ {\isasymRightarrow}\ bool{\isachardoublequoteclose}\ \isakeywordTWO{where}\isanewline
\ \ {\isachardoublequoteopen}min{\isacharunderscore}{\kern0pt}sol{\isacharunderscore}{\kern0pt}ineq{\isacharunderscore}{\kern0pt}sys\ sys\ sol\ {\isasymequiv}\isanewline
\ \ \ \ solves{\isacharunderscore}{\kern0pt}ineq{\isacharunderscore}{\kern0pt}sys\ sys\ sol\ {\isasymand}\ {\isacharparenleft}{\kern0pt}{\isasymforall}sol{\isacharprime}{\kern0pt}{\isachardot}{\kern0pt}\ solves{\isacharunderscore}{\kern0pt}ineq{\isacharunderscore}{\kern0pt}sys\ sys\ sol{\isacharprime}{\kern0pt}\ {\isasymlongrightarrow}\ {\isacharparenleft}{\kern0pt}{\isasymforall}x{\isachardot}{\kern0pt}\ sol\ x\ {\isasymsubseteq}\ sol{\isacharprime}{\kern0pt}\ x{\isacharparenright}{\kern0pt}{\isacharparenright}{\kern0pt}{\isachardoublequoteclose}%
\begin{isamarkuptext}%
The previous definitions can easily be extended to the second type of systems of equations
where the Parikh image is applied on both sides of each equation:%
\end{isamarkuptext}\isamarkuptrue%
\isakeywordONE{definition}\isamarkupfalse%
\ solves{\isacharunderscore}{\kern0pt}ineq{\isacharunderscore}{\kern0pt}comm\ {\isacharcolon}{\kern0pt}{\isacharcolon}{\kern0pt}\ {\isachardoublequoteopen}nat\ {\isasymRightarrow}\ {\isacharprime}{\kern0pt}a\ rlexp\ {\isasymRightarrow}\ {\isacharprime}{\kern0pt}a\ valuation\ {\isasymRightarrow}\ bool{\isachardoublequoteclose}\ \isakeywordTWO{where}\isanewline
\ \ {\isachardoublequoteopen}solves{\isacharunderscore}{\kern0pt}ineq{\isacharunderscore}{\kern0pt}comm\ x\ eq\ v\ {\isasymequiv}\ {\isasymPsi}\ {\isacharparenleft}{\kern0pt}eval\ eq\ v{\isacharparenright}{\kern0pt}\ {\isasymsubseteq}\ {\isasymPsi}\ {\isacharparenleft}{\kern0pt}v\ x{\isacharparenright}{\kern0pt}{\isachardoublequoteclose}\isanewline
\isanewline
\isakeywordONE{definition}\isamarkupfalse%
\ solves{\isacharunderscore}{\kern0pt}ineq{\isacharunderscore}{\kern0pt}sys{\isacharunderscore}{\kern0pt}comm\ {\isacharcolon}{\kern0pt}{\isacharcolon}{\kern0pt}\ {\isachardoublequoteopen}{\isacharprime}{\kern0pt}a\ eq{\isacharunderscore}{\kern0pt}sys\ {\isasymRightarrow}\ {\isacharprime}{\kern0pt}a\ valuation\ {\isasymRightarrow}\ bool{\isachardoublequoteclose}\ \isakeywordTWO{where}\isanewline
\ \ {\isachardoublequoteopen}solves{\isacharunderscore}{\kern0pt}ineq{\isacharunderscore}{\kern0pt}sys{\isacharunderscore}{\kern0pt}comm\ sys\ v\ {\isasymequiv}\ {\isasymforall}i\ {\isacharless}{\kern0pt}\ length\ sys{\isachardot}{\kern0pt}\ solves{\isacharunderscore}{\kern0pt}ineq{\isacharunderscore}{\kern0pt}comm\ i\ {\isacharparenleft}{\kern0pt}sys\ {\isacharbang}{\kern0pt}\ i{\isacharparenright}{\kern0pt}\ v{\isachardoublequoteclose}\isanewline
\isanewline
\isakeywordONE{definition}\isamarkupfalse%
\ min{\isacharunderscore}{\kern0pt}sol{\isacharunderscore}{\kern0pt}ineq{\isacharunderscore}{\kern0pt}sys{\isacharunderscore}{\kern0pt}comm\ {\isacharcolon}{\kern0pt}{\isacharcolon}{\kern0pt}\ {\isachardoublequoteopen}{\isacharprime}{\kern0pt}a\ eq{\isacharunderscore}{\kern0pt}sys\ {\isasymRightarrow}\ {\isacharprime}{\kern0pt}a\ valuation\ {\isasymRightarrow}\ bool{\isachardoublequoteclose}\ \isakeywordTWO{where}\isanewline
\ \ {\isachardoublequoteopen}min{\isacharunderscore}{\kern0pt}sol{\isacharunderscore}{\kern0pt}ineq{\isacharunderscore}{\kern0pt}sys{\isacharunderscore}{\kern0pt}comm\ sys\ sol\ {\isasymequiv}\isanewline
\ \ \ \ solves{\isacharunderscore}{\kern0pt}ineq{\isacharunderscore}{\kern0pt}sys{\isacharunderscore}{\kern0pt}comm\ sys\ sol\ {\isasymand}\isanewline
\ \ \ \ {\isacharparenleft}{\kern0pt}{\isasymforall}sol{\isacharprime}{\kern0pt}{\isachardot}{\kern0pt}\ solves{\isacharunderscore}{\kern0pt}ineq{\isacharunderscore}{\kern0pt}sys{\isacharunderscore}{\kern0pt}comm\ sys\ sol{\isacharprime}{\kern0pt}\ {\isasymlongrightarrow}\ {\isacharparenleft}{\kern0pt}{\isasymforall}x{\isachardot}{\kern0pt}\ {\isasymPsi}\ {\isacharparenleft}{\kern0pt}sol\ x{\isacharparenright}{\kern0pt}\ {\isasymsubseteq}\ {\isasymPsi}\ {\isacharparenleft}{\kern0pt}sol{\isacharprime}{\kern0pt}\ x{\isacharparenright}{\kern0pt}{\isacharparenright}{\kern0pt}{\isacharparenright}{\kern0pt}{\isachardoublequoteclose}%
\begin{isamarkuptext}%
Substitution into each equation of a system:%
\end{isamarkuptext}\isamarkuptrue%
\isakeywordONE{definition}\isamarkupfalse%
\ subst{\isacharunderscore}{\kern0pt}sys\ {\isacharcolon}{\kern0pt}{\isacharcolon}{\kern0pt}\ {\isachardoublequoteopen}{\isacharparenleft}{\kern0pt}nat\ {\isasymRightarrow}\ {\isacharprime}{\kern0pt}a\ rlexp{\isacharparenright}{\kern0pt}\ {\isasymRightarrow}\ {\isacharprime}{\kern0pt}a\ eq{\isacharunderscore}{\kern0pt}sys\ {\isasymRightarrow}\ {\isacharprime}{\kern0pt}a\ eq{\isacharunderscore}{\kern0pt}sys{\isachardoublequoteclose}\ \isakeywordTWO{where}\isanewline
\ \ {\isachardoublequoteopen}subst{\isacharunderscore}{\kern0pt}sys\ {\isasymequiv}\ map\ {\isasymcirc}\ subst{\isachardoublequoteclose}\isanewline
\isanewline
\isakeywordONE{lemma}\isamarkupfalse%
\ subst{\isacharunderscore}{\kern0pt}sys{\isacharunderscore}{\kern0pt}subst{\isacharcolon}{\kern0pt}\isanewline
\ \ \isakeywordTWO{assumes}\ {\isachardoublequoteopen}i\ {\isacharless}{\kern0pt}\ length\ sys{\isachardoublequoteclose}\isanewline
\ \ \isakeywordTWO{shows}\ {\isachardoublequoteopen}{\isacharparenleft}{\kern0pt}subst{\isacharunderscore}{\kern0pt}sys\ s\ sys{\isacharparenright}{\kern0pt}\ {\isacharbang}{\kern0pt}\ i\ {\isacharequal}{\kern0pt}\ subst\ s\ {\isacharparenleft}{\kern0pt}sys\ {\isacharbang}{\kern0pt}\ i{\isacharparenright}{\kern0pt}{\isachardoublequoteclose}\isanewline
%
\isadelimproof
\ \ %
\endisadelimproof
%
\isatagproof
\isakeywordONE{unfolding}\isamarkupfalse%
\ subst{\isacharunderscore}{\kern0pt}sys{\isacharunderscore}{\kern0pt}def\ \isakeywordONE{by}\isamarkupfalse%
\ {\isacharparenleft}{\kern0pt}simp\ add{\isacharcolon}{\kern0pt}\ assms{\isacharparenright}{\kern0pt}%
\endisatagproof
{\isafoldproof}%
%
\isadelimproof
%
\endisadelimproof
%
\isadelimdocument
%
\endisadelimdocument
%
\isatagdocument
%
\isamarkupsubsection{Partial solutions of systems of equations%
}
\isamarkuptrue%
%
\endisatagdocument
{\isafolddocument}%
%
\isadelimdocument
%
\endisadelimdocument
%
\begin{isamarkuptext}%
We introduce partial solutions, i.e. solutions which might depend on one or multiple
variables. They are therefore not represented as languages, but as regular language expressions.
\isa{sol} is a partial solution of the \isa{x}-th equation if and only if it solves the equation
independently on the values of the other variables:%
\end{isamarkuptext}\isamarkuptrue%
\isakeywordONE{definition}\isamarkupfalse%
\ partial{\isacharunderscore}{\kern0pt}sol{\isacharunderscore}{\kern0pt}ineq\ {\isacharcolon}{\kern0pt}{\isacharcolon}{\kern0pt}\ {\isachardoublequoteopen}nat\ {\isasymRightarrow}\ {\isacharprime}{\kern0pt}a\ rlexp\ {\isasymRightarrow}\ {\isacharprime}{\kern0pt}a\ rlexp\ {\isasymRightarrow}\ bool{\isachardoublequoteclose}\ \isakeywordTWO{where}\isanewline
\ \ {\isachardoublequoteopen}partial{\isacharunderscore}{\kern0pt}sol{\isacharunderscore}{\kern0pt}ineq\ x\ eq\ sol\ {\isasymequiv}\ {\isasymforall}v{\isachardot}{\kern0pt}\ v\ x\ {\isacharequal}{\kern0pt}\ eval\ sol\ v\ {\isasymlongrightarrow}\ solves{\isacharunderscore}{\kern0pt}ineq{\isacharunderscore}{\kern0pt}comm\ x\ eq\ v{\isachardoublequoteclose}%
\begin{isamarkuptext}%
We generalize the previous definition to partial solutions of whole systems of equations:
\isa{sols} maps each variable \isa{i} to a regular language expression representing the partial solution
of the \isa{i}-th equation. A partial solution of the whole system is then defined as follows:%
\end{isamarkuptext}\isamarkuptrue%
\isakeywordONE{definition}\isamarkupfalse%
\ solution{\isacharunderscore}{\kern0pt}ineq{\isacharunderscore}{\kern0pt}sys\ {\isacharcolon}{\kern0pt}{\isacharcolon}{\kern0pt}\ {\isachardoublequoteopen}{\isacharprime}{\kern0pt}a\ eq{\isacharunderscore}{\kern0pt}sys\ {\isasymRightarrow}\ {\isacharparenleft}{\kern0pt}nat\ {\isasymRightarrow}\ {\isacharprime}{\kern0pt}a\ rlexp{\isacharparenright}{\kern0pt}\ {\isasymRightarrow}\ bool{\isachardoublequoteclose}\ \isakeywordTWO{where}\isanewline
\ \ {\isachardoublequoteopen}solution{\isacharunderscore}{\kern0pt}ineq{\isacharunderscore}{\kern0pt}sys\ sys\ sols\ {\isasymequiv}\ {\isasymforall}v{\isachardot}{\kern0pt}\ {\isacharparenleft}{\kern0pt}{\isasymforall}x{\isachardot}{\kern0pt}\ v\ x\ {\isacharequal}{\kern0pt}\ eval\ {\isacharparenleft}{\kern0pt}sols\ x{\isacharparenright}{\kern0pt}\ v{\isacharparenright}{\kern0pt}\ {\isasymlongrightarrow}\ solves{\isacharunderscore}{\kern0pt}ineq{\isacharunderscore}{\kern0pt}sys{\isacharunderscore}{\kern0pt}comm\ sys\ v{\isachardoublequoteclose}%
\begin{isamarkuptext}%
Given the \isa{x}-th equation \isa{eq}, \isa{sol} is a minimal partial solution of this equation if and
only if
\begin{enumerate}
\item \textit{sol} is a partial solution of \textit{eq}
\item \textit{sol} is a proper partial solution (i.e. it does not depend on \textit{x}) and only
  depends on variables occurring in the equation \textit{eq}
\item no partial solution of the equation \textit{eq} is smaller than \textit{sol}
\end{enumerate}%
\end{isamarkuptext}\isamarkuptrue%
\isakeywordONE{definition}\isamarkupfalse%
\ partial{\isacharunderscore}{\kern0pt}min{\isacharunderscore}{\kern0pt}sol{\isacharunderscore}{\kern0pt}one{\isacharunderscore}{\kern0pt}ineq\ {\isacharcolon}{\kern0pt}{\isacharcolon}{\kern0pt}\ {\isachardoublequoteopen}nat\ {\isasymRightarrow}\ {\isacharprime}{\kern0pt}a\ rlexp\ {\isasymRightarrow}\ {\isacharprime}{\kern0pt}a\ rlexp\ {\isasymRightarrow}\ bool{\isachardoublequoteclose}\ \isakeywordTWO{where}\isanewline
\ \ {\isachardoublequoteopen}partial{\isacharunderscore}{\kern0pt}min{\isacharunderscore}{\kern0pt}sol{\isacharunderscore}{\kern0pt}one{\isacharunderscore}{\kern0pt}ineq\ x\ eq\ sol\ {\isasymequiv}\isanewline
\ \ \ \ partial{\isacharunderscore}{\kern0pt}sol{\isacharunderscore}{\kern0pt}ineq\ x\ eq\ sol\ {\isasymand}\isanewline
\ \ \ \ vars\ sol\ {\isasymsubseteq}\ vars\ eq\ {\isacharminus}{\kern0pt}\ {\isacharbraceleft}{\kern0pt}x{\isacharbraceright}{\kern0pt}\ {\isasymand}\isanewline
\ \ \ \ {\isacharparenleft}{\kern0pt}{\isasymforall}sol{\isacharprime}{\kern0pt}\ v{\isacharprime}{\kern0pt}{\isachardot}{\kern0pt}\ solves{\isacharunderscore}{\kern0pt}ineq{\isacharunderscore}{\kern0pt}comm\ x\ eq\ v{\isacharprime}{\kern0pt}\ {\isasymand}\ v{\isacharprime}{\kern0pt}\ x\ {\isacharequal}{\kern0pt}\ eval\ sol{\isacharprime}{\kern0pt}\ v{\isacharprime}{\kern0pt}\isanewline
\ \ \ \ \ \ \ \ \ \ \ \ \ \ \ {\isasymlongrightarrow}\ {\isasymPsi}\ {\isacharparenleft}{\kern0pt}eval\ sol\ v{\isacharprime}{\kern0pt}{\isacharparenright}{\kern0pt}\ {\isasymsubseteq}\ {\isasymPsi}\ {\isacharparenleft}{\kern0pt}v{\isacharprime}{\kern0pt}\ x{\isacharparenright}{\kern0pt}{\isacharparenright}{\kern0pt}{\isachardoublequoteclose}%
\begin{isamarkuptext}%
Given a whole system of equations \isa{sys}, we can generalize the previous definition such that
\isa{sols} is a minimal solution (possibly dependent on the variables $X_n, X_{n+1}, \dots$) of
the first \isa{n} equations. Besides the three conditions described above, we introduce a forth
condition: \isa{sols\ i\ {\isacharequal}{\kern0pt}\ Var\ i} for \isa{i\ {\isasymge}\ n}, i.e. \isa{sols} assigns only spurious solutions to the
equations which are not yet solved:%
\end{isamarkuptext}\isamarkuptrue%
\isakeywordONE{definition}\isamarkupfalse%
\ partial{\isacharunderscore}{\kern0pt}min{\isacharunderscore}{\kern0pt}sol{\isacharunderscore}{\kern0pt}ineq{\isacharunderscore}{\kern0pt}sys\ {\isacharcolon}{\kern0pt}{\isacharcolon}{\kern0pt}\ {\isachardoublequoteopen}nat\ {\isasymRightarrow}\ {\isacharprime}{\kern0pt}a\ eq{\isacharunderscore}{\kern0pt}sys\ {\isasymRightarrow}\ {\isacharparenleft}{\kern0pt}nat\ {\isasymRightarrow}\ {\isacharprime}{\kern0pt}a\ rlexp{\isacharparenright}{\kern0pt}\ {\isasymRightarrow}\ bool{\isachardoublequoteclose}\ \isakeywordTWO{where}\isanewline
\ \ {\isachardoublequoteopen}partial{\isacharunderscore}{\kern0pt}min{\isacharunderscore}{\kern0pt}sol{\isacharunderscore}{\kern0pt}ineq{\isacharunderscore}{\kern0pt}sys\ n\ sys\ sols\ {\isasymequiv}\isanewline
\ \ \ \ solution{\isacharunderscore}{\kern0pt}ineq{\isacharunderscore}{\kern0pt}sys\ {\isacharparenleft}{\kern0pt}take\ n\ sys{\isacharparenright}{\kern0pt}\ sols\ {\isasymand}\isanewline
\ \ \ \ {\isacharparenleft}{\kern0pt}{\isasymforall}i\ {\isasymge}\ n{\isachardot}{\kern0pt}\ sols\ i\ {\isacharequal}{\kern0pt}\ Var\ i{\isacharparenright}{\kern0pt}\ {\isasymand}\isanewline
\ \ \ \ {\isacharparenleft}{\kern0pt}{\isasymforall}i\ {\isacharless}{\kern0pt}\ n{\isachardot}{\kern0pt}\ {\isasymforall}x\ {\isasymin}\ vars\ {\isacharparenleft}{\kern0pt}sols\ i{\isacharparenright}{\kern0pt}{\isachardot}{\kern0pt}\ x\ {\isasymge}\ n\ {\isasymand}\ x\ {\isacharless}{\kern0pt}\ length\ sys{\isacharparenright}{\kern0pt}\ {\isasymand}\isanewline
\ \ \ \ {\isacharparenleft}{\kern0pt}{\isasymforall}sols{\isacharprime}{\kern0pt}\ v{\isacharprime}{\kern0pt}{\isachardot}{\kern0pt}\ {\isacharparenleft}{\kern0pt}{\isasymforall}x{\isachardot}{\kern0pt}\ v{\isacharprime}{\kern0pt}\ x\ {\isacharequal}{\kern0pt}\ eval\ {\isacharparenleft}{\kern0pt}sols{\isacharprime}{\kern0pt}\ x{\isacharparenright}{\kern0pt}\ v{\isacharprime}{\kern0pt}{\isacharparenright}{\kern0pt}\isanewline
\ \ \ \ \ \ \ \ \ \ \ \ \ \ \ \ \ \ {\isasymand}\ solves{\isacharunderscore}{\kern0pt}ineq{\isacharunderscore}{\kern0pt}sys{\isacharunderscore}{\kern0pt}comm\ {\isacharparenleft}{\kern0pt}take\ n\ sys{\isacharparenright}{\kern0pt}\ v{\isacharprime}{\kern0pt}\isanewline
\ \ \ \ \ \ \ \ \ \ \ \ \ \ \ \ \ \ {\isasymlongrightarrow}\ {\isacharparenleft}{\kern0pt}{\isasymforall}i{\isachardot}{\kern0pt}\ {\isasymPsi}\ {\isacharparenleft}{\kern0pt}eval\ {\isacharparenleft}{\kern0pt}sols\ i{\isacharparenright}{\kern0pt}\ v{\isacharprime}{\kern0pt}{\isacharparenright}{\kern0pt}\ {\isasymsubseteq}\ {\isasymPsi}\ {\isacharparenleft}{\kern0pt}v{\isacharprime}{\kern0pt}\ i{\isacharparenright}{\kern0pt}{\isacharparenright}{\kern0pt}{\isacharparenright}{\kern0pt}{\isachardoublequoteclose}%
\begin{isamarkuptext}%
If the Parikh image of two equations \isa{f} and \isa{g} is identical on all valuations, then their
minimal partial solutions are identical, too:%
\end{isamarkuptext}\isamarkuptrue%
\isakeywordONE{lemma}\isamarkupfalse%
\ same{\isacharunderscore}{\kern0pt}min{\isacharunderscore}{\kern0pt}sol{\isacharunderscore}{\kern0pt}if{\isacharunderscore}{\kern0pt}same{\isacharunderscore}{\kern0pt}parikh{\isacharunderscore}{\kern0pt}img{\isacharcolon}{\kern0pt}\isanewline
\ \ \isakeywordTWO{assumes}\ same{\isacharunderscore}{\kern0pt}parikh{\isacharunderscore}{\kern0pt}img{\isacharcolon}{\kern0pt}\ {\isachardoublequoteopen}{\isasymforall}v{\isachardot}{\kern0pt}\ {\isasymPsi}\ {\isacharparenleft}{\kern0pt}eval\ f\ v{\isacharparenright}{\kern0pt}\ {\isacharequal}{\kern0pt}\ {\isasymPsi}\ {\isacharparenleft}{\kern0pt}eval\ g\ v{\isacharparenright}{\kern0pt}{\isachardoublequoteclose}\isanewline
\ \ \ \ \ \ \isakeywordTWO{and}\ same{\isacharunderscore}{\kern0pt}vars{\isacharcolon}{\kern0pt}\ \ \ \ \ \ \ {\isachardoublequoteopen}vars\ f\ {\isacharminus}{\kern0pt}\ {\isacharbraceleft}{\kern0pt}x{\isacharbraceright}{\kern0pt}\ {\isacharequal}{\kern0pt}\ vars\ g\ {\isacharminus}{\kern0pt}\ {\isacharbraceleft}{\kern0pt}x{\isacharbraceright}{\kern0pt}{\isachardoublequoteclose}\isanewline
\ \ \ \ \ \ \isakeywordTWO{and}\ minimal{\isacharunderscore}{\kern0pt}sol{\isacharcolon}{\kern0pt}\ \ \ \ \ {\isachardoublequoteopen}partial{\isacharunderscore}{\kern0pt}min{\isacharunderscore}{\kern0pt}sol{\isacharunderscore}{\kern0pt}one{\isacharunderscore}{\kern0pt}ineq\ x\ f\ sol{\isachardoublequoteclose}\isanewline
\ \ \ \ \isakeywordTWO{shows}\ \ \ \ \ \ \ \ \ \ \ \ \ \ \ \ \ \ {\isachardoublequoteopen}partial{\isacharunderscore}{\kern0pt}min{\isacharunderscore}{\kern0pt}sol{\isacharunderscore}{\kern0pt}one{\isacharunderscore}{\kern0pt}ineq\ x\ g\ sol{\isachardoublequoteclose}\isanewline
%
\isadelimproof
%
\endisadelimproof
%
\isatagproof
\isakeywordONE{proof}\isamarkupfalse%
\ {\isacharminus}{\kern0pt}\isanewline
\ \ \isakeywordONE{from}\isamarkupfalse%
\ minimal{\isacharunderscore}{\kern0pt}sol\ \isakeywordONE{have}\isamarkupfalse%
\ {\isachardoublequoteopen}vars\ sol\ {\isasymsubseteq}\ vars\ g\ {\isacharminus}{\kern0pt}\ {\isacharbraceleft}{\kern0pt}x{\isacharbraceright}{\kern0pt}{\isachardoublequoteclose}\isanewline
\ \ \ \ \isakeywordONE{unfolding}\isamarkupfalse%
\ partial{\isacharunderscore}{\kern0pt}min{\isacharunderscore}{\kern0pt}sol{\isacharunderscore}{\kern0pt}one{\isacharunderscore}{\kern0pt}ineq{\isacharunderscore}{\kern0pt}def\ \isakeywordONE{using}\isamarkupfalse%
\ same{\isacharunderscore}{\kern0pt}vars\ \isakeywordONE{by}\isamarkupfalse%
\ blast\isanewline
\ \ \isakeywordONE{moreover}\isamarkupfalse%
\ \isakeywordONE{from}\isamarkupfalse%
\ same{\isacharunderscore}{\kern0pt}parikh{\isacharunderscore}{\kern0pt}img\ minimal{\isacharunderscore}{\kern0pt}sol\ \isakeywordONE{have}\isamarkupfalse%
\ {\isachardoublequoteopen}partial{\isacharunderscore}{\kern0pt}sol{\isacharunderscore}{\kern0pt}ineq\ x\ g\ sol{\isachardoublequoteclose}\isanewline
\ \ \ \ \isakeywordONE{unfolding}\isamarkupfalse%
\ partial{\isacharunderscore}{\kern0pt}min{\isacharunderscore}{\kern0pt}sol{\isacharunderscore}{\kern0pt}one{\isacharunderscore}{\kern0pt}ineq{\isacharunderscore}{\kern0pt}def\ partial{\isacharunderscore}{\kern0pt}sol{\isacharunderscore}{\kern0pt}ineq{\isacharunderscore}{\kern0pt}def\ solves{\isacharunderscore}{\kern0pt}ineq{\isacharunderscore}{\kern0pt}comm{\isacharunderscore}{\kern0pt}def\ \isakeywordONE{by}\isamarkupfalse%
\ simp\isanewline
\ \ \isakeywordONE{moreover}\isamarkupfalse%
\ \isakeywordONE{from}\isamarkupfalse%
\ same{\isacharunderscore}{\kern0pt}parikh{\isacharunderscore}{\kern0pt}img\ minimal{\isacharunderscore}{\kern0pt}sol\ \isakeywordONE{have}\isamarkupfalse%
\ {\isachardoublequoteopen}{\isasymforall}sol{\isacharprime}{\kern0pt}\ v{\isacharprime}{\kern0pt}{\isachardot}{\kern0pt}\ solves{\isacharunderscore}{\kern0pt}ineq{\isacharunderscore}{\kern0pt}comm\ x\ g\ v{\isacharprime}{\kern0pt}\ {\isasymand}\ v{\isacharprime}{\kern0pt}\ x\ {\isacharequal}{\kern0pt}\ eval\ sol{\isacharprime}{\kern0pt}\ v{\isacharprime}{\kern0pt}\isanewline
\ \ \ \ \ \ \ \ \ \ \ \ \ \ \ {\isasymlongrightarrow}\ {\isasymPsi}\ {\isacharparenleft}{\kern0pt}eval\ sol\ v{\isacharprime}{\kern0pt}{\isacharparenright}{\kern0pt}\ {\isasymsubseteq}\ {\isasymPsi}\ {\isacharparenleft}{\kern0pt}v{\isacharprime}{\kern0pt}\ x{\isacharparenright}{\kern0pt}{\isachardoublequoteclose}\isanewline
\ \ \ \ \isakeywordONE{unfolding}\isamarkupfalse%
\ partial{\isacharunderscore}{\kern0pt}min{\isacharunderscore}{\kern0pt}sol{\isacharunderscore}{\kern0pt}one{\isacharunderscore}{\kern0pt}ineq{\isacharunderscore}{\kern0pt}def\ solves{\isacharunderscore}{\kern0pt}ineq{\isacharunderscore}{\kern0pt}comm{\isacharunderscore}{\kern0pt}def\ \isakeywordONE{by}\isamarkupfalse%
\ blast\isanewline
\ \ \isakeywordONE{ultimately}\isamarkupfalse%
\ \isakeywordTHREE{show}\isamarkupfalse%
\ {\isacharquery}{\kern0pt}thesis\ \isakeywordONE{unfolding}\isamarkupfalse%
\ partial{\isacharunderscore}{\kern0pt}min{\isacharunderscore}{\kern0pt}sol{\isacharunderscore}{\kern0pt}one{\isacharunderscore}{\kern0pt}ineq{\isacharunderscore}{\kern0pt}def\ \isakeywordONE{by}\isamarkupfalse%
\ fast\isanewline
\isakeywordONE{qed}\isamarkupfalse%
%
\endisatagproof
{\isafoldproof}%
%
\isadelimproof
%
\endisadelimproof
%
\isadelimdocument
%
\endisadelimdocument
%
\isatagdocument
%
\isamarkupsubsection{CFLs as minimal solution of systems of equations%
}
\isamarkuptrue%
%
\endisatagdocument
{\isafolddocument}%
%
\isadelimdocument
%
\endisadelimdocument
%
\begin{isamarkuptext}%
We show that each CFG induces a system of equations of the first type, i.e. without Parikh images,
such that the CFG's language is the minimal solution of the system. First, we describe how to derive
the system of equations from a CFG. This requires us to fix some bijection between the variables in
the system and the non-terminals occurring in the CFG:%
\end{isamarkuptext}\isamarkuptrue%
\isakeywordONE{definition}\isamarkupfalse%
\ bij{\isacharunderscore}{\kern0pt}Nt{\isacharunderscore}{\kern0pt}Var\ {\isacharcolon}{\kern0pt}{\isacharcolon}{\kern0pt}\ {\isachardoublequoteopen}{\isacharprime}{\kern0pt}n\ set\ {\isasymRightarrow}\ {\isacharparenleft}{\kern0pt}nat\ {\isasymRightarrow}\ {\isacharprime}{\kern0pt}n{\isacharparenright}{\kern0pt}\ {\isasymRightarrow}\ {\isacharparenleft}{\kern0pt}{\isacharprime}{\kern0pt}n\ {\isasymRightarrow}\ nat{\isacharparenright}{\kern0pt}\ {\isasymRightarrow}\ bool{\isachardoublequoteclose}\ \isakeywordTWO{where}\isanewline
\ \ {\isachardoublequoteopen}bij{\isacharunderscore}{\kern0pt}Nt{\isacharunderscore}{\kern0pt}Var\ A\ {\isasymgamma}\ {\isasymgamma}{\isacharprime}{\kern0pt}\ {\isasymequiv}\ bij{\isacharunderscore}{\kern0pt}betw\ {\isasymgamma}\ {\isacharbraceleft}{\kern0pt}{\isachardot}{\kern0pt}{\isachardot}{\kern0pt}{\isacharless}{\kern0pt}\ card\ A{\isacharbraceright}{\kern0pt}\ A\ {\isasymand}\ bij{\isacharunderscore}{\kern0pt}betw\ {\isasymgamma}{\isacharprime}{\kern0pt}\ A\ {\isacharbraceleft}{\kern0pt}{\isachardot}{\kern0pt}{\isachardot}{\kern0pt}{\isacharless}{\kern0pt}\ card\ A{\isacharbraceright}{\kern0pt}\isanewline
\ \ \ \ \ \ \ \ \ \ \ \ \ \ \ \ \ \ \ \ \ \ \ \ \ \ {\isasymand}\ {\isacharparenleft}{\kern0pt}{\isasymforall}x\ {\isasymin}\ {\isacharbraceleft}{\kern0pt}{\isachardot}{\kern0pt}{\isachardot}{\kern0pt}{\isacharless}{\kern0pt}\ card\ A{\isacharbraceright}{\kern0pt}{\isachardot}{\kern0pt}\ {\isasymgamma}{\isacharprime}{\kern0pt}\ {\isacharparenleft}{\kern0pt}{\isasymgamma}\ x{\isacharparenright}{\kern0pt}\ {\isacharequal}{\kern0pt}\ x{\isacharparenright}{\kern0pt}\ {\isasymand}\ {\isacharparenleft}{\kern0pt}{\isasymforall}y\ {\isasymin}\ A{\isachardot}{\kern0pt}\ {\isasymgamma}\ {\isacharparenleft}{\kern0pt}{\isasymgamma}{\isacharprime}{\kern0pt}\ y{\isacharparenright}{\kern0pt}\ {\isacharequal}{\kern0pt}\ y{\isacharparenright}{\kern0pt}{\isachardoublequoteclose}\isanewline
\isanewline
\isakeywordONE{lemma}\isamarkupfalse%
\ exists{\isacharunderscore}{\kern0pt}bij{\isacharunderscore}{\kern0pt}Nt{\isacharunderscore}{\kern0pt}Var{\isacharcolon}{\kern0pt}\ {\isachardoublequoteopen}finite\ A\ {\isasymLongrightarrow}\ {\isasymexists}{\isasymgamma}\ {\isasymgamma}{\isacharprime}{\kern0pt}{\isachardot}{\kern0pt}\ bij{\isacharunderscore}{\kern0pt}Nt{\isacharunderscore}{\kern0pt}Var\ A\ {\isasymgamma}\ {\isasymgamma}{\isacharprime}{\kern0pt}{\isachardoublequoteclose}\isanewline
%
\isadelimproof
%
\endisadelimproof
%
\isatagproof
\isakeywordONE{proof}\isamarkupfalse%
\ {\isacharminus}{\kern0pt}\isanewline
\ \ \isakeywordTHREE{assume}\isamarkupfalse%
\ {\isachardoublequoteopen}finite\ A{\isachardoublequoteclose}\isanewline
\ \ \isakeywordONE{then}\isamarkupfalse%
\ \isakeywordONE{have}\isamarkupfalse%
\ {\isachardoublequoteopen}{\isasymexists}{\isasymgamma}{\isachardot}{\kern0pt}\ bij{\isacharunderscore}{\kern0pt}betw\ {\isasymgamma}\ {\isacharbraceleft}{\kern0pt}{\isachardot}{\kern0pt}{\isachardot}{\kern0pt}{\isacharless}{\kern0pt}\ card\ A{\isacharbraceright}{\kern0pt}\ A{\isachardoublequoteclose}\ \isakeywordONE{by}\isamarkupfalse%
\ {\isacharparenleft}{\kern0pt}simp\ add{\isacharcolon}{\kern0pt}\ bij{\isacharunderscore}{\kern0pt}betw{\isacharunderscore}{\kern0pt}iff{\isacharunderscore}{\kern0pt}card{\isacharparenright}{\kern0pt}\isanewline
\ \ \isakeywordONE{then}\isamarkupfalse%
\ \isakeywordTHREE{obtain}\isamarkupfalse%
\ {\isasymgamma}\ \isakeywordTWO{where}\ {\isadigit{1}}{\isacharcolon}{\kern0pt}\ {\isachardoublequoteopen}bij{\isacharunderscore}{\kern0pt}betw\ {\isasymgamma}\ {\isacharbraceleft}{\kern0pt}{\isachardot}{\kern0pt}{\isachardot}{\kern0pt}{\isacharless}{\kern0pt}\ card\ A{\isacharbraceright}{\kern0pt}\ A{\isachardoublequoteclose}\ \isakeywordONE{by}\isamarkupfalse%
\ blast\isanewline
\ \ \isakeywordONE{let}\isamarkupfalse%
\ {\isacharquery}{\kern0pt}{\isasymgamma}{\isacharprime}{\kern0pt}\ {\isacharequal}{\kern0pt}\ {\isachardoublequoteopen}the{\isacharunderscore}{\kern0pt}inv{\isacharunderscore}{\kern0pt}into\ {\isacharbraceleft}{\kern0pt}{\isachardot}{\kern0pt}{\isachardot}{\kern0pt}{\isacharless}{\kern0pt}\ card\ A{\isacharbraceright}{\kern0pt}\ {\isasymgamma}{\isachardoublequoteclose}\isanewline
\ \ \isakeywordONE{from}\isamarkupfalse%
\ the{\isacharunderscore}{\kern0pt}inv{\isacharunderscore}{\kern0pt}into{\isacharunderscore}{\kern0pt}f{\isacharunderscore}{\kern0pt}f\ {\isadigit{1}}\ \isakeywordONE{have}\isamarkupfalse%
\ {\isadigit{2}}{\isacharcolon}{\kern0pt}\ {\isachardoublequoteopen}{\isasymforall}x\ {\isasymin}\ {\isacharbraceleft}{\kern0pt}{\isachardot}{\kern0pt}{\isachardot}{\kern0pt}{\isacharless}{\kern0pt}\ card\ A{\isacharbraceright}{\kern0pt}{\isachardot}{\kern0pt}\ {\isacharquery}{\kern0pt}{\isasymgamma}{\isacharprime}{\kern0pt}\ {\isacharparenleft}{\kern0pt}{\isasymgamma}\ x{\isacharparenright}{\kern0pt}\ {\isacharequal}{\kern0pt}\ x{\isachardoublequoteclose}\ \isakeywordONE{unfolding}\isamarkupfalse%
\ bij{\isacharunderscore}{\kern0pt}betw{\isacharunderscore}{\kern0pt}def\ \isakeywordONE{by}\isamarkupfalse%
\ fast\isanewline
\ \ \isakeywordONE{from}\isamarkupfalse%
\ bij{\isacharunderscore}{\kern0pt}betw{\isacharunderscore}{\kern0pt}the{\isacharunderscore}{\kern0pt}inv{\isacharunderscore}{\kern0pt}into{\isacharbrackleft}{\kern0pt}OF\ {\isadigit{1}}{\isacharbrackright}{\kern0pt}\ \isakeywordONE{have}\isamarkupfalse%
\ {\isadigit{3}}{\isacharcolon}{\kern0pt}\ {\isachardoublequoteopen}bij{\isacharunderscore}{\kern0pt}betw\ {\isacharquery}{\kern0pt}{\isasymgamma}{\isacharprime}{\kern0pt}\ A\ {\isacharbraceleft}{\kern0pt}{\isachardot}{\kern0pt}{\isachardot}{\kern0pt}{\isacharless}{\kern0pt}\ card\ A{\isacharbraceright}{\kern0pt}{\isachardoublequoteclose}\ \isakeywordONE{by}\isamarkupfalse%
\ blast\isanewline
\ \ \isakeywordONE{with}\isamarkupfalse%
\ {\isadigit{1}}\ f{\isacharunderscore}{\kern0pt}the{\isacharunderscore}{\kern0pt}inv{\isacharunderscore}{\kern0pt}into{\isacharunderscore}{\kern0pt}f{\isacharunderscore}{\kern0pt}bij{\isacharunderscore}{\kern0pt}betw\ \isakeywordONE{have}\isamarkupfalse%
\ {\isadigit{4}}{\isacharcolon}{\kern0pt}\ {\isachardoublequoteopen}{\isasymforall}y\ {\isasymin}\ A{\isachardot}{\kern0pt}\ {\isasymgamma}\ {\isacharparenleft}{\kern0pt}{\isacharquery}{\kern0pt}{\isasymgamma}{\isacharprime}{\kern0pt}\ y{\isacharparenright}{\kern0pt}\ {\isacharequal}{\kern0pt}\ y{\isachardoublequoteclose}\ \isakeywordONE{by}\isamarkupfalse%
\ metis\isanewline
\ \ \isakeywordONE{from}\isamarkupfalse%
\ {\isadigit{1}}\ {\isadigit{2}}\ {\isadigit{3}}\ {\isadigit{4}}\ \isakeywordTHREE{show}\isamarkupfalse%
\ {\isacharquery}{\kern0pt}thesis\ \isakeywordONE{unfolding}\isamarkupfalse%
\ bij{\isacharunderscore}{\kern0pt}Nt{\isacharunderscore}{\kern0pt}Var{\isacharunderscore}{\kern0pt}def\ \isakeywordONE{by}\isamarkupfalse%
\ blast\isanewline
\isakeywordONE{qed}\isamarkupfalse%
%
\endisatagproof
{\isafoldproof}%
%
\isadelimproof
\isanewline
%
\endisadelimproof
\isanewline
\isanewline
\isakeywordONE{locale}\isamarkupfalse%
\ CFG{\isacharunderscore}{\kern0pt}eq{\isacharunderscore}{\kern0pt}sys\ {\isacharequal}{\kern0pt}\isanewline
\ \ \isakeywordTWO{fixes}\ P\ {\isacharcolon}{\kern0pt}{\isacharcolon}{\kern0pt}\ {\isachardoublequoteopen}{\isacharparenleft}{\kern0pt}{\isacharprime}{\kern0pt}n{\isacharcomma}{\kern0pt}{\isacharprime}{\kern0pt}a{\isacharparenright}{\kern0pt}\ Prods{\isachardoublequoteclose}\isanewline
\ \ \isakeywordTWO{fixes}\ S\ {\isacharcolon}{\kern0pt}{\isacharcolon}{\kern0pt}\ {\isacharprime}{\kern0pt}n\isanewline
\ \ \isakeywordTWO{fixes}\ {\isasymgamma}\ {\isacharcolon}{\kern0pt}{\isacharcolon}{\kern0pt}\ {\isachardoublequoteopen}nat\ {\isasymRightarrow}\ {\isacharprime}{\kern0pt}n{\isachardoublequoteclose}\isanewline
\ \ \isakeywordTWO{fixes}\ {\isasymgamma}{\isacharprime}{\kern0pt}\ {\isacharcolon}{\kern0pt}{\isacharcolon}{\kern0pt}\ {\isachardoublequoteopen}{\isacharprime}{\kern0pt}n\ {\isasymRightarrow}\ nat{\isachardoublequoteclose}\isanewline
\ \ \isakeywordTWO{assumes}\ finite{\isacharunderscore}{\kern0pt}P{\isacharcolon}{\kern0pt}\ {\isachardoublequoteopen}finite\ P{\isachardoublequoteclose}\isanewline
\ \ \isakeywordTWO{assumes}\ bij{\isacharunderscore}{\kern0pt}{\isasymgamma}{\isacharunderscore}{\kern0pt}{\isasymgamma}{\isacharprime}{\kern0pt}{\isacharcolon}{\kern0pt}\ \ {\isachardoublequoteopen}bij{\isacharunderscore}{\kern0pt}Nt{\isacharunderscore}{\kern0pt}Var\ {\isacharparenleft}{\kern0pt}Nts\ P{\isacharparenright}{\kern0pt}\ {\isasymgamma}\ {\isasymgamma}{\isacharprime}{\kern0pt}{\isachardoublequoteclose}\isanewline
\isakeywordTWO{begin}%
\begin{isamarkuptext}%
The following definitions construct a regular language expression for a single production. This
happens step by step, i.e. starting with a single symbol (terminal or non-terminal) and then extending
this to a single production. The definitions closely follow the definitions \isa{\isaconst{inst{\isacharunderscore}{\kern0pt}sym}},
\isa{\isaconst{concats}} and \isa{\isaconst{inst{\isacharunderscore}{\kern0pt}syms}} in \isa{Context{\isacharunderscore}{\kern0pt}Free{\isacharunderscore}{\kern0pt}Grammar{\isachardot}{\kern0pt}Context{\isacharunderscore}{\kern0pt}Free{\isacharunderscore}{\kern0pt}Language}.%
\end{isamarkuptext}\isamarkuptrue%
\isakeywordONE{definition}\isamarkupfalse%
\ rlexp{\isacharunderscore}{\kern0pt}sym\ {\isacharcolon}{\kern0pt}{\isacharcolon}{\kern0pt}\ {\isachardoublequoteopen}{\isacharparenleft}{\kern0pt}{\isacharprime}{\kern0pt}n{\isacharcomma}{\kern0pt}\ {\isacharprime}{\kern0pt}a{\isacharparenright}{\kern0pt}\ sym\ {\isasymRightarrow}\ {\isacharprime}{\kern0pt}a\ rlexp{\isachardoublequoteclose}\ \isakeywordTWO{where}\isanewline
\ \ {\isachardoublequoteopen}rlexp{\isacharunderscore}{\kern0pt}sym\ s\ {\isacharequal}{\kern0pt}\ {\isacharparenleft}{\kern0pt}case\ s\ of\ Tm\ a\ {\isasymRightarrow}\ Const\ {\isacharbraceleft}{\kern0pt}{\isacharbrackleft}{\kern0pt}a{\isacharbrackright}{\kern0pt}{\isacharbraceright}{\kern0pt}\ {\isacharbar}{\kern0pt}\ Nt\ A\ {\isasymRightarrow}\ Var\ {\isacharparenleft}{\kern0pt}{\isasymgamma}{\isacharprime}{\kern0pt}\ A{\isacharparenright}{\kern0pt}{\isacharparenright}{\kern0pt}{\isachardoublequoteclose}\isanewline
\isanewline
\isakeywordONE{definition}\isamarkupfalse%
\ rlexp{\isacharunderscore}{\kern0pt}concats\ {\isacharcolon}{\kern0pt}{\isacharcolon}{\kern0pt}\ {\isachardoublequoteopen}{\isacharprime}{\kern0pt}a\ rlexp\ list\ {\isasymRightarrow}\ {\isacharprime}{\kern0pt}a\ rlexp{\isachardoublequoteclose}\ \isakeywordTWO{where}\isanewline
\ \ {\isachardoublequoteopen}rlexp{\isacharunderscore}{\kern0pt}concats\ fs\ {\isacharequal}{\kern0pt}\ foldr\ Concat\ fs\ {\isacharparenleft}{\kern0pt}Const\ {\isacharbraceleft}{\kern0pt}{\isacharbrackleft}{\kern0pt}{\isacharbrackright}{\kern0pt}{\isacharbraceright}{\kern0pt}{\isacharparenright}{\kern0pt}{\isachardoublequoteclose}\isanewline
\isanewline
\isakeywordONE{definition}\isamarkupfalse%
\ rlexp{\isacharunderscore}{\kern0pt}syms\ {\isacharcolon}{\kern0pt}{\isacharcolon}{\kern0pt}\ {\isachardoublequoteopen}{\isacharparenleft}{\kern0pt}{\isacharprime}{\kern0pt}n{\isacharcomma}{\kern0pt}\ {\isacharprime}{\kern0pt}a{\isacharparenright}{\kern0pt}\ syms\ {\isasymRightarrow}\ {\isacharprime}{\kern0pt}a\ rlexp{\isachardoublequoteclose}\ \isakeywordTWO{where}\isanewline
\ \ {\isachardoublequoteopen}rlexp{\isacharunderscore}{\kern0pt}syms\ w\ {\isacharequal}{\kern0pt}\ rlexp{\isacharunderscore}{\kern0pt}concats\ {\isacharparenleft}{\kern0pt}map\ rlexp{\isacharunderscore}{\kern0pt}sym\ w{\isacharparenright}{\kern0pt}{\isachardoublequoteclose}%
\begin{isamarkuptext}%
Now it is shown that the regular language expression constructed for a single production
is \isa{\isaconst{reg{\isacharunderscore}{\kern0pt}eval}}. Again, this happens step by step:%
\end{isamarkuptext}\isamarkuptrue%
\isakeywordONE{lemma}\isamarkupfalse%
\ rlexp{\isacharunderscore}{\kern0pt}sym{\isacharunderscore}{\kern0pt}reg{\isacharcolon}{\kern0pt}\ {\isachardoublequoteopen}reg{\isacharunderscore}{\kern0pt}eval\ {\isacharparenleft}{\kern0pt}rlexp{\isacharunderscore}{\kern0pt}sym\ s{\isacharparenright}{\kern0pt}{\isachardoublequoteclose}\isanewline
%
\isadelimproof
%
\endisadelimproof
%
\isatagproof
\isakeywordONE{unfolding}\isamarkupfalse%
\ rlexp{\isacharunderscore}{\kern0pt}sym{\isacharunderscore}{\kern0pt}def\ \isakeywordONE{proof}\isamarkupfalse%
\ {\isacharparenleft}{\kern0pt}induction\ s{\isacharparenright}{\kern0pt}\isanewline
\ \ \isakeywordTHREE{case}\isamarkupfalse%
\ {\isacharparenleft}{\kern0pt}Tm\ x{\isacharparenright}{\kern0pt}\isanewline
\ \ \isakeywordONE{have}\isamarkupfalse%
\ {\isachardoublequoteopen}regular{\isacharunderscore}{\kern0pt}lang\ {\isacharbraceleft}{\kern0pt}{\isacharbrackleft}{\kern0pt}x{\isacharbrackright}{\kern0pt}{\isacharbraceright}{\kern0pt}{\isachardoublequoteclose}\ \isakeywordONE{by}\isamarkupfalse%
\ {\isacharparenleft}{\kern0pt}meson\ lang{\isachardot}{\kern0pt}simps{\isacharparenleft}{\kern0pt}{\isadigit{3}}{\isacharparenright}{\kern0pt}{\isacharparenright}{\kern0pt}\isanewline
\ \ \isakeywordONE{then}\isamarkupfalse%
\ \isakeywordTHREE{show}\isamarkupfalse%
\ {\isacharquery}{\kern0pt}case\ \isakeywordONE{by}\isamarkupfalse%
\ auto\isanewline
\isakeywordONE{qed}\isamarkupfalse%
\ auto%
\endisatagproof
{\isafoldproof}%
%
\isadelimproof
\isanewline
%
\endisadelimproof
\isanewline
\isakeywordONE{lemma}\isamarkupfalse%
\ rlexp{\isacharunderscore}{\kern0pt}concats{\isacharunderscore}{\kern0pt}reg{\isacharcolon}{\kern0pt}\isanewline
\ \ \isakeywordTWO{assumes}\ {\isachardoublequoteopen}{\isasymforall}f\ {\isasymin}\ set\ fs{\isachardot}{\kern0pt}\ reg{\isacharunderscore}{\kern0pt}eval\ f{\isachardoublequoteclose}\isanewline
\ \ \ \ \isakeywordTWO{shows}\ {\isachardoublequoteopen}reg{\isacharunderscore}{\kern0pt}eval\ {\isacharparenleft}{\kern0pt}rlexp{\isacharunderscore}{\kern0pt}concats\ fs{\isacharparenright}{\kern0pt}{\isachardoublequoteclose}\isanewline
%
\isadelimproof
\ \ %
\endisadelimproof
%
\isatagproof
\isakeywordONE{using}\isamarkupfalse%
\ assms\ epsilon{\isacharunderscore}{\kern0pt}regular\ \isakeywordONE{unfolding}\isamarkupfalse%
\ rlexp{\isacharunderscore}{\kern0pt}concats{\isacharunderscore}{\kern0pt}def\ \isakeywordONE{by}\isamarkupfalse%
\ {\isacharparenleft}{\kern0pt}induction\ fs{\isacharparenright}{\kern0pt}\ auto%
\endisatagproof
{\isafoldproof}%
%
\isadelimproof
\isanewline
%
\endisadelimproof
\isanewline
\isakeywordONE{lemma}\isamarkupfalse%
\ rlexp{\isacharunderscore}{\kern0pt}syms{\isacharunderscore}{\kern0pt}reg{\isacharcolon}{\kern0pt}\ {\isachardoublequoteopen}reg{\isacharunderscore}{\kern0pt}eval\ {\isacharparenleft}{\kern0pt}rlexp{\isacharunderscore}{\kern0pt}syms\ w{\isacharparenright}{\kern0pt}{\isachardoublequoteclose}\isanewline
%
\isadelimproof
%
\endisadelimproof
%
\isatagproof
\isakeywordONE{proof}\isamarkupfalse%
\ {\isacharminus}{\kern0pt}\isanewline
\ \ \isakeywordONE{from}\isamarkupfalse%
\ rlexp{\isacharunderscore}{\kern0pt}sym{\isacharunderscore}{\kern0pt}reg\ \isakeywordONE{have}\isamarkupfalse%
\ {\isachardoublequoteopen}{\isasymforall}s\ {\isasymin}\ set\ w{\isachardot}{\kern0pt}\ reg{\isacharunderscore}{\kern0pt}eval\ {\isacharparenleft}{\kern0pt}rlexp{\isacharunderscore}{\kern0pt}sym\ s{\isacharparenright}{\kern0pt}{\isachardoublequoteclose}\ \isakeywordONE{by}\isamarkupfalse%
\ blast\isanewline
\ \ \isakeywordONE{with}\isamarkupfalse%
\ rlexp{\isacharunderscore}{\kern0pt}concats{\isacharunderscore}{\kern0pt}reg\ \isakeywordTHREE{show}\isamarkupfalse%
\ {\isacharquery}{\kern0pt}thesis\ \isakeywordONE{unfolding}\isamarkupfalse%
\ rlexp{\isacharunderscore}{\kern0pt}syms{\isacharunderscore}{\kern0pt}def\isanewline
\ \ \ \ \isakeywordONE{by}\isamarkupfalse%
\ {\isacharparenleft}{\kern0pt}metis\ {\isacharparenleft}{\kern0pt}no{\isacharunderscore}{\kern0pt}types{\isacharcomma}{\kern0pt}\ lifting{\isacharparenright}{\kern0pt}\ image{\isacharunderscore}{\kern0pt}iff\ list{\isachardot}{\kern0pt}set{\isacharunderscore}{\kern0pt}map{\isacharparenright}{\kern0pt}\isanewline
\isakeywordONE{qed}\isamarkupfalse%
%
\endisatagproof
{\isafoldproof}%
%
\isadelimproof
%
\endisadelimproof
%
\begin{isamarkuptext}%
The subsequent lemmas prove that all variables appearing in the regular language
expression of a single production correspond to non-terminals appearing in the production:%
\end{isamarkuptext}\isamarkuptrue%
\isakeywordONE{lemma}\isamarkupfalse%
\ rlexp{\isacharunderscore}{\kern0pt}sym{\isacharunderscore}{\kern0pt}vars{\isacharunderscore}{\kern0pt}Nt{\isacharcolon}{\kern0pt}\isanewline
\ \ \isakeywordTWO{assumes}\ {\isachardoublequoteopen}s\ {\isacharparenleft}{\kern0pt}{\isasymgamma}{\isacharprime}{\kern0pt}\ A{\isacharparenright}{\kern0pt}\ {\isacharequal}{\kern0pt}\ L\ A{\isachardoublequoteclose}\isanewline
\ \ \ \ \isakeywordTWO{shows}\ {\isachardoublequoteopen}vars\ {\isacharparenleft}{\kern0pt}rlexp{\isacharunderscore}{\kern0pt}sym\ {\isacharparenleft}{\kern0pt}Nt\ A{\isacharparenright}{\kern0pt}{\isacharparenright}{\kern0pt}\ {\isacharequal}{\kern0pt}\ {\isacharbraceleft}{\kern0pt}{\isasymgamma}{\isacharprime}{\kern0pt}\ A{\isacharbraceright}{\kern0pt}{\isachardoublequoteclose}\isanewline
%
\isadelimproof
\ \ %
\endisadelimproof
%
\isatagproof
\isakeywordONE{using}\isamarkupfalse%
\ assms\ \isakeywordONE{unfolding}\isamarkupfalse%
\ rlexp{\isacharunderscore}{\kern0pt}sym{\isacharunderscore}{\kern0pt}def\ \isakeywordONE{by}\isamarkupfalse%
\ simp%
\endisatagproof
{\isafoldproof}%
%
\isadelimproof
\isanewline
%
\endisadelimproof
\isanewline
\isakeywordONE{lemma}\isamarkupfalse%
\ rlexp{\isacharunderscore}{\kern0pt}sym{\isacharunderscore}{\kern0pt}vars{\isacharunderscore}{\kern0pt}Tm{\isacharcolon}{\kern0pt}\ {\isachardoublequoteopen}vars\ {\isacharparenleft}{\kern0pt}rlexp{\isacharunderscore}{\kern0pt}sym\ {\isacharparenleft}{\kern0pt}Tm\ x{\isacharparenright}{\kern0pt}{\isacharparenright}{\kern0pt}\ {\isacharequal}{\kern0pt}\ {\isacharbraceleft}{\kern0pt}{\isacharbraceright}{\kern0pt}{\isachardoublequoteclose}\isanewline
%
\isadelimproof
\ \ %
\endisadelimproof
%
\isatagproof
\isakeywordONE{unfolding}\isamarkupfalse%
\ rlexp{\isacharunderscore}{\kern0pt}sym{\isacharunderscore}{\kern0pt}def\ \isakeywordONE{by}\isamarkupfalse%
\ simp%
\endisatagproof
{\isafoldproof}%
%
\isadelimproof
\isanewline
%
\endisadelimproof
\isanewline
\isakeywordONE{lemma}\isamarkupfalse%
\ rlexp{\isacharunderscore}{\kern0pt}concats{\isacharunderscore}{\kern0pt}vars{\isacharcolon}{\kern0pt}\ {\isachardoublequoteopen}vars\ {\isacharparenleft}{\kern0pt}rlexp{\isacharunderscore}{\kern0pt}concats\ fs{\isacharparenright}{\kern0pt}\ {\isacharequal}{\kern0pt}\ {\isasymUnion}{\isacharparenleft}{\kern0pt}vars\ {\isacharbackquote}{\kern0pt}\ set\ fs{\isacharparenright}{\kern0pt}{\isachardoublequoteclose}\isanewline
%
\isadelimproof
\ \ %
\endisadelimproof
%
\isatagproof
\isakeywordONE{unfolding}\isamarkupfalse%
\ rlexp{\isacharunderscore}{\kern0pt}concats{\isacharunderscore}{\kern0pt}def\ \isakeywordONE{by}\isamarkupfalse%
\ {\isacharparenleft}{\kern0pt}induction\ fs{\isacharparenright}{\kern0pt}\ simp{\isacharunderscore}{\kern0pt}all%
\endisatagproof
{\isafoldproof}%
%
\isadelimproof
\isanewline
%
\endisadelimproof
\isanewline
\isanewline
\isakeywordONE{lemma}\isamarkupfalse%
\ insts{\isacharprime}{\kern0pt}{\isacharunderscore}{\kern0pt}vars{\isacharcolon}{\kern0pt}\ {\isachardoublequoteopen}vars\ {\isacharparenleft}{\kern0pt}rlexp{\isacharunderscore}{\kern0pt}syms\ w{\isacharparenright}{\kern0pt}\ {\isasymsubseteq}\ {\isasymgamma}{\isacharprime}{\kern0pt}\ {\isacharbackquote}{\kern0pt}\ nts{\isacharunderscore}{\kern0pt}syms\ w{\isachardoublequoteclose}\isanewline
%
\isadelimproof
%
\endisadelimproof
%
\isatagproof
\isakeywordONE{proof}\isamarkupfalse%
\isanewline
\ \ \isakeywordTHREE{fix}\isamarkupfalse%
\ x\isanewline
\ \ \isakeywordTHREE{assume}\isamarkupfalse%
\ {\isachardoublequoteopen}x\ {\isasymin}\ vars\ {\isacharparenleft}{\kern0pt}rlexp{\isacharunderscore}{\kern0pt}syms\ w{\isacharparenright}{\kern0pt}{\isachardoublequoteclose}\isanewline
\ \ \isakeywordONE{with}\isamarkupfalse%
\ rlexp{\isacharunderscore}{\kern0pt}concats{\isacharunderscore}{\kern0pt}vars\ \isakeywordONE{have}\isamarkupfalse%
\ {\isachardoublequoteopen}x\ {\isasymin}\ {\isasymUnion}{\isacharparenleft}{\kern0pt}vars\ {\isacharbackquote}{\kern0pt}\ set\ {\isacharparenleft}{\kern0pt}map\ rlexp{\isacharunderscore}{\kern0pt}sym\ w{\isacharparenright}{\kern0pt}{\isacharparenright}{\kern0pt}{\isachardoublequoteclose}\isanewline
\ \ \ \ \isakeywordONE{unfolding}\isamarkupfalse%
\ rlexp{\isacharunderscore}{\kern0pt}syms{\isacharunderscore}{\kern0pt}def\ \isakeywordONE{by}\isamarkupfalse%
\ blast\isanewline
\ \ \isakeywordONE{then}\isamarkupfalse%
\ \isakeywordTHREE{obtain}\isamarkupfalse%
\ f\ \isakeywordTWO{where}\ {\isacharasterisk}{\kern0pt}{\isacharcolon}{\kern0pt}\ {\isachardoublequoteopen}f\ {\isasymin}\ set\ {\isacharparenleft}{\kern0pt}map\ rlexp{\isacharunderscore}{\kern0pt}sym\ w{\isacharparenright}{\kern0pt}\ {\isasymand}\ x\ {\isasymin}\ vars\ f{\isachardoublequoteclose}\ \isakeywordONE{by}\isamarkupfalse%
\ blast\isanewline
\ \ \isakeywordONE{then}\isamarkupfalse%
\ \isakeywordTHREE{obtain}\isamarkupfalse%
\ s\ \isakeywordTWO{where}\ {\isacharasterisk}{\kern0pt}{\isacharasterisk}{\kern0pt}{\isacharcolon}{\kern0pt}\ {\isachardoublequoteopen}s\ {\isasymin}\ set\ w\ {\isasymand}\ rlexp{\isacharunderscore}{\kern0pt}sym\ s\ {\isacharequal}{\kern0pt}\ f{\isachardoublequoteclose}\ \isakeywordONE{by}\isamarkupfalse%
\ auto\isanewline
\ \ \isakeywordONE{with}\isamarkupfalse%
\ {\isacharasterisk}{\kern0pt}\ rlexp{\isacharunderscore}{\kern0pt}sym{\isacharunderscore}{\kern0pt}vars{\isacharunderscore}{\kern0pt}Tm\ \isakeywordTHREE{obtain}\isamarkupfalse%
\ A\ \isakeywordTWO{where}\ {\isacharasterisk}{\kern0pt}{\isacharasterisk}{\kern0pt}{\isacharasterisk}{\kern0pt}{\isacharcolon}{\kern0pt}\ {\isachardoublequoteopen}s\ {\isacharequal}{\kern0pt}\ Nt\ A{\isachardoublequoteclose}\ \isakeywordONE{by}\isamarkupfalse%
\ {\isacharparenleft}{\kern0pt}metis\ empty{\isacharunderscore}{\kern0pt}iff\ sym{\isachardot}{\kern0pt}exhaust{\isacharparenright}{\kern0pt}\isanewline
\ \ \isakeywordONE{with}\isamarkupfalse%
\ {\isacharasterisk}{\kern0pt}{\isacharasterisk}{\kern0pt}\ \isakeywordONE{have}\isamarkupfalse%
\ {\isacharasterisk}{\kern0pt}{\isacharasterisk}{\kern0pt}{\isacharasterisk}{\kern0pt}{\isacharasterisk}{\kern0pt}{\isacharcolon}{\kern0pt}\ {\isachardoublequoteopen}A\ {\isasymin}\ nts{\isacharunderscore}{\kern0pt}syms\ w{\isachardoublequoteclose}\ \isakeywordONE{unfolding}\isamarkupfalse%
\ nts{\isacharunderscore}{\kern0pt}syms{\isacharunderscore}{\kern0pt}def\ \isakeywordONE{by}\isamarkupfalse%
\ blast\isanewline
\ \ \isakeywordONE{with}\isamarkupfalse%
\ rlexp{\isacharunderscore}{\kern0pt}sym{\isacharunderscore}{\kern0pt}vars{\isacharunderscore}{\kern0pt}Nt\ \isakeywordONE{have}\isamarkupfalse%
\ {\isachardoublequoteopen}vars\ {\isacharparenleft}{\kern0pt}rlexp{\isacharunderscore}{\kern0pt}sym\ {\isacharparenleft}{\kern0pt}Nt\ A{\isacharparenright}{\kern0pt}{\isacharparenright}{\kern0pt}\ {\isacharequal}{\kern0pt}\ {\isacharbraceleft}{\kern0pt}{\isasymgamma}{\isacharprime}{\kern0pt}\ A{\isacharbraceright}{\kern0pt}{\isachardoublequoteclose}\ \isakeywordONE{by}\isamarkupfalse%
\ blast\isanewline
\ \ \isakeywordONE{with}\isamarkupfalse%
\ {\isacharasterisk}{\kern0pt}\ {\isacharasterisk}{\kern0pt}{\isacharasterisk}{\kern0pt}\ {\isacharasterisk}{\kern0pt}{\isacharasterisk}{\kern0pt}{\isacharasterisk}{\kern0pt}\ {\isacharasterisk}{\kern0pt}{\isacharasterisk}{\kern0pt}{\isacharasterisk}{\kern0pt}{\isacharasterisk}{\kern0pt}\ \isakeywordTHREE{show}\isamarkupfalse%
\ {\isachardoublequoteopen}x\ {\isasymin}\ {\isasymgamma}{\isacharprime}{\kern0pt}\ {\isacharbackquote}{\kern0pt}\ nts{\isacharunderscore}{\kern0pt}syms\ w{\isachardoublequoteclose}\ \isakeywordONE{by}\isamarkupfalse%
\ blast\isanewline
\isakeywordONE{qed}\isamarkupfalse%
%
\endisatagproof
{\isafoldproof}%
%
\isadelimproof
%
\endisadelimproof
%
\begin{isamarkuptext}%
Evaluating the regular language expression of a single production under a valuation
corresponds to instantiating the non-terminals in the production according to the valuation:%
\end{isamarkuptext}\isamarkuptrue%
\isakeywordONE{lemma}\isamarkupfalse%
\ rlexp{\isacharunderscore}{\kern0pt}sym{\isacharunderscore}{\kern0pt}inst{\isacharunderscore}{\kern0pt}Nt{\isacharcolon}{\kern0pt}\isanewline
\ \ \isakeywordTWO{assumes}\ {\isachardoublequoteopen}v\ {\isacharparenleft}{\kern0pt}{\isasymgamma}{\isacharprime}{\kern0pt}\ A{\isacharparenright}{\kern0pt}\ {\isacharequal}{\kern0pt}\ L\ A{\isachardoublequoteclose}\isanewline
\ \ \ \ \isakeywordTWO{shows}\ {\isachardoublequoteopen}eval\ {\isacharparenleft}{\kern0pt}rlexp{\isacharunderscore}{\kern0pt}sym\ {\isacharparenleft}{\kern0pt}Nt\ A{\isacharparenright}{\kern0pt}{\isacharparenright}{\kern0pt}\ v\ {\isacharequal}{\kern0pt}\ inst{\isacharunderscore}{\kern0pt}sym\ L\ {\isacharparenleft}{\kern0pt}Nt\ A{\isacharparenright}{\kern0pt}{\isachardoublequoteclose}\isanewline
%
\isadelimproof
\ \ %
\endisadelimproof
%
\isatagproof
\isakeywordONE{using}\isamarkupfalse%
\ assms\ \isakeywordONE{unfolding}\isamarkupfalse%
\ rlexp{\isacharunderscore}{\kern0pt}sym{\isacharunderscore}{\kern0pt}def\ inst{\isacharunderscore}{\kern0pt}sym{\isacharunderscore}{\kern0pt}def\ \isakeywordONE{by}\isamarkupfalse%
\ force%
\endisatagproof
{\isafoldproof}%
%
\isadelimproof
\isanewline
%
\endisadelimproof
\isanewline
\isakeywordONE{lemma}\isamarkupfalse%
\ rlexp{\isacharunderscore}{\kern0pt}sym{\isacharunderscore}{\kern0pt}inst{\isacharunderscore}{\kern0pt}Tm{\isacharcolon}{\kern0pt}\ {\isachardoublequoteopen}eval\ {\isacharparenleft}{\kern0pt}rlexp{\isacharunderscore}{\kern0pt}sym\ {\isacharparenleft}{\kern0pt}Tm\ a{\isacharparenright}{\kern0pt}{\isacharparenright}{\kern0pt}\ v\ {\isacharequal}{\kern0pt}\ inst{\isacharunderscore}{\kern0pt}sym\ L\ {\isacharparenleft}{\kern0pt}Tm\ a{\isacharparenright}{\kern0pt}{\isachardoublequoteclose}\isanewline
%
\isadelimproof
\ \ %
\endisadelimproof
%
\isatagproof
\isakeywordONE{unfolding}\isamarkupfalse%
\ rlexp{\isacharunderscore}{\kern0pt}sym{\isacharunderscore}{\kern0pt}def\ inst{\isacharunderscore}{\kern0pt}sym{\isacharunderscore}{\kern0pt}def\ \isakeywordONE{by}\isamarkupfalse%
\ force%
\endisatagproof
{\isafoldproof}%
%
\isadelimproof
\isanewline
%
\endisadelimproof
\isanewline
\isakeywordONE{lemma}\isamarkupfalse%
\ rlexp{\isacharunderscore}{\kern0pt}concats{\isacharunderscore}{\kern0pt}concats{\isacharcolon}{\kern0pt}\isanewline
\ \ \isakeywordTWO{assumes}\ {\isachardoublequoteopen}length\ fs\ {\isacharequal}{\kern0pt}\ length\ Ls{\isachardoublequoteclose}\isanewline
\ \ \ \ \ \ \isakeywordTWO{and}\ {\isachardoublequoteopen}{\isasymforall}i\ {\isacharless}{\kern0pt}\ length\ fs{\isachardot}{\kern0pt}\ eval\ {\isacharparenleft}{\kern0pt}fs\ {\isacharbang}{\kern0pt}\ i{\isacharparenright}{\kern0pt}\ v\ {\isacharequal}{\kern0pt}\ Ls\ {\isacharbang}{\kern0pt}\ i{\isachardoublequoteclose}\isanewline
\ \ \ \ \isakeywordTWO{shows}\ {\isachardoublequoteopen}eval\ {\isacharparenleft}{\kern0pt}rlexp{\isacharunderscore}{\kern0pt}concats\ fs{\isacharparenright}{\kern0pt}\ v\ {\isacharequal}{\kern0pt}\ concats\ Ls{\isachardoublequoteclose}\isanewline
%
\isadelimproof
%
\endisadelimproof
%
\isatagproof
\isakeywordONE{using}\isamarkupfalse%
\ assms\ \isakeywordONE{proof}\isamarkupfalse%
\ {\isacharparenleft}{\kern0pt}induction\ fs\ arbitrary{\isacharcolon}{\kern0pt}\ Ls{\isacharparenright}{\kern0pt}\isanewline
\ \ \isakeywordTHREE{case}\isamarkupfalse%
\ Nil\isanewline
\ \ \isakeywordONE{then}\isamarkupfalse%
\ \isakeywordTHREE{show}\isamarkupfalse%
\ {\isacharquery}{\kern0pt}case\ \isakeywordONE{unfolding}\isamarkupfalse%
\ rlexp{\isacharunderscore}{\kern0pt}concats{\isacharunderscore}{\kern0pt}def\ concats{\isacharunderscore}{\kern0pt}def\ \isakeywordONE{by}\isamarkupfalse%
\ simp\isanewline
\isakeywordONE{next}\isamarkupfalse%
\isanewline
\ \ \isakeywordTHREE{case}\isamarkupfalse%
\ {\isacharparenleft}{\kern0pt}Cons\ f{\isadigit{1}}\ fs{\isacharparenright}{\kern0pt}\isanewline
\ \ \isakeywordONE{then}\isamarkupfalse%
\ \isakeywordTHREE{obtain}\isamarkupfalse%
\ L{\isadigit{1}}\ Lr\ \isakeywordTWO{where}\ {\isacharasterisk}{\kern0pt}{\isacharcolon}{\kern0pt}\ {\isachardoublequoteopen}Ls\ {\isacharequal}{\kern0pt}\ L{\isadigit{1}}{\isacharhash}{\kern0pt}Lr{\isachardoublequoteclose}\ \isakeywordONE{by}\isamarkupfalse%
\ {\isacharparenleft}{\kern0pt}metis\ length{\isacharunderscore}{\kern0pt}Suc{\isacharunderscore}{\kern0pt}conv{\isacharparenright}{\kern0pt}\isanewline
\ \ \isakeywordONE{with}\isamarkupfalse%
\ Cons\ \isakeywordONE{have}\isamarkupfalse%
\ {\isachardoublequoteopen}eval\ {\isacharparenleft}{\kern0pt}rlexp{\isacharunderscore}{\kern0pt}concats\ fs{\isacharparenright}{\kern0pt}\ v\ {\isacharequal}{\kern0pt}\ concats\ Lr{\isachardoublequoteclose}\ \isakeywordONE{by}\isamarkupfalse%
\ fastforce\isanewline
\ \ \isakeywordONE{moreover}\isamarkupfalse%
\ \isakeywordONE{from}\isamarkupfalse%
\ Cons{\isachardot}{\kern0pt}prems\ {\isacharasterisk}{\kern0pt}\ \isakeywordONE{have}\isamarkupfalse%
\ {\isachardoublequoteopen}eval\ f{\isadigit{1}}\ v\ {\isacharequal}{\kern0pt}\ L{\isadigit{1}}{\isachardoublequoteclose}\ \isakeywordONE{by}\isamarkupfalse%
\ force\isanewline
\ \ \isakeywordONE{ultimately}\isamarkupfalse%
\ \isakeywordTHREE{show}\isamarkupfalse%
\ {\isacharquery}{\kern0pt}case\ \isakeywordONE{unfolding}\isamarkupfalse%
\ rlexp{\isacharunderscore}{\kern0pt}concats{\isacharunderscore}{\kern0pt}def\ concats{\isacharunderscore}{\kern0pt}def\ \isakeywordONE{by}\isamarkupfalse%
\ {\isacharparenleft}{\kern0pt}simp\ add{\isacharcolon}{\kern0pt}\ {\isachardoublequoteopen}{\isacharasterisk}{\kern0pt}{\isachardoublequoteclose}{\isacharparenright}{\kern0pt}\isanewline
\isakeywordONE{qed}\isamarkupfalse%
%
\endisatagproof
{\isafoldproof}%
%
\isadelimproof
\isanewline
%
\endisadelimproof
\isanewline
\isakeywordONE{lemma}\isamarkupfalse%
\ rlexp{\isacharunderscore}{\kern0pt}syms{\isacharunderscore}{\kern0pt}insts{\isacharcolon}{\kern0pt}\isanewline
\ \ \isakeywordTWO{assumes}\ {\isachardoublequoteopen}{\isasymforall}A\ {\isasymin}\ nts{\isacharunderscore}{\kern0pt}syms\ w{\isachardot}{\kern0pt}\ v\ {\isacharparenleft}{\kern0pt}{\isasymgamma}{\isacharprime}{\kern0pt}\ A{\isacharparenright}{\kern0pt}\ {\isacharequal}{\kern0pt}\ L\ A{\isachardoublequoteclose}\isanewline
\ \ \ \ \isakeywordTWO{shows}\ {\isachardoublequoteopen}eval\ {\isacharparenleft}{\kern0pt}rlexp{\isacharunderscore}{\kern0pt}syms\ w{\isacharparenright}{\kern0pt}\ v\ {\isacharequal}{\kern0pt}\ inst{\isacharunderscore}{\kern0pt}syms\ L\ w{\isachardoublequoteclose}\isanewline
%
\isadelimproof
%
\endisadelimproof
%
\isatagproof
\isakeywordONE{proof}\isamarkupfalse%
\ {\isacharminus}{\kern0pt}\isanewline
\ \ \isakeywordONE{have}\isamarkupfalse%
\ {\isachardoublequoteopen}{\isasymforall}i\ {\isacharless}{\kern0pt}\ length\ w{\isachardot}{\kern0pt}\ eval\ {\isacharparenleft}{\kern0pt}rlexp{\isacharunderscore}{\kern0pt}sym\ {\isacharparenleft}{\kern0pt}w{\isacharbang}{\kern0pt}i{\isacharparenright}{\kern0pt}{\isacharparenright}{\kern0pt}\ v\ {\isacharequal}{\kern0pt}\ inst{\isacharunderscore}{\kern0pt}sym\ L\ {\isacharparenleft}{\kern0pt}w{\isacharbang}{\kern0pt}i{\isacharparenright}{\kern0pt}{\isachardoublequoteclose}\isanewline
\ \ \isakeywordONE{proof}\isamarkupfalse%
\ {\isacharparenleft}{\kern0pt}rule\ allI{\isacharcomma}{\kern0pt}\ rule\ impI{\isacharparenright}{\kern0pt}\isanewline
\ \ \ \ \isakeywordTHREE{fix}\isamarkupfalse%
\ i\isanewline
\ \ \ \ \isakeywordTHREE{assume}\isamarkupfalse%
\ {\isachardoublequoteopen}i\ {\isacharless}{\kern0pt}\ length\ w{\isachardoublequoteclose}\isanewline
\ \ \ \ \isakeywordONE{then}\isamarkupfalse%
\ \isakeywordTHREE{show}\isamarkupfalse%
\ {\isachardoublequoteopen}eval\ {\isacharparenleft}{\kern0pt}rlexp{\isacharunderscore}{\kern0pt}sym\ {\isacharparenleft}{\kern0pt}w\ {\isacharbang}{\kern0pt}\ i{\isacharparenright}{\kern0pt}{\isacharparenright}{\kern0pt}\ v\ {\isacharequal}{\kern0pt}\ inst{\isacharunderscore}{\kern0pt}sym\ L\ {\isacharparenleft}{\kern0pt}w\ {\isacharbang}{\kern0pt}\ i{\isacharparenright}{\kern0pt}{\isachardoublequoteclose}\isanewline
\ \ \ \ \ \ \isakeywordONE{using}\isamarkupfalse%
\ assms\ \isakeywordONE{proof}\isamarkupfalse%
\ {\isacharparenleft}{\kern0pt}induction\ {\isachardoublequoteopen}w{\isacharbang}{\kern0pt}i{\isachardoublequoteclose}{\isacharparenright}{\kern0pt}\isanewline
\ \ \ \ \ \ \isakeywordTHREE{case}\isamarkupfalse%
\ {\isacharparenleft}{\kern0pt}Nt\ A{\isacharparenright}{\kern0pt}\isanewline
\ \ \ \ \ \ \isakeywordONE{then}\isamarkupfalse%
\ \isakeywordONE{have}\isamarkupfalse%
\ {\isachardoublequoteopen}v\ {\isacharparenleft}{\kern0pt}{\isasymgamma}{\isacharprime}{\kern0pt}\ A{\isacharparenright}{\kern0pt}\ {\isacharequal}{\kern0pt}\ L\ A{\isachardoublequoteclose}\ \isakeywordONE{unfolding}\isamarkupfalse%
\ nts{\isacharunderscore}{\kern0pt}syms{\isacharunderscore}{\kern0pt}def\ \isakeywordONE{by}\isamarkupfalse%
\ force\isanewline
\ \ \ \ \ \ \isakeywordONE{with}\isamarkupfalse%
\ rlexp{\isacharunderscore}{\kern0pt}sym{\isacharunderscore}{\kern0pt}inst{\isacharunderscore}{\kern0pt}Nt\ Nt\ \isakeywordTHREE{show}\isamarkupfalse%
\ {\isacharquery}{\kern0pt}case\ \isakeywordONE{by}\isamarkupfalse%
\ metis\isanewline
\ \ \ \ \isakeywordONE{next}\isamarkupfalse%
\isanewline
\ \ \ \ \ \ \isakeywordTHREE{case}\isamarkupfalse%
\ {\isacharparenleft}{\kern0pt}Tm\ x{\isacharparenright}{\kern0pt}\isanewline
\ \ \ \ \ \ \isakeywordONE{with}\isamarkupfalse%
\ rlexp{\isacharunderscore}{\kern0pt}sym{\isacharunderscore}{\kern0pt}inst{\isacharunderscore}{\kern0pt}Tm\ \isakeywordTHREE{show}\isamarkupfalse%
\ {\isacharquery}{\kern0pt}case\ \isakeywordONE{by}\isamarkupfalse%
\ metis\isanewline
\ \ \ \ \isakeywordONE{qed}\isamarkupfalse%
\isanewline
\ \ \isakeywordONE{qed}\isamarkupfalse%
\isanewline
\ \ \isakeywordONE{then}\isamarkupfalse%
\ \isakeywordTHREE{show}\isamarkupfalse%
\ {\isacharquery}{\kern0pt}thesis\ \isakeywordONE{unfolding}\isamarkupfalse%
\ rlexp{\isacharunderscore}{\kern0pt}syms{\isacharunderscore}{\kern0pt}def\ inst{\isacharunderscore}{\kern0pt}syms{\isacharunderscore}{\kern0pt}def\ \isakeywordONE{using}\isamarkupfalse%
\ rlexp{\isacharunderscore}{\kern0pt}concats{\isacharunderscore}{\kern0pt}concats\isanewline
\ \ \ \ \isakeywordONE{by}\isamarkupfalse%
\ {\isacharparenleft}{\kern0pt}metis\ {\isacharparenleft}{\kern0pt}mono{\isacharunderscore}{\kern0pt}tags{\isacharcomma}{\kern0pt}\ lifting{\isacharparenright}{\kern0pt}\ length{\isacharunderscore}{\kern0pt}map\ nth{\isacharunderscore}{\kern0pt}map{\isacharparenright}{\kern0pt}\isanewline
\isakeywordONE{qed}\isamarkupfalse%
%
\endisatagproof
{\isafoldproof}%
%
\isadelimproof
%
\endisadelimproof
%
\begin{isamarkuptext}%
Each non-terminal of the CFG induces some \isa{\isaconst{reg{\isacharunderscore}{\kern0pt}eval}} equation. We do not directly construct
the equation but only prove its existence:%
\end{isamarkuptext}\isamarkuptrue%
\isakeywordONE{lemma}\isamarkupfalse%
\ subst{\isacharunderscore}{\kern0pt}lang{\isacharunderscore}{\kern0pt}rlexp{\isacharcolon}{\kern0pt}\isanewline
\ \ {\isachardoublequoteopen}{\isasymexists}eq{\isachardot}{\kern0pt}\ reg{\isacharunderscore}{\kern0pt}eval\ eq\ {\isasymand}\ vars\ eq\ {\isasymsubseteq}\ {\isasymgamma}{\isacharprime}{\kern0pt}\ {\isacharbackquote}{\kern0pt}\ Nts\ P\isanewline
\ \ \ \ \ \ \ \ \ {\isasymand}\ {\isacharparenleft}{\kern0pt}{\isasymforall}v\ L{\isachardot}{\kern0pt}\ {\isacharparenleft}{\kern0pt}{\isasymforall}A\ {\isasymin}\ Nts\ P{\isachardot}{\kern0pt}\ v\ {\isacharparenleft}{\kern0pt}{\isasymgamma}{\isacharprime}{\kern0pt}\ A{\isacharparenright}{\kern0pt}\ {\isacharequal}{\kern0pt}\ L\ A{\isacharparenright}{\kern0pt}\ {\isasymlongrightarrow}\ eval\ eq\ v\ {\isacharequal}{\kern0pt}\ subst{\isacharunderscore}{\kern0pt}lang\ P\ L\ A{\isacharparenright}{\kern0pt}{\isachardoublequoteclose}\isanewline
%
\isadelimproof
%
\endisadelimproof
%
\isatagproof
\isakeywordONE{proof}\isamarkupfalse%
\ {\isacharminus}{\kern0pt}\isanewline
\ \ \isakeywordONE{let}\isamarkupfalse%
\ {\isacharquery}{\kern0pt}Insts\ {\isacharequal}{\kern0pt}\ {\isachardoublequoteopen}rlexp{\isacharunderscore}{\kern0pt}syms\ {\isacharbackquote}{\kern0pt}\ {\isacharparenleft}{\kern0pt}Rhss\ P\ A{\isacharparenright}{\kern0pt}{\isachardoublequoteclose}\isanewline
\ \ \isakeywordONE{from}\isamarkupfalse%
\ finite{\isacharunderscore}{\kern0pt}Rhss{\isacharbrackleft}{\kern0pt}OF\ finite{\isacharunderscore}{\kern0pt}P{\isacharbrackright}{\kern0pt}\ \isakeywordONE{have}\isamarkupfalse%
\ {\isachardoublequoteopen}finite\ {\isacharquery}{\kern0pt}Insts{\isachardoublequoteclose}\ \isakeywordONE{by}\isamarkupfalse%
\ simp\isanewline
\ \ \isakeywordONE{moreover}\isamarkupfalse%
\ \isakeywordONE{from}\isamarkupfalse%
\ rlexp{\isacharunderscore}{\kern0pt}syms{\isacharunderscore}{\kern0pt}reg\ \isakeywordONE{have}\isamarkupfalse%
\ {\isachardoublequoteopen}{\isasymforall}f\ {\isasymin}\ {\isacharquery}{\kern0pt}Insts{\isachardot}{\kern0pt}\ reg{\isacharunderscore}{\kern0pt}eval\ f{\isachardoublequoteclose}\ \isakeywordONE{by}\isamarkupfalse%
\ blast\isanewline
\ \ \isakeywordONE{ultimately}\isamarkupfalse%
\ \isakeywordTHREE{obtain}\isamarkupfalse%
\ eq\ \isakeywordTWO{where}\ {\isacharasterisk}{\kern0pt}{\isacharcolon}{\kern0pt}\ {\isachardoublequoteopen}reg{\isacharunderscore}{\kern0pt}eval\ eq\ {\isasymand}\ {\isasymUnion}{\isacharparenleft}{\kern0pt}vars\ {\isacharbackquote}{\kern0pt}\ {\isacharquery}{\kern0pt}Insts{\isacharparenright}{\kern0pt}\ {\isacharequal}{\kern0pt}\ vars\ eq\isanewline
\ \ \ \ \ \ \ \ \ \ \ \ \ \ \ \ \ \ \ \ \ \ \ \ \ \ \ \ \ \ \ \ \ \ {\isasymand}\ {\isacharparenleft}{\kern0pt}{\isasymforall}v{\isachardot}{\kern0pt}\ {\isacharparenleft}{\kern0pt}{\isasymUnion}f\ {\isasymin}\ {\isacharquery}{\kern0pt}Insts{\isachardot}{\kern0pt}\ eval\ f\ v{\isacharparenright}{\kern0pt}\ {\isacharequal}{\kern0pt}\ eval\ eq\ v{\isacharparenright}{\kern0pt}{\isachardoublequoteclose}\isanewline
\ \ \ \ \isakeywordONE{using}\isamarkupfalse%
\ finite{\isacharunderscore}{\kern0pt}Union{\isacharunderscore}{\kern0pt}regular\ \isakeywordONE{by}\isamarkupfalse%
\ metis\isanewline
\ \ \isakeywordONE{moreover}\isamarkupfalse%
\ \isakeywordONE{have}\isamarkupfalse%
\ {\isachardoublequoteopen}vars\ eq\ {\isasymsubseteq}\ {\isasymgamma}{\isacharprime}{\kern0pt}\ {\isacharbackquote}{\kern0pt}\ Nts\ P{\isachardoublequoteclose}\isanewline
\ \ \isakeywordONE{proof}\isamarkupfalse%
\isanewline
\ \ \ \ \isakeywordTHREE{fix}\isamarkupfalse%
\ x\isanewline
\ \ \ \ \isakeywordTHREE{assume}\isamarkupfalse%
\ {\isachardoublequoteopen}x\ {\isasymin}\ vars\ eq{\isachardoublequoteclose}\isanewline
\ \ \ \ \isakeywordONE{with}\isamarkupfalse%
\ {\isacharasterisk}{\kern0pt}\ \isakeywordTHREE{obtain}\isamarkupfalse%
\ f\ \isakeywordTWO{where}\ {\isacharasterisk}{\kern0pt}{\isacharasterisk}{\kern0pt}{\isacharcolon}{\kern0pt}\ {\isachardoublequoteopen}f\ {\isasymin}\ {\isacharquery}{\kern0pt}Insts\ {\isasymand}\ x\ {\isasymin}\ vars\ f{\isachardoublequoteclose}\ \isakeywordONE{by}\isamarkupfalse%
\ blast\isanewline
\ \ \ \ \isakeywordONE{then}\isamarkupfalse%
\ \isakeywordTHREE{obtain}\isamarkupfalse%
\ w\ \isakeywordTWO{where}\ {\isacharasterisk}{\kern0pt}{\isacharasterisk}{\kern0pt}{\isacharasterisk}{\kern0pt}{\isacharcolon}{\kern0pt}\ {\isachardoublequoteopen}w\ {\isasymin}\ Rhss\ P\ A\ {\isasymand}\ f\ {\isacharequal}{\kern0pt}\ rlexp{\isacharunderscore}{\kern0pt}syms\ w{\isachardoublequoteclose}\ \isakeywordONE{by}\isamarkupfalse%
\ blast\isanewline
\ \ \ \ \isakeywordONE{with}\isamarkupfalse%
\ {\isacharasterisk}{\kern0pt}{\isacharasterisk}{\kern0pt}\ insts{\isacharprime}{\kern0pt}{\isacharunderscore}{\kern0pt}vars\ \isakeywordONE{have}\isamarkupfalse%
\ {\isachardoublequoteopen}x\ {\isasymin}\ {\isasymgamma}{\isacharprime}{\kern0pt}\ {\isacharbackquote}{\kern0pt}\ nts{\isacharunderscore}{\kern0pt}syms\ w{\isachardoublequoteclose}\ \isakeywordONE{by}\isamarkupfalse%
\ auto\isanewline
\ \ \ \ \isakeywordONE{with}\isamarkupfalse%
\ {\isacharasterisk}{\kern0pt}{\isacharasterisk}{\kern0pt}{\isacharasterisk}{\kern0pt}\ \isakeywordTHREE{show}\isamarkupfalse%
\ {\isachardoublequoteopen}x\ {\isasymin}\ {\isasymgamma}{\isacharprime}{\kern0pt}\ {\isacharbackquote}{\kern0pt}\ Nts\ P{\isachardoublequoteclose}\ \isakeywordONE{unfolding}\isamarkupfalse%
\ Nts{\isacharunderscore}{\kern0pt}def\ Rhss{\isacharunderscore}{\kern0pt}def\ \isakeywordONE{by}\isamarkupfalse%
\ blast\isanewline
\ \ \isakeywordONE{qed}\isamarkupfalse%
\isanewline
\ \ \isakeywordONE{moreover}\isamarkupfalse%
\ \isakeywordONE{have}\isamarkupfalse%
\ {\isachardoublequoteopen}{\isasymforall}v\ L{\isachardot}{\kern0pt}\ {\isacharparenleft}{\kern0pt}{\isasymforall}A\ {\isasymin}\ Nts\ P{\isachardot}{\kern0pt}\ v\ {\isacharparenleft}{\kern0pt}{\isasymgamma}{\isacharprime}{\kern0pt}\ A{\isacharparenright}{\kern0pt}\ {\isacharequal}{\kern0pt}\ L\ A{\isacharparenright}{\kern0pt}\ {\isasymlongrightarrow}\ eval\ eq\ v\ {\isacharequal}{\kern0pt}\ subst{\isacharunderscore}{\kern0pt}lang\ P\ L\ A{\isachardoublequoteclose}\isanewline
\ \ \isakeywordONE{proof}\isamarkupfalse%
\ {\isacharparenleft}{\kern0pt}rule\ allI\ {\isacharbar}{\kern0pt}\ rule\ impI{\isacharparenright}{\kern0pt}{\isacharplus}{\kern0pt}\isanewline
\ \ \ \ \isakeywordTHREE{fix}\isamarkupfalse%
\ v\ {\isacharcolon}{\kern0pt}{\isacharcolon}{\kern0pt}\ {\isachardoublequoteopen}nat\ {\isasymRightarrow}\ {\isacharprime}{\kern0pt}a\ lang{\isachardoublequoteclose}\ \isakeywordTWO{and}\ L\ {\isacharcolon}{\kern0pt}{\isacharcolon}{\kern0pt}\ {\isachardoublequoteopen}{\isacharprime}{\kern0pt}n\ {\isasymRightarrow}\ {\isacharprime}{\kern0pt}a\ lang{\isachardoublequoteclose}\isanewline
\ \ \ \ \isakeywordTHREE{assume}\isamarkupfalse%
\ state{\isacharunderscore}{\kern0pt}L{\isacharcolon}{\kern0pt}\ {\isachardoublequoteopen}{\isasymforall}A\ {\isasymin}\ Nts\ P{\isachardot}{\kern0pt}\ v\ {\isacharparenleft}{\kern0pt}{\isasymgamma}{\isacharprime}{\kern0pt}\ A{\isacharparenright}{\kern0pt}\ {\isacharequal}{\kern0pt}\ L\ A{\isachardoublequoteclose}\isanewline
\ \ \ \ \isakeywordONE{have}\isamarkupfalse%
\ {\isachardoublequoteopen}{\isasymforall}w\ {\isasymin}\ Rhss\ P\ A{\isachardot}{\kern0pt}\ eval\ {\isacharparenleft}{\kern0pt}rlexp{\isacharunderscore}{\kern0pt}syms\ w{\isacharparenright}{\kern0pt}\ v\ {\isacharequal}{\kern0pt}\ inst{\isacharunderscore}{\kern0pt}syms\ L\ w{\isachardoublequoteclose}\isanewline
\ \ \ \ \isakeywordONE{proof}\isamarkupfalse%
\isanewline
\ \ \ \ \ \ \isakeywordTHREE{fix}\isamarkupfalse%
\ w\isanewline
\ \ \ \ \ \ \isakeywordTHREE{assume}\isamarkupfalse%
\ {\isachardoublequoteopen}w\ {\isasymin}\ Rhss\ P\ A{\isachardoublequoteclose}\isanewline
\ \ \ \ \ \ \isakeywordONE{with}\isamarkupfalse%
\ state{\isacharunderscore}{\kern0pt}L\ Nts{\isacharunderscore}{\kern0pt}nts{\isacharunderscore}{\kern0pt}syms\ \isakeywordONE{have}\isamarkupfalse%
\ {\isachardoublequoteopen}{\isasymforall}A\ {\isasymin}\ nts{\isacharunderscore}{\kern0pt}syms\ w{\isachardot}{\kern0pt}\ v\ {\isacharparenleft}{\kern0pt}{\isasymgamma}{\isacharprime}{\kern0pt}\ A{\isacharparenright}{\kern0pt}\ {\isacharequal}{\kern0pt}\ L\ A{\isachardoublequoteclose}\ \isakeywordONE{by}\isamarkupfalse%
\ fast\isanewline
\ \ \ \ \ \ \isakeywordONE{from}\isamarkupfalse%
\ rlexp{\isacharunderscore}{\kern0pt}syms{\isacharunderscore}{\kern0pt}insts{\isacharbrackleft}{\kern0pt}OF\ this{\isacharbrackright}{\kern0pt}\ \isakeywordTHREE{show}\isamarkupfalse%
\ {\isachardoublequoteopen}eval\ {\isacharparenleft}{\kern0pt}rlexp{\isacharunderscore}{\kern0pt}syms\ w{\isacharparenright}{\kern0pt}\ v\ {\isacharequal}{\kern0pt}\ inst{\isacharunderscore}{\kern0pt}syms\ L\ w{\isachardoublequoteclose}\ \isakeywordONE{by}\isamarkupfalse%
\ blast\isanewline
\ \ \ \ \isakeywordONE{qed}\isamarkupfalse%
\isanewline
\ \ \ \ \isakeywordONE{then}\isamarkupfalse%
\ \isakeywordONE{have}\isamarkupfalse%
\ {\isachardoublequoteopen}subst{\isacharunderscore}{\kern0pt}lang\ P\ L\ A\ {\isacharequal}{\kern0pt}\ {\isacharparenleft}{\kern0pt}{\isasymUnion}f\ {\isasymin}\ {\isacharquery}{\kern0pt}Insts{\isachardot}{\kern0pt}\ eval\ f\ v{\isacharparenright}{\kern0pt}{\isachardoublequoteclose}\ \isakeywordONE{unfolding}\isamarkupfalse%
\ subst{\isacharunderscore}{\kern0pt}lang{\isacharunderscore}{\kern0pt}def\ \isakeywordONE{by}\isamarkupfalse%
\ auto\isanewline
\ \ \ \ \isakeywordONE{with}\isamarkupfalse%
\ {\isacharasterisk}{\kern0pt}\ \isakeywordTHREE{show}\isamarkupfalse%
\ {\isachardoublequoteopen}eval\ eq\ v\ {\isacharequal}{\kern0pt}\ subst{\isacharunderscore}{\kern0pt}lang\ P\ L\ A{\isachardoublequoteclose}\ \isakeywordONE{by}\isamarkupfalse%
\ auto\isanewline
\ \ \isakeywordONE{qed}\isamarkupfalse%
\isanewline
\ \ \isakeywordONE{ultimately}\isamarkupfalse%
\ \isakeywordTHREE{show}\isamarkupfalse%
\ {\isacharquery}{\kern0pt}thesis\ \isakeywordONE{by}\isamarkupfalse%
\ auto\isanewline
\isakeywordONE{qed}\isamarkupfalse%
%
\endisatagproof
{\isafoldproof}%
%
\isadelimproof
%
\endisadelimproof
%
\begin{isamarkuptext}%
The whole CFG induces a system of equations. We first define which conditions this system
should fulfill and show its existence in the second step:%
\end{isamarkuptext}\isamarkuptrue%
\isakeywordONE{abbreviation}\isamarkupfalse%
\ {\isachardoublequoteopen}CFG{\isacharunderscore}{\kern0pt}sys\ sys\ {\isasymequiv}\isanewline
\ \ length\ sys\ {\isacharequal}{\kern0pt}\ card\ {\isacharparenleft}{\kern0pt}Nts\ P{\isacharparenright}{\kern0pt}\ {\isasymand}\isanewline
\ \ \ \ {\isacharparenleft}{\kern0pt}{\isasymforall}i\ {\isacharless}{\kern0pt}\ card\ {\isacharparenleft}{\kern0pt}Nts\ P{\isacharparenright}{\kern0pt}{\isachardot}{\kern0pt}\ reg{\isacharunderscore}{\kern0pt}eval\ {\isacharparenleft}{\kern0pt}sys\ {\isacharbang}{\kern0pt}\ i{\isacharparenright}{\kern0pt}\ {\isasymand}\ {\isacharparenleft}{\kern0pt}{\isasymforall}x\ {\isasymin}\ vars\ {\isacharparenleft}{\kern0pt}sys\ {\isacharbang}{\kern0pt}\ i{\isacharparenright}{\kern0pt}{\isachardot}{\kern0pt}\ x\ {\isacharless}{\kern0pt}\ card\ {\isacharparenleft}{\kern0pt}Nts\ P{\isacharparenright}{\kern0pt}{\isacharparenright}{\kern0pt}\isanewline
\ \ \ \ \ \ \ \ \ \ \ \ \ \ \ \ \ \ \ \ \ \ \ \ {\isasymand}\ {\isacharparenleft}{\kern0pt}{\isasymforall}s\ L{\isachardot}{\kern0pt}\ {\isacharparenleft}{\kern0pt}{\isasymforall}A\ {\isasymin}\ Nts\ P{\isachardot}{\kern0pt}\ s\ {\isacharparenleft}{\kern0pt}{\isasymgamma}{\isacharprime}{\kern0pt}\ A{\isacharparenright}{\kern0pt}\ {\isacharequal}{\kern0pt}\ L\ A{\isacharparenright}{\kern0pt}\isanewline
\ \ \ \ \ \ \ \ \ \ \ \ \ \ \ \ \ \ \ \ \ \ \ \ \ \ \ \ {\isasymlongrightarrow}\ eval\ {\isacharparenleft}{\kern0pt}sys\ {\isacharbang}{\kern0pt}\ i{\isacharparenright}{\kern0pt}\ s\ {\isacharequal}{\kern0pt}\ subst{\isacharunderscore}{\kern0pt}lang\ P\ L\ {\isacharparenleft}{\kern0pt}{\isasymgamma}\ i{\isacharparenright}{\kern0pt}{\isacharparenright}{\kern0pt}{\isacharparenright}{\kern0pt}{\isachardoublequoteclose}\isanewline
\isanewline
\isakeywordONE{lemma}\isamarkupfalse%
\ CFG{\isacharunderscore}{\kern0pt}as{\isacharunderscore}{\kern0pt}eq{\isacharunderscore}{\kern0pt}sys{\isacharcolon}{\kern0pt}\ {\isachardoublequoteopen}{\isasymexists}sys{\isachardot}{\kern0pt}\ CFG{\isacharunderscore}{\kern0pt}sys\ sys{\isachardoublequoteclose}\isanewline
%
\isadelimproof
%
\endisadelimproof
%
\isatagproof
\isakeywordONE{proof}\isamarkupfalse%
\ {\isacharminus}{\kern0pt}\isanewline
\ \ \isakeywordONE{from}\isamarkupfalse%
\ bij{\isacharunderscore}{\kern0pt}{\isasymgamma}{\isacharunderscore}{\kern0pt}{\isasymgamma}{\isacharprime}{\kern0pt}\ \isakeywordONE{have}\isamarkupfalse%
\ {\isacharasterisk}{\kern0pt}{\isacharcolon}{\kern0pt}\ {\isachardoublequoteopen}{\isasymAnd}eq{\isachardot}{\kern0pt}\ vars\ eq\ {\isasymsubseteq}\ {\isasymgamma}{\isacharprime}{\kern0pt}\ {\isacharbackquote}{\kern0pt}\ Nts\ P\ {\isasymLongrightarrow}\ {\isasymforall}x\ {\isasymin}\ vars\ eq{\isachardot}{\kern0pt}\ x\ {\isacharless}{\kern0pt}\ card\ {\isacharparenleft}{\kern0pt}Nts\ P{\isacharparenright}{\kern0pt}{\isachardoublequoteclose}\isanewline
\ \ \ \ \isakeywordONE{unfolding}\isamarkupfalse%
\ bij{\isacharunderscore}{\kern0pt}Nt{\isacharunderscore}{\kern0pt}Var{\isacharunderscore}{\kern0pt}def\ bij{\isacharunderscore}{\kern0pt}betw{\isacharunderscore}{\kern0pt}def\ \isakeywordONE{by}\isamarkupfalse%
\ auto\isanewline
\ \ \isakeywordONE{from}\isamarkupfalse%
\ subst{\isacharunderscore}{\kern0pt}lang{\isacharunderscore}{\kern0pt}rlexp\ \isakeywordONE{have}\isamarkupfalse%
\ {\isachardoublequoteopen}{\isasymforall}A{\isachardot}{\kern0pt}\ {\isasymexists}eq{\isachardot}{\kern0pt}\ reg{\isacharunderscore}{\kern0pt}eval\ eq\ {\isasymand}\ vars\ eq\ {\isasymsubseteq}\ {\isasymgamma}{\isacharprime}{\kern0pt}\ {\isacharbackquote}{\kern0pt}\ Nts\ P\ {\isasymand}\isanewline
\ \ \ \ \ \ \ \ \ \ \ \ \ \ \ \ \ \ \ \ \ \ \ \ \ \ \ \ \ {\isacharparenleft}{\kern0pt}{\isasymforall}s\ L{\isachardot}{\kern0pt}\ {\isacharparenleft}{\kern0pt}{\isasymforall}A\ {\isasymin}\ Nts\ P{\isachardot}{\kern0pt}\ s\ {\isacharparenleft}{\kern0pt}{\isasymgamma}{\isacharprime}{\kern0pt}\ A{\isacharparenright}{\kern0pt}\ {\isacharequal}{\kern0pt}\ L\ A{\isacharparenright}{\kern0pt}\ {\isasymlongrightarrow}\ eval\ eq\ s\ {\isacharequal}{\kern0pt}\ subst{\isacharunderscore}{\kern0pt}lang\ P\ L\ A{\isacharparenright}{\kern0pt}{\isachardoublequoteclose}\isanewline
\ \ \ \ \isakeywordONE{by}\isamarkupfalse%
\ blast\isanewline
\ \ \isakeywordONE{with}\isamarkupfalse%
\ bij{\isacharunderscore}{\kern0pt}{\isasymgamma}{\isacharunderscore}{\kern0pt}{\isasymgamma}{\isacharprime}{\kern0pt}\ {\isacharasterisk}{\kern0pt}\ \isakeywordONE{have}\isamarkupfalse%
\ {\isachardoublequoteopen}{\isasymforall}i\ {\isacharless}{\kern0pt}\ card\ {\isacharparenleft}{\kern0pt}Nts\ P{\isacharparenright}{\kern0pt}{\isachardot}{\kern0pt}\ {\isasymexists}eq{\isachardot}{\kern0pt}\ reg{\isacharunderscore}{\kern0pt}eval\ eq\ {\isasymand}\ {\isacharparenleft}{\kern0pt}{\isasymforall}x\ {\isasymin}\ vars\ eq{\isachardot}{\kern0pt}\ x\ {\isacharless}{\kern0pt}\ card\ {\isacharparenleft}{\kern0pt}Nts\ P{\isacharparenright}{\kern0pt}{\isacharparenright}{\kern0pt}\isanewline
\ \ \ \ \ \ \ \ \ \ \ \ \ \ \ \ \ \ \ \ {\isasymand}\ {\isacharparenleft}{\kern0pt}{\isasymforall}s\ L{\isachardot}{\kern0pt}\ {\isacharparenleft}{\kern0pt}{\isasymforall}A\ {\isasymin}\ Nts\ P{\isachardot}{\kern0pt}\ s\ {\isacharparenleft}{\kern0pt}{\isasymgamma}{\isacharprime}{\kern0pt}\ A{\isacharparenright}{\kern0pt}\ {\isacharequal}{\kern0pt}\ L\ A{\isacharparenright}{\kern0pt}\ {\isasymlongrightarrow}\ eval\ eq\ s\ {\isacharequal}{\kern0pt}\ subst{\isacharunderscore}{\kern0pt}lang\ P\ L\ {\isacharparenleft}{\kern0pt}{\isasymgamma}\ i{\isacharparenright}{\kern0pt}{\isacharparenright}{\kern0pt}{\isachardoublequoteclose}\isanewline
\ \ \ \ \isakeywordONE{unfolding}\isamarkupfalse%
\ bij{\isacharunderscore}{\kern0pt}Nt{\isacharunderscore}{\kern0pt}Var{\isacharunderscore}{\kern0pt}def\ \isakeywordONE{by}\isamarkupfalse%
\ metis\isanewline
\ \ \isakeywordONE{with}\isamarkupfalse%
\ Skolem{\isacharunderscore}{\kern0pt}list{\isacharunderscore}{\kern0pt}nth{\isacharbrackleft}{\kern0pt}\isakeywordTWO{where}\ P{\isacharequal}{\kern0pt}{\isachardoublequoteopen}{\isasymlambda}i\ eq{\isachardot}{\kern0pt}\ reg{\isacharunderscore}{\kern0pt}eval\ eq\ {\isasymand}\ {\isacharparenleft}{\kern0pt}{\isasymforall}x\ {\isasymin}\ vars\ eq{\isachardot}{\kern0pt}\ x\ {\isacharless}{\kern0pt}\ card\ {\isacharparenleft}{\kern0pt}Nts\ P{\isacharparenright}{\kern0pt}{\isacharparenright}{\kern0pt}\isanewline
\ \ \ \ \ \ \ \ \ \ \ \ \ \ \ \ \ \ \ \ \ \ \ {\isasymand}\ {\isacharparenleft}{\kern0pt}{\isasymforall}s\ L{\isachardot}{\kern0pt}\ {\isacharparenleft}{\kern0pt}{\isasymforall}A\ {\isasymin}\ Nts\ P{\isachardot}{\kern0pt}\ s\ {\isacharparenleft}{\kern0pt}{\isasymgamma}{\isacharprime}{\kern0pt}\ A{\isacharparenright}{\kern0pt}\ {\isacharequal}{\kern0pt}\ L\ A{\isacharparenright}{\kern0pt}\ {\isasymlongrightarrow}\ eval\ eq\ s\ {\isacharequal}{\kern0pt}\ subst{\isacharunderscore}{\kern0pt}lang\ P\ L\ {\isacharparenleft}{\kern0pt}{\isasymgamma}\ i{\isacharparenright}{\kern0pt}{\isacharparenright}{\kern0pt}{\isachardoublequoteclose}{\isacharbrackright}{\kern0pt}\isanewline
\ \ \ \ \isakeywordTHREE{show}\isamarkupfalse%
\ {\isacharquery}{\kern0pt}thesis\ \isakeywordONE{by}\isamarkupfalse%
\ blast\isanewline
\isakeywordONE{qed}\isamarkupfalse%
%
\endisatagproof
{\isafoldproof}%
%
\isadelimproof
%
\endisadelimproof
%
\begin{isamarkuptext}%
As we have proved that each CFG induces a system of equations, it remains to show that the
 CFG's language is a minimal solution of this system. The first lemma proves that the CFG's language
is a solution and the next two lemmas prove that it is minimal:%
\end{isamarkuptext}\isamarkuptrue%
\isakeywordONE{abbreviation}\isamarkupfalse%
\ {\isachardoublequoteopen}sol\ {\isasymequiv}\ {\isasymlambda}i{\isachardot}{\kern0pt}\ if\ i\ {\isacharless}{\kern0pt}\ card\ {\isacharparenleft}{\kern0pt}Nts\ P{\isacharparenright}{\kern0pt}\ then\ Lang{\isacharunderscore}{\kern0pt}lfp\ P\ {\isacharparenleft}{\kern0pt}{\isasymgamma}\ i{\isacharparenright}{\kern0pt}\ else\ {\isacharbraceleft}{\kern0pt}{\isacharbraceright}{\kern0pt}{\isachardoublequoteclose}\isanewline
\isanewline
\isakeywordONE{lemma}\isamarkupfalse%
\ CFG{\isacharunderscore}{\kern0pt}sys{\isacharunderscore}{\kern0pt}CFL{\isacharunderscore}{\kern0pt}is{\isacharunderscore}{\kern0pt}sol{\isacharcolon}{\kern0pt}\isanewline
\ \ \isakeywordTWO{assumes}\ {\isachardoublequoteopen}CFG{\isacharunderscore}{\kern0pt}sys\ sys{\isachardoublequoteclose}\isanewline
\ \ \isakeywordTWO{shows}\ {\isachardoublequoteopen}solves{\isacharunderscore}{\kern0pt}ineq{\isacharunderscore}{\kern0pt}sys\ sys\ sol{\isachardoublequoteclose}\isanewline
%
\isadelimproof
%
\endisadelimproof
%
\isatagproof
\isakeywordONE{unfolding}\isamarkupfalse%
\ solves{\isacharunderscore}{\kern0pt}ineq{\isacharunderscore}{\kern0pt}sys{\isacharunderscore}{\kern0pt}def\ \isakeywordONE{proof}\isamarkupfalse%
\ {\isacharparenleft}{\kern0pt}rule\ allI{\isacharcomma}{\kern0pt}\ rule\ impI{\isacharparenright}{\kern0pt}\isanewline
\ \ \isakeywordTHREE{fix}\isamarkupfalse%
\ i\isanewline
\ \ \isakeywordTHREE{assume}\isamarkupfalse%
\ {\isachardoublequoteopen}i\ {\isacharless}{\kern0pt}\ length\ sys{\isachardoublequoteclose}\isanewline
\ \ \isakeywordONE{with}\isamarkupfalse%
\ assms\ \isakeywordONE{have}\isamarkupfalse%
\ {\isachardoublequoteopen}i\ {\isacharless}{\kern0pt}\ card\ {\isacharparenleft}{\kern0pt}Nts\ P{\isacharparenright}{\kern0pt}{\isachardoublequoteclose}\ \isakeywordONE{by}\isamarkupfalse%
\ argo\isanewline
\ \ \isakeywordONE{from}\isamarkupfalse%
\ bij{\isacharunderscore}{\kern0pt}{\isasymgamma}{\isacharunderscore}{\kern0pt}{\isasymgamma}{\isacharprime}{\kern0pt}\ \isakeywordONE{have}\isamarkupfalse%
\ {\isacharasterisk}{\kern0pt}{\isacharcolon}{\kern0pt}\ {\isachardoublequoteopen}{\isasymforall}A\ {\isasymin}\ Nts\ P{\isachardot}{\kern0pt}\ sol\ {\isacharparenleft}{\kern0pt}{\isasymgamma}{\isacharprime}{\kern0pt}\ A{\isacharparenright}{\kern0pt}\ {\isacharequal}{\kern0pt}\ Lang{\isacharunderscore}{\kern0pt}lfp\ P\ A{\isachardoublequoteclose}\isanewline
\ \ \ \ \isakeywordONE{unfolding}\isamarkupfalse%
\ bij{\isacharunderscore}{\kern0pt}Nt{\isacharunderscore}{\kern0pt}Var{\isacharunderscore}{\kern0pt}def\ bij{\isacharunderscore}{\kern0pt}betw{\isacharunderscore}{\kern0pt}def\ \isakeywordONE{by}\isamarkupfalse%
\ force\isanewline
\ \ \isakeywordONE{with}\isamarkupfalse%
\ {\isacartoucheopen}i\ {\isacharless}{\kern0pt}\ card\ {\isacharparenleft}{\kern0pt}Nts\ P{\isacharparenright}{\kern0pt}{\isacartoucheclose}\ assms\ \isakeywordONE{have}\isamarkupfalse%
\ {\isachardoublequoteopen}eval\ {\isacharparenleft}{\kern0pt}sys\ {\isacharbang}{\kern0pt}\ i{\isacharparenright}{\kern0pt}\ sol\ {\isacharequal}{\kern0pt}\ subst{\isacharunderscore}{\kern0pt}lang\ P\ {\isacharparenleft}{\kern0pt}Lang{\isacharunderscore}{\kern0pt}lfp\ P{\isacharparenright}{\kern0pt}\ {\isacharparenleft}{\kern0pt}{\isasymgamma}\ i{\isacharparenright}{\kern0pt}{\isachardoublequoteclose}\isanewline
\ \ \ \ \isakeywordONE{by}\isamarkupfalse%
\ presburger\isanewline
\ \ \isakeywordONE{with}\isamarkupfalse%
\ lfp{\isacharunderscore}{\kern0pt}fixpoint{\isacharbrackleft}{\kern0pt}OF\ mono{\isacharunderscore}{\kern0pt}if{\isacharunderscore}{\kern0pt}omega{\isacharunderscore}{\kern0pt}cont{\isacharbrackleft}{\kern0pt}OF\ omega{\isacharunderscore}{\kern0pt}cont{\isacharunderscore}{\kern0pt}Lang{\isacharunderscore}{\kern0pt}lfp{\isacharbrackright}{\kern0pt}{\isacharbrackright}{\kern0pt}\ \isakeywordONE{have}\isamarkupfalse%
\ {\isadigit{1}}{\isacharcolon}{\kern0pt}\ {\isachardoublequoteopen}eval\ {\isacharparenleft}{\kern0pt}sys\ {\isacharbang}{\kern0pt}\ i{\isacharparenright}{\kern0pt}\ sol\ {\isacharequal}{\kern0pt}\ Lang{\isacharunderscore}{\kern0pt}lfp\ P\ {\isacharparenleft}{\kern0pt}{\isasymgamma}\ i{\isacharparenright}{\kern0pt}{\isachardoublequoteclose}\isanewline
\ \ \ \ \isakeywordONE{unfolding}\isamarkupfalse%
\ Lang{\isacharunderscore}{\kern0pt}lfp{\isacharunderscore}{\kern0pt}def\ \isakeywordONE{by}\isamarkupfalse%
\ metis\isanewline
\ \ \isakeywordONE{from}\isamarkupfalse%
\ {\isacartoucheopen}i\ {\isacharless}{\kern0pt}\ card\ {\isacharparenleft}{\kern0pt}Nts\ P{\isacharparenright}{\kern0pt}{\isacartoucheclose}\ bij{\isacharunderscore}{\kern0pt}{\isasymgamma}{\isacharunderscore}{\kern0pt}{\isasymgamma}{\isacharprime}{\kern0pt}\ \isakeywordONE{have}\isamarkupfalse%
\ {\isachardoublequoteopen}{\isasymgamma}\ i\ {\isasymin}\ Nts\ P{\isachardoublequoteclose}\isanewline
\ \ \ \ \isakeywordONE{unfolding}\isamarkupfalse%
\ bij{\isacharunderscore}{\kern0pt}Nt{\isacharunderscore}{\kern0pt}Var{\isacharunderscore}{\kern0pt}def\ \isakeywordONE{using}\isamarkupfalse%
\ bij{\isacharunderscore}{\kern0pt}betwE\ \isakeywordONE{by}\isamarkupfalse%
\ blast\isanewline
\ \ \isakeywordONE{with}\isamarkupfalse%
\ {\isacharasterisk}{\kern0pt}\ \isakeywordONE{have}\isamarkupfalse%
\ {\isachardoublequoteopen}Lang{\isacharunderscore}{\kern0pt}lfp\ P\ {\isacharparenleft}{\kern0pt}{\isasymgamma}\ i{\isacharparenright}{\kern0pt}\ {\isacharequal}{\kern0pt}\ sol\ {\isacharparenleft}{\kern0pt}{\isasymgamma}{\isacharprime}{\kern0pt}\ {\isacharparenleft}{\kern0pt}{\isasymgamma}\ i{\isacharparenright}{\kern0pt}{\isacharparenright}{\kern0pt}{\isachardoublequoteclose}\ \isakeywordONE{by}\isamarkupfalse%
\ auto\isanewline
\ \ \isakeywordONE{also}\isamarkupfalse%
\ \isakeywordONE{have}\isamarkupfalse%
\ {\isachardoublequoteopen}{\isasymdots}\ {\isacharequal}{\kern0pt}\ sol\ i{\isachardoublequoteclose}\ \isakeywordONE{using}\isamarkupfalse%
\ bij{\isacharunderscore}{\kern0pt}{\isasymgamma}{\isacharunderscore}{\kern0pt}{\isasymgamma}{\isacharprime}{\kern0pt}\ {\isacartoucheopen}i\ {\isacharless}{\kern0pt}\ card\ {\isacharparenleft}{\kern0pt}Nts\ P{\isacharparenright}{\kern0pt}{\isacartoucheclose}\ \isakeywordONE{unfolding}\isamarkupfalse%
\ bij{\isacharunderscore}{\kern0pt}Nt{\isacharunderscore}{\kern0pt}Var{\isacharunderscore}{\kern0pt}def\ \isakeywordONE{by}\isamarkupfalse%
\ auto\isanewline
\ \ \isakeywordONE{finally}\isamarkupfalse%
\ \isakeywordTHREE{show}\isamarkupfalse%
\ {\isachardoublequoteopen}eval\ {\isacharparenleft}{\kern0pt}sys\ {\isacharbang}{\kern0pt}\ i{\isacharparenright}{\kern0pt}\ sol\ {\isasymsubseteq}\ sol\ i{\isachardoublequoteclose}\ \isakeywordONE{using}\isamarkupfalse%
\ {\isadigit{1}}\ \isakeywordONE{by}\isamarkupfalse%
\ blast\isanewline
\isakeywordONE{qed}\isamarkupfalse%
%
\endisatagproof
{\isafoldproof}%
%
\isadelimproof
\isanewline
%
\endisadelimproof
\isanewline
\isakeywordONE{lemma}\isamarkupfalse%
\ CFG{\isacharunderscore}{\kern0pt}sys{\isacharunderscore}{\kern0pt}CFL{\isacharunderscore}{\kern0pt}is{\isacharunderscore}{\kern0pt}min{\isacharunderscore}{\kern0pt}aux{\isacharcolon}{\kern0pt}\isanewline
\ \ \isakeywordTWO{assumes}\ {\isachardoublequoteopen}CFG{\isacharunderscore}{\kern0pt}sys\ sys{\isachardoublequoteclose}\isanewline
\ \ \ \ \ \ \isakeywordTWO{and}\ {\isachardoublequoteopen}solves{\isacharunderscore}{\kern0pt}ineq{\isacharunderscore}{\kern0pt}sys\ sys\ sol{\isacharprime}{\kern0pt}{\isachardoublequoteclose}\isanewline
\ \ \ \ \isakeywordTWO{shows}\ {\isachardoublequoteopen}Lang{\isacharunderscore}{\kern0pt}lfp\ P\ {\isasymle}\ {\isacharparenleft}{\kern0pt}{\isasymlambda}A{\isachardot}{\kern0pt}\ sol{\isacharprime}{\kern0pt}\ {\isacharparenleft}{\kern0pt}{\isasymgamma}{\isacharprime}{\kern0pt}\ A{\isacharparenright}{\kern0pt}{\isacharparenright}{\kern0pt}{\isachardoublequoteclose}\ {\isacharparenleft}{\kern0pt}\isakeywordTWO{is}\ {\isachardoublequoteopen}{\isacharunderscore}{\kern0pt}\ {\isasymle}\ {\isacharquery}{\kern0pt}L{\isacharprime}{\kern0pt}{\isachardoublequoteclose}{\isacharparenright}{\kern0pt}\isanewline
%
\isadelimproof
%
\endisadelimproof
%
\isatagproof
\isakeywordONE{proof}\isamarkupfalse%
\ {\isacharminus}{\kern0pt}\isanewline
\ \ \isakeywordONE{have}\isamarkupfalse%
\ {\isachardoublequoteopen}subst{\isacharunderscore}{\kern0pt}lang\ P\ {\isacharquery}{\kern0pt}L{\isacharprime}{\kern0pt}\ A\ {\isasymsubseteq}\ {\isacharquery}{\kern0pt}L{\isacharprime}{\kern0pt}\ A{\isachardoublequoteclose}\ \isakeywordTWO{for}\ A\isanewline
\ \ \isakeywordONE{proof}\isamarkupfalse%
\ {\isacharparenleft}{\kern0pt}cases\ {\isachardoublequoteopen}A\ {\isasymin}\ Nts\ P{\isachardoublequoteclose}{\isacharparenright}{\kern0pt}\isanewline
\ \ \ \ \isakeywordTHREE{case}\isamarkupfalse%
\ True\isanewline
\ \ \ \ \isakeywordONE{with}\isamarkupfalse%
\ assms{\isacharparenleft}{\kern0pt}{\isadigit{1}}{\isacharparenright}{\kern0pt}\ bij{\isacharunderscore}{\kern0pt}{\isasymgamma}{\isacharunderscore}{\kern0pt}{\isasymgamma}{\isacharprime}{\kern0pt}\ \isakeywordONE{have}\isamarkupfalse%
\ {\isachardoublequoteopen}{\isasymgamma}{\isacharprime}{\kern0pt}\ A\ {\isacharless}{\kern0pt}\ length\ sys{\isachardoublequoteclose}\isanewline
\ \ \ \ \ \ \isakeywordONE{unfolding}\isamarkupfalse%
\ bij{\isacharunderscore}{\kern0pt}Nt{\isacharunderscore}{\kern0pt}Var{\isacharunderscore}{\kern0pt}def\ bij{\isacharunderscore}{\kern0pt}betw{\isacharunderscore}{\kern0pt}def\ \isakeywordONE{by}\isamarkupfalse%
\ fastforce\isanewline
\ \ \ \ \isakeywordONE{with}\isamarkupfalse%
\ assms{\isacharparenleft}{\kern0pt}{\isadigit{1}}{\isacharparenright}{\kern0pt}\ bij{\isacharunderscore}{\kern0pt}{\isasymgamma}{\isacharunderscore}{\kern0pt}{\isasymgamma}{\isacharprime}{\kern0pt}\ True\ \isakeywordONE{have}\isamarkupfalse%
\ {\isachardoublequoteopen}subst{\isacharunderscore}{\kern0pt}lang\ P\ {\isacharquery}{\kern0pt}L{\isacharprime}{\kern0pt}\ A\ {\isacharequal}{\kern0pt}\ eval\ {\isacharparenleft}{\kern0pt}sys\ {\isacharbang}{\kern0pt}\ {\isasymgamma}{\isacharprime}{\kern0pt}\ A{\isacharparenright}{\kern0pt}\ sol{\isacharprime}{\kern0pt}{\isachardoublequoteclose}\isanewline
\ \ \ \ \ \ \isakeywordONE{unfolding}\isamarkupfalse%
\ bij{\isacharunderscore}{\kern0pt}Nt{\isacharunderscore}{\kern0pt}Var{\isacharunderscore}{\kern0pt}def\ \isakeywordONE{by}\isamarkupfalse%
\ metis\isanewline
\ \ \ \ \isakeywordONE{also}\isamarkupfalse%
\ \isakeywordONE{from}\isamarkupfalse%
\ True\ assms{\isacharparenleft}{\kern0pt}{\isadigit{2}}{\isacharparenright}{\kern0pt}\ {\isacartoucheopen}{\isasymgamma}{\isacharprime}{\kern0pt}\ A\ {\isacharless}{\kern0pt}\ length\ sys{\isacartoucheclose}\ bij{\isacharunderscore}{\kern0pt}{\isasymgamma}{\isacharunderscore}{\kern0pt}{\isasymgamma}{\isacharprime}{\kern0pt}\ \isakeywordONE{have}\isamarkupfalse%
\ {\isachardoublequoteopen}{\isasymdots}\ {\isasymsubseteq}\ {\isacharquery}{\kern0pt}L{\isacharprime}{\kern0pt}\ A{\isachardoublequoteclose}\isanewline
\ \ \ \ \ \ \isakeywordONE{unfolding}\isamarkupfalse%
\ solves{\isacharunderscore}{\kern0pt}ineq{\isacharunderscore}{\kern0pt}sys{\isacharunderscore}{\kern0pt}def\ bij{\isacharunderscore}{\kern0pt}Nt{\isacharunderscore}{\kern0pt}Var{\isacharunderscore}{\kern0pt}def\ \isakeywordONE{by}\isamarkupfalse%
\ blast\isanewline
\ \ \ \ \isakeywordONE{finally}\isamarkupfalse%
\ \isakeywordTHREE{show}\isamarkupfalse%
\ {\isacharquery}{\kern0pt}thesis\ \isakeywordONE{{\isachardot}{\kern0pt}}\isamarkupfalse%
\isanewline
\ \ \isakeywordONE{next}\isamarkupfalse%
\isanewline
\ \ \ \ \isakeywordTHREE{case}\isamarkupfalse%
\ False\isanewline
\ \ \ \ \isakeywordONE{then}\isamarkupfalse%
\ \isakeywordONE{have}\isamarkupfalse%
\ {\isachardoublequoteopen}Rhss\ P\ A\ {\isacharequal}{\kern0pt}\ {\isacharbraceleft}{\kern0pt}{\isacharbraceright}{\kern0pt}{\isachardoublequoteclose}\ \isakeywordONE{unfolding}\isamarkupfalse%
\ Nts{\isacharunderscore}{\kern0pt}def\ Rhss{\isacharunderscore}{\kern0pt}def\ \isakeywordONE{by}\isamarkupfalse%
\ blast\isanewline
\ \ \ \ \isakeywordONE{with}\isamarkupfalse%
\ False\ \isakeywordTHREE{show}\isamarkupfalse%
\ {\isacharquery}{\kern0pt}thesis\ \isakeywordONE{unfolding}\isamarkupfalse%
\ subst{\isacharunderscore}{\kern0pt}lang{\isacharunderscore}{\kern0pt}def\ \isakeywordONE{by}\isamarkupfalse%
\ simp\isanewline
\ \ \isakeywordONE{qed}\isamarkupfalse%
\isanewline
\ \ \isakeywordONE{then}\isamarkupfalse%
\ \isakeywordONE{have}\isamarkupfalse%
\ {\isachardoublequoteopen}subst{\isacharunderscore}{\kern0pt}lang\ P\ {\isacharquery}{\kern0pt}L{\isacharprime}{\kern0pt}\ {\isasymle}\ {\isacharquery}{\kern0pt}L{\isacharprime}{\kern0pt}{\isachardoublequoteclose}\ \isakeywordONE{by}\isamarkupfalse%
\ {\isacharparenleft}{\kern0pt}simp\ add{\isacharcolon}{\kern0pt}\ le{\isacharunderscore}{\kern0pt}funI{\isacharparenright}{\kern0pt}\isanewline
\ \ \isakeywordONE{from}\isamarkupfalse%
\ lfp{\isacharunderscore}{\kern0pt}lowerbound{\isacharbrackleft}{\kern0pt}of\ {\isachardoublequoteopen}subst{\isacharunderscore}{\kern0pt}lang\ P{\isachardoublequoteclose}{\isacharcomma}{\kern0pt}\ OF\ this{\isacharbrackright}{\kern0pt}\ Lang{\isacharunderscore}{\kern0pt}lfp{\isacharunderscore}{\kern0pt}def\ \isakeywordTHREE{show}\isamarkupfalse%
\ {\isacharquery}{\kern0pt}thesis\ \isakeywordONE{by}\isamarkupfalse%
\ metis\isanewline
\isakeywordONE{qed}\isamarkupfalse%
%
\endisatagproof
{\isafoldproof}%
%
\isadelimproof
\isanewline
%
\endisadelimproof
\isanewline
\isakeywordONE{lemma}\isamarkupfalse%
\ CFG{\isacharunderscore}{\kern0pt}sys{\isacharunderscore}{\kern0pt}CFL{\isacharunderscore}{\kern0pt}is{\isacharunderscore}{\kern0pt}min{\isacharcolon}{\kern0pt}\isanewline
\ \ \isakeywordTWO{assumes}\ {\isachardoublequoteopen}CFG{\isacharunderscore}{\kern0pt}sys\ sys{\isachardoublequoteclose}\isanewline
\ \ \ \ \ \ \isakeywordTWO{and}\ {\isachardoublequoteopen}solves{\isacharunderscore}{\kern0pt}ineq{\isacharunderscore}{\kern0pt}sys\ sys\ sol{\isacharprime}{\kern0pt}{\isachardoublequoteclose}\isanewline
\ \ \ \ \isakeywordTWO{shows}\ {\isachardoublequoteopen}sol\ x\ {\isasymsubseteq}\ sol{\isacharprime}{\kern0pt}\ x{\isachardoublequoteclose}\isanewline
%
\isadelimproof
%
\endisadelimproof
%
\isatagproof
\isakeywordONE{proof}\isamarkupfalse%
\ {\isacharparenleft}{\kern0pt}cases\ {\isachardoublequoteopen}x\ {\isacharless}{\kern0pt}\ card\ {\isacharparenleft}{\kern0pt}Nts\ P{\isacharparenright}{\kern0pt}{\isachardoublequoteclose}{\isacharparenright}{\kern0pt}\isanewline
\ \ \isakeywordTHREE{case}\isamarkupfalse%
\ True\isanewline
\ \ \isakeywordONE{then}\isamarkupfalse%
\ \isakeywordONE{have}\isamarkupfalse%
\ {\isachardoublequoteopen}sol\ x\ {\isacharequal}{\kern0pt}\ Lang{\isacharunderscore}{\kern0pt}lfp\ P\ {\isacharparenleft}{\kern0pt}{\isasymgamma}\ x{\isacharparenright}{\kern0pt}{\isachardoublequoteclose}\ \isakeywordONE{by}\isamarkupfalse%
\ argo\isanewline
\ \ \isakeywordONE{also}\isamarkupfalse%
\ \isakeywordONE{from}\isamarkupfalse%
\ CFG{\isacharunderscore}{\kern0pt}sys{\isacharunderscore}{\kern0pt}CFL{\isacharunderscore}{\kern0pt}is{\isacharunderscore}{\kern0pt}min{\isacharunderscore}{\kern0pt}aux{\isacharbrackleft}{\kern0pt}OF\ assms{\isacharbrackright}{\kern0pt}\ \isakeywordONE{have}\isamarkupfalse%
\ {\isachardoublequoteopen}{\isasymdots}\ {\isasymsubseteq}\ sol{\isacharprime}{\kern0pt}\ {\isacharparenleft}{\kern0pt}{\isasymgamma}{\isacharprime}{\kern0pt}\ {\isacharparenleft}{\kern0pt}{\isasymgamma}\ x{\isacharparenright}{\kern0pt}{\isacharparenright}{\kern0pt}{\isachardoublequoteclose}\ \isakeywordONE{by}\isamarkupfalse%
\ {\isacharparenleft}{\kern0pt}simp\ add{\isacharcolon}{\kern0pt}\ le{\isacharunderscore}{\kern0pt}fun{\isacharunderscore}{\kern0pt}def{\isacharparenright}{\kern0pt}\isanewline
\ \ \isakeywordONE{finally}\isamarkupfalse%
\ \isakeywordTHREE{show}\isamarkupfalse%
\ {\isacharquery}{\kern0pt}thesis\ \isakeywordONE{using}\isamarkupfalse%
\ True\ bij{\isacharunderscore}{\kern0pt}{\isasymgamma}{\isacharunderscore}{\kern0pt}{\isasymgamma}{\isacharprime}{\kern0pt}\ \isakeywordONE{unfolding}\isamarkupfalse%
\ bij{\isacharunderscore}{\kern0pt}Nt{\isacharunderscore}{\kern0pt}Var{\isacharunderscore}{\kern0pt}def\ \isakeywordONE{by}\isamarkupfalse%
\ auto\isanewline
\isakeywordONE{next}\isamarkupfalse%
\isanewline
\ \ \isakeywordTHREE{case}\isamarkupfalse%
\ False\isanewline
\ \ \isakeywordONE{then}\isamarkupfalse%
\ \isakeywordTHREE{show}\isamarkupfalse%
\ {\isacharquery}{\kern0pt}thesis\ \isakeywordONE{by}\isamarkupfalse%
\ auto\isanewline
\isakeywordONE{qed}\isamarkupfalse%
%
\endisatagproof
{\isafoldproof}%
%
\isadelimproof
%
\endisadelimproof
%
\begin{isamarkuptext}%
Lastly we combine all of the previous lemmas into the desired result of this section, namely
that each CFG induces a system of equations such that the CFG's language is a minimal solution of
the system:%
\end{isamarkuptext}\isamarkuptrue%
\isakeywordONE{lemma}\isamarkupfalse%
\ CFL{\isacharunderscore}{\kern0pt}is{\isacharunderscore}{\kern0pt}min{\isacharunderscore}{\kern0pt}sol{\isacharcolon}{\kern0pt}\isanewline
\ \ {\isachardoublequoteopen}{\isasymexists}sys{\isachardot}{\kern0pt}\ {\isacharparenleft}{\kern0pt}{\isasymforall}eq\ {\isasymin}\ set\ sys{\isachardot}{\kern0pt}\ reg{\isacharunderscore}{\kern0pt}eval\ eq{\isacharparenright}{\kern0pt}\ {\isasymand}\ {\isacharparenleft}{\kern0pt}{\isasymforall}eq\ {\isasymin}\ set\ sys{\isachardot}{\kern0pt}\ {\isasymforall}x\ {\isasymin}\ vars\ eq{\isachardot}{\kern0pt}\ x\ {\isacharless}{\kern0pt}\ length\ sys{\isacharparenright}{\kern0pt}\isanewline
\ \ \ \ \ \ \ \ \ \ {\isasymand}\ min{\isacharunderscore}{\kern0pt}sol{\isacharunderscore}{\kern0pt}ineq{\isacharunderscore}{\kern0pt}sys\ sys\ sol{\isachardoublequoteclose}\isanewline
%
\isadelimproof
%
\endisadelimproof
%
\isatagproof
\isakeywordONE{proof}\isamarkupfalse%
\ {\isacharminus}{\kern0pt}\isanewline
\ \ \isakeywordONE{from}\isamarkupfalse%
\ CFG{\isacharunderscore}{\kern0pt}as{\isacharunderscore}{\kern0pt}eq{\isacharunderscore}{\kern0pt}sys\ \isakeywordTHREE{obtain}\isamarkupfalse%
\ sys\ \isakeywordTWO{where}\ {\isacharasterisk}{\kern0pt}{\isacharcolon}{\kern0pt}\ {\isachardoublequoteopen}CFG{\isacharunderscore}{\kern0pt}sys\ sys{\isachardoublequoteclose}\ \isakeywordONE{by}\isamarkupfalse%
\ blast\isanewline
\ \ \isakeywordONE{then}\isamarkupfalse%
\ \isakeywordONE{have}\isamarkupfalse%
\ {\isachardoublequoteopen}length\ sys\ {\isacharequal}{\kern0pt}\ card\ {\isacharparenleft}{\kern0pt}Nts\ P{\isacharparenright}{\kern0pt}{\isachardoublequoteclose}\ \isakeywordONE{by}\isamarkupfalse%
\ blast\isanewline
\ \ \isakeywordONE{moreover}\isamarkupfalse%
\ \isakeywordONE{from}\isamarkupfalse%
\ {\isacharasterisk}{\kern0pt}\ \isakeywordONE{have}\isamarkupfalse%
\ {\isachardoublequoteopen}{\isasymforall}eq\ {\isasymin}\ set\ sys{\isachardot}{\kern0pt}\ reg{\isacharunderscore}{\kern0pt}eval\ eq{\isachardoublequoteclose}\ \isakeywordONE{by}\isamarkupfalse%
\ {\isacharparenleft}{\kern0pt}simp\ add{\isacharcolon}{\kern0pt}\ all{\isacharunderscore}{\kern0pt}set{\isacharunderscore}{\kern0pt}conv{\isacharunderscore}{\kern0pt}all{\isacharunderscore}{\kern0pt}nth{\isacharparenright}{\kern0pt}\isanewline
\ \ \isakeywordONE{moreover}\isamarkupfalse%
\ \isakeywordONE{from}\isamarkupfalse%
\ {\isacharasterisk}{\kern0pt}\ {\isacartoucheopen}length\ sys\ {\isacharequal}{\kern0pt}\ card\ {\isacharparenleft}{\kern0pt}Nts\ P{\isacharparenright}{\kern0pt}{\isacartoucheclose}\ \isakeywordONE{have}\isamarkupfalse%
\ {\isachardoublequoteopen}{\isasymforall}eq\ {\isasymin}\ set\ sys{\isachardot}{\kern0pt}\ {\isasymforall}x\ {\isasymin}\ vars\ eq{\isachardot}{\kern0pt}\ x\ {\isacharless}{\kern0pt}\ length\ sys{\isachardoublequoteclose}\isanewline
\ \ \ \ \isakeywordONE{by}\isamarkupfalse%
\ {\isacharparenleft}{\kern0pt}simp\ add{\isacharcolon}{\kern0pt}\ all{\isacharunderscore}{\kern0pt}set{\isacharunderscore}{\kern0pt}conv{\isacharunderscore}{\kern0pt}all{\isacharunderscore}{\kern0pt}nth{\isacharparenright}{\kern0pt}\isanewline
\ \ \isakeywordONE{moreover}\isamarkupfalse%
\ \isakeywordONE{from}\isamarkupfalse%
\ CFG{\isacharunderscore}{\kern0pt}sys{\isacharunderscore}{\kern0pt}CFL{\isacharunderscore}{\kern0pt}is{\isacharunderscore}{\kern0pt}sol{\isacharbrackleft}{\kern0pt}OF\ {\isacharasterisk}{\kern0pt}{\isacharbrackright}{\kern0pt}\ CFG{\isacharunderscore}{\kern0pt}sys{\isacharunderscore}{\kern0pt}CFL{\isacharunderscore}{\kern0pt}is{\isacharunderscore}{\kern0pt}min{\isacharbrackleft}{\kern0pt}OF\ {\isacharasterisk}{\kern0pt}{\isacharbrackright}{\kern0pt}\isanewline
\ \ \ \ \isakeywordONE{have}\isamarkupfalse%
\ {\isachardoublequoteopen}min{\isacharunderscore}{\kern0pt}sol{\isacharunderscore}{\kern0pt}ineq{\isacharunderscore}{\kern0pt}sys\ sys\ sol{\isachardoublequoteclose}\ \isakeywordONE{unfolding}\isamarkupfalse%
\ min{\isacharunderscore}{\kern0pt}sol{\isacharunderscore}{\kern0pt}ineq{\isacharunderscore}{\kern0pt}sys{\isacharunderscore}{\kern0pt}def\ \isakeywordONE{by}\isamarkupfalse%
\ blast\isanewline
\ \ \isakeywordONE{ultimately}\isamarkupfalse%
\ \isakeywordTHREE{show}\isamarkupfalse%
\ {\isacharquery}{\kern0pt}thesis\ \isakeywordONE{by}\isamarkupfalse%
\ blast\isanewline
\isakeywordONE{qed}\isamarkupfalse%
%
\endisatagproof
{\isafoldproof}%
%
\isadelimproof
\isanewline
%
\endisadelimproof
\isanewline
\isakeywordTWO{end}\isamarkupfalse%
%
\isadelimdocument
%
\endisadelimdocument
%
\isatagdocument
%
\isamarkupsubsection{Relation between the two types of systems of equations%
}
\isamarkuptrue%
%
\endisatagdocument
{\isafolddocument}%
%
\isadelimdocument
%
\endisadelimdocument
%
\begin{isamarkuptext}%
One can simply convert a system \isa{sys} of equations of the second type (i.e. with Parikh
images) into a system of equations of the first type by dropping the Parikh images on both side of
each equation. The following lemmas describe how the two systems are related to each other.

First of all, to any solution \isa{sol} of \isa{sys} exists a valuation whose Parikh image is
identical to that of \isa{sol} and which is a solution of the other system (i.e. the system obtained
by dropping all Parikh images in \isa{sys}). The proof benefits from the result of section 2.7:%
\end{isamarkuptext}\isamarkuptrue%
\isakeywordONE{lemma}\isamarkupfalse%
\ sol{\isacharunderscore}{\kern0pt}comm{\isacharunderscore}{\kern0pt}sol{\isacharcolon}{\kern0pt}\isanewline
\ \ \isakeywordTWO{assumes}\ sol{\isacharunderscore}{\kern0pt}is{\isacharunderscore}{\kern0pt}sol{\isacharunderscore}{\kern0pt}comm{\isacharcolon}{\kern0pt}\ {\isachardoublequoteopen}solves{\isacharunderscore}{\kern0pt}ineq{\isacharunderscore}{\kern0pt}sys{\isacharunderscore}{\kern0pt}comm\ sys\ sol{\isachardoublequoteclose}\isanewline
\ \ \isakeywordTWO{shows}\ \ \ {\isachardoublequoteopen}{\isasymexists}sol{\isacharprime}{\kern0pt}{\isachardot}{\kern0pt}\ {\isacharparenleft}{\kern0pt}{\isasymforall}x{\isachardot}{\kern0pt}\ {\isasymPsi}\ {\isacharparenleft}{\kern0pt}sol\ x{\isacharparenright}{\kern0pt}\ {\isacharequal}{\kern0pt}\ {\isasymPsi}\ {\isacharparenleft}{\kern0pt}sol{\isacharprime}{\kern0pt}\ x{\isacharparenright}{\kern0pt}{\isacharparenright}{\kern0pt}\ {\isasymand}\ solves{\isacharunderscore}{\kern0pt}ineq{\isacharunderscore}{\kern0pt}sys\ sys\ sol{\isacharprime}{\kern0pt}{\isachardoublequoteclose}\isanewline
%
\isadelimproof
%
\endisadelimproof
%
\isatagproof
\isakeywordONE{proof}\isamarkupfalse%
\isanewline
\ \ \isakeywordONE{let}\isamarkupfalse%
\ {\isacharquery}{\kern0pt}sol{\isacharprime}{\kern0pt}\ {\isacharequal}{\kern0pt}\ {\isachardoublequoteopen}{\isasymlambda}x{\isachardot}{\kern0pt}\ {\isasymUnion}{\isacharparenleft}{\kern0pt}parikh{\isacharunderscore}{\kern0pt}img{\isacharunderscore}{\kern0pt}eq{\isacharunderscore}{\kern0pt}class\ {\isacharparenleft}{\kern0pt}sol\ x{\isacharparenright}{\kern0pt}{\isacharparenright}{\kern0pt}{\isachardoublequoteclose}\isanewline
\ \ \isakeywordONE{have}\isamarkupfalse%
\ sol{\isacharprime}{\kern0pt}{\isacharunderscore}{\kern0pt}sol{\isacharcolon}{\kern0pt}\ {\isachardoublequoteopen}{\isasymforall}x{\isachardot}{\kern0pt}\ {\isasymPsi}\ {\isacharparenleft}{\kern0pt}{\isacharquery}{\kern0pt}sol{\isacharprime}{\kern0pt}\ x{\isacharparenright}{\kern0pt}\ {\isacharequal}{\kern0pt}\ {\isasymPsi}\ {\isacharparenleft}{\kern0pt}sol\ x{\isacharparenright}{\kern0pt}{\isachardoublequoteclose}\isanewline
\ \ \ \ \ \ \isakeywordONE{using}\isamarkupfalse%
\ parikh{\isacharunderscore}{\kern0pt}img{\isacharunderscore}{\kern0pt}Union{\isacharunderscore}{\kern0pt}class\ \isakeywordONE{by}\isamarkupfalse%
\ metis\isanewline
\ \ \isakeywordONE{moreover}\isamarkupfalse%
\ \isakeywordONE{have}\isamarkupfalse%
\ {\isachardoublequoteopen}solves{\isacharunderscore}{\kern0pt}ineq{\isacharunderscore}{\kern0pt}sys\ sys\ {\isacharquery}{\kern0pt}sol{\isacharprime}{\kern0pt}{\isachardoublequoteclose}\isanewline
\ \ \isakeywordONE{unfolding}\isamarkupfalse%
\ solves{\isacharunderscore}{\kern0pt}ineq{\isacharunderscore}{\kern0pt}sys{\isacharunderscore}{\kern0pt}def\ \isakeywordONE{proof}\isamarkupfalse%
\ {\isacharparenleft}{\kern0pt}rule\ allI{\isacharcomma}{\kern0pt}\ rule\ impI{\isacharparenright}{\kern0pt}\isanewline
\ \ \ \ \isakeywordTHREE{fix}\isamarkupfalse%
\ i\isanewline
\ \ \ \ \isakeywordTHREE{assume}\isamarkupfalse%
\ {\isachardoublequoteopen}i\ {\isacharless}{\kern0pt}\ length\ sys{\isachardoublequoteclose}\isanewline
\ \ \ \ \isakeywordONE{with}\isamarkupfalse%
\ sol{\isacharunderscore}{\kern0pt}is{\isacharunderscore}{\kern0pt}sol{\isacharunderscore}{\kern0pt}comm\ \isakeywordONE{have}\isamarkupfalse%
\ {\isachardoublequoteopen}{\isasymPsi}\ {\isacharparenleft}{\kern0pt}eval\ {\isacharparenleft}{\kern0pt}sys\ {\isacharbang}{\kern0pt}\ i{\isacharparenright}{\kern0pt}\ sol{\isacharparenright}{\kern0pt}\ {\isasymsubseteq}\ {\isasymPsi}\ {\isacharparenleft}{\kern0pt}sol\ i{\isacharparenright}{\kern0pt}{\isachardoublequoteclose}\isanewline
\ \ \ \ \ \ \isakeywordONE{unfolding}\isamarkupfalse%
\ solves{\isacharunderscore}{\kern0pt}ineq{\isacharunderscore}{\kern0pt}sys{\isacharunderscore}{\kern0pt}comm{\isacharunderscore}{\kern0pt}def\ solves{\isacharunderscore}{\kern0pt}ineq{\isacharunderscore}{\kern0pt}comm{\isacharunderscore}{\kern0pt}def\ \isakeywordONE{by}\isamarkupfalse%
\ blast\isanewline
\ \ \ \ \isakeywordONE{moreover}\isamarkupfalse%
\ \isakeywordONE{from}\isamarkupfalse%
\ sol{\isacharprime}{\kern0pt}{\isacharunderscore}{\kern0pt}sol\ \isakeywordONE{have}\isamarkupfalse%
\ {\isachardoublequoteopen}{\isasymPsi}\ {\isacharparenleft}{\kern0pt}eval\ {\isacharparenleft}{\kern0pt}sys\ {\isacharbang}{\kern0pt}\ i{\isacharparenright}{\kern0pt}\ {\isacharquery}{\kern0pt}sol{\isacharprime}{\kern0pt}{\isacharparenright}{\kern0pt}\ {\isacharequal}{\kern0pt}\ {\isasymPsi}\ {\isacharparenleft}{\kern0pt}eval\ {\isacharparenleft}{\kern0pt}sys\ {\isacharbang}{\kern0pt}\ i{\isacharparenright}{\kern0pt}\ sol{\isacharparenright}{\kern0pt}{\isachardoublequoteclose}\isanewline
\ \ \ \ \ \ \isakeywordONE{using}\isamarkupfalse%
\ rlexp{\isacharunderscore}{\kern0pt}mono{\isacharunderscore}{\kern0pt}parikh{\isacharunderscore}{\kern0pt}eq\ \isakeywordONE{by}\isamarkupfalse%
\ meson\isanewline
\ \ \ \ \isakeywordONE{ultimately}\isamarkupfalse%
\ \isakeywordONE{have}\isamarkupfalse%
\ {\isachardoublequoteopen}{\isasymPsi}\ {\isacharparenleft}{\kern0pt}eval\ {\isacharparenleft}{\kern0pt}sys\ {\isacharbang}{\kern0pt}\ i{\isacharparenright}{\kern0pt}\ {\isacharquery}{\kern0pt}sol{\isacharprime}{\kern0pt}{\isacharparenright}{\kern0pt}\ {\isasymsubseteq}\ {\isasymPsi}\ {\isacharparenleft}{\kern0pt}sol\ i{\isacharparenright}{\kern0pt}{\isachardoublequoteclose}\ \isakeywordONE{by}\isamarkupfalse%
\ simp\isanewline
\ \ \ \ \isakeywordONE{then}\isamarkupfalse%
\ \isakeywordTHREE{show}\isamarkupfalse%
\ {\isachardoublequoteopen}eval\ {\isacharparenleft}{\kern0pt}sys\ {\isacharbang}{\kern0pt}\ i{\isacharparenright}{\kern0pt}\ {\isacharquery}{\kern0pt}sol{\isacharprime}{\kern0pt}\ {\isasymsubseteq}\ {\isacharquery}{\kern0pt}sol{\isacharprime}{\kern0pt}\ i{\isachardoublequoteclose}\ \isakeywordONE{using}\isamarkupfalse%
\ subseteq{\isacharunderscore}{\kern0pt}comm{\isacharunderscore}{\kern0pt}subseteq\ \isakeywordONE{by}\isamarkupfalse%
\ metis\isanewline
\ \ \isakeywordONE{qed}\isamarkupfalse%
\isanewline
\ \ \isakeywordONE{ultimately}\isamarkupfalse%
\ \isakeywordTHREE{show}\isamarkupfalse%
\ {\isachardoublequoteopen}{\isacharparenleft}{\kern0pt}{\isasymforall}x{\isachardot}{\kern0pt}\ {\isasymPsi}\ {\isacharparenleft}{\kern0pt}sol\ x{\isacharparenright}{\kern0pt}\ {\isacharequal}{\kern0pt}\ {\isasymPsi}\ {\isacharparenleft}{\kern0pt}{\isacharquery}{\kern0pt}sol{\isacharprime}{\kern0pt}\ x{\isacharparenright}{\kern0pt}{\isacharparenright}{\kern0pt}\ {\isasymand}\ solves{\isacharunderscore}{\kern0pt}ineq{\isacharunderscore}{\kern0pt}sys\ sys\ {\isacharquery}{\kern0pt}sol{\isacharprime}{\kern0pt}{\isachardoublequoteclose}\isanewline
\ \ \ \ \isakeywordONE{by}\isamarkupfalse%
\ simp\isanewline
\isakeywordONE{qed}\isamarkupfalse%
%
\endisatagproof
{\isafoldproof}%
%
\isadelimproof
%
\endisadelimproof
%
\begin{isamarkuptext}%
The converse works similarly: Given a minimal solution \isa{sol} of the system \isa{sys} of the first type,
then \isa{sol} is also a minimal solution to the system obtained by converting \isa{sys} into a system of the second
type (which can be achieved by applying the Parikh image on both sides of each equation):%
\end{isamarkuptext}\isamarkuptrue%
\isakeywordONE{lemma}\isamarkupfalse%
\ min{\isacharunderscore}{\kern0pt}sol{\isacharunderscore}{\kern0pt}min{\isacharunderscore}{\kern0pt}sol{\isacharunderscore}{\kern0pt}comm{\isacharcolon}{\kern0pt}\isanewline
\ \ \isakeywordTWO{assumes}\ {\isachardoublequoteopen}min{\isacharunderscore}{\kern0pt}sol{\isacharunderscore}{\kern0pt}ineq{\isacharunderscore}{\kern0pt}sys\ sys\ sol{\isachardoublequoteclose}\isanewline
\ \ \ \ \isakeywordTWO{shows}\ {\isachardoublequoteopen}min{\isacharunderscore}{\kern0pt}sol{\isacharunderscore}{\kern0pt}ineq{\isacharunderscore}{\kern0pt}sys{\isacharunderscore}{\kern0pt}comm\ sys\ sol{\isachardoublequoteclose}\isanewline
%
\isadelimproof
%
\endisadelimproof
%
\isatagproof
\isakeywordONE{unfolding}\isamarkupfalse%
\ min{\isacharunderscore}{\kern0pt}sol{\isacharunderscore}{\kern0pt}ineq{\isacharunderscore}{\kern0pt}sys{\isacharunderscore}{\kern0pt}comm{\isacharunderscore}{\kern0pt}def\ \isakeywordONE{proof}\isamarkupfalse%
\isanewline
\ \ \isakeywordONE{from}\isamarkupfalse%
\ assms\ \isakeywordTHREE{show}\isamarkupfalse%
\ {\isachardoublequoteopen}solves{\isacharunderscore}{\kern0pt}ineq{\isacharunderscore}{\kern0pt}sys{\isacharunderscore}{\kern0pt}comm\ sys\ sol{\isachardoublequoteclose}\isanewline
\ \ \ \ \isakeywordONE{unfolding}\isamarkupfalse%
\ min{\isacharunderscore}{\kern0pt}sol{\isacharunderscore}{\kern0pt}ineq{\isacharunderscore}{\kern0pt}sys{\isacharunderscore}{\kern0pt}def\ min{\isacharunderscore}{\kern0pt}sol{\isacharunderscore}{\kern0pt}ineq{\isacharunderscore}{\kern0pt}sys{\isacharunderscore}{\kern0pt}comm{\isacharunderscore}{\kern0pt}def\ solves{\isacharunderscore}{\kern0pt}ineq{\isacharunderscore}{\kern0pt}sys{\isacharunderscore}{\kern0pt}def\isanewline
\ \ \ \ \ \ solves{\isacharunderscore}{\kern0pt}ineq{\isacharunderscore}{\kern0pt}sys{\isacharunderscore}{\kern0pt}comm{\isacharunderscore}{\kern0pt}def\ solves{\isacharunderscore}{\kern0pt}ineq{\isacharunderscore}{\kern0pt}comm{\isacharunderscore}{\kern0pt}def\ \isakeywordONE{by}\isamarkupfalse%
\ {\isacharparenleft}{\kern0pt}simp\ add{\isacharcolon}{\kern0pt}\ parikh{\isacharunderscore}{\kern0pt}img{\isacharunderscore}{\kern0pt}mono{\isacharparenright}{\kern0pt}\isanewline
\ \ \isakeywordTHREE{show}\isamarkupfalse%
\ {\isachardoublequoteopen}\ {\isasymforall}sol{\isacharprime}{\kern0pt}{\isachardot}{\kern0pt}\ solves{\isacharunderscore}{\kern0pt}ineq{\isacharunderscore}{\kern0pt}sys{\isacharunderscore}{\kern0pt}comm\ sys\ sol{\isacharprime}{\kern0pt}\ {\isasymlongrightarrow}\ {\isacharparenleft}{\kern0pt}{\isasymforall}x{\isachardot}{\kern0pt}\ {\isasymPsi}\ {\isacharparenleft}{\kern0pt}sol\ x{\isacharparenright}{\kern0pt}\ {\isasymsubseteq}\ {\isasymPsi}\ {\isacharparenleft}{\kern0pt}sol{\isacharprime}{\kern0pt}\ x{\isacharparenright}{\kern0pt}{\isacharparenright}{\kern0pt}{\isachardoublequoteclose}\isanewline
\ \ \isakeywordONE{proof}\isamarkupfalse%
\ {\isacharparenleft}{\kern0pt}rule\ allI{\isacharcomma}{\kern0pt}\ rule\ impI{\isacharparenright}{\kern0pt}\isanewline
\ \ \ \ \isakeywordTHREE{fix}\isamarkupfalse%
\ sol{\isacharprime}{\kern0pt}\isanewline
\ \ \ \ \isakeywordTHREE{assume}\isamarkupfalse%
\ {\isachardoublequoteopen}solves{\isacharunderscore}{\kern0pt}ineq{\isacharunderscore}{\kern0pt}sys{\isacharunderscore}{\kern0pt}comm\ sys\ sol{\isacharprime}{\kern0pt}{\isachardoublequoteclose}\isanewline
\ \ \ \ \isakeywordONE{with}\isamarkupfalse%
\ sol{\isacharunderscore}{\kern0pt}comm{\isacharunderscore}{\kern0pt}sol\ \isakeywordTHREE{obtain}\isamarkupfalse%
\ sol{\isacharprime}{\kern0pt}{\isacharprime}{\kern0pt}\ \isakeywordTWO{where}\ sol{\isacharprime}{\kern0pt}{\isacharprime}{\kern0pt}{\isacharunderscore}{\kern0pt}intro{\isacharcolon}{\kern0pt}\isanewline
\ \ \ \ \ \ {\isachardoublequoteopen}{\isacharparenleft}{\kern0pt}{\isasymforall}x{\isachardot}{\kern0pt}\ {\isasymPsi}\ {\isacharparenleft}{\kern0pt}sol{\isacharprime}{\kern0pt}\ x{\isacharparenright}{\kern0pt}\ {\isacharequal}{\kern0pt}\ {\isasymPsi}\ {\isacharparenleft}{\kern0pt}sol{\isacharprime}{\kern0pt}{\isacharprime}{\kern0pt}\ x{\isacharparenright}{\kern0pt}{\isacharparenright}{\kern0pt}\ {\isasymand}\ solves{\isacharunderscore}{\kern0pt}ineq{\isacharunderscore}{\kern0pt}sys\ sys\ sol{\isacharprime}{\kern0pt}{\isacharprime}{\kern0pt}{\isachardoublequoteclose}\ \isakeywordONE{by}\isamarkupfalse%
\ meson\isanewline
\ \ \ \ \isakeywordONE{with}\isamarkupfalse%
\ assms\ \isakeywordONE{have}\isamarkupfalse%
\ {\isachardoublequoteopen}{\isasymforall}x{\isachardot}{\kern0pt}\ sol\ x\ {\isasymsubseteq}\ sol{\isacharprime}{\kern0pt}{\isacharprime}{\kern0pt}\ x{\isachardoublequoteclose}\ \isakeywordONE{unfolding}\isamarkupfalse%
\ min{\isacharunderscore}{\kern0pt}sol{\isacharunderscore}{\kern0pt}ineq{\isacharunderscore}{\kern0pt}sys{\isacharunderscore}{\kern0pt}def\ \isakeywordONE{by}\isamarkupfalse%
\ auto\isanewline
\ \ \ \ \isakeywordONE{with}\isamarkupfalse%
\ sol{\isacharprime}{\kern0pt}{\isacharprime}{\kern0pt}{\isacharunderscore}{\kern0pt}intro\ \isakeywordTHREE{show}\isamarkupfalse%
\ {\isachardoublequoteopen}{\isasymforall}x{\isachardot}{\kern0pt}\ {\isasymPsi}\ {\isacharparenleft}{\kern0pt}sol\ x{\isacharparenright}{\kern0pt}\ {\isasymsubseteq}\ {\isasymPsi}\ {\isacharparenleft}{\kern0pt}sol{\isacharprime}{\kern0pt}\ x{\isacharparenright}{\kern0pt}{\isachardoublequoteclose}\isanewline
\ \ \ \ \ \ \isakeywordONE{using}\isamarkupfalse%
\ parikh{\isacharunderscore}{\kern0pt}img{\isacharunderscore}{\kern0pt}mono\ \isakeywordONE{by}\isamarkupfalse%
\ metis\isanewline
\ \ \isakeywordONE{qed}\isamarkupfalse%
\isanewline
\isakeywordONE{qed}\isamarkupfalse%
%
\endisatagproof
{\isafoldproof}%
%
\isadelimproof
%
\endisadelimproof
%
\begin{isamarkuptext}%
All minimal solutions of a system of the second type have the same Parikh image:%
\end{isamarkuptext}\isamarkuptrue%
\isakeywordONE{lemma}\isamarkupfalse%
\ min{\isacharunderscore}{\kern0pt}sol{\isacharunderscore}{\kern0pt}comm{\isacharunderscore}{\kern0pt}unique{\isacharcolon}{\kern0pt}\isanewline
\ \ \isakeywordTWO{assumes}\ sol{\isadigit{1}}{\isacharunderscore}{\kern0pt}is{\isacharunderscore}{\kern0pt}min{\isacharunderscore}{\kern0pt}sol{\isacharcolon}{\kern0pt}\ {\isachardoublequoteopen}min{\isacharunderscore}{\kern0pt}sol{\isacharunderscore}{\kern0pt}ineq{\isacharunderscore}{\kern0pt}sys{\isacharunderscore}{\kern0pt}comm\ sys\ sol{\isadigit{1}}{\isachardoublequoteclose}\isanewline
\ \ \ \ \ \ \isakeywordTWO{and}\ sol{\isadigit{2}}{\isacharunderscore}{\kern0pt}is{\isacharunderscore}{\kern0pt}min{\isacharunderscore}{\kern0pt}sol{\isacharcolon}{\kern0pt}\ {\isachardoublequoteopen}min{\isacharunderscore}{\kern0pt}sol{\isacharunderscore}{\kern0pt}ineq{\isacharunderscore}{\kern0pt}sys{\isacharunderscore}{\kern0pt}comm\ sys\ sol{\isadigit{2}}{\isachardoublequoteclose}\isanewline
\ \ \ \ \isakeywordTWO{shows}\ \ \ \ \ \ \ \ \ \ \ \ \ \ \ \ \ \ {\isachardoublequoteopen}{\isasymPsi}\ {\isacharparenleft}{\kern0pt}sol{\isadigit{1}}\ x{\isacharparenright}{\kern0pt}\ {\isacharequal}{\kern0pt}\ {\isasymPsi}\ {\isacharparenleft}{\kern0pt}sol{\isadigit{2}}\ x{\isacharparenright}{\kern0pt}{\isachardoublequoteclose}\isanewline
%
\isadelimproof
%
\endisadelimproof
%
\isatagproof
\isakeywordONE{proof}\isamarkupfalse%
\ {\isacharminus}{\kern0pt}\isanewline
\ \ \isakeywordONE{from}\isamarkupfalse%
\ sol{\isadigit{1}}{\isacharunderscore}{\kern0pt}is{\isacharunderscore}{\kern0pt}min{\isacharunderscore}{\kern0pt}sol\ sol{\isadigit{2}}{\isacharunderscore}{\kern0pt}is{\isacharunderscore}{\kern0pt}min{\isacharunderscore}{\kern0pt}sol\ \isakeywordONE{have}\isamarkupfalse%
\ {\isachardoublequoteopen}{\isasymPsi}\ {\isacharparenleft}{\kern0pt}sol{\isadigit{1}}\ x{\isacharparenright}{\kern0pt}\ {\isasymsubseteq}\ {\isasymPsi}\ {\isacharparenleft}{\kern0pt}sol{\isadigit{2}}\ x{\isacharparenright}{\kern0pt}{\isachardoublequoteclose}\isanewline
\ \ \ \ \isakeywordONE{unfolding}\isamarkupfalse%
\ min{\isacharunderscore}{\kern0pt}sol{\isacharunderscore}{\kern0pt}ineq{\isacharunderscore}{\kern0pt}sys{\isacharunderscore}{\kern0pt}comm{\isacharunderscore}{\kern0pt}def\ \isakeywordONE{by}\isamarkupfalse%
\ simp\isanewline
\ \ \isakeywordONE{moreover}\isamarkupfalse%
\ \isakeywordONE{from}\isamarkupfalse%
\ sol{\isadigit{1}}{\isacharunderscore}{\kern0pt}is{\isacharunderscore}{\kern0pt}min{\isacharunderscore}{\kern0pt}sol\ sol{\isadigit{2}}{\isacharunderscore}{\kern0pt}is{\isacharunderscore}{\kern0pt}min{\isacharunderscore}{\kern0pt}sol\ \isakeywordONE{have}\isamarkupfalse%
\ {\isachardoublequoteopen}{\isasymPsi}\ {\isacharparenleft}{\kern0pt}sol{\isadigit{2}}\ x{\isacharparenright}{\kern0pt}\ {\isasymsubseteq}\ {\isasymPsi}\ {\isacharparenleft}{\kern0pt}sol{\isadigit{1}}\ x{\isacharparenright}{\kern0pt}{\isachardoublequoteclose}\isanewline
\ \ \ \ \isakeywordONE{unfolding}\isamarkupfalse%
\ min{\isacharunderscore}{\kern0pt}sol{\isacharunderscore}{\kern0pt}ineq{\isacharunderscore}{\kern0pt}sys{\isacharunderscore}{\kern0pt}comm{\isacharunderscore}{\kern0pt}def\ \isakeywordONE{by}\isamarkupfalse%
\ simp\isanewline
\ \ \isakeywordONE{ultimately}\isamarkupfalse%
\ \isakeywordTHREE{show}\isamarkupfalse%
\ {\isacharquery}{\kern0pt}thesis\ \isakeywordONE{by}\isamarkupfalse%
\ blast\isanewline
\isakeywordONE{qed}\isamarkupfalse%
%
\endisatagproof
{\isafoldproof}%
%
\isadelimproof
\isanewline
%
\endisadelimproof
%
\isadelimtheory
\isanewline
%
\endisadelimtheory
%
\isatagtheory
\isakeywordTWO{end}\isamarkupfalse%
%
\endisatagtheory
{\isafoldtheory}%
%
\isadelimtheory
%
\endisadelimtheory
%
\end{isabellebody}%
\endinput
%:%file=~/studium/semester_7/semantik/homeworks/AIST/Parikh/Eq_Sys.thy%:%
%:%11=1%:%
%:%27=3%:%
%:%28=3%:%
%:%29=4%:%
%:%30=5%:%
%:%31=6%:%
%:%32=7%:%
%:%41=9%:%
%:%42=10%:%
%:%43=11%:%
%:%52=14%:%
%:%64=16%:%
%:%65=17%:%
%:%66=18%:%
%:%67=19%:%
%:%68=20%:%
%:%69=21%:%
%:%70=22%:%
%:%71=23%:%
%:%73=25%:%
%:%74=25%:%
%:%76=28%:%
%:%77=29%:%
%:%78=30%:%
%:%80=31%:%
%:%81=31%:%
%:%82=32%:%
%:%83=33%:%
%:%84=34%:%
%:%85=34%:%
%:%86=35%:%
%:%89=39%:%
%:%90=40%:%
%:%92=41%:%
%:%93=41%:%
%:%94=42%:%
%:%95=43%:%
%:%96=44%:%
%:%97=44%:%
%:%98=45%:%
%:%99=46%:%
%:%100=47%:%
%:%101=47%:%
%:%102=48%:%
%:%106=52%:%
%:%108=53%:%
%:%109=53%:%
%:%110=54%:%
%:%111=55%:%
%:%112=56%:%
%:%113=56%:%
%:%114=57%:%
%:%115=58%:%
%:%118=59%:%
%:%122=59%:%
%:%123=59%:%
%:%124=59%:%
%:%138=62%:%
%:%150=64%:%
%:%151=65%:%
%:%152=66%:%
%:%153=67%:%
%:%155=68%:%
%:%156=68%:%
%:%157=69%:%
%:%159=71%:%
%:%160=72%:%
%:%161=73%:%
%:%163=74%:%
%:%164=74%:%
%:%165=75%:%
%:%167=77%:%
%:%168=78%:%
%:%169=79%:%
%:%176=85%:%
%:%177=85%:%
%:%178=86%:%
%:%184=92%:%
%:%185=93%:%
%:%186=94%:%
%:%187=95%:%
%:%188=96%:%
%:%190=97%:%
%:%191=97%:%
%:%192=98%:%
%:%200=107%:%
%:%201=108%:%
%:%203=109%:%
%:%204=109%:%
%:%205=110%:%
%:%206=111%:%
%:%207=112%:%
%:%208=113%:%
%:%215=114%:%
%:%216=114%:%
%:%217=115%:%
%:%218=115%:%
%:%219=115%:%
%:%220=116%:%
%:%221=116%:%
%:%222=116%:%
%:%223=116%:%
%:%224=117%:%
%:%225=117%:%
%:%226=117%:%
%:%227=117%:%
%:%228=118%:%
%:%229=118%:%
%:%230=118%:%
%:%231=119%:%
%:%232=119%:%
%:%233=119%:%
%:%234=119%:%
%:%235=120%:%
%:%236=121%:%
%:%237=121%:%
%:%238=121%:%
%:%239=122%:%
%:%240=122%:%
%:%241=122%:%
%:%242=122%:%
%:%243=122%:%
%:%244=123%:%
%:%259=127%:%
%:%271=129%:%
%:%272=130%:%
%:%273=131%:%
%:%274=132%:%
%:%276=134%:%
%:%277=134%:%
%:%278=135%:%
%:%279=136%:%
%:%280=137%:%
%:%281=138%:%
%:%282=138%:%
%:%289=139%:%
%:%290=139%:%
%:%291=140%:%
%:%292=140%:%
%:%293=141%:%
%:%294=141%:%
%:%295=141%:%
%:%296=141%:%
%:%297=142%:%
%:%298=142%:%
%:%299=142%:%
%:%300=142%:%
%:%301=143%:%
%:%302=143%:%
%:%303=144%:%
%:%304=144%:%
%:%305=144%:%
%:%306=144%:%
%:%307=144%:%
%:%308=145%:%
%:%309=145%:%
%:%310=145%:%
%:%311=145%:%
%:%312=146%:%
%:%313=146%:%
%:%314=146%:%
%:%315=146%:%
%:%316=147%:%
%:%317=147%:%
%:%318=147%:%
%:%319=147%:%
%:%320=147%:%
%:%321=148%:%
%:%327=148%:%
%:%330=149%:%
%:%331=150%:%
%:%332=151%:%
%:%333=151%:%
%:%334=152%:%
%:%335=153%:%
%:%336=154%:%
%:%337=155%:%
%:%338=156%:%
%:%339=157%:%
%:%340=158%:%
%:%342=160%:%
%:%343=161%:%
%:%344=162%:%
%:%345=163%:%
%:%347=165%:%
%:%348=165%:%
%:%349=166%:%
%:%350=167%:%
%:%351=168%:%
%:%352=168%:%
%:%353=169%:%
%:%354=170%:%
%:%355=171%:%
%:%356=171%:%
%:%357=172%:%
%:%359=175%:%
%:%360=176%:%
%:%362=178%:%
%:%363=178%:%
%:%370=179%:%
%:%371=179%:%
%:%372=179%:%
%:%373=180%:%
%:%374=180%:%
%:%375=181%:%
%:%376=181%:%
%:%377=181%:%
%:%378=182%:%
%:%379=182%:%
%:%380=182%:%
%:%381=182%:%
%:%382=183%:%
%:%383=183%:%
%:%388=183%:%
%:%391=184%:%
%:%392=185%:%
%:%393=185%:%
%:%394=186%:%
%:%395=187%:%
%:%398=188%:%
%:%402=188%:%
%:%403=188%:%
%:%404=188%:%
%:%405=188%:%
%:%410=188%:%
%:%413=189%:%
%:%414=190%:%
%:%415=190%:%
%:%422=191%:%
%:%423=191%:%
%:%424=192%:%
%:%425=192%:%
%:%426=192%:%
%:%427=192%:%
%:%428=193%:%
%:%429=193%:%
%:%430=193%:%
%:%431=193%:%
%:%432=194%:%
%:%433=194%:%
%:%434=195%:%
%:%444=198%:%
%:%445=199%:%
%:%447=201%:%
%:%448=201%:%
%:%449=202%:%
%:%450=203%:%
%:%453=204%:%
%:%457=204%:%
%:%458=204%:%
%:%459=204%:%
%:%460=204%:%
%:%465=204%:%
%:%468=205%:%
%:%469=206%:%
%:%470=206%:%
%:%473=207%:%
%:%477=207%:%
%:%478=207%:%
%:%479=207%:%
%:%484=207%:%
%:%487=208%:%
%:%488=209%:%
%:%489=209%:%
%:%492=210%:%
%:%496=210%:%
%:%497=210%:%
%:%498=210%:%
%:%503=210%:%
%:%506=211%:%
%:%507=212%:%
%:%508=213%:%
%:%509=213%:%
%:%516=214%:%
%:%517=214%:%
%:%518=215%:%
%:%519=215%:%
%:%520=216%:%
%:%521=216%:%
%:%522=217%:%
%:%523=217%:%
%:%524=217%:%
%:%525=218%:%
%:%526=218%:%
%:%527=218%:%
%:%528=219%:%
%:%529=219%:%
%:%530=219%:%
%:%531=219%:%
%:%532=220%:%
%:%533=220%:%
%:%534=220%:%
%:%535=220%:%
%:%536=221%:%
%:%537=221%:%
%:%538=221%:%
%:%539=221%:%
%:%540=222%:%
%:%541=222%:%
%:%542=222%:%
%:%543=222%:%
%:%544=222%:%
%:%545=223%:%
%:%546=223%:%
%:%547=223%:%
%:%548=223%:%
%:%549=224%:%
%:%550=224%:%
%:%551=224%:%
%:%552=224%:%
%:%553=225%:%
%:%563=228%:%
%:%564=229%:%
%:%566=231%:%
%:%567=231%:%
%:%568=232%:%
%:%569=233%:%
%:%572=234%:%
%:%576=234%:%
%:%577=234%:%
%:%578=234%:%
%:%579=234%:%
%:%584=234%:%
%:%587=235%:%
%:%588=236%:%
%:%589=236%:%
%:%592=237%:%
%:%596=237%:%
%:%597=237%:%
%:%598=237%:%
%:%603=237%:%
%:%606=238%:%
%:%607=239%:%
%:%608=239%:%
%:%609=240%:%
%:%610=241%:%
%:%611=242%:%
%:%618=243%:%
%:%619=243%:%
%:%620=243%:%
%:%621=244%:%
%:%622=244%:%
%:%623=245%:%
%:%624=245%:%
%:%625=245%:%
%:%626=245%:%
%:%627=245%:%
%:%628=246%:%
%:%629=246%:%
%:%630=247%:%
%:%631=247%:%
%:%632=248%:%
%:%633=248%:%
%:%634=248%:%
%:%635=248%:%
%:%636=249%:%
%:%637=249%:%
%:%638=249%:%
%:%639=249%:%
%:%640=250%:%
%:%641=250%:%
%:%642=250%:%
%:%643=250%:%
%:%644=250%:%
%:%645=251%:%
%:%646=251%:%
%:%647=251%:%
%:%648=251%:%
%:%649=251%:%
%:%650=252%:%
%:%656=252%:%
%:%659=253%:%
%:%660=254%:%
%:%661=254%:%
%:%662=255%:%
%:%663=256%:%
%:%670=257%:%
%:%671=257%:%
%:%672=258%:%
%:%673=258%:%
%:%674=259%:%
%:%675=259%:%
%:%676=260%:%
%:%677=260%:%
%:%678=261%:%
%:%679=261%:%
%:%680=262%:%
%:%681=262%:%
%:%682=262%:%
%:%683=263%:%
%:%684=263%:%
%:%685=263%:%
%:%686=264%:%
%:%687=264%:%
%:%688=265%:%
%:%689=265%:%
%:%690=265%:%
%:%691=265%:%
%:%692=265%:%
%:%693=266%:%
%:%694=266%:%
%:%695=266%:%
%:%696=266%:%
%:%697=267%:%
%:%698=267%:%
%:%699=268%:%
%:%700=268%:%
%:%701=269%:%
%:%702=269%:%
%:%703=269%:%
%:%704=269%:%
%:%705=270%:%
%:%706=270%:%
%:%707=271%:%
%:%708=271%:%
%:%709=272%:%
%:%710=272%:%
%:%711=272%:%
%:%712=272%:%
%:%713=272%:%
%:%714=273%:%
%:%715=273%:%
%:%716=274%:%
%:%726=277%:%
%:%727=278%:%
%:%729=279%:%
%:%730=279%:%
%:%731=280%:%
%:%732=281%:%
%:%739=282%:%
%:%740=282%:%
%:%741=283%:%
%:%742=283%:%
%:%743=284%:%
%:%744=284%:%
%:%745=284%:%
%:%746=284%:%
%:%747=285%:%
%:%748=285%:%
%:%749=285%:%
%:%750=285%:%
%:%751=285%:%
%:%752=286%:%
%:%753=286%:%
%:%754=286%:%
%:%755=287%:%
%:%756=288%:%
%:%757=288%:%
%:%758=288%:%
%:%759=289%:%
%:%760=289%:%
%:%761=289%:%
%:%762=290%:%
%:%763=290%:%
%:%764=291%:%
%:%765=291%:%
%:%766=292%:%
%:%767=292%:%
%:%768=293%:%
%:%769=293%:%
%:%770=293%:%
%:%771=293%:%
%:%772=294%:%
%:%773=294%:%
%:%774=294%:%
%:%775=294%:%
%:%776=295%:%
%:%777=295%:%
%:%778=295%:%
%:%779=295%:%
%:%780=296%:%
%:%781=296%:%
%:%782=296%:%
%:%783=296%:%
%:%784=296%:%
%:%785=297%:%
%:%786=297%:%
%:%787=298%:%
%:%788=298%:%
%:%789=298%:%
%:%790=299%:%
%:%791=299%:%
%:%792=300%:%
%:%793=300%:%
%:%794=301%:%
%:%795=301%:%
%:%796=302%:%
%:%797=302%:%
%:%798=303%:%
%:%799=303%:%
%:%800=304%:%
%:%801=304%:%
%:%802=305%:%
%:%803=305%:%
%:%804=306%:%
%:%805=306%:%
%:%806=306%:%
%:%807=306%:%
%:%808=307%:%
%:%809=307%:%
%:%810=307%:%
%:%811=307%:%
%:%812=308%:%
%:%813=308%:%
%:%814=309%:%
%:%815=309%:%
%:%816=309%:%
%:%817=309%:%
%:%818=309%:%
%:%819=310%:%
%:%820=310%:%
%:%821=310%:%
%:%822=310%:%
%:%823=311%:%
%:%824=311%:%
%:%825=312%:%
%:%826=312%:%
%:%827=312%:%
%:%828=312%:%
%:%829=313%:%
%:%839=316%:%
%:%840=317%:%
%:%842=319%:%
%:%843=319%:%
%:%847=323%:%
%:%848=324%:%
%:%849=325%:%
%:%850=325%:%
%:%857=326%:%
%:%858=326%:%
%:%859=327%:%
%:%860=327%:%
%:%861=327%:%
%:%862=328%:%
%:%863=328%:%
%:%864=328%:%
%:%865=329%:%
%:%866=329%:%
%:%867=329%:%
%:%868=330%:%
%:%869=331%:%
%:%870=331%:%
%:%871=332%:%
%:%872=332%:%
%:%873=332%:%
%:%874=333%:%
%:%875=334%:%
%:%876=334%:%
%:%877=334%:%
%:%878=335%:%
%:%879=335%:%
%:%880=336%:%
%:%881=337%:%
%:%882=337%:%
%:%883=337%:%
%:%884=338%:%
%:%894=341%:%
%:%895=342%:%
%:%896=343%:%
%:%898=345%:%
%:%899=345%:%
%:%900=346%:%
%:%901=347%:%
%:%902=347%:%
%:%903=348%:%
%:%904=349%:%
%:%911=350%:%
%:%912=350%:%
%:%913=350%:%
%:%914=351%:%
%:%915=351%:%
%:%916=352%:%
%:%917=352%:%
%:%918=353%:%
%:%919=353%:%
%:%920=353%:%
%:%921=353%:%
%:%922=354%:%
%:%923=354%:%
%:%924=354%:%
%:%925=355%:%
%:%926=355%:%
%:%927=355%:%
%:%928=356%:%
%:%929=356%:%
%:%930=356%:%
%:%931=357%:%
%:%932=357%:%
%:%933=358%:%
%:%934=358%:%
%:%935=358%:%
%:%936=359%:%
%:%937=359%:%
%:%938=359%:%
%:%939=360%:%
%:%940=360%:%
%:%941=360%:%
%:%942=361%:%
%:%943=361%:%
%:%944=361%:%
%:%945=361%:%
%:%946=362%:%
%:%947=362%:%
%:%948=362%:%
%:%949=362%:%
%:%950=363%:%
%:%951=363%:%
%:%952=363%:%
%:%953=363%:%
%:%954=363%:%
%:%955=363%:%
%:%956=364%:%
%:%957=364%:%
%:%958=364%:%
%:%959=364%:%
%:%960=364%:%
%:%961=365%:%
%:%967=365%:%
%:%970=366%:%
%:%971=367%:%
%:%972=367%:%
%:%973=368%:%
%:%974=369%:%
%:%975=370%:%
%:%982=371%:%
%:%983=371%:%
%:%984=372%:%
%:%985=372%:%
%:%986=373%:%
%:%987=373%:%
%:%988=374%:%
%:%989=374%:%
%:%990=375%:%
%:%991=375%:%
%:%992=375%:%
%:%993=376%:%
%:%994=376%:%
%:%995=376%:%
%:%996=377%:%
%:%997=377%:%
%:%998=377%:%
%:%999=378%:%
%:%1000=378%:%
%:%1001=378%:%
%:%1002=379%:%
%:%1003=379%:%
%:%1004=379%:%
%:%1005=379%:%
%:%1006=380%:%
%:%1007=380%:%
%:%1008=380%:%
%:%1009=381%:%
%:%1010=381%:%
%:%1011=381%:%
%:%1012=381%:%
%:%1013=382%:%
%:%1014=382%:%
%:%1015=383%:%
%:%1016=383%:%
%:%1017=384%:%
%:%1018=384%:%
%:%1019=384%:%
%:%1020=384%:%
%:%1021=384%:%
%:%1022=385%:%
%:%1023=385%:%
%:%1024=385%:%
%:%1025=385%:%
%:%1026=385%:%
%:%1027=386%:%
%:%1028=386%:%
%:%1029=387%:%
%:%1030=387%:%
%:%1031=387%:%
%:%1032=387%:%
%:%1033=388%:%
%:%1034=388%:%
%:%1035=388%:%
%:%1036=388%:%
%:%1037=389%:%
%:%1043=389%:%
%:%1046=390%:%
%:%1047=391%:%
%:%1048=391%:%
%:%1049=392%:%
%:%1050=393%:%
%:%1051=394%:%
%:%1058=395%:%
%:%1059=395%:%
%:%1060=396%:%
%:%1061=396%:%
%:%1062=397%:%
%:%1063=397%:%
%:%1064=397%:%
%:%1065=397%:%
%:%1066=398%:%
%:%1067=398%:%
%:%1068=398%:%
%:%1069=398%:%
%:%1070=398%:%
%:%1071=399%:%
%:%1072=399%:%
%:%1073=399%:%
%:%1074=399%:%
%:%1075=399%:%
%:%1076=399%:%
%:%1077=400%:%
%:%1078=400%:%
%:%1079=401%:%
%:%1080=401%:%
%:%1081=402%:%
%:%1082=402%:%
%:%1083=402%:%
%:%1084=402%:%
%:%1085=403%:%
%:%1095=406%:%
%:%1096=407%:%
%:%1097=408%:%
%:%1099=409%:%
%:%1100=409%:%
%:%1101=410%:%
%:%1102=411%:%
%:%1109=412%:%
%:%1110=412%:%
%:%1111=413%:%
%:%1112=413%:%
%:%1113=413%:%
%:%1114=413%:%
%:%1115=414%:%
%:%1116=414%:%
%:%1117=414%:%
%:%1118=414%:%
%:%1119=415%:%
%:%1120=415%:%
%:%1121=415%:%
%:%1122=415%:%
%:%1123=415%:%
%:%1124=416%:%
%:%1125=416%:%
%:%1126=416%:%
%:%1127=416%:%
%:%1128=417%:%
%:%1129=417%:%
%:%1130=418%:%
%:%1131=418%:%
%:%1132=418%:%
%:%1133=419%:%
%:%1134=419%:%
%:%1135=419%:%
%:%1136=419%:%
%:%1137=420%:%
%:%1138=420%:%
%:%1139=420%:%
%:%1140=420%:%
%:%1141=421%:%
%:%1147=421%:%
%:%1150=422%:%
%:%1151=423%:%
%:%1159=426%:%
%:%1171=428%:%
%:%1172=429%:%
%:%1173=430%:%
%:%1174=431%:%
%:%1175=432%:%
%:%1176=433%:%
%:%1177=434%:%
%:%1179=435%:%
%:%1180=435%:%
%:%1181=436%:%
%:%1182=437%:%
%:%1189=438%:%
%:%1190=438%:%
%:%1191=439%:%
%:%1192=439%:%
%:%1193=440%:%
%:%1194=440%:%
%:%1195=441%:%
%:%1196=441%:%
%:%1197=441%:%
%:%1198=442%:%
%:%1199=442%:%
%:%1200=442%:%
%:%1201=443%:%
%:%1202=443%:%
%:%1203=443%:%
%:%1204=444%:%
%:%1205=444%:%
%:%1206=445%:%
%:%1207=445%:%
%:%1208=446%:%
%:%1209=446%:%
%:%1210=446%:%
%:%1211=447%:%
%:%1212=447%:%
%:%1213=447%:%
%:%1214=448%:%
%:%1215=448%:%
%:%1216=448%:%
%:%1217=448%:%
%:%1218=449%:%
%:%1219=449%:%
%:%1220=449%:%
%:%1221=450%:%
%:%1222=450%:%
%:%1223=450%:%
%:%1224=450%:%
%:%1225=451%:%
%:%1226=451%:%
%:%1227=451%:%
%:%1228=451%:%
%:%1229=451%:%
%:%1230=452%:%
%:%1231=452%:%
%:%1232=453%:%
%:%1233=453%:%
%:%1234=453%:%
%:%1235=454%:%
%:%1236=454%:%
%:%1237=455%:%
%:%1247=457%:%
%:%1248=458%:%
%:%1249=459%:%
%:%1251=460%:%
%:%1252=460%:%
%:%1253=461%:%
%:%1254=462%:%
%:%1261=463%:%
%:%1262=463%:%
%:%1263=463%:%
%:%1264=464%:%
%:%1265=464%:%
%:%1266=464%:%
%:%1267=465%:%
%:%1268=465%:%
%:%1269=466%:%
%:%1270=466%:%
%:%1271=467%:%
%:%1272=467%:%
%:%1273=468%:%
%:%1274=468%:%
%:%1275=469%:%
%:%1276=469%:%
%:%1277=470%:%
%:%1278=470%:%
%:%1279=471%:%
%:%1280=471%:%
%:%1281=471%:%
%:%1282=472%:%
%:%1283=472%:%
%:%1284=473%:%
%:%1285=473%:%
%:%1286=473%:%
%:%1287=473%:%
%:%1288=473%:%
%:%1289=474%:%
%:%1290=474%:%
%:%1291=474%:%
%:%1292=475%:%
%:%1293=475%:%
%:%1294=475%:%
%:%1295=476%:%
%:%1296=476%:%
%:%1297=477%:%
%:%1307=479%:%
%:%1309=480%:%
%:%1310=480%:%
%:%1311=481%:%
%:%1312=482%:%
%:%1313=483%:%
%:%1320=484%:%
%:%1321=484%:%
%:%1322=485%:%
%:%1323=485%:%
%:%1324=485%:%
%:%1325=486%:%
%:%1326=486%:%
%:%1327=486%:%
%:%1328=487%:%
%:%1329=487%:%
%:%1330=487%:%
%:%1331=487%:%
%:%1332=488%:%
%:%1333=488%:%
%:%1334=488%:%
%:%1335=489%:%
%:%1336=489%:%
%:%1337=489%:%
%:%1338=489%:%
%:%1339=490%:%
%:%1345=490%:%
%:%1350=491%:%
%:%1355=492%:%

%
\begin{isabellebody}%
\setisabellecontext{Pilling}%
%
\isadelimdocument
%
\endisadelimdocument
%
\isatagdocument
%
\isamarkupsection{Pilling's proof of Parikh's theorem%
}
\isamarkuptrue%
%
\endisatagdocument
{\isafolddocument}%
%
\isadelimdocument
%
\endisadelimdocument
%
\isadelimtheory
%
\endisadelimtheory
%
\isatagtheory
\isakeywordONE{theory}\isamarkupfalse%
\ Pilling\isanewline
\ \ \isakeywordTWO{imports}\ \isanewline
\ \ \ \ {\isachardoublequoteopen}Reg{\isacharunderscore}{\kern0pt}Lang{\isacharunderscore}{\kern0pt}Exp{\isacharunderscore}{\kern0pt}Eqns{\isachardoublequoteclose}\isanewline
\isakeywordTWO{begin}%
\endisatagtheory
{\isafoldtheory}%
%
\isadelimtheory
%
\endisadelimtheory
%
\begin{isamarkuptext}%
We prove Parikh's theorem, closely following Pilling's proof \cite{Pilling}. The rough
idea is as follows: As seen above, each CFG can be interpreted as a system of equations of the
first type and we can easily convert it into a system of the second type by applying the Parikh
image on both sides of each equation. Pilling now shows that there is a regular solution to this
system and that this solution is furthermore minimal.
Using the relations explored in the last section we prove that the CFG's language is a minimal
solution of the same sytem and hence that the Parikh image of the CFG's language and of the regular
solution must be identical; this finishes the proof of Parikh's theorem.

Notably, while in \cite{Pilling} Pilling proves an auxiliary lemma first and applies this lemma in
the proof of the main theorem, we were able to complete the whole proof without using the lemma.%
\end{isamarkuptext}\isamarkuptrue%
%
\isadelimdocument
%
\endisadelimdocument
%
\isatagdocument
%
\isamarkupsubsection{Special representation of regular language expressions%
}
\isamarkuptrue%
%
\endisatagdocument
{\isafolddocument}%
%
\isadelimdocument
%
\endisadelimdocument
%
\begin{isamarkuptext}%
To each regular language expression and variable \isa{x} corresponds a second regular language
expression with the same Parikh image and of the form depicted in equation (3) in \cite{Pilling}.
We call regular language expressions of this form "bipartite regular language expressions" since
they decompose into two subexpressions where one of them contains the variable \isa{x} and the other
one does not:%
\end{isamarkuptext}\isamarkuptrue%
\isakeywordONE{definition}\isamarkupfalse%
\ bipart{\isacharunderscore}{\kern0pt}rlexp\ {\isacharcolon}{\kern0pt}{\isacharcolon}{\kern0pt}\ {\isachardoublequoteopen}nat\ {\isasymRightarrow}\ {\isacharprime}{\kern0pt}a\ rlexp\ {\isasymRightarrow}\ bool{\isachardoublequoteclose}\ \isakeywordTWO{where}\isanewline
\ \ {\isachardoublequoteopen}bipart{\isacharunderscore}{\kern0pt}rlexp\ x\ f\ {\isasymequiv}\ {\isasymexists}p\ q{\isachardot}{\kern0pt}\ reg{\isacharunderscore}{\kern0pt}eval\ p\ {\isasymand}\ reg{\isacharunderscore}{\kern0pt}eval\ q\ {\isasymand}\isanewline
\ \ \ \ f\ {\isacharequal}{\kern0pt}\ Union\ p\ {\isacharparenleft}{\kern0pt}Concat\ q\ {\isacharparenleft}{\kern0pt}Var\ x{\isacharparenright}{\kern0pt}{\isacharparenright}{\kern0pt}\ {\isasymand}\ x\ {\isasymnotin}\ vars\ p{\isachardoublequoteclose}%
\begin{isamarkuptext}%
All bipartite regular language expressions evaluate to regular languages. Additionally,
for each \isa{\isaconst{reg{\isacharunderscore}{\kern0pt}eval}} regular language expression and variable \isa{x}, there exists a bipartite
regular language expression with identical Parikh image and almost identical set of variables.
While the first proof is simple, the second one is more complex and needs the results of the
sections 2.3 and 2.4:%
\end{isamarkuptext}\isamarkuptrue%
\isakeywordONE{lemma}\isamarkupfalse%
\ {\isachardoublequoteopen}bipart{\isacharunderscore}{\kern0pt}rlexp\ x\ f\ {\isasymLongrightarrow}\ reg{\isacharunderscore}{\kern0pt}eval\ f{\isachardoublequoteclose}\isanewline
%
\isadelimproof
\ \ %
\endisadelimproof
%
\isatagproof
\isakeywordONE{unfolding}\isamarkupfalse%
\ bipart{\isacharunderscore}{\kern0pt}rlexp{\isacharunderscore}{\kern0pt}def\ \isakeywordONE{by}\isamarkupfalse%
\ fastforce%
\endisatagproof
{\isafoldproof}%
%
\isadelimproof
\isanewline
%
\endisadelimproof
\isanewline
\isanewline
\isakeywordONE{lemma}\isamarkupfalse%
\ reg{\isacharunderscore}{\kern0pt}eval{\isacharunderscore}{\kern0pt}bipart{\isacharunderscore}{\kern0pt}rlexp{\isacharunderscore}{\kern0pt}Variable{\isacharcolon}{\kern0pt}\ {\isachardoublequoteopen}{\isasymexists}f{\isacharprime}{\kern0pt}{\isachardot}{\kern0pt}\ bipart{\isacharunderscore}{\kern0pt}rlexp\ x\ f{\isacharprime}{\kern0pt}\ {\isasymand}\ vars\ f{\isacharprime}{\kern0pt}\ {\isacharequal}{\kern0pt}\ vars\ {\isacharparenleft}{\kern0pt}Var\ y{\isacharparenright}{\kern0pt}\ {\isasymunion}\ {\isacharbraceleft}{\kern0pt}x{\isacharbraceright}{\kern0pt}\isanewline
\ \ \ \ \ \ \ \ \ \ \ \ \ \ \ \ \ \ \ \ \ \ \ \ \ \ \ \ \ \ \ \ \ \ \ \ \ \ \ \ {\isasymand}\ {\isacharparenleft}{\kern0pt}{\isasymforall}v{\isachardot}{\kern0pt}\ {\isasymPsi}\ {\isacharparenleft}{\kern0pt}eval\ {\isacharparenleft}{\kern0pt}Var\ y{\isacharparenright}{\kern0pt}\ v{\isacharparenright}{\kern0pt}\ {\isacharequal}{\kern0pt}\ {\isasymPsi}\ {\isacharparenleft}{\kern0pt}eval\ f{\isacharprime}{\kern0pt}\ v{\isacharparenright}{\kern0pt}{\isacharparenright}{\kern0pt}{\isachardoublequoteclose}\isanewline
%
\isadelimproof
%
\endisadelimproof
%
\isatagproof
\isakeywordONE{proof}\isamarkupfalse%
\ {\isacharparenleft}{\kern0pt}cases\ {\isachardoublequoteopen}x\ {\isacharequal}{\kern0pt}\ y{\isachardoublequoteclose}{\isacharparenright}{\kern0pt}\isanewline
\isakeywordONE{let}\isamarkupfalse%
\ {\isacharquery}{\kern0pt}f{\isacharprime}{\kern0pt}\ {\isacharequal}{\kern0pt}\ {\isachardoublequoteopen}Union\ {\isacharparenleft}{\kern0pt}Const\ {\isacharbraceleft}{\kern0pt}{\isacharbraceright}{\kern0pt}{\isacharparenright}{\kern0pt}\ {\isacharparenleft}{\kern0pt}Concat\ {\isacharparenleft}{\kern0pt}Const\ {\isacharbraceleft}{\kern0pt}{\isacharbrackleft}{\kern0pt}{\isacharbrackright}{\kern0pt}{\isacharbraceright}{\kern0pt}{\isacharparenright}{\kern0pt}\ {\isacharparenleft}{\kern0pt}Var\ x{\isacharparenright}{\kern0pt}{\isacharparenright}{\kern0pt}{\isachardoublequoteclose}\isanewline
\ \ \isakeywordTHREE{case}\isamarkupfalse%
\ True\isanewline
\ \ \isakeywordONE{then}\isamarkupfalse%
\ \isakeywordONE{have}\isamarkupfalse%
\ {\isachardoublequoteopen}bipart{\isacharunderscore}{\kern0pt}rlexp\ x\ {\isacharquery}{\kern0pt}f{\isacharprime}{\kern0pt}{\isachardoublequoteclose}\isanewline
\ \ \ \ \isakeywordONE{unfolding}\isamarkupfalse%
\ bipart{\isacharunderscore}{\kern0pt}rlexp{\isacharunderscore}{\kern0pt}def\ \isakeywordONE{using}\isamarkupfalse%
\ emptyset{\isacharunderscore}{\kern0pt}regular\ epsilon{\isacharunderscore}{\kern0pt}regular\ \isakeywordONE{by}\isamarkupfalse%
\ fastforce\isanewline
\ \ \isakeywordONE{moreover}\isamarkupfalse%
\ \isakeywordONE{have}\isamarkupfalse%
\ {\isachardoublequoteopen}eval\ {\isacharquery}{\kern0pt}f{\isacharprime}{\kern0pt}\ v\ {\isacharequal}{\kern0pt}\ eval\ {\isacharparenleft}{\kern0pt}Var\ y{\isacharparenright}{\kern0pt}\ v{\isachardoublequoteclose}\ \isakeywordTWO{for}\ v\ {\isacharcolon}{\kern0pt}{\isacharcolon}{\kern0pt}\ {\isachardoublequoteopen}{\isacharprime}{\kern0pt}a\ valuation{\isachardoublequoteclose}\ \isakeywordONE{using}\isamarkupfalse%
\ True\ \isakeywordONE{by}\isamarkupfalse%
\ simp\isanewline
\ \ \isakeywordONE{moreover}\isamarkupfalse%
\ \isakeywordONE{have}\isamarkupfalse%
\ {\isachardoublequoteopen}vars\ {\isacharquery}{\kern0pt}f{\isacharprime}{\kern0pt}\ {\isacharequal}{\kern0pt}\ vars\ {\isacharparenleft}{\kern0pt}Var\ y{\isacharparenright}{\kern0pt}\ {\isasymunion}\ {\isacharbraceleft}{\kern0pt}x{\isacharbraceright}{\kern0pt}{\isachardoublequoteclose}\ \isakeywordONE{using}\isamarkupfalse%
\ True\ \isakeywordONE{by}\isamarkupfalse%
\ simp\isanewline
\ \ \isakeywordONE{ultimately}\isamarkupfalse%
\ \isakeywordTHREE{show}\isamarkupfalse%
\ {\isacharquery}{\kern0pt}thesis\ \isakeywordONE{by}\isamarkupfalse%
\ metis\isanewline
\isakeywordONE{next}\isamarkupfalse%
\isanewline
\ \ \isakeywordONE{let}\isamarkupfalse%
\ {\isacharquery}{\kern0pt}f{\isacharprime}{\kern0pt}\ {\isacharequal}{\kern0pt}\ {\isachardoublequoteopen}Union\ {\isacharparenleft}{\kern0pt}Var\ y{\isacharparenright}{\kern0pt}\ {\isacharparenleft}{\kern0pt}Concat\ {\isacharparenleft}{\kern0pt}Const\ {\isacharbraceleft}{\kern0pt}{\isacharbraceright}{\kern0pt}{\isacharparenright}{\kern0pt}\ {\isacharparenleft}{\kern0pt}Var\ x{\isacharparenright}{\kern0pt}{\isacharparenright}{\kern0pt}{\isachardoublequoteclose}\isanewline
\ \ \isakeywordTHREE{case}\isamarkupfalse%
\ False\isanewline
\ \ \isakeywordONE{then}\isamarkupfalse%
\ \isakeywordONE{have}\isamarkupfalse%
\ {\isachardoublequoteopen}bipart{\isacharunderscore}{\kern0pt}rlexp\ x\ {\isacharquery}{\kern0pt}f{\isacharprime}{\kern0pt}{\isachardoublequoteclose}\isanewline
\ \ \ \ \isakeywordONE{unfolding}\isamarkupfalse%
\ bipart{\isacharunderscore}{\kern0pt}rlexp{\isacharunderscore}{\kern0pt}def\ \isakeywordONE{using}\isamarkupfalse%
\ emptyset{\isacharunderscore}{\kern0pt}regular\ epsilon{\isacharunderscore}{\kern0pt}regular\ \isakeywordONE{by}\isamarkupfalse%
\ fastforce\isanewline
\ \ \isakeywordONE{moreover}\isamarkupfalse%
\ \isakeywordONE{have}\isamarkupfalse%
\ {\isachardoublequoteopen}eval\ {\isacharquery}{\kern0pt}f{\isacharprime}{\kern0pt}\ v\ {\isacharequal}{\kern0pt}\ eval\ {\isacharparenleft}{\kern0pt}Var\ y{\isacharparenright}{\kern0pt}\ v{\isachardoublequoteclose}\ \isakeywordTWO{for}\ v\ {\isacharcolon}{\kern0pt}{\isacharcolon}{\kern0pt}\ {\isachardoublequoteopen}{\isacharprime}{\kern0pt}a\ valuation{\isachardoublequoteclose}\ \isakeywordONE{using}\isamarkupfalse%
\ False\ \isakeywordONE{by}\isamarkupfalse%
\ simp\isanewline
\ \ \isakeywordONE{moreover}\isamarkupfalse%
\ \isakeywordONE{have}\isamarkupfalse%
\ {\isachardoublequoteopen}vars\ {\isacharquery}{\kern0pt}f{\isacharprime}{\kern0pt}\ {\isacharequal}{\kern0pt}\ vars\ {\isacharparenleft}{\kern0pt}Var\ y{\isacharparenright}{\kern0pt}\ {\isasymunion}\ {\isacharbraceleft}{\kern0pt}x{\isacharbraceright}{\kern0pt}{\isachardoublequoteclose}\ \isakeywordONE{by}\isamarkupfalse%
\ simp\isanewline
\ \ \isakeywordONE{ultimately}\isamarkupfalse%
\ \isakeywordTHREE{show}\isamarkupfalse%
\ {\isacharquery}{\kern0pt}thesis\ \isakeywordONE{by}\isamarkupfalse%
\ metis\isanewline
\isakeywordONE{qed}\isamarkupfalse%
%
\endisatagproof
{\isafoldproof}%
%
\isadelimproof
\isanewline
%
\endisadelimproof
\isanewline
\isakeywordONE{lemma}\isamarkupfalse%
\ reg{\isacharunderscore}{\kern0pt}eval{\isacharunderscore}{\kern0pt}bipart{\isacharunderscore}{\kern0pt}rlexp{\isacharunderscore}{\kern0pt}Const{\isacharcolon}{\kern0pt}\isanewline
\ \ \isakeywordTWO{assumes}\ {\isachardoublequoteopen}regular{\isacharunderscore}{\kern0pt}lang\ l{\isachardoublequoteclose}\isanewline
\ \ \ \ \isakeywordTWO{shows}\ {\isachardoublequoteopen}{\isasymexists}f{\isacharprime}{\kern0pt}{\isachardot}{\kern0pt}\ bipart{\isacharunderscore}{\kern0pt}rlexp\ x\ f{\isacharprime}{\kern0pt}\ {\isasymand}\ vars\ f{\isacharprime}{\kern0pt}\ {\isacharequal}{\kern0pt}\ vars\ {\isacharparenleft}{\kern0pt}Const\ l{\isacharparenright}{\kern0pt}\ {\isasymunion}\ {\isacharbraceleft}{\kern0pt}x{\isacharbraceright}{\kern0pt}\isanewline
\ \ \ \ \ \ \ \ \ \ \ \ \ \ \ \ {\isasymand}\ {\isacharparenleft}{\kern0pt}{\isasymforall}v{\isachardot}{\kern0pt}\ {\isasymPsi}\ {\isacharparenleft}{\kern0pt}eval\ {\isacharparenleft}{\kern0pt}Const\ l{\isacharparenright}{\kern0pt}\ v{\isacharparenright}{\kern0pt}\ {\isacharequal}{\kern0pt}\ {\isasymPsi}\ {\isacharparenleft}{\kern0pt}eval\ f{\isacharprime}{\kern0pt}\ v{\isacharparenright}{\kern0pt}{\isacharparenright}{\kern0pt}{\isachardoublequoteclose}\isanewline
%
\isadelimproof
%
\endisadelimproof
%
\isatagproof
\isakeywordONE{proof}\isamarkupfalse%
\ {\isacharminus}{\kern0pt}\isanewline
\ \ \isakeywordONE{let}\isamarkupfalse%
\ {\isacharquery}{\kern0pt}f{\isacharprime}{\kern0pt}\ {\isacharequal}{\kern0pt}\ {\isachardoublequoteopen}Union\ {\isacharparenleft}{\kern0pt}Const\ l{\isacharparenright}{\kern0pt}\ {\isacharparenleft}{\kern0pt}Concat\ {\isacharparenleft}{\kern0pt}Const\ {\isacharbraceleft}{\kern0pt}{\isacharbraceright}{\kern0pt}{\isacharparenright}{\kern0pt}\ {\isacharparenleft}{\kern0pt}Var\ x{\isacharparenright}{\kern0pt}{\isacharparenright}{\kern0pt}{\isachardoublequoteclose}\isanewline
\ \ \isakeywordONE{have}\isamarkupfalse%
\ {\isachardoublequoteopen}bipart{\isacharunderscore}{\kern0pt}rlexp\ x\ {\isacharquery}{\kern0pt}f{\isacharprime}{\kern0pt}{\isachardoublequoteclose}\isanewline
\ \ \ \ \isakeywordONE{unfolding}\isamarkupfalse%
\ bipart{\isacharunderscore}{\kern0pt}rlexp{\isacharunderscore}{\kern0pt}def\ \isakeywordONE{using}\isamarkupfalse%
\ assms\ emptyset{\isacharunderscore}{\kern0pt}regular\ \isakeywordONE{by}\isamarkupfalse%
\ simp\isanewline
\ \ \isakeywordONE{moreover}\isamarkupfalse%
\ \isakeywordONE{have}\isamarkupfalse%
\ {\isachardoublequoteopen}eval\ {\isacharquery}{\kern0pt}f{\isacharprime}{\kern0pt}\ v\ {\isacharequal}{\kern0pt}\ eval\ {\isacharparenleft}{\kern0pt}Const\ l{\isacharparenright}{\kern0pt}\ v{\isachardoublequoteclose}\ \isakeywordTWO{for}\ v\ {\isacharcolon}{\kern0pt}{\isacharcolon}{\kern0pt}\ {\isachardoublequoteopen}{\isacharprime}{\kern0pt}a\ valuation{\isachardoublequoteclose}\ \isakeywordONE{by}\isamarkupfalse%
\ simp\isanewline
\ \ \isakeywordONE{moreover}\isamarkupfalse%
\ \isakeywordONE{have}\isamarkupfalse%
\ {\isachardoublequoteopen}vars\ {\isacharquery}{\kern0pt}f{\isacharprime}{\kern0pt}\ {\isacharequal}{\kern0pt}\ vars\ {\isacharparenleft}{\kern0pt}Const\ l{\isacharparenright}{\kern0pt}\ {\isasymunion}\ {\isacharbraceleft}{\kern0pt}x{\isacharbraceright}{\kern0pt}{\isachardoublequoteclose}\ \isakeywordONE{by}\isamarkupfalse%
\ simp\ \isanewline
\ \ \isakeywordONE{ultimately}\isamarkupfalse%
\ \isakeywordTHREE{show}\isamarkupfalse%
\ {\isacharquery}{\kern0pt}thesis\ \isakeywordONE{by}\isamarkupfalse%
\ metis\isanewline
\isakeywordONE{qed}\isamarkupfalse%
%
\endisatagproof
{\isafoldproof}%
%
\isadelimproof
\isanewline
%
\endisadelimproof
\isanewline
\isakeywordONE{lemma}\isamarkupfalse%
\ reg{\isacharunderscore}{\kern0pt}eval{\isacharunderscore}{\kern0pt}bipart{\isacharunderscore}{\kern0pt}rlexp{\isacharunderscore}{\kern0pt}Union{\isacharcolon}{\kern0pt}\isanewline
\ \ \isakeywordTWO{assumes}\ {\isachardoublequoteopen}{\isasymexists}f{\isacharprime}{\kern0pt}{\isachardot}{\kern0pt}\ bipart{\isacharunderscore}{\kern0pt}rlexp\ x\ f{\isacharprime}{\kern0pt}\ {\isasymand}\ vars\ f{\isacharprime}{\kern0pt}\ {\isacharequal}{\kern0pt}\ vars\ f{\isadigit{1}}\ {\isasymunion}\ {\isacharbraceleft}{\kern0pt}x{\isacharbraceright}{\kern0pt}\ {\isasymand}\isanewline
\ \ \ \ \ \ \ \ \ \ \ \ \ \ \ \ {\isacharparenleft}{\kern0pt}{\isasymforall}v{\isachardot}{\kern0pt}\ {\isasymPsi}\ {\isacharparenleft}{\kern0pt}eval\ f{\isadigit{1}}\ v{\isacharparenright}{\kern0pt}\ {\isacharequal}{\kern0pt}\ {\isasymPsi}\ {\isacharparenleft}{\kern0pt}eval\ f{\isacharprime}{\kern0pt}\ v{\isacharparenright}{\kern0pt}{\isacharparenright}{\kern0pt}{\isachardoublequoteclose}\isanewline
\ \ \ \ \ \ \ \ \ \ {\isachardoublequoteopen}{\isasymexists}f{\isacharprime}{\kern0pt}{\isachardot}{\kern0pt}\ bipart{\isacharunderscore}{\kern0pt}rlexp\ x\ f{\isacharprime}{\kern0pt}\ {\isasymand}\ vars\ f{\isacharprime}{\kern0pt}\ {\isacharequal}{\kern0pt}\ vars\ f{\isadigit{2}}\ {\isasymunion}\ {\isacharbraceleft}{\kern0pt}x{\isacharbraceright}{\kern0pt}\ {\isasymand}\isanewline
\ \ \ \ \ \ \ \ \ \ \ \ \ \ \ \ {\isacharparenleft}{\kern0pt}{\isasymforall}v{\isachardot}{\kern0pt}\ {\isasymPsi}\ {\isacharparenleft}{\kern0pt}eval\ f{\isadigit{2}}\ v{\isacharparenright}{\kern0pt}\ {\isacharequal}{\kern0pt}\ {\isasymPsi}\ {\isacharparenleft}{\kern0pt}eval\ f{\isacharprime}{\kern0pt}\ v{\isacharparenright}{\kern0pt}{\isacharparenright}{\kern0pt}{\isachardoublequoteclose}\isanewline
\ \ \ \ \isakeywordTWO{shows}\ {\isachardoublequoteopen}{\isasymexists}f{\isacharprime}{\kern0pt}{\isachardot}{\kern0pt}\ bipart{\isacharunderscore}{\kern0pt}rlexp\ x\ f{\isacharprime}{\kern0pt}\ {\isasymand}\ vars\ f{\isacharprime}{\kern0pt}\ {\isacharequal}{\kern0pt}\ vars\ {\isacharparenleft}{\kern0pt}Union\ f{\isadigit{1}}\ f{\isadigit{2}}{\isacharparenright}{\kern0pt}\ {\isasymunion}\ {\isacharbraceleft}{\kern0pt}x{\isacharbraceright}{\kern0pt}\ {\isasymand}\isanewline
\ \ \ \ \ \ \ \ \ \ \ \ \ \ \ \ {\isacharparenleft}{\kern0pt}{\isasymforall}v{\isachardot}{\kern0pt}\ {\isasymPsi}\ {\isacharparenleft}{\kern0pt}eval\ {\isacharparenleft}{\kern0pt}Union\ f{\isadigit{1}}\ f{\isadigit{2}}{\isacharparenright}{\kern0pt}\ v{\isacharparenright}{\kern0pt}\ {\isacharequal}{\kern0pt}\ {\isasymPsi}\ {\isacharparenleft}{\kern0pt}eval\ f{\isacharprime}{\kern0pt}\ v{\isacharparenright}{\kern0pt}{\isacharparenright}{\kern0pt}{\isachardoublequoteclose}\isanewline
%
\isadelimproof
%
\endisadelimproof
%
\isatagproof
\isakeywordONE{proof}\isamarkupfalse%
\ {\isacharminus}{\kern0pt}\isanewline
\ \ \isakeywordONE{from}\isamarkupfalse%
\ assms\ \isakeywordTHREE{obtain}\isamarkupfalse%
\ f{\isadigit{1}}{\isacharprime}{\kern0pt}\ f{\isadigit{2}}{\isacharprime}{\kern0pt}\ \isakeywordTWO{where}\ f{\isadigit{1}}{\isacharprime}{\kern0pt}{\isacharunderscore}{\kern0pt}intro{\isacharcolon}{\kern0pt}\ {\isachardoublequoteopen}bipart{\isacharunderscore}{\kern0pt}rlexp\ x\ f{\isadigit{1}}{\isacharprime}{\kern0pt}\ {\isasymand}\ vars\ f{\isadigit{1}}{\isacharprime}{\kern0pt}\ {\isacharequal}{\kern0pt}\ vars\ f{\isadigit{1}}\ {\isasymunion}\ {\isacharbraceleft}{\kern0pt}x{\isacharbraceright}{\kern0pt}\ {\isasymand}\isanewline
\ \ \ \ \ \ {\isacharparenleft}{\kern0pt}{\isasymforall}v{\isachardot}{\kern0pt}\ {\isasymPsi}\ {\isacharparenleft}{\kern0pt}eval\ f{\isadigit{1}}\ v{\isacharparenright}{\kern0pt}\ {\isacharequal}{\kern0pt}\ {\isasymPsi}\ {\isacharparenleft}{\kern0pt}eval\ f{\isadigit{1}}{\isacharprime}{\kern0pt}\ v{\isacharparenright}{\kern0pt}{\isacharparenright}{\kern0pt}{\isachardoublequoteclose}\isanewline
\ \ \ \ \isakeywordTWO{and}\ f{\isadigit{2}}{\isacharprime}{\kern0pt}{\isacharunderscore}{\kern0pt}intro{\isacharcolon}{\kern0pt}\ {\isachardoublequoteopen}bipart{\isacharunderscore}{\kern0pt}rlexp\ x\ f{\isadigit{2}}{\isacharprime}{\kern0pt}\ {\isasymand}\ vars\ f{\isadigit{2}}{\isacharprime}{\kern0pt}\ {\isacharequal}{\kern0pt}\ vars\ f{\isadigit{2}}\ {\isasymunion}\ {\isacharbraceleft}{\kern0pt}x{\isacharbraceright}{\kern0pt}\ {\isasymand}\isanewline
\ \ \ \ \ \ {\isacharparenleft}{\kern0pt}{\isasymforall}v{\isachardot}{\kern0pt}\ {\isasymPsi}\ {\isacharparenleft}{\kern0pt}eval\ f{\isadigit{2}}\ v{\isacharparenright}{\kern0pt}\ {\isacharequal}{\kern0pt}\ {\isasymPsi}\ {\isacharparenleft}{\kern0pt}eval\ f{\isadigit{2}}{\isacharprime}{\kern0pt}\ v{\isacharparenright}{\kern0pt}{\isacharparenright}{\kern0pt}{\isachardoublequoteclose}\isanewline
\ \ \ \ \isakeywordONE{by}\isamarkupfalse%
\ auto\isanewline
\ \ \isakeywordONE{then}\isamarkupfalse%
\ \isakeywordTHREE{obtain}\isamarkupfalse%
\ p{\isadigit{1}}\ q{\isadigit{1}}\ p{\isadigit{2}}\ q{\isadigit{2}}\ \isakeywordTWO{where}\ p{\isadigit{1}}{\isacharunderscore}{\kern0pt}q{\isadigit{1}}{\isacharunderscore}{\kern0pt}intro{\isacharcolon}{\kern0pt}\ {\isachardoublequoteopen}reg{\isacharunderscore}{\kern0pt}eval\ p{\isadigit{1}}\ {\isasymand}\ reg{\isacharunderscore}{\kern0pt}eval\ q{\isadigit{1}}\ {\isasymand}\isanewline
\ \ \ \ f{\isadigit{1}}{\isacharprime}{\kern0pt}\ {\isacharequal}{\kern0pt}\ Union\ p{\isadigit{1}}\ {\isacharparenleft}{\kern0pt}Concat\ q{\isadigit{1}}\ {\isacharparenleft}{\kern0pt}Var\ x{\isacharparenright}{\kern0pt}{\isacharparenright}{\kern0pt}\ {\isasymand}\ {\isacharparenleft}{\kern0pt}{\isasymforall}y\ {\isasymin}\ vars\ p{\isadigit{1}}{\isachardot}{\kern0pt}\ y\ {\isasymnoteq}\ x{\isacharparenright}{\kern0pt}{\isachardoublequoteclose}\isanewline
\ \ \ \ \isakeywordTWO{and}\ p{\isadigit{2}}{\isacharunderscore}{\kern0pt}q{\isadigit{2}}{\isacharunderscore}{\kern0pt}intro{\isacharcolon}{\kern0pt}\ {\isachardoublequoteopen}reg{\isacharunderscore}{\kern0pt}eval\ p{\isadigit{2}}\ {\isasymand}\ reg{\isacharunderscore}{\kern0pt}eval\ q{\isadigit{2}}\ {\isasymand}\ f{\isadigit{2}}{\isacharprime}{\kern0pt}\ {\isacharequal}{\kern0pt}\ Union\ p{\isadigit{2}}\ {\isacharparenleft}{\kern0pt}Concat\ q{\isadigit{2}}\ {\isacharparenleft}{\kern0pt}Var\ x{\isacharparenright}{\kern0pt}{\isacharparenright}{\kern0pt}\ {\isasymand}\isanewline
\ \ \ \ {\isacharparenleft}{\kern0pt}{\isasymforall}y\ {\isasymin}\ vars\ p{\isadigit{2}}{\isachardot}{\kern0pt}\ y\ {\isasymnoteq}\ x{\isacharparenright}{\kern0pt}{\isachardoublequoteclose}\ \isakeywordONE{unfolding}\isamarkupfalse%
\ bipart{\isacharunderscore}{\kern0pt}rlexp{\isacharunderscore}{\kern0pt}def\ \isakeywordONE{by}\isamarkupfalse%
\ auto\isanewline
\ \ \isakeywordONE{let}\isamarkupfalse%
\ {\isacharquery}{\kern0pt}f{\isacharprime}{\kern0pt}\ {\isacharequal}{\kern0pt}\ {\isachardoublequoteopen}Union\ {\isacharparenleft}{\kern0pt}Union\ p{\isadigit{1}}\ p{\isadigit{2}}{\isacharparenright}{\kern0pt}\ {\isacharparenleft}{\kern0pt}Concat\ {\isacharparenleft}{\kern0pt}Union\ q{\isadigit{1}}\ q{\isadigit{2}}{\isacharparenright}{\kern0pt}\ {\isacharparenleft}{\kern0pt}Var\ x{\isacharparenright}{\kern0pt}{\isacharparenright}{\kern0pt}{\isachardoublequoteclose}\isanewline
\ \ \isakeywordONE{have}\isamarkupfalse%
\ {\isachardoublequoteopen}bipart{\isacharunderscore}{\kern0pt}rlexp\ x\ {\isacharquery}{\kern0pt}f{\isacharprime}{\kern0pt}{\isachardoublequoteclose}\ \isakeywordONE{unfolding}\isamarkupfalse%
\ bipart{\isacharunderscore}{\kern0pt}rlexp{\isacharunderscore}{\kern0pt}def\ \isakeywordONE{using}\isamarkupfalse%
\ p{\isadigit{1}}{\isacharunderscore}{\kern0pt}q{\isadigit{1}}{\isacharunderscore}{\kern0pt}intro\ p{\isadigit{2}}{\isacharunderscore}{\kern0pt}q{\isadigit{2}}{\isacharunderscore}{\kern0pt}intro\ \isakeywordONE{by}\isamarkupfalse%
\ auto\isanewline
\ \ \isakeywordONE{moreover}\isamarkupfalse%
\ \isakeywordONE{have}\isamarkupfalse%
\ {\isachardoublequoteopen}{\isasymPsi}\ {\isacharparenleft}{\kern0pt}eval\ {\isacharquery}{\kern0pt}f{\isacharprime}{\kern0pt}\ v{\isacharparenright}{\kern0pt}\ {\isacharequal}{\kern0pt}\ {\isasymPsi}\ {\isacharparenleft}{\kern0pt}eval\ {\isacharparenleft}{\kern0pt}Union\ f{\isadigit{1}}\ f{\isadigit{2}}{\isacharparenright}{\kern0pt}\ v{\isacharparenright}{\kern0pt}{\isachardoublequoteclose}\ \isakeywordTWO{for}\ v\isanewline
\ \ \ \ \isakeywordONE{using}\isamarkupfalse%
\ p{\isadigit{1}}{\isacharunderscore}{\kern0pt}q{\isadigit{1}}{\isacharunderscore}{\kern0pt}intro\ p{\isadigit{2}}{\isacharunderscore}{\kern0pt}q{\isadigit{2}}{\isacharunderscore}{\kern0pt}intro\ f{\isadigit{1}}{\isacharprime}{\kern0pt}{\isacharunderscore}{\kern0pt}intro\ f{\isadigit{2}}{\isacharprime}{\kern0pt}{\isacharunderscore}{\kern0pt}intro\isanewline
\ \ \ \ \isakeywordONE{by}\isamarkupfalse%
\ {\isacharparenleft}{\kern0pt}simp\ add{\isacharcolon}{\kern0pt}\ conc{\isacharunderscore}{\kern0pt}Un{\isacharunderscore}{\kern0pt}distrib{\isacharparenleft}{\kern0pt}{\isadigit{2}}{\isacharparenright}{\kern0pt}\ sup{\isacharunderscore}{\kern0pt}assoc\ sup{\isacharunderscore}{\kern0pt}left{\isacharunderscore}{\kern0pt}commute{\isacharparenright}{\kern0pt}\isanewline
\ \ \isakeywordONE{moreover}\isamarkupfalse%
\ \isakeywordONE{from}\isamarkupfalse%
\ f{\isadigit{1}}{\isacharprime}{\kern0pt}{\isacharunderscore}{\kern0pt}intro\ f{\isadigit{2}}{\isacharprime}{\kern0pt}{\isacharunderscore}{\kern0pt}intro\ p{\isadigit{1}}{\isacharunderscore}{\kern0pt}q{\isadigit{1}}{\isacharunderscore}{\kern0pt}intro\ p{\isadigit{2}}{\isacharunderscore}{\kern0pt}q{\isadigit{2}}{\isacharunderscore}{\kern0pt}intro\isanewline
\ \ \ \ \isakeywordONE{have}\isamarkupfalse%
\ {\isachardoublequoteopen}vars\ {\isacharquery}{\kern0pt}f{\isacharprime}{\kern0pt}\ {\isacharequal}{\kern0pt}\ vars\ {\isacharparenleft}{\kern0pt}Union\ f{\isadigit{1}}\ f{\isadigit{2}}{\isacharparenright}{\kern0pt}\ {\isasymunion}\ {\isacharbraceleft}{\kern0pt}x{\isacharbraceright}{\kern0pt}{\isachardoublequoteclose}\ \isakeywordONE{by}\isamarkupfalse%
\ auto\isanewline
\ \ \isakeywordONE{ultimately}\isamarkupfalse%
\ \isakeywordTHREE{show}\isamarkupfalse%
\ {\isacharquery}{\kern0pt}thesis\ \isakeywordONE{by}\isamarkupfalse%
\ metis\isanewline
\isakeywordONE{qed}\isamarkupfalse%
%
\endisatagproof
{\isafoldproof}%
%
\isadelimproof
\isanewline
%
\endisadelimproof
\isanewline
\isakeywordONE{lemma}\isamarkupfalse%
\ reg{\isacharunderscore}{\kern0pt}eval{\isacharunderscore}{\kern0pt}bipart{\isacharunderscore}{\kern0pt}rlexp{\isacharunderscore}{\kern0pt}Concat{\isacharcolon}{\kern0pt}\isanewline
\ \ \isakeywordTWO{assumes}\ {\isachardoublequoteopen}{\isasymexists}f{\isacharprime}{\kern0pt}{\isachardot}{\kern0pt}\ bipart{\isacharunderscore}{\kern0pt}rlexp\ x\ f{\isacharprime}{\kern0pt}\ {\isasymand}\ vars\ f{\isacharprime}{\kern0pt}\ {\isacharequal}{\kern0pt}\ vars\ f{\isadigit{1}}\ {\isasymunion}\ {\isacharbraceleft}{\kern0pt}x{\isacharbraceright}{\kern0pt}\ {\isasymand}\isanewline
\ \ \ \ \ \ \ \ \ \ \ \ \ \ \ \ {\isacharparenleft}{\kern0pt}{\isasymforall}v{\isachardot}{\kern0pt}\ {\isasymPsi}\ {\isacharparenleft}{\kern0pt}eval\ f{\isadigit{1}}\ v{\isacharparenright}{\kern0pt}\ {\isacharequal}{\kern0pt}\ {\isasymPsi}\ {\isacharparenleft}{\kern0pt}eval\ f{\isacharprime}{\kern0pt}\ v{\isacharparenright}{\kern0pt}{\isacharparenright}{\kern0pt}{\isachardoublequoteclose}\isanewline
\ \ \ \ \ \ \ \ \ \ {\isachardoublequoteopen}{\isasymexists}f{\isacharprime}{\kern0pt}{\isachardot}{\kern0pt}\ bipart{\isacharunderscore}{\kern0pt}rlexp\ x\ f{\isacharprime}{\kern0pt}\ {\isasymand}\ vars\ f{\isacharprime}{\kern0pt}\ {\isacharequal}{\kern0pt}\ vars\ f{\isadigit{2}}\ {\isasymunion}\ {\isacharbraceleft}{\kern0pt}x{\isacharbraceright}{\kern0pt}\ {\isasymand}\isanewline
\ \ \ \ \ \ \ \ \ \ \ \ \ \ \ \ {\isacharparenleft}{\kern0pt}{\isasymforall}v{\isachardot}{\kern0pt}\ {\isasymPsi}\ {\isacharparenleft}{\kern0pt}eval\ f{\isadigit{2}}\ v{\isacharparenright}{\kern0pt}\ {\isacharequal}{\kern0pt}\ {\isasymPsi}\ {\isacharparenleft}{\kern0pt}eval\ f{\isacharprime}{\kern0pt}\ v{\isacharparenright}{\kern0pt}{\isacharparenright}{\kern0pt}{\isachardoublequoteclose}\isanewline
\ \ \ \ \isakeywordTWO{shows}\ {\isachardoublequoteopen}{\isasymexists}f{\isacharprime}{\kern0pt}{\isachardot}{\kern0pt}\ bipart{\isacharunderscore}{\kern0pt}rlexp\ x\ f{\isacharprime}{\kern0pt}\ {\isasymand}\ vars\ f{\isacharprime}{\kern0pt}\ {\isacharequal}{\kern0pt}\ vars\ {\isacharparenleft}{\kern0pt}Concat\ f{\isadigit{1}}\ f{\isadigit{2}}{\isacharparenright}{\kern0pt}\ {\isasymunion}\ {\isacharbraceleft}{\kern0pt}x{\isacharbraceright}{\kern0pt}\ {\isasymand}\isanewline
\ \ \ \ \ \ \ \ \ \ \ \ \ \ \ \ {\isacharparenleft}{\kern0pt}{\isasymforall}v{\isachardot}{\kern0pt}\ {\isasymPsi}\ {\isacharparenleft}{\kern0pt}eval\ {\isacharparenleft}{\kern0pt}Concat\ f{\isadigit{1}}\ f{\isadigit{2}}{\isacharparenright}{\kern0pt}\ v{\isacharparenright}{\kern0pt}\ {\isacharequal}{\kern0pt}\ {\isasymPsi}\ {\isacharparenleft}{\kern0pt}eval\ f{\isacharprime}{\kern0pt}\ v{\isacharparenright}{\kern0pt}{\isacharparenright}{\kern0pt}{\isachardoublequoteclose}\isanewline
%
\isadelimproof
%
\endisadelimproof
%
\isatagproof
\isakeywordONE{proof}\isamarkupfalse%
\ {\isacharminus}{\kern0pt}\isanewline
\ \ \isakeywordONE{from}\isamarkupfalse%
\ assms\ \isakeywordTHREE{obtain}\isamarkupfalse%
\ f{\isadigit{1}}{\isacharprime}{\kern0pt}\ f{\isadigit{2}}{\isacharprime}{\kern0pt}\ \isakeywordTWO{where}\ f{\isadigit{1}}{\isacharprime}{\kern0pt}{\isacharunderscore}{\kern0pt}intro{\isacharcolon}{\kern0pt}\ {\isachardoublequoteopen}bipart{\isacharunderscore}{\kern0pt}rlexp\ x\ f{\isadigit{1}}{\isacharprime}{\kern0pt}\ {\isasymand}\ vars\ f{\isadigit{1}}{\isacharprime}{\kern0pt}\ {\isacharequal}{\kern0pt}\ vars\ f{\isadigit{1}}\ {\isasymunion}\ {\isacharbraceleft}{\kern0pt}x{\isacharbraceright}{\kern0pt}\ {\isasymand}\isanewline
\ \ \ \ \ \ {\isacharparenleft}{\kern0pt}{\isasymforall}v{\isachardot}{\kern0pt}\ {\isasymPsi}\ {\isacharparenleft}{\kern0pt}eval\ f{\isadigit{1}}\ v{\isacharparenright}{\kern0pt}\ {\isacharequal}{\kern0pt}\ {\isasymPsi}\ {\isacharparenleft}{\kern0pt}eval\ f{\isadigit{1}}{\isacharprime}{\kern0pt}\ v{\isacharparenright}{\kern0pt}{\isacharparenright}{\kern0pt}{\isachardoublequoteclose}\isanewline
\ \ \ \ \isakeywordTWO{and}\ f{\isadigit{2}}{\isacharprime}{\kern0pt}{\isacharunderscore}{\kern0pt}intro{\isacharcolon}{\kern0pt}\ {\isachardoublequoteopen}bipart{\isacharunderscore}{\kern0pt}rlexp\ x\ f{\isadigit{2}}{\isacharprime}{\kern0pt}\ {\isasymand}\ vars\ f{\isadigit{2}}{\isacharprime}{\kern0pt}\ {\isacharequal}{\kern0pt}\ vars\ f{\isadigit{2}}\ {\isasymunion}\ {\isacharbraceleft}{\kern0pt}x{\isacharbraceright}{\kern0pt}\ {\isasymand}\isanewline
\ \ \ \ \ \ {\isacharparenleft}{\kern0pt}{\isasymforall}v{\isachardot}{\kern0pt}\ {\isasymPsi}\ {\isacharparenleft}{\kern0pt}eval\ f{\isadigit{2}}\ v{\isacharparenright}{\kern0pt}\ {\isacharequal}{\kern0pt}\ {\isasymPsi}\ {\isacharparenleft}{\kern0pt}eval\ f{\isadigit{2}}{\isacharprime}{\kern0pt}\ v{\isacharparenright}{\kern0pt}{\isacharparenright}{\kern0pt}{\isachardoublequoteclose}\isanewline
\ \ \ \ \isakeywordONE{by}\isamarkupfalse%
\ auto\isanewline
\ \ \isakeywordONE{then}\isamarkupfalse%
\ \isakeywordTHREE{obtain}\isamarkupfalse%
\ p{\isadigit{1}}\ q{\isadigit{1}}\ p{\isadigit{2}}\ q{\isadigit{2}}\ \isakeywordTWO{where}\ p{\isadigit{1}}{\isacharunderscore}{\kern0pt}q{\isadigit{1}}{\isacharunderscore}{\kern0pt}intro{\isacharcolon}{\kern0pt}\ {\isachardoublequoteopen}reg{\isacharunderscore}{\kern0pt}eval\ p{\isadigit{1}}\ {\isasymand}\ reg{\isacharunderscore}{\kern0pt}eval\ q{\isadigit{1}}\ {\isasymand}\isanewline
\ \ \ \ f{\isadigit{1}}{\isacharprime}{\kern0pt}\ {\isacharequal}{\kern0pt}\ Union\ p{\isadigit{1}}\ {\isacharparenleft}{\kern0pt}Concat\ q{\isadigit{1}}\ {\isacharparenleft}{\kern0pt}Var\ x{\isacharparenright}{\kern0pt}{\isacharparenright}{\kern0pt}\ {\isasymand}\ {\isacharparenleft}{\kern0pt}{\isasymforall}y\ {\isasymin}\ vars\ p{\isadigit{1}}{\isachardot}{\kern0pt}\ y\ {\isasymnoteq}\ x{\isacharparenright}{\kern0pt}{\isachardoublequoteclose}\isanewline
\ \ \ \ \isakeywordTWO{and}\ p{\isadigit{2}}{\isacharunderscore}{\kern0pt}q{\isadigit{2}}{\isacharunderscore}{\kern0pt}intro{\isacharcolon}{\kern0pt}\ {\isachardoublequoteopen}reg{\isacharunderscore}{\kern0pt}eval\ p{\isadigit{2}}\ {\isasymand}\ reg{\isacharunderscore}{\kern0pt}eval\ q{\isadigit{2}}\ {\isasymand}\ f{\isadigit{2}}{\isacharprime}{\kern0pt}\ {\isacharequal}{\kern0pt}\ Union\ p{\isadigit{2}}\ {\isacharparenleft}{\kern0pt}Concat\ q{\isadigit{2}}\ {\isacharparenleft}{\kern0pt}Var\ x{\isacharparenright}{\kern0pt}{\isacharparenright}{\kern0pt}\ {\isasymand}\isanewline
\ \ \ \ {\isacharparenleft}{\kern0pt}{\isasymforall}y\ {\isasymin}\ vars\ p{\isadigit{2}}{\isachardot}{\kern0pt}\ y\ {\isasymnoteq}\ x{\isacharparenright}{\kern0pt}{\isachardoublequoteclose}\ \isakeywordONE{unfolding}\isamarkupfalse%
\ bipart{\isacharunderscore}{\kern0pt}rlexp{\isacharunderscore}{\kern0pt}def\ \isakeywordONE{by}\isamarkupfalse%
\ auto\isanewline
\ \ \isakeywordONE{let}\isamarkupfalse%
\ {\isacharquery}{\kern0pt}q{\isacharprime}{\kern0pt}\ {\isacharequal}{\kern0pt}\ {\isachardoublequoteopen}Union\ {\isacharparenleft}{\kern0pt}Concat\ q{\isadigit{1}}\ {\isacharparenleft}{\kern0pt}Concat\ {\isacharparenleft}{\kern0pt}Var\ x{\isacharparenright}{\kern0pt}\ q{\isadigit{2}}{\isacharparenright}{\kern0pt}{\isacharparenright}{\kern0pt}\ {\isacharparenleft}{\kern0pt}Union\ {\isacharparenleft}{\kern0pt}Concat\ p{\isadigit{1}}\ q{\isadigit{2}}{\isacharparenright}{\kern0pt}\ {\isacharparenleft}{\kern0pt}Concat\ q{\isadigit{1}}\ p{\isadigit{2}}{\isacharparenright}{\kern0pt}{\isacharparenright}{\kern0pt}{\isachardoublequoteclose}\isanewline
\ \ \isakeywordONE{let}\isamarkupfalse%
\ {\isacharquery}{\kern0pt}f{\isacharprime}{\kern0pt}\ {\isacharequal}{\kern0pt}\ {\isachardoublequoteopen}Union\ {\isacharparenleft}{\kern0pt}Concat\ p{\isadigit{1}}\ p{\isadigit{2}}{\isacharparenright}{\kern0pt}\ {\isacharparenleft}{\kern0pt}Concat\ {\isacharquery}{\kern0pt}q{\isacharprime}{\kern0pt}\ {\isacharparenleft}{\kern0pt}Var\ x{\isacharparenright}{\kern0pt}{\isacharparenright}{\kern0pt}{\isachardoublequoteclose}\isanewline
\ \ \isakeywordONE{have}\isamarkupfalse%
\ {\isachardoublequoteopen}{\isasymforall}v{\isachardot}{\kern0pt}\ {\isacharparenleft}{\kern0pt}{\isasymPsi}\ {\isacharparenleft}{\kern0pt}eval\ {\isacharparenleft}{\kern0pt}Concat\ f{\isadigit{1}}\ f{\isadigit{2}}{\isacharparenright}{\kern0pt}\ v{\isacharparenright}{\kern0pt}\ {\isacharequal}{\kern0pt}\ {\isasymPsi}\ {\isacharparenleft}{\kern0pt}eval\ {\isacharquery}{\kern0pt}f{\isacharprime}{\kern0pt}\ v{\isacharparenright}{\kern0pt}{\isacharparenright}{\kern0pt}{\isachardoublequoteclose}\isanewline
\ \ \isakeywordONE{proof}\isamarkupfalse%
\ {\isacharparenleft}{\kern0pt}rule\ allI{\isacharparenright}{\kern0pt}\isanewline
\ \ \ \ \isakeywordTHREE{fix}\isamarkupfalse%
\ v\isanewline
\ \ \ \ \isakeywordONE{have}\isamarkupfalse%
\ f{\isadigit{2}}{\isacharunderscore}{\kern0pt}subst{\isacharcolon}{\kern0pt}\ {\isachardoublequoteopen}{\isasymPsi}\ {\isacharparenleft}{\kern0pt}eval\ f{\isadigit{2}}\ v{\isacharparenright}{\kern0pt}\ {\isacharequal}{\kern0pt}\ {\isasymPsi}\ {\isacharparenleft}{\kern0pt}eval\ p{\isadigit{2}}\ v\ {\isasymunion}\ eval\ q{\isadigit{2}}\ v\ {\isacharat}{\kern0pt}{\isacharat}{\kern0pt}\ v\ x{\isacharparenright}{\kern0pt}{\isachardoublequoteclose}\isanewline
\ \ \ \ \ \ \isakeywordONE{using}\isamarkupfalse%
\ p{\isadigit{2}}{\isacharunderscore}{\kern0pt}q{\isadigit{2}}{\isacharunderscore}{\kern0pt}intro\ f{\isadigit{2}}{\isacharprime}{\kern0pt}{\isacharunderscore}{\kern0pt}intro\ \isakeywordONE{by}\isamarkupfalse%
\ auto\isanewline
\ \ \ \ \isakeywordONE{have}\isamarkupfalse%
\ {\isachardoublequoteopen}{\isasymPsi}\ {\isacharparenleft}{\kern0pt}eval\ {\isacharparenleft}{\kern0pt}Concat\ f{\isadigit{1}}\ f{\isadigit{2}}{\isacharparenright}{\kern0pt}\ v{\isacharparenright}{\kern0pt}\ {\isacharequal}{\kern0pt}\ {\isasymPsi}\ {\isacharparenleft}{\kern0pt}{\isacharparenleft}{\kern0pt}eval\ p{\isadigit{1}}\ v\ {\isasymunion}\ eval\ q{\isadigit{1}}\ v\ {\isacharat}{\kern0pt}{\isacharat}{\kern0pt}\ v\ x{\isacharparenright}{\kern0pt}\ {\isacharat}{\kern0pt}{\isacharat}{\kern0pt}\ eval\ f{\isadigit{2}}\ v{\isacharparenright}{\kern0pt}{\isachardoublequoteclose}\isanewline
\ \ \ \ \ \ \isakeywordONE{using}\isamarkupfalse%
\ p{\isadigit{1}}{\isacharunderscore}{\kern0pt}q{\isadigit{1}}{\isacharunderscore}{\kern0pt}intro\ f{\isadigit{1}}{\isacharprime}{\kern0pt}{\isacharunderscore}{\kern0pt}intro\isanewline
\ \ \ \ \ \ \isakeywordONE{by}\isamarkupfalse%
\ {\isacharparenleft}{\kern0pt}metis\ eval{\isachardot}{\kern0pt}simps{\isacharparenleft}{\kern0pt}{\isadigit{1}}{\isacharparenright}{\kern0pt}\ eval{\isachardot}{\kern0pt}simps{\isacharparenleft}{\kern0pt}{\isadigit{3}}{\isacharparenright}{\kern0pt}\ eval{\isachardot}{\kern0pt}simps{\isacharparenleft}{\kern0pt}{\isadigit{4}}{\isacharparenright}{\kern0pt}\ parikh{\isacharunderscore}{\kern0pt}conc{\isacharunderscore}{\kern0pt}right{\isacharparenright}{\kern0pt}\isanewline
\ \ \ \ \isakeywordONE{also}\isamarkupfalse%
\ \isakeywordONE{have}\isamarkupfalse%
\ {\isachardoublequoteopen}{\isasymdots}\ {\isacharequal}{\kern0pt}\ {\isasymPsi}\ {\isacharparenleft}{\kern0pt}eval\ p{\isadigit{1}}\ v\ {\isacharat}{\kern0pt}{\isacharat}{\kern0pt}\ eval\ f{\isadigit{2}}\ v\ {\isasymunion}\ eval\ q{\isadigit{1}}\ v\ {\isacharat}{\kern0pt}{\isacharat}{\kern0pt}\ v\ x\ {\isacharat}{\kern0pt}{\isacharat}{\kern0pt}\ eval\ f{\isadigit{2}}\ v{\isacharparenright}{\kern0pt}{\isachardoublequoteclose}\isanewline
\ \ \ \ \ \ \isakeywordONE{by}\isamarkupfalse%
\ {\isacharparenleft}{\kern0pt}simp\ add{\isacharcolon}{\kern0pt}\ conc{\isacharunderscore}{\kern0pt}Un{\isacharunderscore}{\kern0pt}distrib{\isacharparenleft}{\kern0pt}{\isadigit{2}}{\isacharparenright}{\kern0pt}\ conc{\isacharunderscore}{\kern0pt}assoc{\isacharparenright}{\kern0pt}\isanewline
\ \ \ \ \isakeywordONE{also}\isamarkupfalse%
\ \isakeywordONE{have}\isamarkupfalse%
\ {\isachardoublequoteopen}{\isasymdots}\ {\isacharequal}{\kern0pt}\ {\isasymPsi}\ {\isacharparenleft}{\kern0pt}eval\ p{\isadigit{1}}\ v\ {\isacharat}{\kern0pt}{\isacharat}{\kern0pt}\ {\isacharparenleft}{\kern0pt}eval\ p{\isadigit{2}}\ v\ {\isasymunion}\ eval\ q{\isadigit{2}}\ v\ {\isacharat}{\kern0pt}{\isacharat}{\kern0pt}\ v\ x{\isacharparenright}{\kern0pt}\isanewline
\ \ \ \ \ \ \ \ {\isasymunion}\ eval\ q{\isadigit{1}}\ v\ {\isacharat}{\kern0pt}{\isacharat}{\kern0pt}\ v\ x\ {\isacharat}{\kern0pt}{\isacharat}{\kern0pt}\ {\isacharparenleft}{\kern0pt}eval\ p{\isadigit{2}}\ v\ {\isasymunion}\ eval\ q{\isadigit{2}}\ v\ {\isacharat}{\kern0pt}{\isacharat}{\kern0pt}\ v\ x{\isacharparenright}{\kern0pt}{\isacharparenright}{\kern0pt}{\isachardoublequoteclose}\isanewline
\ \ \ \ \ \ \isakeywordONE{using}\isamarkupfalse%
\ f{\isadigit{2}}{\isacharunderscore}{\kern0pt}subst\ \isakeywordONE{by}\isamarkupfalse%
\ {\isacharparenleft}{\kern0pt}smt\ {\isacharparenleft}{\kern0pt}verit{\isacharcomma}{\kern0pt}\ ccfv{\isacharunderscore}{\kern0pt}threshold{\isacharparenright}{\kern0pt}\ parikh{\isacharunderscore}{\kern0pt}conc{\isacharunderscore}{\kern0pt}right\ parikh{\isacharunderscore}{\kern0pt}img{\isacharunderscore}{\kern0pt}Un\ parikh{\isacharunderscore}{\kern0pt}img{\isacharunderscore}{\kern0pt}commut{\isacharparenright}{\kern0pt}\isanewline
\ \ \ \ \isakeywordONE{also}\isamarkupfalse%
\ \isakeywordONE{have}\isamarkupfalse%
\ {\isachardoublequoteopen}{\isasymdots}\ {\isacharequal}{\kern0pt}\ {\isasymPsi}\ {\isacharparenleft}{\kern0pt}eval\ p{\isadigit{1}}\ v\ {\isacharat}{\kern0pt}{\isacharat}{\kern0pt}\ eval\ p{\isadigit{2}}\ v\ {\isasymunion}\ {\isacharparenleft}{\kern0pt}eval\ p{\isadigit{1}}\ v\ {\isacharat}{\kern0pt}{\isacharat}{\kern0pt}\ eval\ q{\isadigit{2}}\ v\ {\isacharat}{\kern0pt}{\isacharat}{\kern0pt}\ v\ x\ {\isasymunion}\isanewline
\ \ \ \ \ \ \ \ eval\ q{\isadigit{1}}\ v\ {\isacharat}{\kern0pt}{\isacharat}{\kern0pt}\ eval\ p{\isadigit{2}}\ v\ {\isacharat}{\kern0pt}{\isacharat}{\kern0pt}\ v\ x\ {\isasymunion}\ eval\ q{\isadigit{1}}\ v\ {\isacharat}{\kern0pt}{\isacharat}{\kern0pt}\ v\ x\ {\isacharat}{\kern0pt}{\isacharat}{\kern0pt}\ eval\ q{\isadigit{2}}\ v\ {\isacharat}{\kern0pt}{\isacharat}{\kern0pt}\ v\ x{\isacharparenright}{\kern0pt}{\isacharparenright}{\kern0pt}{\isachardoublequoteclose}\isanewline
\ \ \ \ \ \ \isakeywordONE{using}\isamarkupfalse%
\ parikh{\isacharunderscore}{\kern0pt}img{\isacharunderscore}{\kern0pt}commut\ \isakeywordONE{by}\isamarkupfalse%
\ {\isacharparenleft}{\kern0pt}smt\ {\isacharparenleft}{\kern0pt}z{\isadigit{3}}{\isacharparenright}{\kern0pt}\ conc{\isacharunderscore}{\kern0pt}Un{\isacharunderscore}{\kern0pt}distrib{\isacharparenleft}{\kern0pt}{\isadigit{1}}{\isacharparenright}{\kern0pt}\ parikh{\isacharunderscore}{\kern0pt}conc{\isacharunderscore}{\kern0pt}right\ parikh{\isacharunderscore}{\kern0pt}img{\isacharunderscore}{\kern0pt}Un\ sup{\isacharunderscore}{\kern0pt}assoc{\isacharparenright}{\kern0pt}\isanewline
\ \ \ \ \isakeywordONE{also}\isamarkupfalse%
\ \isakeywordONE{have}\isamarkupfalse%
\ {\isachardoublequoteopen}{\isasymdots}\ {\isacharequal}{\kern0pt}\ {\isasymPsi}\ {\isacharparenleft}{\kern0pt}eval\ p{\isadigit{1}}\ v\ {\isacharat}{\kern0pt}{\isacharat}{\kern0pt}\ eval\ p{\isadigit{2}}\ v\ {\isasymunion}\ {\isacharparenleft}{\kern0pt}eval\ p{\isadigit{1}}\ v\ {\isacharat}{\kern0pt}{\isacharat}{\kern0pt}\ eval\ q{\isadigit{2}}\ v\ {\isasymunion}\isanewline
\ \ \ \ \ \ \ \ eval\ q{\isadigit{1}}\ v\ {\isacharat}{\kern0pt}{\isacharat}{\kern0pt}\ eval\ p{\isadigit{2}}\ v\ {\isasymunion}\ eval\ q{\isadigit{1}}\ v\ {\isacharat}{\kern0pt}{\isacharat}{\kern0pt}\ v\ x\ {\isacharat}{\kern0pt}{\isacharat}{\kern0pt}\ eval\ q{\isadigit{2}}\ v{\isacharparenright}{\kern0pt}\ {\isacharat}{\kern0pt}{\isacharat}{\kern0pt}\ v\ x{\isacharparenright}{\kern0pt}{\isachardoublequoteclose}\isanewline
\ \ \ \ \ \ \isakeywordONE{by}\isamarkupfalse%
\ {\isacharparenleft}{\kern0pt}simp\ add{\isacharcolon}{\kern0pt}\ conc{\isacharunderscore}{\kern0pt}Un{\isacharunderscore}{\kern0pt}distrib{\isacharparenleft}{\kern0pt}{\isadigit{2}}{\isacharparenright}{\kern0pt}\ conc{\isacharunderscore}{\kern0pt}assoc{\isacharparenright}{\kern0pt}\isanewline
\ \ \ \ \isakeywordONE{also}\isamarkupfalse%
\ \isakeywordONE{have}\isamarkupfalse%
\ {\isachardoublequoteopen}{\isasymdots}\ {\isacharequal}{\kern0pt}\ {\isasymPsi}\ {\isacharparenleft}{\kern0pt}eval\ {\isacharquery}{\kern0pt}f{\isacharprime}{\kern0pt}\ v{\isacharparenright}{\kern0pt}{\isachardoublequoteclose}\isanewline
\ \ \ \ \ \ \isakeywordONE{by}\isamarkupfalse%
\ {\isacharparenleft}{\kern0pt}simp\ add{\isacharcolon}{\kern0pt}\ Un{\isacharunderscore}{\kern0pt}commute{\isacharparenright}{\kern0pt}\isanewline
\ \ \ \ \isakeywordONE{finally}\isamarkupfalse%
\ \isakeywordTHREE{show}\isamarkupfalse%
\ {\isachardoublequoteopen}{\isasymPsi}\ {\isacharparenleft}{\kern0pt}eval\ {\isacharparenleft}{\kern0pt}Concat\ f{\isadigit{1}}\ f{\isadigit{2}}{\isacharparenright}{\kern0pt}\ v{\isacharparenright}{\kern0pt}\ {\isacharequal}{\kern0pt}\ {\isasymPsi}\ {\isacharparenleft}{\kern0pt}eval\ {\isacharquery}{\kern0pt}f{\isacharprime}{\kern0pt}\ v{\isacharparenright}{\kern0pt}{\isachardoublequoteclose}\ \isakeywordONE{{\isachardot}{\kern0pt}}\isamarkupfalse%
\isanewline
\ \ \isakeywordONE{qed}\isamarkupfalse%
\isanewline
\ \ \isakeywordONE{moreover}\isamarkupfalse%
\ \isakeywordONE{have}\isamarkupfalse%
\ {\isachardoublequoteopen}bipart{\isacharunderscore}{\kern0pt}rlexp\ x\ {\isacharquery}{\kern0pt}f{\isacharprime}{\kern0pt}{\isachardoublequoteclose}\ \isakeywordONE{unfolding}\isamarkupfalse%
\ bipart{\isacharunderscore}{\kern0pt}rlexp{\isacharunderscore}{\kern0pt}def\ \isakeywordONE{using}\isamarkupfalse%
\ p{\isadigit{1}}{\isacharunderscore}{\kern0pt}q{\isadigit{1}}{\isacharunderscore}{\kern0pt}intro\ p{\isadigit{2}}{\isacharunderscore}{\kern0pt}q{\isadigit{2}}{\isacharunderscore}{\kern0pt}intro\ \isakeywordONE{by}\isamarkupfalse%
\ auto\isanewline
\ \ \isakeywordONE{moreover}\isamarkupfalse%
\ \isakeywordONE{from}\isamarkupfalse%
\ f{\isadigit{1}}{\isacharprime}{\kern0pt}{\isacharunderscore}{\kern0pt}intro\ f{\isadigit{2}}{\isacharprime}{\kern0pt}{\isacharunderscore}{\kern0pt}intro\ p{\isadigit{1}}{\isacharunderscore}{\kern0pt}q{\isadigit{1}}{\isacharunderscore}{\kern0pt}intro\ p{\isadigit{2}}{\isacharunderscore}{\kern0pt}q{\isadigit{2}}{\isacharunderscore}{\kern0pt}intro\isanewline
\ \ \ \ \isakeywordONE{have}\isamarkupfalse%
\ {\isachardoublequoteopen}vars\ {\isacharquery}{\kern0pt}f{\isacharprime}{\kern0pt}\ {\isacharequal}{\kern0pt}\ vars\ {\isacharparenleft}{\kern0pt}Concat\ f{\isadigit{1}}\ f{\isadigit{2}}{\isacharparenright}{\kern0pt}\ {\isasymunion}\ {\isacharbraceleft}{\kern0pt}x{\isacharbraceright}{\kern0pt}{\isachardoublequoteclose}\ \isakeywordONE{by}\isamarkupfalse%
\ auto\isanewline
\ \ \isakeywordONE{ultimately}\isamarkupfalse%
\ \isakeywordTHREE{show}\isamarkupfalse%
\ {\isacharquery}{\kern0pt}thesis\ \isakeywordONE{by}\isamarkupfalse%
\ metis\isanewline
\isakeywordONE{qed}\isamarkupfalse%
%
\endisatagproof
{\isafoldproof}%
%
\isadelimproof
\isanewline
%
\endisadelimproof
\isanewline
\isakeywordONE{lemma}\isamarkupfalse%
\ reg{\isacharunderscore}{\kern0pt}eval{\isacharunderscore}{\kern0pt}bipart{\isacharunderscore}{\kern0pt}rlexp{\isacharunderscore}{\kern0pt}Star{\isacharcolon}{\kern0pt}\isanewline
\ \ \isakeywordTWO{assumes}\ {\isachardoublequoteopen}{\isasymexists}f{\isacharprime}{\kern0pt}{\isachardot}{\kern0pt}\ bipart{\isacharunderscore}{\kern0pt}rlexp\ x\ f{\isacharprime}{\kern0pt}\ {\isasymand}\ vars\ f{\isacharprime}{\kern0pt}\ {\isacharequal}{\kern0pt}\ vars\ f\ {\isasymunion}\ {\isacharbraceleft}{\kern0pt}x{\isacharbraceright}{\kern0pt}\isanewline
\ \ \ \ \ \ \ \ \ \ \ \ \ \ \ \ {\isasymand}\ {\isacharparenleft}{\kern0pt}{\isasymforall}v{\isachardot}{\kern0pt}\ {\isasymPsi}\ {\isacharparenleft}{\kern0pt}eval\ f\ v{\isacharparenright}{\kern0pt}\ {\isacharequal}{\kern0pt}\ {\isasymPsi}\ {\isacharparenleft}{\kern0pt}eval\ f{\isacharprime}{\kern0pt}\ v{\isacharparenright}{\kern0pt}{\isacharparenright}{\kern0pt}{\isachardoublequoteclose}\isanewline
\ \ \isakeywordTWO{shows}\ {\isachardoublequoteopen}{\isasymexists}f{\isacharprime}{\kern0pt}{\isachardot}{\kern0pt}\ bipart{\isacharunderscore}{\kern0pt}rlexp\ x\ f{\isacharprime}{\kern0pt}\ {\isasymand}\ vars\ f{\isacharprime}{\kern0pt}\ {\isacharequal}{\kern0pt}\ vars\ {\isacharparenleft}{\kern0pt}Star\ f{\isacharparenright}{\kern0pt}\ {\isasymunion}\ {\isacharbraceleft}{\kern0pt}x{\isacharbraceright}{\kern0pt}\isanewline
\ \ \ \ \ \ \ \ \ \ \ \ \ \ \ \ {\isasymand}\ {\isacharparenleft}{\kern0pt}{\isasymforall}v{\isachardot}{\kern0pt}\ {\isasymPsi}\ {\isacharparenleft}{\kern0pt}eval\ {\isacharparenleft}{\kern0pt}Star\ f{\isacharparenright}{\kern0pt}\ v{\isacharparenright}{\kern0pt}\ {\isacharequal}{\kern0pt}\ {\isasymPsi}\ {\isacharparenleft}{\kern0pt}eval\ f{\isacharprime}{\kern0pt}\ v{\isacharparenright}{\kern0pt}{\isacharparenright}{\kern0pt}{\isachardoublequoteclose}\isanewline
%
\isadelimproof
%
\endisadelimproof
%
\isatagproof
\isakeywordONE{proof}\isamarkupfalse%
\ {\isacharminus}{\kern0pt}\isanewline
\ \ \isakeywordONE{from}\isamarkupfalse%
\ assms\ \isakeywordTHREE{obtain}\isamarkupfalse%
\ f{\isacharprime}{\kern0pt}\ \isakeywordTWO{where}\ f{\isacharprime}{\kern0pt}{\isacharunderscore}{\kern0pt}intro{\isacharcolon}{\kern0pt}\ {\isachardoublequoteopen}bipart{\isacharunderscore}{\kern0pt}rlexp\ x\ f{\isacharprime}{\kern0pt}\ {\isasymand}\ vars\ f{\isacharprime}{\kern0pt}\ {\isacharequal}{\kern0pt}\ vars\ f\ {\isasymunion}\ {\isacharbraceleft}{\kern0pt}x{\isacharbraceright}{\kern0pt}\ {\isasymand}\isanewline
\ \ \ \ \ \ {\isacharparenleft}{\kern0pt}{\isasymforall}v{\isachardot}{\kern0pt}\ {\isasymPsi}\ {\isacharparenleft}{\kern0pt}eval\ f\ v{\isacharparenright}{\kern0pt}\ {\isacharequal}{\kern0pt}\ {\isasymPsi}\ {\isacharparenleft}{\kern0pt}eval\ f{\isacharprime}{\kern0pt}\ v{\isacharparenright}{\kern0pt}{\isacharparenright}{\kern0pt}{\isachardoublequoteclose}\ \isakeywordONE{by}\isamarkupfalse%
\ auto\isanewline
\ \ \isakeywordONE{then}\isamarkupfalse%
\ \isakeywordTHREE{obtain}\isamarkupfalse%
\ p\ q\ \isakeywordTWO{where}\ p{\isacharunderscore}{\kern0pt}q{\isacharunderscore}{\kern0pt}intro{\isacharcolon}{\kern0pt}\ {\isachardoublequoteopen}reg{\isacharunderscore}{\kern0pt}eval\ p\ {\isasymand}\ reg{\isacharunderscore}{\kern0pt}eval\ q\ {\isasymand}\isanewline
\ \ \ \ f{\isacharprime}{\kern0pt}\ {\isacharequal}{\kern0pt}\ Union\ p\ {\isacharparenleft}{\kern0pt}Concat\ q\ {\isacharparenleft}{\kern0pt}Var\ x{\isacharparenright}{\kern0pt}{\isacharparenright}{\kern0pt}\ {\isasymand}\ {\isacharparenleft}{\kern0pt}{\isasymforall}y\ {\isasymin}\ vars\ p{\isachardot}{\kern0pt}\ y\ {\isasymnoteq}\ x{\isacharparenright}{\kern0pt}{\isachardoublequoteclose}\ \isakeywordONE{unfolding}\isamarkupfalse%
\ bipart{\isacharunderscore}{\kern0pt}rlexp{\isacharunderscore}{\kern0pt}def\ \isakeywordONE{by}\isamarkupfalse%
\ auto\isanewline
\ \ \isakeywordONE{let}\isamarkupfalse%
\ {\isacharquery}{\kern0pt}q{\isacharunderscore}{\kern0pt}new\ {\isacharequal}{\kern0pt}\ {\isachardoublequoteopen}Concat\ {\isacharparenleft}{\kern0pt}Star\ p{\isacharparenright}{\kern0pt}\ {\isacharparenleft}{\kern0pt}Concat\ {\isacharparenleft}{\kern0pt}Star\ {\isacharparenleft}{\kern0pt}Concat\ q\ {\isacharparenleft}{\kern0pt}Var\ x{\isacharparenright}{\kern0pt}{\isacharparenright}{\kern0pt}{\isacharparenright}{\kern0pt}\ {\isacharparenleft}{\kern0pt}Concat\ {\isacharparenleft}{\kern0pt}Star\ {\isacharparenleft}{\kern0pt}Concat\ q\ {\isacharparenleft}{\kern0pt}Var\ x{\isacharparenright}{\kern0pt}{\isacharparenright}{\kern0pt}{\isacharparenright}{\kern0pt}\ q{\isacharparenright}{\kern0pt}{\isacharparenright}{\kern0pt}{\isachardoublequoteclose}\isanewline
\ \ \isakeywordONE{let}\isamarkupfalse%
\ {\isacharquery}{\kern0pt}f{\isacharunderscore}{\kern0pt}new\ {\isacharequal}{\kern0pt}\ {\isachardoublequoteopen}Union\ {\isacharparenleft}{\kern0pt}Star\ p{\isacharparenright}{\kern0pt}\ {\isacharparenleft}{\kern0pt}Concat\ {\isacharquery}{\kern0pt}q{\isacharunderscore}{\kern0pt}new\ {\isacharparenleft}{\kern0pt}Var\ x{\isacharparenright}{\kern0pt}{\isacharparenright}{\kern0pt}{\isachardoublequoteclose}\isanewline
\ \ \isakeywordONE{have}\isamarkupfalse%
\ {\isachardoublequoteopen}{\isasymforall}v{\isachardot}{\kern0pt}\ {\isacharparenleft}{\kern0pt}{\isasymPsi}\ {\isacharparenleft}{\kern0pt}eval\ {\isacharparenleft}{\kern0pt}Star\ f{\isacharparenright}{\kern0pt}\ v{\isacharparenright}{\kern0pt}\ {\isacharequal}{\kern0pt}\ {\isasymPsi}\ {\isacharparenleft}{\kern0pt}eval\ {\isacharquery}{\kern0pt}f{\isacharunderscore}{\kern0pt}new\ v{\isacharparenright}{\kern0pt}{\isacharparenright}{\kern0pt}{\isachardoublequoteclose}\isanewline
\ \ \isakeywordONE{proof}\isamarkupfalse%
\ {\isacharparenleft}{\kern0pt}rule\ allI{\isacharparenright}{\kern0pt}\isanewline
\ \ \ \ \isakeywordTHREE{fix}\isamarkupfalse%
\ v\isanewline
\ \ \ \ \isakeywordONE{have}\isamarkupfalse%
\ {\isachardoublequoteopen}{\isasymPsi}\ {\isacharparenleft}{\kern0pt}eval\ {\isacharparenleft}{\kern0pt}Star\ f{\isacharparenright}{\kern0pt}\ v{\isacharparenright}{\kern0pt}\ {\isacharequal}{\kern0pt}\ {\isasymPsi}\ {\isacharparenleft}{\kern0pt}star\ {\isacharparenleft}{\kern0pt}eval\ p\ v\ {\isasymunion}\ eval\ q\ v\ {\isacharat}{\kern0pt}{\isacharat}{\kern0pt}\ v\ x{\isacharparenright}{\kern0pt}{\isacharparenright}{\kern0pt}{\isachardoublequoteclose}\isanewline
\ \ \ \ \ \ \isakeywordONE{using}\isamarkupfalse%
\ f{\isacharprime}{\kern0pt}{\isacharunderscore}{\kern0pt}intro\ parikh{\isacharunderscore}{\kern0pt}star{\isacharunderscore}{\kern0pt}mono{\isacharunderscore}{\kern0pt}eq\ p{\isacharunderscore}{\kern0pt}q{\isacharunderscore}{\kern0pt}intro\isanewline
\ \ \ \ \ \ \isakeywordONE{by}\isamarkupfalse%
\ {\isacharparenleft}{\kern0pt}metis\ eval{\isachardot}{\kern0pt}simps{\isacharparenleft}{\kern0pt}{\isadigit{1}}{\isacharparenright}{\kern0pt}\ eval{\isachardot}{\kern0pt}simps{\isacharparenleft}{\kern0pt}{\isadigit{3}}{\isacharparenright}{\kern0pt}\ eval{\isachardot}{\kern0pt}simps{\isacharparenleft}{\kern0pt}{\isadigit{4}}{\isacharparenright}{\kern0pt}\ eval{\isachardot}{\kern0pt}simps{\isacharparenleft}{\kern0pt}{\isadigit{5}}{\isacharparenright}{\kern0pt}{\isacharparenright}{\kern0pt}\isanewline
\ \ \ \ \isakeywordONE{also}\isamarkupfalse%
\ \isakeywordONE{have}\isamarkupfalse%
\ {\isachardoublequoteopen}{\isasymdots}\ {\isacharequal}{\kern0pt}\ {\isasymPsi}\ {\isacharparenleft}{\kern0pt}star\ {\isacharparenleft}{\kern0pt}eval\ p\ v{\isacharparenright}{\kern0pt}\ {\isacharat}{\kern0pt}{\isacharat}{\kern0pt}\ star\ {\isacharparenleft}{\kern0pt}eval\ q\ v\ {\isacharat}{\kern0pt}{\isacharat}{\kern0pt}\ v\ x{\isacharparenright}{\kern0pt}{\isacharparenright}{\kern0pt}{\isachardoublequoteclose}\isanewline
\ \ \ \ \ \ \isakeywordONE{using}\isamarkupfalse%
\ parikh{\isacharunderscore}{\kern0pt}img{\isacharunderscore}{\kern0pt}star\ \isakeywordONE{by}\isamarkupfalse%
\ blast\isanewline
\ \ \ \ \isakeywordONE{also}\isamarkupfalse%
\ \isakeywordONE{have}\isamarkupfalse%
\ {\isachardoublequoteopen}{\isasymdots}\ {\isacharequal}{\kern0pt}\ {\isasymPsi}\ {\isacharparenleft}{\kern0pt}star\ {\isacharparenleft}{\kern0pt}eval\ p\ v{\isacharparenright}{\kern0pt}\ {\isacharat}{\kern0pt}{\isacharat}{\kern0pt}\isanewline
\ \ \ \ \ \ \ \ star\ {\isacharparenleft}{\kern0pt}{\isacharbraceleft}{\kern0pt}{\isacharbrackleft}{\kern0pt}{\isacharbrackright}{\kern0pt}{\isacharbraceright}{\kern0pt}\ {\isasymunion}\ star\ {\isacharparenleft}{\kern0pt}eval\ q\ v\ {\isacharat}{\kern0pt}{\isacharat}{\kern0pt}\ v\ x{\isacharparenright}{\kern0pt}\ {\isacharat}{\kern0pt}{\isacharat}{\kern0pt}\ eval\ q\ v\ {\isacharat}{\kern0pt}{\isacharat}{\kern0pt}\ v\ x{\isacharparenright}{\kern0pt}{\isacharparenright}{\kern0pt}{\isachardoublequoteclose}\isanewline
\ \ \ \ \ \ \isakeywordONE{by}\isamarkupfalse%
\ {\isacharparenleft}{\kern0pt}metis\ Un{\isacharunderscore}{\kern0pt}commute\ conc{\isacharunderscore}{\kern0pt}star{\isacharunderscore}{\kern0pt}comm\ star{\isacharunderscore}{\kern0pt}idemp\ star{\isacharunderscore}{\kern0pt}unfold{\isacharunderscore}{\kern0pt}left{\isacharparenright}{\kern0pt}\isanewline
\ \ \ \ \isakeywordONE{also}\isamarkupfalse%
\ \isakeywordONE{have}\isamarkupfalse%
\ {\isachardoublequoteopen}{\isasymdots}\ {\isacharequal}{\kern0pt}\ {\isasymPsi}\ {\isacharparenleft}{\kern0pt}star\ {\isacharparenleft}{\kern0pt}eval\ p\ v{\isacharparenright}{\kern0pt}\ {\isacharat}{\kern0pt}{\isacharat}{\kern0pt}\ star\ {\isacharparenleft}{\kern0pt}star\ {\isacharparenleft}{\kern0pt}eval\ q\ v\ {\isacharat}{\kern0pt}{\isacharat}{\kern0pt}\ v\ x{\isacharparenright}{\kern0pt}\ {\isacharat}{\kern0pt}{\isacharat}{\kern0pt}\ eval\ q\ v\ {\isacharat}{\kern0pt}{\isacharat}{\kern0pt}\ v\ x{\isacharparenright}{\kern0pt}{\isacharparenright}{\kern0pt}{\isachardoublequoteclose}\isanewline
\ \ \ \ \ \ \isakeywordONE{by}\isamarkupfalse%
\ auto\isanewline
\ \ \ \ \isakeywordONE{also}\isamarkupfalse%
\ \isakeywordONE{have}\isamarkupfalse%
\ {\isachardoublequoteopen}{\isasymdots}\ {\isacharequal}{\kern0pt}\ {\isasymPsi}\ {\isacharparenleft}{\kern0pt}star\ {\isacharparenleft}{\kern0pt}eval\ p\ v{\isacharparenright}{\kern0pt}\ {\isacharat}{\kern0pt}{\isacharat}{\kern0pt}\ {\isacharparenleft}{\kern0pt}{\isacharbraceleft}{\kern0pt}{\isacharbrackleft}{\kern0pt}{\isacharbrackright}{\kern0pt}{\isacharbraceright}{\kern0pt}\ {\isasymunion}\ star\ {\isacharparenleft}{\kern0pt}eval\ q\ v\ {\isacharat}{\kern0pt}{\isacharat}{\kern0pt}\ v\ x{\isacharparenright}{\kern0pt}\isanewline
\ \ \ \ \ \ \ \ {\isacharat}{\kern0pt}{\isacharat}{\kern0pt}\ star\ {\isacharparenleft}{\kern0pt}eval\ q\ v\ {\isacharat}{\kern0pt}{\isacharat}{\kern0pt}\ v\ x{\isacharparenright}{\kern0pt}\ {\isacharat}{\kern0pt}{\isacharat}{\kern0pt}\ eval\ q\ v\ {\isacharat}{\kern0pt}{\isacharat}{\kern0pt}\ v\ x{\isacharparenright}{\kern0pt}{\isacharparenright}{\kern0pt}{\isachardoublequoteclose}\isanewline
\ \ \ \ \ \ \isakeywordONE{using}\isamarkupfalse%
\ parikh{\isacharunderscore}{\kern0pt}img{\isacharunderscore}{\kern0pt}star{\isadigit{2}}\ parikh{\isacharunderscore}{\kern0pt}conc{\isacharunderscore}{\kern0pt}left\ \isakeywordONE{by}\isamarkupfalse%
\ blast\isanewline
\ \ \ \ \isakeywordONE{also}\isamarkupfalse%
\ \isakeywordONE{have}\isamarkupfalse%
\ {\isachardoublequoteopen}{\isasymdots}\ {\isacharequal}{\kern0pt}\ {\isasymPsi}\ {\isacharparenleft}{\kern0pt}star\ {\isacharparenleft}{\kern0pt}eval\ p\ v{\isacharparenright}{\kern0pt}\ {\isacharat}{\kern0pt}{\isacharat}{\kern0pt}\ {\isacharbraceleft}{\kern0pt}{\isacharbrackleft}{\kern0pt}{\isacharbrackright}{\kern0pt}{\isacharbraceright}{\kern0pt}\ {\isasymunion}\ star\ {\isacharparenleft}{\kern0pt}eval\ p\ v{\isacharparenright}{\kern0pt}\ {\isacharat}{\kern0pt}{\isacharat}{\kern0pt}\ star\ {\isacharparenleft}{\kern0pt}eval\ q\ v\ {\isacharat}{\kern0pt}{\isacharat}{\kern0pt}\ v\ x{\isacharparenright}{\kern0pt}\isanewline
\ \ \ \ \ \ \ \ {\isacharat}{\kern0pt}{\isacharat}{\kern0pt}\ star\ {\isacharparenleft}{\kern0pt}eval\ q\ v\ {\isacharat}{\kern0pt}{\isacharat}{\kern0pt}\ v\ x{\isacharparenright}{\kern0pt}\ {\isacharat}{\kern0pt}{\isacharat}{\kern0pt}\ eval\ q\ v\ {\isacharat}{\kern0pt}{\isacharat}{\kern0pt}\ v\ x{\isacharparenright}{\kern0pt}{\isachardoublequoteclose}\ \isakeywordONE{by}\isamarkupfalse%
\ {\isacharparenleft}{\kern0pt}metis\ conc{\isacharunderscore}{\kern0pt}Un{\isacharunderscore}{\kern0pt}distrib{\isacharparenleft}{\kern0pt}{\isadigit{1}}{\isacharparenright}{\kern0pt}{\isacharparenright}{\kern0pt}\isanewline
\ \ \ \ \isakeywordONE{also}\isamarkupfalse%
\ \isakeywordONE{have}\isamarkupfalse%
\ {\isachardoublequoteopen}{\isasymdots}\ {\isacharequal}{\kern0pt}\ {\isasymPsi}\ {\isacharparenleft}{\kern0pt}eval\ {\isacharquery}{\kern0pt}f{\isacharunderscore}{\kern0pt}new\ v{\isacharparenright}{\kern0pt}{\isachardoublequoteclose}\ \isakeywordONE{by}\isamarkupfalse%
\ {\isacharparenleft}{\kern0pt}simp\ add{\isacharcolon}{\kern0pt}\ conc{\isacharunderscore}{\kern0pt}assoc{\isacharparenright}{\kern0pt}\isanewline
\ \ \ \ \isakeywordONE{finally}\isamarkupfalse%
\ \isakeywordTHREE{show}\isamarkupfalse%
\ {\isachardoublequoteopen}{\isasymPsi}\ {\isacharparenleft}{\kern0pt}eval\ {\isacharparenleft}{\kern0pt}Star\ f{\isacharparenright}{\kern0pt}\ v{\isacharparenright}{\kern0pt}\ {\isacharequal}{\kern0pt}\ {\isasymPsi}\ {\isacharparenleft}{\kern0pt}eval\ {\isacharquery}{\kern0pt}f{\isacharunderscore}{\kern0pt}new\ v{\isacharparenright}{\kern0pt}{\isachardoublequoteclose}\ \isakeywordONE{{\isachardot}{\kern0pt}}\isamarkupfalse%
\isanewline
\ \ \isakeywordONE{qed}\isamarkupfalse%
\isanewline
\ \ \isakeywordONE{moreover}\isamarkupfalse%
\ \isakeywordONE{have}\isamarkupfalse%
\ {\isachardoublequoteopen}bipart{\isacharunderscore}{\kern0pt}rlexp\ x\ {\isacharquery}{\kern0pt}f{\isacharunderscore}{\kern0pt}new{\isachardoublequoteclose}\ \isakeywordONE{unfolding}\isamarkupfalse%
\ bipart{\isacharunderscore}{\kern0pt}rlexp{\isacharunderscore}{\kern0pt}def\ \isakeywordONE{using}\isamarkupfalse%
\ p{\isacharunderscore}{\kern0pt}q{\isacharunderscore}{\kern0pt}intro\ \isakeywordONE{by}\isamarkupfalse%
\ fastforce\isanewline
\ \ \isakeywordONE{moreover}\isamarkupfalse%
\ \isakeywordONE{from}\isamarkupfalse%
\ f{\isacharprime}{\kern0pt}{\isacharunderscore}{\kern0pt}intro\ p{\isacharunderscore}{\kern0pt}q{\isacharunderscore}{\kern0pt}intro\ \isakeywordONE{have}\isamarkupfalse%
\ {\isachardoublequoteopen}vars\ {\isacharquery}{\kern0pt}f{\isacharunderscore}{\kern0pt}new\ {\isacharequal}{\kern0pt}\ vars\ {\isacharparenleft}{\kern0pt}Star\ f{\isacharparenright}{\kern0pt}\ {\isasymunion}\ {\isacharbraceleft}{\kern0pt}x{\isacharbraceright}{\kern0pt}{\isachardoublequoteclose}\ \isakeywordONE{by}\isamarkupfalse%
\ auto\isanewline
\ \ \isakeywordONE{ultimately}\isamarkupfalse%
\ \isakeywordTHREE{show}\isamarkupfalse%
\ {\isacharquery}{\kern0pt}thesis\ \isakeywordONE{by}\isamarkupfalse%
\ metis\isanewline
\isakeywordONE{qed}\isamarkupfalse%
%
\endisatagproof
{\isafoldproof}%
%
\isadelimproof
\isanewline
%
\endisadelimproof
\isanewline
\isakeywordONE{lemma}\isamarkupfalse%
\ reg{\isacharunderscore}{\kern0pt}eval{\isacharunderscore}{\kern0pt}bipart{\isacharunderscore}{\kern0pt}rlexp{\isacharcolon}{\kern0pt}\ {\isachardoublequoteopen}reg{\isacharunderscore}{\kern0pt}eval\ f\ {\isasymLongrightarrow}\isanewline
\ \ \ \ {\isasymexists}f{\isacharprime}{\kern0pt}{\isachardot}{\kern0pt}\ bipart{\isacharunderscore}{\kern0pt}rlexp\ x\ f{\isacharprime}{\kern0pt}\ {\isasymand}\ vars\ f{\isacharprime}{\kern0pt}\ {\isacharequal}{\kern0pt}\ vars\ f\ {\isasymunion}\ {\isacharbraceleft}{\kern0pt}x{\isacharbraceright}{\kern0pt}\ {\isasymand}\isanewline
\ \ \ \ \ \ \ \ \ {\isacharparenleft}{\kern0pt}{\isasymforall}s{\isachardot}{\kern0pt}\ {\isasymPsi}\ {\isacharparenleft}{\kern0pt}eval\ f\ s{\isacharparenright}{\kern0pt}\ {\isacharequal}{\kern0pt}\ {\isasymPsi}\ {\isacharparenleft}{\kern0pt}eval\ f{\isacharprime}{\kern0pt}\ s{\isacharparenright}{\kern0pt}{\isacharparenright}{\kern0pt}{\isachardoublequoteclose}\isanewline
%
\isadelimproof
%
\endisadelimproof
%
\isatagproof
\isakeywordONE{proof}\isamarkupfalse%
\ {\isacharparenleft}{\kern0pt}induction\ f\ rule{\isacharcolon}{\kern0pt}\ reg{\isacharunderscore}{\kern0pt}eval{\isachardot}{\kern0pt}induct{\isacharparenright}{\kern0pt}\isanewline
\ \ \isakeywordTHREE{case}\isamarkupfalse%
\ {\isacharparenleft}{\kern0pt}{\isadigit{1}}\ uu{\isacharparenright}{\kern0pt}\isanewline
\ \ \isakeywordONE{from}\isamarkupfalse%
\ reg{\isacharunderscore}{\kern0pt}eval{\isacharunderscore}{\kern0pt}bipart{\isacharunderscore}{\kern0pt}rlexp{\isacharunderscore}{\kern0pt}Variable\ \isakeywordTHREE{show}\isamarkupfalse%
\ {\isacharquery}{\kern0pt}case\ \isakeywordONE{by}\isamarkupfalse%
\ blast\isanewline
\isakeywordONE{next}\isamarkupfalse%
\isanewline
\ \ \isakeywordTHREE{case}\isamarkupfalse%
\ {\isacharparenleft}{\kern0pt}{\isadigit{2}}\ l{\isacharparenright}{\kern0pt}\isanewline
\ \ \isakeywordONE{then}\isamarkupfalse%
\ \isakeywordONE{have}\isamarkupfalse%
\ {\isachardoublequoteopen}regular{\isacharunderscore}{\kern0pt}lang\ l{\isachardoublequoteclose}\ \isakeywordONE{by}\isamarkupfalse%
\ simp\isanewline
\ \ \isakeywordONE{from}\isamarkupfalse%
\ reg{\isacharunderscore}{\kern0pt}eval{\isacharunderscore}{\kern0pt}bipart{\isacharunderscore}{\kern0pt}rlexp{\isacharunderscore}{\kern0pt}Const{\isacharbrackleft}{\kern0pt}OF\ this{\isacharbrackright}{\kern0pt}\ \isakeywordTHREE{show}\isamarkupfalse%
\ {\isacharquery}{\kern0pt}case\ \isakeywordONE{by}\isamarkupfalse%
\ blast\isanewline
\isakeywordONE{next}\isamarkupfalse%
\isanewline
\ \ \isakeywordTHREE{case}\isamarkupfalse%
\ {\isacharparenleft}{\kern0pt}{\isadigit{3}}\ f\ g{\isacharparenright}{\kern0pt}\isanewline
\ \ \isakeywordONE{then}\isamarkupfalse%
\ \isakeywordONE{have}\isamarkupfalse%
\ {\isachardoublequoteopen}{\isasymexists}f{\isacharprime}{\kern0pt}{\isachardot}{\kern0pt}\ bipart{\isacharunderscore}{\kern0pt}rlexp\ x\ f{\isacharprime}{\kern0pt}\ {\isasymand}\ vars\ f{\isacharprime}{\kern0pt}\ {\isacharequal}{\kern0pt}\ vars\ f\ {\isasymunion}\ {\isacharbraceleft}{\kern0pt}x{\isacharbraceright}{\kern0pt}\ {\isasymand}\ {\isacharparenleft}{\kern0pt}{\isasymforall}v{\isachardot}{\kern0pt}\ {\isasymPsi}\ {\isacharparenleft}{\kern0pt}eval\ f\ v{\isacharparenright}{\kern0pt}\ {\isacharequal}{\kern0pt}\ {\isasymPsi}\ {\isacharparenleft}{\kern0pt}eval\ f{\isacharprime}{\kern0pt}\ v{\isacharparenright}{\kern0pt}{\isacharparenright}{\kern0pt}{\isachardoublequoteclose}\isanewline
\ \ \ \ \ \ \ \ \ \ \ \ {\isachardoublequoteopen}{\isasymexists}f{\isacharprime}{\kern0pt}{\isachardot}{\kern0pt}\ bipart{\isacharunderscore}{\kern0pt}rlexp\ x\ f{\isacharprime}{\kern0pt}\ {\isasymand}\ vars\ f{\isacharprime}{\kern0pt}\ {\isacharequal}{\kern0pt}\ vars\ g\ {\isasymunion}\ {\isacharbraceleft}{\kern0pt}x{\isacharbraceright}{\kern0pt}\ {\isasymand}\ {\isacharparenleft}{\kern0pt}{\isasymforall}v{\isachardot}{\kern0pt}\ {\isasymPsi}\ {\isacharparenleft}{\kern0pt}eval\ g\ v{\isacharparenright}{\kern0pt}\ {\isacharequal}{\kern0pt}\ {\isasymPsi}\ {\isacharparenleft}{\kern0pt}eval\ f{\isacharprime}{\kern0pt}\ v{\isacharparenright}{\kern0pt}{\isacharparenright}{\kern0pt}{\isachardoublequoteclose}\isanewline
\ \ \ \ \isakeywordONE{by}\isamarkupfalse%
\ auto\isanewline
\ \ \isakeywordONE{from}\isamarkupfalse%
\ reg{\isacharunderscore}{\kern0pt}eval{\isacharunderscore}{\kern0pt}bipart{\isacharunderscore}{\kern0pt}rlexp{\isacharunderscore}{\kern0pt}Union{\isacharbrackleft}{\kern0pt}OF\ this{\isacharbrackright}{\kern0pt}\ \isakeywordTHREE{show}\isamarkupfalse%
\ {\isacharquery}{\kern0pt}case\ \isakeywordONE{by}\isamarkupfalse%
\ blast\isanewline
\isakeywordONE{next}\isamarkupfalse%
\isanewline
\ \ \isakeywordTHREE{case}\isamarkupfalse%
\ {\isacharparenleft}{\kern0pt}{\isadigit{4}}\ f\ g{\isacharparenright}{\kern0pt}\isanewline
\ \ \isakeywordONE{then}\isamarkupfalse%
\ \isakeywordONE{have}\isamarkupfalse%
\ {\isachardoublequoteopen}{\isasymexists}f{\isacharprime}{\kern0pt}{\isachardot}{\kern0pt}\ bipart{\isacharunderscore}{\kern0pt}rlexp\ x\ f{\isacharprime}{\kern0pt}\ {\isasymand}\ vars\ f{\isacharprime}{\kern0pt}\ {\isacharequal}{\kern0pt}\ vars\ f\ {\isasymunion}\ {\isacharbraceleft}{\kern0pt}x{\isacharbraceright}{\kern0pt}\ {\isasymand}\ {\isacharparenleft}{\kern0pt}{\isasymforall}v{\isachardot}{\kern0pt}\ {\isasymPsi}\ {\isacharparenleft}{\kern0pt}eval\ f\ v{\isacharparenright}{\kern0pt}\ {\isacharequal}{\kern0pt}\ {\isasymPsi}\ {\isacharparenleft}{\kern0pt}eval\ f{\isacharprime}{\kern0pt}\ v{\isacharparenright}{\kern0pt}{\isacharparenright}{\kern0pt}{\isachardoublequoteclose}\isanewline
\ \ \ \ \ \ \ \ \ \ \ \ {\isachardoublequoteopen}{\isasymexists}f{\isacharprime}{\kern0pt}{\isachardot}{\kern0pt}\ bipart{\isacharunderscore}{\kern0pt}rlexp\ x\ f{\isacharprime}{\kern0pt}\ {\isasymand}\ vars\ f{\isacharprime}{\kern0pt}\ {\isacharequal}{\kern0pt}\ vars\ g\ {\isasymunion}\ {\isacharbraceleft}{\kern0pt}x{\isacharbraceright}{\kern0pt}\ {\isasymand}\ {\isacharparenleft}{\kern0pt}{\isasymforall}v{\isachardot}{\kern0pt}\ {\isasymPsi}\ {\isacharparenleft}{\kern0pt}eval\ g\ v{\isacharparenright}{\kern0pt}\ {\isacharequal}{\kern0pt}\ {\isasymPsi}\ {\isacharparenleft}{\kern0pt}eval\ f{\isacharprime}{\kern0pt}\ v{\isacharparenright}{\kern0pt}{\isacharparenright}{\kern0pt}{\isachardoublequoteclose}\isanewline
\ \ \ \ \isakeywordONE{by}\isamarkupfalse%
\ auto\isanewline
\ \ \isakeywordONE{from}\isamarkupfalse%
\ reg{\isacharunderscore}{\kern0pt}eval{\isacharunderscore}{\kern0pt}bipart{\isacharunderscore}{\kern0pt}rlexp{\isacharunderscore}{\kern0pt}Concat{\isacharbrackleft}{\kern0pt}OF\ this{\isacharbrackright}{\kern0pt}\ \isakeywordTHREE{show}\isamarkupfalse%
\ {\isacharquery}{\kern0pt}case\ \isakeywordONE{by}\isamarkupfalse%
\ blast\isanewline
\isakeywordONE{next}\isamarkupfalse%
\isanewline
\ \ \isakeywordTHREE{case}\isamarkupfalse%
\ {\isacharparenleft}{\kern0pt}{\isadigit{5}}\ f{\isacharparenright}{\kern0pt}\isanewline
\ \ \isakeywordONE{then}\isamarkupfalse%
\ \isakeywordONE{have}\isamarkupfalse%
\ {\isachardoublequoteopen}{\isasymexists}f{\isacharprime}{\kern0pt}{\isachardot}{\kern0pt}\ bipart{\isacharunderscore}{\kern0pt}rlexp\ x\ f{\isacharprime}{\kern0pt}\ {\isasymand}\ vars\ f{\isacharprime}{\kern0pt}\ {\isacharequal}{\kern0pt}\ vars\ f\ {\isasymunion}\ {\isacharbraceleft}{\kern0pt}x{\isacharbraceright}{\kern0pt}\ {\isasymand}\ {\isacharparenleft}{\kern0pt}{\isasymforall}v{\isachardot}{\kern0pt}\ {\isasymPsi}\ {\isacharparenleft}{\kern0pt}eval\ f\ v{\isacharparenright}{\kern0pt}\ {\isacharequal}{\kern0pt}\ {\isasymPsi}\ {\isacharparenleft}{\kern0pt}eval\ f{\isacharprime}{\kern0pt}\ v{\isacharparenright}{\kern0pt}{\isacharparenright}{\kern0pt}{\isachardoublequoteclose}\isanewline
\ \ \ \ \isakeywordONE{by}\isamarkupfalse%
\ auto\isanewline
\ \ \isakeywordONE{from}\isamarkupfalse%
\ reg{\isacharunderscore}{\kern0pt}eval{\isacharunderscore}{\kern0pt}bipart{\isacharunderscore}{\kern0pt}rlexp{\isacharunderscore}{\kern0pt}Star{\isacharbrackleft}{\kern0pt}OF\ this{\isacharbrackright}{\kern0pt}\ \isakeywordTHREE{show}\isamarkupfalse%
\ {\isacharquery}{\kern0pt}case\ \isakeywordONE{by}\isamarkupfalse%
\ blast\isanewline
\isakeywordONE{qed}\isamarkupfalse%
%
\endisatagproof
{\isafoldproof}%
%
\isadelimproof
%
\endisadelimproof
%
\isadelimdocument
%
\endisadelimdocument
%
\isatagdocument
%
\isamarkupsubsection{Minimal solution for a single equation%
}
\isamarkuptrue%
%
\endisatagdocument
{\isafolddocument}%
%
\isadelimdocument
%
\endisadelimdocument
%
\begin{isamarkuptext}%
The aim is to prove that every system of equations of the second type
has some minimal solution which is \isa{\isaconst{reg{\isacharunderscore}{\kern0pt}eval}}. In this section, we prove this property
only for the case of a single equation. First we assume that the equation is bipartite but later
in this section we will abandon this assumption.%
\end{isamarkuptext}\isamarkuptrue%
\isakeywordONE{locale}\isamarkupfalse%
\ single{\isacharunderscore}{\kern0pt}bipartite{\isacharunderscore}{\kern0pt}eq\ {\isacharequal}{\kern0pt}\isanewline
\ \ \isakeywordTWO{fixes}\ x\ {\isacharcolon}{\kern0pt}{\isacharcolon}{\kern0pt}\ {\isachardoublequoteopen}nat{\isachardoublequoteclose}\isanewline
\ \ \isakeywordTWO{fixes}\ p\ {\isacharcolon}{\kern0pt}{\isacharcolon}{\kern0pt}\ {\isachardoublequoteopen}{\isacharprime}{\kern0pt}a\ rlexp{\isachardoublequoteclose}\isanewline
\ \ \isakeywordTWO{fixes}\ q\ {\isacharcolon}{\kern0pt}{\isacharcolon}{\kern0pt}\ {\isachardoublequoteopen}{\isacharprime}{\kern0pt}a\ rlexp{\isachardoublequoteclose}\isanewline
\ \ \isakeywordTWO{assumes}\ p{\isacharunderscore}{\kern0pt}reg{\isacharcolon}{\kern0pt}\ \ \ \ \ \ {\isachardoublequoteopen}reg{\isacharunderscore}{\kern0pt}eval\ p{\isachardoublequoteclose}\isanewline
\ \ \isakeywordTWO{assumes}\ q{\isacharunderscore}{\kern0pt}reg{\isacharcolon}{\kern0pt}\ \ \ \ \ \ {\isachardoublequoteopen}reg{\isacharunderscore}{\kern0pt}eval\ q{\isachardoublequoteclose}\isanewline
\ \ \isakeywordTWO{assumes}\ x{\isacharunderscore}{\kern0pt}not{\isacharunderscore}{\kern0pt}in{\isacharunderscore}{\kern0pt}p{\isacharcolon}{\kern0pt}\ {\isachardoublequoteopen}x\ {\isasymnotin}\ vars\ p{\isachardoublequoteclose}\isanewline
\isakeywordTWO{begin}%
\begin{isamarkuptext}%
The equation and the minimal solution look as follows. Here, \isa{x} describes the variable whose
solution is to be determined. In the subsequent lemmas, we prove that the solution is \isa{\isaconst{reg{\isacharunderscore}{\kern0pt}eval}}
and fulfills each of the three conditions of the predicate \isa{\isaconst{partial{\isacharunderscore}{\kern0pt}min{\isacharunderscore}{\kern0pt}sol{\isacharunderscore}{\kern0pt}one{\isacharunderscore}{\kern0pt}ineq}}.
In particular, we will use the lemmas of the sections 2.5 and 2.6 here:%
\end{isamarkuptext}\isamarkuptrue%
\isakeywordONE{abbreviation}\isamarkupfalse%
\ {\isachardoublequoteopen}eq\ {\isasymequiv}\ Union\ p\ {\isacharparenleft}{\kern0pt}Concat\ q\ {\isacharparenleft}{\kern0pt}Var\ x{\isacharparenright}{\kern0pt}{\isacharparenright}{\kern0pt}{\isachardoublequoteclose}\isanewline
\isakeywordONE{abbreviation}\isamarkupfalse%
\ {\isachardoublequoteopen}sol\ {\isasymequiv}\ Concat\ {\isacharparenleft}{\kern0pt}Star\ {\isacharparenleft}{\kern0pt}subst\ {\isacharparenleft}{\kern0pt}Var{\isacharparenleft}{\kern0pt}x\ {\isacharcolon}{\kern0pt}{\isacharequal}{\kern0pt}\ p{\isacharparenright}{\kern0pt}{\isacharparenright}{\kern0pt}\ q{\isacharparenright}{\kern0pt}{\isacharparenright}{\kern0pt}\ p{\isachardoublequoteclose}\isanewline
\isanewline
\isakeywordONE{lemma}\isamarkupfalse%
\ sol{\isacharunderscore}{\kern0pt}is{\isacharunderscore}{\kern0pt}reg{\isacharcolon}{\kern0pt}\ {\isachardoublequoteopen}reg{\isacharunderscore}{\kern0pt}eval\ sol{\isachardoublequoteclose}\isanewline
%
\isadelimproof
%
\endisadelimproof
%
\isatagproof
\isakeywordONE{proof}\isamarkupfalse%
\ {\isacharminus}{\kern0pt}\isanewline
\ \ \isakeywordONE{from}\isamarkupfalse%
\ p{\isacharunderscore}{\kern0pt}reg\ q{\isacharunderscore}{\kern0pt}reg\ \isakeywordONE{have}\isamarkupfalse%
\ r{\isacharunderscore}{\kern0pt}reg{\isacharcolon}{\kern0pt}\ {\isachardoublequoteopen}reg{\isacharunderscore}{\kern0pt}eval\ {\isacharparenleft}{\kern0pt}subst\ {\isacharparenleft}{\kern0pt}Var{\isacharparenleft}{\kern0pt}x\ {\isacharcolon}{\kern0pt}{\isacharequal}{\kern0pt}\ p{\isacharparenright}{\kern0pt}{\isacharparenright}{\kern0pt}\ q{\isacharparenright}{\kern0pt}{\isachardoublequoteclose}\isanewline
\ \ \ \ \isakeywordONE{using}\isamarkupfalse%
\ subst{\isacharunderscore}{\kern0pt}reg{\isacharunderscore}{\kern0pt}eval{\isacharunderscore}{\kern0pt}update\ \isakeywordONE{by}\isamarkupfalse%
\ auto\isanewline
\ \ \isakeywordONE{with}\isamarkupfalse%
\ p{\isacharunderscore}{\kern0pt}reg\ \isakeywordTHREE{show}\isamarkupfalse%
\ {\isachardoublequoteopen}reg{\isacharunderscore}{\kern0pt}eval\ sol{\isachardoublequoteclose}\ \isakeywordONE{by}\isamarkupfalse%
\ auto\isanewline
\isakeywordONE{qed}\isamarkupfalse%
%
\endisatagproof
{\isafoldproof}%
%
\isadelimproof
\isanewline
%
\endisadelimproof
\isanewline
\isakeywordONE{lemma}\isamarkupfalse%
\ sol{\isacharunderscore}{\kern0pt}vars{\isacharcolon}{\kern0pt}\ {\isachardoublequoteopen}vars\ sol\ {\isasymsubseteq}\ vars\ eq\ {\isacharminus}{\kern0pt}\ {\isacharbraceleft}{\kern0pt}x{\isacharbraceright}{\kern0pt}{\isachardoublequoteclose}\isanewline
%
\isadelimproof
%
\endisadelimproof
%
\isatagproof
\isakeywordONE{proof}\isamarkupfalse%
\ {\isacharminus}{\kern0pt}\isanewline
\ \ \isakeywordONE{let}\isamarkupfalse%
\ {\isacharquery}{\kern0pt}upd\ {\isacharequal}{\kern0pt}\ {\isachardoublequoteopen}Var{\isacharparenleft}{\kern0pt}x\ {\isacharcolon}{\kern0pt}{\isacharequal}{\kern0pt}\ p{\isacharparenright}{\kern0pt}{\isachardoublequoteclose}\isanewline
\ \ \isakeywordONE{let}\isamarkupfalse%
\ {\isacharquery}{\kern0pt}subst{\isacharunderscore}{\kern0pt}q\ {\isacharequal}{\kern0pt}\ {\isachardoublequoteopen}subst\ {\isacharquery}{\kern0pt}upd\ q{\isachardoublequoteclose}\isanewline
\ \ \isakeywordONE{from}\isamarkupfalse%
\ x{\isacharunderscore}{\kern0pt}not{\isacharunderscore}{\kern0pt}in{\isacharunderscore}{\kern0pt}p\ \isakeywordONE{have}\isamarkupfalse%
\ vars{\isacharunderscore}{\kern0pt}p{\isacharcolon}{\kern0pt}\ {\isachardoublequoteopen}vars\ p\ {\isasymsubseteq}\ vars\ eq\ {\isacharminus}{\kern0pt}\ {\isacharbraceleft}{\kern0pt}x{\isacharbraceright}{\kern0pt}{\isachardoublequoteclose}\ \isakeywordONE{by}\isamarkupfalse%
\ fastforce\isanewline
\ \ \isakeywordONE{moreover}\isamarkupfalse%
\ \isakeywordONE{have}\isamarkupfalse%
\ {\isachardoublequoteopen}vars\ p\ {\isasymunion}\ vars\ q\ {\isasymsubseteq}\ vars\ eq{\isachardoublequoteclose}\ \isakeywordONE{by}\isamarkupfalse%
\ auto\isanewline
\ \ \isakeywordONE{ultimately}\isamarkupfalse%
\ \isakeywordONE{have}\isamarkupfalse%
\ {\isachardoublequoteopen}vars\ {\isacharquery}{\kern0pt}subst{\isacharunderscore}{\kern0pt}q\ {\isasymsubseteq}\ vars\ eq\ {\isacharminus}{\kern0pt}\ {\isacharbraceleft}{\kern0pt}x{\isacharbraceright}{\kern0pt}{\isachardoublequoteclose}\ \isakeywordONE{using}\isamarkupfalse%
\ vars{\isacharunderscore}{\kern0pt}subst{\isacharunderscore}{\kern0pt}upd{\isacharunderscore}{\kern0pt}upper\ \isakeywordONE{by}\isamarkupfalse%
\ blast\isanewline
\ \ \isakeywordONE{with}\isamarkupfalse%
\ x{\isacharunderscore}{\kern0pt}not{\isacharunderscore}{\kern0pt}in{\isacharunderscore}{\kern0pt}p\ \isakeywordTHREE{show}\isamarkupfalse%
\ {\isacharquery}{\kern0pt}thesis\ \isakeywordONE{by}\isamarkupfalse%
\ auto\isanewline
\isakeywordONE{qed}\isamarkupfalse%
%
\endisatagproof
{\isafoldproof}%
%
\isadelimproof
\isanewline
%
\endisadelimproof
\isanewline
\isakeywordONE{lemma}\isamarkupfalse%
\ sol{\isacharunderscore}{\kern0pt}is{\isacharunderscore}{\kern0pt}sol{\isacharunderscore}{\kern0pt}ineq{\isacharcolon}{\kern0pt}\ {\isachardoublequoteopen}partial{\isacharunderscore}{\kern0pt}sol{\isacharunderscore}{\kern0pt}ineq\ x\ eq\ sol{\isachardoublequoteclose}\isanewline
%
\isadelimproof
%
\endisadelimproof
%
\isatagproof
\isakeywordONE{unfolding}\isamarkupfalse%
\ partial{\isacharunderscore}{\kern0pt}sol{\isacharunderscore}{\kern0pt}ineq{\isacharunderscore}{\kern0pt}def\ \isakeywordONE{proof}\isamarkupfalse%
\ {\isacharparenleft}{\kern0pt}rule\ allI{\isacharcomma}{\kern0pt}\ rule\ impI{\isacharparenright}{\kern0pt}\isanewline
\ \ \isakeywordTHREE{fix}\isamarkupfalse%
\ v\isanewline
\ \ \isakeywordTHREE{assume}\isamarkupfalse%
\ x{\isacharunderscore}{\kern0pt}is{\isacharunderscore}{\kern0pt}sol{\isacharcolon}{\kern0pt}\ {\isachardoublequoteopen}v\ x\ {\isacharequal}{\kern0pt}\ eval\ sol\ v{\isachardoublequoteclose}\isanewline
\ \ \isakeywordONE{let}\isamarkupfalse%
\ {\isacharquery}{\kern0pt}r\ {\isacharequal}{\kern0pt}\ {\isachardoublequoteopen}subst\ {\isacharparenleft}{\kern0pt}Var\ {\isacharparenleft}{\kern0pt}x\ {\isacharcolon}{\kern0pt}{\isacharequal}{\kern0pt}\ p{\isacharparenright}{\kern0pt}{\isacharparenright}{\kern0pt}\ q{\isachardoublequoteclose}\isanewline
\ \ \isakeywordONE{let}\isamarkupfalse%
\ {\isacharquery}{\kern0pt}upd\ {\isacharequal}{\kern0pt}\ {\isachardoublequoteopen}Var{\isacharparenleft}{\kern0pt}x\ {\isacharcolon}{\kern0pt}{\isacharequal}{\kern0pt}\ sol{\isacharparenright}{\kern0pt}{\isachardoublequoteclose}\isanewline
\ \ \isakeywordONE{let}\isamarkupfalse%
\ {\isacharquery}{\kern0pt}q{\isacharunderscore}{\kern0pt}subst\ {\isacharequal}{\kern0pt}\ {\isachardoublequoteopen}subst\ {\isacharquery}{\kern0pt}upd\ q{\isachardoublequoteclose}\isanewline
\ \ \isakeywordONE{let}\isamarkupfalse%
\ {\isacharquery}{\kern0pt}eq{\isacharunderscore}{\kern0pt}subst\ {\isacharequal}{\kern0pt}\ {\isachardoublequoteopen}subst\ {\isacharquery}{\kern0pt}upd\ eq{\isachardoublequoteclose}\isanewline
\ \ \isakeywordONE{have}\isamarkupfalse%
\ homogeneous{\isacharunderscore}{\kern0pt}app{\isacharcolon}{\kern0pt}\ {\isachardoublequoteopen}{\isasymPsi}\ {\isacharparenleft}{\kern0pt}eval\ {\isacharquery}{\kern0pt}q{\isacharunderscore}{\kern0pt}subst\ v{\isacharparenright}{\kern0pt}\ {\isasymsubseteq}\ {\isasymPsi}\ {\isacharparenleft}{\kern0pt}eval\ {\isacharparenleft}{\kern0pt}Concat\ {\isacharparenleft}{\kern0pt}Star\ {\isacharquery}{\kern0pt}r{\isacharparenright}{\kern0pt}\ {\isacharquery}{\kern0pt}r{\isacharparenright}{\kern0pt}\ v{\isacharparenright}{\kern0pt}{\isachardoublequoteclose}\isanewline
\ \ \ \ \isakeywordONE{using}\isamarkupfalse%
\ rlexp{\isacharunderscore}{\kern0pt}homogeneous\ \isakeywordONE{by}\isamarkupfalse%
\ blast\isanewline
\ \ \isakeywordONE{from}\isamarkupfalse%
\ x{\isacharunderscore}{\kern0pt}not{\isacharunderscore}{\kern0pt}in{\isacharunderscore}{\kern0pt}p\ \isakeywordONE{have}\isamarkupfalse%
\ {\isachardoublequoteopen}eval\ {\isacharparenleft}{\kern0pt}subst\ {\isacharquery}{\kern0pt}upd\ p{\isacharparenright}{\kern0pt}\ v\ {\isacharequal}{\kern0pt}\ eval\ p\ v{\isachardoublequoteclose}\ \isakeywordONE{using}\isamarkupfalse%
\ eval{\isacharunderscore}{\kern0pt}vars{\isacharunderscore}{\kern0pt}subst{\isacharbrackleft}{\kern0pt}of\ p{\isacharbrackright}{\kern0pt}\ \isakeywordONE{by}\isamarkupfalse%
\ simp\isanewline
\ \ \isakeywordONE{then}\isamarkupfalse%
\ \isakeywordONE{have}\isamarkupfalse%
\ {\isachardoublequoteopen}{\isasymPsi}\ {\isacharparenleft}{\kern0pt}eval\ {\isacharquery}{\kern0pt}eq{\isacharunderscore}{\kern0pt}subst\ v{\isacharparenright}{\kern0pt}\ {\isacharequal}{\kern0pt}\ {\isasymPsi}\ {\isacharparenleft}{\kern0pt}eval\ p\ v\ {\isasymunion}\ eval\ {\isacharquery}{\kern0pt}q{\isacharunderscore}{\kern0pt}subst\ v\ {\isacharat}{\kern0pt}{\isacharat}{\kern0pt}\ eval\ sol\ v{\isacharparenright}{\kern0pt}{\isachardoublequoteclose}\isanewline
\ \ \ \ \isakeywordONE{by}\isamarkupfalse%
\ simp\isanewline
\ \ \isakeywordONE{also}\isamarkupfalse%
\ \isakeywordONE{have}\isamarkupfalse%
\ {\isachardoublequoteopen}{\isasymdots}\ {\isasymsubseteq}\ {\isasymPsi}\ {\isacharparenleft}{\kern0pt}eval\ p\ v\ {\isasymunion}\ eval\ {\isacharparenleft}{\kern0pt}Concat\ {\isacharparenleft}{\kern0pt}Star\ {\isacharquery}{\kern0pt}r{\isacharparenright}{\kern0pt}\ {\isacharquery}{\kern0pt}r{\isacharparenright}{\kern0pt}\ v\ {\isacharat}{\kern0pt}{\isacharat}{\kern0pt}\ eval\ sol\ v{\isacharparenright}{\kern0pt}{\isachardoublequoteclose}\isanewline
\ \ \ \ \isakeywordONE{using}\isamarkupfalse%
\ homogeneous{\isacharunderscore}{\kern0pt}app\ \isakeywordONE{by}\isamarkupfalse%
\ {\isacharparenleft}{\kern0pt}metis\ dual{\isacharunderscore}{\kern0pt}order{\isachardot}{\kern0pt}refl\ parikh{\isacharunderscore}{\kern0pt}conc{\isacharunderscore}{\kern0pt}right{\isacharunderscore}{\kern0pt}subset\ parikh{\isacharunderscore}{\kern0pt}img{\isacharunderscore}{\kern0pt}Un\ sup{\isachardot}{\kern0pt}mono{\isacharparenright}{\kern0pt}\isanewline
\ \ \isakeywordONE{also}\isamarkupfalse%
\ \isakeywordONE{have}\isamarkupfalse%
\ {\isachardoublequoteopen}{\isasymdots}\ {\isacharequal}{\kern0pt}\ {\isasymPsi}\ {\isacharparenleft}{\kern0pt}eval\ p\ v{\isacharparenright}{\kern0pt}\ {\isasymunion}\isanewline
\ \ \ \ \ \ {\isasymPsi}\ {\isacharparenleft}{\kern0pt}star\ {\isacharparenleft}{\kern0pt}eval\ {\isacharquery}{\kern0pt}r\ v{\isacharparenright}{\kern0pt}\ {\isacharat}{\kern0pt}{\isacharat}{\kern0pt}\ eval\ {\isacharquery}{\kern0pt}r\ v\ {\isacharat}{\kern0pt}{\isacharat}{\kern0pt}\ star\ {\isacharparenleft}{\kern0pt}eval\ {\isacharquery}{\kern0pt}r\ v{\isacharparenright}{\kern0pt}\ {\isacharat}{\kern0pt}{\isacharat}{\kern0pt}\ eval\ p\ v{\isacharparenright}{\kern0pt}{\isachardoublequoteclose}\isanewline
\ \ \ \ \isakeywordONE{by}\isamarkupfalse%
\ {\isacharparenleft}{\kern0pt}simp\ add{\isacharcolon}{\kern0pt}\ conc{\isacharunderscore}{\kern0pt}assoc{\isacharparenright}{\kern0pt}\isanewline
\ \ \isakeywordONE{also}\isamarkupfalse%
\ \isakeywordONE{have}\isamarkupfalse%
\ {\isachardoublequoteopen}{\isasymdots}\ {\isacharequal}{\kern0pt}\ {\isasymPsi}\ {\isacharparenleft}{\kern0pt}eval\ p\ v{\isacharparenright}{\kern0pt}\ {\isasymunion}\isanewline
\ \ \ \ \ \ {\isasymPsi}\ {\isacharparenleft}{\kern0pt}eval\ {\isacharquery}{\kern0pt}r\ v\ {\isacharat}{\kern0pt}{\isacharat}{\kern0pt}\ star\ {\isacharparenleft}{\kern0pt}eval\ {\isacharquery}{\kern0pt}r\ v{\isacharparenright}{\kern0pt}\ {\isacharat}{\kern0pt}{\isacharat}{\kern0pt}\ eval\ p\ v{\isacharparenright}{\kern0pt}{\isachardoublequoteclose}\isanewline
\ \ \ \ \isakeywordONE{using}\isamarkupfalse%
\ parikh{\isacharunderscore}{\kern0pt}img{\isacharunderscore}{\kern0pt}commut\ conc{\isacharunderscore}{\kern0pt}star{\isacharunderscore}{\kern0pt}star\ \isakeywordONE{by}\isamarkupfalse%
\ {\isacharparenleft}{\kern0pt}smt\ {\isacharparenleft}{\kern0pt}verit{\isacharcomma}{\kern0pt}\ best{\isacharparenright}{\kern0pt}\ conc{\isacharunderscore}{\kern0pt}assoc\ conc{\isacharunderscore}{\kern0pt}star{\isacharunderscore}{\kern0pt}comm{\isacharparenright}{\kern0pt}\isanewline
\ \ \isakeywordONE{also}\isamarkupfalse%
\ \isakeywordONE{have}\isamarkupfalse%
\ {\isachardoublequoteopen}{\isasymdots}\ {\isacharequal}{\kern0pt}\ {\isasymPsi}\ {\isacharparenleft}{\kern0pt}star\ {\isacharparenleft}{\kern0pt}eval\ {\isacharquery}{\kern0pt}r\ v{\isacharparenright}{\kern0pt}\ {\isacharat}{\kern0pt}{\isacharat}{\kern0pt}\ eval\ p\ v{\isacharparenright}{\kern0pt}{\isachardoublequoteclose}\isanewline
\ \ \ \ \isakeywordONE{using}\isamarkupfalse%
\ star{\isacharunderscore}{\kern0pt}unfold{\isacharunderscore}{\kern0pt}left\isanewline
\ \ \ \ \isakeywordONE{by}\isamarkupfalse%
\ {\isacharparenleft}{\kern0pt}smt\ {\isacharparenleft}{\kern0pt}verit{\isacharparenright}{\kern0pt}\ conc{\isacharunderscore}{\kern0pt}Un{\isacharunderscore}{\kern0pt}distrib{\isacharparenleft}{\kern0pt}{\isadigit{2}}{\isacharparenright}{\kern0pt}\ conc{\isacharunderscore}{\kern0pt}assoc\ conc{\isacharunderscore}{\kern0pt}epsilon{\isacharparenleft}{\kern0pt}{\isadigit{1}}{\isacharparenright}{\kern0pt}\ parikh{\isacharunderscore}{\kern0pt}img{\isacharunderscore}{\kern0pt}Un\ sup{\isacharunderscore}{\kern0pt}commute{\isacharparenright}{\kern0pt}\isanewline
\ \ \isakeywordONE{finally}\isamarkupfalse%
\ \isakeywordONE{have}\isamarkupfalse%
\ {\isacharasterisk}{\kern0pt}{\isacharcolon}{\kern0pt}\ {\isachardoublequoteopen}{\isasymPsi}\ {\isacharparenleft}{\kern0pt}eval\ {\isacharquery}{\kern0pt}eq{\isacharunderscore}{\kern0pt}subst\ v{\isacharparenright}{\kern0pt}\ {\isasymsubseteq}\ {\isasymPsi}\ {\isacharparenleft}{\kern0pt}v\ x{\isacharparenright}{\kern0pt}{\isachardoublequoteclose}\ \isakeywordONE{using}\isamarkupfalse%
\ x{\isacharunderscore}{\kern0pt}is{\isacharunderscore}{\kern0pt}sol\ \isakeywordONE{by}\isamarkupfalse%
\ simp\isanewline
\ \ \isakeywordONE{from}\isamarkupfalse%
\ x{\isacharunderscore}{\kern0pt}is{\isacharunderscore}{\kern0pt}sol\ \isakeywordONE{have}\isamarkupfalse%
\ {\isachardoublequoteopen}v\ {\isacharequal}{\kern0pt}\ v{\isacharparenleft}{\kern0pt}x\ {\isacharcolon}{\kern0pt}{\isacharequal}{\kern0pt}\ eval\ sol\ v{\isacharparenright}{\kern0pt}{\isachardoublequoteclose}\ \isakeywordONE{using}\isamarkupfalse%
\ fun{\isacharunderscore}{\kern0pt}upd{\isacharunderscore}{\kern0pt}triv\ \isakeywordONE{by}\isamarkupfalse%
\ metis\isanewline
\ \ \isakeywordONE{then}\isamarkupfalse%
\ \isakeywordONE{have}\isamarkupfalse%
\ {\isachardoublequoteopen}eval\ eq\ v\ {\isacharequal}{\kern0pt}\ eval\ {\isacharparenleft}{\kern0pt}subst\ {\isacharparenleft}{\kern0pt}Var{\isacharparenleft}{\kern0pt}x\ {\isacharcolon}{\kern0pt}{\isacharequal}{\kern0pt}\ sol{\isacharparenright}{\kern0pt}{\isacharparenright}{\kern0pt}\ eq{\isacharparenright}{\kern0pt}\ v{\isachardoublequoteclose}\isanewline
\ \ \ \ \isakeywordONE{using}\isamarkupfalse%
\ substitution{\isacharunderscore}{\kern0pt}lemma{\isacharunderscore}{\kern0pt}upd{\isacharbrackleft}{\kern0pt}\isakeywordTWO{where}\ f{\isacharequal}{\kern0pt}eq{\isacharbrackright}{\kern0pt}\ \isakeywordONE{by}\isamarkupfalse%
\ presburger\isanewline
\ \ \isakeywordONE{with}\isamarkupfalse%
\ {\isacharasterisk}{\kern0pt}\ \isakeywordTHREE{show}\isamarkupfalse%
\ {\isachardoublequoteopen}solves{\isacharunderscore}{\kern0pt}ineq{\isacharunderscore}{\kern0pt}comm\ x\ eq\ v{\isachardoublequoteclose}\ \isakeywordONE{unfolding}\isamarkupfalse%
\ solves{\isacharunderscore}{\kern0pt}ineq{\isacharunderscore}{\kern0pt}comm{\isacharunderscore}{\kern0pt}def\ \isakeywordONE{by}\isamarkupfalse%
\ argo\isanewline
\isakeywordONE{qed}\isamarkupfalse%
%
\endisatagproof
{\isafoldproof}%
%
\isadelimproof
\isanewline
%
\endisadelimproof
\isanewline
\isakeywordONE{lemma}\isamarkupfalse%
\ sol{\isacharunderscore}{\kern0pt}is{\isacharunderscore}{\kern0pt}minimal{\isacharcolon}{\kern0pt}\isanewline
\ \ \isakeywordTWO{assumes}\ is{\isacharunderscore}{\kern0pt}sol{\isacharcolon}{\kern0pt}\ \ \ \ {\isachardoublequoteopen}solves{\isacharunderscore}{\kern0pt}ineq{\isacharunderscore}{\kern0pt}comm\ x\ eq\ v{\isachardoublequoteclose}\isanewline
\ \ \ \ \ \ \isakeywordTWO{and}\ sol{\isacharprime}{\kern0pt}{\isacharunderscore}{\kern0pt}s{\isacharcolon}{\kern0pt}\ \ \ \ {\isachardoublequoteopen}v\ x\ {\isacharequal}{\kern0pt}\ eval\ sol{\isacharprime}{\kern0pt}\ v{\isachardoublequoteclose}\isanewline
\ \ \ \ \isakeywordTWO{shows}\ \ \ \ \ \ \ \ \ \ \ \ {\isachardoublequoteopen}{\isasymPsi}\ {\isacharparenleft}{\kern0pt}eval\ sol\ v{\isacharparenright}{\kern0pt}\ {\isasymsubseteq}\ {\isasymPsi}\ {\isacharparenleft}{\kern0pt}v\ x{\isacharparenright}{\kern0pt}{\isachardoublequoteclose}\isanewline
%
\isadelimproof
%
\endisadelimproof
%
\isatagproof
\isakeywordONE{proof}\isamarkupfalse%
\ {\isacharminus}{\kern0pt}\isanewline
\ \ \isakeywordONE{from}\isamarkupfalse%
\ is{\isacharunderscore}{\kern0pt}sol\ sol{\isacharprime}{\kern0pt}{\isacharunderscore}{\kern0pt}s\ \isakeywordONE{have}\isamarkupfalse%
\ is{\isacharunderscore}{\kern0pt}sol{\isacharprime}{\kern0pt}{\isacharcolon}{\kern0pt}\ {\isachardoublequoteopen}{\isasymPsi}\ {\isacharparenleft}{\kern0pt}eval\ q\ v\ {\isacharat}{\kern0pt}{\isacharat}{\kern0pt}\ v\ x\ {\isasymunion}\ eval\ p\ v{\isacharparenright}{\kern0pt}\ {\isasymsubseteq}\ {\isasymPsi}\ {\isacharparenleft}{\kern0pt}v\ x{\isacharparenright}{\kern0pt}{\isachardoublequoteclose}\isanewline
\ \ \ \ \isakeywordONE{unfolding}\isamarkupfalse%
\ solves{\isacharunderscore}{\kern0pt}ineq{\isacharunderscore}{\kern0pt}comm{\isacharunderscore}{\kern0pt}def\ \isakeywordONE{by}\isamarkupfalse%
\ simp\isanewline
\ \ \isakeywordONE{then}\isamarkupfalse%
\ \isakeywordONE{have}\isamarkupfalse%
\ {\isadigit{1}}{\isacharcolon}{\kern0pt}\ {\isachardoublequoteopen}{\isasymPsi}\ {\isacharparenleft}{\kern0pt}eval\ {\isacharparenleft}{\kern0pt}Concat\ {\isacharparenleft}{\kern0pt}Star\ q{\isacharparenright}{\kern0pt}\ p{\isacharparenright}{\kern0pt}\ v{\isacharparenright}{\kern0pt}\ {\isasymsubseteq}\ {\isasymPsi}\ {\isacharparenleft}{\kern0pt}v\ x{\isacharparenright}{\kern0pt}{\isachardoublequoteclose}\isanewline
\ \ \ \ \isakeywordONE{using}\isamarkupfalse%
\ parikh{\isacharunderscore}{\kern0pt}img{\isacharunderscore}{\kern0pt}arden\ \isakeywordONE{by}\isamarkupfalse%
\ auto\isanewline
\ \ \isakeywordONE{from}\isamarkupfalse%
\ is{\isacharunderscore}{\kern0pt}sol{\isacharprime}{\kern0pt}\ \isakeywordONE{have}\isamarkupfalse%
\ {\isachardoublequoteopen}{\isasymPsi}\ {\isacharparenleft}{\kern0pt}eval\ p\ v{\isacharparenright}{\kern0pt}\ {\isasymsubseteq}\ {\isasymPsi}\ {\isacharparenleft}{\kern0pt}eval\ {\isacharparenleft}{\kern0pt}Var\ x{\isacharparenright}{\kern0pt}\ v{\isacharparenright}{\kern0pt}{\isachardoublequoteclose}\ \isakeywordONE{by}\isamarkupfalse%
\ auto\isanewline
\ \ \isakeywordONE{then}\isamarkupfalse%
\ \isakeywordONE{have}\isamarkupfalse%
\ {\isachardoublequoteopen}{\isasymPsi}\ {\isacharparenleft}{\kern0pt}eval\ {\isacharparenleft}{\kern0pt}subst\ {\isacharparenleft}{\kern0pt}Var{\isacharparenleft}{\kern0pt}x\ {\isacharcolon}{\kern0pt}{\isacharequal}{\kern0pt}\ p{\isacharparenright}{\kern0pt}{\isacharparenright}{\kern0pt}\ q{\isacharparenright}{\kern0pt}\ v{\isacharparenright}{\kern0pt}\ {\isasymsubseteq}\ {\isasymPsi}\ {\isacharparenleft}{\kern0pt}eval\ q\ v{\isacharparenright}{\kern0pt}{\isachardoublequoteclose}\isanewline
\ \ \ \ \isakeywordONE{using}\isamarkupfalse%
\ parikh{\isacharunderscore}{\kern0pt}img{\isacharunderscore}{\kern0pt}subst{\isacharunderscore}{\kern0pt}mono{\isacharunderscore}{\kern0pt}upd\ \isakeywordONE{by}\isamarkupfalse%
\ {\isacharparenleft}{\kern0pt}metis\ fun{\isacharunderscore}{\kern0pt}upd{\isacharunderscore}{\kern0pt}triv\ subst{\isacharunderscore}{\kern0pt}id{\isacharparenright}{\kern0pt}\isanewline
\ \ \isakeywordONE{then}\isamarkupfalse%
\ \isakeywordONE{have}\isamarkupfalse%
\ {\isachardoublequoteopen}{\isasymPsi}\ {\isacharparenleft}{\kern0pt}eval\ {\isacharparenleft}{\kern0pt}Star\ {\isacharparenleft}{\kern0pt}subst\ {\isacharparenleft}{\kern0pt}Var{\isacharparenleft}{\kern0pt}x\ {\isacharcolon}{\kern0pt}{\isacharequal}{\kern0pt}\ p{\isacharparenright}{\kern0pt}{\isacharparenright}{\kern0pt}\ q{\isacharparenright}{\kern0pt}{\isacharparenright}{\kern0pt}\ v{\isacharparenright}{\kern0pt}\ {\isasymsubseteq}\ {\isasymPsi}\ {\isacharparenleft}{\kern0pt}eval\ {\isacharparenleft}{\kern0pt}Star\ q{\isacharparenright}{\kern0pt}\ v{\isacharparenright}{\kern0pt}{\isachardoublequoteclose}\isanewline
\ \ \ \ \isakeywordONE{using}\isamarkupfalse%
\ parikh{\isacharunderscore}{\kern0pt}star{\isacharunderscore}{\kern0pt}mono\ \isakeywordONE{by}\isamarkupfalse%
\ auto\isanewline
\ \ \isakeywordONE{then}\isamarkupfalse%
\ \isakeywordONE{have}\isamarkupfalse%
\ {\isachardoublequoteopen}{\isasymPsi}\ {\isacharparenleft}{\kern0pt}eval\ sol\ v{\isacharparenright}{\kern0pt}\ {\isasymsubseteq}\ {\isasymPsi}\ {\isacharparenleft}{\kern0pt}eval\ {\isacharparenleft}{\kern0pt}Concat\ {\isacharparenleft}{\kern0pt}Star\ q{\isacharparenright}{\kern0pt}\ p{\isacharparenright}{\kern0pt}\ v{\isacharparenright}{\kern0pt}{\isachardoublequoteclose}\isanewline
\ \ \ \ \isakeywordONE{using}\isamarkupfalse%
\ parikh{\isacharunderscore}{\kern0pt}conc{\isacharunderscore}{\kern0pt}right{\isacharunderscore}{\kern0pt}subset\ \isakeywordONE{by}\isamarkupfalse%
\ {\isacharparenleft}{\kern0pt}metis\ eval{\isachardot}{\kern0pt}simps{\isacharparenleft}{\kern0pt}{\isadigit{4}}{\isacharparenright}{\kern0pt}{\isacharparenright}{\kern0pt}\isanewline
\ \ \isakeywordONE{with}\isamarkupfalse%
\ {\isadigit{1}}\ \isakeywordTHREE{show}\isamarkupfalse%
\ {\isacharquery}{\kern0pt}thesis\ \isakeywordONE{by}\isamarkupfalse%
\ fast\isanewline
\isakeywordONE{qed}\isamarkupfalse%
%
\endisatagproof
{\isafoldproof}%
%
\isadelimproof
%
\endisadelimproof
%
\begin{isamarkuptext}%
In summary, \isa{sol} is a minimal partial solution and it is \isa{\isaconst{reg{\isacharunderscore}{\kern0pt}eval}}:%
\end{isamarkuptext}\isamarkuptrue%
\isakeywordONE{lemma}\isamarkupfalse%
\ sol{\isacharunderscore}{\kern0pt}is{\isacharunderscore}{\kern0pt}minimal{\isacharunderscore}{\kern0pt}reg{\isacharunderscore}{\kern0pt}sol{\isacharcolon}{\kern0pt}\isanewline
\ \ {\isachardoublequoteopen}reg{\isacharunderscore}{\kern0pt}eval\ sol\ {\isasymand}\ partial{\isacharunderscore}{\kern0pt}min{\isacharunderscore}{\kern0pt}sol{\isacharunderscore}{\kern0pt}one{\isacharunderscore}{\kern0pt}ineq\ x\ eq\ sol{\isachardoublequoteclose}\isanewline
%
\isadelimproof
\ \ %
\endisadelimproof
%
\isatagproof
\isakeywordONE{unfolding}\isamarkupfalse%
\ partial{\isacharunderscore}{\kern0pt}min{\isacharunderscore}{\kern0pt}sol{\isacharunderscore}{\kern0pt}one{\isacharunderscore}{\kern0pt}ineq{\isacharunderscore}{\kern0pt}def\isanewline
\ \ \isakeywordONE{using}\isamarkupfalse%
\ sol{\isacharunderscore}{\kern0pt}is{\isacharunderscore}{\kern0pt}reg\ sol{\isacharunderscore}{\kern0pt}vars\ sol{\isacharunderscore}{\kern0pt}is{\isacharunderscore}{\kern0pt}sol{\isacharunderscore}{\kern0pt}ineq\ sol{\isacharunderscore}{\kern0pt}is{\isacharunderscore}{\kern0pt}minimal\isanewline
\ \ \isakeywordONE{by}\isamarkupfalse%
\ blast%
\endisatagproof
{\isafoldproof}%
%
\isadelimproof
\isanewline
%
\endisadelimproof
\isanewline
\isakeywordTWO{end}\isamarkupfalse%
%
\begin{isamarkuptext}%
As announced at the beginning of this section, we now extend the previous result to arbitrary
equations, i.e.\ we show that each equation has some minimal partial solution which is
\isa{\isaconst{reg{\isacharunderscore}{\kern0pt}eval}}:%
\end{isamarkuptext}\isamarkuptrue%
\isakeywordONE{lemma}\isamarkupfalse%
\ exists{\isacharunderscore}{\kern0pt}minimal{\isacharunderscore}{\kern0pt}reg{\isacharunderscore}{\kern0pt}sol{\isacharcolon}{\kern0pt}\isanewline
\ \ \isakeywordTWO{assumes}\ eq{\isacharunderscore}{\kern0pt}reg{\isacharcolon}{\kern0pt}\ {\isachardoublequoteopen}reg{\isacharunderscore}{\kern0pt}eval\ eq{\isachardoublequoteclose}\isanewline
\ \ \isakeywordTWO{shows}\ {\isachardoublequoteopen}{\isasymexists}sol{\isachardot}{\kern0pt}\ reg{\isacharunderscore}{\kern0pt}eval\ sol\ {\isasymand}\ partial{\isacharunderscore}{\kern0pt}min{\isacharunderscore}{\kern0pt}sol{\isacharunderscore}{\kern0pt}one{\isacharunderscore}{\kern0pt}ineq\ x\ eq\ sol{\isachardoublequoteclose}\isanewline
%
\isadelimproof
%
\endisadelimproof
%
\isatagproof
\isakeywordONE{proof}\isamarkupfalse%
\ {\isacharminus}{\kern0pt}\isanewline
\ \ \isakeywordONE{from}\isamarkupfalse%
\ reg{\isacharunderscore}{\kern0pt}eval{\isacharunderscore}{\kern0pt}bipart{\isacharunderscore}{\kern0pt}rlexp{\isacharbrackleft}{\kern0pt}OF\ eq{\isacharunderscore}{\kern0pt}reg{\isacharbrackright}{\kern0pt}\ \isakeywordTHREE{obtain}\isamarkupfalse%
\ eq{\isacharprime}{\kern0pt}\isanewline
\ \ \ \ \isakeywordTWO{where}\ eq{\isacharprime}{\kern0pt}{\isacharunderscore}{\kern0pt}intro{\isacharcolon}{\kern0pt}\ {\isachardoublequoteopen}bipart{\isacharunderscore}{\kern0pt}rlexp\ x\ eq{\isacharprime}{\kern0pt}\ {\isasymand}\ vars\ eq{\isacharprime}{\kern0pt}\ {\isacharequal}{\kern0pt}\ vars\ eq\ {\isasymunion}\ {\isacharbraceleft}{\kern0pt}x{\isacharbraceright}{\kern0pt}\ {\isasymand}\isanewline
\ \ \ \ \ \ \ \ \ \ \ \ \ \ \ \ \ \ \ \ {\isacharparenleft}{\kern0pt}{\isasymforall}v{\isachardot}{\kern0pt}\ {\isasymPsi}\ {\isacharparenleft}{\kern0pt}eval\ eq\ v{\isacharparenright}{\kern0pt}\ {\isacharequal}{\kern0pt}\ {\isasymPsi}\ {\isacharparenleft}{\kern0pt}eval\ eq{\isacharprime}{\kern0pt}\ v{\isacharparenright}{\kern0pt}{\isacharparenright}{\kern0pt}{\isachardoublequoteclose}\ \isakeywordONE{by}\isamarkupfalse%
\ blast\isanewline
\ \ \isakeywordONE{then}\isamarkupfalse%
\ \isakeywordTHREE{obtain}\isamarkupfalse%
\ p\ q\isanewline
\ \ \ \ \isakeywordTWO{where}\ p{\isacharunderscore}{\kern0pt}q{\isacharunderscore}{\kern0pt}intro{\isacharcolon}{\kern0pt}\ {\isachardoublequoteopen}reg{\isacharunderscore}{\kern0pt}eval\ p\ {\isasymand}\ reg{\isacharunderscore}{\kern0pt}eval\ q\ {\isasymand}\ eq{\isacharprime}{\kern0pt}\ {\isacharequal}{\kern0pt}\ Union\ p\ {\isacharparenleft}{\kern0pt}Concat\ q\ {\isacharparenleft}{\kern0pt}Var\ x{\isacharparenright}{\kern0pt}{\isacharparenright}{\kern0pt}\ {\isasymand}\ x\ {\isasymnotin}\ vars\ p{\isachardoublequoteclose}\isanewline
\ \ \ \ \isakeywordONE{unfolding}\isamarkupfalse%
\ bipart{\isacharunderscore}{\kern0pt}rlexp{\isacharunderscore}{\kern0pt}def\ \isakeywordONE{by}\isamarkupfalse%
\ blast\isanewline
\ \ \isakeywordONE{let}\isamarkupfalse%
\ {\isacharquery}{\kern0pt}sol\ {\isacharequal}{\kern0pt}\ {\isachardoublequoteopen}Concat\ {\isacharparenleft}{\kern0pt}Star\ {\isacharparenleft}{\kern0pt}subst\ {\isacharparenleft}{\kern0pt}Var{\isacharparenleft}{\kern0pt}x\ {\isacharcolon}{\kern0pt}{\isacharequal}{\kern0pt}\ p{\isacharparenright}{\kern0pt}{\isacharparenright}{\kern0pt}\ q{\isacharparenright}{\kern0pt}{\isacharparenright}{\kern0pt}\ p{\isachardoublequoteclose}\isanewline
\ \ \isakeywordONE{from}\isamarkupfalse%
\ p{\isacharunderscore}{\kern0pt}q{\isacharunderscore}{\kern0pt}intro\ \isakeywordONE{have}\isamarkupfalse%
\ sol{\isacharunderscore}{\kern0pt}prop{\isacharcolon}{\kern0pt}\ {\isachardoublequoteopen}reg{\isacharunderscore}{\kern0pt}eval\ {\isacharquery}{\kern0pt}sol\ {\isasymand}\ partial{\isacharunderscore}{\kern0pt}min{\isacharunderscore}{\kern0pt}sol{\isacharunderscore}{\kern0pt}one{\isacharunderscore}{\kern0pt}ineq\ x\ eq{\isacharprime}{\kern0pt}\ {\isacharquery}{\kern0pt}sol{\isachardoublequoteclose}\isanewline
\ \ \ \ \isakeywordONE{using}\isamarkupfalse%
\ single{\isacharunderscore}{\kern0pt}bipartite{\isacharunderscore}{\kern0pt}eq{\isachardot}{\kern0pt}sol{\isacharunderscore}{\kern0pt}is{\isacharunderscore}{\kern0pt}minimal{\isacharunderscore}{\kern0pt}reg{\isacharunderscore}{\kern0pt}sol\ \isakeywordONE{unfolding}\isamarkupfalse%
\ single{\isacharunderscore}{\kern0pt}bipartite{\isacharunderscore}{\kern0pt}eq{\isacharunderscore}{\kern0pt}def\ \isakeywordONE{by}\isamarkupfalse%
\ blast\isanewline
\ \ \isakeywordONE{with}\isamarkupfalse%
\ eq{\isacharprime}{\kern0pt}{\isacharunderscore}{\kern0pt}intro\ \isakeywordONE{have}\isamarkupfalse%
\ {\isachardoublequoteopen}partial{\isacharunderscore}{\kern0pt}min{\isacharunderscore}{\kern0pt}sol{\isacharunderscore}{\kern0pt}one{\isacharunderscore}{\kern0pt}ineq\ x\ eq\ {\isacharquery}{\kern0pt}sol{\isachardoublequoteclose}\isanewline
\ \ \ \ \isakeywordONE{using}\isamarkupfalse%
\ same{\isacharunderscore}{\kern0pt}min{\isacharunderscore}{\kern0pt}sol{\isacharunderscore}{\kern0pt}if{\isacharunderscore}{\kern0pt}same{\isacharunderscore}{\kern0pt}parikh{\isacharunderscore}{\kern0pt}img\ \isakeywordONE{by}\isamarkupfalse%
\ blast\isanewline
\ \ \isakeywordONE{with}\isamarkupfalse%
\ sol{\isacharunderscore}{\kern0pt}prop\ \isakeywordTHREE{show}\isamarkupfalse%
\ {\isacharquery}{\kern0pt}thesis\ \isakeywordONE{by}\isamarkupfalse%
\ blast\isanewline
\isakeywordONE{qed}\isamarkupfalse%
%
\endisatagproof
{\isafoldproof}%
%
\isadelimproof
%
\endisadelimproof
%
\isadelimdocument
%
\endisadelimdocument
%
\isatagdocument
%
\isamarkupsubsection{Minimal solution of the whole system of equations%
}
\isamarkuptrue%
%
\endisatagdocument
{\isafolddocument}%
%
\isadelimdocument
%
\endisadelimdocument
%
\begin{isamarkuptext}%
In this section we will extend the last section's result to whole systems of equations.
For this purpose, we will show by induction on \isa{r} that the first \isa{r} equations have
some minimal partial solution which is \isa{\isaconst{reg{\isacharunderscore}{\kern0pt}eval}}.

We start with the centerpiece of the induction step: If a \isa{\isaconst{reg{\isacharunderscore}{\kern0pt}eval}} and minimal partial solution
\isa{sols} exists for the first \isa{r} equations and furthermore a \isa{\isaconst{reg{\isacharunderscore}{\kern0pt}eval}} and minimal partial solution
\isa{sol{\isacharunderscore}{\kern0pt}r} exists for the \isa{r}-th equation, then there exists a \isa{\isaconst{reg{\isacharunderscore}{\kern0pt}eval}} and minimal partial solution
for the first \isa{Suc\ r} equations as well.%
\end{isamarkuptext}\isamarkuptrue%
\isakeywordONE{locale}\isamarkupfalse%
\ min{\isacharunderscore}{\kern0pt}sol{\isacharunderscore}{\kern0pt}induction{\isacharunderscore}{\kern0pt}step\ {\isacharequal}{\kern0pt}\isanewline
\ \ \isakeywordTWO{fixes}\ r\ {\isacharcolon}{\kern0pt}{\isacharcolon}{\kern0pt}\ nat\isanewline
\ \ \ \ \isakeywordTWO{and}\ sys\ {\isacharcolon}{\kern0pt}{\isacharcolon}{\kern0pt}\ {\isachardoublequoteopen}{\isacharprime}{\kern0pt}a\ eq{\isacharunderscore}{\kern0pt}sys{\isachardoublequoteclose}\isanewline
\ \ \ \ \isakeywordTWO{and}\ sols\ {\isacharcolon}{\kern0pt}{\isacharcolon}{\kern0pt}\ {\isachardoublequoteopen}nat\ {\isasymRightarrow}\ {\isacharprime}{\kern0pt}a\ rlexp{\isachardoublequoteclose}\isanewline
\ \ \ \ \isakeywordTWO{and}\ sol{\isacharunderscore}{\kern0pt}r\ {\isacharcolon}{\kern0pt}{\isacharcolon}{\kern0pt}\ {\isachardoublequoteopen}{\isacharprime}{\kern0pt}a\ rlexp{\isachardoublequoteclose}\isanewline
\ \ \isakeywordTWO{assumes}\ eqs{\isacharunderscore}{\kern0pt}reg{\isacharcolon}{\kern0pt}\ \ \ \ \ \ {\isachardoublequoteopen}{\isasymforall}eq\ {\isasymin}\ set\ sys{\isachardot}{\kern0pt}\ reg{\isacharunderscore}{\kern0pt}eval\ eq{\isachardoublequoteclose}\isanewline
\ \ \ \ \ \ \isakeywordTWO{and}\ sys{\isacharunderscore}{\kern0pt}valid{\isacharcolon}{\kern0pt}\ \ \ \ {\isachardoublequoteopen}{\isasymforall}eq\ {\isasymin}\ set\ sys{\isachardot}{\kern0pt}\ {\isasymforall}x\ {\isasymin}\ vars\ eq{\isachardot}{\kern0pt}\ x\ {\isacharless}{\kern0pt}\ length\ sys{\isachardoublequoteclose}\isanewline
\ \ \ \ \ \ \isakeywordTWO{and}\ r{\isacharunderscore}{\kern0pt}valid{\isacharcolon}{\kern0pt}\ \ \ \ \ \ {\isachardoublequoteopen}r\ {\isacharless}{\kern0pt}\ length\ sys{\isachardoublequoteclose}\isanewline
\ \ \ \ \ \ \isakeywordTWO{and}\ sols{\isacharunderscore}{\kern0pt}is{\isacharunderscore}{\kern0pt}sol{\isacharcolon}{\kern0pt}\ \ {\isachardoublequoteopen}partial{\isacharunderscore}{\kern0pt}min{\isacharunderscore}{\kern0pt}sol{\isacharunderscore}{\kern0pt}ineq{\isacharunderscore}{\kern0pt}sys\ r\ sys\ sols{\isachardoublequoteclose}\isanewline
\ \ \ \ \ \ \isakeywordTWO{and}\ sols{\isacharunderscore}{\kern0pt}reg{\isacharcolon}{\kern0pt}\ \ \ \ \ {\isachardoublequoteopen}{\isasymforall}i{\isachardot}{\kern0pt}\ reg{\isacharunderscore}{\kern0pt}eval\ {\isacharparenleft}{\kern0pt}sols\ i{\isacharparenright}{\kern0pt}{\isachardoublequoteclose}\isanewline
\ \ \ \ \ \ \isakeywordTWO{and}\ sol{\isacharunderscore}{\kern0pt}r{\isacharunderscore}{\kern0pt}is{\isacharunderscore}{\kern0pt}sol{\isacharcolon}{\kern0pt}\ {\isachardoublequoteopen}partial{\isacharunderscore}{\kern0pt}min{\isacharunderscore}{\kern0pt}sol{\isacharunderscore}{\kern0pt}one{\isacharunderscore}{\kern0pt}ineq\ r\ {\isacharparenleft}{\kern0pt}subst{\isacharunderscore}{\kern0pt}sys\ sols\ sys\ {\isacharbang}{\kern0pt}\ r{\isacharparenright}{\kern0pt}\ sol{\isacharunderscore}{\kern0pt}r{\isachardoublequoteclose}\isanewline
\ \ \ \ \ \ \isakeywordTWO{and}\ sol{\isacharunderscore}{\kern0pt}r{\isacharunderscore}{\kern0pt}reg{\isacharcolon}{\kern0pt}\ \ \ \ {\isachardoublequoteopen}reg{\isacharunderscore}{\kern0pt}eval\ sol{\isacharunderscore}{\kern0pt}r{\isachardoublequoteclose}\isanewline
\isakeywordTWO{begin}%
\begin{isamarkuptext}%
Throughout the proof, a modified system of equations will be occasionally used to simplify
the proof; this modified system is obtained by substituting the partial solutions of
the first \isa{r} equations into the original system. Additionally
we retrieve a partial solution for the first \isa{Suc\ r} equations - named \isa{sols{\isacharprime}{\kern0pt}} - by substituting the partial
solution of the \isa{r}-th equation into the partial solutions of each of the first \isa{r} equations:%
\end{isamarkuptext}\isamarkuptrue%
\isakeywordONE{abbreviation}\isamarkupfalse%
\ {\isachardoublequoteopen}sys{\isacharprime}{\kern0pt}\ {\isasymequiv}\ subst{\isacharunderscore}{\kern0pt}sys\ sols\ sys{\isachardoublequoteclose}\isanewline
\isakeywordONE{abbreviation}\isamarkupfalse%
\ {\isachardoublequoteopen}sols{\isacharprime}{\kern0pt}\ {\isasymequiv}\ {\isasymlambda}i{\isachardot}{\kern0pt}\ subst\ {\isacharparenleft}{\kern0pt}Var{\isacharparenleft}{\kern0pt}r\ {\isacharcolon}{\kern0pt}{\isacharequal}{\kern0pt}\ sol{\isacharunderscore}{\kern0pt}r{\isacharparenright}{\kern0pt}{\isacharparenright}{\kern0pt}\ {\isacharparenleft}{\kern0pt}sols\ i{\isacharparenright}{\kern0pt}{\isachardoublequoteclose}\isanewline
\isanewline
\isakeywordONE{lemma}\isamarkupfalse%
\ sols{\isacharprime}{\kern0pt}{\isacharunderscore}{\kern0pt}r{\isacharcolon}{\kern0pt}\ {\isachardoublequoteopen}sols{\isacharprime}{\kern0pt}\ r\ {\isacharequal}{\kern0pt}\ sol{\isacharunderscore}{\kern0pt}r{\isachardoublequoteclose}\isanewline
%
\isadelimproof
\ \ %
\endisadelimproof
%
\isatagproof
\isakeywordONE{using}\isamarkupfalse%
\ sols{\isacharunderscore}{\kern0pt}is{\isacharunderscore}{\kern0pt}sol\ \isakeywordONE{unfolding}\isamarkupfalse%
\ partial{\isacharunderscore}{\kern0pt}min{\isacharunderscore}{\kern0pt}sol{\isacharunderscore}{\kern0pt}ineq{\isacharunderscore}{\kern0pt}sys{\isacharunderscore}{\kern0pt}def\ \isakeywordONE{by}\isamarkupfalse%
\ simp%
\endisatagproof
{\isafoldproof}%
%
\isadelimproof
%
\endisadelimproof
%
\begin{isamarkuptext}%
The next lemmas show that \isa{\isaconst{sols{\isacharprime}{\kern0pt}}} is still \isa{\isaconst{reg{\isacharunderscore}{\kern0pt}eval}} and that it complies with
each of the four conditions defined by the predicate \isa{\isaconst{partial{\isacharunderscore}{\kern0pt}min{\isacharunderscore}{\kern0pt}sol{\isacharunderscore}{\kern0pt}ineq{\isacharunderscore}{\kern0pt}sys}}:%
\end{isamarkuptext}\isamarkuptrue%
\isakeywordONE{lemma}\isamarkupfalse%
\ sols{\isacharprime}{\kern0pt}{\isacharunderscore}{\kern0pt}reg{\isacharcolon}{\kern0pt}\ {\isachardoublequoteopen}{\isasymforall}i{\isachardot}{\kern0pt}\ reg{\isacharunderscore}{\kern0pt}eval\ {\isacharparenleft}{\kern0pt}sols{\isacharprime}{\kern0pt}\ i{\isacharparenright}{\kern0pt}{\isachardoublequoteclose}\isanewline
%
\isadelimproof
\ \ %
\endisadelimproof
%
\isatagproof
\isakeywordONE{using}\isamarkupfalse%
\ sols{\isacharunderscore}{\kern0pt}reg\ sol{\isacharunderscore}{\kern0pt}r{\isacharunderscore}{\kern0pt}reg\ \isakeywordONE{using}\isamarkupfalse%
\ subst{\isacharunderscore}{\kern0pt}reg{\isacharunderscore}{\kern0pt}eval{\isacharunderscore}{\kern0pt}update\ \isakeywordONE{by}\isamarkupfalse%
\ blast%
\endisatagproof
{\isafoldproof}%
%
\isadelimproof
\isanewline
%
\endisadelimproof
\isanewline
\isakeywordONE{lemma}\isamarkupfalse%
\ sols{\isacharprime}{\kern0pt}{\isacharunderscore}{\kern0pt}is{\isacharunderscore}{\kern0pt}sol{\isacharcolon}{\kern0pt}\ {\isachardoublequoteopen}solution{\isacharunderscore}{\kern0pt}ineq{\isacharunderscore}{\kern0pt}sys\ {\isacharparenleft}{\kern0pt}take\ {\isacharparenleft}{\kern0pt}Suc\ r{\isacharparenright}{\kern0pt}\ sys{\isacharparenright}{\kern0pt}\ sols{\isacharprime}{\kern0pt}{\isachardoublequoteclose}\isanewline
%
\isadelimproof
%
\endisadelimproof
%
\isatagproof
\isakeywordONE{unfolding}\isamarkupfalse%
\ solution{\isacharunderscore}{\kern0pt}ineq{\isacharunderscore}{\kern0pt}sys{\isacharunderscore}{\kern0pt}def\ \isakeywordONE{proof}\isamarkupfalse%
\ {\isacharparenleft}{\kern0pt}rule\ allI{\isacharcomma}{\kern0pt}\ rule\ impI{\isacharparenright}{\kern0pt}\isanewline
\ \ \isakeywordTHREE{fix}\isamarkupfalse%
\ v\isanewline
\ \ \isakeywordTHREE{assume}\isamarkupfalse%
\ s{\isacharunderscore}{\kern0pt}sols{\isacharprime}{\kern0pt}{\isacharcolon}{\kern0pt}\ {\isachardoublequoteopen}{\isasymforall}x{\isachardot}{\kern0pt}\ v\ x\ {\isacharequal}{\kern0pt}\ eval\ {\isacharparenleft}{\kern0pt}sols{\isacharprime}{\kern0pt}\ x{\isacharparenright}{\kern0pt}\ v{\isachardoublequoteclose}\isanewline
\ \ \isakeywordONE{from}\isamarkupfalse%
\ sols{\isacharprime}{\kern0pt}{\isacharunderscore}{\kern0pt}r\ s{\isacharunderscore}{\kern0pt}sols{\isacharprime}{\kern0pt}\ \isakeywordONE{have}\isamarkupfalse%
\ s{\isacharunderscore}{\kern0pt}r{\isacharunderscore}{\kern0pt}sol{\isacharunderscore}{\kern0pt}r{\isacharcolon}{\kern0pt}\ {\isachardoublequoteopen}v\ r\ {\isacharequal}{\kern0pt}\ eval\ sol{\isacharunderscore}{\kern0pt}r\ v{\isachardoublequoteclose}\ \isakeywordONE{by}\isamarkupfalse%
\ simp\isanewline
\ \ \isakeywordONE{with}\isamarkupfalse%
\ s{\isacharunderscore}{\kern0pt}sols{\isacharprime}{\kern0pt}\ \isakeywordONE{have}\isamarkupfalse%
\ s{\isacharunderscore}{\kern0pt}sols{\isacharcolon}{\kern0pt}\ {\isachardoublequoteopen}v\ x\ {\isacharequal}{\kern0pt}\ eval\ {\isacharparenleft}{\kern0pt}sols\ x{\isacharparenright}{\kern0pt}\ v{\isachardoublequoteclose}\ \isakeywordTWO{for}\ x\isanewline
\ \ \ \ \isakeywordONE{using}\isamarkupfalse%
\ substitution{\isacharunderscore}{\kern0pt}lemma{\isacharunderscore}{\kern0pt}upd{\isacharbrackleft}{\kern0pt}\isakeywordTWO{where}\ f{\isacharequal}{\kern0pt}{\isachardoublequoteopen}sols\ x{\isachardoublequoteclose}{\isacharbrackright}{\kern0pt}\ \isakeywordONE{by}\isamarkupfalse%
\ {\isacharparenleft}{\kern0pt}auto\ simp\ add{\isacharcolon}{\kern0pt}\ fun{\isacharunderscore}{\kern0pt}upd{\isacharunderscore}{\kern0pt}idem{\isacharparenright}{\kern0pt}\isanewline
\ \ \isakeywordONE{with}\isamarkupfalse%
\ sols{\isacharunderscore}{\kern0pt}is{\isacharunderscore}{\kern0pt}sol\ \isakeywordONE{have}\isamarkupfalse%
\ solves{\isacharunderscore}{\kern0pt}r{\isacharunderscore}{\kern0pt}sys{\isacharcolon}{\kern0pt}\ {\isachardoublequoteopen}solves{\isacharunderscore}{\kern0pt}ineq{\isacharunderscore}{\kern0pt}sys{\isacharunderscore}{\kern0pt}comm\ {\isacharparenleft}{\kern0pt}take\ r\ sys{\isacharparenright}{\kern0pt}\ v{\isachardoublequoteclose}\isanewline
\ \ \ \ \isakeywordONE{unfolding}\isamarkupfalse%
\ partial{\isacharunderscore}{\kern0pt}min{\isacharunderscore}{\kern0pt}sol{\isacharunderscore}{\kern0pt}ineq{\isacharunderscore}{\kern0pt}sys{\isacharunderscore}{\kern0pt}def\ solution{\isacharunderscore}{\kern0pt}ineq{\isacharunderscore}{\kern0pt}sys{\isacharunderscore}{\kern0pt}def\ \isakeywordONE{by}\isamarkupfalse%
\ meson\isanewline
\ \ \isakeywordONE{have}\isamarkupfalse%
\ {\isachardoublequoteopen}eval\ {\isacharparenleft}{\kern0pt}sys\ {\isacharbang}{\kern0pt}\ r{\isacharparenright}{\kern0pt}\ {\isacharparenleft}{\kern0pt}{\isasymlambda}y{\isachardot}{\kern0pt}\ eval\ {\isacharparenleft}{\kern0pt}sols\ y{\isacharparenright}{\kern0pt}\ v{\isacharparenright}{\kern0pt}\ {\isacharequal}{\kern0pt}\ eval\ {\isacharparenleft}{\kern0pt}sys{\isacharprime}{\kern0pt}\ {\isacharbang}{\kern0pt}\ r{\isacharparenright}{\kern0pt}\ v{\isachardoublequoteclose}\isanewline
\ \ \ \ \isakeywordONE{using}\isamarkupfalse%
\ substitution{\isacharunderscore}{\kern0pt}lemma{\isacharbrackleft}{\kern0pt}of\ {\isachardoublequoteopen}{\isasymlambda}y{\isachardot}{\kern0pt}\ eval\ {\isacharparenleft}{\kern0pt}sols\ y{\isacharparenright}{\kern0pt}\ v{\isachardoublequoteclose}{\isacharbrackright}{\kern0pt}\isanewline
\ \ \ \ \isakeywordONE{by}\isamarkupfalse%
\ {\isacharparenleft}{\kern0pt}simp\ add{\isacharcolon}{\kern0pt}\ r{\isacharunderscore}{\kern0pt}valid\ Suc{\isacharunderscore}{\kern0pt}le{\isacharunderscore}{\kern0pt}lessD\ subst{\isacharunderscore}{\kern0pt}sys{\isacharunderscore}{\kern0pt}subst{\isacharparenright}{\kern0pt}\isanewline
\ \ \isakeywordONE{with}\isamarkupfalse%
\ s{\isacharunderscore}{\kern0pt}sols\ \isakeywordONE{have}\isamarkupfalse%
\ {\isachardoublequoteopen}eval\ {\isacharparenleft}{\kern0pt}sys\ {\isacharbang}{\kern0pt}\ r{\isacharparenright}{\kern0pt}\ v\ {\isacharequal}{\kern0pt}\ eval\ {\isacharparenleft}{\kern0pt}sys{\isacharprime}{\kern0pt}\ {\isacharbang}{\kern0pt}\ r{\isacharparenright}{\kern0pt}\ v{\isachardoublequoteclose}\isanewline
\ \ \ \ \isakeywordONE{by}\isamarkupfalse%
\ {\isacharparenleft}{\kern0pt}metis\ {\isacharparenleft}{\kern0pt}mono{\isacharunderscore}{\kern0pt}tags{\isacharcomma}{\kern0pt}\ lifting{\isacharparenright}{\kern0pt}\ eval{\isacharunderscore}{\kern0pt}vars{\isacharparenright}{\kern0pt}\isanewline
\ \ \isakeywordONE{with}\isamarkupfalse%
\ sol{\isacharunderscore}{\kern0pt}r{\isacharunderscore}{\kern0pt}is{\isacharunderscore}{\kern0pt}sol\ s{\isacharunderscore}{\kern0pt}r{\isacharunderscore}{\kern0pt}sol{\isacharunderscore}{\kern0pt}r\ \isakeywordONE{have}\isamarkupfalse%
\ {\isachardoublequoteopen}{\isasymPsi}\ {\isacharparenleft}{\kern0pt}eval\ {\isacharparenleft}{\kern0pt}sys\ {\isacharbang}{\kern0pt}\ r{\isacharparenright}{\kern0pt}\ v{\isacharparenright}{\kern0pt}\ {\isasymsubseteq}\ {\isasymPsi}\ {\isacharparenleft}{\kern0pt}v\ r{\isacharparenright}{\kern0pt}{\isachardoublequoteclose}\isanewline
\ \ \ \ \isakeywordONE{unfolding}\isamarkupfalse%
\ partial{\isacharunderscore}{\kern0pt}min{\isacharunderscore}{\kern0pt}sol{\isacharunderscore}{\kern0pt}one{\isacharunderscore}{\kern0pt}ineq{\isacharunderscore}{\kern0pt}def\ partial{\isacharunderscore}{\kern0pt}sol{\isacharunderscore}{\kern0pt}ineq{\isacharunderscore}{\kern0pt}def\ solves{\isacharunderscore}{\kern0pt}ineq{\isacharunderscore}{\kern0pt}comm{\isacharunderscore}{\kern0pt}def\ \isakeywordONE{by}\isamarkupfalse%
\ simp\isanewline
\ \ \isakeywordONE{with}\isamarkupfalse%
\ solves{\isacharunderscore}{\kern0pt}r{\isacharunderscore}{\kern0pt}sys\ \isakeywordTHREE{show}\isamarkupfalse%
\ {\isachardoublequoteopen}solves{\isacharunderscore}{\kern0pt}ineq{\isacharunderscore}{\kern0pt}sys{\isacharunderscore}{\kern0pt}comm\ {\isacharparenleft}{\kern0pt}take\ {\isacharparenleft}{\kern0pt}Suc\ r{\isacharparenright}{\kern0pt}\ sys{\isacharparenright}{\kern0pt}\ v{\isachardoublequoteclose}\isanewline
\ \ \ \ \isakeywordONE{unfolding}\isamarkupfalse%
\ solves{\isacharunderscore}{\kern0pt}ineq{\isacharunderscore}{\kern0pt}sys{\isacharunderscore}{\kern0pt}comm{\isacharunderscore}{\kern0pt}def\ solves{\isacharunderscore}{\kern0pt}ineq{\isacharunderscore}{\kern0pt}comm{\isacharunderscore}{\kern0pt}def\ \isakeywordONE{by}\isamarkupfalse%
\ {\isacharparenleft}{\kern0pt}auto\ simp\ add{\isacharcolon}{\kern0pt}\ less{\isacharunderscore}{\kern0pt}Suc{\isacharunderscore}{\kern0pt}eq{\isacharparenright}{\kern0pt}\isanewline
\isakeywordONE{qed}\isamarkupfalse%
%
\endisatagproof
{\isafoldproof}%
%
\isadelimproof
\isanewline
%
\endisadelimproof
\isanewline
\isakeywordONE{lemma}\isamarkupfalse%
\ sols{\isacharprime}{\kern0pt}{\isacharunderscore}{\kern0pt}min{\isacharcolon}{\kern0pt}\ {\isachardoublequoteopen}{\isasymforall}sols{\isadigit{2}}\ v{\isadigit{2}}{\isachardot}{\kern0pt}\ {\isacharparenleft}{\kern0pt}{\isasymforall}x{\isachardot}{\kern0pt}\ v{\isadigit{2}}\ x\ {\isacharequal}{\kern0pt}\ eval\ {\isacharparenleft}{\kern0pt}sols{\isadigit{2}}\ x{\isacharparenright}{\kern0pt}\ v{\isadigit{2}}{\isacharparenright}{\kern0pt}\isanewline
\ \ \ \ \ \ \ \ \ \ \ \ \ \ \ \ \ \ \ {\isasymand}\ solves{\isacharunderscore}{\kern0pt}ineq{\isacharunderscore}{\kern0pt}sys{\isacharunderscore}{\kern0pt}comm\ {\isacharparenleft}{\kern0pt}take\ {\isacharparenleft}{\kern0pt}Suc\ r{\isacharparenright}{\kern0pt}\ sys{\isacharparenright}{\kern0pt}\ v{\isadigit{2}}\isanewline
\ \ \ \ \ \ \ \ \ \ \ \ \ \ \ \ \ \ \ {\isasymlongrightarrow}\ {\isacharparenleft}{\kern0pt}{\isasymforall}i{\isachardot}{\kern0pt}\ {\isasymPsi}\ {\isacharparenleft}{\kern0pt}eval\ {\isacharparenleft}{\kern0pt}sols{\isacharprime}{\kern0pt}\ i{\isacharparenright}{\kern0pt}\ v{\isadigit{2}}{\isacharparenright}{\kern0pt}\ {\isasymsubseteq}\ {\isasymPsi}\ {\isacharparenleft}{\kern0pt}v{\isadigit{2}}\ i{\isacharparenright}{\kern0pt}{\isacharparenright}{\kern0pt}{\isachardoublequoteclose}\isanewline
%
\isadelimproof
%
\endisadelimproof
%
\isatagproof
\isakeywordONE{proof}\isamarkupfalse%
\ {\isacharparenleft}{\kern0pt}rule\ allI\ {\isacharbar}{\kern0pt}\ rule\ impI{\isacharparenright}{\kern0pt}{\isacharplus}{\kern0pt}\isanewline
\ \ \isakeywordTHREE{fix}\isamarkupfalse%
\ sols{\isadigit{2}}\ v{\isadigit{2}}\ i\isanewline
\ \ \isakeywordTHREE{assume}\isamarkupfalse%
\ as{\isacharcolon}{\kern0pt}\ {\isachardoublequoteopen}{\isacharparenleft}{\kern0pt}{\isasymforall}x{\isachardot}{\kern0pt}\ v{\isadigit{2}}\ x\ {\isacharequal}{\kern0pt}\ eval\ {\isacharparenleft}{\kern0pt}sols{\isadigit{2}}\ x{\isacharparenright}{\kern0pt}\ v{\isadigit{2}}{\isacharparenright}{\kern0pt}\ {\isasymand}\ solves{\isacharunderscore}{\kern0pt}ineq{\isacharunderscore}{\kern0pt}sys{\isacharunderscore}{\kern0pt}comm\ {\isacharparenleft}{\kern0pt}take\ {\isacharparenleft}{\kern0pt}Suc\ r{\isacharparenright}{\kern0pt}\ sys{\isacharparenright}{\kern0pt}\ v{\isadigit{2}}{\isachardoublequoteclose}\isanewline
\ \ \isakeywordONE{then}\isamarkupfalse%
\ \isakeywordONE{have}\isamarkupfalse%
\ {\isachardoublequoteopen}solves{\isacharunderscore}{\kern0pt}ineq{\isacharunderscore}{\kern0pt}sys{\isacharunderscore}{\kern0pt}comm\ {\isacharparenleft}{\kern0pt}take\ r\ sys{\isacharparenright}{\kern0pt}\ v{\isadigit{2}}{\isachardoublequoteclose}\ \isakeywordONE{unfolding}\isamarkupfalse%
\ solves{\isacharunderscore}{\kern0pt}ineq{\isacharunderscore}{\kern0pt}sys{\isacharunderscore}{\kern0pt}comm{\isacharunderscore}{\kern0pt}def\ \isakeywordONE{by}\isamarkupfalse%
\ fastforce\isanewline
\ \ \isakeywordONE{with}\isamarkupfalse%
\ as\ sols{\isacharunderscore}{\kern0pt}is{\isacharunderscore}{\kern0pt}sol\ \isakeywordONE{have}\isamarkupfalse%
\ sols{\isacharunderscore}{\kern0pt}s{\isadigit{2}}{\isacharcolon}{\kern0pt}\ {\isachardoublequoteopen}{\isasymPsi}\ {\isacharparenleft}{\kern0pt}eval\ {\isacharparenleft}{\kern0pt}sols\ i{\isacharparenright}{\kern0pt}\ v{\isadigit{2}}{\isacharparenright}{\kern0pt}\ {\isasymsubseteq}\ {\isasymPsi}\ {\isacharparenleft}{\kern0pt}v{\isadigit{2}}\ i{\isacharparenright}{\kern0pt}{\isachardoublequoteclose}\ \isakeywordTWO{for}\ i\isanewline
\ \ \ \ \isakeywordONE{unfolding}\isamarkupfalse%
\ partial{\isacharunderscore}{\kern0pt}min{\isacharunderscore}{\kern0pt}sol{\isacharunderscore}{\kern0pt}ineq{\isacharunderscore}{\kern0pt}sys{\isacharunderscore}{\kern0pt}def\ \isakeywordONE{by}\isamarkupfalse%
\ auto\isanewline
\ \ \isakeywordONE{have}\isamarkupfalse%
\ {\isachardoublequoteopen}eval\ {\isacharparenleft}{\kern0pt}sys{\isacharprime}{\kern0pt}\ {\isacharbang}{\kern0pt}\ r{\isacharparenright}{\kern0pt}\ v{\isadigit{2}}\ {\isacharequal}{\kern0pt}\ eval\ {\isacharparenleft}{\kern0pt}sys\ {\isacharbang}{\kern0pt}\ r{\isacharparenright}{\kern0pt}\ {\isacharparenleft}{\kern0pt}{\isasymlambda}i{\isachardot}{\kern0pt}\ eval\ {\isacharparenleft}{\kern0pt}sols\ i{\isacharparenright}{\kern0pt}\ v{\isadigit{2}}{\isacharparenright}{\kern0pt}{\isachardoublequoteclose}\isanewline
\ \ \ \ \isakeywordONE{unfolding}\isamarkupfalse%
\ subst{\isacharunderscore}{\kern0pt}sys{\isacharunderscore}{\kern0pt}def\ \isakeywordONE{using}\isamarkupfalse%
\ substitution{\isacharunderscore}{\kern0pt}lemma{\isacharbrackleft}{\kern0pt}\isakeywordTWO{where}\ f{\isacharequal}{\kern0pt}{\isachardoublequoteopen}sys\ {\isacharbang}{\kern0pt}\ r{\isachardoublequoteclose}{\isacharbrackright}{\kern0pt}\isanewline
\ \ \ \ \isakeywordONE{by}\isamarkupfalse%
\ {\isacharparenleft}{\kern0pt}simp\ add{\isacharcolon}{\kern0pt}\ r{\isacharunderscore}{\kern0pt}valid\ Suc{\isacharunderscore}{\kern0pt}le{\isacharunderscore}{\kern0pt}lessD{\isacharparenright}{\kern0pt}\isanewline
\ \ \isakeywordONE{with}\isamarkupfalse%
\ sols{\isacharunderscore}{\kern0pt}s{\isadigit{2}}\ \isakeywordONE{have}\isamarkupfalse%
\ {\isachardoublequoteopen}{\isasymPsi}\ {\isacharparenleft}{\kern0pt}eval\ {\isacharparenleft}{\kern0pt}sys{\isacharprime}{\kern0pt}\ {\isacharbang}{\kern0pt}\ r{\isacharparenright}{\kern0pt}\ v{\isadigit{2}}{\isacharparenright}{\kern0pt}\ {\isasymsubseteq}\ {\isasymPsi}\ {\isacharparenleft}{\kern0pt}eval\ {\isacharparenleft}{\kern0pt}sys\ {\isacharbang}{\kern0pt}\ r{\isacharparenright}{\kern0pt}\ v{\isadigit{2}}{\isacharparenright}{\kern0pt}{\isachardoublequoteclose}\isanewline
\ \ \ \ \isakeywordONE{using}\isamarkupfalse%
\ rlexp{\isacharunderscore}{\kern0pt}mono{\isacharunderscore}{\kern0pt}parikh{\isacharbrackleft}{\kern0pt}of\ {\isachardoublequoteopen}sys\ {\isacharbang}{\kern0pt}\ r{\isachardoublequoteclose}{\isacharbrackright}{\kern0pt}\ \isakeywordONE{by}\isamarkupfalse%
\ auto\isanewline
\ \ \isakeywordONE{with}\isamarkupfalse%
\ as\ \isakeywordONE{have}\isamarkupfalse%
\ {\isachardoublequoteopen}solves{\isacharunderscore}{\kern0pt}ineq{\isacharunderscore}{\kern0pt}comm\ r\ {\isacharparenleft}{\kern0pt}sys{\isacharprime}{\kern0pt}\ {\isacharbang}{\kern0pt}\ r{\isacharparenright}{\kern0pt}\ v{\isadigit{2}}{\isachardoublequoteclose}\isanewline
\ \ \ \ \isakeywordONE{unfolding}\isamarkupfalse%
\ solves{\isacharunderscore}{\kern0pt}ineq{\isacharunderscore}{\kern0pt}sys{\isacharunderscore}{\kern0pt}comm{\isacharunderscore}{\kern0pt}def\ solves{\isacharunderscore}{\kern0pt}ineq{\isacharunderscore}{\kern0pt}comm{\isacharunderscore}{\kern0pt}def\ \isakeywordONE{using}\isamarkupfalse%
\ r{\isacharunderscore}{\kern0pt}valid\ \isakeywordONE{by}\isamarkupfalse%
\ force\isanewline
\ \ \isakeywordONE{with}\isamarkupfalse%
\ as\ sol{\isacharunderscore}{\kern0pt}r{\isacharunderscore}{\kern0pt}is{\isacharunderscore}{\kern0pt}sol\ \isakeywordONE{have}\isamarkupfalse%
\ sol{\isacharunderscore}{\kern0pt}r{\isacharunderscore}{\kern0pt}min{\isacharcolon}{\kern0pt}\ {\isachardoublequoteopen}{\isasymPsi}\ {\isacharparenleft}{\kern0pt}eval\ sol{\isacharunderscore}{\kern0pt}r\ v{\isadigit{2}}{\isacharparenright}{\kern0pt}\ {\isasymsubseteq}\ {\isasymPsi}\ {\isacharparenleft}{\kern0pt}v{\isadigit{2}}\ r{\isacharparenright}{\kern0pt}{\isachardoublequoteclose}\isanewline
\ \ \ \ \isakeywordONE{unfolding}\isamarkupfalse%
\ partial{\isacharunderscore}{\kern0pt}min{\isacharunderscore}{\kern0pt}sol{\isacharunderscore}{\kern0pt}one{\isacharunderscore}{\kern0pt}ineq{\isacharunderscore}{\kern0pt}def\ \isakeywordONE{by}\isamarkupfalse%
\ blast\isanewline
\ \ \isakeywordONE{let}\isamarkupfalse%
\ {\isacharquery}{\kern0pt}v{\isacharprime}{\kern0pt}\ {\isacharequal}{\kern0pt}\ {\isachardoublequoteopen}v{\isadigit{2}}{\isacharparenleft}{\kern0pt}r\ {\isacharcolon}{\kern0pt}{\isacharequal}{\kern0pt}\ eval\ sol{\isacharunderscore}{\kern0pt}r\ v{\isadigit{2}}{\isacharparenright}{\kern0pt}{\isachardoublequoteclose}\isanewline
\ \ \isakeywordONE{from}\isamarkupfalse%
\ sol{\isacharunderscore}{\kern0pt}r{\isacharunderscore}{\kern0pt}min\ \isakeywordONE{have}\isamarkupfalse%
\ {\isachardoublequoteopen}{\isasymPsi}\ {\isacharparenleft}{\kern0pt}{\isacharquery}{\kern0pt}v{\isacharprime}{\kern0pt}\ i{\isacharparenright}{\kern0pt}\ {\isasymsubseteq}\ {\isasymPsi}\ {\isacharparenleft}{\kern0pt}v{\isadigit{2}}\ i{\isacharparenright}{\kern0pt}{\isachardoublequoteclose}\ \isakeywordTWO{for}\ i\ \isakeywordONE{by}\isamarkupfalse%
\ simp\isanewline
\ \ \isakeywordONE{with}\isamarkupfalse%
\ sols{\isacharunderscore}{\kern0pt}s{\isadigit{2}}\ \isakeywordTHREE{show}\isamarkupfalse%
\ {\isachardoublequoteopen}{\isasymPsi}\ {\isacharparenleft}{\kern0pt}eval\ {\isacharparenleft}{\kern0pt}sols{\isacharprime}{\kern0pt}\ i{\isacharparenright}{\kern0pt}\ v{\isadigit{2}}{\isacharparenright}{\kern0pt}\ {\isasymsubseteq}\ {\isasymPsi}\ {\isacharparenleft}{\kern0pt}v{\isadigit{2}}\ i{\isacharparenright}{\kern0pt}{\isachardoublequoteclose}\isanewline
\ \ \ \ \isakeywordONE{using}\isamarkupfalse%
\ substitution{\isacharunderscore}{\kern0pt}lemma{\isacharunderscore}{\kern0pt}upd{\isacharbrackleft}{\kern0pt}\isakeywordTWO{where}\ f{\isacharequal}{\kern0pt}{\isachardoublequoteopen}sols\ i{\isachardoublequoteclose}{\isacharbrackright}{\kern0pt}\ rlexp{\isacharunderscore}{\kern0pt}mono{\isacharunderscore}{\kern0pt}parikh{\isacharbrackleft}{\kern0pt}of\ {\isachardoublequoteopen}sols\ i{\isachardoublequoteclose}\ {\isacharquery}{\kern0pt}v{\isacharprime}{\kern0pt}\ v{\isadigit{2}}{\isacharbrackright}{\kern0pt}\ \isakeywordONE{by}\isamarkupfalse%
\ force\isanewline
\isakeywordONE{qed}\isamarkupfalse%
%
\endisatagproof
{\isafoldproof}%
%
\isadelimproof
\isanewline
%
\endisadelimproof
\isanewline
\isakeywordONE{lemma}\isamarkupfalse%
\ sols{\isacharprime}{\kern0pt}{\isacharunderscore}{\kern0pt}vars{\isacharunderscore}{\kern0pt}gt{\isacharunderscore}{\kern0pt}r{\isacharcolon}{\kern0pt}\ {\isachardoublequoteopen}{\isasymforall}i\ {\isasymge}\ Suc\ r{\isachardot}{\kern0pt}\ sols{\isacharprime}{\kern0pt}\ i\ {\isacharequal}{\kern0pt}\ Var\ i{\isachardoublequoteclose}\isanewline
%
\isadelimproof
\ \ %
\endisadelimproof
%
\isatagproof
\isakeywordONE{using}\isamarkupfalse%
\ sols{\isacharunderscore}{\kern0pt}is{\isacharunderscore}{\kern0pt}sol\ \isakeywordONE{unfolding}\isamarkupfalse%
\ partial{\isacharunderscore}{\kern0pt}min{\isacharunderscore}{\kern0pt}sol{\isacharunderscore}{\kern0pt}ineq{\isacharunderscore}{\kern0pt}sys{\isacharunderscore}{\kern0pt}def\ \isakeywordONE{by}\isamarkupfalse%
\ auto%
\endisatagproof
{\isafoldproof}%
%
\isadelimproof
\isanewline
%
\endisadelimproof
\isanewline
\isakeywordONE{lemma}\isamarkupfalse%
\ sols{\isacharprime}{\kern0pt}{\isacharunderscore}{\kern0pt}vars{\isacharunderscore}{\kern0pt}leq{\isacharunderscore}{\kern0pt}r{\isacharcolon}{\kern0pt}\ {\isachardoublequoteopen}{\isasymforall}i\ {\isacharless}{\kern0pt}\ Suc\ r{\isachardot}{\kern0pt}\ {\isasymforall}x\ {\isasymin}\ vars\ {\isacharparenleft}{\kern0pt}sols{\isacharprime}{\kern0pt}\ i{\isacharparenright}{\kern0pt}{\isachardot}{\kern0pt}\ x\ {\isasymge}\ Suc\ r\ {\isasymand}\ x\ {\isacharless}{\kern0pt}\ length\ sys{\isachardoublequoteclose}\isanewline
%
\isadelimproof
%
\endisadelimproof
%
\isatagproof
\isakeywordONE{proof}\isamarkupfalse%
\ {\isacharminus}{\kern0pt}\isanewline
\ \ \isakeywordONE{from}\isamarkupfalse%
\ sols{\isacharunderscore}{\kern0pt}is{\isacharunderscore}{\kern0pt}sol\ \isakeywordONE{have}\isamarkupfalse%
\ {\isachardoublequoteopen}{\isasymforall}i\ {\isacharless}{\kern0pt}\ r{\isachardot}{\kern0pt}\ {\isasymforall}x\ {\isasymin}\ vars\ {\isacharparenleft}{\kern0pt}sols\ i{\isacharparenright}{\kern0pt}{\isachardot}{\kern0pt}\ x\ {\isasymge}\ r\ {\isasymand}\ x\ {\isacharless}{\kern0pt}\ length\ sys{\isachardoublequoteclose}\isanewline
\ \ \ \ \isakeywordONE{unfolding}\isamarkupfalse%
\ partial{\isacharunderscore}{\kern0pt}min{\isacharunderscore}{\kern0pt}sol{\isacharunderscore}{\kern0pt}ineq{\isacharunderscore}{\kern0pt}sys{\isacharunderscore}{\kern0pt}def\ \isakeywordONE{by}\isamarkupfalse%
\ simp\isanewline
\ \ \isakeywordONE{with}\isamarkupfalse%
\ sols{\isacharunderscore}{\kern0pt}is{\isacharunderscore}{\kern0pt}sol\ \isakeywordONE{have}\isamarkupfalse%
\ vars{\isacharunderscore}{\kern0pt}sols{\isacharcolon}{\kern0pt}\ {\isachardoublequoteopen}{\isasymforall}i\ {\isacharless}{\kern0pt}\ length\ sys{\isachardot}{\kern0pt}\ {\isasymforall}x\ {\isasymin}\ vars\ {\isacharparenleft}{\kern0pt}sols\ i{\isacharparenright}{\kern0pt}{\isachardot}{\kern0pt}\ x\ {\isasymge}\ r\ {\isasymand}\ x\ {\isacharless}{\kern0pt}\ length\ sys{\isachardoublequoteclose}\isanewline
\ \ \ \ \isakeywordONE{unfolding}\isamarkupfalse%
\ partial{\isacharunderscore}{\kern0pt}min{\isacharunderscore}{\kern0pt}sol{\isacharunderscore}{\kern0pt}ineq{\isacharunderscore}{\kern0pt}sys{\isacharunderscore}{\kern0pt}def\ \isakeywordONE{by}\isamarkupfalse%
\ {\isacharparenleft}{\kern0pt}metis\ empty{\isacharunderscore}{\kern0pt}iff\ insert{\isacharunderscore}{\kern0pt}iff\ leI\ vars{\isachardot}{\kern0pt}simps{\isacharparenleft}{\kern0pt}{\isadigit{1}}{\isacharparenright}{\kern0pt}{\isacharparenright}{\kern0pt}\isanewline
\ \ \isakeywordONE{with}\isamarkupfalse%
\ sys{\isacharunderscore}{\kern0pt}valid\ \isakeywordONE{have}\isamarkupfalse%
\ {\isachardoublequoteopen}{\isasymforall}x\ {\isasymin}\ vars\ {\isacharparenleft}{\kern0pt}subst\ sols\ {\isacharparenleft}{\kern0pt}sys\ {\isacharbang}{\kern0pt}\ i{\isacharparenright}{\kern0pt}{\isacharparenright}{\kern0pt}{\isachardot}{\kern0pt}\ x\ {\isasymge}\ r\ {\isasymand}\ x\ {\isacharless}{\kern0pt}\ length\ sys{\isachardoublequoteclose}\ \isakeywordTWO{if}\ {\isachardoublequoteopen}i\ {\isacharless}{\kern0pt}\ length\ sys{\isachardoublequoteclose}\ \isakeywordTWO{for}\ i\isanewline
\ \ \ \ \isakeywordONE{using}\isamarkupfalse%
\ vars{\isacharunderscore}{\kern0pt}subst{\isacharbrackleft}{\kern0pt}of\ sols\ {\isachardoublequoteopen}sys\ {\isacharbang}{\kern0pt}\ i{\isachardoublequoteclose}{\isacharbrackright}{\kern0pt}\ that\ \isakeywordONE{by}\isamarkupfalse%
\ {\isacharparenleft}{\kern0pt}metis\ UN{\isacharunderscore}{\kern0pt}E\ nth{\isacharunderscore}{\kern0pt}mem{\isacharparenright}{\kern0pt}\isanewline
\ \ \isakeywordONE{then}\isamarkupfalse%
\ \isakeywordONE{have}\isamarkupfalse%
\ {\isachardoublequoteopen}{\isasymforall}x\ {\isasymin}\ vars\ {\isacharparenleft}{\kern0pt}sys{\isacharprime}{\kern0pt}\ {\isacharbang}{\kern0pt}\ i{\isacharparenright}{\kern0pt}{\isachardot}{\kern0pt}\ x\ {\isasymge}\ r\ {\isasymand}\ x\ {\isacharless}{\kern0pt}\ length\ sys{\isachardoublequoteclose}\ \isakeywordTWO{if}\ {\isachardoublequoteopen}i\ {\isacharless}{\kern0pt}\ length\ sys{\isachardoublequoteclose}\ \isakeywordTWO{for}\ i\isanewline
\ \ \ \ \isakeywordONE{unfolding}\isamarkupfalse%
\ subst{\isacharunderscore}{\kern0pt}sys{\isacharunderscore}{\kern0pt}def\ \isakeywordONE{using}\isamarkupfalse%
\ r{\isacharunderscore}{\kern0pt}valid\ that\ \isakeywordONE{by}\isamarkupfalse%
\ auto\isanewline
\ \ \isakeywordONE{moreover}\isamarkupfalse%
\ \isakeywordONE{from}\isamarkupfalse%
\ sol{\isacharunderscore}{\kern0pt}r{\isacharunderscore}{\kern0pt}is{\isacharunderscore}{\kern0pt}sol\ \isakeywordONE{have}\isamarkupfalse%
\ {\isachardoublequoteopen}vars\ {\isacharparenleft}{\kern0pt}sol{\isacharunderscore}{\kern0pt}r{\isacharparenright}{\kern0pt}\ {\isasymsubseteq}\ vars\ {\isacharparenleft}{\kern0pt}sys{\isacharprime}{\kern0pt}\ {\isacharbang}{\kern0pt}\ r{\isacharparenright}{\kern0pt}\ {\isacharminus}{\kern0pt}\ {\isacharbraceleft}{\kern0pt}r{\isacharbraceright}{\kern0pt}{\isachardoublequoteclose}\isanewline
\ \ \ \ \isakeywordONE{unfolding}\isamarkupfalse%
\ partial{\isacharunderscore}{\kern0pt}min{\isacharunderscore}{\kern0pt}sol{\isacharunderscore}{\kern0pt}one{\isacharunderscore}{\kern0pt}ineq{\isacharunderscore}{\kern0pt}def\ \isakeywordONE{by}\isamarkupfalse%
\ simp\isanewline
\ \ \isakeywordONE{ultimately}\isamarkupfalse%
\ \isakeywordONE{have}\isamarkupfalse%
\ vars{\isacharunderscore}{\kern0pt}sol{\isacharunderscore}{\kern0pt}r{\isacharcolon}{\kern0pt}\ {\isachardoublequoteopen}{\isasymforall}x\ {\isasymin}\ vars\ sol{\isacharunderscore}{\kern0pt}r{\isachardot}{\kern0pt}\ x\ {\isachargreater}{\kern0pt}\ r\ {\isasymand}\ x\ {\isacharless}{\kern0pt}\ length\ sys{\isachardoublequoteclose}\isanewline
\ \ \ \ \isakeywordONE{unfolding}\isamarkupfalse%
\ partial{\isacharunderscore}{\kern0pt}min{\isacharunderscore}{\kern0pt}sol{\isacharunderscore}{\kern0pt}one{\isacharunderscore}{\kern0pt}ineq{\isacharunderscore}{\kern0pt}def\ \isakeywordONE{using}\isamarkupfalse%
\ r{\isacharunderscore}{\kern0pt}valid\isanewline
\ \ \ \ \isakeywordONE{by}\isamarkupfalse%
\ {\isacharparenleft}{\kern0pt}metis\ DiffE\ insertCI\ nat{\isacharunderscore}{\kern0pt}less{\isacharunderscore}{\kern0pt}le\ subsetD{\isacharparenright}{\kern0pt}\isanewline
\ \ \isakeywordONE{moreover}\isamarkupfalse%
\ \isakeywordONE{have}\isamarkupfalse%
\ {\isachardoublequoteopen}vars\ {\isacharparenleft}{\kern0pt}sols{\isacharprime}{\kern0pt}\ i{\isacharparenright}{\kern0pt}\ {\isasymsubseteq}\ vars\ {\isacharparenleft}{\kern0pt}sols\ i{\isacharparenright}{\kern0pt}\ {\isacharminus}{\kern0pt}\ {\isacharbraceleft}{\kern0pt}r{\isacharbraceright}{\kern0pt}\ {\isasymunion}\ vars\ sol{\isacharunderscore}{\kern0pt}r{\isachardoublequoteclose}\ \isakeywordTWO{if}\ {\isachardoublequoteopen}i\ {\isacharless}{\kern0pt}\ length\ sys{\isachardoublequoteclose}\ \isakeywordTWO{for}\ i\isanewline
\ \ \ \ \isakeywordONE{using}\isamarkupfalse%
\ vars{\isacharunderscore}{\kern0pt}subst{\isacharunderscore}{\kern0pt}upd{\isacharunderscore}{\kern0pt}upper\ \isakeywordONE{by}\isamarkupfalse%
\ meson\isanewline
\ \ \isakeywordONE{ultimately}\isamarkupfalse%
\ \isakeywordONE{have}\isamarkupfalse%
\ {\isachardoublequoteopen}{\isasymforall}x\ {\isasymin}\ vars\ {\isacharparenleft}{\kern0pt}sols{\isacharprime}{\kern0pt}\ i{\isacharparenright}{\kern0pt}{\isachardot}{\kern0pt}\ x\ {\isachargreater}{\kern0pt}\ r\ {\isasymand}\ x\ {\isacharless}{\kern0pt}\ length\ sys{\isachardoublequoteclose}\ \isakeywordTWO{if}\ {\isachardoublequoteopen}i\ {\isacharless}{\kern0pt}\ length\ sys{\isachardoublequoteclose}\ \isakeywordTWO{for}\ i\isanewline
\ \ \ \ \isakeywordONE{using}\isamarkupfalse%
\ vars{\isacharunderscore}{\kern0pt}sols\ that\ \isakeywordONE{by}\isamarkupfalse%
\ fastforce\isanewline
\ \ \isakeywordONE{then}\isamarkupfalse%
\ \isakeywordTHREE{show}\isamarkupfalse%
\ {\isacharquery}{\kern0pt}thesis\ \isakeywordONE{by}\isamarkupfalse%
\ {\isacharparenleft}{\kern0pt}meson\ r{\isacharunderscore}{\kern0pt}valid\ Suc{\isacharunderscore}{\kern0pt}le{\isacharunderscore}{\kern0pt}eq\ dual{\isacharunderscore}{\kern0pt}order{\isachardot}{\kern0pt}strict{\isacharunderscore}{\kern0pt}trans{\isadigit{1}}{\isacharparenright}{\kern0pt}\isanewline
\isakeywordONE{qed}\isamarkupfalse%
%
\endisatagproof
{\isafoldproof}%
%
\isadelimproof
%
\endisadelimproof
%
\begin{isamarkuptext}%
In summary, \isa{\isaconst{sols{\isacharprime}{\kern0pt}}} is a minimal partial solution of the first \isa{Suc\ r} equations. This
allows us to prove the centerpiece of the induction step in the next lemma, namely that there exists
a \isa{\isaconst{reg{\isacharunderscore}{\kern0pt}eval}} and minimal partial solution for the first \isa{Suc\ r} equations:%
\end{isamarkuptext}\isamarkuptrue%
\isakeywordONE{lemma}\isamarkupfalse%
\ sols{\isacharprime}{\kern0pt}{\isacharunderscore}{\kern0pt}is{\isacharunderscore}{\kern0pt}min{\isacharunderscore}{\kern0pt}sol{\isacharcolon}{\kern0pt}\ {\isachardoublequoteopen}partial{\isacharunderscore}{\kern0pt}min{\isacharunderscore}{\kern0pt}sol{\isacharunderscore}{\kern0pt}ineq{\isacharunderscore}{\kern0pt}sys\ {\isacharparenleft}{\kern0pt}Suc\ r{\isacharparenright}{\kern0pt}\ sys\ sols{\isacharprime}{\kern0pt}{\isachardoublequoteclose}\isanewline
%
\isadelimproof
\ \ %
\endisadelimproof
%
\isatagproof
\isakeywordONE{unfolding}\isamarkupfalse%
\ partial{\isacharunderscore}{\kern0pt}min{\isacharunderscore}{\kern0pt}sol{\isacharunderscore}{\kern0pt}ineq{\isacharunderscore}{\kern0pt}sys{\isacharunderscore}{\kern0pt}def\isanewline
\ \ \isakeywordONE{using}\isamarkupfalse%
\ sols{\isacharprime}{\kern0pt}{\isacharunderscore}{\kern0pt}is{\isacharunderscore}{\kern0pt}sol\ sols{\isacharprime}{\kern0pt}{\isacharunderscore}{\kern0pt}min\ sols{\isacharprime}{\kern0pt}{\isacharunderscore}{\kern0pt}vars{\isacharunderscore}{\kern0pt}gt{\isacharunderscore}{\kern0pt}r\ sols{\isacharprime}{\kern0pt}{\isacharunderscore}{\kern0pt}vars{\isacharunderscore}{\kern0pt}leq{\isacharunderscore}{\kern0pt}r\isanewline
\ \ \isakeywordONE{by}\isamarkupfalse%
\ blast%
\endisatagproof
{\isafoldproof}%
%
\isadelimproof
\isanewline
%
\endisadelimproof
\isanewline
\isakeywordONE{lemma}\isamarkupfalse%
\ exists{\isacharunderscore}{\kern0pt}min{\isacharunderscore}{\kern0pt}sol{\isacharunderscore}{\kern0pt}Suc{\isacharunderscore}{\kern0pt}r{\isacharcolon}{\kern0pt}\isanewline
\ \ {\isachardoublequoteopen}{\isasymexists}sols{\isacharprime}{\kern0pt}{\isachardot}{\kern0pt}\ partial{\isacharunderscore}{\kern0pt}min{\isacharunderscore}{\kern0pt}sol{\isacharunderscore}{\kern0pt}ineq{\isacharunderscore}{\kern0pt}sys\ {\isacharparenleft}{\kern0pt}Suc\ r{\isacharparenright}{\kern0pt}\ sys\ sols{\isacharprime}{\kern0pt}\ {\isasymand}\ {\isacharparenleft}{\kern0pt}{\isasymforall}i{\isachardot}{\kern0pt}\ reg{\isacharunderscore}{\kern0pt}eval\ {\isacharparenleft}{\kern0pt}sols{\isacharprime}{\kern0pt}\ i{\isacharparenright}{\kern0pt}{\isacharparenright}{\kern0pt}{\isachardoublequoteclose}\isanewline
%
\isadelimproof
\ \ %
\endisadelimproof
%
\isatagproof
\isakeywordONE{using}\isamarkupfalse%
\ sols{\isacharprime}{\kern0pt}{\isacharunderscore}{\kern0pt}reg\ sols{\isacharprime}{\kern0pt}{\isacharunderscore}{\kern0pt}is{\isacharunderscore}{\kern0pt}min{\isacharunderscore}{\kern0pt}sol\ \isakeywordONE{by}\isamarkupfalse%
\ blast%
\endisatagproof
{\isafoldproof}%
%
\isadelimproof
\isanewline
%
\endisadelimproof
\isanewline
\isakeywordTWO{end}\isamarkupfalse%
%
\begin{isamarkuptext}%
Now follows the actual induction proof: For every \isa{r}, there exists a \isa{\isaconst{reg{\isacharunderscore}{\kern0pt}eval}} and minimal partial
solution of the first \isa{r} equations. This then implies that there also exists a regular and minimal (non-partial)
solution of the whole system:%
\end{isamarkuptext}\isamarkuptrue%
\isakeywordONE{lemma}\isamarkupfalse%
\ exists{\isacharunderscore}{\kern0pt}minimal{\isacharunderscore}{\kern0pt}reg{\isacharunderscore}{\kern0pt}sol{\isacharunderscore}{\kern0pt}sys{\isacharunderscore}{\kern0pt}aux{\isacharcolon}{\kern0pt}\isanewline
\ \ \isakeywordTWO{assumes}\ eqs{\isacharunderscore}{\kern0pt}reg{\isacharcolon}{\kern0pt}\ \ \ {\isachardoublequoteopen}{\isasymforall}eq\ {\isasymin}\ set\ sys{\isachardot}{\kern0pt}\ reg{\isacharunderscore}{\kern0pt}eval\ eq{\isachardoublequoteclose}\isanewline
\ \ \ \ \ \ \isakeywordTWO{and}\ sys{\isacharunderscore}{\kern0pt}valid{\isacharcolon}{\kern0pt}\ {\isachardoublequoteopen}{\isasymforall}eq\ {\isasymin}\ set\ sys{\isachardot}{\kern0pt}\ {\isasymforall}x\ {\isasymin}\ vars\ eq{\isachardot}{\kern0pt}\ x\ {\isacharless}{\kern0pt}\ length\ sys{\isachardoublequoteclose}\isanewline
\ \ \ \ \ \ \isakeywordTWO{and}\ r{\isacharunderscore}{\kern0pt}valid{\isacharcolon}{\kern0pt}\ \ \ {\isachardoublequoteopen}r\ {\isasymle}\ length\ sys{\isachardoublequoteclose}\ \ \ \isanewline
\ \ \ \ \isakeywordTWO{shows}\ \ \ \ \ \ \ \ \ \ \ \ {\isachardoublequoteopen}{\isasymexists}sols{\isachardot}{\kern0pt}\ partial{\isacharunderscore}{\kern0pt}min{\isacharunderscore}{\kern0pt}sol{\isacharunderscore}{\kern0pt}ineq{\isacharunderscore}{\kern0pt}sys\ r\ sys\ sols\ {\isasymand}\ {\isacharparenleft}{\kern0pt}{\isasymforall}i{\isachardot}{\kern0pt}\ reg{\isacharunderscore}{\kern0pt}eval\ {\isacharparenleft}{\kern0pt}sols\ i{\isacharparenright}{\kern0pt}{\isacharparenright}{\kern0pt}{\isachardoublequoteclose}\isanewline
%
\isadelimproof
%
\endisadelimproof
%
\isatagproof
\isakeywordONE{using}\isamarkupfalse%
\ r{\isacharunderscore}{\kern0pt}valid\ \isakeywordONE{proof}\isamarkupfalse%
\ {\isacharparenleft}{\kern0pt}induction\ r{\isacharparenright}{\kern0pt}\isanewline
\ \ \isakeywordTHREE{case}\isamarkupfalse%
\ {\isadigit{0}}\isanewline
\ \ \isakeywordONE{have}\isamarkupfalse%
\ {\isachardoublequoteopen}solution{\isacharunderscore}{\kern0pt}ineq{\isacharunderscore}{\kern0pt}sys\ {\isacharparenleft}{\kern0pt}take\ {\isadigit{0}}\ sys{\isacharparenright}{\kern0pt}\ Var{\isachardoublequoteclose}\isanewline
\ \ \ \ \isakeywordONE{unfolding}\isamarkupfalse%
\ solution{\isacharunderscore}{\kern0pt}ineq{\isacharunderscore}{\kern0pt}sys{\isacharunderscore}{\kern0pt}def\ solves{\isacharunderscore}{\kern0pt}ineq{\isacharunderscore}{\kern0pt}sys{\isacharunderscore}{\kern0pt}comm{\isacharunderscore}{\kern0pt}def\ \isakeywordONE{by}\isamarkupfalse%
\ simp\isanewline
\ \ \isakeywordONE{then}\isamarkupfalse%
\ \isakeywordTHREE{show}\isamarkupfalse%
\ {\isacharquery}{\kern0pt}case\ \isakeywordONE{unfolding}\isamarkupfalse%
\ partial{\isacharunderscore}{\kern0pt}min{\isacharunderscore}{\kern0pt}sol{\isacharunderscore}{\kern0pt}ineq{\isacharunderscore}{\kern0pt}sys{\isacharunderscore}{\kern0pt}def\ \isakeywordONE{by}\isamarkupfalse%
\ auto\isanewline
\isakeywordONE{next}\isamarkupfalse%
\isanewline
\ \ \isakeywordTHREE{case}\isamarkupfalse%
\ {\isacharparenleft}{\kern0pt}Suc\ r{\isacharparenright}{\kern0pt}\isanewline
\ \ \isakeywordONE{then}\isamarkupfalse%
\ \isakeywordTHREE{obtain}\isamarkupfalse%
\ sols\ \isakeywordTWO{where}\ sols{\isacharunderscore}{\kern0pt}intro{\isacharcolon}{\kern0pt}\ {\isachardoublequoteopen}partial{\isacharunderscore}{\kern0pt}min{\isacharunderscore}{\kern0pt}sol{\isacharunderscore}{\kern0pt}ineq{\isacharunderscore}{\kern0pt}sys\ r\ sys\ sols\ {\isasymand}\ {\isacharparenleft}{\kern0pt}{\isasymforall}i{\isachardot}{\kern0pt}\ reg{\isacharunderscore}{\kern0pt}eval\ {\isacharparenleft}{\kern0pt}sols\ i{\isacharparenright}{\kern0pt}{\isacharparenright}{\kern0pt}{\isachardoublequoteclose}\isanewline
\ \ \ \ \isakeywordONE{by}\isamarkupfalse%
\ auto\isanewline
\ \ \isakeywordONE{let}\isamarkupfalse%
\ {\isacharquery}{\kern0pt}sys{\isacharprime}{\kern0pt}\ {\isacharequal}{\kern0pt}\ {\isachardoublequoteopen}subst{\isacharunderscore}{\kern0pt}sys\ sols\ sys{\isachardoublequoteclose}\isanewline
\ \ \isakeywordONE{from}\isamarkupfalse%
\ eqs{\isacharunderscore}{\kern0pt}reg\ Suc{\isachardot}{\kern0pt}prems\ \isakeywordONE{have}\isamarkupfalse%
\ {\isachardoublequoteopen}reg{\isacharunderscore}{\kern0pt}eval\ {\isacharparenleft}{\kern0pt}sys\ {\isacharbang}{\kern0pt}\ r{\isacharparenright}{\kern0pt}{\isachardoublequoteclose}\ \isakeywordONE{by}\isamarkupfalse%
\ simp\isanewline
\ \ \isakeywordONE{with}\isamarkupfalse%
\ sols{\isacharunderscore}{\kern0pt}intro\ Suc{\isachardot}{\kern0pt}prems\ \isakeywordONE{have}\isamarkupfalse%
\ sys{\isacharunderscore}{\kern0pt}r{\isacharunderscore}{\kern0pt}reg{\isacharcolon}{\kern0pt}\ {\isachardoublequoteopen}reg{\isacharunderscore}{\kern0pt}eval\ {\isacharparenleft}{\kern0pt}{\isacharquery}{\kern0pt}sys{\isacharprime}{\kern0pt}\ {\isacharbang}{\kern0pt}\ r{\isacharparenright}{\kern0pt}{\isachardoublequoteclose}\isanewline
\ \ \ \ \isakeywordONE{using}\isamarkupfalse%
\ subst{\isacharunderscore}{\kern0pt}reg{\isacharunderscore}{\kern0pt}eval{\isacharbrackleft}{\kern0pt}of\ {\isachardoublequoteopen}sys\ {\isacharbang}{\kern0pt}\ r{\isachardoublequoteclose}{\isacharbrackright}{\kern0pt}\ subst{\isacharunderscore}{\kern0pt}sys{\isacharunderscore}{\kern0pt}subst{\isacharbrackleft}{\kern0pt}of\ r\ sys{\isacharbrackright}{\kern0pt}\ \isakeywordONE{by}\isamarkupfalse%
\ simp\isanewline
\ \ \isakeywordONE{then}\isamarkupfalse%
\ \isakeywordTHREE{obtain}\isamarkupfalse%
\ sol{\isacharunderscore}{\kern0pt}r\ \isakeywordTWO{where}\ sol{\isacharunderscore}{\kern0pt}r{\isacharunderscore}{\kern0pt}intro{\isacharcolon}{\kern0pt}\isanewline
\ \ \ \ {\isachardoublequoteopen}reg{\isacharunderscore}{\kern0pt}eval\ sol{\isacharunderscore}{\kern0pt}r\ {\isasymand}\ partial{\isacharunderscore}{\kern0pt}min{\isacharunderscore}{\kern0pt}sol{\isacharunderscore}{\kern0pt}one{\isacharunderscore}{\kern0pt}ineq\ r\ {\isacharparenleft}{\kern0pt}{\isacharquery}{\kern0pt}sys{\isacharprime}{\kern0pt}\ {\isacharbang}{\kern0pt}\ r{\isacharparenright}{\kern0pt}\ sol{\isacharunderscore}{\kern0pt}r{\isachardoublequoteclose}\isanewline
\ \ \ \ \isakeywordONE{using}\isamarkupfalse%
\ exists{\isacharunderscore}{\kern0pt}minimal{\isacharunderscore}{\kern0pt}reg{\isacharunderscore}{\kern0pt}sol\ \isakeywordONE{by}\isamarkupfalse%
\ blast\isanewline
\ \ \isakeywordONE{with}\isamarkupfalse%
\ Suc\ sols{\isacharunderscore}{\kern0pt}intro\ sys{\isacharunderscore}{\kern0pt}valid\ eqs{\isacharunderscore}{\kern0pt}reg\ \isakeywordONE{have}\isamarkupfalse%
\ {\isachardoublequoteopen}min{\isacharunderscore}{\kern0pt}sol{\isacharunderscore}{\kern0pt}induction{\isacharunderscore}{\kern0pt}step\ r\ sys\ sols\ sol{\isacharunderscore}{\kern0pt}r{\isachardoublequoteclose}\isanewline
\ \ \ \ \isakeywordONE{unfolding}\isamarkupfalse%
\ min{\isacharunderscore}{\kern0pt}sol{\isacharunderscore}{\kern0pt}induction{\isacharunderscore}{\kern0pt}step{\isacharunderscore}{\kern0pt}def\ \isakeywordONE{by}\isamarkupfalse%
\ force\isanewline
\ \ \isakeywordONE{from}\isamarkupfalse%
\ min{\isacharunderscore}{\kern0pt}sol{\isacharunderscore}{\kern0pt}induction{\isacharunderscore}{\kern0pt}step{\isachardot}{\kern0pt}exists{\isacharunderscore}{\kern0pt}min{\isacharunderscore}{\kern0pt}sol{\isacharunderscore}{\kern0pt}Suc{\isacharunderscore}{\kern0pt}r{\isacharbrackleft}{\kern0pt}OF\ this{\isacharbrackright}{\kern0pt}\ \isakeywordTHREE{show}\isamarkupfalse%
\ {\isacharquery}{\kern0pt}case\ \isakeywordONE{by}\isamarkupfalse%
\ blast\isanewline
\isakeywordONE{qed}\isamarkupfalse%
%
\endisatagproof
{\isafoldproof}%
%
\isadelimproof
\isanewline
%
\endisadelimproof
\isanewline
\isakeywordONE{lemma}\isamarkupfalse%
\ exists{\isacharunderscore}{\kern0pt}minimal{\isacharunderscore}{\kern0pt}reg{\isacharunderscore}{\kern0pt}sol{\isacharunderscore}{\kern0pt}sys{\isacharcolon}{\kern0pt}\isanewline
\ \ \isakeywordTWO{assumes}\ eqs{\isacharunderscore}{\kern0pt}reg{\isacharcolon}{\kern0pt}\ \ \ {\isachardoublequoteopen}{\isasymforall}eq\ {\isasymin}\ set\ sys{\isachardot}{\kern0pt}\ reg{\isacharunderscore}{\kern0pt}eval\ eq{\isachardoublequoteclose}\isanewline
\ \ \ \ \ \ \isakeywordTWO{and}\ sys{\isacharunderscore}{\kern0pt}valid{\isacharcolon}{\kern0pt}\ {\isachardoublequoteopen}{\isasymforall}eq\ {\isasymin}\ set\ sys{\isachardot}{\kern0pt}\ {\isasymforall}x\ {\isasymin}\ vars\ eq{\isachardot}{\kern0pt}\ x\ {\isacharless}{\kern0pt}\ length\ sys{\isachardoublequoteclose}\isanewline
\ \ \ \ \isakeywordTWO{shows}\ \ \ \ \ \ \ \ \ \ \ \ {\isachardoublequoteopen}{\isasymexists}sols{\isachardot}{\kern0pt}\ min{\isacharunderscore}{\kern0pt}sol{\isacharunderscore}{\kern0pt}ineq{\isacharunderscore}{\kern0pt}sys{\isacharunderscore}{\kern0pt}comm\ sys\ sols\ {\isasymand}\ {\isacharparenleft}{\kern0pt}{\isasymforall}i{\isachardot}{\kern0pt}\ regular{\isacharunderscore}{\kern0pt}lang\ {\isacharparenleft}{\kern0pt}sols\ i{\isacharparenright}{\kern0pt}{\isacharparenright}{\kern0pt}{\isachardoublequoteclose}\isanewline
%
\isadelimproof
%
\endisadelimproof
%
\isatagproof
\isakeywordONE{proof}\isamarkupfalse%
\ {\isacharminus}{\kern0pt}\isanewline
\ \ \isakeywordONE{from}\isamarkupfalse%
\ eqs{\isacharunderscore}{\kern0pt}reg\ sys{\isacharunderscore}{\kern0pt}valid\ \isakeywordONE{have}\isamarkupfalse%
\isanewline
\ \ \ \ {\isachardoublequoteopen}{\isasymexists}sols{\isachardot}{\kern0pt}\ partial{\isacharunderscore}{\kern0pt}min{\isacharunderscore}{\kern0pt}sol{\isacharunderscore}{\kern0pt}ineq{\isacharunderscore}{\kern0pt}sys\ {\isacharparenleft}{\kern0pt}length\ sys{\isacharparenright}{\kern0pt}\ sys\ sols\ {\isasymand}\ {\isacharparenleft}{\kern0pt}{\isasymforall}i{\isachardot}{\kern0pt}\ reg{\isacharunderscore}{\kern0pt}eval\ {\isacharparenleft}{\kern0pt}sols\ i{\isacharparenright}{\kern0pt}{\isacharparenright}{\kern0pt}{\isachardoublequoteclose}\isanewline
\ \ \ \ \isakeywordONE{using}\isamarkupfalse%
\ exists{\isacharunderscore}{\kern0pt}minimal{\isacharunderscore}{\kern0pt}reg{\isacharunderscore}{\kern0pt}sol{\isacharunderscore}{\kern0pt}sys{\isacharunderscore}{\kern0pt}aux\ \isakeywordONE{by}\isamarkupfalse%
\ blast\isanewline
\ \ \isakeywordONE{then}\isamarkupfalse%
\ \isakeywordTHREE{obtain}\isamarkupfalse%
\ sols\ \isakeywordTWO{where}\isanewline
\ \ \ \ sols{\isacharunderscore}{\kern0pt}intro{\isacharcolon}{\kern0pt}\ {\isachardoublequoteopen}partial{\isacharunderscore}{\kern0pt}min{\isacharunderscore}{\kern0pt}sol{\isacharunderscore}{\kern0pt}ineq{\isacharunderscore}{\kern0pt}sys\ {\isacharparenleft}{\kern0pt}length\ sys{\isacharparenright}{\kern0pt}\ sys\ sols\ {\isasymand}\ {\isacharparenleft}{\kern0pt}{\isasymforall}i{\isachardot}{\kern0pt}\ reg{\isacharunderscore}{\kern0pt}eval\ {\isacharparenleft}{\kern0pt}sols\ i{\isacharparenright}{\kern0pt}{\isacharparenright}{\kern0pt}{\isachardoublequoteclose}\isanewline
\ \ \ \ \isakeywordONE{by}\isamarkupfalse%
\ blast\isanewline
\ \ \isakeywordONE{then}\isamarkupfalse%
\ \isakeywordONE{have}\isamarkupfalse%
\ {\isachardoublequoteopen}const{\isacharunderscore}{\kern0pt}rlexp\ {\isacharparenleft}{\kern0pt}sols\ i{\isacharparenright}{\kern0pt}{\isachardoublequoteclose}\ \isakeywordTWO{if}\ {\isachardoublequoteopen}i\ {\isacharless}{\kern0pt}\ length\ sys{\isachardoublequoteclose}\ \isakeywordTWO{for}\ i\isanewline
\ \ \ \ \isakeywordONE{using}\isamarkupfalse%
\ that\ \isakeywordONE{unfolding}\isamarkupfalse%
\ partial{\isacharunderscore}{\kern0pt}min{\isacharunderscore}{\kern0pt}sol{\isacharunderscore}{\kern0pt}ineq{\isacharunderscore}{\kern0pt}sys{\isacharunderscore}{\kern0pt}def\ \isakeywordONE{by}\isamarkupfalse%
\ {\isacharparenleft}{\kern0pt}meson\ equals{\isadigit{0}}I\ leD{\isacharparenright}{\kern0pt}\isanewline
\ \ \isakeywordONE{with}\isamarkupfalse%
\ sols{\isacharunderscore}{\kern0pt}intro\ \isakeywordONE{have}\isamarkupfalse%
\ {\isachardoublequoteopen}{\isasymexists}l{\isachardot}{\kern0pt}\ regular{\isacharunderscore}{\kern0pt}lang\ l\ {\isasymand}\ {\isacharparenleft}{\kern0pt}{\isasymforall}v{\isachardot}{\kern0pt}\ eval\ {\isacharparenleft}{\kern0pt}sols\ i{\isacharparenright}{\kern0pt}\ v\ {\isacharequal}{\kern0pt}\ l{\isacharparenright}{\kern0pt}{\isachardoublequoteclose}\ \isakeywordTWO{if}\ {\isachardoublequoteopen}i\ {\isacharless}{\kern0pt}\ length\ sys{\isachardoublequoteclose}\ \isakeywordTWO{for}\ i\isanewline
\ \ \ \ \isakeywordONE{using}\isamarkupfalse%
\ that\ const{\isacharunderscore}{\kern0pt}rlexp{\isacharunderscore}{\kern0pt}regular{\isacharunderscore}{\kern0pt}lang\ \isakeywordONE{by}\isamarkupfalse%
\ metis\isanewline
\ \ \isakeywordONE{then}\isamarkupfalse%
\ \isakeywordTHREE{obtain}\isamarkupfalse%
\ ls\ \isakeywordTWO{where}\ ls{\isacharunderscore}{\kern0pt}intro{\isacharcolon}{\kern0pt}\ {\isachardoublequoteopen}{\isasymforall}i\ {\isacharless}{\kern0pt}\ length\ sys{\isachardot}{\kern0pt}\ regular{\isacharunderscore}{\kern0pt}lang\ {\isacharparenleft}{\kern0pt}ls\ i{\isacharparenright}{\kern0pt}\ {\isasymand}\ {\isacharparenleft}{\kern0pt}{\isasymforall}v{\isachardot}{\kern0pt}\ eval\ {\isacharparenleft}{\kern0pt}sols\ i{\isacharparenright}{\kern0pt}\ v\ {\isacharequal}{\kern0pt}\ ls\ i{\isacharparenright}{\kern0pt}{\isachardoublequoteclose}\isanewline
\ \ \ \ \isakeywordONE{by}\isamarkupfalse%
\ metis\isanewline
\ \ \isakeywordONE{let}\isamarkupfalse%
\ {\isacharquery}{\kern0pt}ls{\isacharprime}{\kern0pt}\ {\isacharequal}{\kern0pt}\ {\isachardoublequoteopen}{\isasymlambda}i{\isachardot}{\kern0pt}\ if\ i\ {\isacharless}{\kern0pt}\ length\ sys\ then\ ls\ i\ else\ {\isacharbraceleft}{\kern0pt}{\isacharbraceright}{\kern0pt}{\isachardoublequoteclose}\isanewline
\ \ \isakeywordONE{from}\isamarkupfalse%
\ ls{\isacharunderscore}{\kern0pt}intro\ \isakeywordONE{have}\isamarkupfalse%
\ ls{\isacharprime}{\kern0pt}{\isacharunderscore}{\kern0pt}intro{\isacharcolon}{\kern0pt}\isanewline
\ \ \ \ {\isachardoublequoteopen}{\isacharparenleft}{\kern0pt}{\isasymforall}i\ {\isacharless}{\kern0pt}\ length\ sys{\isachardot}{\kern0pt}\ regular{\isacharunderscore}{\kern0pt}lang\ {\isacharparenleft}{\kern0pt}{\isacharquery}{\kern0pt}ls{\isacharprime}{\kern0pt}\ i{\isacharparenright}{\kern0pt}\ {\isasymand}\ {\isacharparenleft}{\kern0pt}{\isasymforall}v{\isachardot}{\kern0pt}\ eval\ {\isacharparenleft}{\kern0pt}sols\ i{\isacharparenright}{\kern0pt}\ v\ {\isacharequal}{\kern0pt}\ {\isacharquery}{\kern0pt}ls{\isacharprime}{\kern0pt}\ i{\isacharparenright}{\kern0pt}{\isacharparenright}{\kern0pt}\isanewline
\ \ \ \ \ {\isasymand}\ {\isacharparenleft}{\kern0pt}{\isasymforall}i\ {\isasymge}\ length\ sys{\isachardot}{\kern0pt}\ {\isacharquery}{\kern0pt}ls{\isacharprime}{\kern0pt}\ i\ {\isacharequal}{\kern0pt}\ {\isacharbraceleft}{\kern0pt}{\isacharbraceright}{\kern0pt}{\isacharparenright}{\kern0pt}{\isachardoublequoteclose}\ \isakeywordONE{by}\isamarkupfalse%
\ force\isanewline
\ \ \isakeywordONE{then}\isamarkupfalse%
\ \isakeywordONE{have}\isamarkupfalse%
\ ls{\isacharprime}{\kern0pt}{\isacharunderscore}{\kern0pt}regular{\isacharcolon}{\kern0pt}\ {\isachardoublequoteopen}regular{\isacharunderscore}{\kern0pt}lang\ {\isacharparenleft}{\kern0pt}{\isacharquery}{\kern0pt}ls{\isacharprime}{\kern0pt}\ i{\isacharparenright}{\kern0pt}{\isachardoublequoteclose}\ \isakeywordTWO{for}\ i\ \isakeywordONE{by}\isamarkupfalse%
\ {\isacharparenleft}{\kern0pt}meson\ lang{\isachardot}{\kern0pt}simps{\isacharparenleft}{\kern0pt}{\isadigit{1}}{\isacharparenright}{\kern0pt}{\isacharparenright}{\kern0pt}\isanewline
\ \ \isakeywordONE{from}\isamarkupfalse%
\ ls{\isacharprime}{\kern0pt}{\isacharunderscore}{\kern0pt}intro\ sols{\isacharunderscore}{\kern0pt}intro\ \isakeywordONE{have}\isamarkupfalse%
\ {\isachardoublequoteopen}solves{\isacharunderscore}{\kern0pt}ineq{\isacharunderscore}{\kern0pt}sys{\isacharunderscore}{\kern0pt}comm\ sys\ {\isacharquery}{\kern0pt}ls{\isacharprime}{\kern0pt}{\isachardoublequoteclose}\isanewline
\ \ \ \ \isakeywordONE{unfolding}\isamarkupfalse%
\ partial{\isacharunderscore}{\kern0pt}min{\isacharunderscore}{\kern0pt}sol{\isacharunderscore}{\kern0pt}ineq{\isacharunderscore}{\kern0pt}sys{\isacharunderscore}{\kern0pt}def\ solution{\isacharunderscore}{\kern0pt}ineq{\isacharunderscore}{\kern0pt}sys{\isacharunderscore}{\kern0pt}def\isanewline
\ \ \ \ \isakeywordONE{by}\isamarkupfalse%
\ {\isacharparenleft}{\kern0pt}smt\ {\isacharparenleft}{\kern0pt}verit{\isacharparenright}{\kern0pt}\ eval{\isachardot}{\kern0pt}simps{\isacharparenleft}{\kern0pt}{\isadigit{1}}{\isacharparenright}{\kern0pt}\ linorder{\isacharunderscore}{\kern0pt}not{\isacharunderscore}{\kern0pt}less\ nless{\isacharunderscore}{\kern0pt}le\ take{\isacharunderscore}{\kern0pt}all{\isacharunderscore}{\kern0pt}iff{\isacharparenright}{\kern0pt}\isanewline
\ \ \isakeywordONE{moreover}\isamarkupfalse%
\ \isakeywordONE{have}\isamarkupfalse%
\ {\isachardoublequoteopen}{\isasymforall}sol{\isacharprime}{\kern0pt}{\isachardot}{\kern0pt}\ solves{\isacharunderscore}{\kern0pt}ineq{\isacharunderscore}{\kern0pt}sys{\isacharunderscore}{\kern0pt}comm\ sys\ sol{\isacharprime}{\kern0pt}\ {\isasymlongrightarrow}\ {\isacharparenleft}{\kern0pt}{\isasymforall}x{\isachardot}{\kern0pt}\ {\isasymPsi}\ {\isacharparenleft}{\kern0pt}{\isacharquery}{\kern0pt}ls{\isacharprime}{\kern0pt}\ x{\isacharparenright}{\kern0pt}\ {\isasymsubseteq}\ {\isasymPsi}\ {\isacharparenleft}{\kern0pt}sol{\isacharprime}{\kern0pt}\ x{\isacharparenright}{\kern0pt}{\isacharparenright}{\kern0pt}{\isachardoublequoteclose}\isanewline
\ \ \isakeywordONE{proof}\isamarkupfalse%
\ {\isacharparenleft}{\kern0pt}rule\ allI{\isacharcomma}{\kern0pt}\ rule\ impI{\isacharparenright}{\kern0pt}\isanewline
\ \ \ \ \isakeywordTHREE{fix}\isamarkupfalse%
\ sol{\isacharprime}{\kern0pt}\ x\isanewline
\ \ \ \ \isakeywordTHREE{assume}\isamarkupfalse%
\ as{\isacharcolon}{\kern0pt}\ {\isachardoublequoteopen}solves{\isacharunderscore}{\kern0pt}ineq{\isacharunderscore}{\kern0pt}sys{\isacharunderscore}{\kern0pt}comm\ sys\ sol{\isacharprime}{\kern0pt}{\isachardoublequoteclose}\isanewline
\ \ \ \ \isakeywordONE{let}\isamarkupfalse%
\ {\isacharquery}{\kern0pt}sol{\isacharunderscore}{\kern0pt}rlexps\ {\isacharequal}{\kern0pt}\ {\isachardoublequoteopen}{\isasymlambda}i{\isachardot}{\kern0pt}\ Const\ {\isacharparenleft}{\kern0pt}sol{\isacharprime}{\kern0pt}\ i{\isacharparenright}{\kern0pt}{\isachardoublequoteclose}\isanewline
\ \ \ \ \isakeywordONE{from}\isamarkupfalse%
\ as\ \isakeywordONE{have}\isamarkupfalse%
\ {\isachardoublequoteopen}solves{\isacharunderscore}{\kern0pt}ineq{\isacharunderscore}{\kern0pt}sys{\isacharunderscore}{\kern0pt}comm\ {\isacharparenleft}{\kern0pt}take\ {\isacharparenleft}{\kern0pt}length\ sys{\isacharparenright}{\kern0pt}\ sys{\isacharparenright}{\kern0pt}\ sol{\isacharprime}{\kern0pt}{\isachardoublequoteclose}\ \isakeywordONE{by}\isamarkupfalse%
\ simp\isanewline
\ \ \ \ \isakeywordONE{moreover}\isamarkupfalse%
\ \isakeywordONE{have}\isamarkupfalse%
\ {\isachardoublequoteopen}sol{\isacharprime}{\kern0pt}\ x\ {\isacharequal}{\kern0pt}\ eval\ {\isacharparenleft}{\kern0pt}{\isacharquery}{\kern0pt}sol{\isacharunderscore}{\kern0pt}rlexps\ x{\isacharparenright}{\kern0pt}\ sol{\isacharprime}{\kern0pt}{\isachardoublequoteclose}\ \isakeywordTWO{for}\ x\ \isakeywordONE{by}\isamarkupfalse%
\ simp\isanewline
\ \ \ \ \isakeywordONE{ultimately}\isamarkupfalse%
\ \isakeywordTHREE{show}\isamarkupfalse%
\ {\isachardoublequoteopen}{\isasymforall}x{\isachardot}{\kern0pt}\ {\isasymPsi}\ {\isacharparenleft}{\kern0pt}{\isacharquery}{\kern0pt}ls{\isacharprime}{\kern0pt}\ x{\isacharparenright}{\kern0pt}\ {\isasymsubseteq}\ {\isasymPsi}\ {\isacharparenleft}{\kern0pt}sol{\isacharprime}{\kern0pt}\ x{\isacharparenright}{\kern0pt}{\isachardoublequoteclose}\isanewline
\ \ \ \ \ \ \isakeywordONE{using}\isamarkupfalse%
\ sols{\isacharunderscore}{\kern0pt}intro\ \isakeywordONE{unfolding}\isamarkupfalse%
\ partial{\isacharunderscore}{\kern0pt}min{\isacharunderscore}{\kern0pt}sol{\isacharunderscore}{\kern0pt}ineq{\isacharunderscore}{\kern0pt}sys{\isacharunderscore}{\kern0pt}def\isanewline
\ \ \ \ \ \ \isakeywordONE{by}\isamarkupfalse%
\ {\isacharparenleft}{\kern0pt}smt\ {\isacharparenleft}{\kern0pt}verit{\isacharparenright}{\kern0pt}\ empty{\isacharunderscore}{\kern0pt}subsetI\ eval{\isachardot}{\kern0pt}simps{\isacharparenleft}{\kern0pt}{\isadigit{1}}{\isacharparenright}{\kern0pt}\ ls{\isacharprime}{\kern0pt}{\isacharunderscore}{\kern0pt}intro\ parikh{\isacharunderscore}{\kern0pt}img{\isacharunderscore}{\kern0pt}mono{\isacharparenright}{\kern0pt}\isanewline
\ \ \isakeywordONE{qed}\isamarkupfalse%
\isanewline
\ \ \isakeywordONE{ultimately}\isamarkupfalse%
\ \isakeywordONE{have}\isamarkupfalse%
\ {\isachardoublequoteopen}min{\isacharunderscore}{\kern0pt}sol{\isacharunderscore}{\kern0pt}ineq{\isacharunderscore}{\kern0pt}sys{\isacharunderscore}{\kern0pt}comm\ sys\ {\isacharquery}{\kern0pt}ls{\isacharprime}{\kern0pt}{\isachardoublequoteclose}\ \isakeywordONE{unfolding}\isamarkupfalse%
\ min{\isacharunderscore}{\kern0pt}sol{\isacharunderscore}{\kern0pt}ineq{\isacharunderscore}{\kern0pt}sys{\isacharunderscore}{\kern0pt}comm{\isacharunderscore}{\kern0pt}def\ \isakeywordONE{by}\isamarkupfalse%
\ blast\isanewline
\ \ \isakeywordONE{with}\isamarkupfalse%
\ ls{\isacharprime}{\kern0pt}{\isacharunderscore}{\kern0pt}regular\ \isakeywordTHREE{show}\isamarkupfalse%
\ {\isacharquery}{\kern0pt}thesis\ \isakeywordONE{by}\isamarkupfalse%
\ blast\isanewline
\isakeywordONE{qed}\isamarkupfalse%
%
\endisatagproof
{\isafoldproof}%
%
\isadelimproof
%
\endisadelimproof
%
\isadelimdocument
%
\endisadelimdocument
%
\isatagdocument
%
\isamarkupsubsection{Parikh's theorem%
}
\isamarkuptrue%
%
\endisatagdocument
{\isafolddocument}%
%
\isadelimdocument
%
\endisadelimdocument
%
\begin{isamarkuptext}%
Finally we are able to prove Parikh's theorem, i.e.\ that to each context free grammar exists
a regular language with identical Parikh image:%
\end{isamarkuptext}\isamarkuptrue%
\isakeywordONE{theorem}\isamarkupfalse%
\ Parikh{\isacharcolon}{\kern0pt}\isanewline
\ \ \isakeywordTWO{assumes}\ {\isachardoublequoteopen}CFL\ {\isacharparenleft}{\kern0pt}TYPE{\isacharparenleft}{\kern0pt}{\isacharprime}{\kern0pt}n{\isacharparenright}{\kern0pt}{\isacharparenright}{\kern0pt}\ L{\isachardoublequoteclose}\isanewline
\ \ \isakeywordTWO{shows}\ \ \ {\isachardoublequoteopen}{\isasymexists}L{\isacharprime}{\kern0pt}{\isachardot}{\kern0pt}\ regular{\isacharunderscore}{\kern0pt}lang\ L{\isacharprime}{\kern0pt}\ {\isasymand}\ {\isasymPsi}\ L\ {\isacharequal}{\kern0pt}\ {\isasymPsi}\ L{\isacharprime}{\kern0pt}{\isachardoublequoteclose}\isanewline
%
\isadelimproof
%
\endisadelimproof
%
\isatagproof
\isakeywordONE{proof}\isamarkupfalse%
\ {\isacharminus}{\kern0pt}\isanewline
\ \ \isakeywordONE{from}\isamarkupfalse%
\ assms\ \isakeywordTHREE{obtain}\isamarkupfalse%
\ P\ \isakeywordTWO{and}\ S{\isacharcolon}{\kern0pt}{\isacharcolon}{\kern0pt}{\isacharprime}{\kern0pt}n\ \isakeywordTWO{where}\ {\isacharasterisk}{\kern0pt}{\isacharcolon}{\kern0pt}\ {\isachardoublequoteopen}L\ {\isacharequal}{\kern0pt}\ Lang\ P\ S\ {\isasymand}\ finite\ P{\isachardoublequoteclose}\ \isakeywordONE{unfolding}\isamarkupfalse%
\ CFL{\isacharunderscore}{\kern0pt}def\ \isakeywordONE{by}\isamarkupfalse%
\ blast\isanewline
\ \ \isakeywordTHREE{show}\isamarkupfalse%
\ {\isacharquery}{\kern0pt}thesis\isanewline
\ \ \isakeywordONE{proof}\isamarkupfalse%
\ {\isacharparenleft}{\kern0pt}cases\ {\isachardoublequoteopen}S\ {\isasymin}\ Nts\ P{\isachardoublequoteclose}{\isacharparenright}{\kern0pt}\isanewline
\ \ \ \ \isakeywordTHREE{case}\isamarkupfalse%
\ True\isanewline
\ \ \ \ \isakeywordONE{from}\isamarkupfalse%
\ {\isacharasterisk}{\kern0pt}\ finite{\isacharunderscore}{\kern0pt}Nts\ exists{\isacharunderscore}{\kern0pt}bij{\isacharunderscore}{\kern0pt}Nt{\isacharunderscore}{\kern0pt}Var\ \isakeywordTHREE{obtain}\isamarkupfalse%
\ {\isasymgamma}\ {\isasymgamma}{\isacharprime}{\kern0pt}\ \isakeywordTWO{where}\ {\isacharasterisk}{\kern0pt}{\isacharasterisk}{\kern0pt}{\isacharcolon}{\kern0pt}\ {\isachardoublequoteopen}bij{\isacharunderscore}{\kern0pt}Nt{\isacharunderscore}{\kern0pt}Var\ {\isacharparenleft}{\kern0pt}Nts\ P{\isacharparenright}{\kern0pt}\ {\isasymgamma}\ {\isasymgamma}{\isacharprime}{\kern0pt}{\isachardoublequoteclose}\ \isakeywordONE{by}\isamarkupfalse%
\ metis\isanewline
\ \ \ \ \isakeywordONE{let}\isamarkupfalse%
\ {\isacharquery}{\kern0pt}sol\ {\isacharequal}{\kern0pt}\ {\isachardoublequoteopen}{\isasymlambda}i{\isachardot}{\kern0pt}\ if\ i\ {\isacharless}{\kern0pt}\ card\ {\isacharparenleft}{\kern0pt}Nts\ P{\isacharparenright}{\kern0pt}\ then\ Lang{\isacharunderscore}{\kern0pt}lfp\ P\ {\isacharparenleft}{\kern0pt}{\isasymgamma}\ i{\isacharparenright}{\kern0pt}\ else\ {\isacharbraceleft}{\kern0pt}{\isacharbraceright}{\kern0pt}{\isachardoublequoteclose}\isanewline
\ \ \ \ \isakeywordONE{from}\isamarkupfalse%
\ {\isacharasterisk}{\kern0pt}{\isacharasterisk}{\kern0pt}\ True\ \isakeywordONE{have}\isamarkupfalse%
\ {\isachardoublequoteopen}{\isasymgamma}{\isacharprime}{\kern0pt}\ S\ {\isacharless}{\kern0pt}\ card\ {\isacharparenleft}{\kern0pt}Nts\ P{\isacharparenright}{\kern0pt}{\isachardoublequoteclose}\ {\isachardoublequoteopen}{\isasymgamma}\ {\isacharparenleft}{\kern0pt}{\isasymgamma}{\isacharprime}{\kern0pt}\ S{\isacharparenright}{\kern0pt}\ {\isacharequal}{\kern0pt}\ S{\isachardoublequoteclose}\isanewline
\ \ \ \ \ \ \isakeywordONE{unfolding}\isamarkupfalse%
\ bij{\isacharunderscore}{\kern0pt}Nt{\isacharunderscore}{\kern0pt}Var{\isacharunderscore}{\kern0pt}def\ bij{\isacharunderscore}{\kern0pt}betw{\isacharunderscore}{\kern0pt}def\ \isakeywordONE{by}\isamarkupfalse%
\ auto\isanewline
\ \ \ \ \isakeywordONE{with}\isamarkupfalse%
\ Lang{\isacharunderscore}{\kern0pt}lfp{\isacharunderscore}{\kern0pt}eq{\isacharunderscore}{\kern0pt}Lang\ \isakeywordONE{have}\isamarkupfalse%
\ {\isacharasterisk}{\kern0pt}{\isacharasterisk}{\kern0pt}{\isacharasterisk}{\kern0pt}{\isacharcolon}{\kern0pt}\ {\isachardoublequoteopen}Lang\ P\ S\ {\isacharequal}{\kern0pt}\ {\isacharquery}{\kern0pt}sol\ {\isacharparenleft}{\kern0pt}{\isasymgamma}{\isacharprime}{\kern0pt}\ S{\isacharparenright}{\kern0pt}{\isachardoublequoteclose}\ \isakeywordONE{by}\isamarkupfalse%
\ metis\isanewline
\ \ \ \ \isakeywordONE{from}\isamarkupfalse%
\ {\isacharasterisk}{\kern0pt}\ {\isacharasterisk}{\kern0pt}{\isacharasterisk}{\kern0pt}\ CFG{\isacharunderscore}{\kern0pt}eq{\isacharunderscore}{\kern0pt}sys{\isachardot}{\kern0pt}CFL{\isacharunderscore}{\kern0pt}is{\isacharunderscore}{\kern0pt}min{\isacharunderscore}{\kern0pt}sol\ \isakeywordTHREE{obtain}\isamarkupfalse%
\ sys\isanewline
\ \ \ \ \ \ \isakeywordTWO{where}\ sys{\isacharunderscore}{\kern0pt}intro{\isacharcolon}{\kern0pt}\ {\isachardoublequoteopen}{\isacharparenleft}{\kern0pt}{\isasymforall}eq\ {\isasymin}\ set\ sys{\isachardot}{\kern0pt}\ reg{\isacharunderscore}{\kern0pt}eval\ eq{\isacharparenright}{\kern0pt}\ {\isasymand}\ {\isacharparenleft}{\kern0pt}{\isasymforall}eq\ {\isasymin}\ set\ sys{\isachardot}{\kern0pt}\ {\isasymforall}x\ {\isasymin}\ vars\ eq{\isachardot}{\kern0pt}\ x\ {\isacharless}{\kern0pt}\ length\ sys{\isacharparenright}{\kern0pt}\isanewline
\ \ \ \ \ \ \ \ \ \ \ \ \ \ \ \ \ \ \ \ \ \ \ \ {\isasymand}\ min{\isacharunderscore}{\kern0pt}sol{\isacharunderscore}{\kern0pt}ineq{\isacharunderscore}{\kern0pt}sys\ sys\ {\isacharquery}{\kern0pt}sol{\isachardoublequoteclose}\isanewline
\ \ \ \ \ \ \isakeywordONE{unfolding}\isamarkupfalse%
\ CFG{\isacharunderscore}{\kern0pt}eq{\isacharunderscore}{\kern0pt}sys{\isacharunderscore}{\kern0pt}def\ \isakeywordONE{by}\isamarkupfalse%
\ blast\isanewline
\ \ \ \ \isakeywordONE{with}\isamarkupfalse%
\ min{\isacharunderscore}{\kern0pt}sol{\isacharunderscore}{\kern0pt}min{\isacharunderscore}{\kern0pt}sol{\isacharunderscore}{\kern0pt}comm\ \isakeywordONE{have}\isamarkupfalse%
\ sol{\isacharunderscore}{\kern0pt}is{\isacharunderscore}{\kern0pt}min{\isacharunderscore}{\kern0pt}sol{\isacharcolon}{\kern0pt}\ {\isachardoublequoteopen}min{\isacharunderscore}{\kern0pt}sol{\isacharunderscore}{\kern0pt}ineq{\isacharunderscore}{\kern0pt}sys{\isacharunderscore}{\kern0pt}comm\ sys\ {\isacharquery}{\kern0pt}sol{\isachardoublequoteclose}\ \isakeywordONE{by}\isamarkupfalse%
\ fast\isanewline
\ \ \ \ \isakeywordONE{from}\isamarkupfalse%
\ sys{\isacharunderscore}{\kern0pt}intro\ exists{\isacharunderscore}{\kern0pt}minimal{\isacharunderscore}{\kern0pt}reg{\isacharunderscore}{\kern0pt}sol{\isacharunderscore}{\kern0pt}sys\ \isakeywordTHREE{obtain}\isamarkupfalse%
\ sol{\isacharprime}{\kern0pt}\ \isakeywordTWO{where}\isanewline
\ \ \ \ \ \ sol{\isacharprime}{\kern0pt}{\isacharunderscore}{\kern0pt}intro{\isacharcolon}{\kern0pt}\ {\isachardoublequoteopen}min{\isacharunderscore}{\kern0pt}sol{\isacharunderscore}{\kern0pt}ineq{\isacharunderscore}{\kern0pt}sys{\isacharunderscore}{\kern0pt}comm\ sys\ sol{\isacharprime}{\kern0pt}\ {\isasymand}\ regular{\isacharunderscore}{\kern0pt}lang\ {\isacharparenleft}{\kern0pt}sol{\isacharprime}{\kern0pt}\ {\isacharparenleft}{\kern0pt}{\isasymgamma}{\isacharprime}{\kern0pt}\ S{\isacharparenright}{\kern0pt}{\isacharparenright}{\kern0pt}{\isachardoublequoteclose}\ \isakeywordONE{by}\isamarkupfalse%
\ fastforce\isanewline
\ \ \ \ \isakeywordONE{with}\isamarkupfalse%
\ sol{\isacharunderscore}{\kern0pt}is{\isacharunderscore}{\kern0pt}min{\isacharunderscore}{\kern0pt}sol\ min{\isacharunderscore}{\kern0pt}sol{\isacharunderscore}{\kern0pt}comm{\isacharunderscore}{\kern0pt}unique\ \isakeywordONE{have}\isamarkupfalse%
\ {\isachardoublequoteopen}{\isasymPsi}\ {\isacharparenleft}{\kern0pt}{\isacharquery}{\kern0pt}sol\ {\isacharparenleft}{\kern0pt}{\isasymgamma}{\isacharprime}{\kern0pt}\ S{\isacharparenright}{\kern0pt}{\isacharparenright}{\kern0pt}\ {\isacharequal}{\kern0pt}\ {\isasymPsi}\ {\isacharparenleft}{\kern0pt}sol{\isacharprime}{\kern0pt}\ {\isacharparenleft}{\kern0pt}{\isasymgamma}{\isacharprime}{\kern0pt}\ S{\isacharparenright}{\kern0pt}{\isacharparenright}{\kern0pt}{\isachardoublequoteclose}\isanewline
\ \ \ \ \ \ \isakeywordONE{by}\isamarkupfalse%
\ blast\isanewline
\ \ \ \ \isakeywordONE{with}\isamarkupfalse%
\ {\isacharasterisk}{\kern0pt}\ {\isacharasterisk}{\kern0pt}{\isacharasterisk}{\kern0pt}{\isacharasterisk}{\kern0pt}\ sol{\isacharprime}{\kern0pt}{\isacharunderscore}{\kern0pt}intro\ \isakeywordTHREE{show}\isamarkupfalse%
\ {\isacharquery}{\kern0pt}thesis\ \isakeywordONE{by}\isamarkupfalse%
\ auto\isanewline
\ \ \isakeywordONE{next}\isamarkupfalse%
\isanewline
\ \ \ \ \isakeywordTHREE{case}\isamarkupfalse%
\ False\isanewline
\ \ \ \ \isakeywordONE{with}\isamarkupfalse%
\ Nts{\isacharunderscore}{\kern0pt}Lhss{\isacharunderscore}{\kern0pt}Rhs{\isacharunderscore}{\kern0pt}Nts\ \isakeywordONE{have}\isamarkupfalse%
\ {\isachardoublequoteopen}S\ {\isasymnotin}\ Lhss\ P{\isachardoublequoteclose}\ \isakeywordONE{by}\isamarkupfalse%
\ fast\isanewline
\ \ \ \ \isakeywordONE{from}\isamarkupfalse%
\ Lang{\isacharunderscore}{\kern0pt}empty{\isacharunderscore}{\kern0pt}if{\isacharunderscore}{\kern0pt}notin{\isacharunderscore}{\kern0pt}Lhss{\isacharbrackleft}{\kern0pt}OF\ this{\isacharbrackright}{\kern0pt}\ {\isacharasterisk}{\kern0pt}\ \isakeywordTHREE{show}\isamarkupfalse%
\ {\isacharquery}{\kern0pt}thesis\ \isakeywordONE{by}\isamarkupfalse%
\ {\isacharparenleft}{\kern0pt}metis\ lang{\isachardot}{\kern0pt}simps{\isacharparenleft}{\kern0pt}{\isadigit{1}}{\isacharparenright}{\kern0pt}{\isacharparenright}{\kern0pt}\isanewline
\ \ \isakeywordONE{qed}\isamarkupfalse%
\isanewline
\isakeywordONE{qed}\isamarkupfalse%
%
\endisatagproof
{\isafoldproof}%
%
\isadelimproof
\isanewline
%
\endisadelimproof
%
\isadelimtheory
\isanewline
%
\endisadelimtheory
%
\isatagtheory
\isakeywordTWO{end}\isamarkupfalse%
%
\endisatagtheory
{\isafoldtheory}%
%
\isadelimtheory
%
\endisadelimtheory
%
\end{isabellebody}%
\endinput
%:%file=~/studium/semester_7/semantik/homeworks/AIST/Parikh/Pilling.thy%:%
%:%11=1%:%
%:%27=3%:%
%:%28=3%:%
%:%29=4%:%
%:%30=5%:%
%:%31=6%:%
%:%40=9%:%
%:%41=10%:%
%:%42=11%:%
%:%43=12%:%
%:%44=13%:%
%:%45=14%:%
%:%46=15%:%
%:%47=16%:%
%:%48=17%:%
%:%49=18%:%
%:%50=19%:%
%:%59=22%:%
%:%71=24%:%
%:%72=25%:%
%:%73=26%:%
%:%74=27%:%
%:%75=28%:%
%:%77=29%:%
%:%78=29%:%
%:%79=30%:%
%:%82=34%:%
%:%83=35%:%
%:%84=36%:%
%:%85=37%:%
%:%86=38%:%
%:%88=39%:%
%:%89=39%:%
%:%92=40%:%
%:%96=40%:%
%:%97=40%:%
%:%98=40%:%
%:%103=40%:%
%:%106=41%:%
%:%107=42%:%
%:%108=43%:%
%:%109=43%:%
%:%110=44%:%
%:%117=45%:%
%:%118=45%:%
%:%119=46%:%
%:%120=46%:%
%:%121=47%:%
%:%122=47%:%
%:%123=48%:%
%:%124=48%:%
%:%125=48%:%
%:%126=49%:%
%:%127=49%:%
%:%128=49%:%
%:%129=49%:%
%:%130=50%:%
%:%131=50%:%
%:%132=50%:%
%:%133=50%:%
%:%134=50%:%
%:%135=51%:%
%:%136=51%:%
%:%137=51%:%
%:%138=51%:%
%:%139=51%:%
%:%140=52%:%
%:%141=52%:%
%:%142=52%:%
%:%143=52%:%
%:%144=53%:%
%:%145=53%:%
%:%146=54%:%
%:%147=54%:%
%:%148=55%:%
%:%149=55%:%
%:%150=56%:%
%:%151=56%:%
%:%152=56%:%
%:%153=57%:%
%:%154=57%:%
%:%155=57%:%
%:%156=57%:%
%:%157=58%:%
%:%158=58%:%
%:%159=58%:%
%:%160=58%:%
%:%161=58%:%
%:%162=59%:%
%:%163=59%:%
%:%164=59%:%
%:%165=59%:%
%:%166=60%:%
%:%167=60%:%
%:%168=60%:%
%:%169=60%:%
%:%170=61%:%
%:%176=61%:%
%:%179=62%:%
%:%180=63%:%
%:%181=63%:%
%:%182=64%:%
%:%183=65%:%
%:%184=66%:%
%:%191=67%:%
%:%192=67%:%
%:%193=68%:%
%:%194=68%:%
%:%195=69%:%
%:%196=69%:%
%:%197=70%:%
%:%198=70%:%
%:%199=70%:%
%:%200=70%:%
%:%201=71%:%
%:%202=71%:%
%:%203=71%:%
%:%204=71%:%
%:%205=72%:%
%:%206=72%:%
%:%207=72%:%
%:%208=72%:%
%:%209=73%:%
%:%210=73%:%
%:%211=73%:%
%:%212=73%:%
%:%213=74%:%
%:%219=74%:%
%:%222=75%:%
%:%223=76%:%
%:%224=76%:%
%:%225=77%:%
%:%226=78%:%
%:%227=79%:%
%:%228=80%:%
%:%229=81%:%
%:%230=82%:%
%:%237=83%:%
%:%238=83%:%
%:%239=84%:%
%:%240=84%:%
%:%241=84%:%
%:%242=85%:%
%:%243=86%:%
%:%244=87%:%
%:%245=88%:%
%:%246=88%:%
%:%247=89%:%
%:%248=89%:%
%:%249=89%:%
%:%250=90%:%
%:%251=91%:%
%:%252=92%:%
%:%253=92%:%
%:%254=92%:%
%:%255=93%:%
%:%256=93%:%
%:%257=94%:%
%:%258=94%:%
%:%259=94%:%
%:%260=94%:%
%:%261=94%:%
%:%262=95%:%
%:%263=95%:%
%:%264=95%:%
%:%265=96%:%
%:%266=96%:%
%:%267=97%:%
%:%268=97%:%
%:%269=98%:%
%:%270=98%:%
%:%271=98%:%
%:%272=99%:%
%:%273=99%:%
%:%274=99%:%
%:%275=100%:%
%:%276=100%:%
%:%277=100%:%
%:%278=100%:%
%:%279=101%:%
%:%285=101%:%
%:%288=102%:%
%:%289=103%:%
%:%290=103%:%
%:%291=104%:%
%:%292=105%:%
%:%293=106%:%
%:%294=107%:%
%:%295=108%:%
%:%296=109%:%
%:%303=110%:%
%:%304=110%:%
%:%305=111%:%
%:%306=111%:%
%:%307=111%:%
%:%308=112%:%
%:%309=113%:%
%:%310=114%:%
%:%311=115%:%
%:%312=115%:%
%:%313=116%:%
%:%314=116%:%
%:%315=116%:%
%:%316=117%:%
%:%317=118%:%
%:%318=119%:%
%:%319=119%:%
%:%320=119%:%
%:%321=120%:%
%:%322=120%:%
%:%323=121%:%
%:%324=121%:%
%:%325=122%:%
%:%326=122%:%
%:%327=123%:%
%:%328=123%:%
%:%329=124%:%
%:%330=124%:%
%:%331=125%:%
%:%332=125%:%
%:%333=126%:%
%:%334=126%:%
%:%335=126%:%
%:%336=127%:%
%:%337=127%:%
%:%338=128%:%
%:%339=128%:%
%:%340=129%:%
%:%341=129%:%
%:%342=130%:%
%:%343=130%:%
%:%344=130%:%
%:%345=131%:%
%:%346=131%:%
%:%347=132%:%
%:%348=132%:%
%:%349=132%:%
%:%350=133%:%
%:%351=134%:%
%:%352=134%:%
%:%353=134%:%
%:%354=135%:%
%:%355=135%:%
%:%356=135%:%
%:%357=136%:%
%:%358=137%:%
%:%359=137%:%
%:%360=137%:%
%:%361=138%:%
%:%362=138%:%
%:%363=138%:%
%:%364=139%:%
%:%365=140%:%
%:%366=140%:%
%:%367=141%:%
%:%368=141%:%
%:%369=141%:%
%:%370=142%:%
%:%371=142%:%
%:%372=143%:%
%:%373=143%:%
%:%374=143%:%
%:%375=143%:%
%:%376=144%:%
%:%377=144%:%
%:%378=145%:%
%:%379=145%:%
%:%380=145%:%
%:%381=145%:%
%:%382=145%:%
%:%383=145%:%
%:%384=146%:%
%:%385=146%:%
%:%386=146%:%
%:%387=147%:%
%:%388=147%:%
%:%389=147%:%
%:%390=148%:%
%:%391=148%:%
%:%392=148%:%
%:%393=148%:%
%:%394=149%:%
%:%400=149%:%
%:%403=150%:%
%:%404=151%:%
%:%405=151%:%
%:%406=152%:%
%:%407=153%:%
%:%408=154%:%
%:%409=155%:%
%:%416=156%:%
%:%417=156%:%
%:%418=157%:%
%:%419=157%:%
%:%420=157%:%
%:%421=158%:%
%:%422=158%:%
%:%423=159%:%
%:%424=159%:%
%:%425=159%:%
%:%426=160%:%
%:%427=160%:%
%:%428=160%:%
%:%429=161%:%
%:%430=161%:%
%:%431=162%:%
%:%432=162%:%
%:%433=163%:%
%:%434=163%:%
%:%435=164%:%
%:%436=164%:%
%:%437=165%:%
%:%438=165%:%
%:%439=166%:%
%:%440=166%:%
%:%441=167%:%
%:%442=167%:%
%:%443=168%:%
%:%444=168%:%
%:%445=169%:%
%:%446=169%:%
%:%447=169%:%
%:%448=170%:%
%:%449=170%:%
%:%450=170%:%
%:%451=171%:%
%:%452=171%:%
%:%453=171%:%
%:%454=172%:%
%:%455=173%:%
%:%456=173%:%
%:%457=174%:%
%:%458=174%:%
%:%459=174%:%
%:%460=175%:%
%:%461=175%:%
%:%462=176%:%
%:%463=176%:%
%:%464=176%:%
%:%465=177%:%
%:%466=178%:%
%:%467=178%:%
%:%468=178%:%
%:%469=179%:%
%:%470=179%:%
%:%471=179%:%
%:%472=180%:%
%:%473=180%:%
%:%474=181%:%
%:%475=181%:%
%:%476=181%:%
%:%477=181%:%
%:%478=182%:%
%:%479=182%:%
%:%480=182%:%
%:%481=182%:%
%:%482=183%:%
%:%483=183%:%
%:%484=184%:%
%:%485=184%:%
%:%486=184%:%
%:%487=184%:%
%:%488=184%:%
%:%489=184%:%
%:%490=185%:%
%:%491=185%:%
%:%492=185%:%
%:%493=185%:%
%:%494=185%:%
%:%495=186%:%
%:%496=186%:%
%:%497=186%:%
%:%498=186%:%
%:%499=187%:%
%:%505=187%:%
%:%508=188%:%
%:%509=189%:%
%:%510=189%:%
%:%512=191%:%
%:%519=192%:%
%:%520=192%:%
%:%521=193%:%
%:%522=193%:%
%:%523=194%:%
%:%524=194%:%
%:%525=194%:%
%:%526=194%:%
%:%527=195%:%
%:%528=195%:%
%:%529=196%:%
%:%530=196%:%
%:%531=197%:%
%:%532=197%:%
%:%533=197%:%
%:%534=197%:%
%:%535=198%:%
%:%536=198%:%
%:%537=198%:%
%:%538=198%:%
%:%539=199%:%
%:%540=199%:%
%:%541=200%:%
%:%542=200%:%
%:%543=201%:%
%:%544=201%:%
%:%545=201%:%
%:%546=202%:%
%:%547=203%:%
%:%548=203%:%
%:%549=204%:%
%:%550=204%:%
%:%551=204%:%
%:%552=204%:%
%:%553=205%:%
%:%554=205%:%
%:%555=206%:%
%:%556=206%:%
%:%557=207%:%
%:%558=207%:%
%:%559=207%:%
%:%560=208%:%
%:%561=209%:%
%:%562=209%:%
%:%563=210%:%
%:%564=210%:%
%:%565=210%:%
%:%566=210%:%
%:%567=211%:%
%:%568=211%:%
%:%569=212%:%
%:%570=212%:%
%:%571=213%:%
%:%572=213%:%
%:%573=213%:%
%:%574=214%:%
%:%575=214%:%
%:%576=215%:%
%:%577=215%:%
%:%578=215%:%
%:%579=215%:%
%:%580=216%:%
%:%595=219%:%
%:%607=221%:%
%:%608=222%:%
%:%609=223%:%
%:%610=224%:%
%:%612=226%:%
%:%613=226%:%
%:%614=227%:%
%:%615=228%:%
%:%616=229%:%
%:%617=230%:%
%:%618=231%:%
%:%619=232%:%
%:%620=233%:%
%:%622=235%:%
%:%623=236%:%
%:%624=237%:%
%:%625=238%:%
%:%627=239%:%
%:%628=239%:%
%:%629=240%:%
%:%630=240%:%
%:%631=241%:%
%:%632=242%:%
%:%633=242%:%
%:%640=243%:%
%:%641=243%:%
%:%642=244%:%
%:%643=244%:%
%:%644=244%:%
%:%645=245%:%
%:%646=245%:%
%:%647=245%:%
%:%648=246%:%
%:%649=246%:%
%:%650=246%:%
%:%651=246%:%
%:%652=247%:%
%:%658=247%:%
%:%661=248%:%
%:%662=249%:%
%:%663=249%:%
%:%670=250%:%
%:%671=250%:%
%:%672=251%:%
%:%673=251%:%
%:%674=252%:%
%:%675=252%:%
%:%676=253%:%
%:%677=253%:%
%:%678=253%:%
%:%679=253%:%
%:%680=254%:%
%:%681=254%:%
%:%682=254%:%
%:%683=254%:%
%:%684=255%:%
%:%685=255%:%
%:%686=255%:%
%:%687=255%:%
%:%688=255%:%
%:%689=256%:%
%:%690=256%:%
%:%691=256%:%
%:%692=256%:%
%:%693=257%:%
%:%699=257%:%
%:%702=258%:%
%:%703=259%:%
%:%704=259%:%
%:%711=260%:%
%:%712=260%:%
%:%713=260%:%
%:%714=261%:%
%:%715=261%:%
%:%716=262%:%
%:%717=262%:%
%:%718=263%:%
%:%719=263%:%
%:%720=264%:%
%:%721=264%:%
%:%722=265%:%
%:%723=265%:%
%:%724=266%:%
%:%725=266%:%
%:%726=267%:%
%:%727=267%:%
%:%728=268%:%
%:%729=268%:%
%:%730=268%:%
%:%731=269%:%
%:%732=269%:%
%:%733=269%:%
%:%734=269%:%
%:%735=269%:%
%:%736=270%:%
%:%737=270%:%
%:%738=270%:%
%:%739=271%:%
%:%740=271%:%
%:%741=272%:%
%:%742=272%:%
%:%743=272%:%
%:%744=273%:%
%:%745=273%:%
%:%746=273%:%
%:%747=274%:%
%:%748=274%:%
%:%749=274%:%
%:%750=275%:%
%:%751=276%:%
%:%752=276%:%
%:%753=277%:%
%:%754=277%:%
%:%755=277%:%
%:%756=278%:%
%:%757=279%:%
%:%758=279%:%
%:%759=279%:%
%:%760=280%:%
%:%761=280%:%
%:%762=280%:%
%:%763=281%:%
%:%764=281%:%
%:%765=282%:%
%:%766=282%:%
%:%767=283%:%
%:%768=283%:%
%:%769=283%:%
%:%770=283%:%
%:%771=283%:%
%:%772=284%:%
%:%773=284%:%
%:%774=284%:%
%:%775=284%:%
%:%776=284%:%
%:%777=285%:%
%:%778=285%:%
%:%779=285%:%
%:%780=286%:%
%:%781=286%:%
%:%782=286%:%
%:%783=287%:%
%:%784=287%:%
%:%785=287%:%
%:%786=287%:%
%:%787=287%:%
%:%788=288%:%
%:%794=288%:%
%:%797=289%:%
%:%798=290%:%
%:%799=290%:%
%:%800=291%:%
%:%801=292%:%
%:%802=293%:%
%:%809=294%:%
%:%810=294%:%
%:%811=295%:%
%:%812=295%:%
%:%813=295%:%
%:%814=296%:%
%:%815=296%:%
%:%816=296%:%
%:%817=297%:%
%:%818=297%:%
%:%819=297%:%
%:%820=298%:%
%:%821=298%:%
%:%822=298%:%
%:%823=299%:%
%:%824=299%:%
%:%825=299%:%
%:%826=299%:%
%:%827=300%:%
%:%828=300%:%
%:%829=300%:%
%:%830=301%:%
%:%831=301%:%
%:%832=301%:%
%:%833=302%:%
%:%834=302%:%
%:%835=302%:%
%:%836=303%:%
%:%837=303%:%
%:%838=303%:%
%:%839=304%:%
%:%840=304%:%
%:%841=304%:%
%:%842=305%:%
%:%843=305%:%
%:%844=305%:%
%:%845=306%:%
%:%846=306%:%
%:%847=306%:%
%:%848=306%:%
%:%849=307%:%
%:%859=309%:%
%:%861=310%:%
%:%862=310%:%
%:%863=311%:%
%:%866=312%:%
%:%870=312%:%
%:%871=312%:%
%:%872=313%:%
%:%873=313%:%
%:%874=314%:%
%:%875=314%:%
%:%880=314%:%
%:%883=315%:%
%:%884=316%:%
%:%887=319%:%
%:%888=320%:%
%:%889=321%:%
%:%891=322%:%
%:%892=322%:%
%:%893=323%:%
%:%894=324%:%
%:%901=325%:%
%:%902=325%:%
%:%903=326%:%
%:%904=326%:%
%:%905=326%:%
%:%906=327%:%
%:%907=328%:%
%:%908=328%:%
%:%909=329%:%
%:%910=329%:%
%:%911=329%:%
%:%912=330%:%
%:%913=331%:%
%:%914=331%:%
%:%915=331%:%
%:%916=332%:%
%:%917=332%:%
%:%918=333%:%
%:%919=333%:%
%:%920=333%:%
%:%921=334%:%
%:%922=334%:%
%:%923=334%:%
%:%924=334%:%
%:%925=335%:%
%:%926=335%:%
%:%927=335%:%
%:%928=336%:%
%:%929=336%:%
%:%930=336%:%
%:%931=337%:%
%:%932=337%:%
%:%933=337%:%
%:%934=337%:%
%:%935=338%:%
%:%950=341%:%
%:%962=343%:%
%:%963=344%:%
%:%964=345%:%
%:%965=346%:%
%:%966=347%:%
%:%967=348%:%
%:%968=349%:%
%:%969=350%:%
%:%971=352%:%
%:%972=352%:%
%:%973=353%:%
%:%974=354%:%
%:%975=355%:%
%:%976=356%:%
%:%977=357%:%
%:%978=358%:%
%:%979=359%:%
%:%980=360%:%
%:%981=361%:%
%:%982=362%:%
%:%983=363%:%
%:%984=364%:%
%:%986=366%:%
%:%987=367%:%
%:%988=368%:%
%:%989=369%:%
%:%990=370%:%
%:%992=371%:%
%:%993=371%:%
%:%994=372%:%
%:%995=372%:%
%:%996=373%:%
%:%997=374%:%
%:%998=374%:%
%:%1001=375%:%
%:%1005=375%:%
%:%1006=375%:%
%:%1007=375%:%
%:%1008=375%:%
%:%1017=378%:%
%:%1018=379%:%
%:%1020=380%:%
%:%1021=380%:%
%:%1024=381%:%
%:%1028=381%:%
%:%1029=381%:%
%:%1030=381%:%
%:%1031=381%:%
%:%1036=381%:%
%:%1039=382%:%
%:%1040=383%:%
%:%1041=383%:%
%:%1048=384%:%
%:%1049=384%:%
%:%1050=384%:%
%:%1051=385%:%
%:%1052=385%:%
%:%1053=386%:%
%:%1054=386%:%
%:%1055=387%:%
%:%1056=387%:%
%:%1057=387%:%
%:%1058=387%:%
%:%1059=388%:%
%:%1060=388%:%
%:%1061=388%:%
%:%1062=389%:%
%:%1063=389%:%
%:%1064=389%:%
%:%1065=390%:%
%:%1066=390%:%
%:%1067=390%:%
%:%1068=391%:%
%:%1069=391%:%
%:%1070=391%:%
%:%1071=392%:%
%:%1072=392%:%
%:%1073=393%:%
%:%1074=393%:%
%:%1075=394%:%
%:%1076=394%:%
%:%1077=395%:%
%:%1078=395%:%
%:%1079=395%:%
%:%1080=396%:%
%:%1081=396%:%
%:%1082=397%:%
%:%1083=397%:%
%:%1084=397%:%
%:%1085=398%:%
%:%1086=398%:%
%:%1087=398%:%
%:%1088=399%:%
%:%1089=399%:%
%:%1090=399%:%
%:%1091=400%:%
%:%1092=400%:%
%:%1093=400%:%
%:%1094=401%:%
%:%1100=401%:%
%:%1103=402%:%
%:%1104=403%:%
%:%1105=403%:%
%:%1107=405%:%
%:%1114=406%:%
%:%1115=406%:%
%:%1116=407%:%
%:%1117=407%:%
%:%1118=408%:%
%:%1119=408%:%
%:%1120=409%:%
%:%1121=409%:%
%:%1122=409%:%
%:%1123=409%:%
%:%1124=409%:%
%:%1125=410%:%
%:%1126=410%:%
%:%1127=410%:%
%:%1128=411%:%
%:%1129=411%:%
%:%1130=411%:%
%:%1131=412%:%
%:%1132=412%:%
%:%1133=413%:%
%:%1134=413%:%
%:%1135=413%:%
%:%1136=414%:%
%:%1137=414%:%
%:%1138=415%:%
%:%1139=415%:%
%:%1140=415%:%
%:%1141=416%:%
%:%1142=416%:%
%:%1143=416%:%
%:%1144=417%:%
%:%1145=417%:%
%:%1146=417%:%
%:%1147=418%:%
%:%1148=418%:%
%:%1149=418%:%
%:%1150=418%:%
%:%1151=419%:%
%:%1152=419%:%
%:%1153=419%:%
%:%1154=420%:%
%:%1155=420%:%
%:%1156=420%:%
%:%1157=421%:%
%:%1158=421%:%
%:%1159=422%:%
%:%1160=422%:%
%:%1161=422%:%
%:%1162=422%:%
%:%1163=423%:%
%:%1164=423%:%
%:%1165=423%:%
%:%1166=424%:%
%:%1167=424%:%
%:%1168=424%:%
%:%1169=425%:%
%:%1175=425%:%
%:%1178=426%:%
%:%1179=427%:%
%:%1180=427%:%
%:%1183=428%:%
%:%1187=428%:%
%:%1188=428%:%
%:%1189=428%:%
%:%1190=428%:%
%:%1195=428%:%
%:%1198=429%:%
%:%1199=430%:%
%:%1200=430%:%
%:%1207=431%:%
%:%1208=431%:%
%:%1209=432%:%
%:%1210=432%:%
%:%1211=432%:%
%:%1212=433%:%
%:%1213=433%:%
%:%1214=433%:%
%:%1215=434%:%
%:%1216=434%:%
%:%1217=434%:%
%:%1218=435%:%
%:%1219=435%:%
%:%1220=435%:%
%:%1221=436%:%
%:%1222=436%:%
%:%1223=436%:%
%:%1224=437%:%
%:%1225=437%:%
%:%1226=437%:%
%:%1227=438%:%
%:%1228=438%:%
%:%1229=438%:%
%:%1230=439%:%
%:%1231=439%:%
%:%1232=439%:%
%:%1233=439%:%
%:%1234=440%:%
%:%1235=440%:%
%:%1236=440%:%
%:%1237=440%:%
%:%1238=441%:%
%:%1239=441%:%
%:%1240=441%:%
%:%1241=442%:%
%:%1242=442%:%
%:%1243=442%:%
%:%1244=443%:%
%:%1245=443%:%
%:%1246=443%:%
%:%1247=444%:%
%:%1248=444%:%
%:%1249=445%:%
%:%1250=445%:%
%:%1251=445%:%
%:%1252=446%:%
%:%1253=446%:%
%:%1254=446%:%
%:%1255=447%:%
%:%1256=447%:%
%:%1257=447%:%
%:%1258=448%:%
%:%1259=448%:%
%:%1260=448%:%
%:%1261=449%:%
%:%1262=449%:%
%:%1263=449%:%
%:%1264=449%:%
%:%1265=450%:%
%:%1275=453%:%
%:%1276=454%:%
%:%1277=455%:%
%:%1279=456%:%
%:%1280=456%:%
%:%1283=457%:%
%:%1287=457%:%
%:%1288=457%:%
%:%1289=458%:%
%:%1290=458%:%
%:%1291=459%:%
%:%1292=459%:%
%:%1297=459%:%
%:%1300=460%:%
%:%1301=461%:%
%:%1302=461%:%
%:%1303=462%:%
%:%1306=463%:%
%:%1310=463%:%
%:%1311=463%:%
%:%1312=463%:%
%:%1317=463%:%
%:%1320=464%:%
%:%1321=465%:%
%:%1324=468%:%
%:%1325=469%:%
%:%1326=470%:%
%:%1328=471%:%
%:%1329=471%:%
%:%1330=472%:%
%:%1331=473%:%
%:%1332=474%:%
%:%1333=475%:%
%:%1340=476%:%
%:%1341=476%:%
%:%1342=476%:%
%:%1343=477%:%
%:%1344=477%:%
%:%1345=478%:%
%:%1346=478%:%
%:%1347=479%:%
%:%1348=479%:%
%:%1349=479%:%
%:%1350=480%:%
%:%1351=480%:%
%:%1352=480%:%
%:%1353=480%:%
%:%1354=480%:%
%:%1355=481%:%
%:%1356=481%:%
%:%1357=482%:%
%:%1358=482%:%
%:%1359=483%:%
%:%1360=483%:%
%:%1361=483%:%
%:%1362=484%:%
%:%1363=484%:%
%:%1364=485%:%
%:%1365=485%:%
%:%1366=486%:%
%:%1367=486%:%
%:%1368=486%:%
%:%1369=486%:%
%:%1370=487%:%
%:%1371=487%:%
%:%1372=487%:%
%:%1373=488%:%
%:%1374=488%:%
%:%1375=488%:%
%:%1376=489%:%
%:%1377=489%:%
%:%1378=489%:%
%:%1379=490%:%
%:%1380=491%:%
%:%1381=491%:%
%:%1382=491%:%
%:%1383=492%:%
%:%1384=492%:%
%:%1385=492%:%
%:%1386=493%:%
%:%1387=493%:%
%:%1388=493%:%
%:%1389=494%:%
%:%1390=494%:%
%:%1391=494%:%
%:%1392=494%:%
%:%1393=495%:%
%:%1399=495%:%
%:%1402=496%:%
%:%1403=497%:%
%:%1404=497%:%
%:%1405=498%:%
%:%1406=499%:%
%:%1407=500%:%
%:%1414=501%:%
%:%1415=501%:%
%:%1416=502%:%
%:%1417=502%:%
%:%1418=502%:%
%:%1419=503%:%
%:%1420=504%:%
%:%1421=504%:%
%:%1422=504%:%
%:%1423=505%:%
%:%1424=505%:%
%:%1425=505%:%
%:%1426=506%:%
%:%1427=507%:%
%:%1428=507%:%
%:%1429=508%:%
%:%1430=508%:%
%:%1431=508%:%
%:%1432=509%:%
%:%1433=509%:%
%:%1434=509%:%
%:%1435=509%:%
%:%1436=510%:%
%:%1437=510%:%
%:%1438=510%:%
%:%1439=511%:%
%:%1440=511%:%
%:%1441=511%:%
%:%1442=512%:%
%:%1443=512%:%
%:%1444=512%:%
%:%1445=513%:%
%:%1446=513%:%
%:%1447=514%:%
%:%1448=514%:%
%:%1449=515%:%
%:%1450=515%:%
%:%1451=515%:%
%:%1452=516%:%
%:%1453=517%:%
%:%1454=517%:%
%:%1455=518%:%
%:%1456=518%:%
%:%1457=518%:%
%:%1458=518%:%
%:%1459=519%:%
%:%1460=519%:%
%:%1461=519%:%
%:%1462=520%:%
%:%1463=520%:%
%:%1464=521%:%
%:%1465=521%:%
%:%1466=522%:%
%:%1467=522%:%
%:%1468=522%:%
%:%1469=523%:%
%:%1470=523%:%
%:%1471=524%:%
%:%1472=524%:%
%:%1473=525%:%
%:%1474=525%:%
%:%1475=526%:%
%:%1476=526%:%
%:%1477=527%:%
%:%1478=527%:%
%:%1479=527%:%
%:%1480=527%:%
%:%1481=528%:%
%:%1482=528%:%
%:%1483=528%:%
%:%1484=528%:%
%:%1485=529%:%
%:%1486=529%:%
%:%1487=529%:%
%:%1488=530%:%
%:%1489=530%:%
%:%1490=530%:%
%:%1491=531%:%
%:%1492=531%:%
%:%1493=532%:%
%:%1494=532%:%
%:%1495=533%:%
%:%1496=533%:%
%:%1497=533%:%
%:%1498=533%:%
%:%1499=533%:%
%:%1500=534%:%
%:%1501=534%:%
%:%1502=534%:%
%:%1503=534%:%
%:%1504=535%:%
%:%1519=539%:%
%:%1531=541%:%
%:%1532=542%:%
%:%1534=543%:%
%:%1535=543%:%
%:%1536=544%:%
%:%1537=545%:%
%:%1544=546%:%
%:%1545=546%:%
%:%1546=547%:%
%:%1547=547%:%
%:%1548=547%:%
%:%1549=547%:%
%:%1550=547%:%
%:%1551=548%:%
%:%1552=548%:%
%:%1553=549%:%
%:%1554=549%:%
%:%1555=550%:%
%:%1556=550%:%
%:%1557=551%:%
%:%1558=551%:%
%:%1559=551%:%
%:%1560=551%:%
%:%1561=552%:%
%:%1562=552%:%
%:%1563=553%:%
%:%1564=553%:%
%:%1565=553%:%
%:%1566=554%:%
%:%1567=554%:%
%:%1568=554%:%
%:%1569=555%:%
%:%1570=555%:%
%:%1571=555%:%
%:%1572=555%:%
%:%1573=556%:%
%:%1574=556%:%
%:%1575=556%:%
%:%1576=557%:%
%:%1577=558%:%
%:%1578=559%:%
%:%1579=559%:%
%:%1580=559%:%
%:%1581=560%:%
%:%1582=560%:%
%:%1583=560%:%
%:%1584=560%:%
%:%1585=561%:%
%:%1586=561%:%
%:%1587=561%:%
%:%1588=562%:%
%:%1589=562%:%
%:%1590=563%:%
%:%1591=563%:%
%:%1592=563%:%
%:%1593=564%:%
%:%1594=564%:%
%:%1595=565%:%
%:%1596=565%:%
%:%1597=565%:%
%:%1598=565%:%
%:%1599=566%:%
%:%1600=566%:%
%:%1601=567%:%
%:%1602=567%:%
%:%1603=568%:%
%:%1604=568%:%
%:%1605=568%:%
%:%1606=568%:%
%:%1607=569%:%
%:%1608=569%:%
%:%1609=569%:%
%:%1610=569%:%
%:%1611=570%:%
%:%1612=570%:%
%:%1613=571%:%
%:%1619=571%:%
%:%1624=572%:%
%:%1629=573%:%



\bibliographystyle{abbrv}
\bibliography{root}

\end{document}

%%% Local Variables:
%%% mode: latex
%%% TeX-master: t
%%% End:
