\documentclass[11pt,a4paper]{article}
\usepackage[T1]{fontenc}
\usepackage{isabelle,isabellesym}

% further packages required for unusual symbols (see also
% isabellesym.sty), use only when needed

\usepackage{amssymb}

%\usepackage[only,bigsqcap,bigparallel,fatsemi,interleave,sslash]{stmaryrd}
  %for \<Sqinter>, \<Parallel>, \<Zsemi>, \<Parallel>, \<sslash>

% this should be the last package used
\usepackage{pdfsetup}

% urls in roman style, theory text in math-similar italics
\urlstyle{rm}
\isabellestyle{literal}

\begin{document}

\title{Greibach Normal Form}
\author{Alexander Haberl and Tobias Nipkow and Akihisa Yamada}
\maketitle

\begin{abstract}
This theory formalizes a method to transform a set of productions into the Greibach Normal Form (GNF) \cite{Greibach}. For purposes of this theory, we only consider one part of the GNF definition, that every production starts with a terminal. This means that the tail of a right-hand side can contain other terminals. To express this, we have defined the formalisation of $GNF\_hd$, which only checks this property.

The main idea behind this method is to bring the productions into a $triangular$ form, where nonterminal $Ai$ does not depend on nonterminals $A_i, …, A_n$. Then every $A_0$ production must already start with a terminal, and we can bring all productions into GNF by expanding the head nonterminals in order, starting with A1 productions.

A drawback of this approach is that the resulting GNF grammar can have exponential size, with respect to the number of productions, which is caused by the successive expansion of the head nonterminals.
\end{abstract}

% sane default for proof documents
\parindent 0pt\parskip 0.5ex

% generated text of all theories
%
\begin{isabellebody}%
\setisabellecontext{Reg{\isacharunderscore}{\kern0pt}Lang{\isacharunderscore}{\kern0pt}Exp}%
%
\isadelimdocument
%
\endisadelimdocument
%
\isatagdocument
%
\isamarkupsection{Regular language expressions%
}
\isamarkuptrue%
%
\endisatagdocument
{\isafolddocument}%
%
\isadelimdocument
%
\endisadelimdocument
%
\isadelimtheory
%
\endisadelimtheory
%
\isatagtheory
\isakeywordONE{theory}\isamarkupfalse%
\ Reg{\isacharunderscore}{\kern0pt}Lang{\isacharunderscore}{\kern0pt}Exp\isanewline
\ \ \isakeywordTWO{imports}\ \isanewline
\ \ \ \ {\isachardoublequoteopen}Regular{\isacharminus}{\kern0pt}Sets{\isachardot}{\kern0pt}Regular{\isacharunderscore}{\kern0pt}Exp{\isachardoublequoteclose}\isanewline
\isakeywordTWO{begin}%
\endisatagtheory
{\isafoldtheory}%
%
\isadelimtheory
%
\endisadelimtheory
%
\isadelimdocument
%
\endisadelimdocument
%
\isatagdocument
%
\isamarkupsubsection{Definition%
}
\isamarkuptrue%
%
\endisatagdocument
{\isafolddocument}%
%
\isadelimdocument
%
\endisadelimdocument
%
\begin{isamarkuptext}%
We introduce regular language expressions which will be the building blocks of the systems of
equations defined later. Regular language expressions can contain both constant languages and
variable languages where variables are natural numbers for simplicity. Given a valuation, i.e.\ an
instantiation of each variable with a language, the regular language expression can be evaluated,
yielding a language.%
\end{isamarkuptext}\isamarkuptrue%
\isakeywordONE{datatype}\isamarkupfalse%
\ {\isacharprime}{\kern0pt}a\ rlexp\ {\isacharequal}{\kern0pt}\ Var\ nat\ \ \ \ \ \ \ \ \ \ \ \ \ \ \ \ \ \ \ \ \ \ \ \ \ \ \isanewline
\ \ \ \ \ \ \ \ \ \ \ \ \ \ \ \ \ \ {\isacharbar}{\kern0pt}\ Const\ {\isachardoublequoteopen}{\isacharprime}{\kern0pt}a\ lang{\isachardoublequoteclose}\ \ \ \ \ \ \ \ \ \ \ \ \ \ \ \ \ \ \isanewline
\ \ \ \ \ \ \ \ \ \ \ \ \ \ \ \ \ \ {\isacharbar}{\kern0pt}\ Union\ {\isachardoublequoteopen}{\isacharprime}{\kern0pt}a\ rlexp{\isachardoublequoteclose}\ {\isachardoublequoteopen}{\isacharprime}{\kern0pt}a\ rlexp{\isachardoublequoteclose}\isanewline
\ \ \ \ \ \ \ \ \ \ \ \ \ \ \ \ \ \ {\isacharbar}{\kern0pt}\ Concat\ {\isachardoublequoteopen}{\isacharprime}{\kern0pt}a\ rlexp{\isachardoublequoteclose}\ {\isachardoublequoteopen}{\isacharprime}{\kern0pt}a\ rlexp{\isachardoublequoteclose}\ \ \ \ \ \isanewline
\ \ \ \ \ \ \ \ \ \ \ \ \ \ \ \ \ \ {\isacharbar}{\kern0pt}\ Star\ {\isachardoublequoteopen}{\isacharprime}{\kern0pt}a\ rlexp{\isachardoublequoteclose}\ \ \ \ \ \ \ \ \ \ \ \ \ \ \ \ \ \ \isanewline
\isanewline
\isakeywordONE{type{\isacharunderscore}{\kern0pt}synonym}\isamarkupfalse%
\ {\isacharprime}{\kern0pt}a\ valuation\ {\isacharequal}{\kern0pt}\ {\isachardoublequoteopen}nat\ {\isasymRightarrow}\ {\isacharprime}{\kern0pt}a\ lang{\isachardoublequoteclose}\isanewline
\isanewline
\isakeywordONE{primrec}\isamarkupfalse%
\ eval\ {\isacharcolon}{\kern0pt}{\isacharcolon}{\kern0pt}\ {\isachardoublequoteopen}{\isacharprime}{\kern0pt}a\ rlexp\ {\isasymRightarrow}\ {\isacharprime}{\kern0pt}a\ valuation\ {\isasymRightarrow}\ {\isacharprime}{\kern0pt}a\ lang{\isachardoublequoteclose}\ \isakeywordTWO{where}\isanewline
\ \ {\isachardoublequoteopen}eval\ {\isacharparenleft}{\kern0pt}Var\ n{\isacharparenright}{\kern0pt}\ v\ {\isacharequal}{\kern0pt}\ v\ n{\isachardoublequoteclose}\ {\isacharbar}{\kern0pt}\isanewline
\ \ {\isachardoublequoteopen}eval\ {\isacharparenleft}{\kern0pt}Const\ l{\isacharparenright}{\kern0pt}\ {\isacharunderscore}{\kern0pt}\ {\isacharequal}{\kern0pt}\ l{\isachardoublequoteclose}\ {\isacharbar}{\kern0pt}\isanewline
\ \ {\isachardoublequoteopen}eval\ {\isacharparenleft}{\kern0pt}Union\ f\ g{\isacharparenright}{\kern0pt}\ v\ {\isacharequal}{\kern0pt}\ eval\ f\ v\ {\isasymunion}\ eval\ g\ v{\isachardoublequoteclose}\ {\isacharbar}{\kern0pt}\isanewline
\ \ {\isachardoublequoteopen}eval\ {\isacharparenleft}{\kern0pt}Concat\ f\ g{\isacharparenright}{\kern0pt}\ v\ {\isacharequal}{\kern0pt}\ eval\ f\ v\ {\isacharat}{\kern0pt}{\isacharat}{\kern0pt}\ eval\ g\ v{\isachardoublequoteclose}\ {\isacharbar}{\kern0pt}\isanewline
\ \ {\isachardoublequoteopen}eval\ {\isacharparenleft}{\kern0pt}Star\ f{\isacharparenright}{\kern0pt}\ v\ {\isacharequal}{\kern0pt}\ star\ {\isacharparenleft}{\kern0pt}eval\ f\ v{\isacharparenright}{\kern0pt}{\isachardoublequoteclose}\isanewline
\isanewline
\isakeywordONE{primrec}\isamarkupfalse%
\ vars\ {\isacharcolon}{\kern0pt}{\isacharcolon}{\kern0pt}\ {\isachardoublequoteopen}{\isacharprime}{\kern0pt}a\ rlexp\ {\isasymRightarrow}\ nat\ set{\isachardoublequoteclose}\ \isakeywordTWO{where}\isanewline
\ \ {\isachardoublequoteopen}vars\ {\isacharparenleft}{\kern0pt}Var\ n{\isacharparenright}{\kern0pt}\ {\isacharequal}{\kern0pt}\ {\isacharbraceleft}{\kern0pt}n{\isacharbraceright}{\kern0pt}{\isachardoublequoteclose}\ {\isacharbar}{\kern0pt}\isanewline
\ \ {\isachardoublequoteopen}vars\ {\isacharparenleft}{\kern0pt}Const\ {\isacharunderscore}{\kern0pt}{\isacharparenright}{\kern0pt}\ {\isacharequal}{\kern0pt}\ {\isacharbraceleft}{\kern0pt}{\isacharbraceright}{\kern0pt}{\isachardoublequoteclose}\ {\isacharbar}{\kern0pt}\isanewline
\ \ {\isachardoublequoteopen}vars\ {\isacharparenleft}{\kern0pt}Union\ f\ g{\isacharparenright}{\kern0pt}\ {\isacharequal}{\kern0pt}\ vars\ f\ {\isasymunion}\ vars\ g{\isachardoublequoteclose}\ {\isacharbar}{\kern0pt}\isanewline
\ \ {\isachardoublequoteopen}vars\ {\isacharparenleft}{\kern0pt}Concat\ f\ g{\isacharparenright}{\kern0pt}\ {\isacharequal}{\kern0pt}\ vars\ f\ {\isasymunion}\ vars\ g{\isachardoublequoteclose}\ {\isacharbar}{\kern0pt}\isanewline
\ \ {\isachardoublequoteopen}vars\ {\isacharparenleft}{\kern0pt}Star\ f{\isacharparenright}{\kern0pt}\ {\isacharequal}{\kern0pt}\ vars\ f{\isachardoublequoteclose}%
\begin{isamarkuptext}%
Given some regular language expression, substituting each occurrence of a variable \isa{i} by
the regular language expression \isa{s\ i} yields the following regular language expression:%
\end{isamarkuptext}\isamarkuptrue%
\isakeywordONE{primrec}\isamarkupfalse%
\ subst\ {\isacharcolon}{\kern0pt}{\isacharcolon}{\kern0pt}\ {\isachardoublequoteopen}{\isacharparenleft}{\kern0pt}nat\ {\isasymRightarrow}\ {\isacharprime}{\kern0pt}a\ rlexp{\isacharparenright}{\kern0pt}\ {\isasymRightarrow}\ {\isacharprime}{\kern0pt}a\ rlexp\ {\isasymRightarrow}\ {\isacharprime}{\kern0pt}a\ rlexp{\isachardoublequoteclose}\ \isakeywordTWO{where}\isanewline
\ \ {\isachardoublequoteopen}subst\ s\ {\isacharparenleft}{\kern0pt}Var\ n{\isacharparenright}{\kern0pt}\ {\isacharequal}{\kern0pt}\ s\ n{\isachardoublequoteclose}\ {\isacharbar}{\kern0pt}\isanewline
\ \ {\isachardoublequoteopen}subst\ {\isacharunderscore}{\kern0pt}\ {\isacharparenleft}{\kern0pt}Const\ l{\isacharparenright}{\kern0pt}\ {\isacharequal}{\kern0pt}\ Const\ l{\isachardoublequoteclose}\ {\isacharbar}{\kern0pt}\isanewline
\ \ {\isachardoublequoteopen}subst\ s\ {\isacharparenleft}{\kern0pt}Union\ f\ g{\isacharparenright}{\kern0pt}\ {\isacharequal}{\kern0pt}\ Union\ {\isacharparenleft}{\kern0pt}subst\ s\ f{\isacharparenright}{\kern0pt}\ {\isacharparenleft}{\kern0pt}subst\ s\ g{\isacharparenright}{\kern0pt}{\isachardoublequoteclose}\ {\isacharbar}{\kern0pt}\isanewline
\ \ {\isachardoublequoteopen}subst\ s\ {\isacharparenleft}{\kern0pt}Concat\ f\ g{\isacharparenright}{\kern0pt}\ {\isacharequal}{\kern0pt}\ Concat\ {\isacharparenleft}{\kern0pt}subst\ s\ f{\isacharparenright}{\kern0pt}\ {\isacharparenleft}{\kern0pt}subst\ s\ g{\isacharparenright}{\kern0pt}{\isachardoublequoteclose}\ {\isacharbar}{\kern0pt}\isanewline
\ \ {\isachardoublequoteopen}subst\ s\ {\isacharparenleft}{\kern0pt}Star\ f{\isacharparenright}{\kern0pt}\ {\isacharequal}{\kern0pt}\ Star\ {\isacharparenleft}{\kern0pt}subst\ s\ f{\isacharparenright}{\kern0pt}{\isachardoublequoteclose}%
\isadelimdocument
%
\endisadelimdocument
%
\isatagdocument
%
\isamarkupsubsection{Basic lemmas%
}
\isamarkuptrue%
%
\endisatagdocument
{\isafolddocument}%
%
\isadelimdocument
%
\endisadelimdocument
\isakeywordONE{lemma}\isamarkupfalse%
\ substitution{\isacharunderscore}{\kern0pt}lemma{\isacharcolon}{\kern0pt}\isanewline
\ \ \isakeywordTWO{assumes}\ {\isachardoublequoteopen}{\isasymforall}i{\isachardot}{\kern0pt}\ v{\isacharprime}{\kern0pt}\ i\ {\isacharequal}{\kern0pt}\ eval\ {\isacharparenleft}{\kern0pt}upd\ i{\isacharparenright}{\kern0pt}\ v{\isachardoublequoteclose}\isanewline
\ \ \isakeywordTWO{shows}\ {\isachardoublequoteopen}eval\ {\isacharparenleft}{\kern0pt}subst\ upd\ f{\isacharparenright}{\kern0pt}\ v\ {\isacharequal}{\kern0pt}\ eval\ f\ v{\isacharprime}{\kern0pt}{\isachardoublequoteclose}\isanewline
%
\isadelimproof
\ \ %
\endisadelimproof
%
\isatagproof
\isakeywordONE{by}\isamarkupfalse%
\ {\isacharparenleft}{\kern0pt}induction\ f\ rule{\isacharcolon}{\kern0pt}\ rlexp{\isachardot}{\kern0pt}induct{\isacharparenright}{\kern0pt}\ {\isacharparenleft}{\kern0pt}use\ assms\ \isakeywordTWO{in}\ auto{\isacharparenright}{\kern0pt}%
\endisatagproof
{\isafoldproof}%
%
\isadelimproof
\isanewline
%
\endisadelimproof
\isanewline
\isakeywordONE{lemma}\isamarkupfalse%
\ substitution{\isacharunderscore}{\kern0pt}lemma{\isacharunderscore}{\kern0pt}upd{\isacharcolon}{\kern0pt}\isanewline
\ \ {\isachardoublequoteopen}eval\ {\isacharparenleft}{\kern0pt}subst\ {\isacharparenleft}{\kern0pt}Var{\isacharparenleft}{\kern0pt}x\ {\isacharcolon}{\kern0pt}{\isacharequal}{\kern0pt}\ f{\isacharprime}{\kern0pt}{\isacharparenright}{\kern0pt}{\isacharparenright}{\kern0pt}\ f{\isacharparenright}{\kern0pt}\ v\ {\isacharequal}{\kern0pt}\ eval\ f\ {\isacharparenleft}{\kern0pt}v{\isacharparenleft}{\kern0pt}x\ {\isacharcolon}{\kern0pt}{\isacharequal}{\kern0pt}\ eval\ f{\isacharprime}{\kern0pt}\ v{\isacharparenright}{\kern0pt}{\isacharparenright}{\kern0pt}{\isachardoublequoteclose}\isanewline
%
\isadelimproof
\ \ %
\endisadelimproof
%
\isatagproof
\isakeywordONE{using}\isamarkupfalse%
\ substitution{\isacharunderscore}{\kern0pt}lemma{\isacharbrackleft}{\kern0pt}of\ {\isachardoublequoteopen}v{\isacharparenleft}{\kern0pt}x\ {\isacharcolon}{\kern0pt}{\isacharequal}{\kern0pt}\ eval\ f{\isacharprime}{\kern0pt}\ v{\isacharparenright}{\kern0pt}{\isachardoublequoteclose}{\isacharbrackright}{\kern0pt}\ \isakeywordONE{by}\isamarkupfalse%
\ force%
\endisatagproof
{\isafoldproof}%
%
\isadelimproof
\isanewline
%
\endisadelimproof
\isanewline
\isakeywordONE{lemma}\isamarkupfalse%
\ subst{\isacharunderscore}{\kern0pt}id{\isacharcolon}{\kern0pt}\ {\isachardoublequoteopen}eval\ {\isacharparenleft}{\kern0pt}subst\ Var\ f{\isacharparenright}{\kern0pt}\ v\ {\isacharequal}{\kern0pt}\ eval\ f\ v{\isachardoublequoteclose}\isanewline
%
\isadelimproof
\ \ %
\endisadelimproof
%
\isatagproof
\isakeywordONE{using}\isamarkupfalse%
\ substitution{\isacharunderscore}{\kern0pt}lemma{\isacharbrackleft}{\kern0pt}of\ v{\isacharbrackright}{\kern0pt}\ \isakeywordONE{by}\isamarkupfalse%
\ simp%
\endisatagproof
{\isafoldproof}%
%
\isadelimproof
\isanewline
%
\endisadelimproof
\isanewline
\isakeywordONE{lemma}\isamarkupfalse%
\ vars{\isacharunderscore}{\kern0pt}subst{\isacharcolon}{\kern0pt}\ {\isachardoublequoteopen}vars\ {\isacharparenleft}{\kern0pt}subst\ upd\ f{\isacharparenright}{\kern0pt}\ {\isacharequal}{\kern0pt}\ {\isacharparenleft}{\kern0pt}{\isasymUnion}x\ {\isasymin}\ vars\ f{\isachardot}{\kern0pt}\ vars\ {\isacharparenleft}{\kern0pt}upd\ x{\isacharparenright}{\kern0pt}{\isacharparenright}{\kern0pt}{\isachardoublequoteclose}\isanewline
%
\isadelimproof
\ \ %
\endisadelimproof
%
\isatagproof
\isakeywordONE{by}\isamarkupfalse%
\ {\isacharparenleft}{\kern0pt}induction\ f{\isacharparenright}{\kern0pt}\ auto%
\endisatagproof
{\isafoldproof}%
%
\isadelimproof
\isanewline
%
\endisadelimproof
\isanewline
\isakeywordONE{lemma}\isamarkupfalse%
\ vars{\isacharunderscore}{\kern0pt}subst{\isacharunderscore}{\kern0pt}upd{\isacharunderscore}{\kern0pt}upper{\isacharcolon}{\kern0pt}\ {\isachardoublequoteopen}vars\ {\isacharparenleft}{\kern0pt}subst\ {\isacharparenleft}{\kern0pt}Var{\isacharparenleft}{\kern0pt}x\ {\isacharcolon}{\kern0pt}{\isacharequal}{\kern0pt}\ fx{\isacharparenright}{\kern0pt}{\isacharparenright}{\kern0pt}\ f{\isacharparenright}{\kern0pt}\ {\isasymsubseteq}\ vars\ f\ {\isacharminus}{\kern0pt}\ {\isacharbraceleft}{\kern0pt}x{\isacharbraceright}{\kern0pt}\ {\isasymunion}\ vars\ fx{\isachardoublequoteclose}\isanewline
%
\isadelimproof
%
\endisadelimproof
%
\isatagproof
\isakeywordONE{proof}\isamarkupfalse%
\isanewline
\ \ \isakeywordTHREE{fix}\isamarkupfalse%
\ y\isanewline
\ \ \isakeywordONE{let}\isamarkupfalse%
\ {\isacharquery}{\kern0pt}upd\ {\isacharequal}{\kern0pt}\ {\isachardoublequoteopen}Var{\isacharparenleft}{\kern0pt}x\ {\isacharcolon}{\kern0pt}{\isacharequal}{\kern0pt}\ fx{\isacharparenright}{\kern0pt}{\isachardoublequoteclose}\isanewline
\ \ \isakeywordTHREE{assume}\isamarkupfalse%
\ {\isachardoublequoteopen}y\ {\isasymin}\ vars\ {\isacharparenleft}{\kern0pt}subst\ {\isacharquery}{\kern0pt}upd\ f{\isacharparenright}{\kern0pt}{\isachardoublequoteclose}\isanewline
\ \ \isakeywordONE{then}\isamarkupfalse%
\ \isakeywordTHREE{obtain}\isamarkupfalse%
\ y{\isacharprime}{\kern0pt}\ \isakeywordTWO{where}\ {\isachardoublequoteopen}y{\isacharprime}{\kern0pt}\ {\isasymin}\ vars\ f\ {\isasymand}\ y\ {\isasymin}\ vars\ {\isacharparenleft}{\kern0pt}{\isacharquery}{\kern0pt}upd\ y{\isacharprime}{\kern0pt}{\isacharparenright}{\kern0pt}{\isachardoublequoteclose}\ \isakeywordONE{using}\isamarkupfalse%
\ vars{\isacharunderscore}{\kern0pt}subst\ \isakeywordONE{by}\isamarkupfalse%
\ blast\isanewline
\ \ \isakeywordONE{then}\isamarkupfalse%
\ \isakeywordTHREE{show}\isamarkupfalse%
\ {\isachardoublequoteopen}y\ {\isasymin}\ vars\ f\ {\isacharminus}{\kern0pt}\ {\isacharbraceleft}{\kern0pt}x{\isacharbraceright}{\kern0pt}\ {\isasymunion}\ vars\ fx{\isachardoublequoteclose}\ \isakeywordONE{by}\isamarkupfalse%
\ {\isacharparenleft}{\kern0pt}cases\ {\isachardoublequoteopen}x\ {\isacharequal}{\kern0pt}\ y{\isacharprime}{\kern0pt}{\isachardoublequoteclose}{\isacharparenright}{\kern0pt}\ auto\isanewline
\isakeywordONE{qed}\isamarkupfalse%
%
\endisatagproof
{\isafoldproof}%
%
\isadelimproof
\isanewline
%
\endisadelimproof
\isanewline
\isanewline
\isakeywordONE{lemma}\isamarkupfalse%
\ eval{\isacharunderscore}{\kern0pt}vars{\isacharcolon}{\kern0pt}\isanewline
\ \ \isakeywordTWO{assumes}\ {\isachardoublequoteopen}{\isasymforall}i\ {\isasymin}\ vars\ f{\isachardot}{\kern0pt}\ s\ i\ {\isacharequal}{\kern0pt}\ s{\isacharprime}{\kern0pt}\ i{\isachardoublequoteclose}\isanewline
\ \ \isakeywordTWO{shows}\ {\isachardoublequoteopen}eval\ f\ s\ {\isacharequal}{\kern0pt}\ eval\ f\ s{\isacharprime}{\kern0pt}{\isachardoublequoteclose}\isanewline
%
\isadelimproof
\ \ %
\endisadelimproof
%
\isatagproof
\isakeywordONE{using}\isamarkupfalse%
\ assms\ \isakeywordONE{by}\isamarkupfalse%
\ {\isacharparenleft}{\kern0pt}induction\ f{\isacharparenright}{\kern0pt}\ auto%
\endisatagproof
{\isafoldproof}%
%
\isadelimproof
\isanewline
%
\endisadelimproof
\isanewline
\isakeywordONE{lemma}\isamarkupfalse%
\ eval{\isacharunderscore}{\kern0pt}vars{\isacharunderscore}{\kern0pt}subst{\isacharcolon}{\kern0pt}\isanewline
\ \ \isakeywordTWO{assumes}\ {\isachardoublequoteopen}{\isasymforall}i\ {\isasymin}\ vars\ f{\isachardot}{\kern0pt}\ v\ i\ {\isacharequal}{\kern0pt}\ eval\ {\isacharparenleft}{\kern0pt}upd\ i{\isacharparenright}{\kern0pt}\ v{\isachardoublequoteclose}\isanewline
\ \ \isakeywordTWO{shows}\ {\isachardoublequoteopen}eval\ {\isacharparenleft}{\kern0pt}subst\ upd\ f{\isacharparenright}{\kern0pt}\ v\ {\isacharequal}{\kern0pt}\ eval\ f\ v{\isachardoublequoteclose}\isanewline
%
\isadelimproof
%
\endisadelimproof
%
\isatagproof
\isakeywordONE{proof}\isamarkupfalse%
\ {\isacharminus}{\kern0pt}\isanewline
\ \ \isakeywordONE{let}\isamarkupfalse%
\ {\isacharquery}{\kern0pt}v{\isacharprime}{\kern0pt}\ {\isacharequal}{\kern0pt}\ {\isachardoublequoteopen}{\isasymlambda}i{\isachardot}{\kern0pt}\ if\ i\ {\isasymin}\ vars\ f\ then\ v\ i\ else\ eval\ {\isacharparenleft}{\kern0pt}upd\ i{\isacharparenright}{\kern0pt}\ v{\isachardoublequoteclose}\isanewline
\ \ \isakeywordONE{let}\isamarkupfalse%
\ {\isacharquery}{\kern0pt}v{\isacharprime}{\kern0pt}{\isacharprime}{\kern0pt}\ {\isacharequal}{\kern0pt}\ {\isachardoublequoteopen}{\isasymlambda}i{\isachardot}{\kern0pt}\ eval\ {\isacharparenleft}{\kern0pt}upd\ i{\isacharparenright}{\kern0pt}\ v{\isachardoublequoteclose}\isanewline
\ \ \isakeywordONE{have}\isamarkupfalse%
\ v{\isacharprime}{\kern0pt}{\isacharunderscore}{\kern0pt}v{\isacharprime}{\kern0pt}{\isacharprime}{\kern0pt}{\isacharcolon}{\kern0pt}\ {\isachardoublequoteopen}{\isacharquery}{\kern0pt}v{\isacharprime}{\kern0pt}\ i\ {\isacharequal}{\kern0pt}\ {\isacharquery}{\kern0pt}v{\isacharprime}{\kern0pt}{\isacharprime}{\kern0pt}\ i{\isachardoublequoteclose}\ \isakeywordTWO{for}\ i\ \isakeywordONE{using}\isamarkupfalse%
\ assms\ \isakeywordONE{by}\isamarkupfalse%
\ simp\isanewline
\ \ \isakeywordONE{then}\isamarkupfalse%
\ \isakeywordONE{have}\isamarkupfalse%
\ v{\isacharunderscore}{\kern0pt}v{\isacharprime}{\kern0pt}{\isacharprime}{\kern0pt}{\isacharcolon}{\kern0pt}\ {\isachardoublequoteopen}{\isasymforall}i{\isachardot}{\kern0pt}\ {\isacharquery}{\kern0pt}v{\isacharprime}{\kern0pt}{\isacharprime}{\kern0pt}\ i\ {\isacharequal}{\kern0pt}\ eval\ {\isacharparenleft}{\kern0pt}upd\ i{\isacharparenright}{\kern0pt}\ v{\isachardoublequoteclose}\ \isakeywordONE{by}\isamarkupfalse%
\ simp\isanewline
\ \ \isakeywordONE{from}\isamarkupfalse%
\ assms\ \isakeywordONE{have}\isamarkupfalse%
\ {\isachardoublequoteopen}eval\ f\ v\ {\isacharequal}{\kern0pt}\ eval\ f\ {\isacharquery}{\kern0pt}v{\isacharprime}{\kern0pt}{\isachardoublequoteclose}\ \isakeywordONE{using}\isamarkupfalse%
\ eval{\isacharunderscore}{\kern0pt}vars{\isacharbrackleft}{\kern0pt}of\ f{\isacharbrackright}{\kern0pt}\ \isakeywordONE{by}\isamarkupfalse%
\ simp\isanewline
\ \ \isakeywordONE{also}\isamarkupfalse%
\ \isakeywordONE{have}\isamarkupfalse%
\ {\isachardoublequoteopen}{\isasymdots}\ {\isacharequal}{\kern0pt}\ eval\ {\isacharparenleft}{\kern0pt}subst\ upd\ f{\isacharparenright}{\kern0pt}\ v{\isachardoublequoteclose}\isanewline
\ \ \ \ \isakeywordONE{using}\isamarkupfalse%
\ assms\ substitution{\isacharunderscore}{\kern0pt}lemma{\isacharbrackleft}{\kern0pt}OF\ v{\isacharunderscore}{\kern0pt}v{\isacharprime}{\kern0pt}{\isacharprime}{\kern0pt}{\isacharcomma}{\kern0pt}\ of\ f{\isacharbrackright}{\kern0pt}\ \isakeywordONE{by}\isamarkupfalse%
\ {\isacharparenleft}{\kern0pt}simp\ add{\isacharcolon}{\kern0pt}\ eval{\isacharunderscore}{\kern0pt}vars{\isacharparenright}{\kern0pt}\isanewline
\ \ \isakeywordONE{finally}\isamarkupfalse%
\ \isakeywordTHREE{show}\isamarkupfalse%
\ {\isacharquery}{\kern0pt}thesis\ \isakeywordONE{by}\isamarkupfalse%
\ simp\isanewline
\isakeywordONE{qed}\isamarkupfalse%
%
\endisatagproof
{\isafoldproof}%
%
\isadelimproof
%
\endisadelimproof
%
\begin{isamarkuptext}%
\isa{\isaconst{eval}\ \isafree{f}} is monotone:%
\end{isamarkuptext}\isamarkuptrue%
\isakeywordONE{lemma}\isamarkupfalse%
\ rlexp{\isacharunderscore}{\kern0pt}mono{\isacharcolon}{\kern0pt}\isanewline
\ \ \isakeywordTWO{assumes}\ {\isachardoublequoteopen}{\isasymforall}i\ {\isasymin}\ vars\ f{\isachardot}{\kern0pt}\ v\ i\ {\isasymsubseteq}\ v{\isacharprime}{\kern0pt}\ i{\isachardoublequoteclose}\isanewline
\ \ \isakeywordTWO{shows}\ {\isachardoublequoteopen}eval\ f\ v\ {\isasymsubseteq}\ eval\ f\ v{\isacharprime}{\kern0pt}{\isachardoublequoteclose}\isanewline
%
\isadelimproof
%
\endisadelimproof
%
\isatagproof
\isakeywordONE{using}\isamarkupfalse%
\ assms\ \isakeywordONE{proof}\isamarkupfalse%
\ {\isacharparenleft}{\kern0pt}induction\ f\ rule{\isacharcolon}{\kern0pt}\ rlexp{\isachardot}{\kern0pt}induct{\isacharparenright}{\kern0pt}\isanewline
\ \ \isakeywordTHREE{case}\isamarkupfalse%
\ {\isacharparenleft}{\kern0pt}Star\ x{\isacharparenright}{\kern0pt}\isanewline
\ \ \isakeywordONE{then}\isamarkupfalse%
\ \isakeywordTHREE{show}\isamarkupfalse%
\ {\isacharquery}{\kern0pt}case\isanewline
\ \ \ \ \isakeywordONE{by}\isamarkupfalse%
\ {\isacharparenleft}{\kern0pt}smt\ {\isacharparenleft}{\kern0pt}verit{\isacharcomma}{\kern0pt}\ best{\isacharparenright}{\kern0pt}\ eval{\isachardot}{\kern0pt}simps{\isacharparenleft}{\kern0pt}{\isadigit{5}}{\isacharparenright}{\kern0pt}\ in{\isacharunderscore}{\kern0pt}star{\isacharunderscore}{\kern0pt}iff{\isacharunderscore}{\kern0pt}concat\ order{\isacharunderscore}{\kern0pt}trans\ subsetI\ vars{\isachardot}{\kern0pt}simps{\isacharparenleft}{\kern0pt}{\isadigit{5}}{\isacharparenright}{\kern0pt}{\isacharparenright}{\kern0pt}\isanewline
\isakeywordONE{qed}\isamarkupfalse%
\ fastforce{\isacharplus}{\kern0pt}%
\endisatagproof
{\isafoldproof}%
%
\isadelimproof
%
\endisadelimproof
%
\isadelimdocument
%
\endisadelimdocument
%
\isatagdocument
%
\isamarkupsubsection{Continuity%
}
\isamarkuptrue%
%
\endisatagdocument
{\isafolddocument}%
%
\isadelimdocument
%
\endisadelimdocument
\isakeywordONE{lemma}\isamarkupfalse%
\ langpow{\isacharunderscore}{\kern0pt}mono{\isacharcolon}{\kern0pt}\isanewline
\ \ \isakeywordTWO{fixes}\ A\ {\isacharcolon}{\kern0pt}{\isacharcolon}{\kern0pt}\ {\isachardoublequoteopen}{\isacharprime}{\kern0pt}a\ lang{\isachardoublequoteclose}\isanewline
\ \ \isakeywordTWO{assumes}\ {\isachardoublequoteopen}A\ {\isasymsubseteq}\ B{\isachardoublequoteclose}\isanewline
\ \ \isakeywordTWO{shows}\ {\isachardoublequoteopen}A\ {\isacharcircum}{\kern0pt}{\isacharcircum}{\kern0pt}\ n\ {\isasymsubseteq}\ B\ {\isacharcircum}{\kern0pt}{\isacharcircum}{\kern0pt}\ n{\isachardoublequoteclose}\isanewline
%
\isadelimproof
\ \ %
\endisadelimproof
%
\isatagproof
\isakeywordONE{by}\isamarkupfalse%
\ {\isacharparenleft}{\kern0pt}induction\ n{\isacharparenright}{\kern0pt}\ {\isacharparenleft}{\kern0pt}use\ assms\ conc{\isacharunderscore}{\kern0pt}mono{\isacharbrackleft}{\kern0pt}of\ A\ B{\isacharbrackright}{\kern0pt}\ \isakeywordTWO{in}\ auto{\isacharparenright}{\kern0pt}%
\endisatagproof
{\isafoldproof}%
%
\isadelimproof
\isanewline
%
\endisadelimproof
\isanewline
\isakeywordONE{lemma}\isamarkupfalse%
\ rlexp{\isacharunderscore}{\kern0pt}cont{\isacharunderscore}{\kern0pt}aux{\isadigit{1}}{\isacharcolon}{\kern0pt}\isanewline
\ \ \isakeywordTWO{assumes}\ {\isachardoublequoteopen}{\isasymforall}i{\isachardot}{\kern0pt}\ v\ i\ {\isasymle}\ v\ {\isacharparenleft}{\kern0pt}Suc\ i{\isacharparenright}{\kern0pt}{\isachardoublequoteclose}\isanewline
\ \ \ \ \ \ \isakeywordTWO{and}\ {\isachardoublequoteopen}w\ {\isasymin}\ {\isacharparenleft}{\kern0pt}{\isasymUnion}i{\isachardot}{\kern0pt}\ eval\ f\ {\isacharparenleft}{\kern0pt}v\ i{\isacharparenright}{\kern0pt}{\isacharparenright}{\kern0pt}{\isachardoublequoteclose}\isanewline
\ \ \ \ \isakeywordTWO{shows}\ {\isachardoublequoteopen}w\ {\isasymin}\ eval\ f\ {\isacharparenleft}{\kern0pt}{\isasymlambda}x{\isachardot}{\kern0pt}\ {\isasymUnion}i{\isachardot}{\kern0pt}\ v\ i\ x{\isacharparenright}{\kern0pt}{\isachardoublequoteclose}\isanewline
%
\isadelimproof
%
\endisadelimproof
%
\isatagproof
\isakeywordONE{proof}\isamarkupfalse%
\ {\isacharminus}{\kern0pt}\isanewline
\ \ \isakeywordONE{from}\isamarkupfalse%
\ assms{\isacharparenleft}{\kern0pt}{\isadigit{2}}{\isacharparenright}{\kern0pt}\ \isakeywordTHREE{obtain}\isamarkupfalse%
\ n\ \isakeywordTWO{where}\ n{\isacharunderscore}{\kern0pt}intro{\isacharcolon}{\kern0pt}\ {\isachardoublequoteopen}w\ {\isasymin}\ eval\ f\ {\isacharparenleft}{\kern0pt}v\ n{\isacharparenright}{\kern0pt}{\isachardoublequoteclose}\ \isakeywordONE{by}\isamarkupfalse%
\ auto\isanewline
\ \ \isakeywordONE{have}\isamarkupfalse%
\ {\isachardoublequoteopen}v\ n\ x\ {\isasymsubseteq}\ {\isacharparenleft}{\kern0pt}{\isasymUnion}i{\isachardot}{\kern0pt}\ v\ i\ x{\isacharparenright}{\kern0pt}{\isachardoublequoteclose}\ \isakeywordTWO{for}\ x\ \isakeywordONE{by}\isamarkupfalse%
\ auto\isanewline
\ \ \isakeywordONE{with}\isamarkupfalse%
\ n{\isacharunderscore}{\kern0pt}intro\ \isakeywordTHREE{show}\isamarkupfalse%
\ {\isachardoublequoteopen}{\isacharquery}{\kern0pt}thesis{\isachardoublequoteclose}\isanewline
\ \ \ \ \isakeywordONE{using}\isamarkupfalse%
\ rlexp{\isacharunderscore}{\kern0pt}mono{\isacharbrackleft}{\kern0pt}\isakeywordTWO{where}\ v{\isacharequal}{\kern0pt}{\isachardoublequoteopen}v\ n{\isachardoublequoteclose}\ \isakeywordTWO{and}\ v{\isacharprime}{\kern0pt}{\isacharequal}{\kern0pt}{\isachardoublequoteopen}{\isasymlambda}x{\isachardot}{\kern0pt}\ {\isasymUnion}i{\isachardot}{\kern0pt}\ v\ i\ x{\isachardoublequoteclose}{\isacharbrackright}{\kern0pt}\ \isakeywordONE{by}\isamarkupfalse%
\ auto\isanewline
\isakeywordONE{qed}\isamarkupfalse%
%
\endisatagproof
{\isafoldproof}%
%
\isadelimproof
\isanewline
%
\endisadelimproof
\isanewline
\isakeywordONE{lemma}\isamarkupfalse%
\ langpow{\isacharunderscore}{\kern0pt}Union{\isacharunderscore}{\kern0pt}eval{\isacharcolon}{\kern0pt}\isanewline
\ \ \isakeywordTWO{assumes}\ {\isachardoublequoteopen}{\isasymforall}i{\isachardot}{\kern0pt}\ v\ i\ {\isasymle}\ v\ {\isacharparenleft}{\kern0pt}Suc\ i{\isacharparenright}{\kern0pt}{\isachardoublequoteclose}\isanewline
\ \ \ \ \ \ \isakeywordTWO{and}\ {\isachardoublequoteopen}w\ {\isasymin}\ {\isacharparenleft}{\kern0pt}{\isasymUnion}i{\isachardot}{\kern0pt}\ eval\ f\ {\isacharparenleft}{\kern0pt}v\ i{\isacharparenright}{\kern0pt}{\isacharparenright}{\kern0pt}\ {\isacharcircum}{\kern0pt}{\isacharcircum}{\kern0pt}\ n{\isachardoublequoteclose}\isanewline
\ \ \ \ \isakeywordTWO{shows}\ {\isachardoublequoteopen}w\ {\isasymin}\ {\isacharparenleft}{\kern0pt}{\isasymUnion}i{\isachardot}{\kern0pt}\ eval\ f\ {\isacharparenleft}{\kern0pt}v\ i{\isacharparenright}{\kern0pt}\ {\isacharcircum}{\kern0pt}{\isacharcircum}{\kern0pt}\ n{\isacharparenright}{\kern0pt}{\isachardoublequoteclose}\isanewline
%
\isadelimproof
%
\endisadelimproof
%
\isatagproof
\isakeywordONE{using}\isamarkupfalse%
\ assms{\isacharparenleft}{\kern0pt}{\isadigit{2}}{\isacharparenright}{\kern0pt}\ \isakeywordONE{proof}\isamarkupfalse%
\ {\isacharparenleft}{\kern0pt}induction\ n\ arbitrary{\isacharcolon}{\kern0pt}\ w{\isacharparenright}{\kern0pt}\isanewline
\ \ \isakeywordTHREE{case}\isamarkupfalse%
\ {\isadigit{0}}\isanewline
\ \ \isakeywordONE{then}\isamarkupfalse%
\ \isakeywordTHREE{show}\isamarkupfalse%
\ {\isacharquery}{\kern0pt}case\ \isakeywordONE{by}\isamarkupfalse%
\ simp\isanewline
\isakeywordONE{next}\isamarkupfalse%
\isanewline
\ \ \isakeywordTHREE{case}\isamarkupfalse%
\ {\isacharparenleft}{\kern0pt}Suc\ n{\isacharparenright}{\kern0pt}\isanewline
\ \ \isakeywordONE{then}\isamarkupfalse%
\ \isakeywordTHREE{obtain}\isamarkupfalse%
\ u\ u{\isacharprime}{\kern0pt}\ \isakeywordTWO{where}\ w{\isacharunderscore}{\kern0pt}decomp{\isacharcolon}{\kern0pt}\ {\isachardoublequoteopen}w\ {\isacharequal}{\kern0pt}\ u{\isacharat}{\kern0pt}u{\isacharprime}{\kern0pt}{\isachardoublequoteclose}\ \isakeywordTWO{and}\isanewline
\ \ \ \ {\isachardoublequoteopen}u\ {\isasymin}\ {\isacharparenleft}{\kern0pt}{\isasymUnion}i{\isachardot}{\kern0pt}\ eval\ f\ {\isacharparenleft}{\kern0pt}v\ i{\isacharparenright}{\kern0pt}{\isacharparenright}{\kern0pt}\ {\isasymand}\ u{\isacharprime}{\kern0pt}\ {\isasymin}\ {\isacharparenleft}{\kern0pt}{\isasymUnion}i{\isachardot}{\kern0pt}\ eval\ f\ {\isacharparenleft}{\kern0pt}v\ i{\isacharparenright}{\kern0pt}{\isacharparenright}{\kern0pt}\ {\isacharcircum}{\kern0pt}{\isacharcircum}{\kern0pt}\ n{\isachardoublequoteclose}\ \isakeywordONE{by}\isamarkupfalse%
\ fastforce\isanewline
\ \ \isakeywordONE{with}\isamarkupfalse%
\ Suc\ \isakeywordONE{have}\isamarkupfalse%
\ {\isachardoublequoteopen}u\ {\isasymin}\ {\isacharparenleft}{\kern0pt}{\isasymUnion}i{\isachardot}{\kern0pt}\ eval\ f\ {\isacharparenleft}{\kern0pt}v\ i{\isacharparenright}{\kern0pt}{\isacharparenright}{\kern0pt}\ {\isasymand}\ u{\isacharprime}{\kern0pt}\ {\isasymin}\ {\isacharparenleft}{\kern0pt}{\isasymUnion}i{\isachardot}{\kern0pt}\ eval\ f\ {\isacharparenleft}{\kern0pt}v\ i{\isacharparenright}{\kern0pt}\ {\isacharcircum}{\kern0pt}{\isacharcircum}{\kern0pt}\ n{\isacharparenright}{\kern0pt}{\isachardoublequoteclose}\ \isakeywordONE{by}\isamarkupfalse%
\ auto\isanewline
\ \ \isakeywordONE{then}\isamarkupfalse%
\ \isakeywordTHREE{obtain}\isamarkupfalse%
\ i\ j\ \isakeywordTWO{where}\ i{\isacharunderscore}{\kern0pt}intro{\isacharcolon}{\kern0pt}\ {\isachardoublequoteopen}u\ {\isasymin}\ eval\ f\ {\isacharparenleft}{\kern0pt}v\ i{\isacharparenright}{\kern0pt}{\isachardoublequoteclose}\ \isakeywordTWO{and}\ j{\isacharunderscore}{\kern0pt}intro{\isacharcolon}{\kern0pt}\ {\isachardoublequoteopen}u{\isacharprime}{\kern0pt}\ {\isasymin}\ eval\ f\ {\isacharparenleft}{\kern0pt}v\ j{\isacharparenright}{\kern0pt}\ {\isacharcircum}{\kern0pt}{\isacharcircum}{\kern0pt}\ n{\isachardoublequoteclose}\ \isakeywordONE{by}\isamarkupfalse%
\ blast\isanewline
\ \ \isakeywordONE{let}\isamarkupfalse%
\ {\isacharquery}{\kern0pt}m\ {\isacharequal}{\kern0pt}\ {\isachardoublequoteopen}max\ i\ j{\isachardoublequoteclose}\isanewline
\ \ \isakeywordONE{from}\isamarkupfalse%
\ i{\isacharunderscore}{\kern0pt}intro\ Suc{\isachardot}{\kern0pt}prems{\isacharparenleft}{\kern0pt}{\isadigit{1}}{\isacharparenright}{\kern0pt}\ assms{\isacharparenleft}{\kern0pt}{\isadigit{1}}{\isacharparenright}{\kern0pt}\ rlexp{\isacharunderscore}{\kern0pt}mono\ \isakeywordONE{have}\isamarkupfalse%
\ {\isadigit{1}}{\isacharcolon}{\kern0pt}\ {\isachardoublequoteopen}u\ {\isasymin}\ eval\ f\ {\isacharparenleft}{\kern0pt}v\ {\isacharquery}{\kern0pt}m{\isacharparenright}{\kern0pt}{\isachardoublequoteclose}\isanewline
\ \ \ \ \isakeywordONE{by}\isamarkupfalse%
\ {\isacharparenleft}{\kern0pt}metis\ le{\isacharunderscore}{\kern0pt}fun{\isacharunderscore}{\kern0pt}def\ lift{\isacharunderscore}{\kern0pt}Suc{\isacharunderscore}{\kern0pt}mono{\isacharunderscore}{\kern0pt}le\ max{\isachardot}{\kern0pt}cobounded{\isadigit{1}}\ subset{\isacharunderscore}{\kern0pt}eq{\isacharparenright}{\kern0pt}\isanewline
\ \ \isakeywordONE{from}\isamarkupfalse%
\ Suc{\isachardot}{\kern0pt}prems{\isacharparenleft}{\kern0pt}{\isadigit{1}}{\isacharparenright}{\kern0pt}\ assms\ {\isacharparenleft}{\kern0pt}{\isadigit{1}}{\isacharparenright}{\kern0pt}\ rlexp{\isacharunderscore}{\kern0pt}mono\ \isakeywordONE{have}\isamarkupfalse%
\ {\isachardoublequoteopen}eval\ f\ {\isacharparenleft}{\kern0pt}v\ j{\isacharparenright}{\kern0pt}\ {\isasymsubseteq}\ eval\ f\ {\isacharparenleft}{\kern0pt}v\ {\isacharquery}{\kern0pt}m{\isacharparenright}{\kern0pt}{\isachardoublequoteclose}\isanewline
\ \ \ \ \isakeywordONE{by}\isamarkupfalse%
\ {\isacharparenleft}{\kern0pt}metis\ le{\isacharunderscore}{\kern0pt}fun{\isacharunderscore}{\kern0pt}def\ lift{\isacharunderscore}{\kern0pt}Suc{\isacharunderscore}{\kern0pt}mono{\isacharunderscore}{\kern0pt}le\ max{\isachardot}{\kern0pt}cobounded{\isadigit{2}}{\isacharparenright}{\kern0pt}\isanewline
\ \ \isakeywordONE{with}\isamarkupfalse%
\ j{\isacharunderscore}{\kern0pt}intro\ langpow{\isacharunderscore}{\kern0pt}mono\ \isakeywordONE{have}\isamarkupfalse%
\ {\isadigit{2}}{\isacharcolon}{\kern0pt}\ {\isachardoublequoteopen}u{\isacharprime}{\kern0pt}\ {\isasymin}\ eval\ f\ {\isacharparenleft}{\kern0pt}v\ {\isacharquery}{\kern0pt}m{\isacharparenright}{\kern0pt}\ {\isacharcircum}{\kern0pt}{\isacharcircum}{\kern0pt}\ n{\isachardoublequoteclose}\ \isakeywordONE{by}\isamarkupfalse%
\ auto\isanewline
\ \ \isakeywordONE{from}\isamarkupfalse%
\ {\isadigit{1}}\ {\isadigit{2}}\ \isakeywordTHREE{show}\isamarkupfalse%
\ {\isacharquery}{\kern0pt}case\ \isakeywordONE{using}\isamarkupfalse%
\ w{\isacharunderscore}{\kern0pt}decomp\ \isakeywordONE{by}\isamarkupfalse%
\ auto\isanewline
\isakeywordONE{qed}\isamarkupfalse%
%
\endisatagproof
{\isafoldproof}%
%
\isadelimproof
\isanewline
%
\endisadelimproof
\isanewline
\isakeywordONE{lemma}\isamarkupfalse%
\ rlexp{\isacharunderscore}{\kern0pt}cont{\isacharunderscore}{\kern0pt}aux{\isadigit{2}}{\isacharcolon}{\kern0pt}\isanewline
\ \ \isakeywordTWO{assumes}\ {\isachardoublequoteopen}{\isasymforall}i{\isachardot}{\kern0pt}\ v\ i\ {\isasymle}\ v\ {\isacharparenleft}{\kern0pt}Suc\ i{\isacharparenright}{\kern0pt}{\isachardoublequoteclose}\isanewline
\ \ \ \ \ \ \isakeywordTWO{and}\ {\isachardoublequoteopen}w\ {\isasymin}\ eval\ f\ {\isacharparenleft}{\kern0pt}{\isasymlambda}x{\isachardot}{\kern0pt}\ {\isasymUnion}i{\isachardot}{\kern0pt}\ v\ i\ x{\isacharparenright}{\kern0pt}{\isachardoublequoteclose}\isanewline
\ \ \ \ \isakeywordTWO{shows}\ {\isachardoublequoteopen}w\ {\isasymin}\ {\isacharparenleft}{\kern0pt}{\isasymUnion}i{\isachardot}{\kern0pt}\ eval\ f\ {\isacharparenleft}{\kern0pt}v\ i{\isacharparenright}{\kern0pt}{\isacharparenright}{\kern0pt}{\isachardoublequoteclose}\isanewline
%
\isadelimproof
%
\endisadelimproof
%
\isatagproof
\isakeywordONE{using}\isamarkupfalse%
\ assms{\isacharparenleft}{\kern0pt}{\isadigit{2}}{\isacharparenright}{\kern0pt}\ \isakeywordONE{proof}\isamarkupfalse%
\ {\isacharparenleft}{\kern0pt}induction\ f\ arbitrary{\isacharcolon}{\kern0pt}\ w\ rule{\isacharcolon}{\kern0pt}\ rlexp{\isachardot}{\kern0pt}induct{\isacharparenright}{\kern0pt}\isanewline
\ \ \isakeywordTHREE{case}\isamarkupfalse%
\ {\isacharparenleft}{\kern0pt}Concat\ f\ g{\isacharparenright}{\kern0pt}\isanewline
\ \ \isakeywordONE{then}\isamarkupfalse%
\ \isakeywordTHREE{obtain}\isamarkupfalse%
\ u\ u{\isacharprime}{\kern0pt}\ \isakeywordTWO{where}\ w{\isacharunderscore}{\kern0pt}decomp{\isacharcolon}{\kern0pt}\ {\isachardoublequoteopen}w\ {\isacharequal}{\kern0pt}\ u{\isacharat}{\kern0pt}u{\isacharprime}{\kern0pt}{\isachardoublequoteclose}\isanewline
\ \ \ \ \isakeywordTWO{and}\ {\isachardoublequoteopen}u\ {\isasymin}\ eval\ f\ {\isacharparenleft}{\kern0pt}{\isasymlambda}x{\isachardot}{\kern0pt}\ {\isasymUnion}i{\isachardot}{\kern0pt}\ v\ i\ x{\isacharparenright}{\kern0pt}\ {\isasymand}\ u{\isacharprime}{\kern0pt}\ {\isasymin}\ eval\ g\ {\isacharparenleft}{\kern0pt}{\isasymlambda}x{\isachardot}{\kern0pt}\ {\isasymUnion}i{\isachardot}{\kern0pt}\ v\ i\ x{\isacharparenright}{\kern0pt}{\isachardoublequoteclose}\ \isakeywordONE{by}\isamarkupfalse%
\ auto\isanewline
\ \ \isakeywordONE{with}\isamarkupfalse%
\ Concat\ \isakeywordONE{have}\isamarkupfalse%
\ {\isachardoublequoteopen}u\ {\isasymin}\ {\isacharparenleft}{\kern0pt}{\isasymUnion}i{\isachardot}{\kern0pt}\ eval\ f\ {\isacharparenleft}{\kern0pt}v\ i{\isacharparenright}{\kern0pt}{\isacharparenright}{\kern0pt}\ {\isasymand}\ u{\isacharprime}{\kern0pt}\ {\isasymin}\ {\isacharparenleft}{\kern0pt}{\isasymUnion}i{\isachardot}{\kern0pt}\ eval\ g\ {\isacharparenleft}{\kern0pt}v\ i{\isacharparenright}{\kern0pt}{\isacharparenright}{\kern0pt}{\isachardoublequoteclose}\ \isakeywordONE{by}\isamarkupfalse%
\ auto\isanewline
\ \ \isakeywordONE{then}\isamarkupfalse%
\ \isakeywordTHREE{obtain}\isamarkupfalse%
\ i\ j\ \isakeywordTWO{where}\ i{\isacharunderscore}{\kern0pt}intro{\isacharcolon}{\kern0pt}\ {\isachardoublequoteopen}u\ {\isasymin}\ eval\ f\ {\isacharparenleft}{\kern0pt}v\ i{\isacharparenright}{\kern0pt}{\isachardoublequoteclose}\ \isakeywordTWO{and}\ j{\isacharunderscore}{\kern0pt}intro{\isacharcolon}{\kern0pt}\ {\isachardoublequoteopen}u{\isacharprime}{\kern0pt}\ {\isasymin}\ eval\ g\ {\isacharparenleft}{\kern0pt}v\ j{\isacharparenright}{\kern0pt}{\isachardoublequoteclose}\ \isakeywordONE{by}\isamarkupfalse%
\ blast\isanewline
\ \ \isakeywordONE{let}\isamarkupfalse%
\ {\isacharquery}{\kern0pt}m\ {\isacharequal}{\kern0pt}\ {\isachardoublequoteopen}max\ i\ j{\isachardoublequoteclose}\isanewline
\ \ \isakeywordONE{from}\isamarkupfalse%
\ i{\isacharunderscore}{\kern0pt}intro\ Concat{\isachardot}{\kern0pt}prems{\isacharparenleft}{\kern0pt}{\isadigit{1}}{\isacharparenright}{\kern0pt}\ assms{\isacharparenleft}{\kern0pt}{\isadigit{1}}{\isacharparenright}{\kern0pt}\ rlexp{\isacharunderscore}{\kern0pt}mono\ \isakeywordONE{have}\isamarkupfalse%
\ {\isachardoublequoteopen}u\ {\isasymin}\ eval\ f\ {\isacharparenleft}{\kern0pt}v\ {\isacharquery}{\kern0pt}m{\isacharparenright}{\kern0pt}{\isachardoublequoteclose}\isanewline
\ \ \ \ \isakeywordONE{by}\isamarkupfalse%
\ {\isacharparenleft}{\kern0pt}metis\ le{\isacharunderscore}{\kern0pt}fun{\isacharunderscore}{\kern0pt}def\ lift{\isacharunderscore}{\kern0pt}Suc{\isacharunderscore}{\kern0pt}mono{\isacharunderscore}{\kern0pt}le\ max{\isachardot}{\kern0pt}cobounded{\isadigit{1}}\ subset{\isacharunderscore}{\kern0pt}eq{\isacharparenright}{\kern0pt}\isanewline
\ \ \isakeywordONE{moreover}\isamarkupfalse%
\ \isakeywordONE{from}\isamarkupfalse%
\ j{\isacharunderscore}{\kern0pt}intro\ Concat{\isachardot}{\kern0pt}prems{\isacharparenleft}{\kern0pt}{\isadigit{1}}{\isacharparenright}{\kern0pt}\ assms{\isacharparenleft}{\kern0pt}{\isadigit{1}}{\isacharparenright}{\kern0pt}\ rlexp{\isacharunderscore}{\kern0pt}mono\ \isakeywordONE{have}\isamarkupfalse%
\ {\isachardoublequoteopen}u{\isacharprime}{\kern0pt}\ {\isasymin}\ eval\ g\ {\isacharparenleft}{\kern0pt}v\ {\isacharquery}{\kern0pt}m{\isacharparenright}{\kern0pt}{\isachardoublequoteclose}\isanewline
\ \ \ \ \isakeywordONE{by}\isamarkupfalse%
\ {\isacharparenleft}{\kern0pt}metis\ le{\isacharunderscore}{\kern0pt}fun{\isacharunderscore}{\kern0pt}def\ lift{\isacharunderscore}{\kern0pt}Suc{\isacharunderscore}{\kern0pt}mono{\isacharunderscore}{\kern0pt}le\ max{\isachardot}{\kern0pt}cobounded{\isadigit{2}}\ subset{\isacharunderscore}{\kern0pt}eq{\isacharparenright}{\kern0pt}\isanewline
\ \ \isakeywordONE{ultimately}\isamarkupfalse%
\ \isakeywordTHREE{show}\isamarkupfalse%
\ {\isacharquery}{\kern0pt}case\ \isakeywordONE{using}\isamarkupfalse%
\ w{\isacharunderscore}{\kern0pt}decomp\ \isakeywordONE{by}\isamarkupfalse%
\ auto\isanewline
\isakeywordONE{next}\isamarkupfalse%
\isanewline
\ \ \isakeywordTHREE{case}\isamarkupfalse%
\ {\isacharparenleft}{\kern0pt}Star\ f{\isacharparenright}{\kern0pt}\isanewline
\ \ \isakeywordONE{then}\isamarkupfalse%
\ \isakeywordTHREE{obtain}\isamarkupfalse%
\ n\ \isakeywordTWO{where}\ n{\isacharunderscore}{\kern0pt}intro{\isacharcolon}{\kern0pt}\ {\isachardoublequoteopen}w\ {\isasymin}\ {\isacharparenleft}{\kern0pt}eval\ f\ {\isacharparenleft}{\kern0pt}{\isasymlambda}x{\isachardot}{\kern0pt}\ {\isasymUnion}i{\isachardot}{\kern0pt}\ v\ i\ x{\isacharparenright}{\kern0pt}{\isacharparenright}{\kern0pt}\ {\isacharcircum}{\kern0pt}{\isacharcircum}{\kern0pt}\ n{\isachardoublequoteclose}\isanewline
\ \ \ \ \isakeywordONE{using}\isamarkupfalse%
\ eval{\isachardot}{\kern0pt}simps{\isacharparenleft}{\kern0pt}{\isadigit{5}}{\isacharparenright}{\kern0pt}\ star{\isacharunderscore}{\kern0pt}pow\ \isakeywordONE{by}\isamarkupfalse%
\ blast\isanewline
\ \ \isakeywordONE{with}\isamarkupfalse%
\ Star\ \isakeywordONE{have}\isamarkupfalse%
\ {\isachardoublequoteopen}w\ {\isasymin}\ {\isacharparenleft}{\kern0pt}{\isasymUnion}i{\isachardot}{\kern0pt}\ eval\ f\ {\isacharparenleft}{\kern0pt}v\ i{\isacharparenright}{\kern0pt}{\isacharparenright}{\kern0pt}\ {\isacharcircum}{\kern0pt}{\isacharcircum}{\kern0pt}\ n{\isachardoublequoteclose}\ \isakeywordONE{using}\isamarkupfalse%
\ langpow{\isacharunderscore}{\kern0pt}mono\ \isakeywordONE{by}\isamarkupfalse%
\ blast\isanewline
\ \ \isakeywordONE{with}\isamarkupfalse%
\ Star{\isachardot}{\kern0pt}prems\ assms\ \isakeywordONE{have}\isamarkupfalse%
\ {\isachardoublequoteopen}w\ {\isasymin}\ {\isacharparenleft}{\kern0pt}{\isasymUnion}i{\isachardot}{\kern0pt}\ eval\ f\ {\isacharparenleft}{\kern0pt}v\ i{\isacharparenright}{\kern0pt}\ {\isacharcircum}{\kern0pt}{\isacharcircum}{\kern0pt}\ n{\isacharparenright}{\kern0pt}{\isachardoublequoteclose}\ \isakeywordONE{using}\isamarkupfalse%
\ langpow{\isacharunderscore}{\kern0pt}Union{\isacharunderscore}{\kern0pt}eval\ \isakeywordONE{by}\isamarkupfalse%
\ auto\isanewline
\ \ \isakeywordONE{then}\isamarkupfalse%
\ \isakeywordTHREE{show}\isamarkupfalse%
\ {\isacharquery}{\kern0pt}case\ \isakeywordONE{by}\isamarkupfalse%
\ {\isacharparenleft}{\kern0pt}auto\ simp\ add{\isacharcolon}{\kern0pt}\ star{\isacharunderscore}{\kern0pt}def{\isacharparenright}{\kern0pt}\isanewline
\isakeywordONE{qed}\isamarkupfalse%
\ fastforce{\isacharplus}{\kern0pt}%
\endisatagproof
{\isafoldproof}%
%
\isadelimproof
%
\endisadelimproof
%
\begin{isamarkuptext}%
Now we prove that \isa{\isaconst{eval}\ \isafree{f}} is continuous. This result is not needed in the further
proof, but it is interesting anyway:%
\end{isamarkuptext}\isamarkuptrue%
\isakeywordONE{lemma}\isamarkupfalse%
\ rlexp{\isacharunderscore}{\kern0pt}cont{\isacharcolon}{\kern0pt}\isanewline
\ \ \isakeywordTWO{assumes}\ {\isachardoublequoteopen}{\isasymforall}i{\isachardot}{\kern0pt}\ v\ i\ {\isasymle}\ v\ {\isacharparenleft}{\kern0pt}Suc\ i{\isacharparenright}{\kern0pt}{\isachardoublequoteclose}\isanewline
\ \ \isakeywordTWO{shows}\ {\isachardoublequoteopen}eval\ f\ {\isacharparenleft}{\kern0pt}{\isasymlambda}x{\isachardot}{\kern0pt}\ {\isasymUnion}i{\isachardot}{\kern0pt}\ v\ i\ x{\isacharparenright}{\kern0pt}\ {\isacharequal}{\kern0pt}\ {\isacharparenleft}{\kern0pt}{\isasymUnion}i{\isachardot}{\kern0pt}\ eval\ f\ {\isacharparenleft}{\kern0pt}v\ i{\isacharparenright}{\kern0pt}{\isacharparenright}{\kern0pt}{\isachardoublequoteclose}\isanewline
%
\isadelimproof
%
\endisadelimproof
%
\isatagproof
\isakeywordONE{proof}\isamarkupfalse%
\isanewline
\ \ \isakeywordONE{from}\isamarkupfalse%
\ assms\ \isakeywordTHREE{show}\isamarkupfalse%
\ {\isachardoublequoteopen}eval\ f\ {\isacharparenleft}{\kern0pt}{\isasymlambda}x{\isachardot}{\kern0pt}\ {\isasymUnion}i{\isachardot}{\kern0pt}\ v\ i\ x{\isacharparenright}{\kern0pt}\ {\isasymsubseteq}\ {\isacharparenleft}{\kern0pt}{\isasymUnion}i{\isachardot}{\kern0pt}\ eval\ f\ {\isacharparenleft}{\kern0pt}v\ i{\isacharparenright}{\kern0pt}{\isacharparenright}{\kern0pt}{\isachardoublequoteclose}\ \isakeywordONE{using}\isamarkupfalse%
\ rlexp{\isacharunderscore}{\kern0pt}cont{\isacharunderscore}{\kern0pt}aux{\isadigit{2}}\ \isakeywordONE{by}\isamarkupfalse%
\ auto\isanewline
\ \ \isakeywordONE{from}\isamarkupfalse%
\ assms\ \isakeywordTHREE{show}\isamarkupfalse%
\ {\isachardoublequoteopen}{\isacharparenleft}{\kern0pt}{\isasymUnion}i{\isachardot}{\kern0pt}\ eval\ f\ {\isacharparenleft}{\kern0pt}v\ i{\isacharparenright}{\kern0pt}{\isacharparenright}{\kern0pt}\ {\isasymsubseteq}\ eval\ f\ {\isacharparenleft}{\kern0pt}{\isasymlambda}x{\isachardot}{\kern0pt}\ {\isasymUnion}i{\isachardot}{\kern0pt}\ v\ i\ x{\isacharparenright}{\kern0pt}{\isachardoublequoteclose}\ \isakeywordONE{using}\isamarkupfalse%
\ rlexp{\isacharunderscore}{\kern0pt}cont{\isacharunderscore}{\kern0pt}aux{\isadigit{1}}\ \isakeywordONE{by}\isamarkupfalse%
\ blast\isanewline
\isakeywordONE{qed}\isamarkupfalse%
%
\endisatagproof
{\isafoldproof}%
%
\isadelimproof
%
\endisadelimproof
%
\isadelimdocument
%
\endisadelimdocument
%
\isatagdocument
%
\isamarkupsubsection{Regular language expressions which evaluate to regular languages%
}
\isamarkuptrue%
%
\endisatagdocument
{\isafolddocument}%
%
\isadelimdocument
%
\endisadelimdocument
%
\begin{isamarkuptext}%
Evaluating regular language expressions can yield non-regular languages even if
the valuation maps each variable to a regular language. This is because \isa{\isaconst{Const}} may introduce
non-regular languages.
We therefore define the following predicate which guarantees that a regular language expression
\isa{f} yields a regular language if the valuation maps all variables occurring in \isa{f} to some regular
language. This is achieved by only allowing regular languages as constants.
However, note that this predicate is just an under-approximation, i.e.\ there exist regular language
expressions which do not satisfy this predicate but evaluate to regular languages anyway.%
\end{isamarkuptext}\isamarkuptrue%
\isakeywordONE{fun}\isamarkupfalse%
\ reg{\isacharunderscore}{\kern0pt}eval\ {\isacharcolon}{\kern0pt}{\isacharcolon}{\kern0pt}\ {\isachardoublequoteopen}{\isacharprime}{\kern0pt}a\ rlexp\ {\isasymRightarrow}\ bool{\isachardoublequoteclose}\ \isakeywordTWO{where}\isanewline
\ \ {\isachardoublequoteopen}reg{\isacharunderscore}{\kern0pt}eval\ {\isacharparenleft}{\kern0pt}Var\ {\isacharunderscore}{\kern0pt}{\isacharparenright}{\kern0pt}\ {\isasymlongleftrightarrow}\ True{\isachardoublequoteclose}\ {\isacharbar}{\kern0pt}\isanewline
\ \ {\isachardoublequoteopen}reg{\isacharunderscore}{\kern0pt}eval\ {\isacharparenleft}{\kern0pt}Const\ l{\isacharparenright}{\kern0pt}\ {\isasymlongleftrightarrow}\ regular{\isacharunderscore}{\kern0pt}lang\ l{\isachardoublequoteclose}\ {\isacharbar}{\kern0pt}\isanewline
\ \ {\isachardoublequoteopen}reg{\isacharunderscore}{\kern0pt}eval\ {\isacharparenleft}{\kern0pt}Union\ f\ g{\isacharparenright}{\kern0pt}\ {\isasymlongleftrightarrow}\ reg{\isacharunderscore}{\kern0pt}eval\ f\ {\isasymand}\ reg{\isacharunderscore}{\kern0pt}eval\ g{\isachardoublequoteclose}\ {\isacharbar}{\kern0pt}\isanewline
\ \ {\isachardoublequoteopen}reg{\isacharunderscore}{\kern0pt}eval\ {\isacharparenleft}{\kern0pt}Concat\ f\ g{\isacharparenright}{\kern0pt}\ {\isasymlongleftrightarrow}\ reg{\isacharunderscore}{\kern0pt}eval\ f\ {\isasymand}\ reg{\isacharunderscore}{\kern0pt}eval\ g{\isachardoublequoteclose}\ {\isacharbar}{\kern0pt}\isanewline
\ \ {\isachardoublequoteopen}reg{\isacharunderscore}{\kern0pt}eval\ {\isacharparenleft}{\kern0pt}Star\ f{\isacharparenright}{\kern0pt}\ {\isasymlongleftrightarrow}\ reg{\isacharunderscore}{\kern0pt}eval\ f{\isachardoublequoteclose}\isanewline
\isanewline
\isanewline
\isakeywordONE{lemma}\isamarkupfalse%
\ emptyset{\isacharunderscore}{\kern0pt}regular{\isacharcolon}{\kern0pt}\ {\isachardoublequoteopen}reg{\isacharunderscore}{\kern0pt}eval\ {\isacharparenleft}{\kern0pt}Const\ {\isacharbraceleft}{\kern0pt}{\isacharbraceright}{\kern0pt}{\isacharparenright}{\kern0pt}{\isachardoublequoteclose}\isanewline
%
\isadelimproof
\ \ %
\endisadelimproof
%
\isatagproof
\isakeywordONE{using}\isamarkupfalse%
\ lang{\isachardot}{\kern0pt}simps{\isacharparenleft}{\kern0pt}{\isadigit{1}}{\isacharparenright}{\kern0pt}\ reg{\isacharunderscore}{\kern0pt}eval{\isachardot}{\kern0pt}simps{\isacharparenleft}{\kern0pt}{\isadigit{2}}{\isacharparenright}{\kern0pt}\ \isakeywordONE{by}\isamarkupfalse%
\ blast%
\endisatagproof
{\isafoldproof}%
%
\isadelimproof
\isanewline
%
\endisadelimproof
\isanewline
\isakeywordONE{lemma}\isamarkupfalse%
\ epsilon{\isacharunderscore}{\kern0pt}regular{\isacharcolon}{\kern0pt}\ {\isachardoublequoteopen}reg{\isacharunderscore}{\kern0pt}eval\ {\isacharparenleft}{\kern0pt}Const\ {\isacharbraceleft}{\kern0pt}{\isacharbrackleft}{\kern0pt}{\isacharbrackright}{\kern0pt}{\isacharbraceright}{\kern0pt}{\isacharparenright}{\kern0pt}{\isachardoublequoteclose}\isanewline
%
\isadelimproof
\ \ %
\endisadelimproof
%
\isatagproof
\isakeywordONE{using}\isamarkupfalse%
\ lang{\isachardot}{\kern0pt}simps{\isacharparenleft}{\kern0pt}{\isadigit{2}}{\isacharparenright}{\kern0pt}\ reg{\isacharunderscore}{\kern0pt}eval{\isachardot}{\kern0pt}simps{\isacharparenleft}{\kern0pt}{\isadigit{2}}{\isacharparenright}{\kern0pt}\ \isakeywordONE{by}\isamarkupfalse%
\ blast%
\endisatagproof
{\isafoldproof}%
%
\isadelimproof
%
\endisadelimproof
%
\begin{isamarkuptext}%
If the valuation \isa{v} maps all variables occurring in the regular language function \isa{f} to
a regular language, then evaluating \isa{f} again yields a regular language:%
\end{isamarkuptext}\isamarkuptrue%
\isakeywordONE{lemma}\isamarkupfalse%
\ reg{\isacharunderscore}{\kern0pt}eval{\isacharunderscore}{\kern0pt}regular{\isacharcolon}{\kern0pt}\isanewline
\ \ \isakeywordTWO{assumes}\ {\isachardoublequoteopen}reg{\isacharunderscore}{\kern0pt}eval\ f{\isachardoublequoteclose}\isanewline
\ \ \ \ \ \ \isakeywordTWO{and}\ {\isachardoublequoteopen}{\isasymAnd}n{\isachardot}{\kern0pt}\ n\ {\isasymin}\ vars\ f\ {\isasymLongrightarrow}\ regular{\isacharunderscore}{\kern0pt}lang\ {\isacharparenleft}{\kern0pt}v\ n{\isacharparenright}{\kern0pt}{\isachardoublequoteclose}\isanewline
\ \ \ \ \isakeywordTWO{shows}\ {\isachardoublequoteopen}regular{\isacharunderscore}{\kern0pt}lang\ {\isacharparenleft}{\kern0pt}eval\ f\ v{\isacharparenright}{\kern0pt}{\isachardoublequoteclose}\isanewline
%
\isadelimproof
%
\endisadelimproof
%
\isatagproof
\isakeywordONE{using}\isamarkupfalse%
\ assms\ \isakeywordONE{proof}\isamarkupfalse%
\ {\isacharparenleft}{\kern0pt}induction\ f\ rule{\isacharcolon}{\kern0pt}\ reg{\isacharunderscore}{\kern0pt}eval{\isachardot}{\kern0pt}induct{\isacharparenright}{\kern0pt}\isanewline
\ \ \isakeywordTHREE{case}\isamarkupfalse%
\ {\isacharparenleft}{\kern0pt}{\isadigit{3}}\ f\ g{\isacharparenright}{\kern0pt}\isanewline
\ \ \isakeywordONE{then}\isamarkupfalse%
\ \isakeywordTHREE{obtain}\isamarkupfalse%
\ r{\isadigit{1}}\ r{\isadigit{2}}\ \isakeywordTWO{where}\ {\isachardoublequoteopen}Regular{\isacharunderscore}{\kern0pt}Exp{\isachardot}{\kern0pt}lang\ r{\isadigit{1}}\ {\isacharequal}{\kern0pt}\ eval\ f\ v\ {\isasymand}\ Regular{\isacharunderscore}{\kern0pt}Exp{\isachardot}{\kern0pt}lang\ r{\isadigit{2}}\ {\isacharequal}{\kern0pt}\ eval\ g\ v{\isachardoublequoteclose}\ \isakeywordONE{by}\isamarkupfalse%
\ auto\isanewline
\ \ \isakeywordONE{then}\isamarkupfalse%
\ \isakeywordONE{have}\isamarkupfalse%
\ {\isachardoublequoteopen}Regular{\isacharunderscore}{\kern0pt}Exp{\isachardot}{\kern0pt}lang\ {\isacharparenleft}{\kern0pt}Plus\ r{\isadigit{1}}\ r{\isadigit{2}}{\isacharparenright}{\kern0pt}\ {\isacharequal}{\kern0pt}\ eval\ {\isacharparenleft}{\kern0pt}Union\ f\ g{\isacharparenright}{\kern0pt}\ v{\isachardoublequoteclose}\ \isakeywordONE{by}\isamarkupfalse%
\ simp\isanewline
\ \ \isakeywordONE{then}\isamarkupfalse%
\ \isakeywordTHREE{show}\isamarkupfalse%
\ {\isacharquery}{\kern0pt}case\ \isakeywordONE{by}\isamarkupfalse%
\ blast\isanewline
\isakeywordONE{next}\isamarkupfalse%
\isanewline
\ \ \isakeywordTHREE{case}\isamarkupfalse%
\ {\isacharparenleft}{\kern0pt}{\isadigit{4}}\ f\ g{\isacharparenright}{\kern0pt}\isanewline
\ \ \isakeywordONE{then}\isamarkupfalse%
\ \isakeywordTHREE{obtain}\isamarkupfalse%
\ r{\isadigit{1}}\ r{\isadigit{2}}\ \isakeywordTWO{where}\ {\isachardoublequoteopen}Regular{\isacharunderscore}{\kern0pt}Exp{\isachardot}{\kern0pt}lang\ r{\isadigit{1}}\ {\isacharequal}{\kern0pt}\ eval\ f\ v\ {\isasymand}\ Regular{\isacharunderscore}{\kern0pt}Exp{\isachardot}{\kern0pt}lang\ r{\isadigit{2}}\ {\isacharequal}{\kern0pt}\ eval\ g\ v{\isachardoublequoteclose}\ \isakeywordONE{by}\isamarkupfalse%
\ auto\isanewline
\ \ \isakeywordONE{then}\isamarkupfalse%
\ \isakeywordONE{have}\isamarkupfalse%
\ {\isachardoublequoteopen}Regular{\isacharunderscore}{\kern0pt}Exp{\isachardot}{\kern0pt}lang\ {\isacharparenleft}{\kern0pt}Times\ r{\isadigit{1}}\ r{\isadigit{2}}{\isacharparenright}{\kern0pt}\ {\isacharequal}{\kern0pt}\ eval\ {\isacharparenleft}{\kern0pt}Concat\ f\ g{\isacharparenright}{\kern0pt}\ v{\isachardoublequoteclose}\ \isakeywordONE{by}\isamarkupfalse%
\ simp\isanewline
\ \ \isakeywordONE{then}\isamarkupfalse%
\ \isakeywordTHREE{show}\isamarkupfalse%
\ {\isacharquery}{\kern0pt}case\ \isakeywordONE{by}\isamarkupfalse%
\ blast\isanewline
\isakeywordONE{next}\isamarkupfalse%
\isanewline
\ \ \isakeywordTHREE{case}\isamarkupfalse%
\ {\isacharparenleft}{\kern0pt}{\isadigit{5}}\ f{\isacharparenright}{\kern0pt}\isanewline
\ \ \isakeywordONE{then}\isamarkupfalse%
\ \isakeywordTHREE{obtain}\isamarkupfalse%
\ r\ \ \isakeywordTWO{where}\ {\isachardoublequoteopen}Regular{\isacharunderscore}{\kern0pt}Exp{\isachardot}{\kern0pt}lang\ r\ {\isacharequal}{\kern0pt}\ eval\ f\ v{\isachardoublequoteclose}\ \isakeywordONE{by}\isamarkupfalse%
\ auto\isanewline
\ \ \isakeywordONE{then}\isamarkupfalse%
\ \isakeywordONE{have}\isamarkupfalse%
\ {\isachardoublequoteopen}Regular{\isacharunderscore}{\kern0pt}Exp{\isachardot}{\kern0pt}lang\ {\isacharparenleft}{\kern0pt}Regular{\isacharunderscore}{\kern0pt}Exp{\isachardot}{\kern0pt}Star\ r{\isacharparenright}{\kern0pt}\ {\isacharequal}{\kern0pt}\ eval\ {\isacharparenleft}{\kern0pt}Star\ f{\isacharparenright}{\kern0pt}\ v{\isachardoublequoteclose}\ \isakeywordONE{by}\isamarkupfalse%
\ simp\isanewline
\ \ \isakeywordONE{then}\isamarkupfalse%
\ \isakeywordTHREE{show}\isamarkupfalse%
\ {\isacharquery}{\kern0pt}case\ \isakeywordONE{by}\isamarkupfalse%
\ blast\isanewline
\isakeywordONE{qed}\isamarkupfalse%
\ simp{\isacharunderscore}{\kern0pt}all%
\endisatagproof
{\isafoldproof}%
%
\isadelimproof
%
\endisadelimproof
%
\begin{isamarkuptext}%
A \isa{\isaconst{reg{\isacharunderscore}{\kern0pt}eval}} regular language expression stays \isa{\isaconst{reg{\isacharunderscore}{\kern0pt}eval}} if all variables are substituted
by \isa{\isaconst{reg{\isacharunderscore}{\kern0pt}eval}} regular language expressions:%
\end{isamarkuptext}\isamarkuptrue%
\isakeywordONE{lemma}\isamarkupfalse%
\ subst{\isacharunderscore}{\kern0pt}reg{\isacharunderscore}{\kern0pt}eval{\isacharcolon}{\kern0pt}\isanewline
\ \ \isakeywordTWO{assumes}\ {\isachardoublequoteopen}reg{\isacharunderscore}{\kern0pt}eval\ f{\isachardoublequoteclose}\isanewline
\ \ \ \ \ \ \isakeywordTWO{and}\ {\isachardoublequoteopen}{\isasymforall}x\ {\isasymin}\ vars\ f{\isachardot}{\kern0pt}\ reg{\isacharunderscore}{\kern0pt}eval\ {\isacharparenleft}{\kern0pt}upd\ x{\isacharparenright}{\kern0pt}{\isachardoublequoteclose}\isanewline
\ \ \ \ \isakeywordTWO{shows}\ {\isachardoublequoteopen}reg{\isacharunderscore}{\kern0pt}eval\ {\isacharparenleft}{\kern0pt}subst\ upd\ f{\isacharparenright}{\kern0pt}{\isachardoublequoteclose}\isanewline
%
\isadelimproof
\ \ %
\endisadelimproof
%
\isatagproof
\isakeywordONE{using}\isamarkupfalse%
\ assms\ \isakeywordONE{by}\isamarkupfalse%
\ {\isacharparenleft}{\kern0pt}induction\ f\ rule{\isacharcolon}{\kern0pt}\ reg{\isacharunderscore}{\kern0pt}eval{\isachardot}{\kern0pt}induct{\isacharparenright}{\kern0pt}\ simp{\isacharunderscore}{\kern0pt}all%
\endisatagproof
{\isafoldproof}%
%
\isadelimproof
\isanewline
%
\endisadelimproof
\isanewline
\isakeywordONE{lemma}\isamarkupfalse%
\ subst{\isacharunderscore}{\kern0pt}reg{\isacharunderscore}{\kern0pt}eval{\isacharunderscore}{\kern0pt}update{\isacharcolon}{\kern0pt}\isanewline
\ \ \isakeywordTWO{assumes}\ {\isachardoublequoteopen}reg{\isacharunderscore}{\kern0pt}eval\ f{\isachardoublequoteclose}\isanewline
\ \ \ \ \ \ \isakeywordTWO{and}\ {\isachardoublequoteopen}reg{\isacharunderscore}{\kern0pt}eval\ g{\isachardoublequoteclose}\isanewline
\ \ \ \ \isakeywordTWO{shows}\ {\isachardoublequoteopen}reg{\isacharunderscore}{\kern0pt}eval\ {\isacharparenleft}{\kern0pt}subst\ {\isacharparenleft}{\kern0pt}Var{\isacharparenleft}{\kern0pt}x\ {\isacharcolon}{\kern0pt}{\isacharequal}{\kern0pt}\ g{\isacharparenright}{\kern0pt}{\isacharparenright}{\kern0pt}\ f{\isacharparenright}{\kern0pt}{\isachardoublequoteclose}\isanewline
%
\isadelimproof
\ \ %
\endisadelimproof
%
\isatagproof
\isakeywordONE{using}\isamarkupfalse%
\ assms\ subst{\isacharunderscore}{\kern0pt}reg{\isacharunderscore}{\kern0pt}eval\ fun{\isacharunderscore}{\kern0pt}upd{\isacharunderscore}{\kern0pt}def\ \isakeywordONE{by}\isamarkupfalse%
\ {\isacharparenleft}{\kern0pt}metis\ reg{\isacharunderscore}{\kern0pt}eval{\isachardot}{\kern0pt}simps{\isacharparenleft}{\kern0pt}{\isadigit{1}}{\isacharparenright}{\kern0pt}{\isacharparenright}{\kern0pt}%
\endisatagproof
{\isafoldproof}%
%
\isadelimproof
%
\endisadelimproof
%
\begin{isamarkuptext}%
For any finite union of \isa{\isaconst{reg{\isacharunderscore}{\kern0pt}eval}} regular language expressions exists a \isa{\isaconst{reg{\isacharunderscore}{\kern0pt}eval}} regular
language expression:%
\end{isamarkuptext}\isamarkuptrue%
\isakeywordONE{lemma}\isamarkupfalse%
\ finite{\isacharunderscore}{\kern0pt}Union{\isacharunderscore}{\kern0pt}regular{\isacharunderscore}{\kern0pt}aux{\isacharcolon}{\kern0pt}\isanewline
\ \ {\isachardoublequoteopen}{\isasymforall}f\ {\isasymin}\ set\ fs{\isachardot}{\kern0pt}\ reg{\isacharunderscore}{\kern0pt}eval\ f\ {\isasymLongrightarrow}\ {\isasymexists}g{\isachardot}{\kern0pt}\ reg{\isacharunderscore}{\kern0pt}eval\ g\ {\isasymand}\ {\isasymUnion}{\isacharparenleft}{\kern0pt}vars\ {\isacharbackquote}{\kern0pt}\ set\ fs{\isacharparenright}{\kern0pt}\ {\isacharequal}{\kern0pt}\ vars\ g\isanewline
\ \ \ \ \ \ \ \ \ \ \ \ \ \ \ \ \ \ \ \ \ \ \ \ \ \ \ \ \ \ \ \ \ \ \ \ \ \ {\isasymand}\ {\isacharparenleft}{\kern0pt}{\isasymforall}v{\isachardot}{\kern0pt}\ {\isacharparenleft}{\kern0pt}{\isasymUnion}f\ {\isasymin}\ set\ fs{\isachardot}{\kern0pt}\ eval\ f\ v{\isacharparenright}{\kern0pt}\ {\isacharequal}{\kern0pt}\ eval\ g\ v{\isacharparenright}{\kern0pt}{\isachardoublequoteclose}\isanewline
%
\isadelimproof
%
\endisadelimproof
%
\isatagproof
\isakeywordONE{proof}\isamarkupfalse%
\ {\isacharparenleft}{\kern0pt}induction\ fs{\isacharparenright}{\kern0pt}\isanewline
\ \ \isakeywordTHREE{case}\isamarkupfalse%
\ Nil\isanewline
\ \ \isakeywordONE{then}\isamarkupfalse%
\ \isakeywordTHREE{show}\isamarkupfalse%
\ {\isacharquery}{\kern0pt}case\ \isakeywordONE{using}\isamarkupfalse%
\ emptyset{\isacharunderscore}{\kern0pt}regular\ \isakeywordONE{by}\isamarkupfalse%
\ fastforce\isanewline
\isakeywordONE{next}\isamarkupfalse%
\isanewline
\ \ \isakeywordTHREE{case}\isamarkupfalse%
\ {\isacharparenleft}{\kern0pt}Cons\ f{\isadigit{1}}\ fs{\isacharparenright}{\kern0pt}\isanewline
\ \ \isakeywordONE{then}\isamarkupfalse%
\ \isakeywordTHREE{obtain}\isamarkupfalse%
\ g\ \isakeywordTWO{where}\ {\isacharasterisk}{\kern0pt}{\isacharcolon}{\kern0pt}\ {\isachardoublequoteopen}reg{\isacharunderscore}{\kern0pt}eval\ g\ {\isasymand}\ {\isasymUnion}{\isacharparenleft}{\kern0pt}vars\ {\isacharbackquote}{\kern0pt}\ set\ fs{\isacharparenright}{\kern0pt}\ {\isacharequal}{\kern0pt}\ vars\ g\isanewline
\ \ \ \ \ \ \ \ \ \ \ \ \ \ \ \ \ \ \ \ \ \ \ \ \ \ {\isasymand}\ {\isacharparenleft}{\kern0pt}{\isasymforall}v{\isachardot}{\kern0pt}\ {\isacharparenleft}{\kern0pt}{\isasymUnion}f{\isasymin}set\ fs{\isachardot}{\kern0pt}\ eval\ f\ v{\isacharparenright}{\kern0pt}\ {\isacharequal}{\kern0pt}\ eval\ g\ v{\isacharparenright}{\kern0pt}{\isachardoublequoteclose}\ \isakeywordONE{by}\isamarkupfalse%
\ auto\isanewline
\ \ \isakeywordONE{let}\isamarkupfalse%
\ {\isacharquery}{\kern0pt}g{\isacharprime}{\kern0pt}\ {\isacharequal}{\kern0pt}\ {\isachardoublequoteopen}Union\ f{\isadigit{1}}\ g{\isachardoublequoteclose}\isanewline
\ \ \isakeywordONE{from}\isamarkupfalse%
\ Cons{\isachardot}{\kern0pt}prems\ {\isacharasterisk}{\kern0pt}\ \isakeywordONE{have}\isamarkupfalse%
\ {\isachardoublequoteopen}reg{\isacharunderscore}{\kern0pt}eval\ {\isacharquery}{\kern0pt}g{\isacharprime}{\kern0pt}\ {\isasymand}\ {\isasymUnion}\ {\isacharparenleft}{\kern0pt}vars\ {\isacharbackquote}{\kern0pt}\ set\ {\isacharparenleft}{\kern0pt}f{\isadigit{1}}\ {\isacharhash}{\kern0pt}\ fs{\isacharparenright}{\kern0pt}{\isacharparenright}{\kern0pt}\ {\isacharequal}{\kern0pt}\ vars\ {\isacharquery}{\kern0pt}g{\isacharprime}{\kern0pt}\isanewline
\ \ \ \ \ \ {\isasymand}\ {\isacharparenleft}{\kern0pt}{\isasymforall}v{\isachardot}{\kern0pt}\ {\isacharparenleft}{\kern0pt}{\isasymUnion}f{\isasymin}set\ {\isacharparenleft}{\kern0pt}f{\isadigit{1}}\ {\isacharhash}{\kern0pt}\ fs{\isacharparenright}{\kern0pt}{\isachardot}{\kern0pt}\ eval\ f\ v{\isacharparenright}{\kern0pt}\ {\isacharequal}{\kern0pt}\ eval\ {\isacharquery}{\kern0pt}g{\isacharprime}{\kern0pt}\ v{\isacharparenright}{\kern0pt}{\isachardoublequoteclose}\ \isakeywordONE{by}\isamarkupfalse%
\ simp\isanewline
\ \ \isakeywordONE{then}\isamarkupfalse%
\ \isakeywordTHREE{show}\isamarkupfalse%
\ {\isacharquery}{\kern0pt}case\ \isakeywordONE{by}\isamarkupfalse%
\ blast\isanewline
\isakeywordONE{qed}\isamarkupfalse%
%
\endisatagproof
{\isafoldproof}%
%
\isadelimproof
\isanewline
%
\endisadelimproof
\isanewline
\isakeywordONE{lemma}\isamarkupfalse%
\ finite{\isacharunderscore}{\kern0pt}Union{\isacharunderscore}{\kern0pt}regular{\isacharcolon}{\kern0pt}\isanewline
\ \ \isakeywordTWO{assumes}\ {\isachardoublequoteopen}finite\ F{\isachardoublequoteclose}\isanewline
\ \ \ \ \ \ \isakeywordTWO{and}\ {\isachardoublequoteopen}{\isasymforall}f\ {\isasymin}\ F{\isachardot}{\kern0pt}\ reg{\isacharunderscore}{\kern0pt}eval\ f{\isachardoublequoteclose}\isanewline
\ \ \ \ \isakeywordTWO{shows}\ {\isachardoublequoteopen}{\isasymexists}g{\isachardot}{\kern0pt}\ reg{\isacharunderscore}{\kern0pt}eval\ g\ {\isasymand}\ {\isasymUnion}{\isacharparenleft}{\kern0pt}vars\ {\isacharbackquote}{\kern0pt}\ F{\isacharparenright}{\kern0pt}\ {\isacharequal}{\kern0pt}\ vars\ g\ {\isasymand}\ {\isacharparenleft}{\kern0pt}{\isasymforall}v{\isachardot}{\kern0pt}\ {\isacharparenleft}{\kern0pt}{\isasymUnion}f{\isasymin}F{\isachardot}{\kern0pt}\ eval\ f\ v{\isacharparenright}{\kern0pt}\ {\isacharequal}{\kern0pt}\ eval\ g\ v{\isacharparenright}{\kern0pt}{\isachardoublequoteclose}\isanewline
%
\isadelimproof
\ \ %
\endisadelimproof
%
\isatagproof
\isakeywordONE{using}\isamarkupfalse%
\ assms\ finite{\isacharunderscore}{\kern0pt}Union{\isacharunderscore}{\kern0pt}regular{\isacharunderscore}{\kern0pt}aux\ finite{\isacharunderscore}{\kern0pt}list\ \isakeywordONE{by}\isamarkupfalse%
\ metis%
\endisatagproof
{\isafoldproof}%
%
\isadelimproof
%
\endisadelimproof
%
\isadelimdocument
%
\endisadelimdocument
%
\isatagdocument
%
\isamarkupsubsection{Constant regular language functions%
}
\isamarkuptrue%
%
\endisatagdocument
{\isafolddocument}%
%
\isadelimdocument
%
\endisadelimdocument
%
\begin{isamarkuptext}%
We call a regular language expression constant if it contains no variables. A constant
regular language expression always evaluates to the same language, independent on the valuation.
Thus, if the constant regular language expression is \isa{\isaconst{reg{\isacharunderscore}{\kern0pt}eval}}, then it evaluates to some
regular language, independent on the valuation.%
\end{isamarkuptext}\isamarkuptrue%
\isakeywordONE{abbreviation}\isamarkupfalse%
\ const{\isacharunderscore}{\kern0pt}rlexp\ {\isacharcolon}{\kern0pt}{\isacharcolon}{\kern0pt}\ {\isachardoublequoteopen}{\isacharprime}{\kern0pt}a\ rlexp\ {\isasymRightarrow}\ bool{\isachardoublequoteclose}\ \isakeywordTWO{where}\isanewline
\ \ {\isachardoublequoteopen}const{\isacharunderscore}{\kern0pt}rlexp\ f\ {\isasymequiv}\ vars\ f\ {\isacharequal}{\kern0pt}\ {\isacharbraceleft}{\kern0pt}{\isacharbraceright}{\kern0pt}{\isachardoublequoteclose}\isanewline
\isanewline
\isakeywordONE{lemma}\isamarkupfalse%
\ const{\isacharunderscore}{\kern0pt}rlexp{\isacharunderscore}{\kern0pt}lang{\isacharcolon}{\kern0pt}\ {\isachardoublequoteopen}const{\isacharunderscore}{\kern0pt}rlexp\ f\ {\isasymLongrightarrow}\ {\isasymexists}l{\isachardot}{\kern0pt}\ {\isasymforall}v{\isachardot}{\kern0pt}\ eval\ f\ v\ {\isacharequal}{\kern0pt}\ l{\isachardoublequoteclose}\isanewline
%
\isadelimproof
\ \ %
\endisadelimproof
%
\isatagproof
\isakeywordONE{by}\isamarkupfalse%
\ {\isacharparenleft}{\kern0pt}induction\ f{\isacharparenright}{\kern0pt}\ auto%
\endisatagproof
{\isafoldproof}%
%
\isadelimproof
\isanewline
%
\endisadelimproof
\isanewline
\isakeywordONE{lemma}\isamarkupfalse%
\ const{\isacharunderscore}{\kern0pt}rlexp{\isacharunderscore}{\kern0pt}regular{\isacharunderscore}{\kern0pt}lang{\isacharcolon}{\kern0pt}\isanewline
\ \ \isakeywordTWO{assumes}\ {\isachardoublequoteopen}const{\isacharunderscore}{\kern0pt}rlexp\ f{\isachardoublequoteclose}\isanewline
\ \ \ \ \ \ \isakeywordTWO{and}\ {\isachardoublequoteopen}reg{\isacharunderscore}{\kern0pt}eval\ f{\isachardoublequoteclose}\isanewline
\ \ \ \ \isakeywordTWO{shows}\ {\isachardoublequoteopen}{\isasymexists}l{\isachardot}{\kern0pt}\ regular{\isacharunderscore}{\kern0pt}lang\ l\ {\isasymand}\ {\isacharparenleft}{\kern0pt}{\isasymforall}v{\isachardot}{\kern0pt}\ eval\ f\ v\ {\isacharequal}{\kern0pt}\ l{\isacharparenright}{\kern0pt}{\isachardoublequoteclose}\isanewline
%
\isadelimproof
\ \ %
\endisadelimproof
%
\isatagproof
\isakeywordONE{using}\isamarkupfalse%
\ assms\ const{\isacharunderscore}{\kern0pt}rlexp{\isacharunderscore}{\kern0pt}lang\ reg{\isacharunderscore}{\kern0pt}eval{\isacharunderscore}{\kern0pt}regular\ \isakeywordONE{by}\isamarkupfalse%
\ fastforce%
\endisatagproof
{\isafoldproof}%
%
\isadelimproof
\isanewline
%
\endisadelimproof
%
\isadelimtheory
\isanewline
%
\endisadelimtheory
%
\isatagtheory
\isakeywordTWO{end}\isamarkupfalse%
%
\endisatagtheory
{\isafoldtheory}%
%
\isadelimtheory
%
\endisadelimtheory
%
\end{isabellebody}%
\endinput
%:%file=~/studium/semester_7/semantik/homeworks/AIST/Parikh/Reg_Lang_Exp.thy%:%
%:%11=3%:%
%:%27=5%:%
%:%28=5%:%
%:%29=6%:%
%:%30=7%:%
%:%31=8%:%
%:%45=11%:%
%:%57=13%:%
%:%58=14%:%
%:%59=15%:%
%:%60=16%:%
%:%61=17%:%
%:%63=18%:%
%:%64=18%:%
%:%65=19%:%
%:%66=20%:%
%:%67=21%:%
%:%68=22%:%
%:%69=23%:%
%:%70=24%:%
%:%71=24%:%
%:%72=25%:%
%:%73=26%:%
%:%74=26%:%
%:%75=27%:%
%:%76=28%:%
%:%77=29%:%
%:%78=30%:%
%:%79=31%:%
%:%80=32%:%
%:%81=33%:%
%:%82=33%:%
%:%83=34%:%
%:%84=35%:%
%:%85=36%:%
%:%86=37%:%
%:%87=38%:%
%:%89=40%:%
%:%90=41%:%
%:%92=42%:%
%:%93=42%:%
%:%94=43%:%
%:%95=44%:%
%:%96=45%:%
%:%97=46%:%
%:%98=47%:%
%:%105=51%:%
%:%115=53%:%
%:%116=53%:%
%:%117=54%:%
%:%118=55%:%
%:%121=56%:%
%:%125=56%:%
%:%126=56%:%
%:%131=56%:%
%:%134=57%:%
%:%135=58%:%
%:%136=58%:%
%:%137=59%:%
%:%140=60%:%
%:%144=60%:%
%:%145=60%:%
%:%146=60%:%
%:%151=60%:%
%:%154=61%:%
%:%155=62%:%
%:%156=62%:%
%:%159=63%:%
%:%163=63%:%
%:%164=63%:%
%:%165=63%:%
%:%170=63%:%
%:%173=64%:%
%:%174=65%:%
%:%175=65%:%
%:%178=66%:%
%:%182=66%:%
%:%183=66%:%
%:%188=66%:%
%:%191=67%:%
%:%192=68%:%
%:%193=68%:%
%:%200=69%:%
%:%201=69%:%
%:%202=70%:%
%:%203=70%:%
%:%204=71%:%
%:%205=71%:%
%:%206=72%:%
%:%207=72%:%
%:%208=73%:%
%:%209=73%:%
%:%210=73%:%
%:%211=73%:%
%:%212=73%:%
%:%213=74%:%
%:%214=74%:%
%:%215=74%:%
%:%216=74%:%
%:%217=75%:%
%:%223=75%:%
%:%226=76%:%
%:%227=77%:%
%:%228=78%:%
%:%229=78%:%
%:%230=79%:%
%:%231=80%:%
%:%234=81%:%
%:%238=81%:%
%:%239=81%:%
%:%240=81%:%
%:%245=81%:%
%:%248=82%:%
%:%249=83%:%
%:%250=83%:%
%:%251=84%:%
%:%252=85%:%
%:%259=86%:%
%:%260=86%:%
%:%261=87%:%
%:%262=87%:%
%:%263=88%:%
%:%264=88%:%
%:%265=89%:%
%:%266=89%:%
%:%267=89%:%
%:%268=89%:%
%:%269=90%:%
%:%270=90%:%
%:%271=90%:%
%:%272=90%:%
%:%273=91%:%
%:%274=91%:%
%:%275=91%:%
%:%276=91%:%
%:%277=91%:%
%:%278=92%:%
%:%279=92%:%
%:%280=92%:%
%:%281=93%:%
%:%282=93%:%
%:%283=93%:%
%:%284=94%:%
%:%285=94%:%
%:%286=94%:%
%:%287=94%:%
%:%288=95%:%
%:%298=98%:%
%:%300=99%:%
%:%301=99%:%
%:%302=100%:%
%:%303=101%:%
%:%310=102%:%
%:%311=102%:%
%:%312=102%:%
%:%313=103%:%
%:%314=103%:%
%:%315=104%:%
%:%316=104%:%
%:%317=104%:%
%:%318=105%:%
%:%319=105%:%
%:%320=106%:%
%:%321=106%:%
%:%335=110%:%
%:%345=112%:%
%:%346=112%:%
%:%347=113%:%
%:%348=114%:%
%:%349=115%:%
%:%352=116%:%
%:%356=116%:%
%:%357=116%:%
%:%362=116%:%
%:%365=117%:%
%:%366=118%:%
%:%367=118%:%
%:%368=119%:%
%:%369=120%:%
%:%370=121%:%
%:%377=122%:%
%:%378=122%:%
%:%379=123%:%
%:%380=123%:%
%:%381=123%:%
%:%382=123%:%
%:%383=124%:%
%:%384=124%:%
%:%385=124%:%
%:%386=125%:%
%:%387=125%:%
%:%388=125%:%
%:%389=126%:%
%:%390=126%:%
%:%391=126%:%
%:%392=127%:%
%:%398=127%:%
%:%401=128%:%
%:%402=129%:%
%:%403=129%:%
%:%404=130%:%
%:%405=131%:%
%:%406=132%:%
%:%413=133%:%
%:%414=133%:%
%:%415=133%:%
%:%416=134%:%
%:%417=134%:%
%:%418=135%:%
%:%419=135%:%
%:%420=135%:%
%:%421=135%:%
%:%422=136%:%
%:%423=136%:%
%:%424=137%:%
%:%425=137%:%
%:%426=138%:%
%:%427=138%:%
%:%428=138%:%
%:%429=139%:%
%:%430=139%:%
%:%431=140%:%
%:%432=140%:%
%:%433=140%:%
%:%434=140%:%
%:%435=141%:%
%:%436=141%:%
%:%437=141%:%
%:%438=141%:%
%:%439=142%:%
%:%440=142%:%
%:%441=143%:%
%:%442=143%:%
%:%443=143%:%
%:%444=144%:%
%:%445=144%:%
%:%446=145%:%
%:%447=145%:%
%:%448=145%:%
%:%449=146%:%
%:%450=146%:%
%:%451=147%:%
%:%452=147%:%
%:%453=147%:%
%:%454=147%:%
%:%455=148%:%
%:%456=148%:%
%:%457=148%:%
%:%458=148%:%
%:%459=148%:%
%:%460=149%:%
%:%466=149%:%
%:%469=150%:%
%:%470=151%:%
%:%471=151%:%
%:%472=152%:%
%:%473=153%:%
%:%474=154%:%
%:%481=155%:%
%:%482=155%:%
%:%483=155%:%
%:%484=156%:%
%:%485=156%:%
%:%486=157%:%
%:%487=157%:%
%:%488=157%:%
%:%489=158%:%
%:%490=158%:%
%:%491=159%:%
%:%492=159%:%
%:%493=159%:%
%:%494=159%:%
%:%495=160%:%
%:%496=160%:%
%:%497=160%:%
%:%498=160%:%
%:%499=161%:%
%:%500=161%:%
%:%501=162%:%
%:%502=162%:%
%:%503=162%:%
%:%504=163%:%
%:%505=163%:%
%:%506=164%:%
%:%507=164%:%
%:%508=164%:%
%:%509=164%:%
%:%510=165%:%
%:%511=165%:%
%:%512=166%:%
%:%513=166%:%
%:%514=166%:%
%:%515=166%:%
%:%516=166%:%
%:%517=167%:%
%:%518=167%:%
%:%519=168%:%
%:%520=168%:%
%:%521=169%:%
%:%522=169%:%
%:%523=169%:%
%:%524=170%:%
%:%525=170%:%
%:%526=170%:%
%:%527=171%:%
%:%528=171%:%
%:%529=171%:%
%:%530=171%:%
%:%531=171%:%
%:%532=172%:%
%:%533=172%:%
%:%534=172%:%
%:%535=172%:%
%:%536=172%:%
%:%537=173%:%
%:%538=173%:%
%:%539=173%:%
%:%540=173%:%
%:%541=174%:%
%:%542=174%:%
%:%551=176%:%
%:%552=177%:%
%:%554=179%:%
%:%555=179%:%
%:%556=180%:%
%:%557=181%:%
%:%564=182%:%
%:%565=182%:%
%:%566=183%:%
%:%567=183%:%
%:%568=183%:%
%:%569=183%:%
%:%570=183%:%
%:%571=184%:%
%:%572=184%:%
%:%573=184%:%
%:%574=184%:%
%:%575=184%:%
%:%576=185%:%
%:%591=189%:%
%:%603=191%:%
%:%604=192%:%
%:%605=193%:%
%:%606=194%:%
%:%607=195%:%
%:%608=196%:%
%:%609=197%:%
%:%610=198%:%
%:%612=200%:%
%:%613=200%:%
%:%614=201%:%
%:%615=202%:%
%:%616=203%:%
%:%617=204%:%
%:%618=205%:%
%:%619=206%:%
%:%620=207%:%
%:%621=208%:%
%:%622=208%:%
%:%625=209%:%
%:%629=209%:%
%:%630=209%:%
%:%631=209%:%
%:%636=209%:%
%:%639=210%:%
%:%640=211%:%
%:%641=211%:%
%:%644=212%:%
%:%648=212%:%
%:%649=212%:%
%:%650=212%:%
%:%659=215%:%
%:%660=216%:%
%:%662=217%:%
%:%663=217%:%
%:%664=218%:%
%:%665=219%:%
%:%666=220%:%
%:%673=221%:%
%:%674=221%:%
%:%675=221%:%
%:%676=222%:%
%:%677=222%:%
%:%678=223%:%
%:%679=223%:%
%:%680=223%:%
%:%681=223%:%
%:%682=224%:%
%:%683=224%:%
%:%684=224%:%
%:%685=224%:%
%:%686=225%:%
%:%687=225%:%
%:%688=225%:%
%:%689=225%:%
%:%690=226%:%
%:%691=226%:%
%:%692=227%:%
%:%693=227%:%
%:%694=228%:%
%:%695=228%:%
%:%696=228%:%
%:%697=228%:%
%:%698=229%:%
%:%699=229%:%
%:%700=229%:%
%:%701=229%:%
%:%702=230%:%
%:%703=230%:%
%:%704=230%:%
%:%705=230%:%
%:%706=231%:%
%:%707=231%:%
%:%708=232%:%
%:%709=232%:%
%:%710=233%:%
%:%711=233%:%
%:%712=233%:%
%:%713=233%:%
%:%714=234%:%
%:%715=234%:%
%:%716=234%:%
%:%717=234%:%
%:%718=235%:%
%:%719=235%:%
%:%720=235%:%
%:%721=235%:%
%:%722=236%:%
%:%723=236%:%
%:%732=239%:%
%:%733=240%:%
%:%735=241%:%
%:%736=241%:%
%:%737=242%:%
%:%738=243%:%
%:%739=244%:%
%:%742=245%:%
%:%746=245%:%
%:%747=245%:%
%:%748=245%:%
%:%753=245%:%
%:%756=246%:%
%:%757=247%:%
%:%758=247%:%
%:%759=248%:%
%:%760=249%:%
%:%761=250%:%
%:%764=251%:%
%:%768=251%:%
%:%769=251%:%
%:%770=251%:%
%:%779=254%:%
%:%780=255%:%
%:%782=256%:%
%:%783=256%:%
%:%784=257%:%
%:%785=258%:%
%:%792=259%:%
%:%793=259%:%
%:%794=260%:%
%:%795=260%:%
%:%796=261%:%
%:%797=261%:%
%:%798=261%:%
%:%799=261%:%
%:%800=261%:%
%:%801=262%:%
%:%802=262%:%
%:%803=263%:%
%:%804=263%:%
%:%805=264%:%
%:%806=264%:%
%:%807=264%:%
%:%808=265%:%
%:%809=265%:%
%:%810=266%:%
%:%811=266%:%
%:%812=267%:%
%:%813=267%:%
%:%814=267%:%
%:%815=268%:%
%:%816=268%:%
%:%817=269%:%
%:%818=269%:%
%:%819=269%:%
%:%820=269%:%
%:%821=270%:%
%:%827=270%:%
%:%830=271%:%
%:%831=272%:%
%:%832=272%:%
%:%833=273%:%
%:%834=274%:%
%:%835=275%:%
%:%838=276%:%
%:%842=276%:%
%:%843=276%:%
%:%844=276%:%
%:%858=280%:%
%:%870=282%:%
%:%871=283%:%
%:%872=284%:%
%:%873=285%:%
%:%875=287%:%
%:%876=287%:%
%:%877=288%:%
%:%878=289%:%
%:%879=290%:%
%:%880=290%:%
%:%883=291%:%
%:%887=291%:%
%:%888=291%:%
%:%893=291%:%
%:%896=292%:%
%:%897=293%:%
%:%898=293%:%
%:%899=294%:%
%:%900=295%:%
%:%901=296%:%
%:%904=297%:%
%:%908=297%:%
%:%909=297%:%
%:%910=297%:%
%:%915=297%:%
%:%920=298%:%
%:%925=299%:%

%
\begin{isabellebody}%
\setisabellecontext{Parikh{\isacharunderscore}{\kern0pt}Img}%
%
\isadelimdocument
%
\endisadelimdocument
%
\isatagdocument
%
\isamarkupsection{Parikh images%
}
\isamarkuptrue%
%
\endisatagdocument
{\isafolddocument}%
%
\isadelimdocument
%
\endisadelimdocument
%
\isadelimtheory
%
\endisadelimtheory
%
\isatagtheory
\isakeywordONE{theory}\isamarkupfalse%
\ Parikh{\isacharunderscore}{\kern0pt}Img\isanewline
\ \ \isakeywordTWO{imports}\ \isanewline
\ \ \ \ {\isachardoublequoteopen}Reg{\isacharunderscore}{\kern0pt}Lang{\isacharunderscore}{\kern0pt}Exp{\isachardoublequoteclose}\isanewline
\ \ \ \ {\isachardoublequoteopen}HOL{\isacharminus}{\kern0pt}Library{\isachardot}{\kern0pt}Multiset{\isachardoublequoteclose}\isanewline
\isakeywordTWO{begin}%
\endisatagtheory
{\isafoldtheory}%
%
\isadelimtheory
%
\endisadelimtheory
%
\isadelimdocument
%
\endisadelimdocument
%
\isatagdocument
%
\isamarkupsubsection{Definition and basic lemmas%
}
\isamarkuptrue%
%
\endisatagdocument
{\isafolddocument}%
%
\isadelimdocument
%
\endisadelimdocument
%
\begin{isamarkuptext}%
The Parikh vector of a finite word describes how often each symbol of the alphabet occurs in the word.
We represent parikh vectors by multisets. The Parikh image of a language \isa{L}, denoted by \isa{{\isasymPsi}\ L},
is then the set of Parikh vectors of all words in the language.%
\end{isamarkuptext}\isamarkuptrue%
\isakeywordONE{abbreviation}\isamarkupfalse%
\ parikh{\isacharunderscore}{\kern0pt}vec\ \isakeywordTWO{where}\isanewline
\ \ {\isachardoublequoteopen}parikh{\isacharunderscore}{\kern0pt}vec\ {\isasymequiv}\ mset{\isachardoublequoteclose}\isanewline
\isanewline
\isakeywordONE{definition}\isamarkupfalse%
\ parikh{\isacharunderscore}{\kern0pt}img\ {\isacharcolon}{\kern0pt}{\isacharcolon}{\kern0pt}\ {\isachardoublequoteopen}{\isacharprime}{\kern0pt}a\ lang\ {\isasymRightarrow}\ {\isacharprime}{\kern0pt}a\ multiset\ set{\isachardoublequoteclose}\ {\isacharparenleft}{\kern0pt}{\isachardoublequoteopen}{\isasymPsi}{\isachardoublequoteclose}{\isacharparenright}{\kern0pt}\ \isakeywordTWO{where}\isanewline
\ \ {\isachardoublequoteopen}{\isasymPsi}\ L\ {\isasymequiv}\ parikh{\isacharunderscore}{\kern0pt}vec\ {\isacharbackquote}{\kern0pt}\ L{\isachardoublequoteclose}\isanewline
\isanewline
\isakeywordONE{lemma}\isamarkupfalse%
\ parikh{\isacharunderscore}{\kern0pt}img{\isacharunderscore}{\kern0pt}Un\ {\isacharbrackleft}{\kern0pt}simp{\isacharbrackright}{\kern0pt}{\isacharcolon}{\kern0pt}\ {\isachardoublequoteopen}{\isasymPsi}\ {\isacharparenleft}{\kern0pt}L{\isadigit{1}}\ {\isasymunion}\ L{\isadigit{2}}{\isacharparenright}{\kern0pt}\ {\isacharequal}{\kern0pt}\ {\isasymPsi}\ L{\isadigit{1}}\ {\isasymunion}\ {\isasymPsi}\ L{\isadigit{2}}{\isachardoublequoteclose}\isanewline
%
\isadelimproof
\ \ %
\endisadelimproof
%
\isatagproof
\isakeywordONE{by}\isamarkupfalse%
\ {\isacharparenleft}{\kern0pt}auto\ simp\ add{\isacharcolon}{\kern0pt}\ parikh{\isacharunderscore}{\kern0pt}img{\isacharunderscore}{\kern0pt}def{\isacharparenright}{\kern0pt}%
\endisatagproof
{\isafoldproof}%
%
\isadelimproof
\isanewline
%
\endisadelimproof
\isanewline
\isakeywordONE{lemma}\isamarkupfalse%
\ parikh{\isacharunderscore}{\kern0pt}img{\isacharunderscore}{\kern0pt}UNION{\isacharcolon}{\kern0pt}\ {\isachardoublequoteopen}{\isasymPsi}\ {\isacharparenleft}{\kern0pt}{\isasymUnion}{\isacharparenleft}{\kern0pt}L\ {\isacharbackquote}{\kern0pt}\ I{\isacharparenright}{\kern0pt}{\isacharparenright}{\kern0pt}\ {\isacharequal}{\kern0pt}\ {\isasymUnion}\ {\isacharparenleft}{\kern0pt}{\isacharparenleft}{\kern0pt}{\isasymlambda}i{\isachardot}{\kern0pt}\ {\isasymPsi}\ {\isacharparenleft}{\kern0pt}L\ i{\isacharparenright}{\kern0pt}{\isacharparenright}{\kern0pt}\ {\isacharbackquote}{\kern0pt}\ I{\isacharparenright}{\kern0pt}{\isachardoublequoteclose}\isanewline
%
\isadelimproof
\ \ %
\endisadelimproof
%
\isatagproof
\isakeywordONE{by}\isamarkupfalse%
\ {\isacharparenleft}{\kern0pt}auto\ simp\ add{\isacharcolon}{\kern0pt}\ parikh{\isacharunderscore}{\kern0pt}img{\isacharunderscore}{\kern0pt}def{\isacharparenright}{\kern0pt}%
\endisatagproof
{\isafoldproof}%
%
\isadelimproof
\isanewline
%
\endisadelimproof
\isanewline
\isakeywordONE{lemma}\isamarkupfalse%
\ parikh{\isacharunderscore}{\kern0pt}img{\isacharunderscore}{\kern0pt}conc{\isacharcolon}{\kern0pt}\ {\isachardoublequoteopen}{\isasymPsi}\ {\isacharparenleft}{\kern0pt}L{\isadigit{1}}\ {\isacharat}{\kern0pt}{\isacharat}{\kern0pt}\ L{\isadigit{2}}{\isacharparenright}{\kern0pt}\ {\isacharequal}{\kern0pt}\ {\isacharbraceleft}{\kern0pt}\ m{\isadigit{1}}\ {\isacharplus}{\kern0pt}\ m{\isadigit{2}}\ {\isacharbar}{\kern0pt}\ m{\isadigit{1}}\ m{\isadigit{2}}{\isachardot}{\kern0pt}\ m{\isadigit{1}}\ {\isasymin}\ {\isasymPsi}\ L{\isadigit{1}}\ {\isasymand}\ m{\isadigit{2}}\ {\isasymin}\ {\isasymPsi}\ L{\isadigit{2}}\ {\isacharbraceright}{\kern0pt}{\isachardoublequoteclose}\isanewline
%
\isadelimproof
\ \ %
\endisadelimproof
%
\isatagproof
\isakeywordONE{unfolding}\isamarkupfalse%
\ parikh{\isacharunderscore}{\kern0pt}img{\isacharunderscore}{\kern0pt}def\ \isakeywordONE{by}\isamarkupfalse%
\ force%
\endisatagproof
{\isafoldproof}%
%
\isadelimproof
\isanewline
%
\endisadelimproof
\isanewline
\isakeywordONE{lemma}\isamarkupfalse%
\ parikh{\isacharunderscore}{\kern0pt}img{\isacharunderscore}{\kern0pt}commut{\isacharcolon}{\kern0pt}\ {\isachardoublequoteopen}{\isasymPsi}\ {\isacharparenleft}{\kern0pt}L{\isadigit{1}}\ {\isacharat}{\kern0pt}{\isacharat}{\kern0pt}\ L{\isadigit{2}}{\isacharparenright}{\kern0pt}\ {\isacharequal}{\kern0pt}\ {\isasymPsi}\ {\isacharparenleft}{\kern0pt}L{\isadigit{2}}\ {\isacharat}{\kern0pt}{\isacharat}{\kern0pt}\ L{\isadigit{1}}{\isacharparenright}{\kern0pt}{\isachardoublequoteclose}\isanewline
%
\isadelimproof
%
\endisadelimproof
%
\isatagproof
\isakeywordONE{proof}\isamarkupfalse%
\ {\isacharminus}{\kern0pt}\isanewline
\ \ \isakeywordONE{have}\isamarkupfalse%
\ {\isachardoublequoteopen}{\isacharbraceleft}{\kern0pt}\ m{\isadigit{1}}\ {\isacharplus}{\kern0pt}\ m{\isadigit{2}}\ {\isacharbar}{\kern0pt}\ m{\isadigit{1}}\ m{\isadigit{2}}{\isachardot}{\kern0pt}\ m{\isadigit{1}}\ {\isasymin}\ {\isasymPsi}\ L{\isadigit{1}}\ {\isasymand}\ m{\isadigit{2}}\ {\isasymin}\ {\isasymPsi}\ L{\isadigit{2}}\ {\isacharbraceright}{\kern0pt}\ {\isacharequal}{\kern0pt}\ \isanewline
\ \ \ \ \ \ \ \ {\isacharbraceleft}{\kern0pt}\ m{\isadigit{2}}\ {\isacharplus}{\kern0pt}\ m{\isadigit{1}}\ {\isacharbar}{\kern0pt}\ m{\isadigit{1}}\ m{\isadigit{2}}{\isachardot}{\kern0pt}\ m{\isadigit{1}}\ {\isasymin}\ {\isasymPsi}\ L{\isadigit{1}}\ {\isasymand}\ m{\isadigit{2}}\ {\isasymin}\ {\isasymPsi}\ L{\isadigit{2}}\ {\isacharbraceright}{\kern0pt}{\isachardoublequoteclose}\isanewline
\ \ \ \ \isakeywordONE{using}\isamarkupfalse%
\ add{\isachardot}{\kern0pt}commute\ \isakeywordONE{by}\isamarkupfalse%
\ blast\isanewline
\ \ \isakeywordONE{then}\isamarkupfalse%
\ \isakeywordTHREE{show}\isamarkupfalse%
\ {\isacharquery}{\kern0pt}thesis\isanewline
\ \ \ \ \isakeywordONE{using}\isamarkupfalse%
\ parikh{\isacharunderscore}{\kern0pt}img{\isacharunderscore}{\kern0pt}conc{\isacharbrackleft}{\kern0pt}of\ L{\isadigit{1}}{\isacharbrackright}{\kern0pt}\ parikh{\isacharunderscore}{\kern0pt}img{\isacharunderscore}{\kern0pt}conc{\isacharbrackleft}{\kern0pt}of\ L{\isadigit{2}}{\isacharbrackright}{\kern0pt}\ \isakeywordONE{by}\isamarkupfalse%
\ auto\isanewline
\isakeywordONE{qed}\isamarkupfalse%
%
\endisatagproof
{\isafoldproof}%
%
\isadelimproof
%
\endisadelimproof
%
\isadelimdocument
%
\endisadelimdocument
%
\isatagdocument
%
\isamarkupsubsection{Monotonicity properties%
}
\isamarkuptrue%
%
\endisatagdocument
{\isafolddocument}%
%
\isadelimdocument
%
\endisadelimdocument
\isakeywordONE{lemma}\isamarkupfalse%
\ parikh{\isacharunderscore}{\kern0pt}img{\isacharunderscore}{\kern0pt}mono{\isacharcolon}{\kern0pt}\ {\isachardoublequoteopen}A\ {\isasymsubseteq}\ B\ {\isasymLongrightarrow}\ {\isasymPsi}\ A\ {\isasymsubseteq}\ {\isasymPsi}\ B{\isachardoublequoteclose}\isanewline
%
\isadelimproof
\ \ %
\endisadelimproof
%
\isatagproof
\isakeywordONE{unfolding}\isamarkupfalse%
\ parikh{\isacharunderscore}{\kern0pt}img{\isacharunderscore}{\kern0pt}def\ \isakeywordONE{by}\isamarkupfalse%
\ fast%
\endisatagproof
{\isafoldproof}%
%
\isadelimproof
\isanewline
%
\endisadelimproof
\isanewline
\isakeywordONE{lemma}\isamarkupfalse%
\ parikh{\isacharunderscore}{\kern0pt}conc{\isacharunderscore}{\kern0pt}right{\isacharunderscore}{\kern0pt}subset{\isacharcolon}{\kern0pt}\ {\isachardoublequoteopen}{\isasymPsi}\ A\ {\isasymsubseteq}\ {\isasymPsi}\ B\ {\isasymLongrightarrow}\ {\isasymPsi}\ {\isacharparenleft}{\kern0pt}A\ {\isacharat}{\kern0pt}{\isacharat}{\kern0pt}\ C{\isacharparenright}{\kern0pt}\ {\isasymsubseteq}\ {\isasymPsi}\ {\isacharparenleft}{\kern0pt}B\ {\isacharat}{\kern0pt}{\isacharat}{\kern0pt}\ C{\isacharparenright}{\kern0pt}{\isachardoublequoteclose}\isanewline
%
\isadelimproof
\ \ %
\endisadelimproof
%
\isatagproof
\isakeywordONE{by}\isamarkupfalse%
\ {\isacharparenleft}{\kern0pt}auto\ simp\ add{\isacharcolon}{\kern0pt}\ parikh{\isacharunderscore}{\kern0pt}img{\isacharunderscore}{\kern0pt}conc{\isacharparenright}{\kern0pt}%
\endisatagproof
{\isafoldproof}%
%
\isadelimproof
\isanewline
%
\endisadelimproof
\isanewline
\isakeywordONE{lemma}\isamarkupfalse%
\ parikh{\isacharunderscore}{\kern0pt}conc{\isacharunderscore}{\kern0pt}left{\isacharunderscore}{\kern0pt}subset{\isacharcolon}{\kern0pt}\ {\isachardoublequoteopen}{\isasymPsi}\ A\ {\isasymsubseteq}\ {\isasymPsi}\ B\ {\isasymLongrightarrow}\ {\isasymPsi}\ {\isacharparenleft}{\kern0pt}C\ {\isacharat}{\kern0pt}{\isacharat}{\kern0pt}\ A{\isacharparenright}{\kern0pt}\ {\isasymsubseteq}\ {\isasymPsi}\ {\isacharparenleft}{\kern0pt}C\ {\isacharat}{\kern0pt}{\isacharat}{\kern0pt}\ B{\isacharparenright}{\kern0pt}{\isachardoublequoteclose}\isanewline
%
\isadelimproof
\ \ %
\endisadelimproof
%
\isatagproof
\isakeywordONE{by}\isamarkupfalse%
\ {\isacharparenleft}{\kern0pt}auto\ simp\ add{\isacharcolon}{\kern0pt}\ parikh{\isacharunderscore}{\kern0pt}img{\isacharunderscore}{\kern0pt}conc{\isacharparenright}{\kern0pt}%
\endisatagproof
{\isafoldproof}%
%
\isadelimproof
\isanewline
%
\endisadelimproof
\isanewline
\isakeywordONE{lemma}\isamarkupfalse%
\ parikh{\isacharunderscore}{\kern0pt}conc{\isacharunderscore}{\kern0pt}subset{\isacharcolon}{\kern0pt}\isanewline
\ \ \isakeywordTWO{assumes}\ {\isachardoublequoteopen}{\isasymPsi}\ A\ {\isasymsubseteq}\ {\isasymPsi}\ C{\isachardoublequoteclose}\isanewline
\ \ \ \ \ \ \isakeywordTWO{and}\ {\isachardoublequoteopen}{\isasymPsi}\ B\ {\isasymsubseteq}\ {\isasymPsi}\ D{\isachardoublequoteclose}\isanewline
\ \ \ \ \isakeywordTWO{shows}\ {\isachardoublequoteopen}{\isasymPsi}\ {\isacharparenleft}{\kern0pt}A\ {\isacharat}{\kern0pt}{\isacharat}{\kern0pt}\ B{\isacharparenright}{\kern0pt}\ {\isasymsubseteq}\ {\isasymPsi}\ {\isacharparenleft}{\kern0pt}C\ {\isacharat}{\kern0pt}{\isacharat}{\kern0pt}\ D{\isacharparenright}{\kern0pt}{\isachardoublequoteclose}\isanewline
%
\isadelimproof
\ \ %
\endisadelimproof
%
\isatagproof
\isakeywordONE{using}\isamarkupfalse%
\ assms\ parikh{\isacharunderscore}{\kern0pt}conc{\isacharunderscore}{\kern0pt}right{\isacharunderscore}{\kern0pt}subset\ parikh{\isacharunderscore}{\kern0pt}conc{\isacharunderscore}{\kern0pt}left{\isacharunderscore}{\kern0pt}subset\ \isakeywordONE{by}\isamarkupfalse%
\ blast%
\endisatagproof
{\isafoldproof}%
%
\isadelimproof
\isanewline
%
\endisadelimproof
\isanewline
\isakeywordONE{lemma}\isamarkupfalse%
\ parikh{\isacharunderscore}{\kern0pt}conc{\isacharunderscore}{\kern0pt}right{\isacharcolon}{\kern0pt}\ {\isachardoublequoteopen}{\isasymPsi}\ A\ {\isacharequal}{\kern0pt}\ {\isasymPsi}\ B\ {\isasymLongrightarrow}\ {\isasymPsi}\ {\isacharparenleft}{\kern0pt}A\ {\isacharat}{\kern0pt}{\isacharat}{\kern0pt}\ C{\isacharparenright}{\kern0pt}\ {\isacharequal}{\kern0pt}\ {\isasymPsi}\ {\isacharparenleft}{\kern0pt}B\ {\isacharat}{\kern0pt}{\isacharat}{\kern0pt}\ C{\isacharparenright}{\kern0pt}{\isachardoublequoteclose}\isanewline
%
\isadelimproof
\ \ %
\endisadelimproof
%
\isatagproof
\isakeywordONE{by}\isamarkupfalse%
\ {\isacharparenleft}{\kern0pt}auto\ simp\ add{\isacharcolon}{\kern0pt}\ parikh{\isacharunderscore}{\kern0pt}img{\isacharunderscore}{\kern0pt}conc{\isacharparenright}{\kern0pt}%
\endisatagproof
{\isafoldproof}%
%
\isadelimproof
\isanewline
%
\endisadelimproof
\isanewline
\isakeywordONE{lemma}\isamarkupfalse%
\ parikh{\isacharunderscore}{\kern0pt}conc{\isacharunderscore}{\kern0pt}left{\isacharcolon}{\kern0pt}\ {\isachardoublequoteopen}{\isasymPsi}\ A\ {\isacharequal}{\kern0pt}\ {\isasymPsi}\ B\ {\isasymLongrightarrow}\ {\isasymPsi}\ {\isacharparenleft}{\kern0pt}C\ {\isacharat}{\kern0pt}{\isacharat}{\kern0pt}\ A{\isacharparenright}{\kern0pt}\ {\isacharequal}{\kern0pt}\ {\isasymPsi}\ {\isacharparenleft}{\kern0pt}C\ {\isacharat}{\kern0pt}{\isacharat}{\kern0pt}\ B{\isacharparenright}{\kern0pt}{\isachardoublequoteclose}\isanewline
%
\isadelimproof
\ \ %
\endisadelimproof
%
\isatagproof
\isakeywordONE{by}\isamarkupfalse%
\ {\isacharparenleft}{\kern0pt}auto\ simp\ add{\isacharcolon}{\kern0pt}\ parikh{\isacharunderscore}{\kern0pt}img{\isacharunderscore}{\kern0pt}conc{\isacharparenright}{\kern0pt}%
\endisatagproof
{\isafoldproof}%
%
\isadelimproof
\isanewline
%
\endisadelimproof
\isanewline
\isakeywordONE{lemma}\isamarkupfalse%
\ parikh{\isacharunderscore}{\kern0pt}pow{\isacharunderscore}{\kern0pt}mono{\isacharcolon}{\kern0pt}\ {\isachardoublequoteopen}{\isasymPsi}\ A\ {\isasymsubseteq}\ {\isasymPsi}\ B\ {\isasymLongrightarrow}\ {\isasymPsi}\ {\isacharparenleft}{\kern0pt}A\ {\isacharcircum}{\kern0pt}{\isacharcircum}{\kern0pt}\ n{\isacharparenright}{\kern0pt}\ {\isasymsubseteq}\ {\isasymPsi}\ {\isacharparenleft}{\kern0pt}B\ {\isacharcircum}{\kern0pt}{\isacharcircum}{\kern0pt}\ n{\isacharparenright}{\kern0pt}{\isachardoublequoteclose}\isanewline
%
\isadelimproof
\ \ %
\endisadelimproof
%
\isatagproof
\isakeywordONE{by}\isamarkupfalse%
\ {\isacharparenleft}{\kern0pt}induction\ n{\isacharparenright}{\kern0pt}\ {\isacharparenleft}{\kern0pt}auto\ simp\ add{\isacharcolon}{\kern0pt}\ parikh{\isacharunderscore}{\kern0pt}img{\isacharunderscore}{\kern0pt}conc{\isacharparenright}{\kern0pt}%
\endisatagproof
{\isafoldproof}%
%
\isadelimproof
\isanewline
%
\endisadelimproof
\isanewline
\isanewline
\isakeywordONE{lemma}\isamarkupfalse%
\ parikh{\isacharunderscore}{\kern0pt}star{\isacharunderscore}{\kern0pt}mono{\isacharcolon}{\kern0pt}\isanewline
\ \ \isakeywordTWO{assumes}\ {\isachardoublequoteopen}{\isasymPsi}\ A\ {\isasymsubseteq}\ {\isasymPsi}\ B{\isachardoublequoteclose}\isanewline
\ \ \isakeywordTWO{shows}\ {\isachardoublequoteopen}{\isasymPsi}\ {\isacharparenleft}{\kern0pt}star\ A{\isacharparenright}{\kern0pt}\ {\isasymsubseteq}\ {\isasymPsi}\ {\isacharparenleft}{\kern0pt}star\ B{\isacharparenright}{\kern0pt}{\isachardoublequoteclose}\isanewline
%
\isadelimproof
%
\endisadelimproof
%
\isatagproof
\isakeywordONE{proof}\isamarkupfalse%
\isanewline
\ \ \isakeywordTHREE{fix}\isamarkupfalse%
\ v\isanewline
\ \ \isakeywordTHREE{assume}\isamarkupfalse%
\ {\isachardoublequoteopen}v\ {\isasymin}\ {\isasymPsi}\ {\isacharparenleft}{\kern0pt}star\ A{\isacharparenright}{\kern0pt}{\isachardoublequoteclose}\isanewline
\ \ \isakeywordONE{then}\isamarkupfalse%
\ \isakeywordTHREE{obtain}\isamarkupfalse%
\ w\ \isakeywordTWO{where}\ w{\isacharunderscore}{\kern0pt}intro{\isacharcolon}{\kern0pt}\ {\isachardoublequoteopen}parikh{\isacharunderscore}{\kern0pt}vec\ w\ {\isacharequal}{\kern0pt}\ v\ {\isasymand}\ w\ {\isasymin}\ star\ A{\isachardoublequoteclose}\ \isakeywordONE{unfolding}\isamarkupfalse%
\ parikh{\isacharunderscore}{\kern0pt}img{\isacharunderscore}{\kern0pt}def\ \isakeywordONE{by}\isamarkupfalse%
\ blast\isanewline
\ \ \isakeywordONE{then}\isamarkupfalse%
\ \isakeywordTHREE{obtain}\isamarkupfalse%
\ n\ \isakeywordTWO{where}\ {\isachardoublequoteopen}w\ {\isasymin}\ A\ {\isacharcircum}{\kern0pt}{\isacharcircum}{\kern0pt}\ n{\isachardoublequoteclose}\ \isakeywordONE{unfolding}\isamarkupfalse%
\ star{\isacharunderscore}{\kern0pt}def\ \isakeywordONE{by}\isamarkupfalse%
\ blast\isanewline
\ \ \isakeywordONE{then}\isamarkupfalse%
\ \isakeywordONE{have}\isamarkupfalse%
\ {\isachardoublequoteopen}v\ {\isasymin}\ {\isasymPsi}\ {\isacharparenleft}{\kern0pt}A\ {\isacharcircum}{\kern0pt}{\isacharcircum}{\kern0pt}\ n{\isacharparenright}{\kern0pt}{\isachardoublequoteclose}\ \isakeywordONE{using}\isamarkupfalse%
\ w{\isacharunderscore}{\kern0pt}intro\ \isakeywordONE{unfolding}\isamarkupfalse%
\ parikh{\isacharunderscore}{\kern0pt}img{\isacharunderscore}{\kern0pt}def\ \isakeywordONE{by}\isamarkupfalse%
\ blast\isanewline
\ \ \isakeywordONE{with}\isamarkupfalse%
\ assms\ \isakeywordONE{have}\isamarkupfalse%
\ {\isachardoublequoteopen}v\ {\isasymin}\ {\isasymPsi}\ {\isacharparenleft}{\kern0pt}B\ {\isacharcircum}{\kern0pt}{\isacharcircum}{\kern0pt}\ n{\isacharparenright}{\kern0pt}{\isachardoublequoteclose}\ \isakeywordONE{using}\isamarkupfalse%
\ parikh{\isacharunderscore}{\kern0pt}pow{\isacharunderscore}{\kern0pt}mono\ \isakeywordONE{by}\isamarkupfalse%
\ blast\isanewline
\ \ \isakeywordONE{then}\isamarkupfalse%
\ \isakeywordTHREE{show}\isamarkupfalse%
\ {\isachardoublequoteopen}v\ {\isasymin}\ {\isasymPsi}\ {\isacharparenleft}{\kern0pt}star\ B{\isacharparenright}{\kern0pt}{\isachardoublequoteclose}\ \isakeywordONE{unfolding}\isamarkupfalse%
\ star{\isacharunderscore}{\kern0pt}def\ \isakeywordONE{using}\isamarkupfalse%
\ parikh{\isacharunderscore}{\kern0pt}img{\isacharunderscore}{\kern0pt}UNION\ \isakeywordONE{by}\isamarkupfalse%
\ fastforce\isanewline
\isakeywordONE{qed}\isamarkupfalse%
%
\endisatagproof
{\isafoldproof}%
%
\isadelimproof
\isanewline
%
\endisadelimproof
\isanewline
\isakeywordONE{lemma}\isamarkupfalse%
\ parikh{\isacharunderscore}{\kern0pt}star{\isacharunderscore}{\kern0pt}mono{\isacharunderscore}{\kern0pt}eq{\isacharcolon}{\kern0pt}\isanewline
\ \ \isakeywordTWO{assumes}\ {\isachardoublequoteopen}{\isasymPsi}\ A\ {\isacharequal}{\kern0pt}\ {\isasymPsi}\ B{\isachardoublequoteclose}\isanewline
\ \ \isakeywordTWO{shows}\ {\isachardoublequoteopen}{\isasymPsi}\ {\isacharparenleft}{\kern0pt}star\ A{\isacharparenright}{\kern0pt}\ {\isacharequal}{\kern0pt}\ {\isasymPsi}\ {\isacharparenleft}{\kern0pt}star\ B{\isacharparenright}{\kern0pt}{\isachardoublequoteclose}\isanewline
%
\isadelimproof
\ \ %
\endisadelimproof
%
\isatagproof
\isakeywordONE{using}\isamarkupfalse%
\ parikh{\isacharunderscore}{\kern0pt}star{\isacharunderscore}{\kern0pt}mono\ \isakeywordONE{by}\isamarkupfalse%
\ {\isacharparenleft}{\kern0pt}metis\ Orderings{\isachardot}{\kern0pt}order{\isacharunderscore}{\kern0pt}eq{\isacharunderscore}{\kern0pt}iff\ assms{\isacharparenright}{\kern0pt}%
\endisatagproof
{\isafoldproof}%
%
\isadelimproof
\isanewline
%
\endisadelimproof
\isanewline
\isanewline
\isakeywordONE{lemma}\isamarkupfalse%
\ parikh{\isacharunderscore}{\kern0pt}img{\isacharunderscore}{\kern0pt}subst{\isacharunderscore}{\kern0pt}mono{\isacharcolon}{\kern0pt}\isanewline
\ \ \isakeywordTWO{assumes}\ {\isachardoublequoteopen}{\isasymforall}i{\isachardot}{\kern0pt}\ {\isasymPsi}\ {\isacharparenleft}{\kern0pt}eval\ {\isacharparenleft}{\kern0pt}A\ i{\isacharparenright}{\kern0pt}\ v{\isacharparenright}{\kern0pt}\ {\isasymsubseteq}\ {\isasymPsi}\ {\isacharparenleft}{\kern0pt}eval\ {\isacharparenleft}{\kern0pt}B\ i{\isacharparenright}{\kern0pt}\ v{\isacharparenright}{\kern0pt}{\isachardoublequoteclose}\isanewline
\ \ \isakeywordTWO{shows}\ {\isachardoublequoteopen}{\isasymPsi}\ {\isacharparenleft}{\kern0pt}eval\ {\isacharparenleft}{\kern0pt}subst\ A\ f{\isacharparenright}{\kern0pt}\ v{\isacharparenright}{\kern0pt}\ {\isasymsubseteq}\ {\isasymPsi}\ {\isacharparenleft}{\kern0pt}eval\ {\isacharparenleft}{\kern0pt}subst\ B\ f{\isacharparenright}{\kern0pt}\ v{\isacharparenright}{\kern0pt}{\isachardoublequoteclose}\isanewline
%
\isadelimproof
%
\endisadelimproof
%
\isatagproof
\isakeywordONE{proof}\isamarkupfalse%
\ {\isacharparenleft}{\kern0pt}induction\ f{\isacharparenright}{\kern0pt}\isanewline
\ \ \isakeywordTHREE{case}\isamarkupfalse%
\ {\isacharparenleft}{\kern0pt}Concat\ f{\isadigit{1}}\ f{\isadigit{2}}{\isacharparenright}{\kern0pt}\isanewline
\ \ \isakeywordONE{then}\isamarkupfalse%
\ \isakeywordONE{have}\isamarkupfalse%
\ {\isachardoublequoteopen}{\isasymPsi}\ {\isacharparenleft}{\kern0pt}eval\ {\isacharparenleft}{\kern0pt}subst\ A\ f{\isadigit{1}}{\isacharparenright}{\kern0pt}\ v\ {\isacharat}{\kern0pt}{\isacharat}{\kern0pt}\ eval\ {\isacharparenleft}{\kern0pt}subst\ A\ f{\isadigit{2}}{\isacharparenright}{\kern0pt}\ v{\isacharparenright}{\kern0pt}\isanewline
\ \ \ \ \ \ \ \ \ \ \ \ \ \ {\isasymsubseteq}\ {\isasymPsi}\ {\isacharparenleft}{\kern0pt}eval\ {\isacharparenleft}{\kern0pt}subst\ B\ f{\isadigit{1}}{\isacharparenright}{\kern0pt}\ v\ {\isacharat}{\kern0pt}{\isacharat}{\kern0pt}\ eval\ {\isacharparenleft}{\kern0pt}subst\ B\ f{\isadigit{2}}{\isacharparenright}{\kern0pt}\ v{\isacharparenright}{\kern0pt}{\isachardoublequoteclose}\isanewline
\ \ \ \ \isakeywordONE{using}\isamarkupfalse%
\ parikh{\isacharunderscore}{\kern0pt}conc{\isacharunderscore}{\kern0pt}subset\ \isakeywordONE{by}\isamarkupfalse%
\ blast\isanewline
\ \ \isakeywordONE{then}\isamarkupfalse%
\ \isakeywordTHREE{show}\isamarkupfalse%
\ {\isacharquery}{\kern0pt}case\ \isakeywordONE{by}\isamarkupfalse%
\ simp\isanewline
\isakeywordONE{next}\isamarkupfalse%
\isanewline
\ \ \isakeywordTHREE{case}\isamarkupfalse%
\ {\isacharparenleft}{\kern0pt}Star\ f{\isacharparenright}{\kern0pt}\isanewline
\ \ \isakeywordONE{then}\isamarkupfalse%
\ \isakeywordONE{have}\isamarkupfalse%
\ {\isachardoublequoteopen}{\isasymPsi}\ {\isacharparenleft}{\kern0pt}star\ {\isacharparenleft}{\kern0pt}eval\ {\isacharparenleft}{\kern0pt}subst\ A\ f{\isacharparenright}{\kern0pt}\ v{\isacharparenright}{\kern0pt}{\isacharparenright}{\kern0pt}\ {\isasymsubseteq}\ {\isasymPsi}\ {\isacharparenleft}{\kern0pt}star\ {\isacharparenleft}{\kern0pt}eval\ {\isacharparenleft}{\kern0pt}subst\ B\ f{\isacharparenright}{\kern0pt}\ v{\isacharparenright}{\kern0pt}{\isacharparenright}{\kern0pt}{\isachardoublequoteclose}\isanewline
\ \ \ \ \isakeywordONE{using}\isamarkupfalse%
\ parikh{\isacharunderscore}{\kern0pt}star{\isacharunderscore}{\kern0pt}mono\ \isakeywordONE{by}\isamarkupfalse%
\ blast\isanewline
\ \ \isakeywordONE{then}\isamarkupfalse%
\ \isakeywordTHREE{show}\isamarkupfalse%
\ {\isacharquery}{\kern0pt}case\ \isakeywordONE{by}\isamarkupfalse%
\ simp\isanewline
\isakeywordONE{qed}\isamarkupfalse%
\ {\isacharparenleft}{\kern0pt}use\ assms{\isacharparenleft}{\kern0pt}{\isadigit{1}}{\isacharparenright}{\kern0pt}\ \isakeywordTWO{in}\ auto{\isacharparenright}{\kern0pt}%
\endisatagproof
{\isafoldproof}%
%
\isadelimproof
\isanewline
%
\endisadelimproof
\isanewline
\isakeywordONE{lemma}\isamarkupfalse%
\ parikh{\isacharunderscore}{\kern0pt}img{\isacharunderscore}{\kern0pt}subst{\isacharunderscore}{\kern0pt}mono{\isacharunderscore}{\kern0pt}upd{\isacharcolon}{\kern0pt}\isanewline
\ \ \isakeywordTWO{assumes}\ {\isachardoublequoteopen}{\isasymPsi}\ {\isacharparenleft}{\kern0pt}eval\ A\ v{\isacharparenright}{\kern0pt}\ {\isasymsubseteq}\ {\isasymPsi}\ {\isacharparenleft}{\kern0pt}eval\ B\ v{\isacharparenright}{\kern0pt}{\isachardoublequoteclose}\isanewline
\ \ \isakeywordTWO{shows}\ {\isachardoublequoteopen}{\isasymPsi}\ {\isacharparenleft}{\kern0pt}eval\ {\isacharparenleft}{\kern0pt}subst\ {\isacharparenleft}{\kern0pt}Var{\isacharparenleft}{\kern0pt}x\ {\isacharcolon}{\kern0pt}{\isacharequal}{\kern0pt}\ A{\isacharparenright}{\kern0pt}{\isacharparenright}{\kern0pt}\ f{\isacharparenright}{\kern0pt}\ v{\isacharparenright}{\kern0pt}\ {\isasymsubseteq}\ {\isasymPsi}\ {\isacharparenleft}{\kern0pt}eval\ {\isacharparenleft}{\kern0pt}subst\ {\isacharparenleft}{\kern0pt}Var{\isacharparenleft}{\kern0pt}x\ {\isacharcolon}{\kern0pt}{\isacharequal}{\kern0pt}\ B{\isacharparenright}{\kern0pt}{\isacharparenright}{\kern0pt}\ f{\isacharparenright}{\kern0pt}\ v{\isacharparenright}{\kern0pt}{\isachardoublequoteclose}\isanewline
%
\isadelimproof
\ \ %
\endisadelimproof
%
\isatagproof
\isakeywordONE{using}\isamarkupfalse%
\ parikh{\isacharunderscore}{\kern0pt}img{\isacharunderscore}{\kern0pt}subst{\isacharunderscore}{\kern0pt}mono{\isacharbrackleft}{\kern0pt}of\ {\isachardoublequoteopen}Var{\isacharparenleft}{\kern0pt}x\ {\isacharcolon}{\kern0pt}{\isacharequal}{\kern0pt}\ A{\isacharparenright}{\kern0pt}{\isachardoublequoteclose}\ v\ {\isachardoublequoteopen}Var{\isacharparenleft}{\kern0pt}x\ {\isacharcolon}{\kern0pt}{\isacharequal}{\kern0pt}\ B{\isacharparenright}{\kern0pt}{\isachardoublequoteclose}{\isacharbrackright}{\kern0pt}\ assms\ \isakeywordONE{by}\isamarkupfalse%
\ auto%
\endisatagproof
{\isafoldproof}%
%
\isadelimproof
\isanewline
%
\endisadelimproof
\isanewline
\isakeywordONE{lemma}\isamarkupfalse%
\ rlexp{\isacharunderscore}{\kern0pt}mono{\isacharunderscore}{\kern0pt}parikh{\isacharcolon}{\kern0pt}\isanewline
\ \ \isakeywordTWO{assumes}\ {\isachardoublequoteopen}{\isasymforall}i\ {\isasymin}\ vars\ f{\isachardot}{\kern0pt}\ {\isasymPsi}\ {\isacharparenleft}{\kern0pt}v\ i{\isacharparenright}{\kern0pt}\ {\isasymsubseteq}\ {\isasymPsi}\ {\isacharparenleft}{\kern0pt}v{\isacharprime}{\kern0pt}\ i{\isacharparenright}{\kern0pt}{\isachardoublequoteclose}\isanewline
\ \ \isakeywordTWO{shows}\ {\isachardoublequoteopen}{\isasymPsi}\ {\isacharparenleft}{\kern0pt}eval\ f\ v{\isacharparenright}{\kern0pt}\ {\isasymsubseteq}\ {\isasymPsi}\ {\isacharparenleft}{\kern0pt}eval\ f\ v{\isacharprime}{\kern0pt}{\isacharparenright}{\kern0pt}{\isachardoublequoteclose}\isanewline
%
\isadelimproof
%
\endisadelimproof
%
\isatagproof
\isakeywordONE{using}\isamarkupfalse%
\ assms\ \isakeywordONE{proof}\isamarkupfalse%
\ {\isacharparenleft}{\kern0pt}induction\ f\ rule{\isacharcolon}{\kern0pt}\ rlexp{\isachardot}{\kern0pt}induct{\isacharparenright}{\kern0pt}\isanewline
\isakeywordTHREE{case}\isamarkupfalse%
\ {\isacharparenleft}{\kern0pt}Concat\ f{\isadigit{1}}\ f{\isadigit{2}}{\isacharparenright}{\kern0pt}\isanewline
\ \ \isakeywordONE{then}\isamarkupfalse%
\ \isakeywordONE{have}\isamarkupfalse%
\ {\isachardoublequoteopen}{\isasymPsi}\ {\isacharparenleft}{\kern0pt}eval\ f{\isadigit{1}}\ v\ {\isacharat}{\kern0pt}{\isacharat}{\kern0pt}\ eval\ f{\isadigit{2}}\ v{\isacharparenright}{\kern0pt}\ {\isasymsubseteq}\ {\isasymPsi}\ {\isacharparenleft}{\kern0pt}eval\ f{\isadigit{1}}\ v{\isacharprime}{\kern0pt}\ {\isacharat}{\kern0pt}{\isacharat}{\kern0pt}\ eval\ f{\isadigit{2}}\ v{\isacharprime}{\kern0pt}{\isacharparenright}{\kern0pt}{\isachardoublequoteclose}\isanewline
\ \ \ \ \isakeywordONE{using}\isamarkupfalse%
\ parikh{\isacharunderscore}{\kern0pt}conc{\isacharunderscore}{\kern0pt}subset\ \isakeywordONE{by}\isamarkupfalse%
\ {\isacharparenleft}{\kern0pt}metis\ UnCI\ vars{\isachardot}{\kern0pt}simps{\isacharparenleft}{\kern0pt}{\isadigit{4}}{\isacharparenright}{\kern0pt}{\isacharparenright}{\kern0pt}\isanewline
\ \ \isakeywordONE{then}\isamarkupfalse%
\ \isakeywordTHREE{show}\isamarkupfalse%
\ {\isacharquery}{\kern0pt}case\ \isakeywordONE{by}\isamarkupfalse%
\ simp\isanewline
\isakeywordONE{qed}\isamarkupfalse%
\ {\isacharparenleft}{\kern0pt}auto\ simp\ add{\isacharcolon}{\kern0pt}\ SUP{\isacharunderscore}{\kern0pt}mono{\isacharprime}{\kern0pt}\ parikh{\isacharunderscore}{\kern0pt}img{\isacharunderscore}{\kern0pt}UNION\ parikh{\isacharunderscore}{\kern0pt}star{\isacharunderscore}{\kern0pt}mono{\isacharparenright}{\kern0pt}%
\endisatagproof
{\isafoldproof}%
%
\isadelimproof
\isanewline
%
\endisadelimproof
\isanewline
\isakeywordONE{lemma}\isamarkupfalse%
\ rlexp{\isacharunderscore}{\kern0pt}mono{\isacharunderscore}{\kern0pt}parikh{\isacharunderscore}{\kern0pt}eq{\isacharcolon}{\kern0pt}\isanewline
\ \ \isakeywordTWO{assumes}\ {\isachardoublequoteopen}{\isasymforall}i\ {\isasymin}\ vars\ f{\isachardot}{\kern0pt}\ {\isasymPsi}\ {\isacharparenleft}{\kern0pt}v\ i{\isacharparenright}{\kern0pt}\ {\isacharequal}{\kern0pt}\ {\isasymPsi}\ {\isacharparenleft}{\kern0pt}v{\isacharprime}{\kern0pt}\ i{\isacharparenright}{\kern0pt}{\isachardoublequoteclose}\isanewline
\ \ \isakeywordTWO{shows}\ {\isachardoublequoteopen}{\isasymPsi}\ {\isacharparenleft}{\kern0pt}eval\ f\ v{\isacharparenright}{\kern0pt}\ {\isacharequal}{\kern0pt}\ {\isasymPsi}\ {\isacharparenleft}{\kern0pt}eval\ f\ v{\isacharprime}{\kern0pt}{\isacharparenright}{\kern0pt}{\isachardoublequoteclose}\isanewline
%
\isadelimproof
\ \ %
\endisadelimproof
%
\isatagproof
\isakeywordONE{using}\isamarkupfalse%
\ assms\ rlexp{\isacharunderscore}{\kern0pt}mono{\isacharunderscore}{\kern0pt}parikh\ \isakeywordONE{by}\isamarkupfalse%
\ blast%
\endisatagproof
{\isafoldproof}%
%
\isadelimproof
%
\endisadelimproof
%
\isadelimdocument
%
\endisadelimdocument
%
\isatagdocument
%
\isamarkupsubsection{$\Psi \; (A \cup B)^* = \Psi \; A^* B^*$%
}
\isamarkuptrue%
%
\endisatagdocument
{\isafolddocument}%
%
\isadelimdocument
%
\endisadelimdocument
%
\begin{isamarkuptext}%
This property is claimed by Pilling in \cite{Pilling} and will be needed later.%
\end{isamarkuptext}\isamarkuptrue%
\isakeywordONE{lemma}\isamarkupfalse%
\ parikh{\isacharunderscore}{\kern0pt}img{\isacharunderscore}{\kern0pt}union{\isacharunderscore}{\kern0pt}pow{\isacharunderscore}{\kern0pt}aux{\isadigit{1}}{\isacharcolon}{\kern0pt}\isanewline
\ \ \isakeywordTWO{assumes}\ {\isachardoublequoteopen}v\ {\isasymin}\ {\isasymPsi}\ {\isacharparenleft}{\kern0pt}{\isacharparenleft}{\kern0pt}A\ {\isasymunion}\ B{\isacharparenright}{\kern0pt}\ {\isacharcircum}{\kern0pt}{\isacharcircum}{\kern0pt}\ n{\isacharparenright}{\kern0pt}{\isachardoublequoteclose}\isanewline
\ \ \ \ \isakeywordTWO{shows}\ {\isachardoublequoteopen}v\ {\isasymin}\ {\isasymPsi}\ {\isacharparenleft}{\kern0pt}{\isasymUnion}i\ {\isasymle}\ n{\isachardot}{\kern0pt}\ A\ {\isacharcircum}{\kern0pt}{\isacharcircum}{\kern0pt}\ i\ {\isacharat}{\kern0pt}{\isacharat}{\kern0pt}\ B\ {\isacharcircum}{\kern0pt}{\isacharcircum}{\kern0pt}\ {\isacharparenleft}{\kern0pt}n{\isacharminus}{\kern0pt}i{\isacharparenright}{\kern0pt}{\isacharparenright}{\kern0pt}{\isachardoublequoteclose}\isanewline
%
\isadelimproof
%
\endisadelimproof
%
\isatagproof
\isakeywordONE{using}\isamarkupfalse%
\ assms\ \isakeywordONE{proof}\isamarkupfalse%
\ {\isacharparenleft}{\kern0pt}induction\ n\ arbitrary{\isacharcolon}{\kern0pt}\ v{\isacharparenright}{\kern0pt}\isanewline
\ \ \isakeywordTHREE{case}\isamarkupfalse%
\ {\isadigit{0}}\isanewline
\ \ \isakeywordONE{then}\isamarkupfalse%
\ \isakeywordTHREE{show}\isamarkupfalse%
\ {\isacharquery}{\kern0pt}case\ \isakeywordONE{by}\isamarkupfalse%
\ simp\isanewline
\isakeywordONE{next}\isamarkupfalse%
\isanewline
\ \ \isakeywordTHREE{case}\isamarkupfalse%
\ {\isacharparenleft}{\kern0pt}Suc\ n{\isacharparenright}{\kern0pt}\isanewline
\ \ \isakeywordONE{then}\isamarkupfalse%
\ \isakeywordTHREE{obtain}\isamarkupfalse%
\ w\ \isakeywordTWO{where}\ w{\isacharunderscore}{\kern0pt}intro{\isacharcolon}{\kern0pt}\ {\isachardoublequoteopen}w\ {\isasymin}\ {\isacharparenleft}{\kern0pt}A\ {\isasymunion}\ B{\isacharparenright}{\kern0pt}\ {\isacharcircum}{\kern0pt}{\isacharcircum}{\kern0pt}\ {\isacharparenleft}{\kern0pt}Suc\ n{\isacharparenright}{\kern0pt}\ {\isasymand}\ parikh{\isacharunderscore}{\kern0pt}vec\ w\ {\isacharequal}{\kern0pt}\ v{\isachardoublequoteclose}\isanewline
\ \ \ \ \isakeywordONE{unfolding}\isamarkupfalse%
\ parikh{\isacharunderscore}{\kern0pt}img{\isacharunderscore}{\kern0pt}def\ \isakeywordONE{by}\isamarkupfalse%
\ auto\isanewline
\ \ \isakeywordONE{then}\isamarkupfalse%
\ \isakeywordTHREE{obtain}\isamarkupfalse%
\ w{\isadigit{1}}\ w{\isadigit{2}}\ \isakeywordTWO{where}\ w{\isadigit{1}}{\isacharunderscore}{\kern0pt}w{\isadigit{2}}{\isacharunderscore}{\kern0pt}intro{\isacharcolon}{\kern0pt}\ {\isachardoublequoteopen}w\ {\isacharequal}{\kern0pt}\ w{\isadigit{1}}{\isacharat}{\kern0pt}w{\isadigit{2}}\ {\isasymand}\ w{\isadigit{1}}\ {\isasymin}\ A\ {\isasymunion}\ B\ {\isasymand}\ w{\isadigit{2}}\ {\isasymin}\ {\isacharparenleft}{\kern0pt}A\ {\isasymunion}\ B{\isacharparenright}{\kern0pt}\ {\isacharcircum}{\kern0pt}{\isacharcircum}{\kern0pt}\ n{\isachardoublequoteclose}\ \isakeywordONE{by}\isamarkupfalse%
\ fastforce\isanewline
\ \ \isakeywordONE{let}\isamarkupfalse%
\ {\isacharquery}{\kern0pt}v{\isadigit{1}}\ {\isacharequal}{\kern0pt}\ {\isachardoublequoteopen}parikh{\isacharunderscore}{\kern0pt}vec\ w{\isadigit{1}}{\isachardoublequoteclose}\ \isakeywordTWO{and}\ {\isacharquery}{\kern0pt}v{\isadigit{2}}\ {\isacharequal}{\kern0pt}\ {\isachardoublequoteopen}parikh{\isacharunderscore}{\kern0pt}vec\ w{\isadigit{2}}{\isachardoublequoteclose}\isanewline
\ \ \isakeywordONE{from}\isamarkupfalse%
\ w{\isadigit{1}}{\isacharunderscore}{\kern0pt}w{\isadigit{2}}{\isacharunderscore}{\kern0pt}intro\ \isakeywordONE{have}\isamarkupfalse%
\ {\isachardoublequoteopen}{\isacharquery}{\kern0pt}v{\isadigit{2}}\ {\isasymin}\ {\isasymPsi}\ {\isacharparenleft}{\kern0pt}{\isacharparenleft}{\kern0pt}A\ {\isasymunion}\ B{\isacharparenright}{\kern0pt}\ {\isacharcircum}{\kern0pt}{\isacharcircum}{\kern0pt}\ n{\isacharparenright}{\kern0pt}{\isachardoublequoteclose}\ \isakeywordONE{unfolding}\isamarkupfalse%
\ parikh{\isacharunderscore}{\kern0pt}img{\isacharunderscore}{\kern0pt}def\ \isakeywordONE{by}\isamarkupfalse%
\ blast\isanewline
\ \ \isakeywordONE{with}\isamarkupfalse%
\ Suc{\isachardot}{\kern0pt}IH\ \isakeywordONE{have}\isamarkupfalse%
\ {\isachardoublequoteopen}{\isacharquery}{\kern0pt}v{\isadigit{2}}\ {\isasymin}\ {\isasymPsi}\ {\isacharparenleft}{\kern0pt}{\isasymUnion}i\ {\isasymle}\ n{\isachardot}{\kern0pt}\ A\ {\isacharcircum}{\kern0pt}{\isacharcircum}{\kern0pt}\ i\ {\isacharat}{\kern0pt}{\isacharat}{\kern0pt}\ B\ {\isacharcircum}{\kern0pt}{\isacharcircum}{\kern0pt}\ {\isacharparenleft}{\kern0pt}n{\isacharminus}{\kern0pt}i{\isacharparenright}{\kern0pt}{\isacharparenright}{\kern0pt}{\isachardoublequoteclose}\ \isakeywordONE{by}\isamarkupfalse%
\ auto\isanewline
\ \ \isakeywordONE{then}\isamarkupfalse%
\ \isakeywordTHREE{obtain}\isamarkupfalse%
\ w{\isadigit{2}}{\isacharprime}{\kern0pt}\ \isakeywordTWO{where}\ w{\isadigit{2}}{\isacharprime}{\kern0pt}{\isacharunderscore}{\kern0pt}intro{\isacharcolon}{\kern0pt}\ {\isachardoublequoteopen}parikh{\isacharunderscore}{\kern0pt}vec\ w{\isadigit{2}}{\isacharprime}{\kern0pt}\ {\isacharequal}{\kern0pt}\ parikh{\isacharunderscore}{\kern0pt}vec\ w{\isadigit{2}}\ {\isasymand}\isanewline
\ \ \ \ \ \ w{\isadigit{2}}{\isacharprime}{\kern0pt}\ {\isasymin}\ {\isacharparenleft}{\kern0pt}{\isasymUnion}i\ {\isasymle}\ n{\isachardot}{\kern0pt}\ A\ {\isacharcircum}{\kern0pt}{\isacharcircum}{\kern0pt}\ i\ {\isacharat}{\kern0pt}{\isacharat}{\kern0pt}\ B\ {\isacharcircum}{\kern0pt}{\isacharcircum}{\kern0pt}\ {\isacharparenleft}{\kern0pt}n{\isacharminus}{\kern0pt}i{\isacharparenright}{\kern0pt}{\isacharparenright}{\kern0pt}{\isachardoublequoteclose}\ \isakeywordONE{unfolding}\isamarkupfalse%
\ parikh{\isacharunderscore}{\kern0pt}img{\isacharunderscore}{\kern0pt}def\ \isakeywordONE{by}\isamarkupfalse%
\ fastforce\isanewline
\ \ \isakeywordONE{then}\isamarkupfalse%
\ \isakeywordTHREE{obtain}\isamarkupfalse%
\ i\ \isakeywordTWO{where}\ i{\isacharunderscore}{\kern0pt}intro{\isacharcolon}{\kern0pt}\ {\isachardoublequoteopen}i\ {\isasymle}\ n\ {\isasymand}\ w{\isadigit{2}}{\isacharprime}{\kern0pt}\ {\isasymin}\ A\ {\isacharcircum}{\kern0pt}{\isacharcircum}{\kern0pt}\ i\ {\isacharat}{\kern0pt}{\isacharat}{\kern0pt}\ B\ {\isacharcircum}{\kern0pt}{\isacharcircum}{\kern0pt}\ {\isacharparenleft}{\kern0pt}n{\isacharminus}{\kern0pt}i{\isacharparenright}{\kern0pt}{\isachardoublequoteclose}\ \isakeywordONE{by}\isamarkupfalse%
\ blast\isanewline
\ \ \isakeywordONE{from}\isamarkupfalse%
\ w{\isadigit{1}}{\isacharunderscore}{\kern0pt}w{\isadigit{2}}{\isacharunderscore}{\kern0pt}intro\ w{\isadigit{2}}{\isacharprime}{\kern0pt}{\isacharunderscore}{\kern0pt}intro\ \isakeywordONE{have}\isamarkupfalse%
\ {\isachardoublequoteopen}parikh{\isacharunderscore}{\kern0pt}vec\ w\ {\isacharequal}{\kern0pt}\ parikh{\isacharunderscore}{\kern0pt}vec\ {\isacharparenleft}{\kern0pt}w{\isadigit{1}}{\isacharat}{\kern0pt}w{\isadigit{2}}{\isacharprime}{\kern0pt}{\isacharparenright}{\kern0pt}{\isachardoublequoteclose}\isanewline
\ \ \ \ \isakeywordONE{by}\isamarkupfalse%
\ simp\isanewline
\ \ \isakeywordONE{moreover}\isamarkupfalse%
\ \isakeywordONE{have}\isamarkupfalse%
\ {\isachardoublequoteopen}parikh{\isacharunderscore}{\kern0pt}vec\ {\isacharparenleft}{\kern0pt}w{\isadigit{1}}{\isacharat}{\kern0pt}w{\isadigit{2}}{\isacharprime}{\kern0pt}{\isacharparenright}{\kern0pt}\ {\isasymin}\ {\isasymPsi}\ {\isacharparenleft}{\kern0pt}{\isasymUnion}i\ {\isasymle}\ Suc\ n{\isachardot}{\kern0pt}\ A\ {\isacharcircum}{\kern0pt}{\isacharcircum}{\kern0pt}\ i\ {\isacharat}{\kern0pt}{\isacharat}{\kern0pt}\ B\ {\isacharcircum}{\kern0pt}{\isacharcircum}{\kern0pt}\ {\isacharparenleft}{\kern0pt}Suc\ n{\isacharminus}{\kern0pt}i{\isacharparenright}{\kern0pt}{\isacharparenright}{\kern0pt}{\isachardoublequoteclose}\isanewline
\ \ \isakeywordONE{proof}\isamarkupfalse%
\ {\isacharparenleft}{\kern0pt}cases\ {\isachardoublequoteopen}w{\isadigit{1}}\ {\isasymin}\ A{\isachardoublequoteclose}{\isacharparenright}{\kern0pt}\isanewline
\ \ \ \ \isakeywordTHREE{case}\isamarkupfalse%
\ True\isanewline
\ \ \ \ \isakeywordONE{with}\isamarkupfalse%
\ i{\isacharunderscore}{\kern0pt}intro\ \isakeywordONE{have}\isamarkupfalse%
\ Suc{\isacharunderscore}{\kern0pt}i{\isacharunderscore}{\kern0pt}valid{\isacharcolon}{\kern0pt}\ {\isachardoublequoteopen}Suc\ i\ {\isasymle}\ Suc\ n{\isachardoublequoteclose}\ \isakeywordTWO{and}\ {\isachardoublequoteopen}w{\isadigit{1}}{\isacharat}{\kern0pt}w{\isadigit{2}}{\isacharprime}{\kern0pt}\ {\isasymin}\ A\ {\isacharcircum}{\kern0pt}{\isacharcircum}{\kern0pt}\ {\isacharparenleft}{\kern0pt}Suc\ i{\isacharparenright}{\kern0pt}\ {\isacharat}{\kern0pt}{\isacharat}{\kern0pt}\ B\ {\isacharcircum}{\kern0pt}{\isacharcircum}{\kern0pt}\ {\isacharparenleft}{\kern0pt}Suc\ n\ {\isacharminus}{\kern0pt}\ Suc\ i{\isacharparenright}{\kern0pt}{\isachardoublequoteclose}\isanewline
\ \ \ \ \ \ \isakeywordONE{by}\isamarkupfalse%
\ {\isacharparenleft}{\kern0pt}auto\ simp\ add{\isacharcolon}{\kern0pt}\ conc{\isacharunderscore}{\kern0pt}assoc{\isacharparenright}{\kern0pt}\isanewline
\ \ \ \ \isakeywordONE{then}\isamarkupfalse%
\ \isakeywordONE{have}\isamarkupfalse%
\ {\isachardoublequoteopen}parikh{\isacharunderscore}{\kern0pt}vec\ {\isacharparenleft}{\kern0pt}w{\isadigit{1}}{\isacharat}{\kern0pt}w{\isadigit{2}}{\isacharprime}{\kern0pt}{\isacharparenright}{\kern0pt}\ {\isasymin}\ {\isasymPsi}\ {\isacharparenleft}{\kern0pt}A\ {\isacharcircum}{\kern0pt}{\isacharcircum}{\kern0pt}\ {\isacharparenleft}{\kern0pt}Suc\ i{\isacharparenright}{\kern0pt}\ {\isacharat}{\kern0pt}{\isacharat}{\kern0pt}\ B\ {\isacharcircum}{\kern0pt}{\isacharcircum}{\kern0pt}\ {\isacharparenleft}{\kern0pt}Suc\ n\ {\isacharminus}{\kern0pt}\ Suc\ i{\isacharparenright}{\kern0pt}{\isacharparenright}{\kern0pt}{\isachardoublequoteclose}\isanewline
\ \ \ \ \ \ \isakeywordONE{unfolding}\isamarkupfalse%
\ parikh{\isacharunderscore}{\kern0pt}img{\isacharunderscore}{\kern0pt}def\ \isakeywordONE{by}\isamarkupfalse%
\ blast\isanewline
\ \ \ \ \isakeywordONE{with}\isamarkupfalse%
\ Suc{\isacharunderscore}{\kern0pt}i{\isacharunderscore}{\kern0pt}valid\ parikh{\isacharunderscore}{\kern0pt}img{\isacharunderscore}{\kern0pt}UNION\ \isakeywordTHREE{show}\isamarkupfalse%
\ {\isacharquery}{\kern0pt}thesis\ \isakeywordONE{by}\isamarkupfalse%
\ fast\isanewline
\ \ \isakeywordONE{next}\isamarkupfalse%
\isanewline
\ \ \ \ \isakeywordTHREE{case}\isamarkupfalse%
\ False\isanewline
\ \ \ \ \isakeywordONE{with}\isamarkupfalse%
\ w{\isadigit{1}}{\isacharunderscore}{\kern0pt}w{\isadigit{2}}{\isacharunderscore}{\kern0pt}intro\ \isakeywordONE{have}\isamarkupfalse%
\ {\isachardoublequoteopen}w{\isadigit{1}}\ {\isasymin}\ B{\isachardoublequoteclose}\ \isakeywordONE{by}\isamarkupfalse%
\ blast\isanewline
\ \ \ \ \isakeywordONE{with}\isamarkupfalse%
\ i{\isacharunderscore}{\kern0pt}intro\ \isakeywordONE{have}\isamarkupfalse%
\ {\isachardoublequoteopen}parikh{\isacharunderscore}{\kern0pt}vec\ {\isacharparenleft}{\kern0pt}w{\isadigit{1}}{\isacharat}{\kern0pt}w{\isadigit{2}}{\isacharprime}{\kern0pt}{\isacharparenright}{\kern0pt}\ {\isasymin}\ {\isasymPsi}\ {\isacharparenleft}{\kern0pt}B\ {\isacharat}{\kern0pt}{\isacharat}{\kern0pt}\ A\ {\isacharcircum}{\kern0pt}{\isacharcircum}{\kern0pt}\ i\ {\isacharat}{\kern0pt}{\isacharat}{\kern0pt}\ B\ {\isacharcircum}{\kern0pt}{\isacharcircum}{\kern0pt}\ {\isacharparenleft}{\kern0pt}n{\isacharminus}{\kern0pt}i{\isacharparenright}{\kern0pt}{\isacharparenright}{\kern0pt}{\isachardoublequoteclose}\isanewline
\ \ \ \ \ \ \isakeywordONE{unfolding}\isamarkupfalse%
\ parikh{\isacharunderscore}{\kern0pt}img{\isacharunderscore}{\kern0pt}def\ \isakeywordONE{by}\isamarkupfalse%
\ blast\isanewline
\ \ \ \ \isakeywordONE{then}\isamarkupfalse%
\ \isakeywordONE{have}\isamarkupfalse%
\ {\isachardoublequoteopen}parikh{\isacharunderscore}{\kern0pt}vec\ {\isacharparenleft}{\kern0pt}w{\isadigit{1}}{\isacharat}{\kern0pt}w{\isadigit{2}}{\isacharprime}{\kern0pt}{\isacharparenright}{\kern0pt}\ {\isasymin}\ {\isasymPsi}\ {\isacharparenleft}{\kern0pt}A\ {\isacharcircum}{\kern0pt}{\isacharcircum}{\kern0pt}\ i\ {\isacharat}{\kern0pt}{\isacharat}{\kern0pt}\ B\ {\isacharcircum}{\kern0pt}{\isacharcircum}{\kern0pt}\ {\isacharparenleft}{\kern0pt}Suc\ n{\isacharminus}{\kern0pt}i{\isacharparenright}{\kern0pt}{\isacharparenright}{\kern0pt}{\isachardoublequoteclose}\isanewline
\ \ \ \ \ \ \isakeywordONE{using}\isamarkupfalse%
\ parikh{\isacharunderscore}{\kern0pt}img{\isacharunderscore}{\kern0pt}commut\ conc{\isacharunderscore}{\kern0pt}assoc\isanewline
\ \ \ \ \ \ \isakeywordONE{by}\isamarkupfalse%
\ {\isacharparenleft}{\kern0pt}metis\ Suc{\isacharunderscore}{\kern0pt}diff{\isacharunderscore}{\kern0pt}le\ conc{\isacharunderscore}{\kern0pt}pow{\isacharunderscore}{\kern0pt}comm\ i{\isacharunderscore}{\kern0pt}intro\ lang{\isacharunderscore}{\kern0pt}pow{\isachardot}{\kern0pt}simps{\isacharparenleft}{\kern0pt}{\isadigit{2}}{\isacharparenright}{\kern0pt}{\isacharparenright}{\kern0pt}\isanewline
\ \ \ \ \isakeywordONE{with}\isamarkupfalse%
\ i{\isacharunderscore}{\kern0pt}intro\ parikh{\isacharunderscore}{\kern0pt}img{\isacharunderscore}{\kern0pt}UNION\ \isakeywordTHREE{show}\isamarkupfalse%
\ {\isacharquery}{\kern0pt}thesis\ \isakeywordONE{by}\isamarkupfalse%
\ fastforce\isanewline
\ \ \isakeywordONE{qed}\isamarkupfalse%
\isanewline
\ \ \isakeywordONE{ultimately}\isamarkupfalse%
\ \isakeywordTHREE{show}\isamarkupfalse%
\ {\isacharquery}{\kern0pt}case\ \isakeywordONE{using}\isamarkupfalse%
\ w{\isacharunderscore}{\kern0pt}intro\ \isakeywordONE{by}\isamarkupfalse%
\ auto\isanewline
\isakeywordONE{qed}\isamarkupfalse%
%
\endisatagproof
{\isafoldproof}%
%
\isadelimproof
\isanewline
%
\endisadelimproof
\isanewline
\isakeywordONE{lemma}\isamarkupfalse%
\ parikh{\isacharunderscore}{\kern0pt}img{\isacharunderscore}{\kern0pt}star{\isacharunderscore}{\kern0pt}aux{\isadigit{1}}{\isacharcolon}{\kern0pt}\isanewline
\ \ \isakeywordTWO{assumes}\ {\isachardoublequoteopen}v\ {\isasymin}\ {\isasymPsi}\ {\isacharparenleft}{\kern0pt}star\ {\isacharparenleft}{\kern0pt}A\ {\isasymunion}\ B{\isacharparenright}{\kern0pt}{\isacharparenright}{\kern0pt}{\isachardoublequoteclose}\isanewline
\ \ \isakeywordTWO{shows}\ {\isachardoublequoteopen}v\ {\isasymin}\ {\isasymPsi}\ {\isacharparenleft}{\kern0pt}star\ A\ {\isacharat}{\kern0pt}{\isacharat}{\kern0pt}\ star\ B{\isacharparenright}{\kern0pt}{\isachardoublequoteclose}\isanewline
%
\isadelimproof
%
\endisadelimproof
%
\isatagproof
\isakeywordONE{proof}\isamarkupfalse%
\ {\isacharminus}{\kern0pt}\isanewline
\ \ \isakeywordONE{from}\isamarkupfalse%
\ assms\ \isakeywordONE{have}\isamarkupfalse%
\ {\isachardoublequoteopen}v\ {\isasymin}\ {\isacharparenleft}{\kern0pt}{\isasymUnion}n{\isachardot}{\kern0pt}\ {\isasymPsi}\ {\isacharparenleft}{\kern0pt}{\isacharparenleft}{\kern0pt}A\ {\isasymunion}\ B{\isacharparenright}{\kern0pt}\ {\isacharcircum}{\kern0pt}{\isacharcircum}{\kern0pt}\ n{\isacharparenright}{\kern0pt}{\isacharparenright}{\kern0pt}{\isachardoublequoteclose}\isanewline
\ \ \ \ \isakeywordONE{unfolding}\isamarkupfalse%
\ star{\isacharunderscore}{\kern0pt}def\ \isakeywordONE{using}\isamarkupfalse%
\ parikh{\isacharunderscore}{\kern0pt}img{\isacharunderscore}{\kern0pt}UNION\ \isakeywordONE{by}\isamarkupfalse%
\ metis\isanewline
\ \ \isakeywordONE{then}\isamarkupfalse%
\ \isakeywordTHREE{obtain}\isamarkupfalse%
\ n\ \isakeywordTWO{where}\ {\isachardoublequoteopen}v\ {\isasymin}\ {\isasymPsi}\ {\isacharparenleft}{\kern0pt}{\isacharparenleft}{\kern0pt}A\ {\isasymunion}\ B{\isacharparenright}{\kern0pt}\ {\isacharcircum}{\kern0pt}{\isacharcircum}{\kern0pt}\ n{\isacharparenright}{\kern0pt}{\isachardoublequoteclose}\ \isakeywordONE{by}\isamarkupfalse%
\ blast\isanewline
\ \ \isakeywordONE{then}\isamarkupfalse%
\ \isakeywordONE{have}\isamarkupfalse%
\ {\isachardoublequoteopen}v\ {\isasymin}\ {\isasymPsi}\ {\isacharparenleft}{\kern0pt}{\isasymUnion}i\ {\isasymle}\ n{\isachardot}{\kern0pt}\ A\ {\isacharcircum}{\kern0pt}{\isacharcircum}{\kern0pt}\ i\ {\isacharat}{\kern0pt}{\isacharat}{\kern0pt}\ B\ {\isacharcircum}{\kern0pt}{\isacharcircum}{\kern0pt}\ {\isacharparenleft}{\kern0pt}n{\isacharminus}{\kern0pt}i{\isacharparenright}{\kern0pt}{\isacharparenright}{\kern0pt}{\isachardoublequoteclose}\isanewline
\ \ \ \ \isakeywordONE{using}\isamarkupfalse%
\ parikh{\isacharunderscore}{\kern0pt}img{\isacharunderscore}{\kern0pt}union{\isacharunderscore}{\kern0pt}pow{\isacharunderscore}{\kern0pt}aux{\isadigit{1}}\ \isakeywordONE{by}\isamarkupfalse%
\ auto\isanewline
\ \ \isakeywordONE{then}\isamarkupfalse%
\ \isakeywordONE{have}\isamarkupfalse%
\ {\isachardoublequoteopen}v\ {\isasymin}\ {\isacharparenleft}{\kern0pt}{\isasymUnion}i{\isasymle}n{\isachardot}{\kern0pt}\ {\isasymPsi}\ {\isacharparenleft}{\kern0pt}A\ {\isacharcircum}{\kern0pt}{\isacharcircum}{\kern0pt}\ i\ {\isacharat}{\kern0pt}{\isacharat}{\kern0pt}\ B\ {\isacharcircum}{\kern0pt}{\isacharcircum}{\kern0pt}\ {\isacharparenleft}{\kern0pt}n{\isacharminus}{\kern0pt}i{\isacharparenright}{\kern0pt}{\isacharparenright}{\kern0pt}{\isacharparenright}{\kern0pt}{\isachardoublequoteclose}\ \isakeywordONE{using}\isamarkupfalse%
\ parikh{\isacharunderscore}{\kern0pt}img{\isacharunderscore}{\kern0pt}UNION\ \isakeywordONE{by}\isamarkupfalse%
\ metis\isanewline
\ \ \isakeywordONE{then}\isamarkupfalse%
\ \isakeywordTHREE{obtain}\isamarkupfalse%
\ i\ \isakeywordTWO{where}\ {\isachardoublequoteopen}i{\isasymle}n\ {\isasymand}\ v\ {\isasymin}\ {\isasymPsi}\ {\isacharparenleft}{\kern0pt}A\ {\isacharcircum}{\kern0pt}{\isacharcircum}{\kern0pt}\ i\ {\isacharat}{\kern0pt}{\isacharat}{\kern0pt}\ B\ {\isacharcircum}{\kern0pt}{\isacharcircum}{\kern0pt}\ {\isacharparenleft}{\kern0pt}n{\isacharminus}{\kern0pt}i{\isacharparenright}{\kern0pt}{\isacharparenright}{\kern0pt}{\isachardoublequoteclose}\ \isakeywordONE{by}\isamarkupfalse%
\ blast\isanewline
\ \ \isakeywordONE{then}\isamarkupfalse%
\ \isakeywordTHREE{obtain}\isamarkupfalse%
\ w\ \isakeywordTWO{where}\ w{\isacharunderscore}{\kern0pt}intro{\isacharcolon}{\kern0pt}\ {\isachardoublequoteopen}parikh{\isacharunderscore}{\kern0pt}vec\ w\ {\isacharequal}{\kern0pt}\ v\ {\isasymand}\ w\ {\isasymin}\ A\ {\isacharcircum}{\kern0pt}{\isacharcircum}{\kern0pt}\ i\ {\isacharat}{\kern0pt}{\isacharat}{\kern0pt}\ B\ {\isacharcircum}{\kern0pt}{\isacharcircum}{\kern0pt}\ {\isacharparenleft}{\kern0pt}n{\isacharminus}{\kern0pt}i{\isacharparenright}{\kern0pt}{\isachardoublequoteclose}\isanewline
\ \ \ \ \isakeywordONE{unfolding}\isamarkupfalse%
\ parikh{\isacharunderscore}{\kern0pt}img{\isacharunderscore}{\kern0pt}def\ \isakeywordONE{by}\isamarkupfalse%
\ blast\isanewline
\ \ \isakeywordONE{then}\isamarkupfalse%
\ \isakeywordTHREE{obtain}\isamarkupfalse%
\ w{\isadigit{1}}\ w{\isadigit{2}}\ \isakeywordTWO{where}\ w{\isacharunderscore}{\kern0pt}decomp{\isacharcolon}{\kern0pt}\ {\isachardoublequoteopen}w{\isacharequal}{\kern0pt}w{\isadigit{1}}{\isacharat}{\kern0pt}w{\isadigit{2}}\ {\isasymand}\ w{\isadigit{1}}\ {\isasymin}\ A\ {\isacharcircum}{\kern0pt}{\isacharcircum}{\kern0pt}\ i\ {\isasymand}\ w{\isadigit{2}}\ {\isasymin}\ B\ {\isacharcircum}{\kern0pt}{\isacharcircum}{\kern0pt}\ {\isacharparenleft}{\kern0pt}n{\isacharminus}{\kern0pt}i{\isacharparenright}{\kern0pt}{\isachardoublequoteclose}\ \isakeywordONE{by}\isamarkupfalse%
\ blast\isanewline
\ \ \isakeywordONE{then}\isamarkupfalse%
\ \isakeywordONE{have}\isamarkupfalse%
\ {\isachardoublequoteopen}w{\isadigit{1}}\ {\isasymin}\ star\ A{\isachardoublequoteclose}\ \isakeywordTWO{and}\ {\isachardoublequoteopen}w{\isadigit{2}}\ {\isasymin}\ star\ B{\isachardoublequoteclose}\ \isakeywordONE{by}\isamarkupfalse%
\ auto\isanewline
\ \ \isakeywordONE{with}\isamarkupfalse%
\ w{\isacharunderscore}{\kern0pt}decomp\ \isakeywordONE{have}\isamarkupfalse%
\ {\isachardoublequoteopen}w\ {\isasymin}\ star\ A\ {\isacharat}{\kern0pt}{\isacharat}{\kern0pt}\ star\ B{\isachardoublequoteclose}\ \isakeywordONE{by}\isamarkupfalse%
\ auto\isanewline
\ \ \isakeywordONE{with}\isamarkupfalse%
\ w{\isacharunderscore}{\kern0pt}intro\ \isakeywordTHREE{show}\isamarkupfalse%
\ {\isacharquery}{\kern0pt}thesis\ \isakeywordONE{unfolding}\isamarkupfalse%
\ parikh{\isacharunderscore}{\kern0pt}img{\isacharunderscore}{\kern0pt}def\ \isakeywordONE{by}\isamarkupfalse%
\ blast\isanewline
\isakeywordONE{qed}\isamarkupfalse%
%
\endisatagproof
{\isafoldproof}%
%
\isadelimproof
\isanewline
%
\endisadelimproof
\isanewline
\isakeywordONE{lemma}\isamarkupfalse%
\ parikh{\isacharunderscore}{\kern0pt}img{\isacharunderscore}{\kern0pt}star{\isacharunderscore}{\kern0pt}aux{\isadigit{2}}{\isacharcolon}{\kern0pt}\isanewline
\ \ \isakeywordTWO{assumes}\ {\isachardoublequoteopen}v\ {\isasymin}\ {\isasymPsi}\ {\isacharparenleft}{\kern0pt}star\ A\ {\isacharat}{\kern0pt}{\isacharat}{\kern0pt}\ star\ B{\isacharparenright}{\kern0pt}{\isachardoublequoteclose}\isanewline
\ \ \isakeywordTWO{shows}\ {\isachardoublequoteopen}v\ {\isasymin}\ {\isasymPsi}\ {\isacharparenleft}{\kern0pt}star\ {\isacharparenleft}{\kern0pt}A\ {\isasymunion}\ B{\isacharparenright}{\kern0pt}{\isacharparenright}{\kern0pt}{\isachardoublequoteclose}\isanewline
%
\isadelimproof
%
\endisadelimproof
%
\isatagproof
\isakeywordONE{proof}\isamarkupfalse%
\ {\isacharminus}{\kern0pt}\isanewline
\ \ \isakeywordONE{from}\isamarkupfalse%
\ assms\ \isakeywordTHREE{obtain}\isamarkupfalse%
\ w\ \isakeywordTWO{where}\ w{\isacharunderscore}{\kern0pt}intro{\isacharcolon}{\kern0pt}\ {\isachardoublequoteopen}parikh{\isacharunderscore}{\kern0pt}vec\ w\ {\isacharequal}{\kern0pt}\ v\ {\isasymand}\ w\ {\isasymin}\ star\ A\ {\isacharat}{\kern0pt}{\isacharat}{\kern0pt}\ star\ B{\isachardoublequoteclose}\isanewline
\ \ \ \ \isakeywordONE{unfolding}\isamarkupfalse%
\ parikh{\isacharunderscore}{\kern0pt}img{\isacharunderscore}{\kern0pt}def\ \isakeywordONE{by}\isamarkupfalse%
\ blast\isanewline
\ \ \isakeywordONE{then}\isamarkupfalse%
\ \isakeywordTHREE{obtain}\isamarkupfalse%
\ w{\isadigit{1}}\ w{\isadigit{2}}\ \isakeywordTWO{where}\ w{\isacharunderscore}{\kern0pt}decomp{\isacharcolon}{\kern0pt}\ {\isachardoublequoteopen}w{\isacharequal}{\kern0pt}w{\isadigit{1}}{\isacharat}{\kern0pt}w{\isadigit{2}}\ {\isasymand}\ w{\isadigit{1}}\ {\isasymin}\ star\ A\ {\isasymand}\ w{\isadigit{2}}\ {\isasymin}\ star\ B{\isachardoublequoteclose}\ \isakeywordONE{by}\isamarkupfalse%
\ blast\isanewline
\ \ \isakeywordONE{then}\isamarkupfalse%
\ \isakeywordTHREE{obtain}\isamarkupfalse%
\ i\ j\ \isakeywordTWO{where}\ {\isachardoublequoteopen}w{\isadigit{1}}\ {\isasymin}\ A\ {\isacharcircum}{\kern0pt}{\isacharcircum}{\kern0pt}\ i{\isachardoublequoteclose}\ \isakeywordTWO{and}\ w{\isadigit{2}}{\isacharunderscore}{\kern0pt}intro{\isacharcolon}{\kern0pt}\ {\isachardoublequoteopen}w{\isadigit{2}}\ {\isasymin}\ B\ {\isacharcircum}{\kern0pt}{\isacharcircum}{\kern0pt}\ j{\isachardoublequoteclose}\ \isakeywordONE{unfolding}\isamarkupfalse%
\ star{\isacharunderscore}{\kern0pt}def\ \isakeywordONE{by}\isamarkupfalse%
\ blast\isanewline
\ \ \isakeywordONE{then}\isamarkupfalse%
\ \isakeywordONE{have}\isamarkupfalse%
\ w{\isadigit{1}}{\isacharunderscore}{\kern0pt}in{\isacharunderscore}{\kern0pt}union{\isacharcolon}{\kern0pt}\ {\isachardoublequoteopen}w{\isadigit{1}}\ {\isasymin}\ {\isacharparenleft}{\kern0pt}A\ {\isasymunion}\ B{\isacharparenright}{\kern0pt}\ {\isacharcircum}{\kern0pt}{\isacharcircum}{\kern0pt}\ i{\isachardoublequoteclose}\ \isakeywordONE{using}\isamarkupfalse%
\ langpow{\isacharunderscore}{\kern0pt}mono\ \isakeywordONE{by}\isamarkupfalse%
\ blast\isanewline
\ \ \isakeywordONE{from}\isamarkupfalse%
\ w{\isadigit{2}}{\isacharunderscore}{\kern0pt}intro\ \isakeywordONE{have}\isamarkupfalse%
\ {\isachardoublequoteopen}w{\isadigit{2}}\ {\isasymin}\ {\isacharparenleft}{\kern0pt}A\ {\isasymunion}\ B{\isacharparenright}{\kern0pt}\ {\isacharcircum}{\kern0pt}{\isacharcircum}{\kern0pt}\ j{\isachardoublequoteclose}\ \isakeywordONE{using}\isamarkupfalse%
\ langpow{\isacharunderscore}{\kern0pt}mono\ \isakeywordONE{by}\isamarkupfalse%
\ blast\isanewline
\ \ \isakeywordONE{with}\isamarkupfalse%
\ w{\isadigit{1}}{\isacharunderscore}{\kern0pt}in{\isacharunderscore}{\kern0pt}union\ w{\isacharunderscore}{\kern0pt}decomp\ \isakeywordONE{have}\isamarkupfalse%
\ {\isachardoublequoteopen}w\ {\isasymin}\ {\isacharparenleft}{\kern0pt}A\ {\isasymunion}\ B{\isacharparenright}{\kern0pt}\ {\isacharcircum}{\kern0pt}{\isacharcircum}{\kern0pt}\ {\isacharparenleft}{\kern0pt}i{\isacharplus}{\kern0pt}j{\isacharparenright}{\kern0pt}{\isachardoublequoteclose}\ \isakeywordONE{using}\isamarkupfalse%
\ lang{\isacharunderscore}{\kern0pt}pow{\isacharunderscore}{\kern0pt}add\ \isakeywordONE{by}\isamarkupfalse%
\ fast\isanewline
\ \ \isakeywordONE{with}\isamarkupfalse%
\ w{\isacharunderscore}{\kern0pt}intro\ \isakeywordTHREE{show}\isamarkupfalse%
\ {\isacharquery}{\kern0pt}thesis\ \isakeywordONE{unfolding}\isamarkupfalse%
\ parikh{\isacharunderscore}{\kern0pt}img{\isacharunderscore}{\kern0pt}def\ \isakeywordONE{by}\isamarkupfalse%
\ auto\isanewline
\isakeywordONE{qed}\isamarkupfalse%
%
\endisatagproof
{\isafoldproof}%
%
\isadelimproof
\isanewline
%
\endisadelimproof
\isanewline
\isakeywordONE{lemma}\isamarkupfalse%
\ parikh{\isacharunderscore}{\kern0pt}img{\isacharunderscore}{\kern0pt}star{\isacharcolon}{\kern0pt}\ {\isachardoublequoteopen}{\isasymPsi}\ {\isacharparenleft}{\kern0pt}star\ {\isacharparenleft}{\kern0pt}A\ {\isasymunion}\ B{\isacharparenright}{\kern0pt}{\isacharparenright}{\kern0pt}\ {\isacharequal}{\kern0pt}\ {\isasymPsi}\ {\isacharparenleft}{\kern0pt}star\ A\ {\isacharat}{\kern0pt}{\isacharat}{\kern0pt}\ star\ B{\isacharparenright}{\kern0pt}{\isachardoublequoteclose}\isanewline
%
\isadelimproof
%
\endisadelimproof
%
\isatagproof
\isakeywordONE{proof}\isamarkupfalse%
\isanewline
\ \ \isakeywordTHREE{show}\isamarkupfalse%
\ {\isachardoublequoteopen}{\isasymPsi}\ {\isacharparenleft}{\kern0pt}star\ {\isacharparenleft}{\kern0pt}A\ {\isasymunion}\ B{\isacharparenright}{\kern0pt}{\isacharparenright}{\kern0pt}\ {\isasymsubseteq}\ {\isasymPsi}\ {\isacharparenleft}{\kern0pt}star\ A\ {\isacharat}{\kern0pt}{\isacharat}{\kern0pt}\ star\ B{\isacharparenright}{\kern0pt}{\isachardoublequoteclose}\ \isakeywordONE{using}\isamarkupfalse%
\ parikh{\isacharunderscore}{\kern0pt}img{\isacharunderscore}{\kern0pt}star{\isacharunderscore}{\kern0pt}aux{\isadigit{1}}\ \isakeywordONE{by}\isamarkupfalse%
\ auto\isanewline
\ \ \isakeywordTHREE{show}\isamarkupfalse%
\ {\isachardoublequoteopen}{\isasymPsi}\ {\isacharparenleft}{\kern0pt}star\ A\ {\isacharat}{\kern0pt}{\isacharat}{\kern0pt}\ star\ B{\isacharparenright}{\kern0pt}\ {\isasymsubseteq}\ {\isasymPsi}\ {\isacharparenleft}{\kern0pt}star\ {\isacharparenleft}{\kern0pt}A\ {\isasymunion}\ B{\isacharparenright}{\kern0pt}{\isacharparenright}{\kern0pt}{\isachardoublequoteclose}\ \isakeywordONE{using}\isamarkupfalse%
\ parikh{\isacharunderscore}{\kern0pt}img{\isacharunderscore}{\kern0pt}star{\isacharunderscore}{\kern0pt}aux{\isadigit{2}}\ \isakeywordONE{by}\isamarkupfalse%
\ auto\isanewline
\isakeywordONE{qed}\isamarkupfalse%
%
\endisatagproof
{\isafoldproof}%
%
\isadelimproof
%
\endisadelimproof
%
\isadelimdocument
%
\endisadelimdocument
%
\isatagdocument
%
\isamarkupsubsection{$\Psi \; (E^* F)^* = \Psi \; \left(\{\varepsilon\} \cup E^* F^* F\right)$%
}
\isamarkuptrue%
%
\endisatagdocument
{\isafolddocument}%
%
\isadelimdocument
%
\endisadelimdocument
%
\begin{isamarkuptext}%
This property (where $\varepsilon$ denotes the empty word) is claimed by
Pilling as well \cite{Pilling}; we will use it later.%
\end{isamarkuptext}\isamarkuptrue%
\isakeywordONE{lemma}\isamarkupfalse%
\ parikh{\isacharunderscore}{\kern0pt}img{\isacharunderscore}{\kern0pt}conc{\isacharunderscore}{\kern0pt}pow{\isacharcolon}{\kern0pt}\ {\isachardoublequoteopen}{\isasymPsi}\ {\isacharparenleft}{\kern0pt}{\isacharparenleft}{\kern0pt}A\ {\isacharat}{\kern0pt}{\isacharat}{\kern0pt}\ B{\isacharparenright}{\kern0pt}\ {\isacharcircum}{\kern0pt}{\isacharcircum}{\kern0pt}\ n{\isacharparenright}{\kern0pt}\ {\isasymsubseteq}\ {\isasymPsi}\ {\isacharparenleft}{\kern0pt}A\ {\isacharcircum}{\kern0pt}{\isacharcircum}{\kern0pt}\ n\ {\isacharat}{\kern0pt}{\isacharat}{\kern0pt}\ B\ {\isacharcircum}{\kern0pt}{\isacharcircum}{\kern0pt}\ n{\isacharparenright}{\kern0pt}{\isachardoublequoteclose}\isanewline
%
\isadelimproof
%
\endisadelimproof
%
\isatagproof
\isakeywordONE{proof}\isamarkupfalse%
\ {\isacharparenleft}{\kern0pt}induction\ n{\isacharparenright}{\kern0pt}\isanewline
\ \ \isakeywordTHREE{case}\isamarkupfalse%
\ {\isacharparenleft}{\kern0pt}Suc\ n{\isacharparenright}{\kern0pt}\isanewline
\ \ \isakeywordONE{then}\isamarkupfalse%
\ \isakeywordONE{have}\isamarkupfalse%
\ {\isachardoublequoteopen}{\isasymPsi}\ {\isacharparenleft}{\kern0pt}{\isacharparenleft}{\kern0pt}A\ {\isacharat}{\kern0pt}{\isacharat}{\kern0pt}\ B{\isacharparenright}{\kern0pt}\ {\isacharcircum}{\kern0pt}{\isacharcircum}{\kern0pt}\ n\ {\isacharat}{\kern0pt}{\isacharat}{\kern0pt}\ A\ {\isacharat}{\kern0pt}{\isacharat}{\kern0pt}\ B{\isacharparenright}{\kern0pt}\ {\isasymsubseteq}\ {\isasymPsi}\ {\isacharparenleft}{\kern0pt}A\ {\isacharcircum}{\kern0pt}{\isacharcircum}{\kern0pt}\ n\ {\isacharat}{\kern0pt}{\isacharat}{\kern0pt}\ B\ {\isacharcircum}{\kern0pt}{\isacharcircum}{\kern0pt}\ n\ {\isacharat}{\kern0pt}{\isacharat}{\kern0pt}\ A\ {\isacharat}{\kern0pt}{\isacharat}{\kern0pt}\ B{\isacharparenright}{\kern0pt}{\isachardoublequoteclose}\isanewline
\ \ \ \ \isakeywordONE{using}\isamarkupfalse%
\ parikh{\isacharunderscore}{\kern0pt}conc{\isacharunderscore}{\kern0pt}right{\isacharunderscore}{\kern0pt}subset\ conc{\isacharunderscore}{\kern0pt}assoc\ \isakeywordONE{by}\isamarkupfalse%
\ metis\isanewline
\ \ \isakeywordONE{also}\isamarkupfalse%
\ \isakeywordONE{have}\isamarkupfalse%
\ {\isachardoublequoteopen}{\isasymdots}\ {\isacharequal}{\kern0pt}\ {\isasymPsi}\ {\isacharparenleft}{\kern0pt}A\ {\isacharcircum}{\kern0pt}{\isacharcircum}{\kern0pt}\ n\ {\isacharat}{\kern0pt}{\isacharat}{\kern0pt}\ A\ {\isacharat}{\kern0pt}{\isacharat}{\kern0pt}\ B\ {\isacharcircum}{\kern0pt}{\isacharcircum}{\kern0pt}\ n\ {\isacharat}{\kern0pt}{\isacharat}{\kern0pt}\ B{\isacharparenright}{\kern0pt}{\isachardoublequoteclose}\isanewline
\ \ \ \ \isakeywordONE{by}\isamarkupfalse%
\ {\isacharparenleft}{\kern0pt}metis\ parikh{\isacharunderscore}{\kern0pt}img{\isacharunderscore}{\kern0pt}commut\ conc{\isacharunderscore}{\kern0pt}assoc\ parikh{\isacharunderscore}{\kern0pt}conc{\isacharunderscore}{\kern0pt}left{\isacharparenright}{\kern0pt}\isanewline
\ \ \isakeywordONE{finally}\isamarkupfalse%
\ \isakeywordTHREE{show}\isamarkupfalse%
\ {\isacharquery}{\kern0pt}case\ \isakeywordONE{by}\isamarkupfalse%
\ {\isacharparenleft}{\kern0pt}simp\ add{\isacharcolon}{\kern0pt}\ conc{\isacharunderscore}{\kern0pt}assoc\ conc{\isacharunderscore}{\kern0pt}pow{\isacharunderscore}{\kern0pt}comm{\isacharparenright}{\kern0pt}\isanewline
\isakeywordONE{qed}\isamarkupfalse%
\ simp%
\endisatagproof
{\isafoldproof}%
%
\isadelimproof
\isanewline
%
\endisadelimproof
\isanewline
\isakeywordONE{lemma}\isamarkupfalse%
\ parikh{\isacharunderscore}{\kern0pt}img{\isacharunderscore}{\kern0pt}conc{\isacharunderscore}{\kern0pt}star{\isacharcolon}{\kern0pt}\ {\isachardoublequoteopen}{\isasymPsi}\ {\isacharparenleft}{\kern0pt}star\ {\isacharparenleft}{\kern0pt}A\ {\isacharat}{\kern0pt}{\isacharat}{\kern0pt}\ B{\isacharparenright}{\kern0pt}{\isacharparenright}{\kern0pt}\ {\isasymsubseteq}\ {\isasymPsi}\ {\isacharparenleft}{\kern0pt}star\ A\ {\isacharat}{\kern0pt}{\isacharat}{\kern0pt}\ star\ B{\isacharparenright}{\kern0pt}{\isachardoublequoteclose}\isanewline
%
\isadelimproof
%
\endisadelimproof
%
\isatagproof
\isakeywordONE{proof}\isamarkupfalse%
\isanewline
\ \ \isakeywordTHREE{fix}\isamarkupfalse%
\ v\isanewline
\ \ \isakeywordTHREE{assume}\isamarkupfalse%
\ {\isachardoublequoteopen}v\ {\isasymin}\ {\isasymPsi}\ {\isacharparenleft}{\kern0pt}star\ {\isacharparenleft}{\kern0pt}A\ {\isacharat}{\kern0pt}{\isacharat}{\kern0pt}\ B{\isacharparenright}{\kern0pt}{\isacharparenright}{\kern0pt}{\isachardoublequoteclose}\isanewline
\ \ \isakeywordONE{then}\isamarkupfalse%
\ \isakeywordONE{have}\isamarkupfalse%
\ {\isachardoublequoteopen}{\isasymexists}n{\isachardot}{\kern0pt}\ v\ {\isasymin}\ {\isasymPsi}\ {\isacharparenleft}{\kern0pt}{\isacharparenleft}{\kern0pt}A\ {\isacharat}{\kern0pt}{\isacharat}{\kern0pt}\ B{\isacharparenright}{\kern0pt}\ {\isacharcircum}{\kern0pt}{\isacharcircum}{\kern0pt}\ n{\isacharparenright}{\kern0pt}{\isachardoublequoteclose}\ \isakeywordONE{unfolding}\isamarkupfalse%
\ star{\isacharunderscore}{\kern0pt}def\ \isakeywordONE{by}\isamarkupfalse%
\ {\isacharparenleft}{\kern0pt}simp\ add{\isacharcolon}{\kern0pt}\ parikh{\isacharunderscore}{\kern0pt}img{\isacharunderscore}{\kern0pt}UNION{\isacharparenright}{\kern0pt}\isanewline
\ \ \isakeywordONE{then}\isamarkupfalse%
\ \isakeywordTHREE{obtain}\isamarkupfalse%
\ n\ \isakeywordTWO{where}\ {\isachardoublequoteopen}v\ {\isasymin}\ {\isasymPsi}\ {\isacharparenleft}{\kern0pt}{\isacharparenleft}{\kern0pt}A\ {\isacharat}{\kern0pt}{\isacharat}{\kern0pt}\ B{\isacharparenright}{\kern0pt}\ {\isacharcircum}{\kern0pt}{\isacharcircum}{\kern0pt}\ n{\isacharparenright}{\kern0pt}{\isachardoublequoteclose}\ \isakeywordONE{by}\isamarkupfalse%
\ blast\isanewline
\ \ \isakeywordONE{with}\isamarkupfalse%
\ parikh{\isacharunderscore}{\kern0pt}img{\isacharunderscore}{\kern0pt}conc{\isacharunderscore}{\kern0pt}pow\ \isakeywordONE{have}\isamarkupfalse%
\ {\isachardoublequoteopen}v\ {\isasymin}\ {\isasymPsi}\ {\isacharparenleft}{\kern0pt}A\ {\isacharcircum}{\kern0pt}{\isacharcircum}{\kern0pt}\ n\ {\isacharat}{\kern0pt}{\isacharat}{\kern0pt}\ B\ {\isacharcircum}{\kern0pt}{\isacharcircum}{\kern0pt}\ n{\isacharparenright}{\kern0pt}{\isachardoublequoteclose}\ \isakeywordONE{by}\isamarkupfalse%
\ fast\isanewline
\ \ \isakeywordONE{then}\isamarkupfalse%
\ \isakeywordONE{have}\isamarkupfalse%
\ {\isachardoublequoteopen}v\ {\isasymin}\ {\isasymPsi}\ {\isacharparenleft}{\kern0pt}A\ {\isacharcircum}{\kern0pt}{\isacharcircum}{\kern0pt}\ n\ {\isacharat}{\kern0pt}{\isacharat}{\kern0pt}\ star\ B{\isacharparenright}{\kern0pt}{\isachardoublequoteclose}\isanewline
\ \ \ \ \isakeywordONE{unfolding}\isamarkupfalse%
\ star{\isacharunderscore}{\kern0pt}def\ \isakeywordONE{using}\isamarkupfalse%
\ parikh{\isacharunderscore}{\kern0pt}conc{\isacharunderscore}{\kern0pt}left{\isacharunderscore}{\kern0pt}subset\isanewline
\ \ \ \ \isakeywordONE{by}\isamarkupfalse%
\ {\isacharparenleft}{\kern0pt}metis\ {\isacharparenleft}{\kern0pt}no{\isacharunderscore}{\kern0pt}types{\isacharcomma}{\kern0pt}\ lifting{\isacharparenright}{\kern0pt}\ Sup{\isacharunderscore}{\kern0pt}upper\ parikh{\isacharunderscore}{\kern0pt}img{\isacharunderscore}{\kern0pt}mono\ rangeI\ subset{\isacharunderscore}{\kern0pt}eq{\isacharparenright}{\kern0pt}\isanewline
\ \ \isakeywordONE{then}\isamarkupfalse%
\ \isakeywordTHREE{show}\isamarkupfalse%
\ {\isachardoublequoteopen}v\ {\isasymin}\ {\isasymPsi}\ {\isacharparenleft}{\kern0pt}star\ A\ {\isacharat}{\kern0pt}{\isacharat}{\kern0pt}\ star\ B{\isacharparenright}{\kern0pt}{\isachardoublequoteclose}\isanewline
\ \ \ \ \isakeywordONE{unfolding}\isamarkupfalse%
\ star{\isacharunderscore}{\kern0pt}def\ \isakeywordONE{using}\isamarkupfalse%
\ parikh{\isacharunderscore}{\kern0pt}conc{\isacharunderscore}{\kern0pt}right{\isacharunderscore}{\kern0pt}subset\isanewline
\ \ \ \ \isakeywordONE{by}\isamarkupfalse%
\ {\isacharparenleft}{\kern0pt}metis\ {\isacharparenleft}{\kern0pt}no{\isacharunderscore}{\kern0pt}types{\isacharcomma}{\kern0pt}\ lifting{\isacharparenright}{\kern0pt}\ Sup{\isacharunderscore}{\kern0pt}upper\ parikh{\isacharunderscore}{\kern0pt}img{\isacharunderscore}{\kern0pt}mono\ rangeI\ subset{\isacharunderscore}{\kern0pt}eq{\isacharparenright}{\kern0pt}\isanewline
\isakeywordONE{qed}\isamarkupfalse%
%
\endisatagproof
{\isafoldproof}%
%
\isadelimproof
\isanewline
%
\endisadelimproof
\isanewline
\isakeywordONE{lemma}\isamarkupfalse%
\ parikh{\isacharunderscore}{\kern0pt}img{\isacharunderscore}{\kern0pt}conc{\isacharunderscore}{\kern0pt}pow{\isadigit{2}}{\isacharcolon}{\kern0pt}\ {\isachardoublequoteopen}{\isasymPsi}\ {\isacharparenleft}{\kern0pt}{\isacharparenleft}{\kern0pt}A\ {\isacharat}{\kern0pt}{\isacharat}{\kern0pt}\ B{\isacharparenright}{\kern0pt}\ {\isacharcircum}{\kern0pt}{\isacharcircum}{\kern0pt}\ Suc\ n{\isacharparenright}{\kern0pt}\ {\isasymsubseteq}\ {\isasymPsi}\ {\isacharparenleft}{\kern0pt}star\ A\ {\isacharat}{\kern0pt}{\isacharat}{\kern0pt}\ star\ B\ {\isacharat}{\kern0pt}{\isacharat}{\kern0pt}\ B{\isacharparenright}{\kern0pt}{\isachardoublequoteclose}\isanewline
%
\isadelimproof
%
\endisadelimproof
%
\isatagproof
\isakeywordONE{proof}\isamarkupfalse%
\isanewline
\ \ \isakeywordTHREE{fix}\isamarkupfalse%
\ v\isanewline
\ \ \isakeywordTHREE{assume}\isamarkupfalse%
\ {\isachardoublequoteopen}v\ {\isasymin}\ {\isasymPsi}\ {\isacharparenleft}{\kern0pt}{\isacharparenleft}{\kern0pt}A\ {\isacharat}{\kern0pt}{\isacharat}{\kern0pt}\ B{\isacharparenright}{\kern0pt}\ {\isacharcircum}{\kern0pt}{\isacharcircum}{\kern0pt}\ Suc\ n{\isacharparenright}{\kern0pt}{\isachardoublequoteclose}\isanewline
\ \ \isakeywordONE{with}\isamarkupfalse%
\ parikh{\isacharunderscore}{\kern0pt}img{\isacharunderscore}{\kern0pt}conc{\isacharunderscore}{\kern0pt}pow\ \isakeywordONE{have}\isamarkupfalse%
\ {\isachardoublequoteopen}v\ {\isasymin}\ {\isasymPsi}\ {\isacharparenleft}{\kern0pt}A\ {\isacharcircum}{\kern0pt}{\isacharcircum}{\kern0pt}\ Suc\ n\ {\isacharat}{\kern0pt}{\isacharat}{\kern0pt}\ B\ {\isacharcircum}{\kern0pt}{\isacharcircum}{\kern0pt}\ n\ {\isacharat}{\kern0pt}{\isacharat}{\kern0pt}\ B{\isacharparenright}{\kern0pt}{\isachardoublequoteclose}\isanewline
\ \ \ \ \isakeywordONE{by}\isamarkupfalse%
\ {\isacharparenleft}{\kern0pt}metis\ conc{\isacharunderscore}{\kern0pt}pow{\isacharunderscore}{\kern0pt}comm\ lang{\isacharunderscore}{\kern0pt}pow{\isachardot}{\kern0pt}simps{\isacharparenleft}{\kern0pt}{\isadigit{2}}{\isacharparenright}{\kern0pt}\ subsetD{\isacharparenright}{\kern0pt}\isanewline
\ \ \isakeywordONE{then}\isamarkupfalse%
\ \isakeywordONE{have}\isamarkupfalse%
\ {\isachardoublequoteopen}v\ {\isasymin}\ {\isasymPsi}\ {\isacharparenleft}{\kern0pt}star\ A\ {\isacharat}{\kern0pt}{\isacharat}{\kern0pt}\ B\ {\isacharcircum}{\kern0pt}{\isacharcircum}{\kern0pt}\ n\ {\isacharat}{\kern0pt}{\isacharat}{\kern0pt}\ B{\isacharparenright}{\kern0pt}{\isachardoublequoteclose}\isanewline
\ \ \ \ \isakeywordONE{unfolding}\isamarkupfalse%
\ star{\isacharunderscore}{\kern0pt}def\ \isakeywordONE{using}\isamarkupfalse%
\ parikh{\isacharunderscore}{\kern0pt}conc{\isacharunderscore}{\kern0pt}right{\isacharunderscore}{\kern0pt}subset\isanewline
\ \ \ \ \isakeywordONE{by}\isamarkupfalse%
\ {\isacharparenleft}{\kern0pt}metis\ {\isacharparenleft}{\kern0pt}no{\isacharunderscore}{\kern0pt}types{\isacharcomma}{\kern0pt}\ lifting{\isacharparenright}{\kern0pt}\ Sup{\isacharunderscore}{\kern0pt}upper\ parikh{\isacharunderscore}{\kern0pt}img{\isacharunderscore}{\kern0pt}mono\ rangeI\ subset{\isacharunderscore}{\kern0pt}eq{\isacharparenright}{\kern0pt}\isanewline
\ \ \isakeywordONE{then}\isamarkupfalse%
\ \isakeywordTHREE{show}\isamarkupfalse%
\ {\isachardoublequoteopen}v\ {\isasymin}\ {\isasymPsi}\ {\isacharparenleft}{\kern0pt}star\ A\ {\isacharat}{\kern0pt}{\isacharat}{\kern0pt}\ star\ B\ {\isacharat}{\kern0pt}{\isacharat}{\kern0pt}\ B{\isacharparenright}{\kern0pt}{\isachardoublequoteclose}\isanewline
\ \ \ \ \isakeywordONE{unfolding}\isamarkupfalse%
\ star{\isacharunderscore}{\kern0pt}def\ \isakeywordONE{using}\isamarkupfalse%
\ parikh{\isacharunderscore}{\kern0pt}conc{\isacharunderscore}{\kern0pt}right{\isacharunderscore}{\kern0pt}subset\ parikh{\isacharunderscore}{\kern0pt}conc{\isacharunderscore}{\kern0pt}left{\isacharunderscore}{\kern0pt}subset\isanewline
\ \ \ \ \isakeywordONE{by}\isamarkupfalse%
\ {\isacharparenleft}{\kern0pt}metis\ {\isacharparenleft}{\kern0pt}no{\isacharunderscore}{\kern0pt}types{\isacharcomma}{\kern0pt}\ lifting{\isacharparenright}{\kern0pt}\ Sup{\isacharunderscore}{\kern0pt}upper\ parikh{\isacharunderscore}{\kern0pt}img{\isacharunderscore}{\kern0pt}mono\ rangeI\ subset{\isacharunderscore}{\kern0pt}eq{\isacharparenright}{\kern0pt}\isanewline
\isakeywordONE{qed}\isamarkupfalse%
%
\endisatagproof
{\isafoldproof}%
%
\isadelimproof
\isanewline
%
\endisadelimproof
\isanewline
\isakeywordONE{lemma}\isamarkupfalse%
\ parikh{\isacharunderscore}{\kern0pt}img{\isacharunderscore}{\kern0pt}star{\isadigit{2}}{\isacharunderscore}{\kern0pt}aux{\isadigit{1}}{\isacharcolon}{\kern0pt}\isanewline
\ \ {\isachardoublequoteopen}{\isasymPsi}\ {\isacharparenleft}{\kern0pt}star\ {\isacharparenleft}{\kern0pt}star\ E\ {\isacharat}{\kern0pt}{\isacharat}{\kern0pt}\ F{\isacharparenright}{\kern0pt}{\isacharparenright}{\kern0pt}\ {\isasymsubseteq}\ {\isasymPsi}\ {\isacharparenleft}{\kern0pt}{\isacharbraceleft}{\kern0pt}{\isacharbrackleft}{\kern0pt}{\isacharbrackright}{\kern0pt}{\isacharbraceright}{\kern0pt}\ {\isasymunion}\ star\ E\ {\isacharat}{\kern0pt}{\isacharat}{\kern0pt}\ star\ F\ {\isacharat}{\kern0pt}{\isacharat}{\kern0pt}\ F{\isacharparenright}{\kern0pt}{\isachardoublequoteclose}\isanewline
%
\isadelimproof
%
\endisadelimproof
%
\isatagproof
\isakeywordONE{proof}\isamarkupfalse%
\isanewline
\ \ \isakeywordTHREE{fix}\isamarkupfalse%
\ v\isanewline
\ \ \isakeywordTHREE{assume}\isamarkupfalse%
\ {\isachardoublequoteopen}v\ {\isasymin}\ {\isasymPsi}\ {\isacharparenleft}{\kern0pt}star\ {\isacharparenleft}{\kern0pt}star\ E\ {\isacharat}{\kern0pt}{\isacharat}{\kern0pt}\ F{\isacharparenright}{\kern0pt}{\isacharparenright}{\kern0pt}{\isachardoublequoteclose}\isanewline
\ \ \isakeywordONE{then}\isamarkupfalse%
\ \isakeywordONE{have}\isamarkupfalse%
\ {\isachardoublequoteopen}{\isasymexists}n{\isachardot}{\kern0pt}\ v\ {\isasymin}\ {\isasymPsi}\ {\isacharparenleft}{\kern0pt}{\isacharparenleft}{\kern0pt}star\ E\ {\isacharat}{\kern0pt}{\isacharat}{\kern0pt}\ F{\isacharparenright}{\kern0pt}\ {\isacharcircum}{\kern0pt}{\isacharcircum}{\kern0pt}\ n{\isacharparenright}{\kern0pt}{\isachardoublequoteclose}\isanewline
\ \ \ \ \isakeywordONE{unfolding}\isamarkupfalse%
\ star{\isacharunderscore}{\kern0pt}def\ \isakeywordONE{by}\isamarkupfalse%
\ {\isacharparenleft}{\kern0pt}simp\ add{\isacharcolon}{\kern0pt}\ parikh{\isacharunderscore}{\kern0pt}img{\isacharunderscore}{\kern0pt}UNION{\isacharparenright}{\kern0pt}\isanewline
\ \ \isakeywordONE{then}\isamarkupfalse%
\ \isakeywordTHREE{obtain}\isamarkupfalse%
\ n\ \isakeywordTWO{where}\ v{\isacharunderscore}{\kern0pt}in{\isacharunderscore}{\kern0pt}pow{\isacharunderscore}{\kern0pt}n{\isacharcolon}{\kern0pt}\ {\isachardoublequoteopen}v\ {\isasymin}\ {\isasymPsi}\ {\isacharparenleft}{\kern0pt}{\isacharparenleft}{\kern0pt}star\ E\ {\isacharat}{\kern0pt}{\isacharat}{\kern0pt}\ F{\isacharparenright}{\kern0pt}\ {\isacharcircum}{\kern0pt}{\isacharcircum}{\kern0pt}\ n{\isacharparenright}{\kern0pt}{\isachardoublequoteclose}\ \isakeywordONE{by}\isamarkupfalse%
\ blast\isanewline
\ \ \isakeywordTHREE{show}\isamarkupfalse%
\ {\isachardoublequoteopen}v\ {\isasymin}\ {\isasymPsi}\ {\isacharparenleft}{\kern0pt}{\isacharbraceleft}{\kern0pt}{\isacharbrackleft}{\kern0pt}{\isacharbrackright}{\kern0pt}{\isacharbraceright}{\kern0pt}\ {\isasymunion}\ star\ E\ {\isacharat}{\kern0pt}{\isacharat}{\kern0pt}\ star\ F\ {\isacharat}{\kern0pt}{\isacharat}{\kern0pt}\ F{\isacharparenright}{\kern0pt}{\isachardoublequoteclose}\isanewline
\ \ \isakeywordONE{proof}\isamarkupfalse%
\ {\isacharparenleft}{\kern0pt}cases\ n{\isacharparenright}{\kern0pt}\isanewline
\ \ \ \ \isakeywordTHREE{case}\isamarkupfalse%
\ {\isadigit{0}}\isanewline
\ \ \ \ \isakeywordONE{with}\isamarkupfalse%
\ v{\isacharunderscore}{\kern0pt}in{\isacharunderscore}{\kern0pt}pow{\isacharunderscore}{\kern0pt}n\ \isakeywordONE{have}\isamarkupfalse%
\ {\isachardoublequoteopen}v\ {\isacharequal}{\kern0pt}\ parikh{\isacharunderscore}{\kern0pt}vec\ {\isacharbrackleft}{\kern0pt}{\isacharbrackright}{\kern0pt}{\isachardoublequoteclose}\ \isakeywordONE{unfolding}\isamarkupfalse%
\ parikh{\isacharunderscore}{\kern0pt}img{\isacharunderscore}{\kern0pt}def\ \isakeywordONE{by}\isamarkupfalse%
\ simp\isanewline
\ \ \ \ \isakeywordONE{then}\isamarkupfalse%
\ \isakeywordTHREE{show}\isamarkupfalse%
\ {\isacharquery}{\kern0pt}thesis\ \isakeywordONE{unfolding}\isamarkupfalse%
\ parikh{\isacharunderscore}{\kern0pt}img{\isacharunderscore}{\kern0pt}def\ \isakeywordONE{by}\isamarkupfalse%
\ blast\isanewline
\ \ \isakeywordONE{next}\isamarkupfalse%
\isanewline
\ \ \ \ \isakeywordTHREE{case}\isamarkupfalse%
\ {\isacharparenleft}{\kern0pt}Suc\ m{\isacharparenright}{\kern0pt}\isanewline
\ \ \ \ \isakeywordONE{with}\isamarkupfalse%
\ parikh{\isacharunderscore}{\kern0pt}img{\isacharunderscore}{\kern0pt}conc{\isacharunderscore}{\kern0pt}pow{\isadigit{2}}\ v{\isacharunderscore}{\kern0pt}in{\isacharunderscore}{\kern0pt}pow{\isacharunderscore}{\kern0pt}n\ \isakeywordONE{have}\isamarkupfalse%
\ {\isachardoublequoteopen}v\ {\isasymin}\ {\isasymPsi}\ {\isacharparenleft}{\kern0pt}star\ {\isacharparenleft}{\kern0pt}star\ E{\isacharparenright}{\kern0pt}\ {\isacharat}{\kern0pt}{\isacharat}{\kern0pt}\ star\ F\ {\isacharat}{\kern0pt}{\isacharat}{\kern0pt}\ F{\isacharparenright}{\kern0pt}{\isachardoublequoteclose}\ \isakeywordONE{by}\isamarkupfalse%
\ blast\isanewline
\ \ \ \ \isakeywordONE{then}\isamarkupfalse%
\ \isakeywordTHREE{show}\isamarkupfalse%
\ {\isacharquery}{\kern0pt}thesis\ \isakeywordONE{by}\isamarkupfalse%
\ {\isacharparenleft}{\kern0pt}metis\ UnCI\ parikh{\isacharunderscore}{\kern0pt}img{\isacharunderscore}{\kern0pt}Un\ star{\isacharunderscore}{\kern0pt}idemp{\isacharparenright}{\kern0pt}\isanewline
\ \ \isakeywordONE{qed}\isamarkupfalse%
\isanewline
\isakeywordONE{qed}\isamarkupfalse%
%
\endisatagproof
{\isafoldproof}%
%
\isadelimproof
\isanewline
%
\endisadelimproof
\isanewline
\isakeywordONE{lemma}\isamarkupfalse%
\ parikh{\isacharunderscore}{\kern0pt}img{\isacharunderscore}{\kern0pt}star{\isadigit{2}}{\isacharunderscore}{\kern0pt}aux{\isadigit{2}}{\isacharcolon}{\kern0pt}\ {\isachardoublequoteopen}{\isasymPsi}\ {\isacharparenleft}{\kern0pt}star\ E\ {\isacharat}{\kern0pt}{\isacharat}{\kern0pt}\ star\ F\ {\isacharat}{\kern0pt}{\isacharat}{\kern0pt}\ F{\isacharparenright}{\kern0pt}\ {\isasymsubseteq}\ {\isasymPsi}\ {\isacharparenleft}{\kern0pt}star\ {\isacharparenleft}{\kern0pt}star\ E\ {\isacharat}{\kern0pt}{\isacharat}{\kern0pt}\ F{\isacharparenright}{\kern0pt}{\isacharparenright}{\kern0pt}{\isachardoublequoteclose}\isanewline
%
\isadelimproof
%
\endisadelimproof
%
\isatagproof
\isakeywordONE{proof}\isamarkupfalse%
\ {\isacharminus}{\kern0pt}\isanewline
\ \ \isakeywordONE{have}\isamarkupfalse%
\ {\isachardoublequoteopen}F\ {\isasymsubseteq}\ star\ E\ {\isacharat}{\kern0pt}{\isacharat}{\kern0pt}\ F{\isachardoublequoteclose}\ \isakeywordONE{unfolding}\isamarkupfalse%
\ star{\isacharunderscore}{\kern0pt}def\ \isakeywordONE{using}\isamarkupfalse%
\ Nil{\isacharunderscore}{\kern0pt}in{\isacharunderscore}{\kern0pt}star\isanewline
\ \ \ \ \isakeywordONE{by}\isamarkupfalse%
\ {\isacharparenleft}{\kern0pt}metis\ concI{\isacharunderscore}{\kern0pt}if{\isacharunderscore}{\kern0pt}Nil{\isadigit{1}}\ star{\isacharunderscore}{\kern0pt}def\ subsetI{\isacharparenright}{\kern0pt}\isanewline
\ \ \isakeywordONE{then}\isamarkupfalse%
\ \isakeywordONE{have}\isamarkupfalse%
\ {\isachardoublequoteopen}{\isasymPsi}\ {\isacharparenleft}{\kern0pt}star\ E\ {\isacharat}{\kern0pt}{\isacharat}{\kern0pt}\ F\ {\isacharat}{\kern0pt}{\isacharat}{\kern0pt}\ star\ F{\isacharparenright}{\kern0pt}\ {\isasymsubseteq}\ {\isasymPsi}\ {\isacharparenleft}{\kern0pt}star\ E\ {\isacharat}{\kern0pt}{\isacharat}{\kern0pt}\ F\ {\isacharat}{\kern0pt}{\isacharat}{\kern0pt}\ star\ {\isacharparenleft}{\kern0pt}star\ E\ {\isacharat}{\kern0pt}{\isacharat}{\kern0pt}\ F{\isacharparenright}{\kern0pt}{\isacharparenright}{\kern0pt}{\isachardoublequoteclose}\isanewline
\ \ \ \ \isakeywordONE{using}\isamarkupfalse%
\ parikh{\isacharunderscore}{\kern0pt}conc{\isacharunderscore}{\kern0pt}left{\isacharunderscore}{\kern0pt}subset\ parikh{\isacharunderscore}{\kern0pt}img{\isacharunderscore}{\kern0pt}mono\ parikh{\isacharunderscore}{\kern0pt}star{\isacharunderscore}{\kern0pt}mono\ \isakeywordONE{by}\isamarkupfalse%
\ meson\isanewline
\ \ \isakeywordONE{also}\isamarkupfalse%
\ \isakeywordONE{have}\isamarkupfalse%
\ {\isachardoublequoteopen}{\isasymdots}\ {\isasymsubseteq}\ {\isasymPsi}\ {\isacharparenleft}{\kern0pt}star\ {\isacharparenleft}{\kern0pt}star\ E\ {\isacharat}{\kern0pt}{\isacharat}{\kern0pt}\ F{\isacharparenright}{\kern0pt}{\isacharparenright}{\kern0pt}{\isachardoublequoteclose}\isanewline
\ \ \ \ \isakeywordONE{by}\isamarkupfalse%
\ {\isacharparenleft}{\kern0pt}metis\ conc{\isacharunderscore}{\kern0pt}assoc\ inf{\isacharunderscore}{\kern0pt}sup{\isacharunderscore}{\kern0pt}ord{\isacharparenleft}{\kern0pt}{\isadigit{3}}{\isacharparenright}{\kern0pt}\ parikh{\isacharunderscore}{\kern0pt}img{\isacharunderscore}{\kern0pt}mono\ star{\isacharunderscore}{\kern0pt}unfold{\isacharunderscore}{\kern0pt}left{\isacharparenright}{\kern0pt}\isanewline
\ \ \isakeywordONE{finally}\isamarkupfalse%
\ \isakeywordTHREE{show}\isamarkupfalse%
\ {\isacharquery}{\kern0pt}thesis\ \isakeywordONE{using}\isamarkupfalse%
\ conc{\isacharunderscore}{\kern0pt}star{\isacharunderscore}{\kern0pt}comm\ \isakeywordONE{by}\isamarkupfalse%
\ metis\isanewline
\isakeywordONE{qed}\isamarkupfalse%
%
\endisatagproof
{\isafoldproof}%
%
\isadelimproof
\isanewline
%
\endisadelimproof
\isanewline
\isakeywordONE{lemma}\isamarkupfalse%
\ parikh{\isacharunderscore}{\kern0pt}img{\isacharunderscore}{\kern0pt}star{\isadigit{2}}{\isacharcolon}{\kern0pt}\ {\isachardoublequoteopen}{\isasymPsi}\ {\isacharparenleft}{\kern0pt}star\ {\isacharparenleft}{\kern0pt}star\ E\ {\isacharat}{\kern0pt}{\isacharat}{\kern0pt}\ F{\isacharparenright}{\kern0pt}{\isacharparenright}{\kern0pt}\ {\isacharequal}{\kern0pt}\ {\isasymPsi}\ {\isacharparenleft}{\kern0pt}{\isacharbraceleft}{\kern0pt}{\isacharbrackleft}{\kern0pt}{\isacharbrackright}{\kern0pt}{\isacharbraceright}{\kern0pt}\ {\isasymunion}\ star\ E\ {\isacharat}{\kern0pt}{\isacharat}{\kern0pt}\ star\ F\ {\isacharat}{\kern0pt}{\isacharat}{\kern0pt}\ F{\isacharparenright}{\kern0pt}{\isachardoublequoteclose}\isanewline
%
\isadelimproof
%
\endisadelimproof
%
\isatagproof
\isakeywordONE{proof}\isamarkupfalse%
\isanewline
\ \ \isakeywordONE{from}\isamarkupfalse%
\ parikh{\isacharunderscore}{\kern0pt}img{\isacharunderscore}{\kern0pt}star{\isadigit{2}}{\isacharunderscore}{\kern0pt}aux{\isadigit{1}}\isanewline
\ \ \ \ \isakeywordTHREE{show}\isamarkupfalse%
\ {\isachardoublequoteopen}{\isasymPsi}\ {\isacharparenleft}{\kern0pt}star\ {\isacharparenleft}{\kern0pt}star\ E\ {\isacharat}{\kern0pt}{\isacharat}{\kern0pt}\ F{\isacharparenright}{\kern0pt}{\isacharparenright}{\kern0pt}\ {\isasymsubseteq}\ {\isasymPsi}\ {\isacharparenleft}{\kern0pt}{\isacharbraceleft}{\kern0pt}{\isacharbrackleft}{\kern0pt}{\isacharbrackright}{\kern0pt}{\isacharbraceright}{\kern0pt}\ {\isasymunion}\ star\ E\ {\isacharat}{\kern0pt}{\isacharat}{\kern0pt}\ star\ F\ {\isacharat}{\kern0pt}{\isacharat}{\kern0pt}\ F{\isacharparenright}{\kern0pt}{\isachardoublequoteclose}\ \isakeywordONE{{\isachardot}{\kern0pt}}\isamarkupfalse%
\isanewline
\ \ \isakeywordONE{from}\isamarkupfalse%
\ parikh{\isacharunderscore}{\kern0pt}img{\isacharunderscore}{\kern0pt}star{\isadigit{2}}{\isacharunderscore}{\kern0pt}aux{\isadigit{2}}\isanewline
\ \ \ \ \isakeywordTHREE{show}\isamarkupfalse%
\ {\isachardoublequoteopen}{\isasymPsi}\ {\isacharparenleft}{\kern0pt}{\isacharbraceleft}{\kern0pt}{\isacharbrackleft}{\kern0pt}{\isacharbrackright}{\kern0pt}{\isacharbraceright}{\kern0pt}\ {\isasymunion}\ star\ E\ {\isacharat}{\kern0pt}{\isacharat}{\kern0pt}\ star\ F\ {\isacharat}{\kern0pt}{\isacharat}{\kern0pt}\ F{\isacharparenright}{\kern0pt}\ {\isasymsubseteq}\ {\isasymPsi}\ {\isacharparenleft}{\kern0pt}star\ {\isacharparenleft}{\kern0pt}star\ E\ {\isacharat}{\kern0pt}{\isacharat}{\kern0pt}\ F{\isacharparenright}{\kern0pt}{\isacharparenright}{\kern0pt}{\isachardoublequoteclose}\isanewline
\ \ \ \ \isakeywordONE{by}\isamarkupfalse%
\ {\isacharparenleft}{\kern0pt}metis\ le{\isacharunderscore}{\kern0pt}sup{\isacharunderscore}{\kern0pt}iff\ parikh{\isacharunderscore}{\kern0pt}img{\isacharunderscore}{\kern0pt}Un\ star{\isacharunderscore}{\kern0pt}unfold{\isacharunderscore}{\kern0pt}left\ sup{\isachardot}{\kern0pt}cobounded{\isadigit{2}}{\isacharparenright}{\kern0pt}\isanewline
\isakeywordONE{qed}\isamarkupfalse%
%
\endisatagproof
{\isafoldproof}%
%
\isadelimproof
%
\endisadelimproof
%
\isadelimdocument
%
\endisadelimdocument
%
\isatagdocument
%
\isamarkupsubsection{A homogeneous-like property for regular functions%
}
\isamarkuptrue%
%
\endisatagdocument
{\isafolddocument}%
%
\isadelimdocument
%
\endisadelimdocument
\isakeywordONE{lemma}\isamarkupfalse%
\ rlexp{\isacharunderscore}{\kern0pt}homogeneous{\isacharunderscore}{\kern0pt}aux{\isacharcolon}{\kern0pt}\isanewline
\ \ \isakeywordTWO{assumes}\ {\isachardoublequoteopen}v\ x\ {\isacharequal}{\kern0pt}\ star\ Y\ {\isacharat}{\kern0pt}{\isacharat}{\kern0pt}\ Z{\isachardoublequoteclose}\isanewline
\ \ \ \ \isakeywordTWO{shows}\ {\isachardoublequoteopen}{\isasymPsi}\ {\isacharparenleft}{\kern0pt}eval\ f\ v{\isacharparenright}{\kern0pt}\ {\isasymsubseteq}\ {\isasymPsi}\ {\isacharparenleft}{\kern0pt}star\ Y\ {\isacharat}{\kern0pt}{\isacharat}{\kern0pt}\ eval\ f\ {\isacharparenleft}{\kern0pt}v{\isacharparenleft}{\kern0pt}x\ {\isacharcolon}{\kern0pt}{\isacharequal}{\kern0pt}\ Z{\isacharparenright}{\kern0pt}{\isacharparenright}{\kern0pt}{\isacharparenright}{\kern0pt}{\isachardoublequoteclose}\isanewline
%
\isadelimproof
%
\endisadelimproof
%
\isatagproof
\isakeywordONE{proof}\isamarkupfalse%
\ {\isacharparenleft}{\kern0pt}induction\ f{\isacharparenright}{\kern0pt}\isanewline
\ \ \isakeywordTHREE{case}\isamarkupfalse%
\ {\isacharparenleft}{\kern0pt}Var\ y{\isacharparenright}{\kern0pt}\isanewline
\ \ \isakeywordTHREE{show}\isamarkupfalse%
\ {\isacharquery}{\kern0pt}case\isanewline
\ \ \isakeywordONE{proof}\isamarkupfalse%
\ {\isacharparenleft}{\kern0pt}cases\ {\isachardoublequoteopen}x\ {\isacharequal}{\kern0pt}\ y{\isachardoublequoteclose}{\isacharparenright}{\kern0pt}\isanewline
\ \ \ \ \isakeywordTHREE{case}\isamarkupfalse%
\ True\isanewline
\ \ \ \ \isakeywordONE{with}\isamarkupfalse%
\ Var\ assms\ \isakeywordTHREE{show}\isamarkupfalse%
\ {\isacharquery}{\kern0pt}thesis\ \isakeywordONE{by}\isamarkupfalse%
\ simp\isanewline
\ \ \isakeywordONE{next}\isamarkupfalse%
\isanewline
\ \ \ \ \isakeywordTHREE{case}\isamarkupfalse%
\ False\isanewline
\ \ \ \ \isakeywordONE{have}\isamarkupfalse%
\ {\isachardoublequoteopen}eval\ {\isacharparenleft}{\kern0pt}Var\ y{\isacharparenright}{\kern0pt}\ v\ {\isasymsubseteq}\ star\ Y\ {\isacharat}{\kern0pt}{\isacharat}{\kern0pt}\ eval\ {\isacharparenleft}{\kern0pt}Var\ y{\isacharparenright}{\kern0pt}\ v{\isachardoublequoteclose}\ \isakeywordONE{by}\isamarkupfalse%
\ {\isacharparenleft}{\kern0pt}metis\ Nil{\isacharunderscore}{\kern0pt}in{\isacharunderscore}{\kern0pt}star\ concI{\isacharunderscore}{\kern0pt}if{\isacharunderscore}{\kern0pt}Nil{\isadigit{1}}\ subsetI{\isacharparenright}{\kern0pt}\isanewline
\ \ \ \ \isakeywordONE{with}\isamarkupfalse%
\ False\ parikh{\isacharunderscore}{\kern0pt}img{\isacharunderscore}{\kern0pt}mono\ \isakeywordTHREE{show}\isamarkupfalse%
\ {\isacharquery}{\kern0pt}thesis\ \isakeywordONE{by}\isamarkupfalse%
\ auto\isanewline
\ \ \isakeywordONE{qed}\isamarkupfalse%
\isanewline
\isakeywordONE{next}\isamarkupfalse%
\isanewline
\ \ \isakeywordTHREE{case}\isamarkupfalse%
\ {\isacharparenleft}{\kern0pt}Const\ l{\isacharparenright}{\kern0pt}\isanewline
\ \ \isakeywordONE{have}\isamarkupfalse%
\ {\isachardoublequoteopen}eval\ {\isacharparenleft}{\kern0pt}Const\ l{\isacharparenright}{\kern0pt}\ v\ {\isasymsubseteq}\ star\ Y\ {\isacharat}{\kern0pt}{\isacharat}{\kern0pt}\ eval\ {\isacharparenleft}{\kern0pt}Const\ l{\isacharparenright}{\kern0pt}\ v{\isachardoublequoteclose}\ \isakeywordONE{using}\isamarkupfalse%
\ concI{\isacharunderscore}{\kern0pt}if{\isacharunderscore}{\kern0pt}Nil{\isadigit{1}}\ \isakeywordONE{by}\isamarkupfalse%
\ blast\isanewline
\ \ \isakeywordONE{then}\isamarkupfalse%
\ \isakeywordTHREE{show}\isamarkupfalse%
\ {\isacharquery}{\kern0pt}case\ \isakeywordONE{by}\isamarkupfalse%
\ {\isacharparenleft}{\kern0pt}simp\ add{\isacharcolon}{\kern0pt}\ parikh{\isacharunderscore}{\kern0pt}img{\isacharunderscore}{\kern0pt}mono{\isacharparenright}{\kern0pt}\isanewline
\isakeywordONE{next}\isamarkupfalse%
\isanewline
\ \ \isakeywordTHREE{case}\isamarkupfalse%
\ {\isacharparenleft}{\kern0pt}Union\ f\ g{\isacharparenright}{\kern0pt}\isanewline
\ \ \isakeywordONE{then}\isamarkupfalse%
\ \isakeywordONE{have}\isamarkupfalse%
\ {\isachardoublequoteopen}{\isasymPsi}\ {\isacharparenleft}{\kern0pt}eval\ {\isacharparenleft}{\kern0pt}Union\ f\ g{\isacharparenright}{\kern0pt}\ v{\isacharparenright}{\kern0pt}\ {\isasymsubseteq}\ {\isasymPsi}\ {\isacharparenleft}{\kern0pt}star\ Y\ {\isacharat}{\kern0pt}{\isacharat}{\kern0pt}\ eval\ f\ {\isacharparenleft}{\kern0pt}v{\isacharparenleft}{\kern0pt}x\ {\isacharcolon}{\kern0pt}{\isacharequal}{\kern0pt}\ Z{\isacharparenright}{\kern0pt}{\isacharparenright}{\kern0pt}\ {\isasymunion}\isanewline
\ \ \ \ \ \ \ \ \ \ \ \ \ \ \ \ \ \ \ \ \ \ \ \ \ \ \ \ \ \ \ \ \ \ \ \ \ \ \ \ \ \ \ \ \ \ \ \ \ \ \ \ \ \ \ \ \ \ \ \ star\ Y\ {\isacharat}{\kern0pt}{\isacharat}{\kern0pt}\ eval\ g\ {\isacharparenleft}{\kern0pt}v{\isacharparenleft}{\kern0pt}x\ {\isacharcolon}{\kern0pt}{\isacharequal}{\kern0pt}\ Z{\isacharparenright}{\kern0pt}{\isacharparenright}{\kern0pt}{\isacharparenright}{\kern0pt}{\isachardoublequoteclose}\isanewline
\ \ \ \ \isakeywordONE{by}\isamarkupfalse%
\ {\isacharparenleft}{\kern0pt}metis\ eval{\isachardot}{\kern0pt}simps{\isacharparenleft}{\kern0pt}{\isadigit{3}}{\isacharparenright}{\kern0pt}\ parikh{\isacharunderscore}{\kern0pt}img{\isacharunderscore}{\kern0pt}Un\ sup{\isachardot}{\kern0pt}mono{\isacharparenright}{\kern0pt}\isanewline
\ \ \isakeywordONE{then}\isamarkupfalse%
\ \isakeywordTHREE{show}\isamarkupfalse%
\ {\isacharquery}{\kern0pt}case\ \isakeywordONE{by}\isamarkupfalse%
\ {\isacharparenleft}{\kern0pt}metis\ conc{\isacharunderscore}{\kern0pt}Un{\isacharunderscore}{\kern0pt}distrib{\isacharparenleft}{\kern0pt}{\isadigit{1}}{\isacharparenright}{\kern0pt}\ eval{\isachardot}{\kern0pt}simps{\isacharparenleft}{\kern0pt}{\isadigit{3}}{\isacharparenright}{\kern0pt}{\isacharparenright}{\kern0pt}\isanewline
\isakeywordONE{next}\isamarkupfalse%
\isanewline
\ \ \isakeywordTHREE{case}\isamarkupfalse%
\ {\isacharparenleft}{\kern0pt}Concat\ f\ g{\isacharparenright}{\kern0pt}\isanewline
\ \ \isakeywordONE{then}\isamarkupfalse%
\ \isakeywordONE{have}\isamarkupfalse%
\ {\isachardoublequoteopen}{\isasymPsi}\ {\isacharparenleft}{\kern0pt}eval\ {\isacharparenleft}{\kern0pt}Concat\ f\ g{\isacharparenright}{\kern0pt}\ v{\isacharparenright}{\kern0pt}\ {\isasymsubseteq}\ {\isasymPsi}\ {\isacharparenleft}{\kern0pt}{\isacharparenleft}{\kern0pt}star\ Y\ {\isacharat}{\kern0pt}{\isacharat}{\kern0pt}\ eval\ f\ {\isacharparenleft}{\kern0pt}v{\isacharparenleft}{\kern0pt}x\ {\isacharcolon}{\kern0pt}{\isacharequal}{\kern0pt}\ Z{\isacharparenright}{\kern0pt}{\isacharparenright}{\kern0pt}{\isacharparenright}{\kern0pt}\isanewline
\ \ \ \ \ \ \ \ \ \ \ \ \ \ \ \ \ \ \ \ \ \ \ \ \ \ \ \ \ \ \ \ \ \ \ \ \ \ \ \ \ \ \ \ \ \ \ \ \ \ \ \ \ \ \ \ \ \ {\isacharat}{\kern0pt}{\isacharat}{\kern0pt}\ star\ Y\ {\isacharat}{\kern0pt}{\isacharat}{\kern0pt}\ eval\ g\ {\isacharparenleft}{\kern0pt}v{\isacharparenleft}{\kern0pt}x\ {\isacharcolon}{\kern0pt}{\isacharequal}{\kern0pt}\ Z{\isacharparenright}{\kern0pt}{\isacharparenright}{\kern0pt}{\isacharparenright}{\kern0pt}{\isachardoublequoteclose}\isanewline
\ \ \ \ \isakeywordONE{by}\isamarkupfalse%
\ {\isacharparenleft}{\kern0pt}metis\ eval{\isachardot}{\kern0pt}simps{\isacharparenleft}{\kern0pt}{\isadigit{4}}{\isacharparenright}{\kern0pt}\ parikh{\isacharunderscore}{\kern0pt}conc{\isacharunderscore}{\kern0pt}subset{\isacharparenright}{\kern0pt}\isanewline
\ \ \isakeywordONE{also}\isamarkupfalse%
\ \isakeywordONE{have}\isamarkupfalse%
\ {\isachardoublequoteopen}{\isasymdots}\ {\isacharequal}{\kern0pt}\ {\isasymPsi}\ {\isacharparenleft}{\kern0pt}star\ Y\ {\isacharat}{\kern0pt}{\isacharat}{\kern0pt}\ star\ Y\ {\isacharat}{\kern0pt}{\isacharat}{\kern0pt}\ eval\ f\ {\isacharparenleft}{\kern0pt}v{\isacharparenleft}{\kern0pt}x\ {\isacharcolon}{\kern0pt}{\isacharequal}{\kern0pt}\ Z{\isacharparenright}{\kern0pt}{\isacharparenright}{\kern0pt}\ {\isacharat}{\kern0pt}{\isacharat}{\kern0pt}\ eval\ g\ {\isacharparenleft}{\kern0pt}v{\isacharparenleft}{\kern0pt}x\ {\isacharcolon}{\kern0pt}{\isacharequal}{\kern0pt}\ Z{\isacharparenright}{\kern0pt}{\isacharparenright}{\kern0pt}{\isacharparenright}{\kern0pt}{\isachardoublequoteclose}\isanewline
\ \ \ \ \isakeywordONE{by}\isamarkupfalse%
\ {\isacharparenleft}{\kern0pt}metis\ conc{\isacharunderscore}{\kern0pt}assoc\ parikh{\isacharunderscore}{\kern0pt}conc{\isacharunderscore}{\kern0pt}right\ parikh{\isacharunderscore}{\kern0pt}img{\isacharunderscore}{\kern0pt}commut{\isacharparenright}{\kern0pt}\isanewline
\ \ \isakeywordONE{also}\isamarkupfalse%
\ \isakeywordONE{have}\isamarkupfalse%
\ {\isachardoublequoteopen}{\isasymdots}\ {\isacharequal}{\kern0pt}\ {\isasymPsi}\ {\isacharparenleft}{\kern0pt}star\ Y\ {\isacharat}{\kern0pt}{\isacharat}{\kern0pt}\ eval\ f\ {\isacharparenleft}{\kern0pt}v{\isacharparenleft}{\kern0pt}x\ {\isacharcolon}{\kern0pt}{\isacharequal}{\kern0pt}\ Z{\isacharparenright}{\kern0pt}{\isacharparenright}{\kern0pt}\ {\isacharat}{\kern0pt}{\isacharat}{\kern0pt}\ eval\ g\ {\isacharparenleft}{\kern0pt}v{\isacharparenleft}{\kern0pt}x\ {\isacharcolon}{\kern0pt}{\isacharequal}{\kern0pt}\ Z{\isacharparenright}{\kern0pt}{\isacharparenright}{\kern0pt}{\isacharparenright}{\kern0pt}{\isachardoublequoteclose}\isanewline
\ \ \ \ \isakeywordONE{by}\isamarkupfalse%
\ {\isacharparenleft}{\kern0pt}metis\ conc{\isacharunderscore}{\kern0pt}assoc\ conc{\isacharunderscore}{\kern0pt}star{\isacharunderscore}{\kern0pt}star{\isacharparenright}{\kern0pt}\isanewline
\ \ \isakeywordONE{finally}\isamarkupfalse%
\ \isakeywordTHREE{show}\isamarkupfalse%
\ {\isacharquery}{\kern0pt}case\ \isakeywordONE{by}\isamarkupfalse%
\ {\isacharparenleft}{\kern0pt}metis\ eval{\isachardot}{\kern0pt}simps{\isacharparenleft}{\kern0pt}{\isadigit{4}}{\isacharparenright}{\kern0pt}{\isacharparenright}{\kern0pt}\isanewline
\isakeywordONE{next}\isamarkupfalse%
\isanewline
\ \ \isakeywordTHREE{case}\isamarkupfalse%
\ {\isacharparenleft}{\kern0pt}Star\ f{\isacharparenright}{\kern0pt}\isanewline
\ \ \isakeywordONE{then}\isamarkupfalse%
\ \isakeywordONE{have}\isamarkupfalse%
\ {\isachardoublequoteopen}{\isasymPsi}\ {\isacharparenleft}{\kern0pt}star\ {\isacharparenleft}{\kern0pt}eval\ f\ v{\isacharparenright}{\kern0pt}{\isacharparenright}{\kern0pt}\ {\isasymsubseteq}\ {\isasymPsi}\ {\isacharparenleft}{\kern0pt}star\ {\isacharparenleft}{\kern0pt}star\ Y\ {\isacharat}{\kern0pt}{\isacharat}{\kern0pt}\ eval\ f\ {\isacharparenleft}{\kern0pt}v{\isacharparenleft}{\kern0pt}x\ {\isacharcolon}{\kern0pt}{\isacharequal}{\kern0pt}\ Z{\isacharparenright}{\kern0pt}{\isacharparenright}{\kern0pt}{\isacharparenright}{\kern0pt}{\isacharparenright}{\kern0pt}{\isachardoublequoteclose}\isanewline
\ \ \ \ \isakeywordONE{using}\isamarkupfalse%
\ parikh{\isacharunderscore}{\kern0pt}star{\isacharunderscore}{\kern0pt}mono\ \isakeywordONE{by}\isamarkupfalse%
\ metis\isanewline
\ \ \isakeywordONE{also}\isamarkupfalse%
\ \isakeywordONE{from}\isamarkupfalse%
\ parikh{\isacharunderscore}{\kern0pt}img{\isacharunderscore}{\kern0pt}conc{\isacharunderscore}{\kern0pt}star\ \isakeywordONE{have}\isamarkupfalse%
\ {\isachardoublequoteopen}{\isasymdots}\ {\isasymsubseteq}\ {\isasymPsi}\ {\isacharparenleft}{\kern0pt}star\ Y\ {\isacharat}{\kern0pt}{\isacharat}{\kern0pt}\ star\ {\isacharparenleft}{\kern0pt}eval\ f\ {\isacharparenleft}{\kern0pt}v{\isacharparenleft}{\kern0pt}x\ {\isacharcolon}{\kern0pt}{\isacharequal}{\kern0pt}\ Z{\isacharparenright}{\kern0pt}{\isacharparenright}{\kern0pt}{\isacharparenright}{\kern0pt}{\isacharparenright}{\kern0pt}{\isachardoublequoteclose}\isanewline
\ \ \ \ \isakeywordONE{by}\isamarkupfalse%
\ fastforce\isanewline
\ \ \isakeywordONE{finally}\isamarkupfalse%
\ \isakeywordTHREE{show}\isamarkupfalse%
\ {\isacharquery}{\kern0pt}case\ \isakeywordONE{by}\isamarkupfalse%
\ {\isacharparenleft}{\kern0pt}metis\ eval{\isachardot}{\kern0pt}simps{\isacharparenleft}{\kern0pt}{\isadigit{5}}{\isacharparenright}{\kern0pt}{\isacharparenright}{\kern0pt}\isanewline
\isakeywordONE{qed}\isamarkupfalse%
%
\endisatagproof
{\isafoldproof}%
%
\isadelimproof
%
\endisadelimproof
%
\begin{isamarkuptext}%
Now we can prove the desired homogeneous-like property which will become useful later.
Notably this property slightly differs from the property claimed in \cite{Pilling}. However, our
property is easier to prove formally and it suffices for the rest of the proof.%
\end{isamarkuptext}\isamarkuptrue%
\isakeywordONE{lemma}\isamarkupfalse%
\ rlexp{\isacharunderscore}{\kern0pt}homogeneous{\isacharcolon}{\kern0pt}\ \ {\isachardoublequoteopen}{\isasymPsi}\ {\isacharparenleft}{\kern0pt}eval\ {\isacharparenleft}{\kern0pt}subst\ {\isacharparenleft}{\kern0pt}Var{\isacharparenleft}{\kern0pt}x\ {\isacharcolon}{\kern0pt}{\isacharequal}{\kern0pt}\ Concat\ {\isacharparenleft}{\kern0pt}Star\ y{\isacharparenright}{\kern0pt}\ z{\isacharparenright}{\kern0pt}{\isacharparenright}{\kern0pt}\ f{\isacharparenright}{\kern0pt}\ v{\isacharparenright}{\kern0pt}\isanewline
\ \ \ \ \ \ \ \ \ \ \ \ \ \ \ \ \ \ \ \ \ \ \ \ \ \ {\isasymsubseteq}\ {\isasymPsi}\ {\isacharparenleft}{\kern0pt}eval\ {\isacharparenleft}{\kern0pt}Concat\ {\isacharparenleft}{\kern0pt}Star\ y{\isacharparenright}{\kern0pt}\ {\isacharparenleft}{\kern0pt}subst\ {\isacharparenleft}{\kern0pt}Var{\isacharparenleft}{\kern0pt}x\ {\isacharcolon}{\kern0pt}{\isacharequal}{\kern0pt}\ z{\isacharparenright}{\kern0pt}{\isacharparenright}{\kern0pt}\ f{\isacharparenright}{\kern0pt}{\isacharparenright}{\kern0pt}\ v{\isacharparenright}{\kern0pt}{\isachardoublequoteclose}\isanewline
\ \ \ \ \ \ \ \ \ \ \ \ \ \ \ \ \ \ \ \ \ \ \ \ \ \ {\isacharparenleft}{\kern0pt}\isakeywordTWO{is}\ {\isachardoublequoteopen}{\isasymPsi}\ {\isacharquery}{\kern0pt}L\ {\isasymsubseteq}\ {\isasymPsi}\ {\isacharquery}{\kern0pt}R{\isachardoublequoteclose}{\isacharparenright}{\kern0pt}\isanewline
%
\isadelimproof
%
\endisadelimproof
%
\isatagproof
\isakeywordONE{proof}\isamarkupfalse%
\ {\isacharminus}{\kern0pt}\isanewline
\ \ \isakeywordONE{let}\isamarkupfalse%
\ {\isacharquery}{\kern0pt}v{\isacharprime}{\kern0pt}\ {\isacharequal}{\kern0pt}\ {\isachardoublequoteopen}v{\isacharparenleft}{\kern0pt}x\ {\isacharcolon}{\kern0pt}{\isacharequal}{\kern0pt}\ star\ {\isacharparenleft}{\kern0pt}eval\ y\ v{\isacharparenright}{\kern0pt}\ {\isacharat}{\kern0pt}{\isacharat}{\kern0pt}\ eval\ z\ v{\isacharparenright}{\kern0pt}{\isachardoublequoteclose}\isanewline
\ \ \isakeywordONE{have}\isamarkupfalse%
\ {\isachardoublequoteopen}{\isasymPsi}\ {\isacharquery}{\kern0pt}L\ {\isacharequal}{\kern0pt}\ {\isasymPsi}\ {\isacharparenleft}{\kern0pt}eval\ f\ {\isacharquery}{\kern0pt}v{\isacharprime}{\kern0pt}{\isacharparenright}{\kern0pt}{\isachardoublequoteclose}\ \isakeywordONE{using}\isamarkupfalse%
\ substitution{\isacharunderscore}{\kern0pt}lemma{\isacharunderscore}{\kern0pt}upd{\isacharbrackleft}{\kern0pt}\isakeywordTWO{where}\ f{\isacharequal}{\kern0pt}f{\isacharbrackright}{\kern0pt}\ \isakeywordONE{by}\isamarkupfalse%
\ simp\isanewline
\ \ \isakeywordONE{also}\isamarkupfalse%
\ \isakeywordONE{have}\isamarkupfalse%
\ {\isachardoublequoteopen}{\isasymdots}\ {\isasymsubseteq}\ {\isasymPsi}\ {\isacharparenleft}{\kern0pt}star\ {\isacharparenleft}{\kern0pt}eval\ y\ v{\isacharparenright}{\kern0pt}\ {\isacharat}{\kern0pt}{\isacharat}{\kern0pt}\ eval\ f\ {\isacharparenleft}{\kern0pt}{\isacharquery}{\kern0pt}v{\isacharprime}{\kern0pt}{\isacharparenleft}{\kern0pt}x\ {\isacharcolon}{\kern0pt}{\isacharequal}{\kern0pt}\ eval\ z\ v{\isacharparenright}{\kern0pt}{\isacharparenright}{\kern0pt}{\isacharparenright}{\kern0pt}{\isachardoublequoteclose}\isanewline
\ \ \ \ \isakeywordONE{using}\isamarkupfalse%
\ rlexp{\isacharunderscore}{\kern0pt}homogeneous{\isacharunderscore}{\kern0pt}aux{\isacharbrackleft}{\kern0pt}of\ {\isacharquery}{\kern0pt}v{\isacharprime}{\kern0pt}{\isacharbrackright}{\kern0pt}\ \isakeywordONE{unfolding}\isamarkupfalse%
\ fun{\isacharunderscore}{\kern0pt}upd{\isacharunderscore}{\kern0pt}def\ \isakeywordONE{by}\isamarkupfalse%
\ auto\isanewline
\ \ \isakeywordONE{also}\isamarkupfalse%
\ \isakeywordONE{have}\isamarkupfalse%
\ {\isachardoublequoteopen}{\isasymdots}\ {\isacharequal}{\kern0pt}\ {\isasymPsi}\ {\isacharquery}{\kern0pt}R{\isachardoublequoteclose}\ \isakeywordONE{using}\isamarkupfalse%
\ substitution{\isacharunderscore}{\kern0pt}lemma{\isacharbrackleft}{\kern0pt}of\ {\isachardoublequoteopen}v{\isacharparenleft}{\kern0pt}x\ {\isacharcolon}{\kern0pt}{\isacharequal}{\kern0pt}\ eval\ z\ v{\isacharparenright}{\kern0pt}{\isachardoublequoteclose}{\isacharbrackright}{\kern0pt}\ \isakeywordONE{by}\isamarkupfalse%
\ simp\isanewline
\ \ \isakeywordONE{finally}\isamarkupfalse%
\ \isakeywordTHREE{show}\isamarkupfalse%
\ {\isacharquery}{\kern0pt}thesis\ \isakeywordONE{{\isachardot}{\kern0pt}}\isamarkupfalse%
\isanewline
\isakeywordONE{qed}\isamarkupfalse%
%
\endisatagproof
{\isafoldproof}%
%
\isadelimproof
%
\endisadelimproof
%
\isadelimdocument
%
\endisadelimdocument
%
\isatagdocument
%
\isamarkupsubsection{Extension of Arden's lemma to Parikh images%
}
\isamarkuptrue%
%
\endisatagdocument
{\isafolddocument}%
%
\isadelimdocument
%
\endisadelimdocument
\isakeywordONE{lemma}\isamarkupfalse%
\ parikh{\isacharunderscore}{\kern0pt}img{\isacharunderscore}{\kern0pt}arden{\isacharunderscore}{\kern0pt}aux{\isacharcolon}{\kern0pt}\isanewline
\ \ \isakeywordTWO{assumes}\ {\isachardoublequoteopen}{\isasymPsi}\ {\isacharparenleft}{\kern0pt}A\ {\isacharat}{\kern0pt}{\isacharat}{\kern0pt}\ X\ {\isasymunion}\ B{\isacharparenright}{\kern0pt}\ {\isasymsubseteq}\ {\isasymPsi}\ X{\isachardoublequoteclose}\isanewline
\ \ \isakeywordTWO{shows}\ {\isachardoublequoteopen}{\isasymPsi}\ {\isacharparenleft}{\kern0pt}A\ {\isacharcircum}{\kern0pt}{\isacharcircum}{\kern0pt}\ n\ {\isacharat}{\kern0pt}{\isacharat}{\kern0pt}\ B{\isacharparenright}{\kern0pt}\ {\isasymsubseteq}\ {\isasymPsi}\ X{\isachardoublequoteclose}\isanewline
%
\isadelimproof
%
\endisadelimproof
%
\isatagproof
\isakeywordONE{proof}\isamarkupfalse%
\ {\isacharparenleft}{\kern0pt}induction\ n{\isacharparenright}{\kern0pt}\isanewline
\ \ \isakeywordTHREE{case}\isamarkupfalse%
\ {\isadigit{0}}\isanewline
\ \ \isakeywordONE{with}\isamarkupfalse%
\ assms\ \isakeywordTHREE{show}\isamarkupfalse%
\ {\isacharquery}{\kern0pt}case\ \isakeywordONE{by}\isamarkupfalse%
\ auto\isanewline
\isakeywordONE{next}\isamarkupfalse%
\isanewline
\ \ \isakeywordTHREE{case}\isamarkupfalse%
\ {\isacharparenleft}{\kern0pt}Suc\ n{\isacharparenright}{\kern0pt}\isanewline
\ \ \isakeywordONE{then}\isamarkupfalse%
\ \isakeywordONE{have}\isamarkupfalse%
\ {\isachardoublequoteopen}{\isasymPsi}\ {\isacharparenleft}{\kern0pt}A\ {\isacharcircum}{\kern0pt}{\isacharcircum}{\kern0pt}\ {\isacharparenleft}{\kern0pt}Suc\ n{\isacharparenright}{\kern0pt}\ {\isacharat}{\kern0pt}{\isacharat}{\kern0pt}\ B{\isacharparenright}{\kern0pt}\ {\isasymsubseteq}\ {\isasymPsi}\ {\isacharparenleft}{\kern0pt}A\ {\isacharat}{\kern0pt}{\isacharat}{\kern0pt}\ A\ {\isacharcircum}{\kern0pt}{\isacharcircum}{\kern0pt}\ n\ {\isacharat}{\kern0pt}{\isacharat}{\kern0pt}B{\isacharparenright}{\kern0pt}{\isachardoublequoteclose}\isanewline
\ \ \ \ \isakeywordONE{by}\isamarkupfalse%
\ {\isacharparenleft}{\kern0pt}simp\ add{\isacharcolon}{\kern0pt}\ conc{\isacharunderscore}{\kern0pt}assoc{\isacharparenright}{\kern0pt}\isanewline
\ \ \isakeywordONE{moreover}\isamarkupfalse%
\ \isakeywordONE{from}\isamarkupfalse%
\ Suc\ parikh{\isacharunderscore}{\kern0pt}conc{\isacharunderscore}{\kern0pt}left\ \isakeywordONE{have}\isamarkupfalse%
\ {\isachardoublequoteopen}{\isasymdots}\ {\isasymsubseteq}\ {\isasymPsi}\ {\isacharparenleft}{\kern0pt}A\ {\isacharat}{\kern0pt}{\isacharat}{\kern0pt}\ X{\isacharparenright}{\kern0pt}{\isachardoublequoteclose}\isanewline
\ \ \ \ \isakeywordONE{by}\isamarkupfalse%
\ {\isacharparenleft}{\kern0pt}metis\ conc{\isacharunderscore}{\kern0pt}Un{\isacharunderscore}{\kern0pt}distrib{\isacharparenleft}{\kern0pt}{\isadigit{1}}{\isacharparenright}{\kern0pt}\ parikh{\isacharunderscore}{\kern0pt}img{\isacharunderscore}{\kern0pt}Un\ sup{\isachardot}{\kern0pt}orderE\ sup{\isachardot}{\kern0pt}orderI{\isacharparenright}{\kern0pt}\isanewline
\ \ \isakeywordONE{moreover}\isamarkupfalse%
\ \isakeywordONE{from}\isamarkupfalse%
\ Suc{\isachardot}{\kern0pt}prems\ assms\ \isakeywordONE{have}\isamarkupfalse%
\ {\isachardoublequoteopen}{\isasymdots}\ {\isasymsubseteq}\ {\isasymPsi}\ X{\isachardoublequoteclose}\ \isakeywordONE{by}\isamarkupfalse%
\ auto\isanewline
\ \ \isakeywordONE{ultimately}\isamarkupfalse%
\ \isakeywordTHREE{show}\isamarkupfalse%
\ {\isacharquery}{\kern0pt}case\ \isakeywordONE{by}\isamarkupfalse%
\ fast\isanewline
\isakeywordONE{qed}\isamarkupfalse%
%
\endisatagproof
{\isafoldproof}%
%
\isadelimproof
\isanewline
%
\endisadelimproof
\isanewline
\isakeywordONE{lemma}\isamarkupfalse%
\ parikh{\isacharunderscore}{\kern0pt}img{\isacharunderscore}{\kern0pt}arden{\isacharcolon}{\kern0pt}\isanewline
\ \ \isakeywordTWO{assumes}\ {\isachardoublequoteopen}{\isasymPsi}\ {\isacharparenleft}{\kern0pt}A\ {\isacharat}{\kern0pt}{\isacharat}{\kern0pt}\ X\ {\isasymunion}\ B{\isacharparenright}{\kern0pt}\ {\isasymsubseteq}\ {\isasymPsi}\ X{\isachardoublequoteclose}\isanewline
\ \ \isakeywordTWO{shows}\ {\isachardoublequoteopen}{\isasymPsi}\ {\isacharparenleft}{\kern0pt}star\ A\ {\isacharat}{\kern0pt}{\isacharat}{\kern0pt}\ B{\isacharparenright}{\kern0pt}\ {\isasymsubseteq}\ {\isasymPsi}\ X{\isachardoublequoteclose}\isanewline
%
\isadelimproof
%
\endisadelimproof
%
\isatagproof
\isakeywordONE{proof}\isamarkupfalse%
\isanewline
\ \ \isakeywordTHREE{fix}\isamarkupfalse%
\ x\isanewline
\ \ \isakeywordTHREE{assume}\isamarkupfalse%
\ {\isachardoublequoteopen}x\ {\isasymin}\ {\isasymPsi}\ {\isacharparenleft}{\kern0pt}star\ A\ {\isacharat}{\kern0pt}{\isacharat}{\kern0pt}\ B{\isacharparenright}{\kern0pt}{\isachardoublequoteclose}\isanewline
\ \ \isakeywordONE{then}\isamarkupfalse%
\ \isakeywordONE{have}\isamarkupfalse%
\ {\isachardoublequoteopen}{\isasymexists}n{\isachardot}{\kern0pt}\ x\ {\isasymin}\ {\isasymPsi}\ {\isacharparenleft}{\kern0pt}A\ {\isacharcircum}{\kern0pt}{\isacharcircum}{\kern0pt}\ n\ {\isacharat}{\kern0pt}{\isacharat}{\kern0pt}\ B{\isacharparenright}{\kern0pt}{\isachardoublequoteclose}\isanewline
\ \ \ \ \isakeywordONE{unfolding}\isamarkupfalse%
\ star{\isacharunderscore}{\kern0pt}def\ \isakeywordONE{by}\isamarkupfalse%
\ {\isacharparenleft}{\kern0pt}simp\ add{\isacharcolon}{\kern0pt}\ conc{\isacharunderscore}{\kern0pt}UNION{\isacharunderscore}{\kern0pt}distrib{\isacharparenleft}{\kern0pt}{\isadigit{2}}{\isacharparenright}{\kern0pt}\ parikh{\isacharunderscore}{\kern0pt}img{\isacharunderscore}{\kern0pt}UNION{\isacharparenright}{\kern0pt}\isanewline
\ \ \isakeywordONE{then}\isamarkupfalse%
\ \isakeywordTHREE{obtain}\isamarkupfalse%
\ n\ \isakeywordTWO{where}\ {\isachardoublequoteopen}x\ {\isasymin}\ {\isasymPsi}\ {\isacharparenleft}{\kern0pt}A\ {\isacharcircum}{\kern0pt}{\isacharcircum}{\kern0pt}\ n\ {\isacharat}{\kern0pt}{\isacharat}{\kern0pt}\ B{\isacharparenright}{\kern0pt}{\isachardoublequoteclose}\ \isakeywordONE{by}\isamarkupfalse%
\ blast\isanewline
\ \ \isakeywordONE{then}\isamarkupfalse%
\ \isakeywordTHREE{show}\isamarkupfalse%
\ {\isachardoublequoteopen}x\ {\isasymin}\ {\isasymPsi}\ X{\isachardoublequoteclose}\ \isakeywordONE{using}\isamarkupfalse%
\ parikh{\isacharunderscore}{\kern0pt}img{\isacharunderscore}{\kern0pt}arden{\isacharunderscore}{\kern0pt}aux{\isacharbrackleft}{\kern0pt}OF\ assms{\isacharbrackright}{\kern0pt}\ \isakeywordONE{by}\isamarkupfalse%
\ fast\isanewline
\isakeywordONE{qed}\isamarkupfalse%
%
\endisatagproof
{\isafoldproof}%
%
\isadelimproof
%
\endisadelimproof
%
\isadelimdocument
%
\endisadelimdocument
%
\isatagdocument
%
\isamarkupsubsection{Equivalence class of languages with identical Parikh image%
}
\isamarkuptrue%
%
\endisatagdocument
{\isafolddocument}%
%
\isadelimdocument
%
\endisadelimdocument
%
\begin{isamarkuptext}%
For a given language \isa{L}, we define the equivalence class of all languages with identical Parikh
image:%
\end{isamarkuptext}\isamarkuptrue%
\isakeywordONE{definition}\isamarkupfalse%
\ parikh{\isacharunderscore}{\kern0pt}img{\isacharunderscore}{\kern0pt}eq{\isacharunderscore}{\kern0pt}class\ {\isacharcolon}{\kern0pt}{\isacharcolon}{\kern0pt}\ {\isachardoublequoteopen}{\isacharprime}{\kern0pt}a\ lang\ {\isasymRightarrow}\ {\isacharprime}{\kern0pt}a\ lang\ set{\isachardoublequoteclose}\ \isakeywordTWO{where}\isanewline
\ \ {\isachardoublequoteopen}parikh{\isacharunderscore}{\kern0pt}img{\isacharunderscore}{\kern0pt}eq{\isacharunderscore}{\kern0pt}class\ L\ {\isasymequiv}\ {\isacharbraceleft}{\kern0pt}L{\isacharprime}{\kern0pt}{\isachardot}{\kern0pt}\ {\isasymPsi}\ L{\isacharprime}{\kern0pt}\ {\isacharequal}{\kern0pt}\ {\isasymPsi}\ L{\isacharbraceright}{\kern0pt}{\isachardoublequoteclose}\isanewline
\isanewline
\isakeywordONE{lemma}\isamarkupfalse%
\ parikh{\isacharunderscore}{\kern0pt}img{\isacharunderscore}{\kern0pt}Union{\isacharunderscore}{\kern0pt}class{\isacharcolon}{\kern0pt}\ {\isachardoublequoteopen}{\isasymPsi}\ A\ {\isacharequal}{\kern0pt}\ {\isasymPsi}\ {\isacharparenleft}{\kern0pt}{\isasymUnion}{\isacharparenleft}{\kern0pt}parikh{\isacharunderscore}{\kern0pt}img{\isacharunderscore}{\kern0pt}eq{\isacharunderscore}{\kern0pt}class\ A{\isacharparenright}{\kern0pt}{\isacharparenright}{\kern0pt}{\isachardoublequoteclose}\isanewline
%
\isadelimproof
%
\endisadelimproof
%
\isatagproof
\isakeywordONE{proof}\isamarkupfalse%
\isanewline
\ \ \isakeywordONE{let}\isamarkupfalse%
\ {\isacharquery}{\kern0pt}A{\isacharprime}{\kern0pt}\ {\isacharequal}{\kern0pt}\ {\isachardoublequoteopen}{\isasymUnion}{\isacharparenleft}{\kern0pt}parikh{\isacharunderscore}{\kern0pt}img{\isacharunderscore}{\kern0pt}eq{\isacharunderscore}{\kern0pt}class\ A{\isacharparenright}{\kern0pt}{\isachardoublequoteclose}\isanewline
\ \ \isakeywordTHREE{show}\isamarkupfalse%
\ {\isachardoublequoteopen}{\isasymPsi}\ A\ {\isasymsubseteq}\ {\isasymPsi}\ {\isacharquery}{\kern0pt}A{\isacharprime}{\kern0pt}{\isachardoublequoteclose}\isanewline
\ \ \ \ \isakeywordONE{unfolding}\isamarkupfalse%
\ parikh{\isacharunderscore}{\kern0pt}img{\isacharunderscore}{\kern0pt}eq{\isacharunderscore}{\kern0pt}class{\isacharunderscore}{\kern0pt}def\ \isakeywordONE{by}\isamarkupfalse%
\ {\isacharparenleft}{\kern0pt}simp\ add{\isacharcolon}{\kern0pt}\ Union{\isacharunderscore}{\kern0pt}upper\ parikh{\isacharunderscore}{\kern0pt}img{\isacharunderscore}{\kern0pt}mono{\isacharparenright}{\kern0pt}\isanewline
\ \ \isakeywordTHREE{show}\isamarkupfalse%
\ {\isachardoublequoteopen}{\isasymPsi}\ {\isacharquery}{\kern0pt}A{\isacharprime}{\kern0pt}\ {\isasymsubseteq}\ {\isasymPsi}\ A{\isachardoublequoteclose}\isanewline
\ \ \isakeywordONE{proof}\isamarkupfalse%
\isanewline
\ \ \ \ \isakeywordTHREE{fix}\isamarkupfalse%
\ v\isanewline
\ \ \ \ \isakeywordTHREE{assume}\isamarkupfalse%
\ {\isachardoublequoteopen}v\ {\isasymin}\ {\isasymPsi}\ {\isacharquery}{\kern0pt}A{\isacharprime}{\kern0pt}{\isachardoublequoteclose}\isanewline
\ \ \ \ \isakeywordONE{then}\isamarkupfalse%
\ \isakeywordTHREE{obtain}\isamarkupfalse%
\ a\ \isakeywordTWO{where}\ a{\isacharunderscore}{\kern0pt}intro{\isacharcolon}{\kern0pt}\ {\isachardoublequoteopen}parikh{\isacharunderscore}{\kern0pt}vec\ a\ {\isacharequal}{\kern0pt}\ v\ {\isasymand}\ a\ {\isasymin}\ {\isacharquery}{\kern0pt}A{\isacharprime}{\kern0pt}{\isachardoublequoteclose}\isanewline
\ \ \ \ \ \ \isakeywordONE{unfolding}\isamarkupfalse%
\ parikh{\isacharunderscore}{\kern0pt}img{\isacharunderscore}{\kern0pt}def\ \isakeywordONE{by}\isamarkupfalse%
\ blast\isanewline
\ \ \ \ \isakeywordONE{then}\isamarkupfalse%
\ \isakeywordTHREE{obtain}\isamarkupfalse%
\ L\ \isakeywordTWO{where}\ L{\isacharunderscore}{\kern0pt}intro{\isacharcolon}{\kern0pt}\ {\isachardoublequoteopen}a\ {\isasymin}\ L\ {\isasymand}\ L\ {\isasymin}\ parikh{\isacharunderscore}{\kern0pt}img{\isacharunderscore}{\kern0pt}eq{\isacharunderscore}{\kern0pt}class\ A{\isachardoublequoteclose}\isanewline
\ \ \ \ \ \ \isakeywordONE{unfolding}\isamarkupfalse%
\ parikh{\isacharunderscore}{\kern0pt}img{\isacharunderscore}{\kern0pt}eq{\isacharunderscore}{\kern0pt}class{\isacharunderscore}{\kern0pt}def\ \isakeywordONE{by}\isamarkupfalse%
\ blast\isanewline
\ \ \ \ \isakeywordONE{then}\isamarkupfalse%
\ \isakeywordONE{have}\isamarkupfalse%
\ {\isachardoublequoteopen}{\isasymPsi}\ L\ {\isacharequal}{\kern0pt}\ {\isasymPsi}\ A{\isachardoublequoteclose}\ \isakeywordONE{unfolding}\isamarkupfalse%
\ parikh{\isacharunderscore}{\kern0pt}img{\isacharunderscore}{\kern0pt}eq{\isacharunderscore}{\kern0pt}class{\isacharunderscore}{\kern0pt}def\ \isakeywordONE{by}\isamarkupfalse%
\ fastforce\isanewline
\ \ \ \ \isakeywordONE{with}\isamarkupfalse%
\ a{\isacharunderscore}{\kern0pt}intro\ L{\isacharunderscore}{\kern0pt}intro\ \isakeywordTHREE{show}\isamarkupfalse%
\ {\isachardoublequoteopen}v\ {\isasymin}\ {\isasymPsi}\ A{\isachardoublequoteclose}\ \isakeywordONE{unfolding}\isamarkupfalse%
\ parikh{\isacharunderscore}{\kern0pt}img{\isacharunderscore}{\kern0pt}def\ \isakeywordONE{by}\isamarkupfalse%
\ blast\isanewline
\ \ \isakeywordONE{qed}\isamarkupfalse%
\isanewline
\isakeywordONE{qed}\isamarkupfalse%
%
\endisatagproof
{\isafoldproof}%
%
\isadelimproof
\isanewline
%
\endisadelimproof
\isanewline
\isakeywordONE{lemma}\isamarkupfalse%
\ subseteq{\isacharunderscore}{\kern0pt}comm{\isacharunderscore}{\kern0pt}subseteq{\isacharcolon}{\kern0pt}\isanewline
\ \ \isakeywordTWO{assumes}\ {\isachardoublequoteopen}{\isasymPsi}\ A\ {\isasymsubseteq}\ {\isasymPsi}\ B{\isachardoublequoteclose}\isanewline
\ \ \isakeywordTWO{shows}\ {\isachardoublequoteopen}A\ {\isasymsubseteq}\ {\isasymUnion}{\isacharparenleft}{\kern0pt}parikh{\isacharunderscore}{\kern0pt}img{\isacharunderscore}{\kern0pt}eq{\isacharunderscore}{\kern0pt}class\ B{\isacharparenright}{\kern0pt}{\isachardoublequoteclose}\ {\isacharparenleft}{\kern0pt}\isakeywordTWO{is}\ {\isachardoublequoteopen}A\ {\isasymsubseteq}\ {\isacharquery}{\kern0pt}B{\isacharprime}{\kern0pt}{\isachardoublequoteclose}{\isacharparenright}{\kern0pt}\isanewline
%
\isadelimproof
%
\endisadelimproof
%
\isatagproof
\isakeywordONE{proof}\isamarkupfalse%
\isanewline
\ \ \isakeywordTHREE{fix}\isamarkupfalse%
\ a\isanewline
\ \ \isakeywordTHREE{assume}\isamarkupfalse%
\ a{\isacharunderscore}{\kern0pt}in{\isacharunderscore}{\kern0pt}A{\isacharcolon}{\kern0pt}\ {\isachardoublequoteopen}a\ {\isasymin}\ A{\isachardoublequoteclose}\isanewline
\ \ \isakeywordONE{from}\isamarkupfalse%
\ assms\ \isakeywordONE{have}\isamarkupfalse%
\ {\isachardoublequoteopen}{\isasymPsi}\ A\ {\isasymsubseteq}\ {\isasymPsi}\ {\isacharquery}{\kern0pt}B{\isacharprime}{\kern0pt}{\isachardoublequoteclose}\isanewline
\ \ \ \ \isakeywordONE{using}\isamarkupfalse%
\ parikh{\isacharunderscore}{\kern0pt}img{\isacharunderscore}{\kern0pt}Union{\isacharunderscore}{\kern0pt}class\ \isakeywordONE{by}\isamarkupfalse%
\ blast\isanewline
\ \ \isakeywordONE{with}\isamarkupfalse%
\ a{\isacharunderscore}{\kern0pt}in{\isacharunderscore}{\kern0pt}A\ \isakeywordONE{have}\isamarkupfalse%
\ vec{\isacharunderscore}{\kern0pt}a{\isacharunderscore}{\kern0pt}in{\isacharunderscore}{\kern0pt}B{\isacharprime}{\kern0pt}{\isacharcolon}{\kern0pt}\ {\isachardoublequoteopen}parikh{\isacharunderscore}{\kern0pt}vec\ a\ {\isasymin}\ {\isasymPsi}\ {\isacharquery}{\kern0pt}B{\isacharprime}{\kern0pt}{\isachardoublequoteclose}\ \isakeywordONE{unfolding}\isamarkupfalse%
\ parikh{\isacharunderscore}{\kern0pt}img{\isacharunderscore}{\kern0pt}def\ \isakeywordONE{by}\isamarkupfalse%
\ fast\isanewline
\ \ \isakeywordONE{then}\isamarkupfalse%
\ \isakeywordONE{have}\isamarkupfalse%
\ {\isachardoublequoteopen}{\isasymexists}b{\isachardot}{\kern0pt}\ parikh{\isacharunderscore}{\kern0pt}vec\ b\ {\isacharequal}{\kern0pt}\ parikh{\isacharunderscore}{\kern0pt}vec\ a\ {\isasymand}\ b\ {\isasymin}\ {\isacharquery}{\kern0pt}B{\isacharprime}{\kern0pt}{\isachardoublequoteclose}\isanewline
\ \ \ \ \isakeywordONE{unfolding}\isamarkupfalse%
\ parikh{\isacharunderscore}{\kern0pt}img{\isacharunderscore}{\kern0pt}def\ \isakeywordONE{by}\isamarkupfalse%
\ fastforce\isanewline
\ \ \isakeywordONE{then}\isamarkupfalse%
\ \isakeywordTHREE{obtain}\isamarkupfalse%
\ b\ \isakeywordTWO{where}\ b{\isacharunderscore}{\kern0pt}intro{\isacharcolon}{\kern0pt}\ {\isachardoublequoteopen}parikh{\isacharunderscore}{\kern0pt}vec\ b\ {\isacharequal}{\kern0pt}\ parikh{\isacharunderscore}{\kern0pt}vec\ a\ {\isasymand}\ b\ {\isasymin}\ {\isacharquery}{\kern0pt}B{\isacharprime}{\kern0pt}{\isachardoublequoteclose}\ \isakeywordONE{by}\isamarkupfalse%
\ blast\isanewline
\ \ \isakeywordONE{with}\isamarkupfalse%
\ vec{\isacharunderscore}{\kern0pt}a{\isacharunderscore}{\kern0pt}in{\isacharunderscore}{\kern0pt}B{\isacharprime}{\kern0pt}\ \isakeywordONE{have}\isamarkupfalse%
\ {\isachardoublequoteopen}{\isasymPsi}\ {\isacharparenleft}{\kern0pt}{\isacharquery}{\kern0pt}B{\isacharprime}{\kern0pt}\ {\isasymunion}\ {\isacharbraceleft}{\kern0pt}a{\isacharbraceright}{\kern0pt}{\isacharparenright}{\kern0pt}\ {\isacharequal}{\kern0pt}\ {\isasymPsi}\ {\isacharquery}{\kern0pt}B{\isacharprime}{\kern0pt}{\isachardoublequoteclose}\isakeywordONE{unfolding}\isamarkupfalse%
\ parikh{\isacharunderscore}{\kern0pt}img{\isacharunderscore}{\kern0pt}def\ \isakeywordONE{by}\isamarkupfalse%
\ blast\isanewline
\ \ \isakeywordONE{with}\isamarkupfalse%
\ parikh{\isacharunderscore}{\kern0pt}img{\isacharunderscore}{\kern0pt}Union{\isacharunderscore}{\kern0pt}class\ \isakeywordONE{have}\isamarkupfalse%
\ {\isachardoublequoteopen}{\isasymPsi}\ {\isacharparenleft}{\kern0pt}{\isacharquery}{\kern0pt}B{\isacharprime}{\kern0pt}\ {\isasymunion}\ {\isacharbraceleft}{\kern0pt}a{\isacharbraceright}{\kern0pt}{\isacharparenright}{\kern0pt}\ {\isacharequal}{\kern0pt}\ {\isasymPsi}\ B{\isachardoublequoteclose}\ \isakeywordONE{by}\isamarkupfalse%
\ blast\isanewline
\ \ \isakeywordONE{then}\isamarkupfalse%
\ \isakeywordTHREE{show}\isamarkupfalse%
\ {\isachardoublequoteopen}a\ {\isasymin}\ {\isacharquery}{\kern0pt}B{\isacharprime}{\kern0pt}{\isachardoublequoteclose}\ \isakeywordONE{unfolding}\isamarkupfalse%
\ parikh{\isacharunderscore}{\kern0pt}img{\isacharunderscore}{\kern0pt}eq{\isacharunderscore}{\kern0pt}class{\isacharunderscore}{\kern0pt}def\ \isakeywordONE{by}\isamarkupfalse%
\ blast\isanewline
\isakeywordONE{qed}\isamarkupfalse%
%
\endisatagproof
{\isafoldproof}%
%
\isadelimproof
\isanewline
%
\endisadelimproof
%
\isadelimtheory
\isanewline
%
\endisadelimtheory
%
\isatagtheory
\isakeywordTWO{end}\isamarkupfalse%
%
\endisatagtheory
{\isafoldtheory}%
%
\isadelimtheory
%
\endisadelimtheory
%
\end{isabellebody}%
\endinput
%:%file=~/studium/semester_7/semantik/homeworks/AIST/Parikh/Parikh_Img.thy%:%
%:%11=1%:%
%:%27=3%:%
%:%28=3%:%
%:%29=4%:%
%:%30=5%:%
%:%31=6%:%
%:%32=7%:%
%:%46=10%:%
%:%58=12%:%
%:%59=13%:%
%:%60=14%:%
%:%62=16%:%
%:%63=16%:%
%:%64=17%:%
%:%65=18%:%
%:%66=19%:%
%:%67=19%:%
%:%68=20%:%
%:%69=21%:%
%:%70=22%:%
%:%71=22%:%
%:%74=23%:%
%:%78=23%:%
%:%79=23%:%
%:%84=23%:%
%:%87=24%:%
%:%88=25%:%
%:%89=25%:%
%:%92=26%:%
%:%96=26%:%
%:%97=26%:%
%:%102=26%:%
%:%105=27%:%
%:%106=28%:%
%:%107=28%:%
%:%110=29%:%
%:%114=29%:%
%:%115=29%:%
%:%116=29%:%
%:%121=29%:%
%:%124=30%:%
%:%125=31%:%
%:%126=31%:%
%:%133=32%:%
%:%134=32%:%
%:%135=33%:%
%:%136=33%:%
%:%137=34%:%
%:%138=35%:%
%:%139=35%:%
%:%140=35%:%
%:%141=36%:%
%:%142=36%:%
%:%143=36%:%
%:%144=37%:%
%:%145=37%:%
%:%146=37%:%
%:%147=38%:%
%:%162=41%:%
%:%172=43%:%
%:%173=43%:%
%:%176=44%:%
%:%180=44%:%
%:%181=44%:%
%:%182=44%:%
%:%187=44%:%
%:%190=45%:%
%:%191=46%:%
%:%192=46%:%
%:%195=47%:%
%:%199=47%:%
%:%200=47%:%
%:%205=47%:%
%:%208=48%:%
%:%209=49%:%
%:%210=49%:%
%:%213=50%:%
%:%217=50%:%
%:%218=50%:%
%:%223=50%:%
%:%226=51%:%
%:%227=52%:%
%:%228=52%:%
%:%229=53%:%
%:%230=54%:%
%:%231=55%:%
%:%234=56%:%
%:%238=56%:%
%:%239=56%:%
%:%240=56%:%
%:%245=56%:%
%:%248=57%:%
%:%249=58%:%
%:%250=58%:%
%:%253=59%:%
%:%257=59%:%
%:%258=59%:%
%:%263=59%:%
%:%266=60%:%
%:%267=61%:%
%:%268=61%:%
%:%271=62%:%
%:%275=62%:%
%:%276=62%:%
%:%281=62%:%
%:%284=63%:%
%:%285=64%:%
%:%286=64%:%
%:%289=65%:%
%:%293=65%:%
%:%294=65%:%
%:%299=65%:%
%:%302=66%:%
%:%303=67%:%
%:%304=68%:%
%:%305=68%:%
%:%306=69%:%
%:%307=70%:%
%:%314=71%:%
%:%315=71%:%
%:%316=72%:%
%:%317=72%:%
%:%318=73%:%
%:%319=73%:%
%:%320=74%:%
%:%321=74%:%
%:%322=74%:%
%:%323=74%:%
%:%324=74%:%
%:%325=75%:%
%:%326=75%:%
%:%327=75%:%
%:%328=75%:%
%:%329=75%:%
%:%330=76%:%
%:%331=76%:%
%:%332=76%:%
%:%333=76%:%
%:%334=76%:%
%:%335=76%:%
%:%336=77%:%
%:%337=77%:%
%:%338=77%:%
%:%339=77%:%
%:%340=77%:%
%:%341=78%:%
%:%342=78%:%
%:%343=78%:%
%:%344=78%:%
%:%345=78%:%
%:%346=78%:%
%:%347=79%:%
%:%353=79%:%
%:%356=80%:%
%:%357=81%:%
%:%358=81%:%
%:%359=82%:%
%:%360=83%:%
%:%363=84%:%
%:%367=84%:%
%:%368=84%:%
%:%369=84%:%
%:%374=84%:%
%:%377=85%:%
%:%378=86%:%
%:%379=87%:%
%:%380=87%:%
%:%381=88%:%
%:%382=89%:%
%:%389=90%:%
%:%390=90%:%
%:%391=91%:%
%:%392=91%:%
%:%393=92%:%
%:%394=92%:%
%:%395=92%:%
%:%396=93%:%
%:%397=94%:%
%:%398=94%:%
%:%399=94%:%
%:%400=95%:%
%:%401=95%:%
%:%402=95%:%
%:%403=95%:%
%:%404=96%:%
%:%405=96%:%
%:%406=97%:%
%:%407=97%:%
%:%408=98%:%
%:%409=98%:%
%:%410=98%:%
%:%411=99%:%
%:%412=99%:%
%:%413=99%:%
%:%414=100%:%
%:%415=100%:%
%:%416=100%:%
%:%417=100%:%
%:%418=101%:%
%:%419=101%:%
%:%424=101%:%
%:%427=102%:%
%:%428=103%:%
%:%429=103%:%
%:%430=104%:%
%:%431=105%:%
%:%434=106%:%
%:%438=106%:%
%:%439=106%:%
%:%440=106%:%
%:%445=106%:%
%:%448=107%:%
%:%449=108%:%
%:%450=108%:%
%:%451=109%:%
%:%452=110%:%
%:%459=111%:%
%:%460=111%:%
%:%461=111%:%
%:%462=112%:%
%:%463=112%:%
%:%464=113%:%
%:%465=113%:%
%:%466=113%:%
%:%467=114%:%
%:%468=114%:%
%:%469=114%:%
%:%470=115%:%
%:%471=115%:%
%:%472=115%:%
%:%473=115%:%
%:%474=116%:%
%:%475=116%:%
%:%480=116%:%
%:%483=117%:%
%:%484=118%:%
%:%485=118%:%
%:%486=119%:%
%:%487=120%:%
%:%490=121%:%
%:%494=121%:%
%:%495=121%:%
%:%496=121%:%
%:%510=125%:%
%:%522=127%:%
%:%524=129%:%
%:%525=129%:%
%:%526=130%:%
%:%527=131%:%
%:%534=132%:%
%:%535=132%:%
%:%536=132%:%
%:%537=133%:%
%:%538=133%:%
%:%539=134%:%
%:%540=134%:%
%:%541=134%:%
%:%542=134%:%
%:%543=135%:%
%:%544=135%:%
%:%545=136%:%
%:%546=136%:%
%:%547=137%:%
%:%548=137%:%
%:%549=137%:%
%:%550=138%:%
%:%551=138%:%
%:%552=138%:%
%:%553=139%:%
%:%554=139%:%
%:%555=139%:%
%:%556=139%:%
%:%557=140%:%
%:%558=140%:%
%:%559=141%:%
%:%560=141%:%
%:%561=141%:%
%:%562=141%:%
%:%563=141%:%
%:%564=142%:%
%:%565=142%:%
%:%566=142%:%
%:%567=142%:%
%:%568=143%:%
%:%569=143%:%
%:%570=143%:%
%:%571=144%:%
%:%572=144%:%
%:%573=144%:%
%:%574=145%:%
%:%575=145%:%
%:%576=145%:%
%:%577=145%:%
%:%578=146%:%
%:%579=146%:%
%:%580=146%:%
%:%581=147%:%
%:%582=147%:%
%:%583=148%:%
%:%584=148%:%
%:%585=148%:%
%:%586=149%:%
%:%587=149%:%
%:%588=150%:%
%:%589=150%:%
%:%590=151%:%
%:%591=151%:%
%:%592=151%:%
%:%593=152%:%
%:%594=152%:%
%:%595=153%:%
%:%596=153%:%
%:%597=153%:%
%:%598=154%:%
%:%599=154%:%
%:%600=154%:%
%:%601=155%:%
%:%602=155%:%
%:%603=155%:%
%:%604=155%:%
%:%605=156%:%
%:%606=156%:%
%:%607=157%:%
%:%608=157%:%
%:%609=158%:%
%:%610=158%:%
%:%611=158%:%
%:%612=158%:%
%:%613=159%:%
%:%614=159%:%
%:%615=159%:%
%:%616=160%:%
%:%617=160%:%
%:%618=160%:%
%:%619=161%:%
%:%620=161%:%
%:%621=161%:%
%:%622=162%:%
%:%623=162%:%
%:%624=163%:%
%:%625=163%:%
%:%626=164%:%
%:%627=164%:%
%:%628=164%:%
%:%629=164%:%
%:%630=165%:%
%:%631=165%:%
%:%632=166%:%
%:%633=166%:%
%:%634=166%:%
%:%635=166%:%
%:%636=166%:%
%:%637=167%:%
%:%643=167%:%
%:%646=168%:%
%:%647=169%:%
%:%648=169%:%
%:%649=170%:%
%:%650=171%:%
%:%657=172%:%
%:%658=172%:%
%:%659=173%:%
%:%660=173%:%
%:%661=173%:%
%:%662=174%:%
%:%663=174%:%
%:%664=174%:%
%:%665=174%:%
%:%666=175%:%
%:%667=175%:%
%:%668=175%:%
%:%669=175%:%
%:%670=176%:%
%:%671=176%:%
%:%672=176%:%
%:%673=177%:%
%:%674=177%:%
%:%675=177%:%
%:%676=178%:%
%:%677=178%:%
%:%678=178%:%
%:%679=178%:%
%:%680=178%:%
%:%681=179%:%
%:%682=179%:%
%:%683=179%:%
%:%684=179%:%
%:%685=180%:%
%:%686=180%:%
%:%687=180%:%
%:%688=181%:%
%:%689=181%:%
%:%690=181%:%
%:%691=182%:%
%:%692=182%:%
%:%693=182%:%
%:%694=182%:%
%:%695=183%:%
%:%696=183%:%
%:%697=183%:%
%:%698=183%:%
%:%699=184%:%
%:%700=184%:%
%:%701=184%:%
%:%702=184%:%
%:%703=185%:%
%:%704=185%:%
%:%705=185%:%
%:%706=185%:%
%:%707=185%:%
%:%708=186%:%
%:%714=186%:%
%:%717=187%:%
%:%718=188%:%
%:%719=188%:%
%:%720=189%:%
%:%721=190%:%
%:%728=191%:%
%:%729=191%:%
%:%730=192%:%
%:%731=192%:%
%:%732=192%:%
%:%733=193%:%
%:%734=193%:%
%:%735=193%:%
%:%736=194%:%
%:%737=194%:%
%:%738=194%:%
%:%739=194%:%
%:%740=195%:%
%:%741=195%:%
%:%742=195%:%
%:%743=195%:%
%:%744=195%:%
%:%745=196%:%
%:%746=196%:%
%:%747=196%:%
%:%748=196%:%
%:%749=196%:%
%:%750=197%:%
%:%751=197%:%
%:%752=197%:%
%:%753=197%:%
%:%754=197%:%
%:%755=198%:%
%:%756=198%:%
%:%757=198%:%
%:%758=198%:%
%:%759=198%:%
%:%760=199%:%
%:%761=199%:%
%:%762=199%:%
%:%763=199%:%
%:%764=199%:%
%:%765=200%:%
%:%771=200%:%
%:%774=201%:%
%:%775=202%:%
%:%776=202%:%
%:%783=203%:%
%:%784=203%:%
%:%785=204%:%
%:%786=204%:%
%:%787=204%:%
%:%788=204%:%
%:%789=205%:%
%:%790=205%:%
%:%791=205%:%
%:%792=205%:%
%:%793=206%:%
%:%808=210%:%
%:%820=212%:%
%:%821=213%:%
%:%823=216%:%
%:%824=216%:%
%:%831=217%:%
%:%832=217%:%
%:%833=218%:%
%:%834=218%:%
%:%835=219%:%
%:%836=219%:%
%:%837=219%:%
%:%838=220%:%
%:%839=220%:%
%:%840=220%:%
%:%841=221%:%
%:%842=221%:%
%:%843=221%:%
%:%844=222%:%
%:%845=222%:%
%:%846=223%:%
%:%847=223%:%
%:%848=223%:%
%:%849=223%:%
%:%850=224%:%
%:%851=224%:%
%:%856=224%:%
%:%859=225%:%
%:%860=226%:%
%:%861=226%:%
%:%868=227%:%
%:%869=227%:%
%:%870=228%:%
%:%871=228%:%
%:%872=229%:%
%:%873=229%:%
%:%874=230%:%
%:%875=230%:%
%:%876=230%:%
%:%877=230%:%
%:%878=230%:%
%:%879=231%:%
%:%880=231%:%
%:%881=231%:%
%:%882=231%:%
%:%883=232%:%
%:%884=232%:%
%:%885=232%:%
%:%886=232%:%
%:%887=233%:%
%:%888=233%:%
%:%889=233%:%
%:%890=234%:%
%:%891=234%:%
%:%892=234%:%
%:%893=235%:%
%:%894=235%:%
%:%895=236%:%
%:%896=236%:%
%:%897=236%:%
%:%898=237%:%
%:%899=237%:%
%:%900=237%:%
%:%901=238%:%
%:%902=238%:%
%:%903=239%:%
%:%909=239%:%
%:%912=240%:%
%:%913=241%:%
%:%914=241%:%
%:%921=242%:%
%:%922=242%:%
%:%923=243%:%
%:%924=243%:%
%:%925=244%:%
%:%926=244%:%
%:%927=245%:%
%:%928=245%:%
%:%929=245%:%
%:%930=246%:%
%:%931=246%:%
%:%932=247%:%
%:%933=247%:%
%:%934=247%:%
%:%935=248%:%
%:%936=248%:%
%:%937=248%:%
%:%938=249%:%
%:%939=249%:%
%:%940=250%:%
%:%941=250%:%
%:%942=250%:%
%:%943=251%:%
%:%944=251%:%
%:%945=251%:%
%:%946=252%:%
%:%947=252%:%
%:%948=253%:%
%:%954=253%:%
%:%957=254%:%
%:%958=255%:%
%:%959=255%:%
%:%960=256%:%
%:%967=257%:%
%:%968=257%:%
%:%969=258%:%
%:%970=258%:%
%:%971=259%:%
%:%972=259%:%
%:%973=260%:%
%:%974=260%:%
%:%975=260%:%
%:%976=261%:%
%:%977=261%:%
%:%978=261%:%
%:%979=262%:%
%:%980=262%:%
%:%981=262%:%
%:%982=262%:%
%:%983=263%:%
%:%984=263%:%
%:%985=264%:%
%:%986=264%:%
%:%987=265%:%
%:%988=265%:%
%:%989=266%:%
%:%990=266%:%
%:%991=266%:%
%:%992=266%:%
%:%993=266%:%
%:%994=267%:%
%:%995=267%:%
%:%996=267%:%
%:%997=267%:%
%:%998=267%:%
%:%999=268%:%
%:%1000=268%:%
%:%1001=269%:%
%:%1002=269%:%
%:%1003=270%:%
%:%1004=270%:%
%:%1005=270%:%
%:%1006=270%:%
%:%1007=271%:%
%:%1008=271%:%
%:%1009=271%:%
%:%1010=271%:%
%:%1011=272%:%
%:%1012=272%:%
%:%1013=273%:%
%:%1019=273%:%
%:%1022=274%:%
%:%1023=275%:%
%:%1024=275%:%
%:%1031=276%:%
%:%1032=276%:%
%:%1033=277%:%
%:%1034=277%:%
%:%1035=277%:%
%:%1036=277%:%
%:%1037=278%:%
%:%1038=278%:%
%:%1039=279%:%
%:%1040=279%:%
%:%1041=279%:%
%:%1042=280%:%
%:%1043=280%:%
%:%1044=280%:%
%:%1045=281%:%
%:%1046=281%:%
%:%1047=281%:%
%:%1048=282%:%
%:%1049=282%:%
%:%1050=283%:%
%:%1051=283%:%
%:%1052=283%:%
%:%1053=283%:%
%:%1054=283%:%
%:%1055=284%:%
%:%1061=284%:%
%:%1064=285%:%
%:%1065=286%:%
%:%1066=286%:%
%:%1073=287%:%
%:%1074=287%:%
%:%1075=288%:%
%:%1076=288%:%
%:%1077=289%:%
%:%1078=289%:%
%:%1079=289%:%
%:%1080=290%:%
%:%1081=290%:%
%:%1082=291%:%
%:%1083=291%:%
%:%1084=292%:%
%:%1085=292%:%
%:%1086=293%:%
%:%1101=297%:%
%:%1111=299%:%
%:%1112=299%:%
%:%1113=300%:%
%:%1114=301%:%
%:%1121=302%:%
%:%1122=302%:%
%:%1123=303%:%
%:%1124=303%:%
%:%1125=304%:%
%:%1126=304%:%
%:%1127=305%:%
%:%1128=305%:%
%:%1129=306%:%
%:%1130=306%:%
%:%1131=307%:%
%:%1132=307%:%
%:%1133=307%:%
%:%1134=307%:%
%:%1135=308%:%
%:%1136=308%:%
%:%1137=309%:%
%:%1138=309%:%
%:%1139=310%:%
%:%1140=310%:%
%:%1141=310%:%
%:%1142=311%:%
%:%1143=311%:%
%:%1144=311%:%
%:%1145=311%:%
%:%1146=312%:%
%:%1147=312%:%
%:%1148=313%:%
%:%1149=313%:%
%:%1150=314%:%
%:%1151=314%:%
%:%1152=315%:%
%:%1153=315%:%
%:%1154=315%:%
%:%1155=315%:%
%:%1156=316%:%
%:%1157=316%:%
%:%1158=316%:%
%:%1159=316%:%
%:%1160=317%:%
%:%1161=317%:%
%:%1162=318%:%
%:%1163=318%:%
%:%1164=319%:%
%:%1165=319%:%
%:%1166=319%:%
%:%1167=320%:%
%:%1168=321%:%
%:%1169=321%:%
%:%1170=322%:%
%:%1171=322%:%
%:%1172=322%:%
%:%1173=322%:%
%:%1174=323%:%
%:%1175=323%:%
%:%1176=324%:%
%:%1177=324%:%
%:%1178=325%:%
%:%1179=325%:%
%:%1180=325%:%
%:%1181=326%:%
%:%1182=327%:%
%:%1183=327%:%
%:%1184=328%:%
%:%1185=328%:%
%:%1186=328%:%
%:%1187=329%:%
%:%1188=329%:%
%:%1189=330%:%
%:%1190=330%:%
%:%1191=330%:%
%:%1192=331%:%
%:%1193=331%:%
%:%1194=332%:%
%:%1195=332%:%
%:%1196=332%:%
%:%1197=332%:%
%:%1198=333%:%
%:%1199=333%:%
%:%1200=334%:%
%:%1201=334%:%
%:%1202=335%:%
%:%1203=335%:%
%:%1204=335%:%
%:%1205=336%:%
%:%1206=336%:%
%:%1207=336%:%
%:%1208=337%:%
%:%1209=337%:%
%:%1210=337%:%
%:%1211=337%:%
%:%1212=338%:%
%:%1213=338%:%
%:%1214=339%:%
%:%1215=339%:%
%:%1216=339%:%
%:%1217=339%:%
%:%1218=340%:%
%:%1228=342%:%
%:%1229=343%:%
%:%1230=344%:%
%:%1232=345%:%
%:%1233=345%:%
%:%1234=346%:%
%:%1235=347%:%
%:%1242=348%:%
%:%1243=348%:%
%:%1244=349%:%
%:%1245=349%:%
%:%1246=350%:%
%:%1247=350%:%
%:%1248=350%:%
%:%1249=350%:%
%:%1250=351%:%
%:%1251=351%:%
%:%1252=351%:%
%:%1253=352%:%
%:%1254=352%:%
%:%1255=352%:%
%:%1256=352%:%
%:%1257=353%:%
%:%1258=353%:%
%:%1259=353%:%
%:%1260=353%:%
%:%1261=353%:%
%:%1262=354%:%
%:%1263=354%:%
%:%1264=354%:%
%:%1265=354%:%
%:%1266=355%:%
%:%1281=358%:%
%:%1291=360%:%
%:%1292=360%:%
%:%1293=361%:%
%:%1294=362%:%
%:%1301=363%:%
%:%1302=363%:%
%:%1303=364%:%
%:%1304=364%:%
%:%1305=365%:%
%:%1306=365%:%
%:%1307=365%:%
%:%1308=365%:%
%:%1309=366%:%
%:%1310=366%:%
%:%1311=367%:%
%:%1312=367%:%
%:%1313=368%:%
%:%1314=368%:%
%:%1315=368%:%
%:%1316=369%:%
%:%1317=369%:%
%:%1318=370%:%
%:%1319=370%:%
%:%1320=370%:%
%:%1321=370%:%
%:%1322=371%:%
%:%1323=371%:%
%:%1324=372%:%
%:%1325=372%:%
%:%1326=372%:%
%:%1327=372%:%
%:%1328=372%:%
%:%1329=373%:%
%:%1330=373%:%
%:%1331=373%:%
%:%1332=373%:%
%:%1333=374%:%
%:%1339=374%:%
%:%1342=375%:%
%:%1343=376%:%
%:%1344=376%:%
%:%1345=377%:%
%:%1346=378%:%
%:%1353=379%:%
%:%1354=379%:%
%:%1355=380%:%
%:%1356=380%:%
%:%1357=381%:%
%:%1358=381%:%
%:%1359=382%:%
%:%1360=382%:%
%:%1361=382%:%
%:%1362=383%:%
%:%1363=383%:%
%:%1364=383%:%
%:%1365=384%:%
%:%1366=384%:%
%:%1367=384%:%
%:%1368=384%:%
%:%1369=385%:%
%:%1370=385%:%
%:%1371=385%:%
%:%1372=385%:%
%:%1373=385%:%
%:%1374=386%:%
%:%1389=389%:%
%:%1401=391%:%
%:%1402=392%:%
%:%1404=394%:%
%:%1405=394%:%
%:%1406=395%:%
%:%1407=396%:%
%:%1408=397%:%
%:%1409=397%:%
%:%1416=398%:%
%:%1417=398%:%
%:%1418=399%:%
%:%1419=399%:%
%:%1420=400%:%
%:%1421=400%:%
%:%1422=401%:%
%:%1423=401%:%
%:%1424=401%:%
%:%1425=402%:%
%:%1426=402%:%
%:%1427=403%:%
%:%1428=403%:%
%:%1429=404%:%
%:%1430=404%:%
%:%1431=405%:%
%:%1432=405%:%
%:%1433=406%:%
%:%1434=406%:%
%:%1435=406%:%
%:%1436=407%:%
%:%1437=407%:%
%:%1438=407%:%
%:%1439=408%:%
%:%1440=408%:%
%:%1441=408%:%
%:%1442=409%:%
%:%1443=409%:%
%:%1444=409%:%
%:%1445=410%:%
%:%1446=410%:%
%:%1447=410%:%
%:%1448=410%:%
%:%1449=410%:%
%:%1450=411%:%
%:%1451=411%:%
%:%1452=411%:%
%:%1453=411%:%
%:%1454=411%:%
%:%1455=412%:%
%:%1456=412%:%
%:%1457=413%:%
%:%1463=413%:%
%:%1466=414%:%
%:%1467=415%:%
%:%1468=415%:%
%:%1469=416%:%
%:%1470=417%:%
%:%1477=418%:%
%:%1478=418%:%
%:%1479=419%:%
%:%1480=419%:%
%:%1481=420%:%
%:%1482=420%:%
%:%1483=421%:%
%:%1484=421%:%
%:%1485=421%:%
%:%1486=422%:%
%:%1487=422%:%
%:%1488=422%:%
%:%1489=423%:%
%:%1490=423%:%
%:%1491=423%:%
%:%1492=423%:%
%:%1493=423%:%
%:%1494=424%:%
%:%1495=424%:%
%:%1496=424%:%
%:%1497=425%:%
%:%1498=425%:%
%:%1499=425%:%
%:%1500=426%:%
%:%1501=426%:%
%:%1502=426%:%
%:%1503=426%:%
%:%1504=427%:%
%:%1505=427%:%
%:%1506=427%:%
%:%1507=427%:%
%:%1508=427%:%
%:%1509=428%:%
%:%1510=428%:%
%:%1511=428%:%
%:%1512=428%:%
%:%1513=429%:%
%:%1514=429%:%
%:%1515=429%:%
%:%1516=429%:%
%:%1517=429%:%
%:%1518=430%:%
%:%1524=430%:%
%:%1529=431%:%
%:%1534=432%:%

%
\begin{isabellebody}%
\setisabellecontext{Eq{\isacharunderscore}{\kern0pt}Sys}%
%
\isadelimdocument
%
\endisadelimdocument
%
\isatagdocument
%
\isamarkupsection{Context free grammars and systems of equations%
}
\isamarkuptrue%
%
\endisatagdocument
{\isafolddocument}%
%
\isadelimdocument
%
\endisadelimdocument
%
\isadelimtheory
%
\endisadelimtheory
%
\isatagtheory
\isakeywordONE{theory}\isamarkupfalse%
\ Eq{\isacharunderscore}{\kern0pt}Sys\isanewline
\ \ \isakeywordTWO{imports}\isanewline
\ \ \ \ {\isachardoublequoteopen}Parikh{\isacharunderscore}{\kern0pt}Img{\isachardoublequoteclose}\isanewline
\ \ \ \ {\isachardoublequoteopen}Context{\isacharunderscore}{\kern0pt}Free{\isacharunderscore}{\kern0pt}Grammar{\isachardot}{\kern0pt}Context{\isacharunderscore}{\kern0pt}Free{\isacharunderscore}{\kern0pt}Language{\isachardoublequoteclose}\isanewline
\isakeywordTWO{begin}%
\endisatagtheory
{\isafoldtheory}%
%
\isadelimtheory
%
\endisadelimtheory
%
\begin{isamarkuptext}%
In this section, we will first introduce two types of systems of
equations. Then we will show that to each CFG correspond two systems of equations - one for both
of the types - and that the language defined by the CFG is a minimal solution of both systems.%
\end{isamarkuptext}\isamarkuptrue%
%
\isadelimdocument
%
\endisadelimdocument
%
\isatagdocument
%
\isamarkupsubsection{Introduction of systems of equations%
}
\isamarkuptrue%
%
\endisatagdocument
{\isafolddocument}%
%
\isadelimdocument
%
\endisadelimdocument
%
\begin{isamarkuptext}%
For the first type of systems, each equation is of the form
$$X_i \supseteq r_i$$
For the second type of systems, each equation is of the form
$$\Psi \; X_i \supseteq \Psi \; r_i$$
i.e. the Parikh image is applied on both sides of each equation.
In both cases, we represent the whole system by a list of regular language expression where each
of the variables \isa{X\isactrlsub {\isadigit{0}}{\isacharcomma}{\kern0pt}\ X\isactrlsub {\isadigit{1}}{\isacharcomma}{\kern0pt}\ {\isasymdots}} is identified by its integer, i.e. \isa{\isaconst{Var}\ \isafree{i}} denotes the variable
\isa{X\isactrlsub i}. The \isa{i}-th item of the list then represents the right-hand side \isa{r\isactrlsub i} of the \isa{i}-th equation:%
\end{isamarkuptext}\isamarkuptrue%
\isakeywordONE{type{\isacharunderscore}{\kern0pt}synonym}\isamarkupfalse%
\ {\isacharprime}{\kern0pt}a\ eq{\isacharunderscore}{\kern0pt}sys\ {\isacharequal}{\kern0pt}\ {\isachardoublequoteopen}{\isacharprime}{\kern0pt}a\ rlexp\ list{\isachardoublequoteclose}%
\begin{isamarkuptext}%
Now we can define what it means for a valuation \isa{v} to solve a system of equations of the
first type, i.e. a system without Parikh images. Afterwards we characterize minimal solutions of
such a system.%
\end{isamarkuptext}\isamarkuptrue%
\isakeywordONE{definition}\isamarkupfalse%
\ solves{\isacharunderscore}{\kern0pt}ineq{\isacharunderscore}{\kern0pt}sys\ {\isacharcolon}{\kern0pt}{\isacharcolon}{\kern0pt}\ {\isachardoublequoteopen}{\isacharprime}{\kern0pt}a\ eq{\isacharunderscore}{\kern0pt}sys\ {\isasymRightarrow}\ {\isacharprime}{\kern0pt}a\ valuation\ {\isasymRightarrow}\ bool{\isachardoublequoteclose}\ \isakeywordTWO{where}\isanewline
\ \ {\isachardoublequoteopen}solves{\isacharunderscore}{\kern0pt}ineq{\isacharunderscore}{\kern0pt}sys\ sys\ v\ {\isasymequiv}\ {\isasymforall}i\ {\isacharless}{\kern0pt}\ length\ sys{\isachardot}{\kern0pt}\ eval\ {\isacharparenleft}{\kern0pt}sys\ {\isacharbang}{\kern0pt}\ i{\isacharparenright}{\kern0pt}\ v\ {\isasymsubseteq}\ v\ i{\isachardoublequoteclose}\isanewline
\isanewline
\isakeywordONE{definition}\isamarkupfalse%
\ min{\isacharunderscore}{\kern0pt}sol{\isacharunderscore}{\kern0pt}ineq{\isacharunderscore}{\kern0pt}sys\ {\isacharcolon}{\kern0pt}{\isacharcolon}{\kern0pt}\ {\isachardoublequoteopen}{\isacharprime}{\kern0pt}a\ eq{\isacharunderscore}{\kern0pt}sys\ {\isasymRightarrow}\ {\isacharprime}{\kern0pt}a\ valuation\ {\isasymRightarrow}\ bool{\isachardoublequoteclose}\ \isakeywordTWO{where}\isanewline
\ \ {\isachardoublequoteopen}min{\isacharunderscore}{\kern0pt}sol{\isacharunderscore}{\kern0pt}ineq{\isacharunderscore}{\kern0pt}sys\ sys\ sol\ {\isasymequiv}\isanewline
\ \ \ \ solves{\isacharunderscore}{\kern0pt}ineq{\isacharunderscore}{\kern0pt}sys\ sys\ sol\ {\isasymand}\ {\isacharparenleft}{\kern0pt}{\isasymforall}sol{\isacharprime}{\kern0pt}{\isachardot}{\kern0pt}\ solves{\isacharunderscore}{\kern0pt}ineq{\isacharunderscore}{\kern0pt}sys\ sys\ sol{\isacharprime}{\kern0pt}\ {\isasymlongrightarrow}\ {\isacharparenleft}{\kern0pt}{\isasymforall}x{\isachardot}{\kern0pt}\ sol\ x\ {\isasymsubseteq}\ sol{\isacharprime}{\kern0pt}\ x{\isacharparenright}{\kern0pt}{\isacharparenright}{\kern0pt}{\isachardoublequoteclose}%
\begin{isamarkuptext}%
The previous definitions can easily be extended to the second type of systems of equations
where the Parikh image is applied on both sides of each equation:%
\end{isamarkuptext}\isamarkuptrue%
\isakeywordONE{definition}\isamarkupfalse%
\ solves{\isacharunderscore}{\kern0pt}ineq{\isacharunderscore}{\kern0pt}comm\ {\isacharcolon}{\kern0pt}{\isacharcolon}{\kern0pt}\ {\isachardoublequoteopen}nat\ {\isasymRightarrow}\ {\isacharprime}{\kern0pt}a\ rlexp\ {\isasymRightarrow}\ {\isacharprime}{\kern0pt}a\ valuation\ {\isasymRightarrow}\ bool{\isachardoublequoteclose}\ \isakeywordTWO{where}\isanewline
\ \ {\isachardoublequoteopen}solves{\isacharunderscore}{\kern0pt}ineq{\isacharunderscore}{\kern0pt}comm\ x\ eq\ v\ {\isasymequiv}\ {\isasymPsi}\ {\isacharparenleft}{\kern0pt}eval\ eq\ v{\isacharparenright}{\kern0pt}\ {\isasymsubseteq}\ {\isasymPsi}\ {\isacharparenleft}{\kern0pt}v\ x{\isacharparenright}{\kern0pt}{\isachardoublequoteclose}\isanewline
\isanewline
\isakeywordONE{definition}\isamarkupfalse%
\ solves{\isacharunderscore}{\kern0pt}ineq{\isacharunderscore}{\kern0pt}sys{\isacharunderscore}{\kern0pt}comm\ {\isacharcolon}{\kern0pt}{\isacharcolon}{\kern0pt}\ {\isachardoublequoteopen}{\isacharprime}{\kern0pt}a\ eq{\isacharunderscore}{\kern0pt}sys\ {\isasymRightarrow}\ {\isacharprime}{\kern0pt}a\ valuation\ {\isasymRightarrow}\ bool{\isachardoublequoteclose}\ \isakeywordTWO{where}\isanewline
\ \ {\isachardoublequoteopen}solves{\isacharunderscore}{\kern0pt}ineq{\isacharunderscore}{\kern0pt}sys{\isacharunderscore}{\kern0pt}comm\ sys\ v\ {\isasymequiv}\ {\isasymforall}i\ {\isacharless}{\kern0pt}\ length\ sys{\isachardot}{\kern0pt}\ solves{\isacharunderscore}{\kern0pt}ineq{\isacharunderscore}{\kern0pt}comm\ i\ {\isacharparenleft}{\kern0pt}sys\ {\isacharbang}{\kern0pt}\ i{\isacharparenright}{\kern0pt}\ v{\isachardoublequoteclose}\isanewline
\isanewline
\isakeywordONE{definition}\isamarkupfalse%
\ min{\isacharunderscore}{\kern0pt}sol{\isacharunderscore}{\kern0pt}ineq{\isacharunderscore}{\kern0pt}sys{\isacharunderscore}{\kern0pt}comm\ {\isacharcolon}{\kern0pt}{\isacharcolon}{\kern0pt}\ {\isachardoublequoteopen}{\isacharprime}{\kern0pt}a\ eq{\isacharunderscore}{\kern0pt}sys\ {\isasymRightarrow}\ {\isacharprime}{\kern0pt}a\ valuation\ {\isasymRightarrow}\ bool{\isachardoublequoteclose}\ \isakeywordTWO{where}\isanewline
\ \ {\isachardoublequoteopen}min{\isacharunderscore}{\kern0pt}sol{\isacharunderscore}{\kern0pt}ineq{\isacharunderscore}{\kern0pt}sys{\isacharunderscore}{\kern0pt}comm\ sys\ sol\ {\isasymequiv}\isanewline
\ \ \ \ solves{\isacharunderscore}{\kern0pt}ineq{\isacharunderscore}{\kern0pt}sys{\isacharunderscore}{\kern0pt}comm\ sys\ sol\ {\isasymand}\isanewline
\ \ \ \ {\isacharparenleft}{\kern0pt}{\isasymforall}sol{\isacharprime}{\kern0pt}{\isachardot}{\kern0pt}\ solves{\isacharunderscore}{\kern0pt}ineq{\isacharunderscore}{\kern0pt}sys{\isacharunderscore}{\kern0pt}comm\ sys\ sol{\isacharprime}{\kern0pt}\ {\isasymlongrightarrow}\ {\isacharparenleft}{\kern0pt}{\isasymforall}x{\isachardot}{\kern0pt}\ {\isasymPsi}\ {\isacharparenleft}{\kern0pt}sol\ x{\isacharparenright}{\kern0pt}\ {\isasymsubseteq}\ {\isasymPsi}\ {\isacharparenleft}{\kern0pt}sol{\isacharprime}{\kern0pt}\ x{\isacharparenright}{\kern0pt}{\isacharparenright}{\kern0pt}{\isacharparenright}{\kern0pt}{\isachardoublequoteclose}%
\begin{isamarkuptext}%
Substitution into each equation of a system:%
\end{isamarkuptext}\isamarkuptrue%
\isakeywordONE{definition}\isamarkupfalse%
\ subst{\isacharunderscore}{\kern0pt}sys\ {\isacharcolon}{\kern0pt}{\isacharcolon}{\kern0pt}\ {\isachardoublequoteopen}{\isacharparenleft}{\kern0pt}nat\ {\isasymRightarrow}\ {\isacharprime}{\kern0pt}a\ rlexp{\isacharparenright}{\kern0pt}\ {\isasymRightarrow}\ {\isacharprime}{\kern0pt}a\ eq{\isacharunderscore}{\kern0pt}sys\ {\isasymRightarrow}\ {\isacharprime}{\kern0pt}a\ eq{\isacharunderscore}{\kern0pt}sys{\isachardoublequoteclose}\ \isakeywordTWO{where}\isanewline
\ \ {\isachardoublequoteopen}subst{\isacharunderscore}{\kern0pt}sys\ {\isasymequiv}\ map\ {\isasymcirc}\ subst{\isachardoublequoteclose}\isanewline
\isanewline
\isakeywordONE{lemma}\isamarkupfalse%
\ subst{\isacharunderscore}{\kern0pt}sys{\isacharunderscore}{\kern0pt}subst{\isacharcolon}{\kern0pt}\isanewline
\ \ \isakeywordTWO{assumes}\ {\isachardoublequoteopen}i\ {\isacharless}{\kern0pt}\ length\ sys{\isachardoublequoteclose}\isanewline
\ \ \isakeywordTWO{shows}\ {\isachardoublequoteopen}{\isacharparenleft}{\kern0pt}subst{\isacharunderscore}{\kern0pt}sys\ s\ sys{\isacharparenright}{\kern0pt}\ {\isacharbang}{\kern0pt}\ i\ {\isacharequal}{\kern0pt}\ subst\ s\ {\isacharparenleft}{\kern0pt}sys\ {\isacharbang}{\kern0pt}\ i{\isacharparenright}{\kern0pt}{\isachardoublequoteclose}\isanewline
%
\isadelimproof
\ \ %
\endisadelimproof
%
\isatagproof
\isakeywordONE{unfolding}\isamarkupfalse%
\ subst{\isacharunderscore}{\kern0pt}sys{\isacharunderscore}{\kern0pt}def\ \isakeywordONE{by}\isamarkupfalse%
\ {\isacharparenleft}{\kern0pt}simp\ add{\isacharcolon}{\kern0pt}\ assms{\isacharparenright}{\kern0pt}%
\endisatagproof
{\isafoldproof}%
%
\isadelimproof
%
\endisadelimproof
%
\isadelimdocument
%
\endisadelimdocument
%
\isatagdocument
%
\isamarkupsubsection{Partial solutions of systems of equations%
}
\isamarkuptrue%
%
\endisatagdocument
{\isafolddocument}%
%
\isadelimdocument
%
\endisadelimdocument
%
\begin{isamarkuptext}%
We introduce partial solutions, i.e. solutions which might depend on one or multiple
variables. They are therefore not represented as languages, but as regular language expressions.
\isa{sol} is a partial solution of the \isa{x}-th equation if and only if it solves the equation
independently on the values of the other variables:%
\end{isamarkuptext}\isamarkuptrue%
\isakeywordONE{definition}\isamarkupfalse%
\ partial{\isacharunderscore}{\kern0pt}sol{\isacharunderscore}{\kern0pt}ineq\ {\isacharcolon}{\kern0pt}{\isacharcolon}{\kern0pt}\ {\isachardoublequoteopen}nat\ {\isasymRightarrow}\ {\isacharprime}{\kern0pt}a\ rlexp\ {\isasymRightarrow}\ {\isacharprime}{\kern0pt}a\ rlexp\ {\isasymRightarrow}\ bool{\isachardoublequoteclose}\ \isakeywordTWO{where}\isanewline
\ \ {\isachardoublequoteopen}partial{\isacharunderscore}{\kern0pt}sol{\isacharunderscore}{\kern0pt}ineq\ x\ eq\ sol\ {\isasymequiv}\ {\isasymforall}v{\isachardot}{\kern0pt}\ v\ x\ {\isacharequal}{\kern0pt}\ eval\ sol\ v\ {\isasymlongrightarrow}\ solves{\isacharunderscore}{\kern0pt}ineq{\isacharunderscore}{\kern0pt}comm\ x\ eq\ v{\isachardoublequoteclose}%
\begin{isamarkuptext}%
We generalize the previous definition to partial solutions of whole systems of equations:
\isa{sols} maps each variable \isa{i} to a regular language expression representing the partial solution
of the \isa{i}-th equation. A partial solution of the whole system is then defined as follows:%
\end{isamarkuptext}\isamarkuptrue%
\isakeywordONE{definition}\isamarkupfalse%
\ solution{\isacharunderscore}{\kern0pt}ineq{\isacharunderscore}{\kern0pt}sys\ {\isacharcolon}{\kern0pt}{\isacharcolon}{\kern0pt}\ {\isachardoublequoteopen}{\isacharprime}{\kern0pt}a\ eq{\isacharunderscore}{\kern0pt}sys\ {\isasymRightarrow}\ {\isacharparenleft}{\kern0pt}nat\ {\isasymRightarrow}\ {\isacharprime}{\kern0pt}a\ rlexp{\isacharparenright}{\kern0pt}\ {\isasymRightarrow}\ bool{\isachardoublequoteclose}\ \isakeywordTWO{where}\isanewline
\ \ {\isachardoublequoteopen}solution{\isacharunderscore}{\kern0pt}ineq{\isacharunderscore}{\kern0pt}sys\ sys\ sols\ {\isasymequiv}\ {\isasymforall}v{\isachardot}{\kern0pt}\ {\isacharparenleft}{\kern0pt}{\isasymforall}x{\isachardot}{\kern0pt}\ v\ x\ {\isacharequal}{\kern0pt}\ eval\ {\isacharparenleft}{\kern0pt}sols\ x{\isacharparenright}{\kern0pt}\ v{\isacharparenright}{\kern0pt}\ {\isasymlongrightarrow}\ solves{\isacharunderscore}{\kern0pt}ineq{\isacharunderscore}{\kern0pt}sys{\isacharunderscore}{\kern0pt}comm\ sys\ v{\isachardoublequoteclose}%
\begin{isamarkuptext}%
Given the \isa{x}-th equation \isa{eq}, \isa{sol} is a minimal partial solution of this equation if and
only if
\begin{enumerate}
\item \textit{sol} is a partial solution of \textit{eq}
\item \textit{sol} is a proper partial solution (i.e. it does not depend on \textit{x}) and only
  depends on variables occurring in the equation \textit{eq}
\item no partial solution of the equation \textit{eq} is smaller than \textit{sol}
\end{enumerate}%
\end{isamarkuptext}\isamarkuptrue%
\isakeywordONE{definition}\isamarkupfalse%
\ partial{\isacharunderscore}{\kern0pt}min{\isacharunderscore}{\kern0pt}sol{\isacharunderscore}{\kern0pt}one{\isacharunderscore}{\kern0pt}ineq\ {\isacharcolon}{\kern0pt}{\isacharcolon}{\kern0pt}\ {\isachardoublequoteopen}nat\ {\isasymRightarrow}\ {\isacharprime}{\kern0pt}a\ rlexp\ {\isasymRightarrow}\ {\isacharprime}{\kern0pt}a\ rlexp\ {\isasymRightarrow}\ bool{\isachardoublequoteclose}\ \isakeywordTWO{where}\isanewline
\ \ {\isachardoublequoteopen}partial{\isacharunderscore}{\kern0pt}min{\isacharunderscore}{\kern0pt}sol{\isacharunderscore}{\kern0pt}one{\isacharunderscore}{\kern0pt}ineq\ x\ eq\ sol\ {\isasymequiv}\isanewline
\ \ \ \ partial{\isacharunderscore}{\kern0pt}sol{\isacharunderscore}{\kern0pt}ineq\ x\ eq\ sol\ {\isasymand}\isanewline
\ \ \ \ vars\ sol\ {\isasymsubseteq}\ vars\ eq\ {\isacharminus}{\kern0pt}\ {\isacharbraceleft}{\kern0pt}x{\isacharbraceright}{\kern0pt}\ {\isasymand}\isanewline
\ \ \ \ {\isacharparenleft}{\kern0pt}{\isasymforall}sol{\isacharprime}{\kern0pt}\ v{\isacharprime}{\kern0pt}{\isachardot}{\kern0pt}\ solves{\isacharunderscore}{\kern0pt}ineq{\isacharunderscore}{\kern0pt}comm\ x\ eq\ v{\isacharprime}{\kern0pt}\ {\isasymand}\ v{\isacharprime}{\kern0pt}\ x\ {\isacharequal}{\kern0pt}\ eval\ sol{\isacharprime}{\kern0pt}\ v{\isacharprime}{\kern0pt}\isanewline
\ \ \ \ \ \ \ \ \ \ \ \ \ \ \ {\isasymlongrightarrow}\ {\isasymPsi}\ {\isacharparenleft}{\kern0pt}eval\ sol\ v{\isacharprime}{\kern0pt}{\isacharparenright}{\kern0pt}\ {\isasymsubseteq}\ {\isasymPsi}\ {\isacharparenleft}{\kern0pt}v{\isacharprime}{\kern0pt}\ x{\isacharparenright}{\kern0pt}{\isacharparenright}{\kern0pt}{\isachardoublequoteclose}%
\begin{isamarkuptext}%
Given a whole system of equations \isa{sys}, we can generalize the previous definition such that
\isa{sols} is a minimal solution (possibly dependent on the variables $X_n, X_{n+1}, \dots$) of
the first \isa{n} equations. Besides the three conditions described above, we introduce a forth
condition: \isa{sols\ i\ {\isacharequal}{\kern0pt}\ Var\ i} for \isa{i\ {\isasymge}\ n}, i.e. \isa{sols} assigns only spurious solutions to the
equations which are not yet solved:%
\end{isamarkuptext}\isamarkuptrue%
\isakeywordONE{definition}\isamarkupfalse%
\ partial{\isacharunderscore}{\kern0pt}min{\isacharunderscore}{\kern0pt}sol{\isacharunderscore}{\kern0pt}ineq{\isacharunderscore}{\kern0pt}sys\ {\isacharcolon}{\kern0pt}{\isacharcolon}{\kern0pt}\ {\isachardoublequoteopen}nat\ {\isasymRightarrow}\ {\isacharprime}{\kern0pt}a\ eq{\isacharunderscore}{\kern0pt}sys\ {\isasymRightarrow}\ {\isacharparenleft}{\kern0pt}nat\ {\isasymRightarrow}\ {\isacharprime}{\kern0pt}a\ rlexp{\isacharparenright}{\kern0pt}\ {\isasymRightarrow}\ bool{\isachardoublequoteclose}\ \isakeywordTWO{where}\isanewline
\ \ {\isachardoublequoteopen}partial{\isacharunderscore}{\kern0pt}min{\isacharunderscore}{\kern0pt}sol{\isacharunderscore}{\kern0pt}ineq{\isacharunderscore}{\kern0pt}sys\ n\ sys\ sols\ {\isasymequiv}\isanewline
\ \ \ \ solution{\isacharunderscore}{\kern0pt}ineq{\isacharunderscore}{\kern0pt}sys\ {\isacharparenleft}{\kern0pt}take\ n\ sys{\isacharparenright}{\kern0pt}\ sols\ {\isasymand}\isanewline
\ \ \ \ {\isacharparenleft}{\kern0pt}{\isasymforall}i\ {\isasymge}\ n{\isachardot}{\kern0pt}\ sols\ i\ {\isacharequal}{\kern0pt}\ Var\ i{\isacharparenright}{\kern0pt}\ {\isasymand}\isanewline
\ \ \ \ {\isacharparenleft}{\kern0pt}{\isasymforall}i\ {\isacharless}{\kern0pt}\ n{\isachardot}{\kern0pt}\ {\isasymforall}x\ {\isasymin}\ vars\ {\isacharparenleft}{\kern0pt}sols\ i{\isacharparenright}{\kern0pt}{\isachardot}{\kern0pt}\ x\ {\isasymge}\ n\ {\isasymand}\ x\ {\isacharless}{\kern0pt}\ length\ sys{\isacharparenright}{\kern0pt}\ {\isasymand}\isanewline
\ \ \ \ {\isacharparenleft}{\kern0pt}{\isasymforall}sols{\isacharprime}{\kern0pt}\ v{\isacharprime}{\kern0pt}{\isachardot}{\kern0pt}\ {\isacharparenleft}{\kern0pt}{\isasymforall}x{\isachardot}{\kern0pt}\ v{\isacharprime}{\kern0pt}\ x\ {\isacharequal}{\kern0pt}\ eval\ {\isacharparenleft}{\kern0pt}sols{\isacharprime}{\kern0pt}\ x{\isacharparenright}{\kern0pt}\ v{\isacharprime}{\kern0pt}{\isacharparenright}{\kern0pt}\isanewline
\ \ \ \ \ \ \ \ \ \ \ \ \ \ \ \ \ \ {\isasymand}\ solves{\isacharunderscore}{\kern0pt}ineq{\isacharunderscore}{\kern0pt}sys{\isacharunderscore}{\kern0pt}comm\ {\isacharparenleft}{\kern0pt}take\ n\ sys{\isacharparenright}{\kern0pt}\ v{\isacharprime}{\kern0pt}\isanewline
\ \ \ \ \ \ \ \ \ \ \ \ \ \ \ \ \ \ {\isasymlongrightarrow}\ {\isacharparenleft}{\kern0pt}{\isasymforall}i{\isachardot}{\kern0pt}\ {\isasymPsi}\ {\isacharparenleft}{\kern0pt}eval\ {\isacharparenleft}{\kern0pt}sols\ i{\isacharparenright}{\kern0pt}\ v{\isacharprime}{\kern0pt}{\isacharparenright}{\kern0pt}\ {\isasymsubseteq}\ {\isasymPsi}\ {\isacharparenleft}{\kern0pt}v{\isacharprime}{\kern0pt}\ i{\isacharparenright}{\kern0pt}{\isacharparenright}{\kern0pt}{\isacharparenright}{\kern0pt}{\isachardoublequoteclose}%
\begin{isamarkuptext}%
If the Parikh image of two equations \isa{f} and \isa{g} is identical on all valuations, then their
minimal partial solutions are identical, too:%
\end{isamarkuptext}\isamarkuptrue%
\isakeywordONE{lemma}\isamarkupfalse%
\ same{\isacharunderscore}{\kern0pt}min{\isacharunderscore}{\kern0pt}sol{\isacharunderscore}{\kern0pt}if{\isacharunderscore}{\kern0pt}same{\isacharunderscore}{\kern0pt}parikh{\isacharunderscore}{\kern0pt}img{\isacharcolon}{\kern0pt}\isanewline
\ \ \isakeywordTWO{assumes}\ same{\isacharunderscore}{\kern0pt}parikh{\isacharunderscore}{\kern0pt}img{\isacharcolon}{\kern0pt}\ {\isachardoublequoteopen}{\isasymforall}v{\isachardot}{\kern0pt}\ {\isasymPsi}\ {\isacharparenleft}{\kern0pt}eval\ f\ v{\isacharparenright}{\kern0pt}\ {\isacharequal}{\kern0pt}\ {\isasymPsi}\ {\isacharparenleft}{\kern0pt}eval\ g\ v{\isacharparenright}{\kern0pt}{\isachardoublequoteclose}\isanewline
\ \ \ \ \ \ \isakeywordTWO{and}\ same{\isacharunderscore}{\kern0pt}vars{\isacharcolon}{\kern0pt}\ \ \ \ \ \ \ {\isachardoublequoteopen}vars\ f\ {\isacharminus}{\kern0pt}\ {\isacharbraceleft}{\kern0pt}x{\isacharbraceright}{\kern0pt}\ {\isacharequal}{\kern0pt}\ vars\ g\ {\isacharminus}{\kern0pt}\ {\isacharbraceleft}{\kern0pt}x{\isacharbraceright}{\kern0pt}{\isachardoublequoteclose}\isanewline
\ \ \ \ \ \ \isakeywordTWO{and}\ minimal{\isacharunderscore}{\kern0pt}sol{\isacharcolon}{\kern0pt}\ \ \ \ \ {\isachardoublequoteopen}partial{\isacharunderscore}{\kern0pt}min{\isacharunderscore}{\kern0pt}sol{\isacharunderscore}{\kern0pt}one{\isacharunderscore}{\kern0pt}ineq\ x\ f\ sol{\isachardoublequoteclose}\isanewline
\ \ \ \ \isakeywordTWO{shows}\ \ \ \ \ \ \ \ \ \ \ \ \ \ \ \ \ \ {\isachardoublequoteopen}partial{\isacharunderscore}{\kern0pt}min{\isacharunderscore}{\kern0pt}sol{\isacharunderscore}{\kern0pt}one{\isacharunderscore}{\kern0pt}ineq\ x\ g\ sol{\isachardoublequoteclose}\isanewline
%
\isadelimproof
%
\endisadelimproof
%
\isatagproof
\isakeywordONE{proof}\isamarkupfalse%
\ {\isacharminus}{\kern0pt}\isanewline
\ \ \isakeywordONE{from}\isamarkupfalse%
\ minimal{\isacharunderscore}{\kern0pt}sol\ \isakeywordONE{have}\isamarkupfalse%
\ {\isachardoublequoteopen}vars\ sol\ {\isasymsubseteq}\ vars\ g\ {\isacharminus}{\kern0pt}\ {\isacharbraceleft}{\kern0pt}x{\isacharbraceright}{\kern0pt}{\isachardoublequoteclose}\isanewline
\ \ \ \ \isakeywordONE{unfolding}\isamarkupfalse%
\ partial{\isacharunderscore}{\kern0pt}min{\isacharunderscore}{\kern0pt}sol{\isacharunderscore}{\kern0pt}one{\isacharunderscore}{\kern0pt}ineq{\isacharunderscore}{\kern0pt}def\ \isakeywordONE{using}\isamarkupfalse%
\ same{\isacharunderscore}{\kern0pt}vars\ \isakeywordONE{by}\isamarkupfalse%
\ blast\isanewline
\ \ \isakeywordONE{moreover}\isamarkupfalse%
\ \isakeywordONE{from}\isamarkupfalse%
\ same{\isacharunderscore}{\kern0pt}parikh{\isacharunderscore}{\kern0pt}img\ minimal{\isacharunderscore}{\kern0pt}sol\ \isakeywordONE{have}\isamarkupfalse%
\ {\isachardoublequoteopen}partial{\isacharunderscore}{\kern0pt}sol{\isacharunderscore}{\kern0pt}ineq\ x\ g\ sol{\isachardoublequoteclose}\isanewline
\ \ \ \ \isakeywordONE{unfolding}\isamarkupfalse%
\ partial{\isacharunderscore}{\kern0pt}min{\isacharunderscore}{\kern0pt}sol{\isacharunderscore}{\kern0pt}one{\isacharunderscore}{\kern0pt}ineq{\isacharunderscore}{\kern0pt}def\ partial{\isacharunderscore}{\kern0pt}sol{\isacharunderscore}{\kern0pt}ineq{\isacharunderscore}{\kern0pt}def\ solves{\isacharunderscore}{\kern0pt}ineq{\isacharunderscore}{\kern0pt}comm{\isacharunderscore}{\kern0pt}def\ \isakeywordONE{by}\isamarkupfalse%
\ simp\isanewline
\ \ \isakeywordONE{moreover}\isamarkupfalse%
\ \isakeywordONE{from}\isamarkupfalse%
\ same{\isacharunderscore}{\kern0pt}parikh{\isacharunderscore}{\kern0pt}img\ minimal{\isacharunderscore}{\kern0pt}sol\ \isakeywordONE{have}\isamarkupfalse%
\ {\isachardoublequoteopen}{\isasymforall}sol{\isacharprime}{\kern0pt}\ v{\isacharprime}{\kern0pt}{\isachardot}{\kern0pt}\ solves{\isacharunderscore}{\kern0pt}ineq{\isacharunderscore}{\kern0pt}comm\ x\ g\ v{\isacharprime}{\kern0pt}\ {\isasymand}\ v{\isacharprime}{\kern0pt}\ x\ {\isacharequal}{\kern0pt}\ eval\ sol{\isacharprime}{\kern0pt}\ v{\isacharprime}{\kern0pt}\isanewline
\ \ \ \ \ \ \ \ \ \ \ \ \ \ \ {\isasymlongrightarrow}\ {\isasymPsi}\ {\isacharparenleft}{\kern0pt}eval\ sol\ v{\isacharprime}{\kern0pt}{\isacharparenright}{\kern0pt}\ {\isasymsubseteq}\ {\isasymPsi}\ {\isacharparenleft}{\kern0pt}v{\isacharprime}{\kern0pt}\ x{\isacharparenright}{\kern0pt}{\isachardoublequoteclose}\isanewline
\ \ \ \ \isakeywordONE{unfolding}\isamarkupfalse%
\ partial{\isacharunderscore}{\kern0pt}min{\isacharunderscore}{\kern0pt}sol{\isacharunderscore}{\kern0pt}one{\isacharunderscore}{\kern0pt}ineq{\isacharunderscore}{\kern0pt}def\ solves{\isacharunderscore}{\kern0pt}ineq{\isacharunderscore}{\kern0pt}comm{\isacharunderscore}{\kern0pt}def\ \isakeywordONE{by}\isamarkupfalse%
\ blast\isanewline
\ \ \isakeywordONE{ultimately}\isamarkupfalse%
\ \isakeywordTHREE{show}\isamarkupfalse%
\ {\isacharquery}{\kern0pt}thesis\ \isakeywordONE{unfolding}\isamarkupfalse%
\ partial{\isacharunderscore}{\kern0pt}min{\isacharunderscore}{\kern0pt}sol{\isacharunderscore}{\kern0pt}one{\isacharunderscore}{\kern0pt}ineq{\isacharunderscore}{\kern0pt}def\ \isakeywordONE{by}\isamarkupfalse%
\ fast\isanewline
\isakeywordONE{qed}\isamarkupfalse%
%
\endisatagproof
{\isafoldproof}%
%
\isadelimproof
%
\endisadelimproof
%
\isadelimdocument
%
\endisadelimdocument
%
\isatagdocument
%
\isamarkupsubsection{CFLs as minimal solution of systems of equations%
}
\isamarkuptrue%
%
\endisatagdocument
{\isafolddocument}%
%
\isadelimdocument
%
\endisadelimdocument
%
\begin{isamarkuptext}%
We show that each CFG induces a system of equations of the first type, i.e. without Parikh images,
such that the CFG's language is the minimal solution of the system. First, we describe how to derive
the system of equations from a CFG. This requires us to fix some bijection between the variables in
the system and the non-terminals occurring in the CFG:%
\end{isamarkuptext}\isamarkuptrue%
\isakeywordONE{definition}\isamarkupfalse%
\ bij{\isacharunderscore}{\kern0pt}Nt{\isacharunderscore}{\kern0pt}Var\ {\isacharcolon}{\kern0pt}{\isacharcolon}{\kern0pt}\ {\isachardoublequoteopen}{\isacharprime}{\kern0pt}n\ set\ {\isasymRightarrow}\ {\isacharparenleft}{\kern0pt}nat\ {\isasymRightarrow}\ {\isacharprime}{\kern0pt}n{\isacharparenright}{\kern0pt}\ {\isasymRightarrow}\ {\isacharparenleft}{\kern0pt}{\isacharprime}{\kern0pt}n\ {\isasymRightarrow}\ nat{\isacharparenright}{\kern0pt}\ {\isasymRightarrow}\ bool{\isachardoublequoteclose}\ \isakeywordTWO{where}\isanewline
\ \ {\isachardoublequoteopen}bij{\isacharunderscore}{\kern0pt}Nt{\isacharunderscore}{\kern0pt}Var\ A\ {\isasymgamma}\ {\isasymgamma}{\isacharprime}{\kern0pt}\ {\isasymequiv}\ bij{\isacharunderscore}{\kern0pt}betw\ {\isasymgamma}\ {\isacharbraceleft}{\kern0pt}{\isachardot}{\kern0pt}{\isachardot}{\kern0pt}{\isacharless}{\kern0pt}\ card\ A{\isacharbraceright}{\kern0pt}\ A\ {\isasymand}\ bij{\isacharunderscore}{\kern0pt}betw\ {\isasymgamma}{\isacharprime}{\kern0pt}\ A\ {\isacharbraceleft}{\kern0pt}{\isachardot}{\kern0pt}{\isachardot}{\kern0pt}{\isacharless}{\kern0pt}\ card\ A{\isacharbraceright}{\kern0pt}\isanewline
\ \ \ \ \ \ \ \ \ \ \ \ \ \ \ \ \ \ \ \ \ \ \ \ \ \ {\isasymand}\ {\isacharparenleft}{\kern0pt}{\isasymforall}x\ {\isasymin}\ {\isacharbraceleft}{\kern0pt}{\isachardot}{\kern0pt}{\isachardot}{\kern0pt}{\isacharless}{\kern0pt}\ card\ A{\isacharbraceright}{\kern0pt}{\isachardot}{\kern0pt}\ {\isasymgamma}{\isacharprime}{\kern0pt}\ {\isacharparenleft}{\kern0pt}{\isasymgamma}\ x{\isacharparenright}{\kern0pt}\ {\isacharequal}{\kern0pt}\ x{\isacharparenright}{\kern0pt}\ {\isasymand}\ {\isacharparenleft}{\kern0pt}{\isasymforall}y\ {\isasymin}\ A{\isachardot}{\kern0pt}\ {\isasymgamma}\ {\isacharparenleft}{\kern0pt}{\isasymgamma}{\isacharprime}{\kern0pt}\ y{\isacharparenright}{\kern0pt}\ {\isacharequal}{\kern0pt}\ y{\isacharparenright}{\kern0pt}{\isachardoublequoteclose}\isanewline
\isanewline
\isakeywordONE{lemma}\isamarkupfalse%
\ exists{\isacharunderscore}{\kern0pt}bij{\isacharunderscore}{\kern0pt}Nt{\isacharunderscore}{\kern0pt}Var{\isacharcolon}{\kern0pt}\ {\isachardoublequoteopen}finite\ A\ {\isasymLongrightarrow}\ {\isasymexists}{\isasymgamma}\ {\isasymgamma}{\isacharprime}{\kern0pt}{\isachardot}{\kern0pt}\ bij{\isacharunderscore}{\kern0pt}Nt{\isacharunderscore}{\kern0pt}Var\ A\ {\isasymgamma}\ {\isasymgamma}{\isacharprime}{\kern0pt}{\isachardoublequoteclose}\isanewline
%
\isadelimproof
%
\endisadelimproof
%
\isatagproof
\isakeywordONE{proof}\isamarkupfalse%
\ {\isacharminus}{\kern0pt}\isanewline
\ \ \isakeywordTHREE{assume}\isamarkupfalse%
\ {\isachardoublequoteopen}finite\ A{\isachardoublequoteclose}\isanewline
\ \ \isakeywordONE{then}\isamarkupfalse%
\ \isakeywordONE{have}\isamarkupfalse%
\ {\isachardoublequoteopen}{\isasymexists}{\isasymgamma}{\isachardot}{\kern0pt}\ bij{\isacharunderscore}{\kern0pt}betw\ {\isasymgamma}\ {\isacharbraceleft}{\kern0pt}{\isachardot}{\kern0pt}{\isachardot}{\kern0pt}{\isacharless}{\kern0pt}\ card\ A{\isacharbraceright}{\kern0pt}\ A{\isachardoublequoteclose}\ \isakeywordONE{by}\isamarkupfalse%
\ {\isacharparenleft}{\kern0pt}simp\ add{\isacharcolon}{\kern0pt}\ bij{\isacharunderscore}{\kern0pt}betw{\isacharunderscore}{\kern0pt}iff{\isacharunderscore}{\kern0pt}card{\isacharparenright}{\kern0pt}\isanewline
\ \ \isakeywordONE{then}\isamarkupfalse%
\ \isakeywordTHREE{obtain}\isamarkupfalse%
\ {\isasymgamma}\ \isakeywordTWO{where}\ {\isadigit{1}}{\isacharcolon}{\kern0pt}\ {\isachardoublequoteopen}bij{\isacharunderscore}{\kern0pt}betw\ {\isasymgamma}\ {\isacharbraceleft}{\kern0pt}{\isachardot}{\kern0pt}{\isachardot}{\kern0pt}{\isacharless}{\kern0pt}\ card\ A{\isacharbraceright}{\kern0pt}\ A{\isachardoublequoteclose}\ \isakeywordONE{by}\isamarkupfalse%
\ blast\isanewline
\ \ \isakeywordONE{let}\isamarkupfalse%
\ {\isacharquery}{\kern0pt}{\isasymgamma}{\isacharprime}{\kern0pt}\ {\isacharequal}{\kern0pt}\ {\isachardoublequoteopen}the{\isacharunderscore}{\kern0pt}inv{\isacharunderscore}{\kern0pt}into\ {\isacharbraceleft}{\kern0pt}{\isachardot}{\kern0pt}{\isachardot}{\kern0pt}{\isacharless}{\kern0pt}\ card\ A{\isacharbraceright}{\kern0pt}\ {\isasymgamma}{\isachardoublequoteclose}\isanewline
\ \ \isakeywordONE{from}\isamarkupfalse%
\ the{\isacharunderscore}{\kern0pt}inv{\isacharunderscore}{\kern0pt}into{\isacharunderscore}{\kern0pt}f{\isacharunderscore}{\kern0pt}f\ {\isadigit{1}}\ \isakeywordONE{have}\isamarkupfalse%
\ {\isadigit{2}}{\isacharcolon}{\kern0pt}\ {\isachardoublequoteopen}{\isasymforall}x\ {\isasymin}\ {\isacharbraceleft}{\kern0pt}{\isachardot}{\kern0pt}{\isachardot}{\kern0pt}{\isacharless}{\kern0pt}\ card\ A{\isacharbraceright}{\kern0pt}{\isachardot}{\kern0pt}\ {\isacharquery}{\kern0pt}{\isasymgamma}{\isacharprime}{\kern0pt}\ {\isacharparenleft}{\kern0pt}{\isasymgamma}\ x{\isacharparenright}{\kern0pt}\ {\isacharequal}{\kern0pt}\ x{\isachardoublequoteclose}\ \isakeywordONE{unfolding}\isamarkupfalse%
\ bij{\isacharunderscore}{\kern0pt}betw{\isacharunderscore}{\kern0pt}def\ \isakeywordONE{by}\isamarkupfalse%
\ fast\isanewline
\ \ \isakeywordONE{from}\isamarkupfalse%
\ bij{\isacharunderscore}{\kern0pt}betw{\isacharunderscore}{\kern0pt}the{\isacharunderscore}{\kern0pt}inv{\isacharunderscore}{\kern0pt}into{\isacharbrackleft}{\kern0pt}OF\ {\isadigit{1}}{\isacharbrackright}{\kern0pt}\ \isakeywordONE{have}\isamarkupfalse%
\ {\isadigit{3}}{\isacharcolon}{\kern0pt}\ {\isachardoublequoteopen}bij{\isacharunderscore}{\kern0pt}betw\ {\isacharquery}{\kern0pt}{\isasymgamma}{\isacharprime}{\kern0pt}\ A\ {\isacharbraceleft}{\kern0pt}{\isachardot}{\kern0pt}{\isachardot}{\kern0pt}{\isacharless}{\kern0pt}\ card\ A{\isacharbraceright}{\kern0pt}{\isachardoublequoteclose}\ \isakeywordONE{by}\isamarkupfalse%
\ blast\isanewline
\ \ \isakeywordONE{with}\isamarkupfalse%
\ {\isadigit{1}}\ f{\isacharunderscore}{\kern0pt}the{\isacharunderscore}{\kern0pt}inv{\isacharunderscore}{\kern0pt}into{\isacharunderscore}{\kern0pt}f{\isacharunderscore}{\kern0pt}bij{\isacharunderscore}{\kern0pt}betw\ \isakeywordONE{have}\isamarkupfalse%
\ {\isadigit{4}}{\isacharcolon}{\kern0pt}\ {\isachardoublequoteopen}{\isasymforall}y\ {\isasymin}\ A{\isachardot}{\kern0pt}\ {\isasymgamma}\ {\isacharparenleft}{\kern0pt}{\isacharquery}{\kern0pt}{\isasymgamma}{\isacharprime}{\kern0pt}\ y{\isacharparenright}{\kern0pt}\ {\isacharequal}{\kern0pt}\ y{\isachardoublequoteclose}\ \isakeywordONE{by}\isamarkupfalse%
\ metis\isanewline
\ \ \isakeywordONE{from}\isamarkupfalse%
\ {\isadigit{1}}\ {\isadigit{2}}\ {\isadigit{3}}\ {\isadigit{4}}\ \isakeywordTHREE{show}\isamarkupfalse%
\ {\isacharquery}{\kern0pt}thesis\ \isakeywordONE{unfolding}\isamarkupfalse%
\ bij{\isacharunderscore}{\kern0pt}Nt{\isacharunderscore}{\kern0pt}Var{\isacharunderscore}{\kern0pt}def\ \isakeywordONE{by}\isamarkupfalse%
\ blast\isanewline
\isakeywordONE{qed}\isamarkupfalse%
%
\endisatagproof
{\isafoldproof}%
%
\isadelimproof
\isanewline
%
\endisadelimproof
\isanewline
\isanewline
\isakeywordONE{locale}\isamarkupfalse%
\ CFG{\isacharunderscore}{\kern0pt}eq{\isacharunderscore}{\kern0pt}sys\ {\isacharequal}{\kern0pt}\isanewline
\ \ \isakeywordTWO{fixes}\ P\ {\isacharcolon}{\kern0pt}{\isacharcolon}{\kern0pt}\ {\isachardoublequoteopen}{\isacharparenleft}{\kern0pt}{\isacharprime}{\kern0pt}n{\isacharcomma}{\kern0pt}{\isacharprime}{\kern0pt}a{\isacharparenright}{\kern0pt}\ Prods{\isachardoublequoteclose}\isanewline
\ \ \isakeywordTWO{fixes}\ S\ {\isacharcolon}{\kern0pt}{\isacharcolon}{\kern0pt}\ {\isacharprime}{\kern0pt}n\isanewline
\ \ \isakeywordTWO{fixes}\ {\isasymgamma}\ {\isacharcolon}{\kern0pt}{\isacharcolon}{\kern0pt}\ {\isachardoublequoteopen}nat\ {\isasymRightarrow}\ {\isacharprime}{\kern0pt}n{\isachardoublequoteclose}\isanewline
\ \ \isakeywordTWO{fixes}\ {\isasymgamma}{\isacharprime}{\kern0pt}\ {\isacharcolon}{\kern0pt}{\isacharcolon}{\kern0pt}\ {\isachardoublequoteopen}{\isacharprime}{\kern0pt}n\ {\isasymRightarrow}\ nat{\isachardoublequoteclose}\isanewline
\ \ \isakeywordTWO{assumes}\ finite{\isacharunderscore}{\kern0pt}P{\isacharcolon}{\kern0pt}\ {\isachardoublequoteopen}finite\ P{\isachardoublequoteclose}\isanewline
\ \ \isakeywordTWO{assumes}\ bij{\isacharunderscore}{\kern0pt}{\isasymgamma}{\isacharunderscore}{\kern0pt}{\isasymgamma}{\isacharprime}{\kern0pt}{\isacharcolon}{\kern0pt}\ \ {\isachardoublequoteopen}bij{\isacharunderscore}{\kern0pt}Nt{\isacharunderscore}{\kern0pt}Var\ {\isacharparenleft}{\kern0pt}Nts\ P{\isacharparenright}{\kern0pt}\ {\isasymgamma}\ {\isasymgamma}{\isacharprime}{\kern0pt}{\isachardoublequoteclose}\isanewline
\isakeywordTWO{begin}%
\begin{isamarkuptext}%
The following definitions construct a regular language expression for a single production. This
happens step by step, i.e. starting with a single symbol (terminal or non-terminal) and then extending
this to a single production. The definitions closely follow the definitions \isa{\isaconst{inst{\isacharunderscore}{\kern0pt}sym}},
\isa{\isaconst{concats}} and \isa{\isaconst{inst{\isacharunderscore}{\kern0pt}syms}} in \isa{Context{\isacharunderscore}{\kern0pt}Free{\isacharunderscore}{\kern0pt}Grammar{\isachardot}{\kern0pt}Context{\isacharunderscore}{\kern0pt}Free{\isacharunderscore}{\kern0pt}Language}.%
\end{isamarkuptext}\isamarkuptrue%
\isakeywordONE{definition}\isamarkupfalse%
\ rlexp{\isacharunderscore}{\kern0pt}sym\ {\isacharcolon}{\kern0pt}{\isacharcolon}{\kern0pt}\ {\isachardoublequoteopen}{\isacharparenleft}{\kern0pt}{\isacharprime}{\kern0pt}n{\isacharcomma}{\kern0pt}\ {\isacharprime}{\kern0pt}a{\isacharparenright}{\kern0pt}\ sym\ {\isasymRightarrow}\ {\isacharprime}{\kern0pt}a\ rlexp{\isachardoublequoteclose}\ \isakeywordTWO{where}\isanewline
\ \ {\isachardoublequoteopen}rlexp{\isacharunderscore}{\kern0pt}sym\ s\ {\isacharequal}{\kern0pt}\ {\isacharparenleft}{\kern0pt}case\ s\ of\ Tm\ a\ {\isasymRightarrow}\ Const\ {\isacharbraceleft}{\kern0pt}{\isacharbrackleft}{\kern0pt}a{\isacharbrackright}{\kern0pt}{\isacharbraceright}{\kern0pt}\ {\isacharbar}{\kern0pt}\ Nt\ A\ {\isasymRightarrow}\ Var\ {\isacharparenleft}{\kern0pt}{\isasymgamma}{\isacharprime}{\kern0pt}\ A{\isacharparenright}{\kern0pt}{\isacharparenright}{\kern0pt}{\isachardoublequoteclose}\isanewline
\isanewline
\isakeywordONE{definition}\isamarkupfalse%
\ rlexp{\isacharunderscore}{\kern0pt}concats\ {\isacharcolon}{\kern0pt}{\isacharcolon}{\kern0pt}\ {\isachardoublequoteopen}{\isacharprime}{\kern0pt}a\ rlexp\ list\ {\isasymRightarrow}\ {\isacharprime}{\kern0pt}a\ rlexp{\isachardoublequoteclose}\ \isakeywordTWO{where}\isanewline
\ \ {\isachardoublequoteopen}rlexp{\isacharunderscore}{\kern0pt}concats\ fs\ {\isacharequal}{\kern0pt}\ foldr\ Concat\ fs\ {\isacharparenleft}{\kern0pt}Const\ {\isacharbraceleft}{\kern0pt}{\isacharbrackleft}{\kern0pt}{\isacharbrackright}{\kern0pt}{\isacharbraceright}{\kern0pt}{\isacharparenright}{\kern0pt}{\isachardoublequoteclose}\isanewline
\isanewline
\isakeywordONE{definition}\isamarkupfalse%
\ rlexp{\isacharunderscore}{\kern0pt}syms\ {\isacharcolon}{\kern0pt}{\isacharcolon}{\kern0pt}\ {\isachardoublequoteopen}{\isacharparenleft}{\kern0pt}{\isacharprime}{\kern0pt}n{\isacharcomma}{\kern0pt}\ {\isacharprime}{\kern0pt}a{\isacharparenright}{\kern0pt}\ syms\ {\isasymRightarrow}\ {\isacharprime}{\kern0pt}a\ rlexp{\isachardoublequoteclose}\ \isakeywordTWO{where}\isanewline
\ \ {\isachardoublequoteopen}rlexp{\isacharunderscore}{\kern0pt}syms\ w\ {\isacharequal}{\kern0pt}\ rlexp{\isacharunderscore}{\kern0pt}concats\ {\isacharparenleft}{\kern0pt}map\ rlexp{\isacharunderscore}{\kern0pt}sym\ w{\isacharparenright}{\kern0pt}{\isachardoublequoteclose}%
\begin{isamarkuptext}%
Now it is shown that the regular language expression constructed for a single production
is \isa{\isaconst{reg{\isacharunderscore}{\kern0pt}eval}}. Again, this happens step by step:%
\end{isamarkuptext}\isamarkuptrue%
\isakeywordONE{lemma}\isamarkupfalse%
\ rlexp{\isacharunderscore}{\kern0pt}sym{\isacharunderscore}{\kern0pt}reg{\isacharcolon}{\kern0pt}\ {\isachardoublequoteopen}reg{\isacharunderscore}{\kern0pt}eval\ {\isacharparenleft}{\kern0pt}rlexp{\isacharunderscore}{\kern0pt}sym\ s{\isacharparenright}{\kern0pt}{\isachardoublequoteclose}\isanewline
%
\isadelimproof
%
\endisadelimproof
%
\isatagproof
\isakeywordONE{unfolding}\isamarkupfalse%
\ rlexp{\isacharunderscore}{\kern0pt}sym{\isacharunderscore}{\kern0pt}def\ \isakeywordONE{proof}\isamarkupfalse%
\ {\isacharparenleft}{\kern0pt}induction\ s{\isacharparenright}{\kern0pt}\isanewline
\ \ \isakeywordTHREE{case}\isamarkupfalse%
\ {\isacharparenleft}{\kern0pt}Tm\ x{\isacharparenright}{\kern0pt}\isanewline
\ \ \isakeywordONE{have}\isamarkupfalse%
\ {\isachardoublequoteopen}regular{\isacharunderscore}{\kern0pt}lang\ {\isacharbraceleft}{\kern0pt}{\isacharbrackleft}{\kern0pt}x{\isacharbrackright}{\kern0pt}{\isacharbraceright}{\kern0pt}{\isachardoublequoteclose}\ \isakeywordONE{by}\isamarkupfalse%
\ {\isacharparenleft}{\kern0pt}meson\ lang{\isachardot}{\kern0pt}simps{\isacharparenleft}{\kern0pt}{\isadigit{3}}{\isacharparenright}{\kern0pt}{\isacharparenright}{\kern0pt}\isanewline
\ \ \isakeywordONE{then}\isamarkupfalse%
\ \isakeywordTHREE{show}\isamarkupfalse%
\ {\isacharquery}{\kern0pt}case\ \isakeywordONE{by}\isamarkupfalse%
\ auto\isanewline
\isakeywordONE{qed}\isamarkupfalse%
\ auto%
\endisatagproof
{\isafoldproof}%
%
\isadelimproof
\isanewline
%
\endisadelimproof
\isanewline
\isakeywordONE{lemma}\isamarkupfalse%
\ rlexp{\isacharunderscore}{\kern0pt}concats{\isacharunderscore}{\kern0pt}reg{\isacharcolon}{\kern0pt}\isanewline
\ \ \isakeywordTWO{assumes}\ {\isachardoublequoteopen}{\isasymforall}f\ {\isasymin}\ set\ fs{\isachardot}{\kern0pt}\ reg{\isacharunderscore}{\kern0pt}eval\ f{\isachardoublequoteclose}\isanewline
\ \ \ \ \isakeywordTWO{shows}\ {\isachardoublequoteopen}reg{\isacharunderscore}{\kern0pt}eval\ {\isacharparenleft}{\kern0pt}rlexp{\isacharunderscore}{\kern0pt}concats\ fs{\isacharparenright}{\kern0pt}{\isachardoublequoteclose}\isanewline
%
\isadelimproof
\ \ %
\endisadelimproof
%
\isatagproof
\isakeywordONE{using}\isamarkupfalse%
\ assms\ epsilon{\isacharunderscore}{\kern0pt}regular\ \isakeywordONE{unfolding}\isamarkupfalse%
\ rlexp{\isacharunderscore}{\kern0pt}concats{\isacharunderscore}{\kern0pt}def\ \isakeywordONE{by}\isamarkupfalse%
\ {\isacharparenleft}{\kern0pt}induction\ fs{\isacharparenright}{\kern0pt}\ auto%
\endisatagproof
{\isafoldproof}%
%
\isadelimproof
\isanewline
%
\endisadelimproof
\isanewline
\isakeywordONE{lemma}\isamarkupfalse%
\ rlexp{\isacharunderscore}{\kern0pt}syms{\isacharunderscore}{\kern0pt}reg{\isacharcolon}{\kern0pt}\ {\isachardoublequoteopen}reg{\isacharunderscore}{\kern0pt}eval\ {\isacharparenleft}{\kern0pt}rlexp{\isacharunderscore}{\kern0pt}syms\ w{\isacharparenright}{\kern0pt}{\isachardoublequoteclose}\isanewline
%
\isadelimproof
%
\endisadelimproof
%
\isatagproof
\isakeywordONE{proof}\isamarkupfalse%
\ {\isacharminus}{\kern0pt}\isanewline
\ \ \isakeywordONE{from}\isamarkupfalse%
\ rlexp{\isacharunderscore}{\kern0pt}sym{\isacharunderscore}{\kern0pt}reg\ \isakeywordONE{have}\isamarkupfalse%
\ {\isachardoublequoteopen}{\isasymforall}s\ {\isasymin}\ set\ w{\isachardot}{\kern0pt}\ reg{\isacharunderscore}{\kern0pt}eval\ {\isacharparenleft}{\kern0pt}rlexp{\isacharunderscore}{\kern0pt}sym\ s{\isacharparenright}{\kern0pt}{\isachardoublequoteclose}\ \isakeywordONE{by}\isamarkupfalse%
\ blast\isanewline
\ \ \isakeywordONE{with}\isamarkupfalse%
\ rlexp{\isacharunderscore}{\kern0pt}concats{\isacharunderscore}{\kern0pt}reg\ \isakeywordTHREE{show}\isamarkupfalse%
\ {\isacharquery}{\kern0pt}thesis\ \isakeywordONE{unfolding}\isamarkupfalse%
\ rlexp{\isacharunderscore}{\kern0pt}syms{\isacharunderscore}{\kern0pt}def\isanewline
\ \ \ \ \isakeywordONE{by}\isamarkupfalse%
\ {\isacharparenleft}{\kern0pt}metis\ {\isacharparenleft}{\kern0pt}no{\isacharunderscore}{\kern0pt}types{\isacharcomma}{\kern0pt}\ lifting{\isacharparenright}{\kern0pt}\ image{\isacharunderscore}{\kern0pt}iff\ list{\isachardot}{\kern0pt}set{\isacharunderscore}{\kern0pt}map{\isacharparenright}{\kern0pt}\isanewline
\isakeywordONE{qed}\isamarkupfalse%
%
\endisatagproof
{\isafoldproof}%
%
\isadelimproof
%
\endisadelimproof
%
\begin{isamarkuptext}%
The subsequent lemmas prove that all variables appearing in the regular language
expression of a single production correspond to non-terminals appearing in the production:%
\end{isamarkuptext}\isamarkuptrue%
\isakeywordONE{lemma}\isamarkupfalse%
\ rlexp{\isacharunderscore}{\kern0pt}sym{\isacharunderscore}{\kern0pt}vars{\isacharunderscore}{\kern0pt}Nt{\isacharcolon}{\kern0pt}\isanewline
\ \ \isakeywordTWO{assumes}\ {\isachardoublequoteopen}s\ {\isacharparenleft}{\kern0pt}{\isasymgamma}{\isacharprime}{\kern0pt}\ A{\isacharparenright}{\kern0pt}\ {\isacharequal}{\kern0pt}\ L\ A{\isachardoublequoteclose}\isanewline
\ \ \ \ \isakeywordTWO{shows}\ {\isachardoublequoteopen}vars\ {\isacharparenleft}{\kern0pt}rlexp{\isacharunderscore}{\kern0pt}sym\ {\isacharparenleft}{\kern0pt}Nt\ A{\isacharparenright}{\kern0pt}{\isacharparenright}{\kern0pt}\ {\isacharequal}{\kern0pt}\ {\isacharbraceleft}{\kern0pt}{\isasymgamma}{\isacharprime}{\kern0pt}\ A{\isacharbraceright}{\kern0pt}{\isachardoublequoteclose}\isanewline
%
\isadelimproof
\ \ %
\endisadelimproof
%
\isatagproof
\isakeywordONE{using}\isamarkupfalse%
\ assms\ \isakeywordONE{unfolding}\isamarkupfalse%
\ rlexp{\isacharunderscore}{\kern0pt}sym{\isacharunderscore}{\kern0pt}def\ \isakeywordONE{by}\isamarkupfalse%
\ simp%
\endisatagproof
{\isafoldproof}%
%
\isadelimproof
\isanewline
%
\endisadelimproof
\isanewline
\isakeywordONE{lemma}\isamarkupfalse%
\ rlexp{\isacharunderscore}{\kern0pt}sym{\isacharunderscore}{\kern0pt}vars{\isacharunderscore}{\kern0pt}Tm{\isacharcolon}{\kern0pt}\ {\isachardoublequoteopen}vars\ {\isacharparenleft}{\kern0pt}rlexp{\isacharunderscore}{\kern0pt}sym\ {\isacharparenleft}{\kern0pt}Tm\ x{\isacharparenright}{\kern0pt}{\isacharparenright}{\kern0pt}\ {\isacharequal}{\kern0pt}\ {\isacharbraceleft}{\kern0pt}{\isacharbraceright}{\kern0pt}{\isachardoublequoteclose}\isanewline
%
\isadelimproof
\ \ %
\endisadelimproof
%
\isatagproof
\isakeywordONE{unfolding}\isamarkupfalse%
\ rlexp{\isacharunderscore}{\kern0pt}sym{\isacharunderscore}{\kern0pt}def\ \isakeywordONE{by}\isamarkupfalse%
\ simp%
\endisatagproof
{\isafoldproof}%
%
\isadelimproof
\isanewline
%
\endisadelimproof
\isanewline
\isakeywordONE{lemma}\isamarkupfalse%
\ rlexp{\isacharunderscore}{\kern0pt}concats{\isacharunderscore}{\kern0pt}vars{\isacharcolon}{\kern0pt}\ {\isachardoublequoteopen}vars\ {\isacharparenleft}{\kern0pt}rlexp{\isacharunderscore}{\kern0pt}concats\ fs{\isacharparenright}{\kern0pt}\ {\isacharequal}{\kern0pt}\ {\isasymUnion}{\isacharparenleft}{\kern0pt}vars\ {\isacharbackquote}{\kern0pt}\ set\ fs{\isacharparenright}{\kern0pt}{\isachardoublequoteclose}\isanewline
%
\isadelimproof
\ \ %
\endisadelimproof
%
\isatagproof
\isakeywordONE{unfolding}\isamarkupfalse%
\ rlexp{\isacharunderscore}{\kern0pt}concats{\isacharunderscore}{\kern0pt}def\ \isakeywordONE{by}\isamarkupfalse%
\ {\isacharparenleft}{\kern0pt}induction\ fs{\isacharparenright}{\kern0pt}\ simp{\isacharunderscore}{\kern0pt}all%
\endisatagproof
{\isafoldproof}%
%
\isadelimproof
\isanewline
%
\endisadelimproof
\isanewline
\isanewline
\isakeywordONE{lemma}\isamarkupfalse%
\ insts{\isacharprime}{\kern0pt}{\isacharunderscore}{\kern0pt}vars{\isacharcolon}{\kern0pt}\ {\isachardoublequoteopen}vars\ {\isacharparenleft}{\kern0pt}rlexp{\isacharunderscore}{\kern0pt}syms\ w{\isacharparenright}{\kern0pt}\ {\isasymsubseteq}\ {\isasymgamma}{\isacharprime}{\kern0pt}\ {\isacharbackquote}{\kern0pt}\ nts{\isacharunderscore}{\kern0pt}syms\ w{\isachardoublequoteclose}\isanewline
%
\isadelimproof
%
\endisadelimproof
%
\isatagproof
\isakeywordONE{proof}\isamarkupfalse%
\isanewline
\ \ \isakeywordTHREE{fix}\isamarkupfalse%
\ x\isanewline
\ \ \isakeywordTHREE{assume}\isamarkupfalse%
\ {\isachardoublequoteopen}x\ {\isasymin}\ vars\ {\isacharparenleft}{\kern0pt}rlexp{\isacharunderscore}{\kern0pt}syms\ w{\isacharparenright}{\kern0pt}{\isachardoublequoteclose}\isanewline
\ \ \isakeywordONE{with}\isamarkupfalse%
\ rlexp{\isacharunderscore}{\kern0pt}concats{\isacharunderscore}{\kern0pt}vars\ \isakeywordONE{have}\isamarkupfalse%
\ {\isachardoublequoteopen}x\ {\isasymin}\ {\isasymUnion}{\isacharparenleft}{\kern0pt}vars\ {\isacharbackquote}{\kern0pt}\ set\ {\isacharparenleft}{\kern0pt}map\ rlexp{\isacharunderscore}{\kern0pt}sym\ w{\isacharparenright}{\kern0pt}{\isacharparenright}{\kern0pt}{\isachardoublequoteclose}\isanewline
\ \ \ \ \isakeywordONE{unfolding}\isamarkupfalse%
\ rlexp{\isacharunderscore}{\kern0pt}syms{\isacharunderscore}{\kern0pt}def\ \isakeywordONE{by}\isamarkupfalse%
\ blast\isanewline
\ \ \isakeywordONE{then}\isamarkupfalse%
\ \isakeywordTHREE{obtain}\isamarkupfalse%
\ f\ \isakeywordTWO{where}\ {\isacharasterisk}{\kern0pt}{\isacharcolon}{\kern0pt}\ {\isachardoublequoteopen}f\ {\isasymin}\ set\ {\isacharparenleft}{\kern0pt}map\ rlexp{\isacharunderscore}{\kern0pt}sym\ w{\isacharparenright}{\kern0pt}\ {\isasymand}\ x\ {\isasymin}\ vars\ f{\isachardoublequoteclose}\ \isakeywordONE{by}\isamarkupfalse%
\ blast\isanewline
\ \ \isakeywordONE{then}\isamarkupfalse%
\ \isakeywordTHREE{obtain}\isamarkupfalse%
\ s\ \isakeywordTWO{where}\ {\isacharasterisk}{\kern0pt}{\isacharasterisk}{\kern0pt}{\isacharcolon}{\kern0pt}\ {\isachardoublequoteopen}s\ {\isasymin}\ set\ w\ {\isasymand}\ rlexp{\isacharunderscore}{\kern0pt}sym\ s\ {\isacharequal}{\kern0pt}\ f{\isachardoublequoteclose}\ \isakeywordONE{by}\isamarkupfalse%
\ auto\isanewline
\ \ \isakeywordONE{with}\isamarkupfalse%
\ {\isacharasterisk}{\kern0pt}\ rlexp{\isacharunderscore}{\kern0pt}sym{\isacharunderscore}{\kern0pt}vars{\isacharunderscore}{\kern0pt}Tm\ \isakeywordTHREE{obtain}\isamarkupfalse%
\ A\ \isakeywordTWO{where}\ {\isacharasterisk}{\kern0pt}{\isacharasterisk}{\kern0pt}{\isacharasterisk}{\kern0pt}{\isacharcolon}{\kern0pt}\ {\isachardoublequoteopen}s\ {\isacharequal}{\kern0pt}\ Nt\ A{\isachardoublequoteclose}\ \isakeywordONE{by}\isamarkupfalse%
\ {\isacharparenleft}{\kern0pt}metis\ empty{\isacharunderscore}{\kern0pt}iff\ sym{\isachardot}{\kern0pt}exhaust{\isacharparenright}{\kern0pt}\isanewline
\ \ \isakeywordONE{with}\isamarkupfalse%
\ {\isacharasterisk}{\kern0pt}{\isacharasterisk}{\kern0pt}\ \isakeywordONE{have}\isamarkupfalse%
\ {\isacharasterisk}{\kern0pt}{\isacharasterisk}{\kern0pt}{\isacharasterisk}{\kern0pt}{\isacharasterisk}{\kern0pt}{\isacharcolon}{\kern0pt}\ {\isachardoublequoteopen}A\ {\isasymin}\ nts{\isacharunderscore}{\kern0pt}syms\ w{\isachardoublequoteclose}\ \isakeywordONE{unfolding}\isamarkupfalse%
\ nts{\isacharunderscore}{\kern0pt}syms{\isacharunderscore}{\kern0pt}def\ \isakeywordONE{by}\isamarkupfalse%
\ blast\isanewline
\ \ \isakeywordONE{with}\isamarkupfalse%
\ rlexp{\isacharunderscore}{\kern0pt}sym{\isacharunderscore}{\kern0pt}vars{\isacharunderscore}{\kern0pt}Nt\ \isakeywordONE{have}\isamarkupfalse%
\ {\isachardoublequoteopen}vars\ {\isacharparenleft}{\kern0pt}rlexp{\isacharunderscore}{\kern0pt}sym\ {\isacharparenleft}{\kern0pt}Nt\ A{\isacharparenright}{\kern0pt}{\isacharparenright}{\kern0pt}\ {\isacharequal}{\kern0pt}\ {\isacharbraceleft}{\kern0pt}{\isasymgamma}{\isacharprime}{\kern0pt}\ A{\isacharbraceright}{\kern0pt}{\isachardoublequoteclose}\ \isakeywordONE{by}\isamarkupfalse%
\ blast\isanewline
\ \ \isakeywordONE{with}\isamarkupfalse%
\ {\isacharasterisk}{\kern0pt}\ {\isacharasterisk}{\kern0pt}{\isacharasterisk}{\kern0pt}\ {\isacharasterisk}{\kern0pt}{\isacharasterisk}{\kern0pt}{\isacharasterisk}{\kern0pt}\ {\isacharasterisk}{\kern0pt}{\isacharasterisk}{\kern0pt}{\isacharasterisk}{\kern0pt}{\isacharasterisk}{\kern0pt}\ \isakeywordTHREE{show}\isamarkupfalse%
\ {\isachardoublequoteopen}x\ {\isasymin}\ {\isasymgamma}{\isacharprime}{\kern0pt}\ {\isacharbackquote}{\kern0pt}\ nts{\isacharunderscore}{\kern0pt}syms\ w{\isachardoublequoteclose}\ \isakeywordONE{by}\isamarkupfalse%
\ blast\isanewline
\isakeywordONE{qed}\isamarkupfalse%
%
\endisatagproof
{\isafoldproof}%
%
\isadelimproof
%
\endisadelimproof
%
\begin{isamarkuptext}%
Evaluating the regular language expression of a single production under a valuation
corresponds to instantiating the non-terminals in the production according to the valuation:%
\end{isamarkuptext}\isamarkuptrue%
\isakeywordONE{lemma}\isamarkupfalse%
\ rlexp{\isacharunderscore}{\kern0pt}sym{\isacharunderscore}{\kern0pt}inst{\isacharunderscore}{\kern0pt}Nt{\isacharcolon}{\kern0pt}\isanewline
\ \ \isakeywordTWO{assumes}\ {\isachardoublequoteopen}v\ {\isacharparenleft}{\kern0pt}{\isasymgamma}{\isacharprime}{\kern0pt}\ A{\isacharparenright}{\kern0pt}\ {\isacharequal}{\kern0pt}\ L\ A{\isachardoublequoteclose}\isanewline
\ \ \ \ \isakeywordTWO{shows}\ {\isachardoublequoteopen}eval\ {\isacharparenleft}{\kern0pt}rlexp{\isacharunderscore}{\kern0pt}sym\ {\isacharparenleft}{\kern0pt}Nt\ A{\isacharparenright}{\kern0pt}{\isacharparenright}{\kern0pt}\ v\ {\isacharequal}{\kern0pt}\ inst{\isacharunderscore}{\kern0pt}sym\ L\ {\isacharparenleft}{\kern0pt}Nt\ A{\isacharparenright}{\kern0pt}{\isachardoublequoteclose}\isanewline
%
\isadelimproof
\ \ %
\endisadelimproof
%
\isatagproof
\isakeywordONE{using}\isamarkupfalse%
\ assms\ \isakeywordONE{unfolding}\isamarkupfalse%
\ rlexp{\isacharunderscore}{\kern0pt}sym{\isacharunderscore}{\kern0pt}def\ inst{\isacharunderscore}{\kern0pt}sym{\isacharunderscore}{\kern0pt}def\ \isakeywordONE{by}\isamarkupfalse%
\ force%
\endisatagproof
{\isafoldproof}%
%
\isadelimproof
\isanewline
%
\endisadelimproof
\isanewline
\isakeywordONE{lemma}\isamarkupfalse%
\ rlexp{\isacharunderscore}{\kern0pt}sym{\isacharunderscore}{\kern0pt}inst{\isacharunderscore}{\kern0pt}Tm{\isacharcolon}{\kern0pt}\ {\isachardoublequoteopen}eval\ {\isacharparenleft}{\kern0pt}rlexp{\isacharunderscore}{\kern0pt}sym\ {\isacharparenleft}{\kern0pt}Tm\ a{\isacharparenright}{\kern0pt}{\isacharparenright}{\kern0pt}\ v\ {\isacharequal}{\kern0pt}\ inst{\isacharunderscore}{\kern0pt}sym\ L\ {\isacharparenleft}{\kern0pt}Tm\ a{\isacharparenright}{\kern0pt}{\isachardoublequoteclose}\isanewline
%
\isadelimproof
\ \ %
\endisadelimproof
%
\isatagproof
\isakeywordONE{unfolding}\isamarkupfalse%
\ rlexp{\isacharunderscore}{\kern0pt}sym{\isacharunderscore}{\kern0pt}def\ inst{\isacharunderscore}{\kern0pt}sym{\isacharunderscore}{\kern0pt}def\ \isakeywordONE{by}\isamarkupfalse%
\ force%
\endisatagproof
{\isafoldproof}%
%
\isadelimproof
\isanewline
%
\endisadelimproof
\isanewline
\isakeywordONE{lemma}\isamarkupfalse%
\ rlexp{\isacharunderscore}{\kern0pt}concats{\isacharunderscore}{\kern0pt}concats{\isacharcolon}{\kern0pt}\isanewline
\ \ \isakeywordTWO{assumes}\ {\isachardoublequoteopen}length\ fs\ {\isacharequal}{\kern0pt}\ length\ Ls{\isachardoublequoteclose}\isanewline
\ \ \ \ \ \ \isakeywordTWO{and}\ {\isachardoublequoteopen}{\isasymforall}i\ {\isacharless}{\kern0pt}\ length\ fs{\isachardot}{\kern0pt}\ eval\ {\isacharparenleft}{\kern0pt}fs\ {\isacharbang}{\kern0pt}\ i{\isacharparenright}{\kern0pt}\ v\ {\isacharequal}{\kern0pt}\ Ls\ {\isacharbang}{\kern0pt}\ i{\isachardoublequoteclose}\isanewline
\ \ \ \ \isakeywordTWO{shows}\ {\isachardoublequoteopen}eval\ {\isacharparenleft}{\kern0pt}rlexp{\isacharunderscore}{\kern0pt}concats\ fs{\isacharparenright}{\kern0pt}\ v\ {\isacharequal}{\kern0pt}\ concats\ Ls{\isachardoublequoteclose}\isanewline
%
\isadelimproof
%
\endisadelimproof
%
\isatagproof
\isakeywordONE{using}\isamarkupfalse%
\ assms\ \isakeywordONE{proof}\isamarkupfalse%
\ {\isacharparenleft}{\kern0pt}induction\ fs\ arbitrary{\isacharcolon}{\kern0pt}\ Ls{\isacharparenright}{\kern0pt}\isanewline
\ \ \isakeywordTHREE{case}\isamarkupfalse%
\ Nil\isanewline
\ \ \isakeywordONE{then}\isamarkupfalse%
\ \isakeywordTHREE{show}\isamarkupfalse%
\ {\isacharquery}{\kern0pt}case\ \isakeywordONE{unfolding}\isamarkupfalse%
\ rlexp{\isacharunderscore}{\kern0pt}concats{\isacharunderscore}{\kern0pt}def\ concats{\isacharunderscore}{\kern0pt}def\ \isakeywordONE{by}\isamarkupfalse%
\ simp\isanewline
\isakeywordONE{next}\isamarkupfalse%
\isanewline
\ \ \isakeywordTHREE{case}\isamarkupfalse%
\ {\isacharparenleft}{\kern0pt}Cons\ f{\isadigit{1}}\ fs{\isacharparenright}{\kern0pt}\isanewline
\ \ \isakeywordONE{then}\isamarkupfalse%
\ \isakeywordTHREE{obtain}\isamarkupfalse%
\ L{\isadigit{1}}\ Lr\ \isakeywordTWO{where}\ {\isacharasterisk}{\kern0pt}{\isacharcolon}{\kern0pt}\ {\isachardoublequoteopen}Ls\ {\isacharequal}{\kern0pt}\ L{\isadigit{1}}{\isacharhash}{\kern0pt}Lr{\isachardoublequoteclose}\ \isakeywordONE{by}\isamarkupfalse%
\ {\isacharparenleft}{\kern0pt}metis\ length{\isacharunderscore}{\kern0pt}Suc{\isacharunderscore}{\kern0pt}conv{\isacharparenright}{\kern0pt}\isanewline
\ \ \isakeywordONE{with}\isamarkupfalse%
\ Cons\ \isakeywordONE{have}\isamarkupfalse%
\ {\isachardoublequoteopen}eval\ {\isacharparenleft}{\kern0pt}rlexp{\isacharunderscore}{\kern0pt}concats\ fs{\isacharparenright}{\kern0pt}\ v\ {\isacharequal}{\kern0pt}\ concats\ Lr{\isachardoublequoteclose}\ \isakeywordONE{by}\isamarkupfalse%
\ fastforce\isanewline
\ \ \isakeywordONE{moreover}\isamarkupfalse%
\ \isakeywordONE{from}\isamarkupfalse%
\ Cons{\isachardot}{\kern0pt}prems\ {\isacharasterisk}{\kern0pt}\ \isakeywordONE{have}\isamarkupfalse%
\ {\isachardoublequoteopen}eval\ f{\isadigit{1}}\ v\ {\isacharequal}{\kern0pt}\ L{\isadigit{1}}{\isachardoublequoteclose}\ \isakeywordONE{by}\isamarkupfalse%
\ force\isanewline
\ \ \isakeywordONE{ultimately}\isamarkupfalse%
\ \isakeywordTHREE{show}\isamarkupfalse%
\ {\isacharquery}{\kern0pt}case\ \isakeywordONE{unfolding}\isamarkupfalse%
\ rlexp{\isacharunderscore}{\kern0pt}concats{\isacharunderscore}{\kern0pt}def\ concats{\isacharunderscore}{\kern0pt}def\ \isakeywordONE{by}\isamarkupfalse%
\ {\isacharparenleft}{\kern0pt}simp\ add{\isacharcolon}{\kern0pt}\ {\isachardoublequoteopen}{\isacharasterisk}{\kern0pt}{\isachardoublequoteclose}{\isacharparenright}{\kern0pt}\isanewline
\isakeywordONE{qed}\isamarkupfalse%
%
\endisatagproof
{\isafoldproof}%
%
\isadelimproof
\isanewline
%
\endisadelimproof
\isanewline
\isakeywordONE{lemma}\isamarkupfalse%
\ rlexp{\isacharunderscore}{\kern0pt}syms{\isacharunderscore}{\kern0pt}insts{\isacharcolon}{\kern0pt}\isanewline
\ \ \isakeywordTWO{assumes}\ {\isachardoublequoteopen}{\isasymforall}A\ {\isasymin}\ nts{\isacharunderscore}{\kern0pt}syms\ w{\isachardot}{\kern0pt}\ v\ {\isacharparenleft}{\kern0pt}{\isasymgamma}{\isacharprime}{\kern0pt}\ A{\isacharparenright}{\kern0pt}\ {\isacharequal}{\kern0pt}\ L\ A{\isachardoublequoteclose}\isanewline
\ \ \ \ \isakeywordTWO{shows}\ {\isachardoublequoteopen}eval\ {\isacharparenleft}{\kern0pt}rlexp{\isacharunderscore}{\kern0pt}syms\ w{\isacharparenright}{\kern0pt}\ v\ {\isacharequal}{\kern0pt}\ inst{\isacharunderscore}{\kern0pt}syms\ L\ w{\isachardoublequoteclose}\isanewline
%
\isadelimproof
%
\endisadelimproof
%
\isatagproof
\isakeywordONE{proof}\isamarkupfalse%
\ {\isacharminus}{\kern0pt}\isanewline
\ \ \isakeywordONE{have}\isamarkupfalse%
\ {\isachardoublequoteopen}{\isasymforall}i\ {\isacharless}{\kern0pt}\ length\ w{\isachardot}{\kern0pt}\ eval\ {\isacharparenleft}{\kern0pt}rlexp{\isacharunderscore}{\kern0pt}sym\ {\isacharparenleft}{\kern0pt}w{\isacharbang}{\kern0pt}i{\isacharparenright}{\kern0pt}{\isacharparenright}{\kern0pt}\ v\ {\isacharequal}{\kern0pt}\ inst{\isacharunderscore}{\kern0pt}sym\ L\ {\isacharparenleft}{\kern0pt}w{\isacharbang}{\kern0pt}i{\isacharparenright}{\kern0pt}{\isachardoublequoteclose}\isanewline
\ \ \isakeywordONE{proof}\isamarkupfalse%
\ {\isacharparenleft}{\kern0pt}rule\ allI{\isacharcomma}{\kern0pt}\ rule\ impI{\isacharparenright}{\kern0pt}\isanewline
\ \ \ \ \isakeywordTHREE{fix}\isamarkupfalse%
\ i\isanewline
\ \ \ \ \isakeywordTHREE{assume}\isamarkupfalse%
\ {\isachardoublequoteopen}i\ {\isacharless}{\kern0pt}\ length\ w{\isachardoublequoteclose}\isanewline
\ \ \ \ \isakeywordONE{then}\isamarkupfalse%
\ \isakeywordTHREE{show}\isamarkupfalse%
\ {\isachardoublequoteopen}eval\ {\isacharparenleft}{\kern0pt}rlexp{\isacharunderscore}{\kern0pt}sym\ {\isacharparenleft}{\kern0pt}w\ {\isacharbang}{\kern0pt}\ i{\isacharparenright}{\kern0pt}{\isacharparenright}{\kern0pt}\ v\ {\isacharequal}{\kern0pt}\ inst{\isacharunderscore}{\kern0pt}sym\ L\ {\isacharparenleft}{\kern0pt}w\ {\isacharbang}{\kern0pt}\ i{\isacharparenright}{\kern0pt}{\isachardoublequoteclose}\isanewline
\ \ \ \ \ \ \isakeywordONE{using}\isamarkupfalse%
\ assms\ \isakeywordONE{proof}\isamarkupfalse%
\ {\isacharparenleft}{\kern0pt}induction\ {\isachardoublequoteopen}w{\isacharbang}{\kern0pt}i{\isachardoublequoteclose}{\isacharparenright}{\kern0pt}\isanewline
\ \ \ \ \ \ \isakeywordTHREE{case}\isamarkupfalse%
\ {\isacharparenleft}{\kern0pt}Nt\ A{\isacharparenright}{\kern0pt}\isanewline
\ \ \ \ \ \ \isakeywordONE{then}\isamarkupfalse%
\ \isakeywordONE{have}\isamarkupfalse%
\ {\isachardoublequoteopen}v\ {\isacharparenleft}{\kern0pt}{\isasymgamma}{\isacharprime}{\kern0pt}\ A{\isacharparenright}{\kern0pt}\ {\isacharequal}{\kern0pt}\ L\ A{\isachardoublequoteclose}\ \isakeywordONE{unfolding}\isamarkupfalse%
\ nts{\isacharunderscore}{\kern0pt}syms{\isacharunderscore}{\kern0pt}def\ \isakeywordONE{by}\isamarkupfalse%
\ force\isanewline
\ \ \ \ \ \ \isakeywordONE{with}\isamarkupfalse%
\ rlexp{\isacharunderscore}{\kern0pt}sym{\isacharunderscore}{\kern0pt}inst{\isacharunderscore}{\kern0pt}Nt\ Nt\ \isakeywordTHREE{show}\isamarkupfalse%
\ {\isacharquery}{\kern0pt}case\ \isakeywordONE{by}\isamarkupfalse%
\ metis\isanewline
\ \ \ \ \isakeywordONE{next}\isamarkupfalse%
\isanewline
\ \ \ \ \ \ \isakeywordTHREE{case}\isamarkupfalse%
\ {\isacharparenleft}{\kern0pt}Tm\ x{\isacharparenright}{\kern0pt}\isanewline
\ \ \ \ \ \ \isakeywordONE{with}\isamarkupfalse%
\ rlexp{\isacharunderscore}{\kern0pt}sym{\isacharunderscore}{\kern0pt}inst{\isacharunderscore}{\kern0pt}Tm\ \isakeywordTHREE{show}\isamarkupfalse%
\ {\isacharquery}{\kern0pt}case\ \isakeywordONE{by}\isamarkupfalse%
\ metis\isanewline
\ \ \ \ \isakeywordONE{qed}\isamarkupfalse%
\isanewline
\ \ \isakeywordONE{qed}\isamarkupfalse%
\isanewline
\ \ \isakeywordONE{then}\isamarkupfalse%
\ \isakeywordTHREE{show}\isamarkupfalse%
\ {\isacharquery}{\kern0pt}thesis\ \isakeywordONE{unfolding}\isamarkupfalse%
\ rlexp{\isacharunderscore}{\kern0pt}syms{\isacharunderscore}{\kern0pt}def\ inst{\isacharunderscore}{\kern0pt}syms{\isacharunderscore}{\kern0pt}def\ \isakeywordONE{using}\isamarkupfalse%
\ rlexp{\isacharunderscore}{\kern0pt}concats{\isacharunderscore}{\kern0pt}concats\isanewline
\ \ \ \ \isakeywordONE{by}\isamarkupfalse%
\ {\isacharparenleft}{\kern0pt}metis\ {\isacharparenleft}{\kern0pt}mono{\isacharunderscore}{\kern0pt}tags{\isacharcomma}{\kern0pt}\ lifting{\isacharparenright}{\kern0pt}\ length{\isacharunderscore}{\kern0pt}map\ nth{\isacharunderscore}{\kern0pt}map{\isacharparenright}{\kern0pt}\isanewline
\isakeywordONE{qed}\isamarkupfalse%
%
\endisatagproof
{\isafoldproof}%
%
\isadelimproof
%
\endisadelimproof
%
\begin{isamarkuptext}%
Each non-terminal of the CFG induces some \isa{\isaconst{reg{\isacharunderscore}{\kern0pt}eval}} equation. We do not directly construct
the equation but only prove its existence:%
\end{isamarkuptext}\isamarkuptrue%
\isakeywordONE{lemma}\isamarkupfalse%
\ subst{\isacharunderscore}{\kern0pt}lang{\isacharunderscore}{\kern0pt}rlexp{\isacharcolon}{\kern0pt}\isanewline
\ \ {\isachardoublequoteopen}{\isasymexists}eq{\isachardot}{\kern0pt}\ reg{\isacharunderscore}{\kern0pt}eval\ eq\ {\isasymand}\ vars\ eq\ {\isasymsubseteq}\ {\isasymgamma}{\isacharprime}{\kern0pt}\ {\isacharbackquote}{\kern0pt}\ Nts\ P\isanewline
\ \ \ \ \ \ \ \ \ {\isasymand}\ {\isacharparenleft}{\kern0pt}{\isasymforall}v\ L{\isachardot}{\kern0pt}\ {\isacharparenleft}{\kern0pt}{\isasymforall}A\ {\isasymin}\ Nts\ P{\isachardot}{\kern0pt}\ v\ {\isacharparenleft}{\kern0pt}{\isasymgamma}{\isacharprime}{\kern0pt}\ A{\isacharparenright}{\kern0pt}\ {\isacharequal}{\kern0pt}\ L\ A{\isacharparenright}{\kern0pt}\ {\isasymlongrightarrow}\ eval\ eq\ v\ {\isacharequal}{\kern0pt}\ subst{\isacharunderscore}{\kern0pt}lang\ P\ L\ A{\isacharparenright}{\kern0pt}{\isachardoublequoteclose}\isanewline
%
\isadelimproof
%
\endisadelimproof
%
\isatagproof
\isakeywordONE{proof}\isamarkupfalse%
\ {\isacharminus}{\kern0pt}\isanewline
\ \ \isakeywordONE{let}\isamarkupfalse%
\ {\isacharquery}{\kern0pt}Insts\ {\isacharequal}{\kern0pt}\ {\isachardoublequoteopen}rlexp{\isacharunderscore}{\kern0pt}syms\ {\isacharbackquote}{\kern0pt}\ {\isacharparenleft}{\kern0pt}Rhss\ P\ A{\isacharparenright}{\kern0pt}{\isachardoublequoteclose}\isanewline
\ \ \isakeywordONE{from}\isamarkupfalse%
\ finite{\isacharunderscore}{\kern0pt}Rhss{\isacharbrackleft}{\kern0pt}OF\ finite{\isacharunderscore}{\kern0pt}P{\isacharbrackright}{\kern0pt}\ \isakeywordONE{have}\isamarkupfalse%
\ {\isachardoublequoteopen}finite\ {\isacharquery}{\kern0pt}Insts{\isachardoublequoteclose}\ \isakeywordONE{by}\isamarkupfalse%
\ simp\isanewline
\ \ \isakeywordONE{moreover}\isamarkupfalse%
\ \isakeywordONE{from}\isamarkupfalse%
\ rlexp{\isacharunderscore}{\kern0pt}syms{\isacharunderscore}{\kern0pt}reg\ \isakeywordONE{have}\isamarkupfalse%
\ {\isachardoublequoteopen}{\isasymforall}f\ {\isasymin}\ {\isacharquery}{\kern0pt}Insts{\isachardot}{\kern0pt}\ reg{\isacharunderscore}{\kern0pt}eval\ f{\isachardoublequoteclose}\ \isakeywordONE{by}\isamarkupfalse%
\ blast\isanewline
\ \ \isakeywordONE{ultimately}\isamarkupfalse%
\ \isakeywordTHREE{obtain}\isamarkupfalse%
\ eq\ \isakeywordTWO{where}\ {\isacharasterisk}{\kern0pt}{\isacharcolon}{\kern0pt}\ {\isachardoublequoteopen}reg{\isacharunderscore}{\kern0pt}eval\ eq\ {\isasymand}\ {\isasymUnion}{\isacharparenleft}{\kern0pt}vars\ {\isacharbackquote}{\kern0pt}\ {\isacharquery}{\kern0pt}Insts{\isacharparenright}{\kern0pt}\ {\isacharequal}{\kern0pt}\ vars\ eq\isanewline
\ \ \ \ \ \ \ \ \ \ \ \ \ \ \ \ \ \ \ \ \ \ \ \ \ \ \ \ \ \ \ \ \ \ {\isasymand}\ {\isacharparenleft}{\kern0pt}{\isasymforall}v{\isachardot}{\kern0pt}\ {\isacharparenleft}{\kern0pt}{\isasymUnion}f\ {\isasymin}\ {\isacharquery}{\kern0pt}Insts{\isachardot}{\kern0pt}\ eval\ f\ v{\isacharparenright}{\kern0pt}\ {\isacharequal}{\kern0pt}\ eval\ eq\ v{\isacharparenright}{\kern0pt}{\isachardoublequoteclose}\isanewline
\ \ \ \ \isakeywordONE{using}\isamarkupfalse%
\ finite{\isacharunderscore}{\kern0pt}Union{\isacharunderscore}{\kern0pt}regular\ \isakeywordONE{by}\isamarkupfalse%
\ metis\isanewline
\ \ \isakeywordONE{moreover}\isamarkupfalse%
\ \isakeywordONE{have}\isamarkupfalse%
\ {\isachardoublequoteopen}vars\ eq\ {\isasymsubseteq}\ {\isasymgamma}{\isacharprime}{\kern0pt}\ {\isacharbackquote}{\kern0pt}\ Nts\ P{\isachardoublequoteclose}\isanewline
\ \ \isakeywordONE{proof}\isamarkupfalse%
\isanewline
\ \ \ \ \isakeywordTHREE{fix}\isamarkupfalse%
\ x\isanewline
\ \ \ \ \isakeywordTHREE{assume}\isamarkupfalse%
\ {\isachardoublequoteopen}x\ {\isasymin}\ vars\ eq{\isachardoublequoteclose}\isanewline
\ \ \ \ \isakeywordONE{with}\isamarkupfalse%
\ {\isacharasterisk}{\kern0pt}\ \isakeywordTHREE{obtain}\isamarkupfalse%
\ f\ \isakeywordTWO{where}\ {\isacharasterisk}{\kern0pt}{\isacharasterisk}{\kern0pt}{\isacharcolon}{\kern0pt}\ {\isachardoublequoteopen}f\ {\isasymin}\ {\isacharquery}{\kern0pt}Insts\ {\isasymand}\ x\ {\isasymin}\ vars\ f{\isachardoublequoteclose}\ \isakeywordONE{by}\isamarkupfalse%
\ blast\isanewline
\ \ \ \ \isakeywordONE{then}\isamarkupfalse%
\ \isakeywordTHREE{obtain}\isamarkupfalse%
\ w\ \isakeywordTWO{where}\ {\isacharasterisk}{\kern0pt}{\isacharasterisk}{\kern0pt}{\isacharasterisk}{\kern0pt}{\isacharcolon}{\kern0pt}\ {\isachardoublequoteopen}w\ {\isasymin}\ Rhss\ P\ A\ {\isasymand}\ f\ {\isacharequal}{\kern0pt}\ rlexp{\isacharunderscore}{\kern0pt}syms\ w{\isachardoublequoteclose}\ \isakeywordONE{by}\isamarkupfalse%
\ blast\isanewline
\ \ \ \ \isakeywordONE{with}\isamarkupfalse%
\ {\isacharasterisk}{\kern0pt}{\isacharasterisk}{\kern0pt}\ insts{\isacharprime}{\kern0pt}{\isacharunderscore}{\kern0pt}vars\ \isakeywordONE{have}\isamarkupfalse%
\ {\isachardoublequoteopen}x\ {\isasymin}\ {\isasymgamma}{\isacharprime}{\kern0pt}\ {\isacharbackquote}{\kern0pt}\ nts{\isacharunderscore}{\kern0pt}syms\ w{\isachardoublequoteclose}\ \isakeywordONE{by}\isamarkupfalse%
\ auto\isanewline
\ \ \ \ \isakeywordONE{with}\isamarkupfalse%
\ {\isacharasterisk}{\kern0pt}{\isacharasterisk}{\kern0pt}{\isacharasterisk}{\kern0pt}\ \isakeywordTHREE{show}\isamarkupfalse%
\ {\isachardoublequoteopen}x\ {\isasymin}\ {\isasymgamma}{\isacharprime}{\kern0pt}\ {\isacharbackquote}{\kern0pt}\ Nts\ P{\isachardoublequoteclose}\ \isakeywordONE{unfolding}\isamarkupfalse%
\ Nts{\isacharunderscore}{\kern0pt}def\ Rhss{\isacharunderscore}{\kern0pt}def\ \isakeywordONE{by}\isamarkupfalse%
\ blast\isanewline
\ \ \isakeywordONE{qed}\isamarkupfalse%
\isanewline
\ \ \isakeywordONE{moreover}\isamarkupfalse%
\ \isakeywordONE{have}\isamarkupfalse%
\ {\isachardoublequoteopen}{\isasymforall}v\ L{\isachardot}{\kern0pt}\ {\isacharparenleft}{\kern0pt}{\isasymforall}A\ {\isasymin}\ Nts\ P{\isachardot}{\kern0pt}\ v\ {\isacharparenleft}{\kern0pt}{\isasymgamma}{\isacharprime}{\kern0pt}\ A{\isacharparenright}{\kern0pt}\ {\isacharequal}{\kern0pt}\ L\ A{\isacharparenright}{\kern0pt}\ {\isasymlongrightarrow}\ eval\ eq\ v\ {\isacharequal}{\kern0pt}\ subst{\isacharunderscore}{\kern0pt}lang\ P\ L\ A{\isachardoublequoteclose}\isanewline
\ \ \isakeywordONE{proof}\isamarkupfalse%
\ {\isacharparenleft}{\kern0pt}rule\ allI\ {\isacharbar}{\kern0pt}\ rule\ impI{\isacharparenright}{\kern0pt}{\isacharplus}{\kern0pt}\isanewline
\ \ \ \ \isakeywordTHREE{fix}\isamarkupfalse%
\ v\ {\isacharcolon}{\kern0pt}{\isacharcolon}{\kern0pt}\ {\isachardoublequoteopen}nat\ {\isasymRightarrow}\ {\isacharprime}{\kern0pt}a\ lang{\isachardoublequoteclose}\ \isakeywordTWO{and}\ L\ {\isacharcolon}{\kern0pt}{\isacharcolon}{\kern0pt}\ {\isachardoublequoteopen}{\isacharprime}{\kern0pt}n\ {\isasymRightarrow}\ {\isacharprime}{\kern0pt}a\ lang{\isachardoublequoteclose}\isanewline
\ \ \ \ \isakeywordTHREE{assume}\isamarkupfalse%
\ state{\isacharunderscore}{\kern0pt}L{\isacharcolon}{\kern0pt}\ {\isachardoublequoteopen}{\isasymforall}A\ {\isasymin}\ Nts\ P{\isachardot}{\kern0pt}\ v\ {\isacharparenleft}{\kern0pt}{\isasymgamma}{\isacharprime}{\kern0pt}\ A{\isacharparenright}{\kern0pt}\ {\isacharequal}{\kern0pt}\ L\ A{\isachardoublequoteclose}\isanewline
\ \ \ \ \isakeywordONE{have}\isamarkupfalse%
\ {\isachardoublequoteopen}{\isasymforall}w\ {\isasymin}\ Rhss\ P\ A{\isachardot}{\kern0pt}\ eval\ {\isacharparenleft}{\kern0pt}rlexp{\isacharunderscore}{\kern0pt}syms\ w{\isacharparenright}{\kern0pt}\ v\ {\isacharequal}{\kern0pt}\ inst{\isacharunderscore}{\kern0pt}syms\ L\ w{\isachardoublequoteclose}\isanewline
\ \ \ \ \isakeywordONE{proof}\isamarkupfalse%
\isanewline
\ \ \ \ \ \ \isakeywordTHREE{fix}\isamarkupfalse%
\ w\isanewline
\ \ \ \ \ \ \isakeywordTHREE{assume}\isamarkupfalse%
\ {\isachardoublequoteopen}w\ {\isasymin}\ Rhss\ P\ A{\isachardoublequoteclose}\isanewline
\ \ \ \ \ \ \isakeywordONE{with}\isamarkupfalse%
\ state{\isacharunderscore}{\kern0pt}L\ Nts{\isacharunderscore}{\kern0pt}nts{\isacharunderscore}{\kern0pt}syms\ \isakeywordONE{have}\isamarkupfalse%
\ {\isachardoublequoteopen}{\isasymforall}A\ {\isasymin}\ nts{\isacharunderscore}{\kern0pt}syms\ w{\isachardot}{\kern0pt}\ v\ {\isacharparenleft}{\kern0pt}{\isasymgamma}{\isacharprime}{\kern0pt}\ A{\isacharparenright}{\kern0pt}\ {\isacharequal}{\kern0pt}\ L\ A{\isachardoublequoteclose}\ \isakeywordONE{by}\isamarkupfalse%
\ fast\isanewline
\ \ \ \ \ \ \isakeywordONE{from}\isamarkupfalse%
\ rlexp{\isacharunderscore}{\kern0pt}syms{\isacharunderscore}{\kern0pt}insts{\isacharbrackleft}{\kern0pt}OF\ this{\isacharbrackright}{\kern0pt}\ \isakeywordTHREE{show}\isamarkupfalse%
\ {\isachardoublequoteopen}eval\ {\isacharparenleft}{\kern0pt}rlexp{\isacharunderscore}{\kern0pt}syms\ w{\isacharparenright}{\kern0pt}\ v\ {\isacharequal}{\kern0pt}\ inst{\isacharunderscore}{\kern0pt}syms\ L\ w{\isachardoublequoteclose}\ \isakeywordONE{by}\isamarkupfalse%
\ blast\isanewline
\ \ \ \ \isakeywordONE{qed}\isamarkupfalse%
\isanewline
\ \ \ \ \isakeywordONE{then}\isamarkupfalse%
\ \isakeywordONE{have}\isamarkupfalse%
\ {\isachardoublequoteopen}subst{\isacharunderscore}{\kern0pt}lang\ P\ L\ A\ {\isacharequal}{\kern0pt}\ {\isacharparenleft}{\kern0pt}{\isasymUnion}f\ {\isasymin}\ {\isacharquery}{\kern0pt}Insts{\isachardot}{\kern0pt}\ eval\ f\ v{\isacharparenright}{\kern0pt}{\isachardoublequoteclose}\ \isakeywordONE{unfolding}\isamarkupfalse%
\ subst{\isacharunderscore}{\kern0pt}lang{\isacharunderscore}{\kern0pt}def\ \isakeywordONE{by}\isamarkupfalse%
\ auto\isanewline
\ \ \ \ \isakeywordONE{with}\isamarkupfalse%
\ {\isacharasterisk}{\kern0pt}\ \isakeywordTHREE{show}\isamarkupfalse%
\ {\isachardoublequoteopen}eval\ eq\ v\ {\isacharequal}{\kern0pt}\ subst{\isacharunderscore}{\kern0pt}lang\ P\ L\ A{\isachardoublequoteclose}\ \isakeywordONE{by}\isamarkupfalse%
\ auto\isanewline
\ \ \isakeywordONE{qed}\isamarkupfalse%
\isanewline
\ \ \isakeywordONE{ultimately}\isamarkupfalse%
\ \isakeywordTHREE{show}\isamarkupfalse%
\ {\isacharquery}{\kern0pt}thesis\ \isakeywordONE{by}\isamarkupfalse%
\ auto\isanewline
\isakeywordONE{qed}\isamarkupfalse%
%
\endisatagproof
{\isafoldproof}%
%
\isadelimproof
%
\endisadelimproof
%
\begin{isamarkuptext}%
The whole CFG induces a system of equations. We first define which conditions this system
should fulfill and show its existence in the second step:%
\end{isamarkuptext}\isamarkuptrue%
\isakeywordONE{abbreviation}\isamarkupfalse%
\ {\isachardoublequoteopen}CFG{\isacharunderscore}{\kern0pt}sys\ sys\ {\isasymequiv}\isanewline
\ \ length\ sys\ {\isacharequal}{\kern0pt}\ card\ {\isacharparenleft}{\kern0pt}Nts\ P{\isacharparenright}{\kern0pt}\ {\isasymand}\isanewline
\ \ \ \ {\isacharparenleft}{\kern0pt}{\isasymforall}i\ {\isacharless}{\kern0pt}\ card\ {\isacharparenleft}{\kern0pt}Nts\ P{\isacharparenright}{\kern0pt}{\isachardot}{\kern0pt}\ reg{\isacharunderscore}{\kern0pt}eval\ {\isacharparenleft}{\kern0pt}sys\ {\isacharbang}{\kern0pt}\ i{\isacharparenright}{\kern0pt}\ {\isasymand}\ {\isacharparenleft}{\kern0pt}{\isasymforall}x\ {\isasymin}\ vars\ {\isacharparenleft}{\kern0pt}sys\ {\isacharbang}{\kern0pt}\ i{\isacharparenright}{\kern0pt}{\isachardot}{\kern0pt}\ x\ {\isacharless}{\kern0pt}\ card\ {\isacharparenleft}{\kern0pt}Nts\ P{\isacharparenright}{\kern0pt}{\isacharparenright}{\kern0pt}\isanewline
\ \ \ \ \ \ \ \ \ \ \ \ \ \ \ \ \ \ \ \ \ \ \ \ {\isasymand}\ {\isacharparenleft}{\kern0pt}{\isasymforall}s\ L{\isachardot}{\kern0pt}\ {\isacharparenleft}{\kern0pt}{\isasymforall}A\ {\isasymin}\ Nts\ P{\isachardot}{\kern0pt}\ s\ {\isacharparenleft}{\kern0pt}{\isasymgamma}{\isacharprime}{\kern0pt}\ A{\isacharparenright}{\kern0pt}\ {\isacharequal}{\kern0pt}\ L\ A{\isacharparenright}{\kern0pt}\isanewline
\ \ \ \ \ \ \ \ \ \ \ \ \ \ \ \ \ \ \ \ \ \ \ \ \ \ \ \ {\isasymlongrightarrow}\ eval\ {\isacharparenleft}{\kern0pt}sys\ {\isacharbang}{\kern0pt}\ i{\isacharparenright}{\kern0pt}\ s\ {\isacharequal}{\kern0pt}\ subst{\isacharunderscore}{\kern0pt}lang\ P\ L\ {\isacharparenleft}{\kern0pt}{\isasymgamma}\ i{\isacharparenright}{\kern0pt}{\isacharparenright}{\kern0pt}{\isacharparenright}{\kern0pt}{\isachardoublequoteclose}\isanewline
\isanewline
\isakeywordONE{lemma}\isamarkupfalse%
\ CFG{\isacharunderscore}{\kern0pt}as{\isacharunderscore}{\kern0pt}eq{\isacharunderscore}{\kern0pt}sys{\isacharcolon}{\kern0pt}\ {\isachardoublequoteopen}{\isasymexists}sys{\isachardot}{\kern0pt}\ CFG{\isacharunderscore}{\kern0pt}sys\ sys{\isachardoublequoteclose}\isanewline
%
\isadelimproof
%
\endisadelimproof
%
\isatagproof
\isakeywordONE{proof}\isamarkupfalse%
\ {\isacharminus}{\kern0pt}\isanewline
\ \ \isakeywordONE{from}\isamarkupfalse%
\ bij{\isacharunderscore}{\kern0pt}{\isasymgamma}{\isacharunderscore}{\kern0pt}{\isasymgamma}{\isacharprime}{\kern0pt}\ \isakeywordONE{have}\isamarkupfalse%
\ {\isacharasterisk}{\kern0pt}{\isacharcolon}{\kern0pt}\ {\isachardoublequoteopen}{\isasymAnd}eq{\isachardot}{\kern0pt}\ vars\ eq\ {\isasymsubseteq}\ {\isasymgamma}{\isacharprime}{\kern0pt}\ {\isacharbackquote}{\kern0pt}\ Nts\ P\ {\isasymLongrightarrow}\ {\isasymforall}x\ {\isasymin}\ vars\ eq{\isachardot}{\kern0pt}\ x\ {\isacharless}{\kern0pt}\ card\ {\isacharparenleft}{\kern0pt}Nts\ P{\isacharparenright}{\kern0pt}{\isachardoublequoteclose}\isanewline
\ \ \ \ \isakeywordONE{unfolding}\isamarkupfalse%
\ bij{\isacharunderscore}{\kern0pt}Nt{\isacharunderscore}{\kern0pt}Var{\isacharunderscore}{\kern0pt}def\ bij{\isacharunderscore}{\kern0pt}betw{\isacharunderscore}{\kern0pt}def\ \isakeywordONE{by}\isamarkupfalse%
\ auto\isanewline
\ \ \isakeywordONE{from}\isamarkupfalse%
\ subst{\isacharunderscore}{\kern0pt}lang{\isacharunderscore}{\kern0pt}rlexp\ \isakeywordONE{have}\isamarkupfalse%
\ {\isachardoublequoteopen}{\isasymforall}A{\isachardot}{\kern0pt}\ {\isasymexists}eq{\isachardot}{\kern0pt}\ reg{\isacharunderscore}{\kern0pt}eval\ eq\ {\isasymand}\ vars\ eq\ {\isasymsubseteq}\ {\isasymgamma}{\isacharprime}{\kern0pt}\ {\isacharbackquote}{\kern0pt}\ Nts\ P\ {\isasymand}\isanewline
\ \ \ \ \ \ \ \ \ \ \ \ \ \ \ \ \ \ \ \ \ \ \ \ \ \ \ \ \ {\isacharparenleft}{\kern0pt}{\isasymforall}s\ L{\isachardot}{\kern0pt}\ {\isacharparenleft}{\kern0pt}{\isasymforall}A\ {\isasymin}\ Nts\ P{\isachardot}{\kern0pt}\ s\ {\isacharparenleft}{\kern0pt}{\isasymgamma}{\isacharprime}{\kern0pt}\ A{\isacharparenright}{\kern0pt}\ {\isacharequal}{\kern0pt}\ L\ A{\isacharparenright}{\kern0pt}\ {\isasymlongrightarrow}\ eval\ eq\ s\ {\isacharequal}{\kern0pt}\ subst{\isacharunderscore}{\kern0pt}lang\ P\ L\ A{\isacharparenright}{\kern0pt}{\isachardoublequoteclose}\isanewline
\ \ \ \ \isakeywordONE{by}\isamarkupfalse%
\ blast\isanewline
\ \ \isakeywordONE{with}\isamarkupfalse%
\ bij{\isacharunderscore}{\kern0pt}{\isasymgamma}{\isacharunderscore}{\kern0pt}{\isasymgamma}{\isacharprime}{\kern0pt}\ {\isacharasterisk}{\kern0pt}\ \isakeywordONE{have}\isamarkupfalse%
\ {\isachardoublequoteopen}{\isasymforall}i\ {\isacharless}{\kern0pt}\ card\ {\isacharparenleft}{\kern0pt}Nts\ P{\isacharparenright}{\kern0pt}{\isachardot}{\kern0pt}\ {\isasymexists}eq{\isachardot}{\kern0pt}\ reg{\isacharunderscore}{\kern0pt}eval\ eq\ {\isasymand}\ {\isacharparenleft}{\kern0pt}{\isasymforall}x\ {\isasymin}\ vars\ eq{\isachardot}{\kern0pt}\ x\ {\isacharless}{\kern0pt}\ card\ {\isacharparenleft}{\kern0pt}Nts\ P{\isacharparenright}{\kern0pt}{\isacharparenright}{\kern0pt}\isanewline
\ \ \ \ \ \ \ \ \ \ \ \ \ \ \ \ \ \ \ \ {\isasymand}\ {\isacharparenleft}{\kern0pt}{\isasymforall}s\ L{\isachardot}{\kern0pt}\ {\isacharparenleft}{\kern0pt}{\isasymforall}A\ {\isasymin}\ Nts\ P{\isachardot}{\kern0pt}\ s\ {\isacharparenleft}{\kern0pt}{\isasymgamma}{\isacharprime}{\kern0pt}\ A{\isacharparenright}{\kern0pt}\ {\isacharequal}{\kern0pt}\ L\ A{\isacharparenright}{\kern0pt}\ {\isasymlongrightarrow}\ eval\ eq\ s\ {\isacharequal}{\kern0pt}\ subst{\isacharunderscore}{\kern0pt}lang\ P\ L\ {\isacharparenleft}{\kern0pt}{\isasymgamma}\ i{\isacharparenright}{\kern0pt}{\isacharparenright}{\kern0pt}{\isachardoublequoteclose}\isanewline
\ \ \ \ \isakeywordONE{unfolding}\isamarkupfalse%
\ bij{\isacharunderscore}{\kern0pt}Nt{\isacharunderscore}{\kern0pt}Var{\isacharunderscore}{\kern0pt}def\ \isakeywordONE{by}\isamarkupfalse%
\ metis\isanewline
\ \ \isakeywordONE{with}\isamarkupfalse%
\ Skolem{\isacharunderscore}{\kern0pt}list{\isacharunderscore}{\kern0pt}nth{\isacharbrackleft}{\kern0pt}\isakeywordTWO{where}\ P{\isacharequal}{\kern0pt}{\isachardoublequoteopen}{\isasymlambda}i\ eq{\isachardot}{\kern0pt}\ reg{\isacharunderscore}{\kern0pt}eval\ eq\ {\isasymand}\ {\isacharparenleft}{\kern0pt}{\isasymforall}x\ {\isasymin}\ vars\ eq{\isachardot}{\kern0pt}\ x\ {\isacharless}{\kern0pt}\ card\ {\isacharparenleft}{\kern0pt}Nts\ P{\isacharparenright}{\kern0pt}{\isacharparenright}{\kern0pt}\isanewline
\ \ \ \ \ \ \ \ \ \ \ \ \ \ \ \ \ \ \ \ \ \ \ {\isasymand}\ {\isacharparenleft}{\kern0pt}{\isasymforall}s\ L{\isachardot}{\kern0pt}\ {\isacharparenleft}{\kern0pt}{\isasymforall}A\ {\isasymin}\ Nts\ P{\isachardot}{\kern0pt}\ s\ {\isacharparenleft}{\kern0pt}{\isasymgamma}{\isacharprime}{\kern0pt}\ A{\isacharparenright}{\kern0pt}\ {\isacharequal}{\kern0pt}\ L\ A{\isacharparenright}{\kern0pt}\ {\isasymlongrightarrow}\ eval\ eq\ s\ {\isacharequal}{\kern0pt}\ subst{\isacharunderscore}{\kern0pt}lang\ P\ L\ {\isacharparenleft}{\kern0pt}{\isasymgamma}\ i{\isacharparenright}{\kern0pt}{\isacharparenright}{\kern0pt}{\isachardoublequoteclose}{\isacharbrackright}{\kern0pt}\isanewline
\ \ \ \ \isakeywordTHREE{show}\isamarkupfalse%
\ {\isacharquery}{\kern0pt}thesis\ \isakeywordONE{by}\isamarkupfalse%
\ blast\isanewline
\isakeywordONE{qed}\isamarkupfalse%
%
\endisatagproof
{\isafoldproof}%
%
\isadelimproof
%
\endisadelimproof
%
\begin{isamarkuptext}%
As we have proved that each CFG induces a system of equations, it remains to show that the
 CFG's language is a minimal solution of this system. The first lemma proves that the CFG's language
is a solution and the next two lemmas prove that it is minimal:%
\end{isamarkuptext}\isamarkuptrue%
\isakeywordONE{abbreviation}\isamarkupfalse%
\ {\isachardoublequoteopen}sol\ {\isasymequiv}\ {\isasymlambda}i{\isachardot}{\kern0pt}\ if\ i\ {\isacharless}{\kern0pt}\ card\ {\isacharparenleft}{\kern0pt}Nts\ P{\isacharparenright}{\kern0pt}\ then\ Lang{\isacharunderscore}{\kern0pt}lfp\ P\ {\isacharparenleft}{\kern0pt}{\isasymgamma}\ i{\isacharparenright}{\kern0pt}\ else\ {\isacharbraceleft}{\kern0pt}{\isacharbraceright}{\kern0pt}{\isachardoublequoteclose}\isanewline
\isanewline
\isakeywordONE{lemma}\isamarkupfalse%
\ CFG{\isacharunderscore}{\kern0pt}sys{\isacharunderscore}{\kern0pt}CFL{\isacharunderscore}{\kern0pt}is{\isacharunderscore}{\kern0pt}sol{\isacharcolon}{\kern0pt}\isanewline
\ \ \isakeywordTWO{assumes}\ {\isachardoublequoteopen}CFG{\isacharunderscore}{\kern0pt}sys\ sys{\isachardoublequoteclose}\isanewline
\ \ \isakeywordTWO{shows}\ {\isachardoublequoteopen}solves{\isacharunderscore}{\kern0pt}ineq{\isacharunderscore}{\kern0pt}sys\ sys\ sol{\isachardoublequoteclose}\isanewline
%
\isadelimproof
%
\endisadelimproof
%
\isatagproof
\isakeywordONE{unfolding}\isamarkupfalse%
\ solves{\isacharunderscore}{\kern0pt}ineq{\isacharunderscore}{\kern0pt}sys{\isacharunderscore}{\kern0pt}def\ \isakeywordONE{proof}\isamarkupfalse%
\ {\isacharparenleft}{\kern0pt}rule\ allI{\isacharcomma}{\kern0pt}\ rule\ impI{\isacharparenright}{\kern0pt}\isanewline
\ \ \isakeywordTHREE{fix}\isamarkupfalse%
\ i\isanewline
\ \ \isakeywordTHREE{assume}\isamarkupfalse%
\ {\isachardoublequoteopen}i\ {\isacharless}{\kern0pt}\ length\ sys{\isachardoublequoteclose}\isanewline
\ \ \isakeywordONE{with}\isamarkupfalse%
\ assms\ \isakeywordONE{have}\isamarkupfalse%
\ {\isachardoublequoteopen}i\ {\isacharless}{\kern0pt}\ card\ {\isacharparenleft}{\kern0pt}Nts\ P{\isacharparenright}{\kern0pt}{\isachardoublequoteclose}\ \isakeywordONE{by}\isamarkupfalse%
\ argo\isanewline
\ \ \isakeywordONE{from}\isamarkupfalse%
\ bij{\isacharunderscore}{\kern0pt}{\isasymgamma}{\isacharunderscore}{\kern0pt}{\isasymgamma}{\isacharprime}{\kern0pt}\ \isakeywordONE{have}\isamarkupfalse%
\ {\isacharasterisk}{\kern0pt}{\isacharcolon}{\kern0pt}\ {\isachardoublequoteopen}{\isasymforall}A\ {\isasymin}\ Nts\ P{\isachardot}{\kern0pt}\ sol\ {\isacharparenleft}{\kern0pt}{\isasymgamma}{\isacharprime}{\kern0pt}\ A{\isacharparenright}{\kern0pt}\ {\isacharequal}{\kern0pt}\ Lang{\isacharunderscore}{\kern0pt}lfp\ P\ A{\isachardoublequoteclose}\isanewline
\ \ \ \ \isakeywordONE{unfolding}\isamarkupfalse%
\ bij{\isacharunderscore}{\kern0pt}Nt{\isacharunderscore}{\kern0pt}Var{\isacharunderscore}{\kern0pt}def\ bij{\isacharunderscore}{\kern0pt}betw{\isacharunderscore}{\kern0pt}def\ \isakeywordONE{by}\isamarkupfalse%
\ force\isanewline
\ \ \isakeywordONE{with}\isamarkupfalse%
\ {\isacartoucheopen}i\ {\isacharless}{\kern0pt}\ card\ {\isacharparenleft}{\kern0pt}Nts\ P{\isacharparenright}{\kern0pt}{\isacartoucheclose}\ assms\ \isakeywordONE{have}\isamarkupfalse%
\ {\isachardoublequoteopen}eval\ {\isacharparenleft}{\kern0pt}sys\ {\isacharbang}{\kern0pt}\ i{\isacharparenright}{\kern0pt}\ sol\ {\isacharequal}{\kern0pt}\ subst{\isacharunderscore}{\kern0pt}lang\ P\ {\isacharparenleft}{\kern0pt}Lang{\isacharunderscore}{\kern0pt}lfp\ P{\isacharparenright}{\kern0pt}\ {\isacharparenleft}{\kern0pt}{\isasymgamma}\ i{\isacharparenright}{\kern0pt}{\isachardoublequoteclose}\isanewline
\ \ \ \ \isakeywordONE{by}\isamarkupfalse%
\ presburger\isanewline
\ \ \isakeywordONE{with}\isamarkupfalse%
\ lfp{\isacharunderscore}{\kern0pt}fixpoint{\isacharbrackleft}{\kern0pt}OF\ mono{\isacharunderscore}{\kern0pt}if{\isacharunderscore}{\kern0pt}omega{\isacharunderscore}{\kern0pt}cont{\isacharbrackleft}{\kern0pt}OF\ omega{\isacharunderscore}{\kern0pt}cont{\isacharunderscore}{\kern0pt}Lang{\isacharunderscore}{\kern0pt}lfp{\isacharbrackright}{\kern0pt}{\isacharbrackright}{\kern0pt}\ \isakeywordONE{have}\isamarkupfalse%
\ {\isadigit{1}}{\isacharcolon}{\kern0pt}\ {\isachardoublequoteopen}eval\ {\isacharparenleft}{\kern0pt}sys\ {\isacharbang}{\kern0pt}\ i{\isacharparenright}{\kern0pt}\ sol\ {\isacharequal}{\kern0pt}\ Lang{\isacharunderscore}{\kern0pt}lfp\ P\ {\isacharparenleft}{\kern0pt}{\isasymgamma}\ i{\isacharparenright}{\kern0pt}{\isachardoublequoteclose}\isanewline
\ \ \ \ \isakeywordONE{unfolding}\isamarkupfalse%
\ Lang{\isacharunderscore}{\kern0pt}lfp{\isacharunderscore}{\kern0pt}def\ \isakeywordONE{by}\isamarkupfalse%
\ metis\isanewline
\ \ \isakeywordONE{from}\isamarkupfalse%
\ {\isacartoucheopen}i\ {\isacharless}{\kern0pt}\ card\ {\isacharparenleft}{\kern0pt}Nts\ P{\isacharparenright}{\kern0pt}{\isacartoucheclose}\ bij{\isacharunderscore}{\kern0pt}{\isasymgamma}{\isacharunderscore}{\kern0pt}{\isasymgamma}{\isacharprime}{\kern0pt}\ \isakeywordONE{have}\isamarkupfalse%
\ {\isachardoublequoteopen}{\isasymgamma}\ i\ {\isasymin}\ Nts\ P{\isachardoublequoteclose}\isanewline
\ \ \ \ \isakeywordONE{unfolding}\isamarkupfalse%
\ bij{\isacharunderscore}{\kern0pt}Nt{\isacharunderscore}{\kern0pt}Var{\isacharunderscore}{\kern0pt}def\ \isakeywordONE{using}\isamarkupfalse%
\ bij{\isacharunderscore}{\kern0pt}betwE\ \isakeywordONE{by}\isamarkupfalse%
\ blast\isanewline
\ \ \isakeywordONE{with}\isamarkupfalse%
\ {\isacharasterisk}{\kern0pt}\ \isakeywordONE{have}\isamarkupfalse%
\ {\isachardoublequoteopen}Lang{\isacharunderscore}{\kern0pt}lfp\ P\ {\isacharparenleft}{\kern0pt}{\isasymgamma}\ i{\isacharparenright}{\kern0pt}\ {\isacharequal}{\kern0pt}\ sol\ {\isacharparenleft}{\kern0pt}{\isasymgamma}{\isacharprime}{\kern0pt}\ {\isacharparenleft}{\kern0pt}{\isasymgamma}\ i{\isacharparenright}{\kern0pt}{\isacharparenright}{\kern0pt}{\isachardoublequoteclose}\ \isakeywordONE{by}\isamarkupfalse%
\ auto\isanewline
\ \ \isakeywordONE{also}\isamarkupfalse%
\ \isakeywordONE{have}\isamarkupfalse%
\ {\isachardoublequoteopen}{\isasymdots}\ {\isacharequal}{\kern0pt}\ sol\ i{\isachardoublequoteclose}\ \isakeywordONE{using}\isamarkupfalse%
\ bij{\isacharunderscore}{\kern0pt}{\isasymgamma}{\isacharunderscore}{\kern0pt}{\isasymgamma}{\isacharprime}{\kern0pt}\ {\isacartoucheopen}i\ {\isacharless}{\kern0pt}\ card\ {\isacharparenleft}{\kern0pt}Nts\ P{\isacharparenright}{\kern0pt}{\isacartoucheclose}\ \isakeywordONE{unfolding}\isamarkupfalse%
\ bij{\isacharunderscore}{\kern0pt}Nt{\isacharunderscore}{\kern0pt}Var{\isacharunderscore}{\kern0pt}def\ \isakeywordONE{by}\isamarkupfalse%
\ auto\isanewline
\ \ \isakeywordONE{finally}\isamarkupfalse%
\ \isakeywordTHREE{show}\isamarkupfalse%
\ {\isachardoublequoteopen}eval\ {\isacharparenleft}{\kern0pt}sys\ {\isacharbang}{\kern0pt}\ i{\isacharparenright}{\kern0pt}\ sol\ {\isasymsubseteq}\ sol\ i{\isachardoublequoteclose}\ \isakeywordONE{using}\isamarkupfalse%
\ {\isadigit{1}}\ \isakeywordONE{by}\isamarkupfalse%
\ blast\isanewline
\isakeywordONE{qed}\isamarkupfalse%
%
\endisatagproof
{\isafoldproof}%
%
\isadelimproof
\isanewline
%
\endisadelimproof
\isanewline
\isakeywordONE{lemma}\isamarkupfalse%
\ CFG{\isacharunderscore}{\kern0pt}sys{\isacharunderscore}{\kern0pt}CFL{\isacharunderscore}{\kern0pt}is{\isacharunderscore}{\kern0pt}min{\isacharunderscore}{\kern0pt}aux{\isacharcolon}{\kern0pt}\isanewline
\ \ \isakeywordTWO{assumes}\ {\isachardoublequoteopen}CFG{\isacharunderscore}{\kern0pt}sys\ sys{\isachardoublequoteclose}\isanewline
\ \ \ \ \ \ \isakeywordTWO{and}\ {\isachardoublequoteopen}solves{\isacharunderscore}{\kern0pt}ineq{\isacharunderscore}{\kern0pt}sys\ sys\ sol{\isacharprime}{\kern0pt}{\isachardoublequoteclose}\isanewline
\ \ \ \ \isakeywordTWO{shows}\ {\isachardoublequoteopen}Lang{\isacharunderscore}{\kern0pt}lfp\ P\ {\isasymle}\ {\isacharparenleft}{\kern0pt}{\isasymlambda}A{\isachardot}{\kern0pt}\ sol{\isacharprime}{\kern0pt}\ {\isacharparenleft}{\kern0pt}{\isasymgamma}{\isacharprime}{\kern0pt}\ A{\isacharparenright}{\kern0pt}{\isacharparenright}{\kern0pt}{\isachardoublequoteclose}\ {\isacharparenleft}{\kern0pt}\isakeywordTWO{is}\ {\isachardoublequoteopen}{\isacharunderscore}{\kern0pt}\ {\isasymle}\ {\isacharquery}{\kern0pt}L{\isacharprime}{\kern0pt}{\isachardoublequoteclose}{\isacharparenright}{\kern0pt}\isanewline
%
\isadelimproof
%
\endisadelimproof
%
\isatagproof
\isakeywordONE{proof}\isamarkupfalse%
\ {\isacharminus}{\kern0pt}\isanewline
\ \ \isakeywordONE{have}\isamarkupfalse%
\ {\isachardoublequoteopen}subst{\isacharunderscore}{\kern0pt}lang\ P\ {\isacharquery}{\kern0pt}L{\isacharprime}{\kern0pt}\ A\ {\isasymsubseteq}\ {\isacharquery}{\kern0pt}L{\isacharprime}{\kern0pt}\ A{\isachardoublequoteclose}\ \isakeywordTWO{for}\ A\isanewline
\ \ \isakeywordONE{proof}\isamarkupfalse%
\ {\isacharparenleft}{\kern0pt}cases\ {\isachardoublequoteopen}A\ {\isasymin}\ Nts\ P{\isachardoublequoteclose}{\isacharparenright}{\kern0pt}\isanewline
\ \ \ \ \isakeywordTHREE{case}\isamarkupfalse%
\ True\isanewline
\ \ \ \ \isakeywordONE{with}\isamarkupfalse%
\ assms{\isacharparenleft}{\kern0pt}{\isadigit{1}}{\isacharparenright}{\kern0pt}\ bij{\isacharunderscore}{\kern0pt}{\isasymgamma}{\isacharunderscore}{\kern0pt}{\isasymgamma}{\isacharprime}{\kern0pt}\ \isakeywordONE{have}\isamarkupfalse%
\ {\isachardoublequoteopen}{\isasymgamma}{\isacharprime}{\kern0pt}\ A\ {\isacharless}{\kern0pt}\ length\ sys{\isachardoublequoteclose}\isanewline
\ \ \ \ \ \ \isakeywordONE{unfolding}\isamarkupfalse%
\ bij{\isacharunderscore}{\kern0pt}Nt{\isacharunderscore}{\kern0pt}Var{\isacharunderscore}{\kern0pt}def\ bij{\isacharunderscore}{\kern0pt}betw{\isacharunderscore}{\kern0pt}def\ \isakeywordONE{by}\isamarkupfalse%
\ fastforce\isanewline
\ \ \ \ \isakeywordONE{with}\isamarkupfalse%
\ assms{\isacharparenleft}{\kern0pt}{\isadigit{1}}{\isacharparenright}{\kern0pt}\ bij{\isacharunderscore}{\kern0pt}{\isasymgamma}{\isacharunderscore}{\kern0pt}{\isasymgamma}{\isacharprime}{\kern0pt}\ True\ \isakeywordONE{have}\isamarkupfalse%
\ {\isachardoublequoteopen}subst{\isacharunderscore}{\kern0pt}lang\ P\ {\isacharquery}{\kern0pt}L{\isacharprime}{\kern0pt}\ A\ {\isacharequal}{\kern0pt}\ eval\ {\isacharparenleft}{\kern0pt}sys\ {\isacharbang}{\kern0pt}\ {\isasymgamma}{\isacharprime}{\kern0pt}\ A{\isacharparenright}{\kern0pt}\ sol{\isacharprime}{\kern0pt}{\isachardoublequoteclose}\isanewline
\ \ \ \ \ \ \isakeywordONE{unfolding}\isamarkupfalse%
\ bij{\isacharunderscore}{\kern0pt}Nt{\isacharunderscore}{\kern0pt}Var{\isacharunderscore}{\kern0pt}def\ \isakeywordONE{by}\isamarkupfalse%
\ metis\isanewline
\ \ \ \ \isakeywordONE{also}\isamarkupfalse%
\ \isakeywordONE{from}\isamarkupfalse%
\ True\ assms{\isacharparenleft}{\kern0pt}{\isadigit{2}}{\isacharparenright}{\kern0pt}\ {\isacartoucheopen}{\isasymgamma}{\isacharprime}{\kern0pt}\ A\ {\isacharless}{\kern0pt}\ length\ sys{\isacartoucheclose}\ bij{\isacharunderscore}{\kern0pt}{\isasymgamma}{\isacharunderscore}{\kern0pt}{\isasymgamma}{\isacharprime}{\kern0pt}\ \isakeywordONE{have}\isamarkupfalse%
\ {\isachardoublequoteopen}{\isasymdots}\ {\isasymsubseteq}\ {\isacharquery}{\kern0pt}L{\isacharprime}{\kern0pt}\ A{\isachardoublequoteclose}\isanewline
\ \ \ \ \ \ \isakeywordONE{unfolding}\isamarkupfalse%
\ solves{\isacharunderscore}{\kern0pt}ineq{\isacharunderscore}{\kern0pt}sys{\isacharunderscore}{\kern0pt}def\ bij{\isacharunderscore}{\kern0pt}Nt{\isacharunderscore}{\kern0pt}Var{\isacharunderscore}{\kern0pt}def\ \isakeywordONE{by}\isamarkupfalse%
\ blast\isanewline
\ \ \ \ \isakeywordONE{finally}\isamarkupfalse%
\ \isakeywordTHREE{show}\isamarkupfalse%
\ {\isacharquery}{\kern0pt}thesis\ \isakeywordONE{{\isachardot}{\kern0pt}}\isamarkupfalse%
\isanewline
\ \ \isakeywordONE{next}\isamarkupfalse%
\isanewline
\ \ \ \ \isakeywordTHREE{case}\isamarkupfalse%
\ False\isanewline
\ \ \ \ \isakeywordONE{then}\isamarkupfalse%
\ \isakeywordONE{have}\isamarkupfalse%
\ {\isachardoublequoteopen}Rhss\ P\ A\ {\isacharequal}{\kern0pt}\ {\isacharbraceleft}{\kern0pt}{\isacharbraceright}{\kern0pt}{\isachardoublequoteclose}\ \isakeywordONE{unfolding}\isamarkupfalse%
\ Nts{\isacharunderscore}{\kern0pt}def\ Rhss{\isacharunderscore}{\kern0pt}def\ \isakeywordONE{by}\isamarkupfalse%
\ blast\isanewline
\ \ \ \ \isakeywordONE{with}\isamarkupfalse%
\ False\ \isakeywordTHREE{show}\isamarkupfalse%
\ {\isacharquery}{\kern0pt}thesis\ \isakeywordONE{unfolding}\isamarkupfalse%
\ subst{\isacharunderscore}{\kern0pt}lang{\isacharunderscore}{\kern0pt}def\ \isakeywordONE{by}\isamarkupfalse%
\ simp\isanewline
\ \ \isakeywordONE{qed}\isamarkupfalse%
\isanewline
\ \ \isakeywordONE{then}\isamarkupfalse%
\ \isakeywordONE{have}\isamarkupfalse%
\ {\isachardoublequoteopen}subst{\isacharunderscore}{\kern0pt}lang\ P\ {\isacharquery}{\kern0pt}L{\isacharprime}{\kern0pt}\ {\isasymle}\ {\isacharquery}{\kern0pt}L{\isacharprime}{\kern0pt}{\isachardoublequoteclose}\ \isakeywordONE{by}\isamarkupfalse%
\ {\isacharparenleft}{\kern0pt}simp\ add{\isacharcolon}{\kern0pt}\ le{\isacharunderscore}{\kern0pt}funI{\isacharparenright}{\kern0pt}\isanewline
\ \ \isakeywordONE{from}\isamarkupfalse%
\ lfp{\isacharunderscore}{\kern0pt}lowerbound{\isacharbrackleft}{\kern0pt}of\ {\isachardoublequoteopen}subst{\isacharunderscore}{\kern0pt}lang\ P{\isachardoublequoteclose}{\isacharcomma}{\kern0pt}\ OF\ this{\isacharbrackright}{\kern0pt}\ Lang{\isacharunderscore}{\kern0pt}lfp{\isacharunderscore}{\kern0pt}def\ \isakeywordTHREE{show}\isamarkupfalse%
\ {\isacharquery}{\kern0pt}thesis\ \isakeywordONE{by}\isamarkupfalse%
\ metis\isanewline
\isakeywordONE{qed}\isamarkupfalse%
%
\endisatagproof
{\isafoldproof}%
%
\isadelimproof
\isanewline
%
\endisadelimproof
\isanewline
\isakeywordONE{lemma}\isamarkupfalse%
\ CFG{\isacharunderscore}{\kern0pt}sys{\isacharunderscore}{\kern0pt}CFL{\isacharunderscore}{\kern0pt}is{\isacharunderscore}{\kern0pt}min{\isacharcolon}{\kern0pt}\isanewline
\ \ \isakeywordTWO{assumes}\ {\isachardoublequoteopen}CFG{\isacharunderscore}{\kern0pt}sys\ sys{\isachardoublequoteclose}\isanewline
\ \ \ \ \ \ \isakeywordTWO{and}\ {\isachardoublequoteopen}solves{\isacharunderscore}{\kern0pt}ineq{\isacharunderscore}{\kern0pt}sys\ sys\ sol{\isacharprime}{\kern0pt}{\isachardoublequoteclose}\isanewline
\ \ \ \ \isakeywordTWO{shows}\ {\isachardoublequoteopen}sol\ x\ {\isasymsubseteq}\ sol{\isacharprime}{\kern0pt}\ x{\isachardoublequoteclose}\isanewline
%
\isadelimproof
%
\endisadelimproof
%
\isatagproof
\isakeywordONE{proof}\isamarkupfalse%
\ {\isacharparenleft}{\kern0pt}cases\ {\isachardoublequoteopen}x\ {\isacharless}{\kern0pt}\ card\ {\isacharparenleft}{\kern0pt}Nts\ P{\isacharparenright}{\kern0pt}{\isachardoublequoteclose}{\isacharparenright}{\kern0pt}\isanewline
\ \ \isakeywordTHREE{case}\isamarkupfalse%
\ True\isanewline
\ \ \isakeywordONE{then}\isamarkupfalse%
\ \isakeywordONE{have}\isamarkupfalse%
\ {\isachardoublequoteopen}sol\ x\ {\isacharequal}{\kern0pt}\ Lang{\isacharunderscore}{\kern0pt}lfp\ P\ {\isacharparenleft}{\kern0pt}{\isasymgamma}\ x{\isacharparenright}{\kern0pt}{\isachardoublequoteclose}\ \isakeywordONE{by}\isamarkupfalse%
\ argo\isanewline
\ \ \isakeywordONE{also}\isamarkupfalse%
\ \isakeywordONE{from}\isamarkupfalse%
\ CFG{\isacharunderscore}{\kern0pt}sys{\isacharunderscore}{\kern0pt}CFL{\isacharunderscore}{\kern0pt}is{\isacharunderscore}{\kern0pt}min{\isacharunderscore}{\kern0pt}aux{\isacharbrackleft}{\kern0pt}OF\ assms{\isacharbrackright}{\kern0pt}\ \isakeywordONE{have}\isamarkupfalse%
\ {\isachardoublequoteopen}{\isasymdots}\ {\isasymsubseteq}\ sol{\isacharprime}{\kern0pt}\ {\isacharparenleft}{\kern0pt}{\isasymgamma}{\isacharprime}{\kern0pt}\ {\isacharparenleft}{\kern0pt}{\isasymgamma}\ x{\isacharparenright}{\kern0pt}{\isacharparenright}{\kern0pt}{\isachardoublequoteclose}\ \isakeywordONE{by}\isamarkupfalse%
\ {\isacharparenleft}{\kern0pt}simp\ add{\isacharcolon}{\kern0pt}\ le{\isacharunderscore}{\kern0pt}fun{\isacharunderscore}{\kern0pt}def{\isacharparenright}{\kern0pt}\isanewline
\ \ \isakeywordONE{finally}\isamarkupfalse%
\ \isakeywordTHREE{show}\isamarkupfalse%
\ {\isacharquery}{\kern0pt}thesis\ \isakeywordONE{using}\isamarkupfalse%
\ True\ bij{\isacharunderscore}{\kern0pt}{\isasymgamma}{\isacharunderscore}{\kern0pt}{\isasymgamma}{\isacharprime}{\kern0pt}\ \isakeywordONE{unfolding}\isamarkupfalse%
\ bij{\isacharunderscore}{\kern0pt}Nt{\isacharunderscore}{\kern0pt}Var{\isacharunderscore}{\kern0pt}def\ \isakeywordONE{by}\isamarkupfalse%
\ auto\isanewline
\isakeywordONE{next}\isamarkupfalse%
\isanewline
\ \ \isakeywordTHREE{case}\isamarkupfalse%
\ False\isanewline
\ \ \isakeywordONE{then}\isamarkupfalse%
\ \isakeywordTHREE{show}\isamarkupfalse%
\ {\isacharquery}{\kern0pt}thesis\ \isakeywordONE{by}\isamarkupfalse%
\ auto\isanewline
\isakeywordONE{qed}\isamarkupfalse%
%
\endisatagproof
{\isafoldproof}%
%
\isadelimproof
%
\endisadelimproof
%
\begin{isamarkuptext}%
Lastly we combine all of the previous lemmas into the desired result of this section, namely
that each CFG induces a system of equations such that the CFG's language is a minimal solution of
the system:%
\end{isamarkuptext}\isamarkuptrue%
\isakeywordONE{lemma}\isamarkupfalse%
\ CFL{\isacharunderscore}{\kern0pt}is{\isacharunderscore}{\kern0pt}min{\isacharunderscore}{\kern0pt}sol{\isacharcolon}{\kern0pt}\isanewline
\ \ {\isachardoublequoteopen}{\isasymexists}sys{\isachardot}{\kern0pt}\ {\isacharparenleft}{\kern0pt}{\isasymforall}eq\ {\isasymin}\ set\ sys{\isachardot}{\kern0pt}\ reg{\isacharunderscore}{\kern0pt}eval\ eq{\isacharparenright}{\kern0pt}\ {\isasymand}\ {\isacharparenleft}{\kern0pt}{\isasymforall}eq\ {\isasymin}\ set\ sys{\isachardot}{\kern0pt}\ {\isasymforall}x\ {\isasymin}\ vars\ eq{\isachardot}{\kern0pt}\ x\ {\isacharless}{\kern0pt}\ length\ sys{\isacharparenright}{\kern0pt}\isanewline
\ \ \ \ \ \ \ \ \ \ {\isasymand}\ min{\isacharunderscore}{\kern0pt}sol{\isacharunderscore}{\kern0pt}ineq{\isacharunderscore}{\kern0pt}sys\ sys\ sol{\isachardoublequoteclose}\isanewline
%
\isadelimproof
%
\endisadelimproof
%
\isatagproof
\isakeywordONE{proof}\isamarkupfalse%
\ {\isacharminus}{\kern0pt}\isanewline
\ \ \isakeywordONE{from}\isamarkupfalse%
\ CFG{\isacharunderscore}{\kern0pt}as{\isacharunderscore}{\kern0pt}eq{\isacharunderscore}{\kern0pt}sys\ \isakeywordTHREE{obtain}\isamarkupfalse%
\ sys\ \isakeywordTWO{where}\ {\isacharasterisk}{\kern0pt}{\isacharcolon}{\kern0pt}\ {\isachardoublequoteopen}CFG{\isacharunderscore}{\kern0pt}sys\ sys{\isachardoublequoteclose}\ \isakeywordONE{by}\isamarkupfalse%
\ blast\isanewline
\ \ \isakeywordONE{then}\isamarkupfalse%
\ \isakeywordONE{have}\isamarkupfalse%
\ {\isachardoublequoteopen}length\ sys\ {\isacharequal}{\kern0pt}\ card\ {\isacharparenleft}{\kern0pt}Nts\ P{\isacharparenright}{\kern0pt}{\isachardoublequoteclose}\ \isakeywordONE{by}\isamarkupfalse%
\ blast\isanewline
\ \ \isakeywordONE{moreover}\isamarkupfalse%
\ \isakeywordONE{from}\isamarkupfalse%
\ {\isacharasterisk}{\kern0pt}\ \isakeywordONE{have}\isamarkupfalse%
\ {\isachardoublequoteopen}{\isasymforall}eq\ {\isasymin}\ set\ sys{\isachardot}{\kern0pt}\ reg{\isacharunderscore}{\kern0pt}eval\ eq{\isachardoublequoteclose}\ \isakeywordONE{by}\isamarkupfalse%
\ {\isacharparenleft}{\kern0pt}simp\ add{\isacharcolon}{\kern0pt}\ all{\isacharunderscore}{\kern0pt}set{\isacharunderscore}{\kern0pt}conv{\isacharunderscore}{\kern0pt}all{\isacharunderscore}{\kern0pt}nth{\isacharparenright}{\kern0pt}\isanewline
\ \ \isakeywordONE{moreover}\isamarkupfalse%
\ \isakeywordONE{from}\isamarkupfalse%
\ {\isacharasterisk}{\kern0pt}\ {\isacartoucheopen}length\ sys\ {\isacharequal}{\kern0pt}\ card\ {\isacharparenleft}{\kern0pt}Nts\ P{\isacharparenright}{\kern0pt}{\isacartoucheclose}\ \isakeywordONE{have}\isamarkupfalse%
\ {\isachardoublequoteopen}{\isasymforall}eq\ {\isasymin}\ set\ sys{\isachardot}{\kern0pt}\ {\isasymforall}x\ {\isasymin}\ vars\ eq{\isachardot}{\kern0pt}\ x\ {\isacharless}{\kern0pt}\ length\ sys{\isachardoublequoteclose}\isanewline
\ \ \ \ \isakeywordONE{by}\isamarkupfalse%
\ {\isacharparenleft}{\kern0pt}simp\ add{\isacharcolon}{\kern0pt}\ all{\isacharunderscore}{\kern0pt}set{\isacharunderscore}{\kern0pt}conv{\isacharunderscore}{\kern0pt}all{\isacharunderscore}{\kern0pt}nth{\isacharparenright}{\kern0pt}\isanewline
\ \ \isakeywordONE{moreover}\isamarkupfalse%
\ \isakeywordONE{from}\isamarkupfalse%
\ CFG{\isacharunderscore}{\kern0pt}sys{\isacharunderscore}{\kern0pt}CFL{\isacharunderscore}{\kern0pt}is{\isacharunderscore}{\kern0pt}sol{\isacharbrackleft}{\kern0pt}OF\ {\isacharasterisk}{\kern0pt}{\isacharbrackright}{\kern0pt}\ CFG{\isacharunderscore}{\kern0pt}sys{\isacharunderscore}{\kern0pt}CFL{\isacharunderscore}{\kern0pt}is{\isacharunderscore}{\kern0pt}min{\isacharbrackleft}{\kern0pt}OF\ {\isacharasterisk}{\kern0pt}{\isacharbrackright}{\kern0pt}\isanewline
\ \ \ \ \isakeywordONE{have}\isamarkupfalse%
\ {\isachardoublequoteopen}min{\isacharunderscore}{\kern0pt}sol{\isacharunderscore}{\kern0pt}ineq{\isacharunderscore}{\kern0pt}sys\ sys\ sol{\isachardoublequoteclose}\ \isakeywordONE{unfolding}\isamarkupfalse%
\ min{\isacharunderscore}{\kern0pt}sol{\isacharunderscore}{\kern0pt}ineq{\isacharunderscore}{\kern0pt}sys{\isacharunderscore}{\kern0pt}def\ \isakeywordONE{by}\isamarkupfalse%
\ blast\isanewline
\ \ \isakeywordONE{ultimately}\isamarkupfalse%
\ \isakeywordTHREE{show}\isamarkupfalse%
\ {\isacharquery}{\kern0pt}thesis\ \isakeywordONE{by}\isamarkupfalse%
\ blast\isanewline
\isakeywordONE{qed}\isamarkupfalse%
%
\endisatagproof
{\isafoldproof}%
%
\isadelimproof
\isanewline
%
\endisadelimproof
\isanewline
\isakeywordTWO{end}\isamarkupfalse%
%
\isadelimdocument
%
\endisadelimdocument
%
\isatagdocument
%
\isamarkupsubsection{Relation between the two types of systems of equations%
}
\isamarkuptrue%
%
\endisatagdocument
{\isafolddocument}%
%
\isadelimdocument
%
\endisadelimdocument
%
\begin{isamarkuptext}%
One can simply convert a system \isa{sys} of equations of the second type (i.e. with Parikh
images) into a system of equations of the first type by dropping the Parikh images on both side of
each equation. The following lemmas describe how the two systems are related to each other.

First of all, to any solution \isa{sol} of \isa{sys} exists a valuation whose Parikh image is
identical to that of \isa{sol} and which is a solution of the other system (i.e. the system obtained
by dropping all Parikh images in \isa{sys}). The proof benefits from the result of section 2.7:%
\end{isamarkuptext}\isamarkuptrue%
\isakeywordONE{lemma}\isamarkupfalse%
\ sol{\isacharunderscore}{\kern0pt}comm{\isacharunderscore}{\kern0pt}sol{\isacharcolon}{\kern0pt}\isanewline
\ \ \isakeywordTWO{assumes}\ sol{\isacharunderscore}{\kern0pt}is{\isacharunderscore}{\kern0pt}sol{\isacharunderscore}{\kern0pt}comm{\isacharcolon}{\kern0pt}\ {\isachardoublequoteopen}solves{\isacharunderscore}{\kern0pt}ineq{\isacharunderscore}{\kern0pt}sys{\isacharunderscore}{\kern0pt}comm\ sys\ sol{\isachardoublequoteclose}\isanewline
\ \ \isakeywordTWO{shows}\ \ \ {\isachardoublequoteopen}{\isasymexists}sol{\isacharprime}{\kern0pt}{\isachardot}{\kern0pt}\ {\isacharparenleft}{\kern0pt}{\isasymforall}x{\isachardot}{\kern0pt}\ {\isasymPsi}\ {\isacharparenleft}{\kern0pt}sol\ x{\isacharparenright}{\kern0pt}\ {\isacharequal}{\kern0pt}\ {\isasymPsi}\ {\isacharparenleft}{\kern0pt}sol{\isacharprime}{\kern0pt}\ x{\isacharparenright}{\kern0pt}{\isacharparenright}{\kern0pt}\ {\isasymand}\ solves{\isacharunderscore}{\kern0pt}ineq{\isacharunderscore}{\kern0pt}sys\ sys\ sol{\isacharprime}{\kern0pt}{\isachardoublequoteclose}\isanewline
%
\isadelimproof
%
\endisadelimproof
%
\isatagproof
\isakeywordONE{proof}\isamarkupfalse%
\isanewline
\ \ \isakeywordONE{let}\isamarkupfalse%
\ {\isacharquery}{\kern0pt}sol{\isacharprime}{\kern0pt}\ {\isacharequal}{\kern0pt}\ {\isachardoublequoteopen}{\isasymlambda}x{\isachardot}{\kern0pt}\ {\isasymUnion}{\isacharparenleft}{\kern0pt}parikh{\isacharunderscore}{\kern0pt}img{\isacharunderscore}{\kern0pt}eq{\isacharunderscore}{\kern0pt}class\ {\isacharparenleft}{\kern0pt}sol\ x{\isacharparenright}{\kern0pt}{\isacharparenright}{\kern0pt}{\isachardoublequoteclose}\isanewline
\ \ \isakeywordONE{have}\isamarkupfalse%
\ sol{\isacharprime}{\kern0pt}{\isacharunderscore}{\kern0pt}sol{\isacharcolon}{\kern0pt}\ {\isachardoublequoteopen}{\isasymforall}x{\isachardot}{\kern0pt}\ {\isasymPsi}\ {\isacharparenleft}{\kern0pt}{\isacharquery}{\kern0pt}sol{\isacharprime}{\kern0pt}\ x{\isacharparenright}{\kern0pt}\ {\isacharequal}{\kern0pt}\ {\isasymPsi}\ {\isacharparenleft}{\kern0pt}sol\ x{\isacharparenright}{\kern0pt}{\isachardoublequoteclose}\isanewline
\ \ \ \ \ \ \isakeywordONE{using}\isamarkupfalse%
\ parikh{\isacharunderscore}{\kern0pt}img{\isacharunderscore}{\kern0pt}Union{\isacharunderscore}{\kern0pt}class\ \isakeywordONE{by}\isamarkupfalse%
\ metis\isanewline
\ \ \isakeywordONE{moreover}\isamarkupfalse%
\ \isakeywordONE{have}\isamarkupfalse%
\ {\isachardoublequoteopen}solves{\isacharunderscore}{\kern0pt}ineq{\isacharunderscore}{\kern0pt}sys\ sys\ {\isacharquery}{\kern0pt}sol{\isacharprime}{\kern0pt}{\isachardoublequoteclose}\isanewline
\ \ \isakeywordONE{unfolding}\isamarkupfalse%
\ solves{\isacharunderscore}{\kern0pt}ineq{\isacharunderscore}{\kern0pt}sys{\isacharunderscore}{\kern0pt}def\ \isakeywordONE{proof}\isamarkupfalse%
\ {\isacharparenleft}{\kern0pt}rule\ allI{\isacharcomma}{\kern0pt}\ rule\ impI{\isacharparenright}{\kern0pt}\isanewline
\ \ \ \ \isakeywordTHREE{fix}\isamarkupfalse%
\ i\isanewline
\ \ \ \ \isakeywordTHREE{assume}\isamarkupfalse%
\ {\isachardoublequoteopen}i\ {\isacharless}{\kern0pt}\ length\ sys{\isachardoublequoteclose}\isanewline
\ \ \ \ \isakeywordONE{with}\isamarkupfalse%
\ sol{\isacharunderscore}{\kern0pt}is{\isacharunderscore}{\kern0pt}sol{\isacharunderscore}{\kern0pt}comm\ \isakeywordONE{have}\isamarkupfalse%
\ {\isachardoublequoteopen}{\isasymPsi}\ {\isacharparenleft}{\kern0pt}eval\ {\isacharparenleft}{\kern0pt}sys\ {\isacharbang}{\kern0pt}\ i{\isacharparenright}{\kern0pt}\ sol{\isacharparenright}{\kern0pt}\ {\isasymsubseteq}\ {\isasymPsi}\ {\isacharparenleft}{\kern0pt}sol\ i{\isacharparenright}{\kern0pt}{\isachardoublequoteclose}\isanewline
\ \ \ \ \ \ \isakeywordONE{unfolding}\isamarkupfalse%
\ solves{\isacharunderscore}{\kern0pt}ineq{\isacharunderscore}{\kern0pt}sys{\isacharunderscore}{\kern0pt}comm{\isacharunderscore}{\kern0pt}def\ solves{\isacharunderscore}{\kern0pt}ineq{\isacharunderscore}{\kern0pt}comm{\isacharunderscore}{\kern0pt}def\ \isakeywordONE{by}\isamarkupfalse%
\ blast\isanewline
\ \ \ \ \isakeywordONE{moreover}\isamarkupfalse%
\ \isakeywordONE{from}\isamarkupfalse%
\ sol{\isacharprime}{\kern0pt}{\isacharunderscore}{\kern0pt}sol\ \isakeywordONE{have}\isamarkupfalse%
\ {\isachardoublequoteopen}{\isasymPsi}\ {\isacharparenleft}{\kern0pt}eval\ {\isacharparenleft}{\kern0pt}sys\ {\isacharbang}{\kern0pt}\ i{\isacharparenright}{\kern0pt}\ {\isacharquery}{\kern0pt}sol{\isacharprime}{\kern0pt}{\isacharparenright}{\kern0pt}\ {\isacharequal}{\kern0pt}\ {\isasymPsi}\ {\isacharparenleft}{\kern0pt}eval\ {\isacharparenleft}{\kern0pt}sys\ {\isacharbang}{\kern0pt}\ i{\isacharparenright}{\kern0pt}\ sol{\isacharparenright}{\kern0pt}{\isachardoublequoteclose}\isanewline
\ \ \ \ \ \ \isakeywordONE{using}\isamarkupfalse%
\ rlexp{\isacharunderscore}{\kern0pt}mono{\isacharunderscore}{\kern0pt}parikh{\isacharunderscore}{\kern0pt}eq\ \isakeywordONE{by}\isamarkupfalse%
\ meson\isanewline
\ \ \ \ \isakeywordONE{ultimately}\isamarkupfalse%
\ \isakeywordONE{have}\isamarkupfalse%
\ {\isachardoublequoteopen}{\isasymPsi}\ {\isacharparenleft}{\kern0pt}eval\ {\isacharparenleft}{\kern0pt}sys\ {\isacharbang}{\kern0pt}\ i{\isacharparenright}{\kern0pt}\ {\isacharquery}{\kern0pt}sol{\isacharprime}{\kern0pt}{\isacharparenright}{\kern0pt}\ {\isasymsubseteq}\ {\isasymPsi}\ {\isacharparenleft}{\kern0pt}sol\ i{\isacharparenright}{\kern0pt}{\isachardoublequoteclose}\ \isakeywordONE{by}\isamarkupfalse%
\ simp\isanewline
\ \ \ \ \isakeywordONE{then}\isamarkupfalse%
\ \isakeywordTHREE{show}\isamarkupfalse%
\ {\isachardoublequoteopen}eval\ {\isacharparenleft}{\kern0pt}sys\ {\isacharbang}{\kern0pt}\ i{\isacharparenright}{\kern0pt}\ {\isacharquery}{\kern0pt}sol{\isacharprime}{\kern0pt}\ {\isasymsubseteq}\ {\isacharquery}{\kern0pt}sol{\isacharprime}{\kern0pt}\ i{\isachardoublequoteclose}\ \isakeywordONE{using}\isamarkupfalse%
\ subseteq{\isacharunderscore}{\kern0pt}comm{\isacharunderscore}{\kern0pt}subseteq\ \isakeywordONE{by}\isamarkupfalse%
\ metis\isanewline
\ \ \isakeywordONE{qed}\isamarkupfalse%
\isanewline
\ \ \isakeywordONE{ultimately}\isamarkupfalse%
\ \isakeywordTHREE{show}\isamarkupfalse%
\ {\isachardoublequoteopen}{\isacharparenleft}{\kern0pt}{\isasymforall}x{\isachardot}{\kern0pt}\ {\isasymPsi}\ {\isacharparenleft}{\kern0pt}sol\ x{\isacharparenright}{\kern0pt}\ {\isacharequal}{\kern0pt}\ {\isasymPsi}\ {\isacharparenleft}{\kern0pt}{\isacharquery}{\kern0pt}sol{\isacharprime}{\kern0pt}\ x{\isacharparenright}{\kern0pt}{\isacharparenright}{\kern0pt}\ {\isasymand}\ solves{\isacharunderscore}{\kern0pt}ineq{\isacharunderscore}{\kern0pt}sys\ sys\ {\isacharquery}{\kern0pt}sol{\isacharprime}{\kern0pt}{\isachardoublequoteclose}\isanewline
\ \ \ \ \isakeywordONE{by}\isamarkupfalse%
\ simp\isanewline
\isakeywordONE{qed}\isamarkupfalse%
%
\endisatagproof
{\isafoldproof}%
%
\isadelimproof
%
\endisadelimproof
%
\begin{isamarkuptext}%
The converse works similarly: Given a minimal solution \isa{sol} of the system \isa{sys} of the first type,
then \isa{sol} is also a minimal solution to the system obtained by converting \isa{sys} into a system of the second
type (which can be achieved by applying the Parikh image on both sides of each equation):%
\end{isamarkuptext}\isamarkuptrue%
\isakeywordONE{lemma}\isamarkupfalse%
\ min{\isacharunderscore}{\kern0pt}sol{\isacharunderscore}{\kern0pt}min{\isacharunderscore}{\kern0pt}sol{\isacharunderscore}{\kern0pt}comm{\isacharcolon}{\kern0pt}\isanewline
\ \ \isakeywordTWO{assumes}\ {\isachardoublequoteopen}min{\isacharunderscore}{\kern0pt}sol{\isacharunderscore}{\kern0pt}ineq{\isacharunderscore}{\kern0pt}sys\ sys\ sol{\isachardoublequoteclose}\isanewline
\ \ \ \ \isakeywordTWO{shows}\ {\isachardoublequoteopen}min{\isacharunderscore}{\kern0pt}sol{\isacharunderscore}{\kern0pt}ineq{\isacharunderscore}{\kern0pt}sys{\isacharunderscore}{\kern0pt}comm\ sys\ sol{\isachardoublequoteclose}\isanewline
%
\isadelimproof
%
\endisadelimproof
%
\isatagproof
\isakeywordONE{unfolding}\isamarkupfalse%
\ min{\isacharunderscore}{\kern0pt}sol{\isacharunderscore}{\kern0pt}ineq{\isacharunderscore}{\kern0pt}sys{\isacharunderscore}{\kern0pt}comm{\isacharunderscore}{\kern0pt}def\ \isakeywordONE{proof}\isamarkupfalse%
\isanewline
\ \ \isakeywordONE{from}\isamarkupfalse%
\ assms\ \isakeywordTHREE{show}\isamarkupfalse%
\ {\isachardoublequoteopen}solves{\isacharunderscore}{\kern0pt}ineq{\isacharunderscore}{\kern0pt}sys{\isacharunderscore}{\kern0pt}comm\ sys\ sol{\isachardoublequoteclose}\isanewline
\ \ \ \ \isakeywordONE{unfolding}\isamarkupfalse%
\ min{\isacharunderscore}{\kern0pt}sol{\isacharunderscore}{\kern0pt}ineq{\isacharunderscore}{\kern0pt}sys{\isacharunderscore}{\kern0pt}def\ min{\isacharunderscore}{\kern0pt}sol{\isacharunderscore}{\kern0pt}ineq{\isacharunderscore}{\kern0pt}sys{\isacharunderscore}{\kern0pt}comm{\isacharunderscore}{\kern0pt}def\ solves{\isacharunderscore}{\kern0pt}ineq{\isacharunderscore}{\kern0pt}sys{\isacharunderscore}{\kern0pt}def\isanewline
\ \ \ \ \ \ solves{\isacharunderscore}{\kern0pt}ineq{\isacharunderscore}{\kern0pt}sys{\isacharunderscore}{\kern0pt}comm{\isacharunderscore}{\kern0pt}def\ solves{\isacharunderscore}{\kern0pt}ineq{\isacharunderscore}{\kern0pt}comm{\isacharunderscore}{\kern0pt}def\ \isakeywordONE{by}\isamarkupfalse%
\ {\isacharparenleft}{\kern0pt}simp\ add{\isacharcolon}{\kern0pt}\ parikh{\isacharunderscore}{\kern0pt}img{\isacharunderscore}{\kern0pt}mono{\isacharparenright}{\kern0pt}\isanewline
\ \ \isakeywordTHREE{show}\isamarkupfalse%
\ {\isachardoublequoteopen}\ {\isasymforall}sol{\isacharprime}{\kern0pt}{\isachardot}{\kern0pt}\ solves{\isacharunderscore}{\kern0pt}ineq{\isacharunderscore}{\kern0pt}sys{\isacharunderscore}{\kern0pt}comm\ sys\ sol{\isacharprime}{\kern0pt}\ {\isasymlongrightarrow}\ {\isacharparenleft}{\kern0pt}{\isasymforall}x{\isachardot}{\kern0pt}\ {\isasymPsi}\ {\isacharparenleft}{\kern0pt}sol\ x{\isacharparenright}{\kern0pt}\ {\isasymsubseteq}\ {\isasymPsi}\ {\isacharparenleft}{\kern0pt}sol{\isacharprime}{\kern0pt}\ x{\isacharparenright}{\kern0pt}{\isacharparenright}{\kern0pt}{\isachardoublequoteclose}\isanewline
\ \ \isakeywordONE{proof}\isamarkupfalse%
\ {\isacharparenleft}{\kern0pt}rule\ allI{\isacharcomma}{\kern0pt}\ rule\ impI{\isacharparenright}{\kern0pt}\isanewline
\ \ \ \ \isakeywordTHREE{fix}\isamarkupfalse%
\ sol{\isacharprime}{\kern0pt}\isanewline
\ \ \ \ \isakeywordTHREE{assume}\isamarkupfalse%
\ {\isachardoublequoteopen}solves{\isacharunderscore}{\kern0pt}ineq{\isacharunderscore}{\kern0pt}sys{\isacharunderscore}{\kern0pt}comm\ sys\ sol{\isacharprime}{\kern0pt}{\isachardoublequoteclose}\isanewline
\ \ \ \ \isakeywordONE{with}\isamarkupfalse%
\ sol{\isacharunderscore}{\kern0pt}comm{\isacharunderscore}{\kern0pt}sol\ \isakeywordTHREE{obtain}\isamarkupfalse%
\ sol{\isacharprime}{\kern0pt}{\isacharprime}{\kern0pt}\ \isakeywordTWO{where}\ sol{\isacharprime}{\kern0pt}{\isacharprime}{\kern0pt}{\isacharunderscore}{\kern0pt}intro{\isacharcolon}{\kern0pt}\isanewline
\ \ \ \ \ \ {\isachardoublequoteopen}{\isacharparenleft}{\kern0pt}{\isasymforall}x{\isachardot}{\kern0pt}\ {\isasymPsi}\ {\isacharparenleft}{\kern0pt}sol{\isacharprime}{\kern0pt}\ x{\isacharparenright}{\kern0pt}\ {\isacharequal}{\kern0pt}\ {\isasymPsi}\ {\isacharparenleft}{\kern0pt}sol{\isacharprime}{\kern0pt}{\isacharprime}{\kern0pt}\ x{\isacharparenright}{\kern0pt}{\isacharparenright}{\kern0pt}\ {\isasymand}\ solves{\isacharunderscore}{\kern0pt}ineq{\isacharunderscore}{\kern0pt}sys\ sys\ sol{\isacharprime}{\kern0pt}{\isacharprime}{\kern0pt}{\isachardoublequoteclose}\ \isakeywordONE{by}\isamarkupfalse%
\ meson\isanewline
\ \ \ \ \isakeywordONE{with}\isamarkupfalse%
\ assms\ \isakeywordONE{have}\isamarkupfalse%
\ {\isachardoublequoteopen}{\isasymforall}x{\isachardot}{\kern0pt}\ sol\ x\ {\isasymsubseteq}\ sol{\isacharprime}{\kern0pt}{\isacharprime}{\kern0pt}\ x{\isachardoublequoteclose}\ \isakeywordONE{unfolding}\isamarkupfalse%
\ min{\isacharunderscore}{\kern0pt}sol{\isacharunderscore}{\kern0pt}ineq{\isacharunderscore}{\kern0pt}sys{\isacharunderscore}{\kern0pt}def\ \isakeywordONE{by}\isamarkupfalse%
\ auto\isanewline
\ \ \ \ \isakeywordONE{with}\isamarkupfalse%
\ sol{\isacharprime}{\kern0pt}{\isacharprime}{\kern0pt}{\isacharunderscore}{\kern0pt}intro\ \isakeywordTHREE{show}\isamarkupfalse%
\ {\isachardoublequoteopen}{\isasymforall}x{\isachardot}{\kern0pt}\ {\isasymPsi}\ {\isacharparenleft}{\kern0pt}sol\ x{\isacharparenright}{\kern0pt}\ {\isasymsubseteq}\ {\isasymPsi}\ {\isacharparenleft}{\kern0pt}sol{\isacharprime}{\kern0pt}\ x{\isacharparenright}{\kern0pt}{\isachardoublequoteclose}\isanewline
\ \ \ \ \ \ \isakeywordONE{using}\isamarkupfalse%
\ parikh{\isacharunderscore}{\kern0pt}img{\isacharunderscore}{\kern0pt}mono\ \isakeywordONE{by}\isamarkupfalse%
\ metis\isanewline
\ \ \isakeywordONE{qed}\isamarkupfalse%
\isanewline
\isakeywordONE{qed}\isamarkupfalse%
%
\endisatagproof
{\isafoldproof}%
%
\isadelimproof
%
\endisadelimproof
%
\begin{isamarkuptext}%
All minimal solutions of a system of the second type have the same Parikh image:%
\end{isamarkuptext}\isamarkuptrue%
\isakeywordONE{lemma}\isamarkupfalse%
\ min{\isacharunderscore}{\kern0pt}sol{\isacharunderscore}{\kern0pt}comm{\isacharunderscore}{\kern0pt}unique{\isacharcolon}{\kern0pt}\isanewline
\ \ \isakeywordTWO{assumes}\ sol{\isadigit{1}}{\isacharunderscore}{\kern0pt}is{\isacharunderscore}{\kern0pt}min{\isacharunderscore}{\kern0pt}sol{\isacharcolon}{\kern0pt}\ {\isachardoublequoteopen}min{\isacharunderscore}{\kern0pt}sol{\isacharunderscore}{\kern0pt}ineq{\isacharunderscore}{\kern0pt}sys{\isacharunderscore}{\kern0pt}comm\ sys\ sol{\isadigit{1}}{\isachardoublequoteclose}\isanewline
\ \ \ \ \ \ \isakeywordTWO{and}\ sol{\isadigit{2}}{\isacharunderscore}{\kern0pt}is{\isacharunderscore}{\kern0pt}min{\isacharunderscore}{\kern0pt}sol{\isacharcolon}{\kern0pt}\ {\isachardoublequoteopen}min{\isacharunderscore}{\kern0pt}sol{\isacharunderscore}{\kern0pt}ineq{\isacharunderscore}{\kern0pt}sys{\isacharunderscore}{\kern0pt}comm\ sys\ sol{\isadigit{2}}{\isachardoublequoteclose}\isanewline
\ \ \ \ \isakeywordTWO{shows}\ \ \ \ \ \ \ \ \ \ \ \ \ \ \ \ \ \ {\isachardoublequoteopen}{\isasymPsi}\ {\isacharparenleft}{\kern0pt}sol{\isadigit{1}}\ x{\isacharparenright}{\kern0pt}\ {\isacharequal}{\kern0pt}\ {\isasymPsi}\ {\isacharparenleft}{\kern0pt}sol{\isadigit{2}}\ x{\isacharparenright}{\kern0pt}{\isachardoublequoteclose}\isanewline
%
\isadelimproof
%
\endisadelimproof
%
\isatagproof
\isakeywordONE{proof}\isamarkupfalse%
\ {\isacharminus}{\kern0pt}\isanewline
\ \ \isakeywordONE{from}\isamarkupfalse%
\ sol{\isadigit{1}}{\isacharunderscore}{\kern0pt}is{\isacharunderscore}{\kern0pt}min{\isacharunderscore}{\kern0pt}sol\ sol{\isadigit{2}}{\isacharunderscore}{\kern0pt}is{\isacharunderscore}{\kern0pt}min{\isacharunderscore}{\kern0pt}sol\ \isakeywordONE{have}\isamarkupfalse%
\ {\isachardoublequoteopen}{\isasymPsi}\ {\isacharparenleft}{\kern0pt}sol{\isadigit{1}}\ x{\isacharparenright}{\kern0pt}\ {\isasymsubseteq}\ {\isasymPsi}\ {\isacharparenleft}{\kern0pt}sol{\isadigit{2}}\ x{\isacharparenright}{\kern0pt}{\isachardoublequoteclose}\isanewline
\ \ \ \ \isakeywordONE{unfolding}\isamarkupfalse%
\ min{\isacharunderscore}{\kern0pt}sol{\isacharunderscore}{\kern0pt}ineq{\isacharunderscore}{\kern0pt}sys{\isacharunderscore}{\kern0pt}comm{\isacharunderscore}{\kern0pt}def\ \isakeywordONE{by}\isamarkupfalse%
\ simp\isanewline
\ \ \isakeywordONE{moreover}\isamarkupfalse%
\ \isakeywordONE{from}\isamarkupfalse%
\ sol{\isadigit{1}}{\isacharunderscore}{\kern0pt}is{\isacharunderscore}{\kern0pt}min{\isacharunderscore}{\kern0pt}sol\ sol{\isadigit{2}}{\isacharunderscore}{\kern0pt}is{\isacharunderscore}{\kern0pt}min{\isacharunderscore}{\kern0pt}sol\ \isakeywordONE{have}\isamarkupfalse%
\ {\isachardoublequoteopen}{\isasymPsi}\ {\isacharparenleft}{\kern0pt}sol{\isadigit{2}}\ x{\isacharparenright}{\kern0pt}\ {\isasymsubseteq}\ {\isasymPsi}\ {\isacharparenleft}{\kern0pt}sol{\isadigit{1}}\ x{\isacharparenright}{\kern0pt}{\isachardoublequoteclose}\isanewline
\ \ \ \ \isakeywordONE{unfolding}\isamarkupfalse%
\ min{\isacharunderscore}{\kern0pt}sol{\isacharunderscore}{\kern0pt}ineq{\isacharunderscore}{\kern0pt}sys{\isacharunderscore}{\kern0pt}comm{\isacharunderscore}{\kern0pt}def\ \isakeywordONE{by}\isamarkupfalse%
\ simp\isanewline
\ \ \isakeywordONE{ultimately}\isamarkupfalse%
\ \isakeywordTHREE{show}\isamarkupfalse%
\ {\isacharquery}{\kern0pt}thesis\ \isakeywordONE{by}\isamarkupfalse%
\ blast\isanewline
\isakeywordONE{qed}\isamarkupfalse%
%
\endisatagproof
{\isafoldproof}%
%
\isadelimproof
\isanewline
%
\endisadelimproof
%
\isadelimtheory
\isanewline
%
\endisadelimtheory
%
\isatagtheory
\isakeywordTWO{end}\isamarkupfalse%
%
\endisatagtheory
{\isafoldtheory}%
%
\isadelimtheory
%
\endisadelimtheory
%
\end{isabellebody}%
\endinput
%:%file=~/studium/semester_7/semantik/homeworks/AIST/Parikh/Eq_Sys.thy%:%
%:%11=1%:%
%:%27=3%:%
%:%28=3%:%
%:%29=4%:%
%:%30=5%:%
%:%31=6%:%
%:%32=7%:%
%:%41=9%:%
%:%42=10%:%
%:%43=11%:%
%:%52=14%:%
%:%64=16%:%
%:%65=17%:%
%:%66=18%:%
%:%67=19%:%
%:%68=20%:%
%:%69=21%:%
%:%70=22%:%
%:%71=23%:%
%:%73=25%:%
%:%74=25%:%
%:%76=28%:%
%:%77=29%:%
%:%78=30%:%
%:%80=31%:%
%:%81=31%:%
%:%82=32%:%
%:%83=33%:%
%:%84=34%:%
%:%85=34%:%
%:%86=35%:%
%:%89=39%:%
%:%90=40%:%
%:%92=41%:%
%:%93=41%:%
%:%94=42%:%
%:%95=43%:%
%:%96=44%:%
%:%97=44%:%
%:%98=45%:%
%:%99=46%:%
%:%100=47%:%
%:%101=47%:%
%:%102=48%:%
%:%106=52%:%
%:%108=53%:%
%:%109=53%:%
%:%110=54%:%
%:%111=55%:%
%:%112=56%:%
%:%113=56%:%
%:%114=57%:%
%:%115=58%:%
%:%118=59%:%
%:%122=59%:%
%:%123=59%:%
%:%124=59%:%
%:%138=62%:%
%:%150=64%:%
%:%151=65%:%
%:%152=66%:%
%:%153=67%:%
%:%155=68%:%
%:%156=68%:%
%:%157=69%:%
%:%159=71%:%
%:%160=72%:%
%:%161=73%:%
%:%163=74%:%
%:%164=74%:%
%:%165=75%:%
%:%167=77%:%
%:%168=78%:%
%:%169=79%:%
%:%176=85%:%
%:%177=85%:%
%:%178=86%:%
%:%184=92%:%
%:%185=93%:%
%:%186=94%:%
%:%187=95%:%
%:%188=96%:%
%:%190=97%:%
%:%191=97%:%
%:%192=98%:%
%:%200=107%:%
%:%201=108%:%
%:%203=109%:%
%:%204=109%:%
%:%205=110%:%
%:%206=111%:%
%:%207=112%:%
%:%208=113%:%
%:%215=114%:%
%:%216=114%:%
%:%217=115%:%
%:%218=115%:%
%:%219=115%:%
%:%220=116%:%
%:%221=116%:%
%:%222=116%:%
%:%223=116%:%
%:%224=117%:%
%:%225=117%:%
%:%226=117%:%
%:%227=117%:%
%:%228=118%:%
%:%229=118%:%
%:%230=118%:%
%:%231=119%:%
%:%232=119%:%
%:%233=119%:%
%:%234=119%:%
%:%235=120%:%
%:%236=121%:%
%:%237=121%:%
%:%238=121%:%
%:%239=122%:%
%:%240=122%:%
%:%241=122%:%
%:%242=122%:%
%:%243=122%:%
%:%244=123%:%
%:%259=127%:%
%:%271=129%:%
%:%272=130%:%
%:%273=131%:%
%:%274=132%:%
%:%276=134%:%
%:%277=134%:%
%:%278=135%:%
%:%279=136%:%
%:%280=137%:%
%:%281=138%:%
%:%282=138%:%
%:%289=139%:%
%:%290=139%:%
%:%291=140%:%
%:%292=140%:%
%:%293=141%:%
%:%294=141%:%
%:%295=141%:%
%:%296=141%:%
%:%297=142%:%
%:%298=142%:%
%:%299=142%:%
%:%300=142%:%
%:%301=143%:%
%:%302=143%:%
%:%303=144%:%
%:%304=144%:%
%:%305=144%:%
%:%306=144%:%
%:%307=144%:%
%:%308=145%:%
%:%309=145%:%
%:%310=145%:%
%:%311=145%:%
%:%312=146%:%
%:%313=146%:%
%:%314=146%:%
%:%315=146%:%
%:%316=147%:%
%:%317=147%:%
%:%318=147%:%
%:%319=147%:%
%:%320=147%:%
%:%321=148%:%
%:%327=148%:%
%:%330=149%:%
%:%331=150%:%
%:%332=151%:%
%:%333=151%:%
%:%334=152%:%
%:%335=153%:%
%:%336=154%:%
%:%337=155%:%
%:%338=156%:%
%:%339=157%:%
%:%340=158%:%
%:%342=160%:%
%:%343=161%:%
%:%344=162%:%
%:%345=163%:%
%:%347=165%:%
%:%348=165%:%
%:%349=166%:%
%:%350=167%:%
%:%351=168%:%
%:%352=168%:%
%:%353=169%:%
%:%354=170%:%
%:%355=171%:%
%:%356=171%:%
%:%357=172%:%
%:%359=175%:%
%:%360=176%:%
%:%362=178%:%
%:%363=178%:%
%:%370=179%:%
%:%371=179%:%
%:%372=179%:%
%:%373=180%:%
%:%374=180%:%
%:%375=181%:%
%:%376=181%:%
%:%377=181%:%
%:%378=182%:%
%:%379=182%:%
%:%380=182%:%
%:%381=182%:%
%:%382=183%:%
%:%383=183%:%
%:%388=183%:%
%:%391=184%:%
%:%392=185%:%
%:%393=185%:%
%:%394=186%:%
%:%395=187%:%
%:%398=188%:%
%:%402=188%:%
%:%403=188%:%
%:%404=188%:%
%:%405=188%:%
%:%410=188%:%
%:%413=189%:%
%:%414=190%:%
%:%415=190%:%
%:%422=191%:%
%:%423=191%:%
%:%424=192%:%
%:%425=192%:%
%:%426=192%:%
%:%427=192%:%
%:%428=193%:%
%:%429=193%:%
%:%430=193%:%
%:%431=193%:%
%:%432=194%:%
%:%433=194%:%
%:%434=195%:%
%:%444=198%:%
%:%445=199%:%
%:%447=201%:%
%:%448=201%:%
%:%449=202%:%
%:%450=203%:%
%:%453=204%:%
%:%457=204%:%
%:%458=204%:%
%:%459=204%:%
%:%460=204%:%
%:%465=204%:%
%:%468=205%:%
%:%469=206%:%
%:%470=206%:%
%:%473=207%:%
%:%477=207%:%
%:%478=207%:%
%:%479=207%:%
%:%484=207%:%
%:%487=208%:%
%:%488=209%:%
%:%489=209%:%
%:%492=210%:%
%:%496=210%:%
%:%497=210%:%
%:%498=210%:%
%:%503=210%:%
%:%506=211%:%
%:%507=212%:%
%:%508=213%:%
%:%509=213%:%
%:%516=214%:%
%:%517=214%:%
%:%518=215%:%
%:%519=215%:%
%:%520=216%:%
%:%521=216%:%
%:%522=217%:%
%:%523=217%:%
%:%524=217%:%
%:%525=218%:%
%:%526=218%:%
%:%527=218%:%
%:%528=219%:%
%:%529=219%:%
%:%530=219%:%
%:%531=219%:%
%:%532=220%:%
%:%533=220%:%
%:%534=220%:%
%:%535=220%:%
%:%536=221%:%
%:%537=221%:%
%:%538=221%:%
%:%539=221%:%
%:%540=222%:%
%:%541=222%:%
%:%542=222%:%
%:%543=222%:%
%:%544=222%:%
%:%545=223%:%
%:%546=223%:%
%:%547=223%:%
%:%548=223%:%
%:%549=224%:%
%:%550=224%:%
%:%551=224%:%
%:%552=224%:%
%:%553=225%:%
%:%563=228%:%
%:%564=229%:%
%:%566=231%:%
%:%567=231%:%
%:%568=232%:%
%:%569=233%:%
%:%572=234%:%
%:%576=234%:%
%:%577=234%:%
%:%578=234%:%
%:%579=234%:%
%:%584=234%:%
%:%587=235%:%
%:%588=236%:%
%:%589=236%:%
%:%592=237%:%
%:%596=237%:%
%:%597=237%:%
%:%598=237%:%
%:%603=237%:%
%:%606=238%:%
%:%607=239%:%
%:%608=239%:%
%:%609=240%:%
%:%610=241%:%
%:%611=242%:%
%:%618=243%:%
%:%619=243%:%
%:%620=243%:%
%:%621=244%:%
%:%622=244%:%
%:%623=245%:%
%:%624=245%:%
%:%625=245%:%
%:%626=245%:%
%:%627=245%:%
%:%628=246%:%
%:%629=246%:%
%:%630=247%:%
%:%631=247%:%
%:%632=248%:%
%:%633=248%:%
%:%634=248%:%
%:%635=248%:%
%:%636=249%:%
%:%637=249%:%
%:%638=249%:%
%:%639=249%:%
%:%640=250%:%
%:%641=250%:%
%:%642=250%:%
%:%643=250%:%
%:%644=250%:%
%:%645=251%:%
%:%646=251%:%
%:%647=251%:%
%:%648=251%:%
%:%649=251%:%
%:%650=252%:%
%:%656=252%:%
%:%659=253%:%
%:%660=254%:%
%:%661=254%:%
%:%662=255%:%
%:%663=256%:%
%:%670=257%:%
%:%671=257%:%
%:%672=258%:%
%:%673=258%:%
%:%674=259%:%
%:%675=259%:%
%:%676=260%:%
%:%677=260%:%
%:%678=261%:%
%:%679=261%:%
%:%680=262%:%
%:%681=262%:%
%:%682=262%:%
%:%683=263%:%
%:%684=263%:%
%:%685=263%:%
%:%686=264%:%
%:%687=264%:%
%:%688=265%:%
%:%689=265%:%
%:%690=265%:%
%:%691=265%:%
%:%692=265%:%
%:%693=266%:%
%:%694=266%:%
%:%695=266%:%
%:%696=266%:%
%:%697=267%:%
%:%698=267%:%
%:%699=268%:%
%:%700=268%:%
%:%701=269%:%
%:%702=269%:%
%:%703=269%:%
%:%704=269%:%
%:%705=270%:%
%:%706=270%:%
%:%707=271%:%
%:%708=271%:%
%:%709=272%:%
%:%710=272%:%
%:%711=272%:%
%:%712=272%:%
%:%713=272%:%
%:%714=273%:%
%:%715=273%:%
%:%716=274%:%
%:%726=277%:%
%:%727=278%:%
%:%729=279%:%
%:%730=279%:%
%:%731=280%:%
%:%732=281%:%
%:%739=282%:%
%:%740=282%:%
%:%741=283%:%
%:%742=283%:%
%:%743=284%:%
%:%744=284%:%
%:%745=284%:%
%:%746=284%:%
%:%747=285%:%
%:%748=285%:%
%:%749=285%:%
%:%750=285%:%
%:%751=285%:%
%:%752=286%:%
%:%753=286%:%
%:%754=286%:%
%:%755=287%:%
%:%756=288%:%
%:%757=288%:%
%:%758=288%:%
%:%759=289%:%
%:%760=289%:%
%:%761=289%:%
%:%762=290%:%
%:%763=290%:%
%:%764=291%:%
%:%765=291%:%
%:%766=292%:%
%:%767=292%:%
%:%768=293%:%
%:%769=293%:%
%:%770=293%:%
%:%771=293%:%
%:%772=294%:%
%:%773=294%:%
%:%774=294%:%
%:%775=294%:%
%:%776=295%:%
%:%777=295%:%
%:%778=295%:%
%:%779=295%:%
%:%780=296%:%
%:%781=296%:%
%:%782=296%:%
%:%783=296%:%
%:%784=296%:%
%:%785=297%:%
%:%786=297%:%
%:%787=298%:%
%:%788=298%:%
%:%789=298%:%
%:%790=299%:%
%:%791=299%:%
%:%792=300%:%
%:%793=300%:%
%:%794=301%:%
%:%795=301%:%
%:%796=302%:%
%:%797=302%:%
%:%798=303%:%
%:%799=303%:%
%:%800=304%:%
%:%801=304%:%
%:%802=305%:%
%:%803=305%:%
%:%804=306%:%
%:%805=306%:%
%:%806=306%:%
%:%807=306%:%
%:%808=307%:%
%:%809=307%:%
%:%810=307%:%
%:%811=307%:%
%:%812=308%:%
%:%813=308%:%
%:%814=309%:%
%:%815=309%:%
%:%816=309%:%
%:%817=309%:%
%:%818=309%:%
%:%819=310%:%
%:%820=310%:%
%:%821=310%:%
%:%822=310%:%
%:%823=311%:%
%:%824=311%:%
%:%825=312%:%
%:%826=312%:%
%:%827=312%:%
%:%828=312%:%
%:%829=313%:%
%:%839=316%:%
%:%840=317%:%
%:%842=319%:%
%:%843=319%:%
%:%847=323%:%
%:%848=324%:%
%:%849=325%:%
%:%850=325%:%
%:%857=326%:%
%:%858=326%:%
%:%859=327%:%
%:%860=327%:%
%:%861=327%:%
%:%862=328%:%
%:%863=328%:%
%:%864=328%:%
%:%865=329%:%
%:%866=329%:%
%:%867=329%:%
%:%868=330%:%
%:%869=331%:%
%:%870=331%:%
%:%871=332%:%
%:%872=332%:%
%:%873=332%:%
%:%874=333%:%
%:%875=334%:%
%:%876=334%:%
%:%877=334%:%
%:%878=335%:%
%:%879=335%:%
%:%880=336%:%
%:%881=337%:%
%:%882=337%:%
%:%883=337%:%
%:%884=338%:%
%:%894=341%:%
%:%895=342%:%
%:%896=343%:%
%:%898=345%:%
%:%899=345%:%
%:%900=346%:%
%:%901=347%:%
%:%902=347%:%
%:%903=348%:%
%:%904=349%:%
%:%911=350%:%
%:%912=350%:%
%:%913=350%:%
%:%914=351%:%
%:%915=351%:%
%:%916=352%:%
%:%917=352%:%
%:%918=353%:%
%:%919=353%:%
%:%920=353%:%
%:%921=353%:%
%:%922=354%:%
%:%923=354%:%
%:%924=354%:%
%:%925=355%:%
%:%926=355%:%
%:%927=355%:%
%:%928=356%:%
%:%929=356%:%
%:%930=356%:%
%:%931=357%:%
%:%932=357%:%
%:%933=358%:%
%:%934=358%:%
%:%935=358%:%
%:%936=359%:%
%:%937=359%:%
%:%938=359%:%
%:%939=360%:%
%:%940=360%:%
%:%941=360%:%
%:%942=361%:%
%:%943=361%:%
%:%944=361%:%
%:%945=361%:%
%:%946=362%:%
%:%947=362%:%
%:%948=362%:%
%:%949=362%:%
%:%950=363%:%
%:%951=363%:%
%:%952=363%:%
%:%953=363%:%
%:%954=363%:%
%:%955=363%:%
%:%956=364%:%
%:%957=364%:%
%:%958=364%:%
%:%959=364%:%
%:%960=364%:%
%:%961=365%:%
%:%967=365%:%
%:%970=366%:%
%:%971=367%:%
%:%972=367%:%
%:%973=368%:%
%:%974=369%:%
%:%975=370%:%
%:%982=371%:%
%:%983=371%:%
%:%984=372%:%
%:%985=372%:%
%:%986=373%:%
%:%987=373%:%
%:%988=374%:%
%:%989=374%:%
%:%990=375%:%
%:%991=375%:%
%:%992=375%:%
%:%993=376%:%
%:%994=376%:%
%:%995=376%:%
%:%996=377%:%
%:%997=377%:%
%:%998=377%:%
%:%999=378%:%
%:%1000=378%:%
%:%1001=378%:%
%:%1002=379%:%
%:%1003=379%:%
%:%1004=379%:%
%:%1005=379%:%
%:%1006=380%:%
%:%1007=380%:%
%:%1008=380%:%
%:%1009=381%:%
%:%1010=381%:%
%:%1011=381%:%
%:%1012=381%:%
%:%1013=382%:%
%:%1014=382%:%
%:%1015=383%:%
%:%1016=383%:%
%:%1017=384%:%
%:%1018=384%:%
%:%1019=384%:%
%:%1020=384%:%
%:%1021=384%:%
%:%1022=385%:%
%:%1023=385%:%
%:%1024=385%:%
%:%1025=385%:%
%:%1026=385%:%
%:%1027=386%:%
%:%1028=386%:%
%:%1029=387%:%
%:%1030=387%:%
%:%1031=387%:%
%:%1032=387%:%
%:%1033=388%:%
%:%1034=388%:%
%:%1035=388%:%
%:%1036=388%:%
%:%1037=389%:%
%:%1043=389%:%
%:%1046=390%:%
%:%1047=391%:%
%:%1048=391%:%
%:%1049=392%:%
%:%1050=393%:%
%:%1051=394%:%
%:%1058=395%:%
%:%1059=395%:%
%:%1060=396%:%
%:%1061=396%:%
%:%1062=397%:%
%:%1063=397%:%
%:%1064=397%:%
%:%1065=397%:%
%:%1066=398%:%
%:%1067=398%:%
%:%1068=398%:%
%:%1069=398%:%
%:%1070=398%:%
%:%1071=399%:%
%:%1072=399%:%
%:%1073=399%:%
%:%1074=399%:%
%:%1075=399%:%
%:%1076=399%:%
%:%1077=400%:%
%:%1078=400%:%
%:%1079=401%:%
%:%1080=401%:%
%:%1081=402%:%
%:%1082=402%:%
%:%1083=402%:%
%:%1084=402%:%
%:%1085=403%:%
%:%1095=406%:%
%:%1096=407%:%
%:%1097=408%:%
%:%1099=409%:%
%:%1100=409%:%
%:%1101=410%:%
%:%1102=411%:%
%:%1109=412%:%
%:%1110=412%:%
%:%1111=413%:%
%:%1112=413%:%
%:%1113=413%:%
%:%1114=413%:%
%:%1115=414%:%
%:%1116=414%:%
%:%1117=414%:%
%:%1118=414%:%
%:%1119=415%:%
%:%1120=415%:%
%:%1121=415%:%
%:%1122=415%:%
%:%1123=415%:%
%:%1124=416%:%
%:%1125=416%:%
%:%1126=416%:%
%:%1127=416%:%
%:%1128=417%:%
%:%1129=417%:%
%:%1130=418%:%
%:%1131=418%:%
%:%1132=418%:%
%:%1133=419%:%
%:%1134=419%:%
%:%1135=419%:%
%:%1136=419%:%
%:%1137=420%:%
%:%1138=420%:%
%:%1139=420%:%
%:%1140=420%:%
%:%1141=421%:%
%:%1147=421%:%
%:%1150=422%:%
%:%1151=423%:%
%:%1159=426%:%
%:%1171=428%:%
%:%1172=429%:%
%:%1173=430%:%
%:%1174=431%:%
%:%1175=432%:%
%:%1176=433%:%
%:%1177=434%:%
%:%1179=435%:%
%:%1180=435%:%
%:%1181=436%:%
%:%1182=437%:%
%:%1189=438%:%
%:%1190=438%:%
%:%1191=439%:%
%:%1192=439%:%
%:%1193=440%:%
%:%1194=440%:%
%:%1195=441%:%
%:%1196=441%:%
%:%1197=441%:%
%:%1198=442%:%
%:%1199=442%:%
%:%1200=442%:%
%:%1201=443%:%
%:%1202=443%:%
%:%1203=443%:%
%:%1204=444%:%
%:%1205=444%:%
%:%1206=445%:%
%:%1207=445%:%
%:%1208=446%:%
%:%1209=446%:%
%:%1210=446%:%
%:%1211=447%:%
%:%1212=447%:%
%:%1213=447%:%
%:%1214=448%:%
%:%1215=448%:%
%:%1216=448%:%
%:%1217=448%:%
%:%1218=449%:%
%:%1219=449%:%
%:%1220=449%:%
%:%1221=450%:%
%:%1222=450%:%
%:%1223=450%:%
%:%1224=450%:%
%:%1225=451%:%
%:%1226=451%:%
%:%1227=451%:%
%:%1228=451%:%
%:%1229=451%:%
%:%1230=452%:%
%:%1231=452%:%
%:%1232=453%:%
%:%1233=453%:%
%:%1234=453%:%
%:%1235=454%:%
%:%1236=454%:%
%:%1237=455%:%
%:%1247=457%:%
%:%1248=458%:%
%:%1249=459%:%
%:%1251=460%:%
%:%1252=460%:%
%:%1253=461%:%
%:%1254=462%:%
%:%1261=463%:%
%:%1262=463%:%
%:%1263=463%:%
%:%1264=464%:%
%:%1265=464%:%
%:%1266=464%:%
%:%1267=465%:%
%:%1268=465%:%
%:%1269=466%:%
%:%1270=466%:%
%:%1271=467%:%
%:%1272=467%:%
%:%1273=468%:%
%:%1274=468%:%
%:%1275=469%:%
%:%1276=469%:%
%:%1277=470%:%
%:%1278=470%:%
%:%1279=471%:%
%:%1280=471%:%
%:%1281=471%:%
%:%1282=472%:%
%:%1283=472%:%
%:%1284=473%:%
%:%1285=473%:%
%:%1286=473%:%
%:%1287=473%:%
%:%1288=473%:%
%:%1289=474%:%
%:%1290=474%:%
%:%1291=474%:%
%:%1292=475%:%
%:%1293=475%:%
%:%1294=475%:%
%:%1295=476%:%
%:%1296=476%:%
%:%1297=477%:%
%:%1307=479%:%
%:%1309=480%:%
%:%1310=480%:%
%:%1311=481%:%
%:%1312=482%:%
%:%1313=483%:%
%:%1320=484%:%
%:%1321=484%:%
%:%1322=485%:%
%:%1323=485%:%
%:%1324=485%:%
%:%1325=486%:%
%:%1326=486%:%
%:%1327=486%:%
%:%1328=487%:%
%:%1329=487%:%
%:%1330=487%:%
%:%1331=487%:%
%:%1332=488%:%
%:%1333=488%:%
%:%1334=488%:%
%:%1335=489%:%
%:%1336=489%:%
%:%1337=489%:%
%:%1338=489%:%
%:%1339=490%:%
%:%1345=490%:%
%:%1350=491%:%
%:%1355=492%:%

%
\begin{isabellebody}%
\setisabellecontext{Pilling}%
%
\isadelimdocument
%
\endisadelimdocument
%
\isatagdocument
%
\isamarkupsection{Pilling's proof of Parikh's theorem%
}
\isamarkuptrue%
%
\endisatagdocument
{\isafolddocument}%
%
\isadelimdocument
%
\endisadelimdocument
%
\isadelimtheory
%
\endisadelimtheory
%
\isatagtheory
\isakeywordONE{theory}\isamarkupfalse%
\ Pilling\isanewline
\ \ \isakeywordTWO{imports}\ \isanewline
\ \ \ \ {\isachardoublequoteopen}Reg{\isacharunderscore}{\kern0pt}Lang{\isacharunderscore}{\kern0pt}Exp{\isacharunderscore}{\kern0pt}Eqns{\isachardoublequoteclose}\isanewline
\isakeywordTWO{begin}%
\endisatagtheory
{\isafoldtheory}%
%
\isadelimtheory
%
\endisadelimtheory
%
\begin{isamarkuptext}%
We prove Parikh's theorem, closely following Pilling's proof \cite{Pilling}. The rough
idea is as follows: As seen above, each CFG can be interpreted as a system of \isa{\isaconst{reg{\isacharunderscore}{\kern0pt}eval}}
equations of the first type and we can easily convert it into a system of the second type by
applying the Parikh image on both sides of each equation. Pilling now shows that there is a
regular solution to this system and that this solution is furthermore minimal.
Using the relations explored in the last section we prove that the CFG's language is a minimal
solution of the same sytem and hence that the Parikh image of the CFG's language and of the regular
solution must be identical; this finishes the proof of Parikh's theorem.

Notably, while in \cite{Pilling} Pilling proves an auxiliary lemma first and applies this lemma in
the proof of the main theorem, we were able to complete the whole proof without using the lemma.%
\end{isamarkuptext}\isamarkuptrue%
%
\isadelimdocument
%
\endisadelimdocument
%
\isatagdocument
%
\isamarkupsubsection{Special representation of regular language expressions%
}
\isamarkuptrue%
%
\endisatagdocument
{\isafolddocument}%
%
\isadelimdocument
%
\endisadelimdocument
%
\begin{isamarkuptext}%
To each \isa{\isaconst{reg{\isacharunderscore}{\kern0pt}eval}} regular language expression and variable \isa{x} corresponds a second
regular language expression with the same Parikh image and of the form depicted in equation (3) in
\cite{Pilling}. We call regular language expressions of this form "bipartite regular language
expressions" since they decompose into two subexpressions where one of them contains the variable
\isa{x} and the other one does not:%
\end{isamarkuptext}\isamarkuptrue%
\isakeywordONE{definition}\isamarkupfalse%
\ bipart{\isacharunderscore}{\kern0pt}rlexp\ {\isacharcolon}{\kern0pt}{\isacharcolon}{\kern0pt}\ {\isachardoublequoteopen}nat\ {\isasymRightarrow}\ {\isacharprime}{\kern0pt}a\ rlexp\ {\isasymRightarrow}\ bool{\isachardoublequoteclose}\ \isakeywordTWO{where}\isanewline
\ \ {\isachardoublequoteopen}bipart{\isacharunderscore}{\kern0pt}rlexp\ x\ f\ {\isasymequiv}\ {\isasymexists}p\ q{\isachardot}{\kern0pt}\ reg{\isacharunderscore}{\kern0pt}eval\ p\ {\isasymand}\ reg{\isacharunderscore}{\kern0pt}eval\ q\ {\isasymand}\isanewline
\ \ \ \ f\ {\isacharequal}{\kern0pt}\ Union\ p\ {\isacharparenleft}{\kern0pt}Concat\ q\ {\isacharparenleft}{\kern0pt}Var\ x{\isacharparenright}{\kern0pt}{\isacharparenright}{\kern0pt}\ {\isasymand}\ x\ {\isasymnotin}\ vars\ p{\isachardoublequoteclose}%
\begin{isamarkuptext}%
All bipartite regular language expressions evaluate to regular languages. Additionally,
for each \isa{\isaconst{reg{\isacharunderscore}{\kern0pt}eval}} regular language expression and variable \isa{x}, there exists a bipartite
regular language expression with identical Parikh image and almost identical set of variables.
While the first proof is simple, the second one is more complex and needs the results of the
sections 2.3 and 2.4:%
\end{isamarkuptext}\isamarkuptrue%
\isakeywordONE{lemma}\isamarkupfalse%
\ {\isachardoublequoteopen}bipart{\isacharunderscore}{\kern0pt}rlexp\ x\ f\ {\isasymLongrightarrow}\ reg{\isacharunderscore}{\kern0pt}eval\ f{\isachardoublequoteclose}\isanewline
%
\isadelimproof
\ \ %
\endisadelimproof
%
\isatagproof
\isakeywordONE{unfolding}\isamarkupfalse%
\ bipart{\isacharunderscore}{\kern0pt}rlexp{\isacharunderscore}{\kern0pt}def\ \isakeywordONE{by}\isamarkupfalse%
\ fastforce%
\endisatagproof
{\isafoldproof}%
%
\isadelimproof
\isanewline
%
\endisadelimproof
\isanewline
\isanewline
\isakeywordONE{lemma}\isamarkupfalse%
\ reg{\isacharunderscore}{\kern0pt}eval{\isacharunderscore}{\kern0pt}bipart{\isacharunderscore}{\kern0pt}rlexp{\isacharunderscore}{\kern0pt}Variable{\isacharcolon}{\kern0pt}\ {\isachardoublequoteopen}{\isasymexists}f{\isacharprime}{\kern0pt}{\isachardot}{\kern0pt}\ bipart{\isacharunderscore}{\kern0pt}rlexp\ x\ f{\isacharprime}{\kern0pt}\ {\isasymand}\ vars\ f{\isacharprime}{\kern0pt}\ {\isacharequal}{\kern0pt}\ vars\ {\isacharparenleft}{\kern0pt}Var\ y{\isacharparenright}{\kern0pt}\ {\isasymunion}\ {\isacharbraceleft}{\kern0pt}x{\isacharbraceright}{\kern0pt}\isanewline
\ \ \ \ \ \ \ \ \ \ \ \ \ \ \ \ \ \ \ \ \ \ \ \ \ \ \ \ \ \ \ \ \ \ \ \ \ \ \ \ {\isasymand}\ {\isacharparenleft}{\kern0pt}{\isasymforall}v{\isachardot}{\kern0pt}\ {\isasymPsi}\ {\isacharparenleft}{\kern0pt}eval\ {\isacharparenleft}{\kern0pt}Var\ y{\isacharparenright}{\kern0pt}\ v{\isacharparenright}{\kern0pt}\ {\isacharequal}{\kern0pt}\ {\isasymPsi}\ {\isacharparenleft}{\kern0pt}eval\ f{\isacharprime}{\kern0pt}\ v{\isacharparenright}{\kern0pt}{\isacharparenright}{\kern0pt}{\isachardoublequoteclose}\isanewline
%
\isadelimproof
%
\endisadelimproof
%
\isatagproof
\isakeywordONE{proof}\isamarkupfalse%
\ {\isacharparenleft}{\kern0pt}cases\ {\isachardoublequoteopen}x\ {\isacharequal}{\kern0pt}\ y{\isachardoublequoteclose}{\isacharparenright}{\kern0pt}\isanewline
\isakeywordONE{let}\isamarkupfalse%
\ {\isacharquery}{\kern0pt}f{\isacharprime}{\kern0pt}\ {\isacharequal}{\kern0pt}\ {\isachardoublequoteopen}Union\ {\isacharparenleft}{\kern0pt}Const\ {\isacharbraceleft}{\kern0pt}{\isacharbraceright}{\kern0pt}{\isacharparenright}{\kern0pt}\ {\isacharparenleft}{\kern0pt}Concat\ {\isacharparenleft}{\kern0pt}Const\ {\isacharbraceleft}{\kern0pt}{\isacharbrackleft}{\kern0pt}{\isacharbrackright}{\kern0pt}{\isacharbraceright}{\kern0pt}{\isacharparenright}{\kern0pt}\ {\isacharparenleft}{\kern0pt}Var\ x{\isacharparenright}{\kern0pt}{\isacharparenright}{\kern0pt}{\isachardoublequoteclose}\isanewline
\ \ \isakeywordTHREE{case}\isamarkupfalse%
\ True\isanewline
\ \ \isakeywordONE{then}\isamarkupfalse%
\ \isakeywordONE{have}\isamarkupfalse%
\ {\isachardoublequoteopen}bipart{\isacharunderscore}{\kern0pt}rlexp\ x\ {\isacharquery}{\kern0pt}f{\isacharprime}{\kern0pt}{\isachardoublequoteclose}\isanewline
\ \ \ \ \isakeywordONE{unfolding}\isamarkupfalse%
\ bipart{\isacharunderscore}{\kern0pt}rlexp{\isacharunderscore}{\kern0pt}def\ \isakeywordONE{using}\isamarkupfalse%
\ emptyset{\isacharunderscore}{\kern0pt}regular\ epsilon{\isacharunderscore}{\kern0pt}regular\ \isakeywordONE{by}\isamarkupfalse%
\ fastforce\isanewline
\ \ \isakeywordONE{moreover}\isamarkupfalse%
\ \isakeywordONE{have}\isamarkupfalse%
\ {\isachardoublequoteopen}eval\ {\isacharquery}{\kern0pt}f{\isacharprime}{\kern0pt}\ v\ {\isacharequal}{\kern0pt}\ eval\ {\isacharparenleft}{\kern0pt}Var\ y{\isacharparenright}{\kern0pt}\ v{\isachardoublequoteclose}\ \isakeywordTWO{for}\ v\ {\isacharcolon}{\kern0pt}{\isacharcolon}{\kern0pt}\ {\isachardoublequoteopen}{\isacharprime}{\kern0pt}a\ valuation{\isachardoublequoteclose}\ \isakeywordONE{using}\isamarkupfalse%
\ True\ \isakeywordONE{by}\isamarkupfalse%
\ simp\isanewline
\ \ \isakeywordONE{moreover}\isamarkupfalse%
\ \isakeywordONE{have}\isamarkupfalse%
\ {\isachardoublequoteopen}vars\ {\isacharquery}{\kern0pt}f{\isacharprime}{\kern0pt}\ {\isacharequal}{\kern0pt}\ vars\ {\isacharparenleft}{\kern0pt}Var\ y{\isacharparenright}{\kern0pt}\ {\isasymunion}\ {\isacharbraceleft}{\kern0pt}x{\isacharbraceright}{\kern0pt}{\isachardoublequoteclose}\ \isakeywordONE{using}\isamarkupfalse%
\ True\ \isakeywordONE{by}\isamarkupfalse%
\ simp\isanewline
\ \ \isakeywordONE{ultimately}\isamarkupfalse%
\ \isakeywordTHREE{show}\isamarkupfalse%
\ {\isacharquery}{\kern0pt}thesis\ \isakeywordONE{by}\isamarkupfalse%
\ metis\isanewline
\isakeywordONE{next}\isamarkupfalse%
\isanewline
\ \ \isakeywordONE{let}\isamarkupfalse%
\ {\isacharquery}{\kern0pt}f{\isacharprime}{\kern0pt}\ {\isacharequal}{\kern0pt}\ {\isachardoublequoteopen}Union\ {\isacharparenleft}{\kern0pt}Var\ y{\isacharparenright}{\kern0pt}\ {\isacharparenleft}{\kern0pt}Concat\ {\isacharparenleft}{\kern0pt}Const\ {\isacharbraceleft}{\kern0pt}{\isacharbraceright}{\kern0pt}{\isacharparenright}{\kern0pt}\ {\isacharparenleft}{\kern0pt}Var\ x{\isacharparenright}{\kern0pt}{\isacharparenright}{\kern0pt}{\isachardoublequoteclose}\isanewline
\ \ \isakeywordTHREE{case}\isamarkupfalse%
\ False\isanewline
\ \ \isakeywordONE{then}\isamarkupfalse%
\ \isakeywordONE{have}\isamarkupfalse%
\ {\isachardoublequoteopen}bipart{\isacharunderscore}{\kern0pt}rlexp\ x\ {\isacharquery}{\kern0pt}f{\isacharprime}{\kern0pt}{\isachardoublequoteclose}\isanewline
\ \ \ \ \isakeywordONE{unfolding}\isamarkupfalse%
\ bipart{\isacharunderscore}{\kern0pt}rlexp{\isacharunderscore}{\kern0pt}def\ \isakeywordONE{using}\isamarkupfalse%
\ emptyset{\isacharunderscore}{\kern0pt}regular\ epsilon{\isacharunderscore}{\kern0pt}regular\ \isakeywordONE{by}\isamarkupfalse%
\ fastforce\isanewline
\ \ \isakeywordONE{moreover}\isamarkupfalse%
\ \isakeywordONE{have}\isamarkupfalse%
\ {\isachardoublequoteopen}eval\ {\isacharquery}{\kern0pt}f{\isacharprime}{\kern0pt}\ v\ {\isacharequal}{\kern0pt}\ eval\ {\isacharparenleft}{\kern0pt}Var\ y{\isacharparenright}{\kern0pt}\ v{\isachardoublequoteclose}\ \isakeywordTWO{for}\ v\ {\isacharcolon}{\kern0pt}{\isacharcolon}{\kern0pt}\ {\isachardoublequoteopen}{\isacharprime}{\kern0pt}a\ valuation{\isachardoublequoteclose}\ \isakeywordONE{using}\isamarkupfalse%
\ False\ \isakeywordONE{by}\isamarkupfalse%
\ simp\isanewline
\ \ \isakeywordONE{moreover}\isamarkupfalse%
\ \isakeywordONE{have}\isamarkupfalse%
\ {\isachardoublequoteopen}vars\ {\isacharquery}{\kern0pt}f{\isacharprime}{\kern0pt}\ {\isacharequal}{\kern0pt}\ vars\ {\isacharparenleft}{\kern0pt}Var\ y{\isacharparenright}{\kern0pt}\ {\isasymunion}\ {\isacharbraceleft}{\kern0pt}x{\isacharbraceright}{\kern0pt}{\isachardoublequoteclose}\ \isakeywordONE{by}\isamarkupfalse%
\ simp\isanewline
\ \ \isakeywordONE{ultimately}\isamarkupfalse%
\ \isakeywordTHREE{show}\isamarkupfalse%
\ {\isacharquery}{\kern0pt}thesis\ \isakeywordONE{by}\isamarkupfalse%
\ metis\isanewline
\isakeywordONE{qed}\isamarkupfalse%
%
\endisatagproof
{\isafoldproof}%
%
\isadelimproof
\isanewline
%
\endisadelimproof
\isanewline
\isakeywordONE{lemma}\isamarkupfalse%
\ reg{\isacharunderscore}{\kern0pt}eval{\isacharunderscore}{\kern0pt}bipart{\isacharunderscore}{\kern0pt}rlexp{\isacharunderscore}{\kern0pt}Const{\isacharcolon}{\kern0pt}\isanewline
\ \ \isakeywordTWO{assumes}\ {\isachardoublequoteopen}regular{\isacharunderscore}{\kern0pt}lang\ l{\isachardoublequoteclose}\isanewline
\ \ \ \ \isakeywordTWO{shows}\ {\isachardoublequoteopen}{\isasymexists}f{\isacharprime}{\kern0pt}{\isachardot}{\kern0pt}\ bipart{\isacharunderscore}{\kern0pt}rlexp\ x\ f{\isacharprime}{\kern0pt}\ {\isasymand}\ vars\ f{\isacharprime}{\kern0pt}\ {\isacharequal}{\kern0pt}\ vars\ {\isacharparenleft}{\kern0pt}Const\ l{\isacharparenright}{\kern0pt}\ {\isasymunion}\ {\isacharbraceleft}{\kern0pt}x{\isacharbraceright}{\kern0pt}\isanewline
\ \ \ \ \ \ \ \ \ \ \ \ \ \ \ \ {\isasymand}\ {\isacharparenleft}{\kern0pt}{\isasymforall}v{\isachardot}{\kern0pt}\ {\isasymPsi}\ {\isacharparenleft}{\kern0pt}eval\ {\isacharparenleft}{\kern0pt}Const\ l{\isacharparenright}{\kern0pt}\ v{\isacharparenright}{\kern0pt}\ {\isacharequal}{\kern0pt}\ {\isasymPsi}\ {\isacharparenleft}{\kern0pt}eval\ f{\isacharprime}{\kern0pt}\ v{\isacharparenright}{\kern0pt}{\isacharparenright}{\kern0pt}{\isachardoublequoteclose}\isanewline
%
\isadelimproof
%
\endisadelimproof
%
\isatagproof
\isakeywordONE{proof}\isamarkupfalse%
\ {\isacharminus}{\kern0pt}\isanewline
\ \ \isakeywordONE{let}\isamarkupfalse%
\ {\isacharquery}{\kern0pt}f{\isacharprime}{\kern0pt}\ {\isacharequal}{\kern0pt}\ {\isachardoublequoteopen}Union\ {\isacharparenleft}{\kern0pt}Const\ l{\isacharparenright}{\kern0pt}\ {\isacharparenleft}{\kern0pt}Concat\ {\isacharparenleft}{\kern0pt}Const\ {\isacharbraceleft}{\kern0pt}{\isacharbraceright}{\kern0pt}{\isacharparenright}{\kern0pt}\ {\isacharparenleft}{\kern0pt}Var\ x{\isacharparenright}{\kern0pt}{\isacharparenright}{\kern0pt}{\isachardoublequoteclose}\isanewline
\ \ \isakeywordONE{have}\isamarkupfalse%
\ {\isachardoublequoteopen}bipart{\isacharunderscore}{\kern0pt}rlexp\ x\ {\isacharquery}{\kern0pt}f{\isacharprime}{\kern0pt}{\isachardoublequoteclose}\isanewline
\ \ \ \ \isakeywordONE{unfolding}\isamarkupfalse%
\ bipart{\isacharunderscore}{\kern0pt}rlexp{\isacharunderscore}{\kern0pt}def\ \isakeywordONE{using}\isamarkupfalse%
\ assms\ emptyset{\isacharunderscore}{\kern0pt}regular\ \isakeywordONE{by}\isamarkupfalse%
\ simp\isanewline
\ \ \isakeywordONE{moreover}\isamarkupfalse%
\ \isakeywordONE{have}\isamarkupfalse%
\ {\isachardoublequoteopen}eval\ {\isacharquery}{\kern0pt}f{\isacharprime}{\kern0pt}\ v\ {\isacharequal}{\kern0pt}\ eval\ {\isacharparenleft}{\kern0pt}Const\ l{\isacharparenright}{\kern0pt}\ v{\isachardoublequoteclose}\ \isakeywordTWO{for}\ v\ {\isacharcolon}{\kern0pt}{\isacharcolon}{\kern0pt}\ {\isachardoublequoteopen}{\isacharprime}{\kern0pt}a\ valuation{\isachardoublequoteclose}\ \isakeywordONE{by}\isamarkupfalse%
\ simp\isanewline
\ \ \isakeywordONE{moreover}\isamarkupfalse%
\ \isakeywordONE{have}\isamarkupfalse%
\ {\isachardoublequoteopen}vars\ {\isacharquery}{\kern0pt}f{\isacharprime}{\kern0pt}\ {\isacharequal}{\kern0pt}\ vars\ {\isacharparenleft}{\kern0pt}Const\ l{\isacharparenright}{\kern0pt}\ {\isasymunion}\ {\isacharbraceleft}{\kern0pt}x{\isacharbraceright}{\kern0pt}{\isachardoublequoteclose}\ \isakeywordONE{by}\isamarkupfalse%
\ simp\ \isanewline
\ \ \isakeywordONE{ultimately}\isamarkupfalse%
\ \isakeywordTHREE{show}\isamarkupfalse%
\ {\isacharquery}{\kern0pt}thesis\ \isakeywordONE{by}\isamarkupfalse%
\ metis\isanewline
\isakeywordONE{qed}\isamarkupfalse%
%
\endisatagproof
{\isafoldproof}%
%
\isadelimproof
\isanewline
%
\endisadelimproof
\isanewline
\isakeywordONE{lemma}\isamarkupfalse%
\ reg{\isacharunderscore}{\kern0pt}eval{\isacharunderscore}{\kern0pt}bipart{\isacharunderscore}{\kern0pt}rlexp{\isacharunderscore}{\kern0pt}Union{\isacharcolon}{\kern0pt}\isanewline
\ \ \isakeywordTWO{assumes}\ {\isachardoublequoteopen}{\isasymexists}f{\isacharprime}{\kern0pt}{\isachardot}{\kern0pt}\ bipart{\isacharunderscore}{\kern0pt}rlexp\ x\ f{\isacharprime}{\kern0pt}\ {\isasymand}\ vars\ f{\isacharprime}{\kern0pt}\ {\isacharequal}{\kern0pt}\ vars\ f{\isadigit{1}}\ {\isasymunion}\ {\isacharbraceleft}{\kern0pt}x{\isacharbraceright}{\kern0pt}\ {\isasymand}\isanewline
\ \ \ \ \ \ \ \ \ \ \ \ \ \ \ \ {\isacharparenleft}{\kern0pt}{\isasymforall}v{\isachardot}{\kern0pt}\ {\isasymPsi}\ {\isacharparenleft}{\kern0pt}eval\ f{\isadigit{1}}\ v{\isacharparenright}{\kern0pt}\ {\isacharequal}{\kern0pt}\ {\isasymPsi}\ {\isacharparenleft}{\kern0pt}eval\ f{\isacharprime}{\kern0pt}\ v{\isacharparenright}{\kern0pt}{\isacharparenright}{\kern0pt}{\isachardoublequoteclose}\isanewline
\ \ \ \ \ \ \ \ \ \ {\isachardoublequoteopen}{\isasymexists}f{\isacharprime}{\kern0pt}{\isachardot}{\kern0pt}\ bipart{\isacharunderscore}{\kern0pt}rlexp\ x\ f{\isacharprime}{\kern0pt}\ {\isasymand}\ vars\ f{\isacharprime}{\kern0pt}\ {\isacharequal}{\kern0pt}\ vars\ f{\isadigit{2}}\ {\isasymunion}\ {\isacharbraceleft}{\kern0pt}x{\isacharbraceright}{\kern0pt}\ {\isasymand}\isanewline
\ \ \ \ \ \ \ \ \ \ \ \ \ \ \ \ {\isacharparenleft}{\kern0pt}{\isasymforall}v{\isachardot}{\kern0pt}\ {\isasymPsi}\ {\isacharparenleft}{\kern0pt}eval\ f{\isadigit{2}}\ v{\isacharparenright}{\kern0pt}\ {\isacharequal}{\kern0pt}\ {\isasymPsi}\ {\isacharparenleft}{\kern0pt}eval\ f{\isacharprime}{\kern0pt}\ v{\isacharparenright}{\kern0pt}{\isacharparenright}{\kern0pt}{\isachardoublequoteclose}\isanewline
\ \ \ \ \isakeywordTWO{shows}\ {\isachardoublequoteopen}{\isasymexists}f{\isacharprime}{\kern0pt}{\isachardot}{\kern0pt}\ bipart{\isacharunderscore}{\kern0pt}rlexp\ x\ f{\isacharprime}{\kern0pt}\ {\isasymand}\ vars\ f{\isacharprime}{\kern0pt}\ {\isacharequal}{\kern0pt}\ vars\ {\isacharparenleft}{\kern0pt}Union\ f{\isadigit{1}}\ f{\isadigit{2}}{\isacharparenright}{\kern0pt}\ {\isasymunion}\ {\isacharbraceleft}{\kern0pt}x{\isacharbraceright}{\kern0pt}\ {\isasymand}\isanewline
\ \ \ \ \ \ \ \ \ \ \ \ \ \ \ \ {\isacharparenleft}{\kern0pt}{\isasymforall}v{\isachardot}{\kern0pt}\ {\isasymPsi}\ {\isacharparenleft}{\kern0pt}eval\ {\isacharparenleft}{\kern0pt}Union\ f{\isadigit{1}}\ f{\isadigit{2}}{\isacharparenright}{\kern0pt}\ v{\isacharparenright}{\kern0pt}\ {\isacharequal}{\kern0pt}\ {\isasymPsi}\ {\isacharparenleft}{\kern0pt}eval\ f{\isacharprime}{\kern0pt}\ v{\isacharparenright}{\kern0pt}{\isacharparenright}{\kern0pt}{\isachardoublequoteclose}\isanewline
%
\isadelimproof
%
\endisadelimproof
%
\isatagproof
\isakeywordONE{proof}\isamarkupfalse%
\ {\isacharminus}{\kern0pt}\isanewline
\ \ \isakeywordONE{from}\isamarkupfalse%
\ assms\ \isakeywordTHREE{obtain}\isamarkupfalse%
\ f{\isadigit{1}}{\isacharprime}{\kern0pt}\ f{\isadigit{2}}{\isacharprime}{\kern0pt}\ \isakeywordTWO{where}\ f{\isadigit{1}}{\isacharprime}{\kern0pt}{\isacharunderscore}{\kern0pt}intro{\isacharcolon}{\kern0pt}\ {\isachardoublequoteopen}bipart{\isacharunderscore}{\kern0pt}rlexp\ x\ f{\isadigit{1}}{\isacharprime}{\kern0pt}\ {\isasymand}\ vars\ f{\isadigit{1}}{\isacharprime}{\kern0pt}\ {\isacharequal}{\kern0pt}\ vars\ f{\isadigit{1}}\ {\isasymunion}\ {\isacharbraceleft}{\kern0pt}x{\isacharbraceright}{\kern0pt}\ {\isasymand}\isanewline
\ \ \ \ \ \ {\isacharparenleft}{\kern0pt}{\isasymforall}v{\isachardot}{\kern0pt}\ {\isasymPsi}\ {\isacharparenleft}{\kern0pt}eval\ f{\isadigit{1}}\ v{\isacharparenright}{\kern0pt}\ {\isacharequal}{\kern0pt}\ {\isasymPsi}\ {\isacharparenleft}{\kern0pt}eval\ f{\isadigit{1}}{\isacharprime}{\kern0pt}\ v{\isacharparenright}{\kern0pt}{\isacharparenright}{\kern0pt}{\isachardoublequoteclose}\isanewline
\ \ \ \ \isakeywordTWO{and}\ f{\isadigit{2}}{\isacharprime}{\kern0pt}{\isacharunderscore}{\kern0pt}intro{\isacharcolon}{\kern0pt}\ {\isachardoublequoteopen}bipart{\isacharunderscore}{\kern0pt}rlexp\ x\ f{\isadigit{2}}{\isacharprime}{\kern0pt}\ {\isasymand}\ vars\ f{\isadigit{2}}{\isacharprime}{\kern0pt}\ {\isacharequal}{\kern0pt}\ vars\ f{\isadigit{2}}\ {\isasymunion}\ {\isacharbraceleft}{\kern0pt}x{\isacharbraceright}{\kern0pt}\ {\isasymand}\isanewline
\ \ \ \ \ \ {\isacharparenleft}{\kern0pt}{\isasymforall}v{\isachardot}{\kern0pt}\ {\isasymPsi}\ {\isacharparenleft}{\kern0pt}eval\ f{\isadigit{2}}\ v{\isacharparenright}{\kern0pt}\ {\isacharequal}{\kern0pt}\ {\isasymPsi}\ {\isacharparenleft}{\kern0pt}eval\ f{\isadigit{2}}{\isacharprime}{\kern0pt}\ v{\isacharparenright}{\kern0pt}{\isacharparenright}{\kern0pt}{\isachardoublequoteclose}\isanewline
\ \ \ \ \isakeywordONE{by}\isamarkupfalse%
\ auto\isanewline
\ \ \isakeywordONE{then}\isamarkupfalse%
\ \isakeywordTHREE{obtain}\isamarkupfalse%
\ p{\isadigit{1}}\ q{\isadigit{1}}\ p{\isadigit{2}}\ q{\isadigit{2}}\ \isakeywordTWO{where}\ p{\isadigit{1}}{\isacharunderscore}{\kern0pt}q{\isadigit{1}}{\isacharunderscore}{\kern0pt}intro{\isacharcolon}{\kern0pt}\ {\isachardoublequoteopen}reg{\isacharunderscore}{\kern0pt}eval\ p{\isadigit{1}}\ {\isasymand}\ reg{\isacharunderscore}{\kern0pt}eval\ q{\isadigit{1}}\ {\isasymand}\isanewline
\ \ \ \ f{\isadigit{1}}{\isacharprime}{\kern0pt}\ {\isacharequal}{\kern0pt}\ Union\ p{\isadigit{1}}\ {\isacharparenleft}{\kern0pt}Concat\ q{\isadigit{1}}\ {\isacharparenleft}{\kern0pt}Var\ x{\isacharparenright}{\kern0pt}{\isacharparenright}{\kern0pt}\ {\isasymand}\ {\isacharparenleft}{\kern0pt}{\isasymforall}y\ {\isasymin}\ vars\ p{\isadigit{1}}{\isachardot}{\kern0pt}\ y\ {\isasymnoteq}\ x{\isacharparenright}{\kern0pt}{\isachardoublequoteclose}\isanewline
\ \ \ \ \isakeywordTWO{and}\ p{\isadigit{2}}{\isacharunderscore}{\kern0pt}q{\isadigit{2}}{\isacharunderscore}{\kern0pt}intro{\isacharcolon}{\kern0pt}\ {\isachardoublequoteopen}reg{\isacharunderscore}{\kern0pt}eval\ p{\isadigit{2}}\ {\isasymand}\ reg{\isacharunderscore}{\kern0pt}eval\ q{\isadigit{2}}\ {\isasymand}\ f{\isadigit{2}}{\isacharprime}{\kern0pt}\ {\isacharequal}{\kern0pt}\ Union\ p{\isadigit{2}}\ {\isacharparenleft}{\kern0pt}Concat\ q{\isadigit{2}}\ {\isacharparenleft}{\kern0pt}Var\ x{\isacharparenright}{\kern0pt}{\isacharparenright}{\kern0pt}\ {\isasymand}\isanewline
\ \ \ \ {\isacharparenleft}{\kern0pt}{\isasymforall}y\ {\isasymin}\ vars\ p{\isadigit{2}}{\isachardot}{\kern0pt}\ y\ {\isasymnoteq}\ x{\isacharparenright}{\kern0pt}{\isachardoublequoteclose}\ \isakeywordONE{unfolding}\isamarkupfalse%
\ bipart{\isacharunderscore}{\kern0pt}rlexp{\isacharunderscore}{\kern0pt}def\ \isakeywordONE{by}\isamarkupfalse%
\ auto\isanewline
\ \ \isakeywordONE{let}\isamarkupfalse%
\ {\isacharquery}{\kern0pt}f{\isacharprime}{\kern0pt}\ {\isacharequal}{\kern0pt}\ {\isachardoublequoteopen}Union\ {\isacharparenleft}{\kern0pt}Union\ p{\isadigit{1}}\ p{\isadigit{2}}{\isacharparenright}{\kern0pt}\ {\isacharparenleft}{\kern0pt}Concat\ {\isacharparenleft}{\kern0pt}Union\ q{\isadigit{1}}\ q{\isadigit{2}}{\isacharparenright}{\kern0pt}\ {\isacharparenleft}{\kern0pt}Var\ x{\isacharparenright}{\kern0pt}{\isacharparenright}{\kern0pt}{\isachardoublequoteclose}\isanewline
\ \ \isakeywordONE{have}\isamarkupfalse%
\ {\isachardoublequoteopen}bipart{\isacharunderscore}{\kern0pt}rlexp\ x\ {\isacharquery}{\kern0pt}f{\isacharprime}{\kern0pt}{\isachardoublequoteclose}\ \isakeywordONE{unfolding}\isamarkupfalse%
\ bipart{\isacharunderscore}{\kern0pt}rlexp{\isacharunderscore}{\kern0pt}def\ \isakeywordONE{using}\isamarkupfalse%
\ p{\isadigit{1}}{\isacharunderscore}{\kern0pt}q{\isadigit{1}}{\isacharunderscore}{\kern0pt}intro\ p{\isadigit{2}}{\isacharunderscore}{\kern0pt}q{\isadigit{2}}{\isacharunderscore}{\kern0pt}intro\ \isakeywordONE{by}\isamarkupfalse%
\ auto\isanewline
\ \ \isakeywordONE{moreover}\isamarkupfalse%
\ \isakeywordONE{have}\isamarkupfalse%
\ {\isachardoublequoteopen}{\isasymPsi}\ {\isacharparenleft}{\kern0pt}eval\ {\isacharquery}{\kern0pt}f{\isacharprime}{\kern0pt}\ v{\isacharparenright}{\kern0pt}\ {\isacharequal}{\kern0pt}\ {\isasymPsi}\ {\isacharparenleft}{\kern0pt}eval\ {\isacharparenleft}{\kern0pt}Union\ f{\isadigit{1}}\ f{\isadigit{2}}{\isacharparenright}{\kern0pt}\ v{\isacharparenright}{\kern0pt}{\isachardoublequoteclose}\ \isakeywordTWO{for}\ v\isanewline
\ \ \ \ \isakeywordONE{using}\isamarkupfalse%
\ p{\isadigit{1}}{\isacharunderscore}{\kern0pt}q{\isadigit{1}}{\isacharunderscore}{\kern0pt}intro\ p{\isadigit{2}}{\isacharunderscore}{\kern0pt}q{\isadigit{2}}{\isacharunderscore}{\kern0pt}intro\ f{\isadigit{1}}{\isacharprime}{\kern0pt}{\isacharunderscore}{\kern0pt}intro\ f{\isadigit{2}}{\isacharprime}{\kern0pt}{\isacharunderscore}{\kern0pt}intro\isanewline
\ \ \ \ \isakeywordONE{by}\isamarkupfalse%
\ {\isacharparenleft}{\kern0pt}simp\ add{\isacharcolon}{\kern0pt}\ conc{\isacharunderscore}{\kern0pt}Un{\isacharunderscore}{\kern0pt}distrib{\isacharparenleft}{\kern0pt}{\isadigit{2}}{\isacharparenright}{\kern0pt}\ sup{\isacharunderscore}{\kern0pt}assoc\ sup{\isacharunderscore}{\kern0pt}left{\isacharunderscore}{\kern0pt}commute{\isacharparenright}{\kern0pt}\isanewline
\ \ \isakeywordONE{moreover}\isamarkupfalse%
\ \isakeywordONE{from}\isamarkupfalse%
\ f{\isadigit{1}}{\isacharprime}{\kern0pt}{\isacharunderscore}{\kern0pt}intro\ f{\isadigit{2}}{\isacharprime}{\kern0pt}{\isacharunderscore}{\kern0pt}intro\ p{\isadigit{1}}{\isacharunderscore}{\kern0pt}q{\isadigit{1}}{\isacharunderscore}{\kern0pt}intro\ p{\isadigit{2}}{\isacharunderscore}{\kern0pt}q{\isadigit{2}}{\isacharunderscore}{\kern0pt}intro\isanewline
\ \ \ \ \isakeywordONE{have}\isamarkupfalse%
\ {\isachardoublequoteopen}vars\ {\isacharquery}{\kern0pt}f{\isacharprime}{\kern0pt}\ {\isacharequal}{\kern0pt}\ vars\ {\isacharparenleft}{\kern0pt}Union\ f{\isadigit{1}}\ f{\isadigit{2}}{\isacharparenright}{\kern0pt}\ {\isasymunion}\ {\isacharbraceleft}{\kern0pt}x{\isacharbraceright}{\kern0pt}{\isachardoublequoteclose}\ \isakeywordONE{by}\isamarkupfalse%
\ auto\isanewline
\ \ \isakeywordONE{ultimately}\isamarkupfalse%
\ \isakeywordTHREE{show}\isamarkupfalse%
\ {\isacharquery}{\kern0pt}thesis\ \isakeywordONE{by}\isamarkupfalse%
\ metis\isanewline
\isakeywordONE{qed}\isamarkupfalse%
%
\endisatagproof
{\isafoldproof}%
%
\isadelimproof
\isanewline
%
\endisadelimproof
\isanewline
\isakeywordONE{lemma}\isamarkupfalse%
\ reg{\isacharunderscore}{\kern0pt}eval{\isacharunderscore}{\kern0pt}bipart{\isacharunderscore}{\kern0pt}rlexp{\isacharunderscore}{\kern0pt}Concat{\isacharcolon}{\kern0pt}\isanewline
\ \ \isakeywordTWO{assumes}\ {\isachardoublequoteopen}{\isasymexists}f{\isacharprime}{\kern0pt}{\isachardot}{\kern0pt}\ bipart{\isacharunderscore}{\kern0pt}rlexp\ x\ f{\isacharprime}{\kern0pt}\ {\isasymand}\ vars\ f{\isacharprime}{\kern0pt}\ {\isacharequal}{\kern0pt}\ vars\ f{\isadigit{1}}\ {\isasymunion}\ {\isacharbraceleft}{\kern0pt}x{\isacharbraceright}{\kern0pt}\ {\isasymand}\isanewline
\ \ \ \ \ \ \ \ \ \ \ \ \ \ \ \ {\isacharparenleft}{\kern0pt}{\isasymforall}v{\isachardot}{\kern0pt}\ {\isasymPsi}\ {\isacharparenleft}{\kern0pt}eval\ f{\isadigit{1}}\ v{\isacharparenright}{\kern0pt}\ {\isacharequal}{\kern0pt}\ {\isasymPsi}\ {\isacharparenleft}{\kern0pt}eval\ f{\isacharprime}{\kern0pt}\ v{\isacharparenright}{\kern0pt}{\isacharparenright}{\kern0pt}{\isachardoublequoteclose}\isanewline
\ \ \ \ \ \ \ \ \ \ {\isachardoublequoteopen}{\isasymexists}f{\isacharprime}{\kern0pt}{\isachardot}{\kern0pt}\ bipart{\isacharunderscore}{\kern0pt}rlexp\ x\ f{\isacharprime}{\kern0pt}\ {\isasymand}\ vars\ f{\isacharprime}{\kern0pt}\ {\isacharequal}{\kern0pt}\ vars\ f{\isadigit{2}}\ {\isasymunion}\ {\isacharbraceleft}{\kern0pt}x{\isacharbraceright}{\kern0pt}\ {\isasymand}\isanewline
\ \ \ \ \ \ \ \ \ \ \ \ \ \ \ \ {\isacharparenleft}{\kern0pt}{\isasymforall}v{\isachardot}{\kern0pt}\ {\isasymPsi}\ {\isacharparenleft}{\kern0pt}eval\ f{\isadigit{2}}\ v{\isacharparenright}{\kern0pt}\ {\isacharequal}{\kern0pt}\ {\isasymPsi}\ {\isacharparenleft}{\kern0pt}eval\ f{\isacharprime}{\kern0pt}\ v{\isacharparenright}{\kern0pt}{\isacharparenright}{\kern0pt}{\isachardoublequoteclose}\isanewline
\ \ \ \ \isakeywordTWO{shows}\ {\isachardoublequoteopen}{\isasymexists}f{\isacharprime}{\kern0pt}{\isachardot}{\kern0pt}\ bipart{\isacharunderscore}{\kern0pt}rlexp\ x\ f{\isacharprime}{\kern0pt}\ {\isasymand}\ vars\ f{\isacharprime}{\kern0pt}\ {\isacharequal}{\kern0pt}\ vars\ {\isacharparenleft}{\kern0pt}Concat\ f{\isadigit{1}}\ f{\isadigit{2}}{\isacharparenright}{\kern0pt}\ {\isasymunion}\ {\isacharbraceleft}{\kern0pt}x{\isacharbraceright}{\kern0pt}\ {\isasymand}\isanewline
\ \ \ \ \ \ \ \ \ \ \ \ \ \ \ \ {\isacharparenleft}{\kern0pt}{\isasymforall}v{\isachardot}{\kern0pt}\ {\isasymPsi}\ {\isacharparenleft}{\kern0pt}eval\ {\isacharparenleft}{\kern0pt}Concat\ f{\isadigit{1}}\ f{\isadigit{2}}{\isacharparenright}{\kern0pt}\ v{\isacharparenright}{\kern0pt}\ {\isacharequal}{\kern0pt}\ {\isasymPsi}\ {\isacharparenleft}{\kern0pt}eval\ f{\isacharprime}{\kern0pt}\ v{\isacharparenright}{\kern0pt}{\isacharparenright}{\kern0pt}{\isachardoublequoteclose}\isanewline
%
\isadelimproof
%
\endisadelimproof
%
\isatagproof
\isakeywordONE{proof}\isamarkupfalse%
\ {\isacharminus}{\kern0pt}\isanewline
\ \ \isakeywordONE{from}\isamarkupfalse%
\ assms\ \isakeywordTHREE{obtain}\isamarkupfalse%
\ f{\isadigit{1}}{\isacharprime}{\kern0pt}\ f{\isadigit{2}}{\isacharprime}{\kern0pt}\ \isakeywordTWO{where}\ f{\isadigit{1}}{\isacharprime}{\kern0pt}{\isacharunderscore}{\kern0pt}intro{\isacharcolon}{\kern0pt}\ {\isachardoublequoteopen}bipart{\isacharunderscore}{\kern0pt}rlexp\ x\ f{\isadigit{1}}{\isacharprime}{\kern0pt}\ {\isasymand}\ vars\ f{\isadigit{1}}{\isacharprime}{\kern0pt}\ {\isacharequal}{\kern0pt}\ vars\ f{\isadigit{1}}\ {\isasymunion}\ {\isacharbraceleft}{\kern0pt}x{\isacharbraceright}{\kern0pt}\ {\isasymand}\isanewline
\ \ \ \ \ \ {\isacharparenleft}{\kern0pt}{\isasymforall}v{\isachardot}{\kern0pt}\ {\isasymPsi}\ {\isacharparenleft}{\kern0pt}eval\ f{\isadigit{1}}\ v{\isacharparenright}{\kern0pt}\ {\isacharequal}{\kern0pt}\ {\isasymPsi}\ {\isacharparenleft}{\kern0pt}eval\ f{\isadigit{1}}{\isacharprime}{\kern0pt}\ v{\isacharparenright}{\kern0pt}{\isacharparenright}{\kern0pt}{\isachardoublequoteclose}\isanewline
\ \ \ \ \isakeywordTWO{and}\ f{\isadigit{2}}{\isacharprime}{\kern0pt}{\isacharunderscore}{\kern0pt}intro{\isacharcolon}{\kern0pt}\ {\isachardoublequoteopen}bipart{\isacharunderscore}{\kern0pt}rlexp\ x\ f{\isadigit{2}}{\isacharprime}{\kern0pt}\ {\isasymand}\ vars\ f{\isadigit{2}}{\isacharprime}{\kern0pt}\ {\isacharequal}{\kern0pt}\ vars\ f{\isadigit{2}}\ {\isasymunion}\ {\isacharbraceleft}{\kern0pt}x{\isacharbraceright}{\kern0pt}\ {\isasymand}\isanewline
\ \ \ \ \ \ {\isacharparenleft}{\kern0pt}{\isasymforall}v{\isachardot}{\kern0pt}\ {\isasymPsi}\ {\isacharparenleft}{\kern0pt}eval\ f{\isadigit{2}}\ v{\isacharparenright}{\kern0pt}\ {\isacharequal}{\kern0pt}\ {\isasymPsi}\ {\isacharparenleft}{\kern0pt}eval\ f{\isadigit{2}}{\isacharprime}{\kern0pt}\ v{\isacharparenright}{\kern0pt}{\isacharparenright}{\kern0pt}{\isachardoublequoteclose}\isanewline
\ \ \ \ \isakeywordONE{by}\isamarkupfalse%
\ auto\isanewline
\ \ \isakeywordONE{then}\isamarkupfalse%
\ \isakeywordTHREE{obtain}\isamarkupfalse%
\ p{\isadigit{1}}\ q{\isadigit{1}}\ p{\isadigit{2}}\ q{\isadigit{2}}\ \isakeywordTWO{where}\ p{\isadigit{1}}{\isacharunderscore}{\kern0pt}q{\isadigit{1}}{\isacharunderscore}{\kern0pt}intro{\isacharcolon}{\kern0pt}\ {\isachardoublequoteopen}reg{\isacharunderscore}{\kern0pt}eval\ p{\isadigit{1}}\ {\isasymand}\ reg{\isacharunderscore}{\kern0pt}eval\ q{\isadigit{1}}\ {\isasymand}\isanewline
\ \ \ \ f{\isadigit{1}}{\isacharprime}{\kern0pt}\ {\isacharequal}{\kern0pt}\ Union\ p{\isadigit{1}}\ {\isacharparenleft}{\kern0pt}Concat\ q{\isadigit{1}}\ {\isacharparenleft}{\kern0pt}Var\ x{\isacharparenright}{\kern0pt}{\isacharparenright}{\kern0pt}\ {\isasymand}\ {\isacharparenleft}{\kern0pt}{\isasymforall}y\ {\isasymin}\ vars\ p{\isadigit{1}}{\isachardot}{\kern0pt}\ y\ {\isasymnoteq}\ x{\isacharparenright}{\kern0pt}{\isachardoublequoteclose}\isanewline
\ \ \ \ \isakeywordTWO{and}\ p{\isadigit{2}}{\isacharunderscore}{\kern0pt}q{\isadigit{2}}{\isacharunderscore}{\kern0pt}intro{\isacharcolon}{\kern0pt}\ {\isachardoublequoteopen}reg{\isacharunderscore}{\kern0pt}eval\ p{\isadigit{2}}\ {\isasymand}\ reg{\isacharunderscore}{\kern0pt}eval\ q{\isadigit{2}}\ {\isasymand}\ f{\isadigit{2}}{\isacharprime}{\kern0pt}\ {\isacharequal}{\kern0pt}\ Union\ p{\isadigit{2}}\ {\isacharparenleft}{\kern0pt}Concat\ q{\isadigit{2}}\ {\isacharparenleft}{\kern0pt}Var\ x{\isacharparenright}{\kern0pt}{\isacharparenright}{\kern0pt}\ {\isasymand}\isanewline
\ \ \ \ {\isacharparenleft}{\kern0pt}{\isasymforall}y\ {\isasymin}\ vars\ p{\isadigit{2}}{\isachardot}{\kern0pt}\ y\ {\isasymnoteq}\ x{\isacharparenright}{\kern0pt}{\isachardoublequoteclose}\ \isakeywordONE{unfolding}\isamarkupfalse%
\ bipart{\isacharunderscore}{\kern0pt}rlexp{\isacharunderscore}{\kern0pt}def\ \isakeywordONE{by}\isamarkupfalse%
\ auto\isanewline
\ \ \isakeywordONE{let}\isamarkupfalse%
\ {\isacharquery}{\kern0pt}q{\isacharprime}{\kern0pt}\ {\isacharequal}{\kern0pt}\ {\isachardoublequoteopen}Union\ {\isacharparenleft}{\kern0pt}Concat\ q{\isadigit{1}}\ {\isacharparenleft}{\kern0pt}Concat\ {\isacharparenleft}{\kern0pt}Var\ x{\isacharparenright}{\kern0pt}\ q{\isadigit{2}}{\isacharparenright}{\kern0pt}{\isacharparenright}{\kern0pt}\ {\isacharparenleft}{\kern0pt}Union\ {\isacharparenleft}{\kern0pt}Concat\ p{\isadigit{1}}\ q{\isadigit{2}}{\isacharparenright}{\kern0pt}\ {\isacharparenleft}{\kern0pt}Concat\ q{\isadigit{1}}\ p{\isadigit{2}}{\isacharparenright}{\kern0pt}{\isacharparenright}{\kern0pt}{\isachardoublequoteclose}\isanewline
\ \ \isakeywordONE{let}\isamarkupfalse%
\ {\isacharquery}{\kern0pt}f{\isacharprime}{\kern0pt}\ {\isacharequal}{\kern0pt}\ {\isachardoublequoteopen}Union\ {\isacharparenleft}{\kern0pt}Concat\ p{\isadigit{1}}\ p{\isadigit{2}}{\isacharparenright}{\kern0pt}\ {\isacharparenleft}{\kern0pt}Concat\ {\isacharquery}{\kern0pt}q{\isacharprime}{\kern0pt}\ {\isacharparenleft}{\kern0pt}Var\ x{\isacharparenright}{\kern0pt}{\isacharparenright}{\kern0pt}{\isachardoublequoteclose}\isanewline
\ \ \isakeywordONE{have}\isamarkupfalse%
\ {\isachardoublequoteopen}{\isasymforall}v{\isachardot}{\kern0pt}\ {\isacharparenleft}{\kern0pt}{\isasymPsi}\ {\isacharparenleft}{\kern0pt}eval\ {\isacharparenleft}{\kern0pt}Concat\ f{\isadigit{1}}\ f{\isadigit{2}}{\isacharparenright}{\kern0pt}\ v{\isacharparenright}{\kern0pt}\ {\isacharequal}{\kern0pt}\ {\isasymPsi}\ {\isacharparenleft}{\kern0pt}eval\ {\isacharquery}{\kern0pt}f{\isacharprime}{\kern0pt}\ v{\isacharparenright}{\kern0pt}{\isacharparenright}{\kern0pt}{\isachardoublequoteclose}\isanewline
\ \ \isakeywordONE{proof}\isamarkupfalse%
\ {\isacharparenleft}{\kern0pt}rule\ allI{\isacharparenright}{\kern0pt}\isanewline
\ \ \ \ \isakeywordTHREE{fix}\isamarkupfalse%
\ v\isanewline
\ \ \ \ \isakeywordONE{have}\isamarkupfalse%
\ f{\isadigit{2}}{\isacharunderscore}{\kern0pt}subst{\isacharcolon}{\kern0pt}\ {\isachardoublequoteopen}{\isasymPsi}\ {\isacharparenleft}{\kern0pt}eval\ f{\isadigit{2}}\ v{\isacharparenright}{\kern0pt}\ {\isacharequal}{\kern0pt}\ {\isasymPsi}\ {\isacharparenleft}{\kern0pt}eval\ p{\isadigit{2}}\ v\ {\isasymunion}\ eval\ q{\isadigit{2}}\ v\ {\isacharat}{\kern0pt}{\isacharat}{\kern0pt}\ v\ x{\isacharparenright}{\kern0pt}{\isachardoublequoteclose}\isanewline
\ \ \ \ \ \ \isakeywordONE{using}\isamarkupfalse%
\ p{\isadigit{2}}{\isacharunderscore}{\kern0pt}q{\isadigit{2}}{\isacharunderscore}{\kern0pt}intro\ f{\isadigit{2}}{\isacharprime}{\kern0pt}{\isacharunderscore}{\kern0pt}intro\ \isakeywordONE{by}\isamarkupfalse%
\ auto\isanewline
\ \ \ \ \isakeywordONE{have}\isamarkupfalse%
\ {\isachardoublequoteopen}{\isasymPsi}\ {\isacharparenleft}{\kern0pt}eval\ {\isacharparenleft}{\kern0pt}Concat\ f{\isadigit{1}}\ f{\isadigit{2}}{\isacharparenright}{\kern0pt}\ v{\isacharparenright}{\kern0pt}\ {\isacharequal}{\kern0pt}\ {\isasymPsi}\ {\isacharparenleft}{\kern0pt}{\isacharparenleft}{\kern0pt}eval\ p{\isadigit{1}}\ v\ {\isasymunion}\ eval\ q{\isadigit{1}}\ v\ {\isacharat}{\kern0pt}{\isacharat}{\kern0pt}\ v\ x{\isacharparenright}{\kern0pt}\ {\isacharat}{\kern0pt}{\isacharat}{\kern0pt}\ eval\ f{\isadigit{2}}\ v{\isacharparenright}{\kern0pt}{\isachardoublequoteclose}\isanewline
\ \ \ \ \ \ \isakeywordONE{using}\isamarkupfalse%
\ p{\isadigit{1}}{\isacharunderscore}{\kern0pt}q{\isadigit{1}}{\isacharunderscore}{\kern0pt}intro\ f{\isadigit{1}}{\isacharprime}{\kern0pt}{\isacharunderscore}{\kern0pt}intro\isanewline
\ \ \ \ \ \ \isakeywordONE{by}\isamarkupfalse%
\ {\isacharparenleft}{\kern0pt}metis\ eval{\isachardot}{\kern0pt}simps{\isacharparenleft}{\kern0pt}{\isadigit{1}}{\isacharparenright}{\kern0pt}\ eval{\isachardot}{\kern0pt}simps{\isacharparenleft}{\kern0pt}{\isadigit{3}}{\isacharparenright}{\kern0pt}\ eval{\isachardot}{\kern0pt}simps{\isacharparenleft}{\kern0pt}{\isadigit{4}}{\isacharparenright}{\kern0pt}\ parikh{\isacharunderscore}{\kern0pt}conc{\isacharunderscore}{\kern0pt}right{\isacharparenright}{\kern0pt}\isanewline
\ \ \ \ \isakeywordONE{also}\isamarkupfalse%
\ \isakeywordONE{have}\isamarkupfalse%
\ {\isachardoublequoteopen}{\isasymdots}\ {\isacharequal}{\kern0pt}\ {\isasymPsi}\ {\isacharparenleft}{\kern0pt}eval\ p{\isadigit{1}}\ v\ {\isacharat}{\kern0pt}{\isacharat}{\kern0pt}\ eval\ f{\isadigit{2}}\ v\ {\isasymunion}\ eval\ q{\isadigit{1}}\ v\ {\isacharat}{\kern0pt}{\isacharat}{\kern0pt}\ v\ x\ {\isacharat}{\kern0pt}{\isacharat}{\kern0pt}\ eval\ f{\isadigit{2}}\ v{\isacharparenright}{\kern0pt}{\isachardoublequoteclose}\isanewline
\ \ \ \ \ \ \isakeywordONE{by}\isamarkupfalse%
\ {\isacharparenleft}{\kern0pt}simp\ add{\isacharcolon}{\kern0pt}\ conc{\isacharunderscore}{\kern0pt}Un{\isacharunderscore}{\kern0pt}distrib{\isacharparenleft}{\kern0pt}{\isadigit{2}}{\isacharparenright}{\kern0pt}\ conc{\isacharunderscore}{\kern0pt}assoc{\isacharparenright}{\kern0pt}\isanewline
\ \ \ \ \isakeywordONE{also}\isamarkupfalse%
\ \isakeywordONE{have}\isamarkupfalse%
\ {\isachardoublequoteopen}{\isasymdots}\ {\isacharequal}{\kern0pt}\ {\isasymPsi}\ {\isacharparenleft}{\kern0pt}eval\ p{\isadigit{1}}\ v\ {\isacharat}{\kern0pt}{\isacharat}{\kern0pt}\ {\isacharparenleft}{\kern0pt}eval\ p{\isadigit{2}}\ v\ {\isasymunion}\ eval\ q{\isadigit{2}}\ v\ {\isacharat}{\kern0pt}{\isacharat}{\kern0pt}\ v\ x{\isacharparenright}{\kern0pt}\isanewline
\ \ \ \ \ \ \ \ {\isasymunion}\ eval\ q{\isadigit{1}}\ v\ {\isacharat}{\kern0pt}{\isacharat}{\kern0pt}\ v\ x\ {\isacharat}{\kern0pt}{\isacharat}{\kern0pt}\ {\isacharparenleft}{\kern0pt}eval\ p{\isadigit{2}}\ v\ {\isasymunion}\ eval\ q{\isadigit{2}}\ v\ {\isacharat}{\kern0pt}{\isacharat}{\kern0pt}\ v\ x{\isacharparenright}{\kern0pt}{\isacharparenright}{\kern0pt}{\isachardoublequoteclose}\isanewline
\ \ \ \ \ \ \isakeywordONE{using}\isamarkupfalse%
\ f{\isadigit{2}}{\isacharunderscore}{\kern0pt}subst\ \isakeywordONE{by}\isamarkupfalse%
\ {\isacharparenleft}{\kern0pt}smt\ {\isacharparenleft}{\kern0pt}verit{\isacharcomma}{\kern0pt}\ ccfv{\isacharunderscore}{\kern0pt}threshold{\isacharparenright}{\kern0pt}\ parikh{\isacharunderscore}{\kern0pt}conc{\isacharunderscore}{\kern0pt}right\ parikh{\isacharunderscore}{\kern0pt}img{\isacharunderscore}{\kern0pt}Un\ parikh{\isacharunderscore}{\kern0pt}img{\isacharunderscore}{\kern0pt}commut{\isacharparenright}{\kern0pt}\isanewline
\ \ \ \ \isakeywordONE{also}\isamarkupfalse%
\ \isakeywordONE{have}\isamarkupfalse%
\ {\isachardoublequoteopen}{\isasymdots}\ {\isacharequal}{\kern0pt}\ {\isasymPsi}\ {\isacharparenleft}{\kern0pt}eval\ p{\isadigit{1}}\ v\ {\isacharat}{\kern0pt}{\isacharat}{\kern0pt}\ eval\ p{\isadigit{2}}\ v\ {\isasymunion}\ {\isacharparenleft}{\kern0pt}eval\ p{\isadigit{1}}\ v\ {\isacharat}{\kern0pt}{\isacharat}{\kern0pt}\ eval\ q{\isadigit{2}}\ v\ {\isacharat}{\kern0pt}{\isacharat}{\kern0pt}\ v\ x\ {\isasymunion}\isanewline
\ \ \ \ \ \ \ \ eval\ q{\isadigit{1}}\ v\ {\isacharat}{\kern0pt}{\isacharat}{\kern0pt}\ eval\ p{\isadigit{2}}\ v\ {\isacharat}{\kern0pt}{\isacharat}{\kern0pt}\ v\ x\ {\isasymunion}\ eval\ q{\isadigit{1}}\ v\ {\isacharat}{\kern0pt}{\isacharat}{\kern0pt}\ v\ x\ {\isacharat}{\kern0pt}{\isacharat}{\kern0pt}\ eval\ q{\isadigit{2}}\ v\ {\isacharat}{\kern0pt}{\isacharat}{\kern0pt}\ v\ x{\isacharparenright}{\kern0pt}{\isacharparenright}{\kern0pt}{\isachardoublequoteclose}\isanewline
\ \ \ \ \ \ \isakeywordONE{using}\isamarkupfalse%
\ parikh{\isacharunderscore}{\kern0pt}img{\isacharunderscore}{\kern0pt}commut\ \isakeywordONE{by}\isamarkupfalse%
\ {\isacharparenleft}{\kern0pt}smt\ {\isacharparenleft}{\kern0pt}z{\isadigit{3}}{\isacharparenright}{\kern0pt}\ conc{\isacharunderscore}{\kern0pt}Un{\isacharunderscore}{\kern0pt}distrib{\isacharparenleft}{\kern0pt}{\isadigit{1}}{\isacharparenright}{\kern0pt}\ parikh{\isacharunderscore}{\kern0pt}conc{\isacharunderscore}{\kern0pt}right\ parikh{\isacharunderscore}{\kern0pt}img{\isacharunderscore}{\kern0pt}Un\ sup{\isacharunderscore}{\kern0pt}assoc{\isacharparenright}{\kern0pt}\isanewline
\ \ \ \ \isakeywordONE{also}\isamarkupfalse%
\ \isakeywordONE{have}\isamarkupfalse%
\ {\isachardoublequoteopen}{\isasymdots}\ {\isacharequal}{\kern0pt}\ {\isasymPsi}\ {\isacharparenleft}{\kern0pt}eval\ p{\isadigit{1}}\ v\ {\isacharat}{\kern0pt}{\isacharat}{\kern0pt}\ eval\ p{\isadigit{2}}\ v\ {\isasymunion}\ {\isacharparenleft}{\kern0pt}eval\ p{\isadigit{1}}\ v\ {\isacharat}{\kern0pt}{\isacharat}{\kern0pt}\ eval\ q{\isadigit{2}}\ v\ {\isasymunion}\isanewline
\ \ \ \ \ \ \ \ eval\ q{\isadigit{1}}\ v\ {\isacharat}{\kern0pt}{\isacharat}{\kern0pt}\ eval\ p{\isadigit{2}}\ v\ {\isasymunion}\ eval\ q{\isadigit{1}}\ v\ {\isacharat}{\kern0pt}{\isacharat}{\kern0pt}\ v\ x\ {\isacharat}{\kern0pt}{\isacharat}{\kern0pt}\ eval\ q{\isadigit{2}}\ v{\isacharparenright}{\kern0pt}\ {\isacharat}{\kern0pt}{\isacharat}{\kern0pt}\ v\ x{\isacharparenright}{\kern0pt}{\isachardoublequoteclose}\isanewline
\ \ \ \ \ \ \isakeywordONE{by}\isamarkupfalse%
\ {\isacharparenleft}{\kern0pt}simp\ add{\isacharcolon}{\kern0pt}\ conc{\isacharunderscore}{\kern0pt}Un{\isacharunderscore}{\kern0pt}distrib{\isacharparenleft}{\kern0pt}{\isadigit{2}}{\isacharparenright}{\kern0pt}\ conc{\isacharunderscore}{\kern0pt}assoc{\isacharparenright}{\kern0pt}\isanewline
\ \ \ \ \isakeywordONE{also}\isamarkupfalse%
\ \isakeywordONE{have}\isamarkupfalse%
\ {\isachardoublequoteopen}{\isasymdots}\ {\isacharequal}{\kern0pt}\ {\isasymPsi}\ {\isacharparenleft}{\kern0pt}eval\ {\isacharquery}{\kern0pt}f{\isacharprime}{\kern0pt}\ v{\isacharparenright}{\kern0pt}{\isachardoublequoteclose}\isanewline
\ \ \ \ \ \ \isakeywordONE{by}\isamarkupfalse%
\ {\isacharparenleft}{\kern0pt}simp\ add{\isacharcolon}{\kern0pt}\ Un{\isacharunderscore}{\kern0pt}commute{\isacharparenright}{\kern0pt}\isanewline
\ \ \ \ \isakeywordONE{finally}\isamarkupfalse%
\ \isakeywordTHREE{show}\isamarkupfalse%
\ {\isachardoublequoteopen}{\isasymPsi}\ {\isacharparenleft}{\kern0pt}eval\ {\isacharparenleft}{\kern0pt}Concat\ f{\isadigit{1}}\ f{\isadigit{2}}{\isacharparenright}{\kern0pt}\ v{\isacharparenright}{\kern0pt}\ {\isacharequal}{\kern0pt}\ {\isasymPsi}\ {\isacharparenleft}{\kern0pt}eval\ {\isacharquery}{\kern0pt}f{\isacharprime}{\kern0pt}\ v{\isacharparenright}{\kern0pt}{\isachardoublequoteclose}\ \isakeywordONE{{\isachardot}{\kern0pt}}\isamarkupfalse%
\isanewline
\ \ \isakeywordONE{qed}\isamarkupfalse%
\isanewline
\ \ \isakeywordONE{moreover}\isamarkupfalse%
\ \isakeywordONE{have}\isamarkupfalse%
\ {\isachardoublequoteopen}bipart{\isacharunderscore}{\kern0pt}rlexp\ x\ {\isacharquery}{\kern0pt}f{\isacharprime}{\kern0pt}{\isachardoublequoteclose}\ \isakeywordONE{unfolding}\isamarkupfalse%
\ bipart{\isacharunderscore}{\kern0pt}rlexp{\isacharunderscore}{\kern0pt}def\ \isakeywordONE{using}\isamarkupfalse%
\ p{\isadigit{1}}{\isacharunderscore}{\kern0pt}q{\isadigit{1}}{\isacharunderscore}{\kern0pt}intro\ p{\isadigit{2}}{\isacharunderscore}{\kern0pt}q{\isadigit{2}}{\isacharunderscore}{\kern0pt}intro\ \isakeywordONE{by}\isamarkupfalse%
\ auto\isanewline
\ \ \isakeywordONE{moreover}\isamarkupfalse%
\ \isakeywordONE{from}\isamarkupfalse%
\ f{\isadigit{1}}{\isacharprime}{\kern0pt}{\isacharunderscore}{\kern0pt}intro\ f{\isadigit{2}}{\isacharprime}{\kern0pt}{\isacharunderscore}{\kern0pt}intro\ p{\isadigit{1}}{\isacharunderscore}{\kern0pt}q{\isadigit{1}}{\isacharunderscore}{\kern0pt}intro\ p{\isadigit{2}}{\isacharunderscore}{\kern0pt}q{\isadigit{2}}{\isacharunderscore}{\kern0pt}intro\isanewline
\ \ \ \ \isakeywordONE{have}\isamarkupfalse%
\ {\isachardoublequoteopen}vars\ {\isacharquery}{\kern0pt}f{\isacharprime}{\kern0pt}\ {\isacharequal}{\kern0pt}\ vars\ {\isacharparenleft}{\kern0pt}Concat\ f{\isadigit{1}}\ f{\isadigit{2}}{\isacharparenright}{\kern0pt}\ {\isasymunion}\ {\isacharbraceleft}{\kern0pt}x{\isacharbraceright}{\kern0pt}{\isachardoublequoteclose}\ \isakeywordONE{by}\isamarkupfalse%
\ auto\isanewline
\ \ \isakeywordONE{ultimately}\isamarkupfalse%
\ \isakeywordTHREE{show}\isamarkupfalse%
\ {\isacharquery}{\kern0pt}thesis\ \isakeywordONE{by}\isamarkupfalse%
\ metis\isanewline
\isakeywordONE{qed}\isamarkupfalse%
%
\endisatagproof
{\isafoldproof}%
%
\isadelimproof
\isanewline
%
\endisadelimproof
\isanewline
\isakeywordONE{lemma}\isamarkupfalse%
\ reg{\isacharunderscore}{\kern0pt}eval{\isacharunderscore}{\kern0pt}bipart{\isacharunderscore}{\kern0pt}rlexp{\isacharunderscore}{\kern0pt}Star{\isacharcolon}{\kern0pt}\isanewline
\ \ \isakeywordTWO{assumes}\ {\isachardoublequoteopen}{\isasymexists}f{\isacharprime}{\kern0pt}{\isachardot}{\kern0pt}\ bipart{\isacharunderscore}{\kern0pt}rlexp\ x\ f{\isacharprime}{\kern0pt}\ {\isasymand}\ vars\ f{\isacharprime}{\kern0pt}\ {\isacharequal}{\kern0pt}\ vars\ f\ {\isasymunion}\ {\isacharbraceleft}{\kern0pt}x{\isacharbraceright}{\kern0pt}\isanewline
\ \ \ \ \ \ \ \ \ \ \ \ \ \ \ \ {\isasymand}\ {\isacharparenleft}{\kern0pt}{\isasymforall}v{\isachardot}{\kern0pt}\ {\isasymPsi}\ {\isacharparenleft}{\kern0pt}eval\ f\ v{\isacharparenright}{\kern0pt}\ {\isacharequal}{\kern0pt}\ {\isasymPsi}\ {\isacharparenleft}{\kern0pt}eval\ f{\isacharprime}{\kern0pt}\ v{\isacharparenright}{\kern0pt}{\isacharparenright}{\kern0pt}{\isachardoublequoteclose}\isanewline
\ \ \isakeywordTWO{shows}\ {\isachardoublequoteopen}{\isasymexists}f{\isacharprime}{\kern0pt}{\isachardot}{\kern0pt}\ bipart{\isacharunderscore}{\kern0pt}rlexp\ x\ f{\isacharprime}{\kern0pt}\ {\isasymand}\ vars\ f{\isacharprime}{\kern0pt}\ {\isacharequal}{\kern0pt}\ vars\ {\isacharparenleft}{\kern0pt}Star\ f{\isacharparenright}{\kern0pt}\ {\isasymunion}\ {\isacharbraceleft}{\kern0pt}x{\isacharbraceright}{\kern0pt}\isanewline
\ \ \ \ \ \ \ \ \ \ \ \ \ \ \ \ {\isasymand}\ {\isacharparenleft}{\kern0pt}{\isasymforall}v{\isachardot}{\kern0pt}\ {\isasymPsi}\ {\isacharparenleft}{\kern0pt}eval\ {\isacharparenleft}{\kern0pt}Star\ f{\isacharparenright}{\kern0pt}\ v{\isacharparenright}{\kern0pt}\ {\isacharequal}{\kern0pt}\ {\isasymPsi}\ {\isacharparenleft}{\kern0pt}eval\ f{\isacharprime}{\kern0pt}\ v{\isacharparenright}{\kern0pt}{\isacharparenright}{\kern0pt}{\isachardoublequoteclose}\isanewline
%
\isadelimproof
%
\endisadelimproof
%
\isatagproof
\isakeywordONE{proof}\isamarkupfalse%
\ {\isacharminus}{\kern0pt}\isanewline
\ \ \isakeywordONE{from}\isamarkupfalse%
\ assms\ \isakeywordTHREE{obtain}\isamarkupfalse%
\ f{\isacharprime}{\kern0pt}\ \isakeywordTWO{where}\ f{\isacharprime}{\kern0pt}{\isacharunderscore}{\kern0pt}intro{\isacharcolon}{\kern0pt}\ {\isachardoublequoteopen}bipart{\isacharunderscore}{\kern0pt}rlexp\ x\ f{\isacharprime}{\kern0pt}\ {\isasymand}\ vars\ f{\isacharprime}{\kern0pt}\ {\isacharequal}{\kern0pt}\ vars\ f\ {\isasymunion}\ {\isacharbraceleft}{\kern0pt}x{\isacharbraceright}{\kern0pt}\ {\isasymand}\isanewline
\ \ \ \ \ \ {\isacharparenleft}{\kern0pt}{\isasymforall}v{\isachardot}{\kern0pt}\ {\isasymPsi}\ {\isacharparenleft}{\kern0pt}eval\ f\ v{\isacharparenright}{\kern0pt}\ {\isacharequal}{\kern0pt}\ {\isasymPsi}\ {\isacharparenleft}{\kern0pt}eval\ f{\isacharprime}{\kern0pt}\ v{\isacharparenright}{\kern0pt}{\isacharparenright}{\kern0pt}{\isachardoublequoteclose}\ \isakeywordONE{by}\isamarkupfalse%
\ auto\isanewline
\ \ \isakeywordONE{then}\isamarkupfalse%
\ \isakeywordTHREE{obtain}\isamarkupfalse%
\ p\ q\ \isakeywordTWO{where}\ p{\isacharunderscore}{\kern0pt}q{\isacharunderscore}{\kern0pt}intro{\isacharcolon}{\kern0pt}\ {\isachardoublequoteopen}reg{\isacharunderscore}{\kern0pt}eval\ p\ {\isasymand}\ reg{\isacharunderscore}{\kern0pt}eval\ q\ {\isasymand}\isanewline
\ \ \ \ f{\isacharprime}{\kern0pt}\ {\isacharequal}{\kern0pt}\ Union\ p\ {\isacharparenleft}{\kern0pt}Concat\ q\ {\isacharparenleft}{\kern0pt}Var\ x{\isacharparenright}{\kern0pt}{\isacharparenright}{\kern0pt}\ {\isasymand}\ {\isacharparenleft}{\kern0pt}{\isasymforall}y\ {\isasymin}\ vars\ p{\isachardot}{\kern0pt}\ y\ {\isasymnoteq}\ x{\isacharparenright}{\kern0pt}{\isachardoublequoteclose}\ \isakeywordONE{unfolding}\isamarkupfalse%
\ bipart{\isacharunderscore}{\kern0pt}rlexp{\isacharunderscore}{\kern0pt}def\ \isakeywordONE{by}\isamarkupfalse%
\ auto\isanewline
\ \ \isakeywordONE{let}\isamarkupfalse%
\ {\isacharquery}{\kern0pt}q{\isacharunderscore}{\kern0pt}new\ {\isacharequal}{\kern0pt}\ {\isachardoublequoteopen}Concat\ {\isacharparenleft}{\kern0pt}Star\ p{\isacharparenright}{\kern0pt}\ {\isacharparenleft}{\kern0pt}Concat\ {\isacharparenleft}{\kern0pt}Star\ {\isacharparenleft}{\kern0pt}Concat\ q\ {\isacharparenleft}{\kern0pt}Var\ x{\isacharparenright}{\kern0pt}{\isacharparenright}{\kern0pt}{\isacharparenright}{\kern0pt}\ {\isacharparenleft}{\kern0pt}Concat\ {\isacharparenleft}{\kern0pt}Star\ {\isacharparenleft}{\kern0pt}Concat\ q\ {\isacharparenleft}{\kern0pt}Var\ x{\isacharparenright}{\kern0pt}{\isacharparenright}{\kern0pt}{\isacharparenright}{\kern0pt}\ q{\isacharparenright}{\kern0pt}{\isacharparenright}{\kern0pt}{\isachardoublequoteclose}\isanewline
\ \ \isakeywordONE{let}\isamarkupfalse%
\ {\isacharquery}{\kern0pt}f{\isacharunderscore}{\kern0pt}new\ {\isacharequal}{\kern0pt}\ {\isachardoublequoteopen}Union\ {\isacharparenleft}{\kern0pt}Star\ p{\isacharparenright}{\kern0pt}\ {\isacharparenleft}{\kern0pt}Concat\ {\isacharquery}{\kern0pt}q{\isacharunderscore}{\kern0pt}new\ {\isacharparenleft}{\kern0pt}Var\ x{\isacharparenright}{\kern0pt}{\isacharparenright}{\kern0pt}{\isachardoublequoteclose}\isanewline
\ \ \isakeywordONE{have}\isamarkupfalse%
\ {\isachardoublequoteopen}{\isasymforall}v{\isachardot}{\kern0pt}\ {\isacharparenleft}{\kern0pt}{\isasymPsi}\ {\isacharparenleft}{\kern0pt}eval\ {\isacharparenleft}{\kern0pt}Star\ f{\isacharparenright}{\kern0pt}\ v{\isacharparenright}{\kern0pt}\ {\isacharequal}{\kern0pt}\ {\isasymPsi}\ {\isacharparenleft}{\kern0pt}eval\ {\isacharquery}{\kern0pt}f{\isacharunderscore}{\kern0pt}new\ v{\isacharparenright}{\kern0pt}{\isacharparenright}{\kern0pt}{\isachardoublequoteclose}\isanewline
\ \ \isakeywordONE{proof}\isamarkupfalse%
\ {\isacharparenleft}{\kern0pt}rule\ allI{\isacharparenright}{\kern0pt}\isanewline
\ \ \ \ \isakeywordTHREE{fix}\isamarkupfalse%
\ v\isanewline
\ \ \ \ \isakeywordONE{have}\isamarkupfalse%
\ {\isachardoublequoteopen}{\isasymPsi}\ {\isacharparenleft}{\kern0pt}eval\ {\isacharparenleft}{\kern0pt}Star\ f{\isacharparenright}{\kern0pt}\ v{\isacharparenright}{\kern0pt}\ {\isacharequal}{\kern0pt}\ {\isasymPsi}\ {\isacharparenleft}{\kern0pt}star\ {\isacharparenleft}{\kern0pt}eval\ p\ v\ {\isasymunion}\ eval\ q\ v\ {\isacharat}{\kern0pt}{\isacharat}{\kern0pt}\ v\ x{\isacharparenright}{\kern0pt}{\isacharparenright}{\kern0pt}{\isachardoublequoteclose}\isanewline
\ \ \ \ \ \ \isakeywordONE{using}\isamarkupfalse%
\ f{\isacharprime}{\kern0pt}{\isacharunderscore}{\kern0pt}intro\ parikh{\isacharunderscore}{\kern0pt}star{\isacharunderscore}{\kern0pt}mono{\isacharunderscore}{\kern0pt}eq\ p{\isacharunderscore}{\kern0pt}q{\isacharunderscore}{\kern0pt}intro\isanewline
\ \ \ \ \ \ \isakeywordONE{by}\isamarkupfalse%
\ {\isacharparenleft}{\kern0pt}metis\ eval{\isachardot}{\kern0pt}simps{\isacharparenleft}{\kern0pt}{\isadigit{1}}{\isacharparenright}{\kern0pt}\ eval{\isachardot}{\kern0pt}simps{\isacharparenleft}{\kern0pt}{\isadigit{3}}{\isacharparenright}{\kern0pt}\ eval{\isachardot}{\kern0pt}simps{\isacharparenleft}{\kern0pt}{\isadigit{4}}{\isacharparenright}{\kern0pt}\ eval{\isachardot}{\kern0pt}simps{\isacharparenleft}{\kern0pt}{\isadigit{5}}{\isacharparenright}{\kern0pt}{\isacharparenright}{\kern0pt}\isanewline
\ \ \ \ \isakeywordONE{also}\isamarkupfalse%
\ \isakeywordONE{have}\isamarkupfalse%
\ {\isachardoublequoteopen}{\isasymdots}\ {\isacharequal}{\kern0pt}\ {\isasymPsi}\ {\isacharparenleft}{\kern0pt}star\ {\isacharparenleft}{\kern0pt}eval\ p\ v{\isacharparenright}{\kern0pt}\ {\isacharat}{\kern0pt}{\isacharat}{\kern0pt}\ star\ {\isacharparenleft}{\kern0pt}eval\ q\ v\ {\isacharat}{\kern0pt}{\isacharat}{\kern0pt}\ v\ x{\isacharparenright}{\kern0pt}{\isacharparenright}{\kern0pt}{\isachardoublequoteclose}\isanewline
\ \ \ \ \ \ \isakeywordONE{using}\isamarkupfalse%
\ parikh{\isacharunderscore}{\kern0pt}img{\isacharunderscore}{\kern0pt}star\ \isakeywordONE{by}\isamarkupfalse%
\ blast\isanewline
\ \ \ \ \isakeywordONE{also}\isamarkupfalse%
\ \isakeywordONE{have}\isamarkupfalse%
\ {\isachardoublequoteopen}{\isasymdots}\ {\isacharequal}{\kern0pt}\ {\isasymPsi}\ {\isacharparenleft}{\kern0pt}star\ {\isacharparenleft}{\kern0pt}eval\ p\ v{\isacharparenright}{\kern0pt}\ {\isacharat}{\kern0pt}{\isacharat}{\kern0pt}\isanewline
\ \ \ \ \ \ \ \ star\ {\isacharparenleft}{\kern0pt}{\isacharbraceleft}{\kern0pt}{\isacharbrackleft}{\kern0pt}{\isacharbrackright}{\kern0pt}{\isacharbraceright}{\kern0pt}\ {\isasymunion}\ star\ {\isacharparenleft}{\kern0pt}eval\ q\ v\ {\isacharat}{\kern0pt}{\isacharat}{\kern0pt}\ v\ x{\isacharparenright}{\kern0pt}\ {\isacharat}{\kern0pt}{\isacharat}{\kern0pt}\ eval\ q\ v\ {\isacharat}{\kern0pt}{\isacharat}{\kern0pt}\ v\ x{\isacharparenright}{\kern0pt}{\isacharparenright}{\kern0pt}{\isachardoublequoteclose}\isanewline
\ \ \ \ \ \ \isakeywordONE{by}\isamarkupfalse%
\ {\isacharparenleft}{\kern0pt}metis\ Un{\isacharunderscore}{\kern0pt}commute\ conc{\isacharunderscore}{\kern0pt}star{\isacharunderscore}{\kern0pt}comm\ star{\isacharunderscore}{\kern0pt}idemp\ star{\isacharunderscore}{\kern0pt}unfold{\isacharunderscore}{\kern0pt}left{\isacharparenright}{\kern0pt}\isanewline
\ \ \ \ \isakeywordONE{also}\isamarkupfalse%
\ \isakeywordONE{have}\isamarkupfalse%
\ {\isachardoublequoteopen}{\isasymdots}\ {\isacharequal}{\kern0pt}\ {\isasymPsi}\ {\isacharparenleft}{\kern0pt}star\ {\isacharparenleft}{\kern0pt}eval\ p\ v{\isacharparenright}{\kern0pt}\ {\isacharat}{\kern0pt}{\isacharat}{\kern0pt}\ star\ {\isacharparenleft}{\kern0pt}star\ {\isacharparenleft}{\kern0pt}eval\ q\ v\ {\isacharat}{\kern0pt}{\isacharat}{\kern0pt}\ v\ x{\isacharparenright}{\kern0pt}\ {\isacharat}{\kern0pt}{\isacharat}{\kern0pt}\ eval\ q\ v\ {\isacharat}{\kern0pt}{\isacharat}{\kern0pt}\ v\ x{\isacharparenright}{\kern0pt}{\isacharparenright}{\kern0pt}{\isachardoublequoteclose}\isanewline
\ \ \ \ \ \ \isakeywordONE{by}\isamarkupfalse%
\ auto\isanewline
\ \ \ \ \isakeywordONE{also}\isamarkupfalse%
\ \isakeywordONE{have}\isamarkupfalse%
\ {\isachardoublequoteopen}{\isasymdots}\ {\isacharequal}{\kern0pt}\ {\isasymPsi}\ {\isacharparenleft}{\kern0pt}star\ {\isacharparenleft}{\kern0pt}eval\ p\ v{\isacharparenright}{\kern0pt}\ {\isacharat}{\kern0pt}{\isacharat}{\kern0pt}\ {\isacharparenleft}{\kern0pt}{\isacharbraceleft}{\kern0pt}{\isacharbrackleft}{\kern0pt}{\isacharbrackright}{\kern0pt}{\isacharbraceright}{\kern0pt}\ {\isasymunion}\ star\ {\isacharparenleft}{\kern0pt}eval\ q\ v\ {\isacharat}{\kern0pt}{\isacharat}{\kern0pt}\ v\ x{\isacharparenright}{\kern0pt}\isanewline
\ \ \ \ \ \ \ \ {\isacharat}{\kern0pt}{\isacharat}{\kern0pt}\ star\ {\isacharparenleft}{\kern0pt}eval\ q\ v\ {\isacharat}{\kern0pt}{\isacharat}{\kern0pt}\ v\ x{\isacharparenright}{\kern0pt}\ {\isacharat}{\kern0pt}{\isacharat}{\kern0pt}\ eval\ q\ v\ {\isacharat}{\kern0pt}{\isacharat}{\kern0pt}\ v\ x{\isacharparenright}{\kern0pt}{\isacharparenright}{\kern0pt}{\isachardoublequoteclose}\isanewline
\ \ \ \ \ \ \isakeywordONE{using}\isamarkupfalse%
\ parikh{\isacharunderscore}{\kern0pt}img{\isacharunderscore}{\kern0pt}star{\isadigit{2}}\ parikh{\isacharunderscore}{\kern0pt}conc{\isacharunderscore}{\kern0pt}left\ \isakeywordONE{by}\isamarkupfalse%
\ blast\isanewline
\ \ \ \ \isakeywordONE{also}\isamarkupfalse%
\ \isakeywordONE{have}\isamarkupfalse%
\ {\isachardoublequoteopen}{\isasymdots}\ {\isacharequal}{\kern0pt}\ {\isasymPsi}\ {\isacharparenleft}{\kern0pt}star\ {\isacharparenleft}{\kern0pt}eval\ p\ v{\isacharparenright}{\kern0pt}\ {\isacharat}{\kern0pt}{\isacharat}{\kern0pt}\ {\isacharbraceleft}{\kern0pt}{\isacharbrackleft}{\kern0pt}{\isacharbrackright}{\kern0pt}{\isacharbraceright}{\kern0pt}\ {\isasymunion}\ star\ {\isacharparenleft}{\kern0pt}eval\ p\ v{\isacharparenright}{\kern0pt}\ {\isacharat}{\kern0pt}{\isacharat}{\kern0pt}\ star\ {\isacharparenleft}{\kern0pt}eval\ q\ v\ {\isacharat}{\kern0pt}{\isacharat}{\kern0pt}\ v\ x{\isacharparenright}{\kern0pt}\isanewline
\ \ \ \ \ \ \ \ {\isacharat}{\kern0pt}{\isacharat}{\kern0pt}\ star\ {\isacharparenleft}{\kern0pt}eval\ q\ v\ {\isacharat}{\kern0pt}{\isacharat}{\kern0pt}\ v\ x{\isacharparenright}{\kern0pt}\ {\isacharat}{\kern0pt}{\isacharat}{\kern0pt}\ eval\ q\ v\ {\isacharat}{\kern0pt}{\isacharat}{\kern0pt}\ v\ x{\isacharparenright}{\kern0pt}{\isachardoublequoteclose}\ \isakeywordONE{by}\isamarkupfalse%
\ {\isacharparenleft}{\kern0pt}metis\ conc{\isacharunderscore}{\kern0pt}Un{\isacharunderscore}{\kern0pt}distrib{\isacharparenleft}{\kern0pt}{\isadigit{1}}{\isacharparenright}{\kern0pt}{\isacharparenright}{\kern0pt}\isanewline
\ \ \ \ \isakeywordONE{also}\isamarkupfalse%
\ \isakeywordONE{have}\isamarkupfalse%
\ {\isachardoublequoteopen}{\isasymdots}\ {\isacharequal}{\kern0pt}\ {\isasymPsi}\ {\isacharparenleft}{\kern0pt}eval\ {\isacharquery}{\kern0pt}f{\isacharunderscore}{\kern0pt}new\ v{\isacharparenright}{\kern0pt}{\isachardoublequoteclose}\ \isakeywordONE{by}\isamarkupfalse%
\ {\isacharparenleft}{\kern0pt}simp\ add{\isacharcolon}{\kern0pt}\ conc{\isacharunderscore}{\kern0pt}assoc{\isacharparenright}{\kern0pt}\isanewline
\ \ \ \ \isakeywordONE{finally}\isamarkupfalse%
\ \isakeywordTHREE{show}\isamarkupfalse%
\ {\isachardoublequoteopen}{\isasymPsi}\ {\isacharparenleft}{\kern0pt}eval\ {\isacharparenleft}{\kern0pt}Star\ f{\isacharparenright}{\kern0pt}\ v{\isacharparenright}{\kern0pt}\ {\isacharequal}{\kern0pt}\ {\isasymPsi}\ {\isacharparenleft}{\kern0pt}eval\ {\isacharquery}{\kern0pt}f{\isacharunderscore}{\kern0pt}new\ v{\isacharparenright}{\kern0pt}{\isachardoublequoteclose}\ \isakeywordONE{{\isachardot}{\kern0pt}}\isamarkupfalse%
\isanewline
\ \ \isakeywordONE{qed}\isamarkupfalse%
\isanewline
\ \ \isakeywordONE{moreover}\isamarkupfalse%
\ \isakeywordONE{have}\isamarkupfalse%
\ {\isachardoublequoteopen}bipart{\isacharunderscore}{\kern0pt}rlexp\ x\ {\isacharquery}{\kern0pt}f{\isacharunderscore}{\kern0pt}new{\isachardoublequoteclose}\ \isakeywordONE{unfolding}\isamarkupfalse%
\ bipart{\isacharunderscore}{\kern0pt}rlexp{\isacharunderscore}{\kern0pt}def\ \isakeywordONE{using}\isamarkupfalse%
\ p{\isacharunderscore}{\kern0pt}q{\isacharunderscore}{\kern0pt}intro\ \isakeywordONE{by}\isamarkupfalse%
\ fastforce\isanewline
\ \ \isakeywordONE{moreover}\isamarkupfalse%
\ \isakeywordONE{from}\isamarkupfalse%
\ f{\isacharprime}{\kern0pt}{\isacharunderscore}{\kern0pt}intro\ p{\isacharunderscore}{\kern0pt}q{\isacharunderscore}{\kern0pt}intro\ \isakeywordONE{have}\isamarkupfalse%
\ {\isachardoublequoteopen}vars\ {\isacharquery}{\kern0pt}f{\isacharunderscore}{\kern0pt}new\ {\isacharequal}{\kern0pt}\ vars\ {\isacharparenleft}{\kern0pt}Star\ f{\isacharparenright}{\kern0pt}\ {\isasymunion}\ {\isacharbraceleft}{\kern0pt}x{\isacharbraceright}{\kern0pt}{\isachardoublequoteclose}\ \isakeywordONE{by}\isamarkupfalse%
\ auto\isanewline
\ \ \isakeywordONE{ultimately}\isamarkupfalse%
\ \isakeywordTHREE{show}\isamarkupfalse%
\ {\isacharquery}{\kern0pt}thesis\ \isakeywordONE{by}\isamarkupfalse%
\ metis\isanewline
\isakeywordONE{qed}\isamarkupfalse%
%
\endisatagproof
{\isafoldproof}%
%
\isadelimproof
\isanewline
%
\endisadelimproof
\isanewline
\isakeywordONE{lemma}\isamarkupfalse%
\ reg{\isacharunderscore}{\kern0pt}eval{\isacharunderscore}{\kern0pt}bipart{\isacharunderscore}{\kern0pt}rlexp{\isacharcolon}{\kern0pt}\ {\isachardoublequoteopen}reg{\isacharunderscore}{\kern0pt}eval\ f\ {\isasymLongrightarrow}\isanewline
\ \ \ \ {\isasymexists}f{\isacharprime}{\kern0pt}{\isachardot}{\kern0pt}\ bipart{\isacharunderscore}{\kern0pt}rlexp\ x\ f{\isacharprime}{\kern0pt}\ {\isasymand}\ vars\ f{\isacharprime}{\kern0pt}\ {\isacharequal}{\kern0pt}\ vars\ f\ {\isasymunion}\ {\isacharbraceleft}{\kern0pt}x{\isacharbraceright}{\kern0pt}\ {\isasymand}\isanewline
\ \ \ \ \ \ \ \ \ {\isacharparenleft}{\kern0pt}{\isasymforall}s{\isachardot}{\kern0pt}\ {\isasymPsi}\ {\isacharparenleft}{\kern0pt}eval\ f\ s{\isacharparenright}{\kern0pt}\ {\isacharequal}{\kern0pt}\ {\isasymPsi}\ {\isacharparenleft}{\kern0pt}eval\ f{\isacharprime}{\kern0pt}\ s{\isacharparenright}{\kern0pt}{\isacharparenright}{\kern0pt}{\isachardoublequoteclose}\isanewline
%
\isadelimproof
%
\endisadelimproof
%
\isatagproof
\isakeywordONE{proof}\isamarkupfalse%
\ {\isacharparenleft}{\kern0pt}induction\ f\ rule{\isacharcolon}{\kern0pt}\ reg{\isacharunderscore}{\kern0pt}eval{\isachardot}{\kern0pt}induct{\isacharparenright}{\kern0pt}\isanewline
\ \ \isakeywordTHREE{case}\isamarkupfalse%
\ {\isacharparenleft}{\kern0pt}{\isadigit{1}}\ uu{\isacharparenright}{\kern0pt}\isanewline
\ \ \isakeywordONE{from}\isamarkupfalse%
\ reg{\isacharunderscore}{\kern0pt}eval{\isacharunderscore}{\kern0pt}bipart{\isacharunderscore}{\kern0pt}rlexp{\isacharunderscore}{\kern0pt}Variable\ \isakeywordTHREE{show}\isamarkupfalse%
\ {\isacharquery}{\kern0pt}case\ \isakeywordONE{by}\isamarkupfalse%
\ blast\isanewline
\isakeywordONE{next}\isamarkupfalse%
\isanewline
\ \ \isakeywordTHREE{case}\isamarkupfalse%
\ {\isacharparenleft}{\kern0pt}{\isadigit{2}}\ l{\isacharparenright}{\kern0pt}\isanewline
\ \ \isakeywordONE{then}\isamarkupfalse%
\ \isakeywordONE{have}\isamarkupfalse%
\ {\isachardoublequoteopen}regular{\isacharunderscore}{\kern0pt}lang\ l{\isachardoublequoteclose}\ \isakeywordONE{by}\isamarkupfalse%
\ simp\isanewline
\ \ \isakeywordONE{from}\isamarkupfalse%
\ reg{\isacharunderscore}{\kern0pt}eval{\isacharunderscore}{\kern0pt}bipart{\isacharunderscore}{\kern0pt}rlexp{\isacharunderscore}{\kern0pt}Const{\isacharbrackleft}{\kern0pt}OF\ this{\isacharbrackright}{\kern0pt}\ \isakeywordTHREE{show}\isamarkupfalse%
\ {\isacharquery}{\kern0pt}case\ \isakeywordONE{by}\isamarkupfalse%
\ blast\isanewline
\isakeywordONE{next}\isamarkupfalse%
\isanewline
\ \ \isakeywordTHREE{case}\isamarkupfalse%
\ {\isacharparenleft}{\kern0pt}{\isadigit{3}}\ f\ g{\isacharparenright}{\kern0pt}\isanewline
\ \ \isakeywordONE{then}\isamarkupfalse%
\ \isakeywordONE{have}\isamarkupfalse%
\ {\isachardoublequoteopen}{\isasymexists}f{\isacharprime}{\kern0pt}{\isachardot}{\kern0pt}\ bipart{\isacharunderscore}{\kern0pt}rlexp\ x\ f{\isacharprime}{\kern0pt}\ {\isasymand}\ vars\ f{\isacharprime}{\kern0pt}\ {\isacharequal}{\kern0pt}\ vars\ f\ {\isasymunion}\ {\isacharbraceleft}{\kern0pt}x{\isacharbraceright}{\kern0pt}\ {\isasymand}\ {\isacharparenleft}{\kern0pt}{\isasymforall}v{\isachardot}{\kern0pt}\ {\isasymPsi}\ {\isacharparenleft}{\kern0pt}eval\ f\ v{\isacharparenright}{\kern0pt}\ {\isacharequal}{\kern0pt}\ {\isasymPsi}\ {\isacharparenleft}{\kern0pt}eval\ f{\isacharprime}{\kern0pt}\ v{\isacharparenright}{\kern0pt}{\isacharparenright}{\kern0pt}{\isachardoublequoteclose}\isanewline
\ \ \ \ \ \ \ \ \ \ \ \ {\isachardoublequoteopen}{\isasymexists}f{\isacharprime}{\kern0pt}{\isachardot}{\kern0pt}\ bipart{\isacharunderscore}{\kern0pt}rlexp\ x\ f{\isacharprime}{\kern0pt}\ {\isasymand}\ vars\ f{\isacharprime}{\kern0pt}\ {\isacharequal}{\kern0pt}\ vars\ g\ {\isasymunion}\ {\isacharbraceleft}{\kern0pt}x{\isacharbraceright}{\kern0pt}\ {\isasymand}\ {\isacharparenleft}{\kern0pt}{\isasymforall}v{\isachardot}{\kern0pt}\ {\isasymPsi}\ {\isacharparenleft}{\kern0pt}eval\ g\ v{\isacharparenright}{\kern0pt}\ {\isacharequal}{\kern0pt}\ {\isasymPsi}\ {\isacharparenleft}{\kern0pt}eval\ f{\isacharprime}{\kern0pt}\ v{\isacharparenright}{\kern0pt}{\isacharparenright}{\kern0pt}{\isachardoublequoteclose}\isanewline
\ \ \ \ \isakeywordONE{by}\isamarkupfalse%
\ auto\isanewline
\ \ \isakeywordONE{from}\isamarkupfalse%
\ reg{\isacharunderscore}{\kern0pt}eval{\isacharunderscore}{\kern0pt}bipart{\isacharunderscore}{\kern0pt}rlexp{\isacharunderscore}{\kern0pt}Union{\isacharbrackleft}{\kern0pt}OF\ this{\isacharbrackright}{\kern0pt}\ \isakeywordTHREE{show}\isamarkupfalse%
\ {\isacharquery}{\kern0pt}case\ \isakeywordONE{by}\isamarkupfalse%
\ blast\isanewline
\isakeywordONE{next}\isamarkupfalse%
\isanewline
\ \ \isakeywordTHREE{case}\isamarkupfalse%
\ {\isacharparenleft}{\kern0pt}{\isadigit{4}}\ f\ g{\isacharparenright}{\kern0pt}\isanewline
\ \ \isakeywordONE{then}\isamarkupfalse%
\ \isakeywordONE{have}\isamarkupfalse%
\ {\isachardoublequoteopen}{\isasymexists}f{\isacharprime}{\kern0pt}{\isachardot}{\kern0pt}\ bipart{\isacharunderscore}{\kern0pt}rlexp\ x\ f{\isacharprime}{\kern0pt}\ {\isasymand}\ vars\ f{\isacharprime}{\kern0pt}\ {\isacharequal}{\kern0pt}\ vars\ f\ {\isasymunion}\ {\isacharbraceleft}{\kern0pt}x{\isacharbraceright}{\kern0pt}\ {\isasymand}\ {\isacharparenleft}{\kern0pt}{\isasymforall}v{\isachardot}{\kern0pt}\ {\isasymPsi}\ {\isacharparenleft}{\kern0pt}eval\ f\ v{\isacharparenright}{\kern0pt}\ {\isacharequal}{\kern0pt}\ {\isasymPsi}\ {\isacharparenleft}{\kern0pt}eval\ f{\isacharprime}{\kern0pt}\ v{\isacharparenright}{\kern0pt}{\isacharparenright}{\kern0pt}{\isachardoublequoteclose}\isanewline
\ \ \ \ \ \ \ \ \ \ \ \ {\isachardoublequoteopen}{\isasymexists}f{\isacharprime}{\kern0pt}{\isachardot}{\kern0pt}\ bipart{\isacharunderscore}{\kern0pt}rlexp\ x\ f{\isacharprime}{\kern0pt}\ {\isasymand}\ vars\ f{\isacharprime}{\kern0pt}\ {\isacharequal}{\kern0pt}\ vars\ g\ {\isasymunion}\ {\isacharbraceleft}{\kern0pt}x{\isacharbraceright}{\kern0pt}\ {\isasymand}\ {\isacharparenleft}{\kern0pt}{\isasymforall}v{\isachardot}{\kern0pt}\ {\isasymPsi}\ {\isacharparenleft}{\kern0pt}eval\ g\ v{\isacharparenright}{\kern0pt}\ {\isacharequal}{\kern0pt}\ {\isasymPsi}\ {\isacharparenleft}{\kern0pt}eval\ f{\isacharprime}{\kern0pt}\ v{\isacharparenright}{\kern0pt}{\isacharparenright}{\kern0pt}{\isachardoublequoteclose}\isanewline
\ \ \ \ \isakeywordONE{by}\isamarkupfalse%
\ auto\isanewline
\ \ \isakeywordONE{from}\isamarkupfalse%
\ reg{\isacharunderscore}{\kern0pt}eval{\isacharunderscore}{\kern0pt}bipart{\isacharunderscore}{\kern0pt}rlexp{\isacharunderscore}{\kern0pt}Concat{\isacharbrackleft}{\kern0pt}OF\ this{\isacharbrackright}{\kern0pt}\ \isakeywordTHREE{show}\isamarkupfalse%
\ {\isacharquery}{\kern0pt}case\ \isakeywordONE{by}\isamarkupfalse%
\ blast\isanewline
\isakeywordONE{next}\isamarkupfalse%
\isanewline
\ \ \isakeywordTHREE{case}\isamarkupfalse%
\ {\isacharparenleft}{\kern0pt}{\isadigit{5}}\ f{\isacharparenright}{\kern0pt}\isanewline
\ \ \isakeywordONE{then}\isamarkupfalse%
\ \isakeywordONE{have}\isamarkupfalse%
\ {\isachardoublequoteopen}{\isasymexists}f{\isacharprime}{\kern0pt}{\isachardot}{\kern0pt}\ bipart{\isacharunderscore}{\kern0pt}rlexp\ x\ f{\isacharprime}{\kern0pt}\ {\isasymand}\ vars\ f{\isacharprime}{\kern0pt}\ {\isacharequal}{\kern0pt}\ vars\ f\ {\isasymunion}\ {\isacharbraceleft}{\kern0pt}x{\isacharbraceright}{\kern0pt}\ {\isasymand}\ {\isacharparenleft}{\kern0pt}{\isasymforall}v{\isachardot}{\kern0pt}\ {\isasymPsi}\ {\isacharparenleft}{\kern0pt}eval\ f\ v{\isacharparenright}{\kern0pt}\ {\isacharequal}{\kern0pt}\ {\isasymPsi}\ {\isacharparenleft}{\kern0pt}eval\ f{\isacharprime}{\kern0pt}\ v{\isacharparenright}{\kern0pt}{\isacharparenright}{\kern0pt}{\isachardoublequoteclose}\isanewline
\ \ \ \ \isakeywordONE{by}\isamarkupfalse%
\ auto\isanewline
\ \ \isakeywordONE{from}\isamarkupfalse%
\ reg{\isacharunderscore}{\kern0pt}eval{\isacharunderscore}{\kern0pt}bipart{\isacharunderscore}{\kern0pt}rlexp{\isacharunderscore}{\kern0pt}Star{\isacharbrackleft}{\kern0pt}OF\ this{\isacharbrackright}{\kern0pt}\ \isakeywordTHREE{show}\isamarkupfalse%
\ {\isacharquery}{\kern0pt}case\ \isakeywordONE{by}\isamarkupfalse%
\ blast\isanewline
\isakeywordONE{qed}\isamarkupfalse%
%
\endisatagproof
{\isafoldproof}%
%
\isadelimproof
%
\endisadelimproof
%
\isadelimdocument
%
\endisadelimdocument
%
\isatagdocument
%
\isamarkupsubsection{Minimal solution for a single equation%
}
\isamarkuptrue%
%
\endisatagdocument
{\isafolddocument}%
%
\isadelimdocument
%
\endisadelimdocument
%
\begin{isamarkuptext}%
The aim is to prove that every system of \isa{\isaconst{reg{\isacharunderscore}{\kern0pt}eval}} equations of the second type
has some minimal solution which is \isa{\isaconst{reg{\isacharunderscore}{\kern0pt}eval}}. In this section, we prove this property
only for the case of a single equation. First we assume that the equation is bipartite but later
in this section we will abandon this assumption.%
\end{isamarkuptext}\isamarkuptrue%
\isakeywordONE{locale}\isamarkupfalse%
\ single{\isacharunderscore}{\kern0pt}bipartite{\isacharunderscore}{\kern0pt}eq\ {\isacharequal}{\kern0pt}\isanewline
\ \ \isakeywordTWO{fixes}\ x\ {\isacharcolon}{\kern0pt}{\isacharcolon}{\kern0pt}\ {\isachardoublequoteopen}nat{\isachardoublequoteclose}\isanewline
\ \ \isakeywordTWO{fixes}\ p\ {\isacharcolon}{\kern0pt}{\isacharcolon}{\kern0pt}\ {\isachardoublequoteopen}{\isacharprime}{\kern0pt}a\ rlexp{\isachardoublequoteclose}\isanewline
\ \ \isakeywordTWO{fixes}\ q\ {\isacharcolon}{\kern0pt}{\isacharcolon}{\kern0pt}\ {\isachardoublequoteopen}{\isacharprime}{\kern0pt}a\ rlexp{\isachardoublequoteclose}\isanewline
\ \ \isakeywordTWO{assumes}\ p{\isacharunderscore}{\kern0pt}reg{\isacharcolon}{\kern0pt}\ \ \ \ \ \ {\isachardoublequoteopen}reg{\isacharunderscore}{\kern0pt}eval\ p{\isachardoublequoteclose}\isanewline
\ \ \isakeywordTWO{assumes}\ q{\isacharunderscore}{\kern0pt}reg{\isacharcolon}{\kern0pt}\ \ \ \ \ \ {\isachardoublequoteopen}reg{\isacharunderscore}{\kern0pt}eval\ q{\isachardoublequoteclose}\isanewline
\ \ \isakeywordTWO{assumes}\ x{\isacharunderscore}{\kern0pt}not{\isacharunderscore}{\kern0pt}in{\isacharunderscore}{\kern0pt}p{\isacharcolon}{\kern0pt}\ {\isachardoublequoteopen}x\ {\isasymnotin}\ vars\ p{\isachardoublequoteclose}\isanewline
\isakeywordTWO{begin}%
\begin{isamarkuptext}%
The equation and the minimal solution look as follows. Here, \isa{x} describes the variable whose
solution is to be determined. In the subsequent lemmas, we prove that the solution is \isa{\isaconst{reg{\isacharunderscore}{\kern0pt}eval}}
and fulfills each of the three conditions of the predicate \isa{\isaconst{partial{\isacharunderscore}{\kern0pt}min{\isacharunderscore}{\kern0pt}sol{\isacharunderscore}{\kern0pt}one{\isacharunderscore}{\kern0pt}ineq}}.
In particular, we will use the lemmas of the sections 2.5 and 2.6 here:%
\end{isamarkuptext}\isamarkuptrue%
\isakeywordONE{abbreviation}\isamarkupfalse%
\ {\isachardoublequoteopen}eq\ {\isasymequiv}\ Union\ p\ {\isacharparenleft}{\kern0pt}Concat\ q\ {\isacharparenleft}{\kern0pt}Var\ x{\isacharparenright}{\kern0pt}{\isacharparenright}{\kern0pt}{\isachardoublequoteclose}\isanewline
\isakeywordONE{abbreviation}\isamarkupfalse%
\ {\isachardoublequoteopen}sol\ {\isasymequiv}\ Concat\ {\isacharparenleft}{\kern0pt}Star\ {\isacharparenleft}{\kern0pt}subst\ {\isacharparenleft}{\kern0pt}Var{\isacharparenleft}{\kern0pt}x\ {\isacharcolon}{\kern0pt}{\isacharequal}{\kern0pt}\ p{\isacharparenright}{\kern0pt}{\isacharparenright}{\kern0pt}\ q{\isacharparenright}{\kern0pt}{\isacharparenright}{\kern0pt}\ p{\isachardoublequoteclose}\isanewline
\isanewline
\isakeywordONE{lemma}\isamarkupfalse%
\ sol{\isacharunderscore}{\kern0pt}is{\isacharunderscore}{\kern0pt}reg{\isacharcolon}{\kern0pt}\ {\isachardoublequoteopen}reg{\isacharunderscore}{\kern0pt}eval\ sol{\isachardoublequoteclose}\isanewline
%
\isadelimproof
%
\endisadelimproof
%
\isatagproof
\isakeywordONE{proof}\isamarkupfalse%
\ {\isacharminus}{\kern0pt}\isanewline
\ \ \isakeywordONE{from}\isamarkupfalse%
\ p{\isacharunderscore}{\kern0pt}reg\ q{\isacharunderscore}{\kern0pt}reg\ \isakeywordONE{have}\isamarkupfalse%
\ r{\isacharunderscore}{\kern0pt}reg{\isacharcolon}{\kern0pt}\ {\isachardoublequoteopen}reg{\isacharunderscore}{\kern0pt}eval\ {\isacharparenleft}{\kern0pt}subst\ {\isacharparenleft}{\kern0pt}Var{\isacharparenleft}{\kern0pt}x\ {\isacharcolon}{\kern0pt}{\isacharequal}{\kern0pt}\ p{\isacharparenright}{\kern0pt}{\isacharparenright}{\kern0pt}\ q{\isacharparenright}{\kern0pt}{\isachardoublequoteclose}\isanewline
\ \ \ \ \isakeywordONE{using}\isamarkupfalse%
\ subst{\isacharunderscore}{\kern0pt}reg{\isacharunderscore}{\kern0pt}eval{\isacharunderscore}{\kern0pt}update\ \isakeywordONE{by}\isamarkupfalse%
\ auto\isanewline
\ \ \isakeywordONE{with}\isamarkupfalse%
\ p{\isacharunderscore}{\kern0pt}reg\ \isakeywordTHREE{show}\isamarkupfalse%
\ {\isachardoublequoteopen}reg{\isacharunderscore}{\kern0pt}eval\ sol{\isachardoublequoteclose}\ \isakeywordONE{by}\isamarkupfalse%
\ auto\isanewline
\isakeywordONE{qed}\isamarkupfalse%
%
\endisatagproof
{\isafoldproof}%
%
\isadelimproof
\isanewline
%
\endisadelimproof
\isanewline
\isakeywordONE{lemma}\isamarkupfalse%
\ sol{\isacharunderscore}{\kern0pt}vars{\isacharcolon}{\kern0pt}\ {\isachardoublequoteopen}vars\ sol\ {\isasymsubseteq}\ vars\ eq\ {\isacharminus}{\kern0pt}\ {\isacharbraceleft}{\kern0pt}x{\isacharbraceright}{\kern0pt}{\isachardoublequoteclose}\isanewline
%
\isadelimproof
%
\endisadelimproof
%
\isatagproof
\isakeywordONE{proof}\isamarkupfalse%
\ {\isacharminus}{\kern0pt}\isanewline
\ \ \isakeywordONE{let}\isamarkupfalse%
\ {\isacharquery}{\kern0pt}upd\ {\isacharequal}{\kern0pt}\ {\isachardoublequoteopen}Var{\isacharparenleft}{\kern0pt}x\ {\isacharcolon}{\kern0pt}{\isacharequal}{\kern0pt}\ p{\isacharparenright}{\kern0pt}{\isachardoublequoteclose}\isanewline
\ \ \isakeywordONE{let}\isamarkupfalse%
\ {\isacharquery}{\kern0pt}subst{\isacharunderscore}{\kern0pt}q\ {\isacharequal}{\kern0pt}\ {\isachardoublequoteopen}subst\ {\isacharquery}{\kern0pt}upd\ q{\isachardoublequoteclose}\isanewline
\ \ \isakeywordONE{from}\isamarkupfalse%
\ x{\isacharunderscore}{\kern0pt}not{\isacharunderscore}{\kern0pt}in{\isacharunderscore}{\kern0pt}p\ \isakeywordONE{have}\isamarkupfalse%
\ vars{\isacharunderscore}{\kern0pt}p{\isacharcolon}{\kern0pt}\ {\isachardoublequoteopen}vars\ p\ {\isasymsubseteq}\ vars\ eq\ {\isacharminus}{\kern0pt}\ {\isacharbraceleft}{\kern0pt}x{\isacharbraceright}{\kern0pt}{\isachardoublequoteclose}\ \isakeywordONE{by}\isamarkupfalse%
\ fastforce\isanewline
\ \ \isakeywordONE{moreover}\isamarkupfalse%
\ \isakeywordONE{have}\isamarkupfalse%
\ {\isachardoublequoteopen}vars\ p\ {\isasymunion}\ vars\ q\ {\isasymsubseteq}\ vars\ eq{\isachardoublequoteclose}\ \isakeywordONE{by}\isamarkupfalse%
\ auto\isanewline
\ \ \isakeywordONE{ultimately}\isamarkupfalse%
\ \isakeywordONE{have}\isamarkupfalse%
\ {\isachardoublequoteopen}vars\ {\isacharquery}{\kern0pt}subst{\isacharunderscore}{\kern0pt}q\ {\isasymsubseteq}\ vars\ eq\ {\isacharminus}{\kern0pt}\ {\isacharbraceleft}{\kern0pt}x{\isacharbraceright}{\kern0pt}{\isachardoublequoteclose}\ \isakeywordONE{using}\isamarkupfalse%
\ vars{\isacharunderscore}{\kern0pt}subst{\isacharunderscore}{\kern0pt}upd{\isacharunderscore}{\kern0pt}upper\ \isakeywordONE{by}\isamarkupfalse%
\ blast\isanewline
\ \ \isakeywordONE{with}\isamarkupfalse%
\ x{\isacharunderscore}{\kern0pt}not{\isacharunderscore}{\kern0pt}in{\isacharunderscore}{\kern0pt}p\ \isakeywordTHREE{show}\isamarkupfalse%
\ {\isacharquery}{\kern0pt}thesis\ \isakeywordONE{by}\isamarkupfalse%
\ auto\isanewline
\isakeywordONE{qed}\isamarkupfalse%
%
\endisatagproof
{\isafoldproof}%
%
\isadelimproof
\isanewline
%
\endisadelimproof
\isanewline
\isakeywordONE{lemma}\isamarkupfalse%
\ sol{\isacharunderscore}{\kern0pt}is{\isacharunderscore}{\kern0pt}sol{\isacharunderscore}{\kern0pt}ineq{\isacharcolon}{\kern0pt}\ {\isachardoublequoteopen}partial{\isacharunderscore}{\kern0pt}sol{\isacharunderscore}{\kern0pt}ineq\ x\ eq\ sol{\isachardoublequoteclose}\isanewline
%
\isadelimproof
%
\endisadelimproof
%
\isatagproof
\isakeywordONE{unfolding}\isamarkupfalse%
\ partial{\isacharunderscore}{\kern0pt}sol{\isacharunderscore}{\kern0pt}ineq{\isacharunderscore}{\kern0pt}def\ \isakeywordONE{proof}\isamarkupfalse%
\ {\isacharparenleft}{\kern0pt}rule\ allI{\isacharcomma}{\kern0pt}\ rule\ impI{\isacharparenright}{\kern0pt}\isanewline
\ \ \isakeywordTHREE{fix}\isamarkupfalse%
\ v\isanewline
\ \ \isakeywordTHREE{assume}\isamarkupfalse%
\ x{\isacharunderscore}{\kern0pt}is{\isacharunderscore}{\kern0pt}sol{\isacharcolon}{\kern0pt}\ {\isachardoublequoteopen}v\ x\ {\isacharequal}{\kern0pt}\ eval\ sol\ v{\isachardoublequoteclose}\isanewline
\ \ \isakeywordONE{let}\isamarkupfalse%
\ {\isacharquery}{\kern0pt}r\ {\isacharequal}{\kern0pt}\ {\isachardoublequoteopen}subst\ {\isacharparenleft}{\kern0pt}Var\ {\isacharparenleft}{\kern0pt}x\ {\isacharcolon}{\kern0pt}{\isacharequal}{\kern0pt}\ p{\isacharparenright}{\kern0pt}{\isacharparenright}{\kern0pt}\ q{\isachardoublequoteclose}\isanewline
\ \ \isakeywordONE{let}\isamarkupfalse%
\ {\isacharquery}{\kern0pt}upd\ {\isacharequal}{\kern0pt}\ {\isachardoublequoteopen}Var{\isacharparenleft}{\kern0pt}x\ {\isacharcolon}{\kern0pt}{\isacharequal}{\kern0pt}\ sol{\isacharparenright}{\kern0pt}{\isachardoublequoteclose}\isanewline
\ \ \isakeywordONE{let}\isamarkupfalse%
\ {\isacharquery}{\kern0pt}q{\isacharunderscore}{\kern0pt}subst\ {\isacharequal}{\kern0pt}\ {\isachardoublequoteopen}subst\ {\isacharquery}{\kern0pt}upd\ q{\isachardoublequoteclose}\isanewline
\ \ \isakeywordONE{let}\isamarkupfalse%
\ {\isacharquery}{\kern0pt}eq{\isacharunderscore}{\kern0pt}subst\ {\isacharequal}{\kern0pt}\ {\isachardoublequoteopen}subst\ {\isacharquery}{\kern0pt}upd\ eq{\isachardoublequoteclose}\isanewline
\ \ \isakeywordONE{have}\isamarkupfalse%
\ homogeneous{\isacharunderscore}{\kern0pt}app{\isacharcolon}{\kern0pt}\ {\isachardoublequoteopen}{\isasymPsi}\ {\isacharparenleft}{\kern0pt}eval\ {\isacharquery}{\kern0pt}q{\isacharunderscore}{\kern0pt}subst\ v{\isacharparenright}{\kern0pt}\ {\isasymsubseteq}\ {\isasymPsi}\ {\isacharparenleft}{\kern0pt}eval\ {\isacharparenleft}{\kern0pt}Concat\ {\isacharparenleft}{\kern0pt}Star\ {\isacharquery}{\kern0pt}r{\isacharparenright}{\kern0pt}\ {\isacharquery}{\kern0pt}r{\isacharparenright}{\kern0pt}\ v{\isacharparenright}{\kern0pt}{\isachardoublequoteclose}\isanewline
\ \ \ \ \isakeywordONE{using}\isamarkupfalse%
\ rlexp{\isacharunderscore}{\kern0pt}homogeneous\ \isakeywordONE{by}\isamarkupfalse%
\ blast\isanewline
\ \ \isakeywordONE{from}\isamarkupfalse%
\ x{\isacharunderscore}{\kern0pt}not{\isacharunderscore}{\kern0pt}in{\isacharunderscore}{\kern0pt}p\ \isakeywordONE{have}\isamarkupfalse%
\ {\isachardoublequoteopen}eval\ {\isacharparenleft}{\kern0pt}subst\ {\isacharquery}{\kern0pt}upd\ p{\isacharparenright}{\kern0pt}\ v\ {\isacharequal}{\kern0pt}\ eval\ p\ v{\isachardoublequoteclose}\ \isakeywordONE{using}\isamarkupfalse%
\ eval{\isacharunderscore}{\kern0pt}vars{\isacharunderscore}{\kern0pt}subst{\isacharbrackleft}{\kern0pt}of\ p{\isacharbrackright}{\kern0pt}\ \isakeywordONE{by}\isamarkupfalse%
\ simp\isanewline
\ \ \isakeywordONE{then}\isamarkupfalse%
\ \isakeywordONE{have}\isamarkupfalse%
\ {\isachardoublequoteopen}{\isasymPsi}\ {\isacharparenleft}{\kern0pt}eval\ {\isacharquery}{\kern0pt}eq{\isacharunderscore}{\kern0pt}subst\ v{\isacharparenright}{\kern0pt}\ {\isacharequal}{\kern0pt}\ {\isasymPsi}\ {\isacharparenleft}{\kern0pt}eval\ p\ v\ {\isasymunion}\ eval\ {\isacharquery}{\kern0pt}q{\isacharunderscore}{\kern0pt}subst\ v\ {\isacharat}{\kern0pt}{\isacharat}{\kern0pt}\ eval\ sol\ v{\isacharparenright}{\kern0pt}{\isachardoublequoteclose}\isanewline
\ \ \ \ \isakeywordONE{by}\isamarkupfalse%
\ simp\isanewline
\ \ \isakeywordONE{also}\isamarkupfalse%
\ \isakeywordONE{have}\isamarkupfalse%
\ {\isachardoublequoteopen}{\isasymdots}\ {\isasymsubseteq}\ {\isasymPsi}\ {\isacharparenleft}{\kern0pt}eval\ p\ v\ {\isasymunion}\ eval\ {\isacharparenleft}{\kern0pt}Concat\ {\isacharparenleft}{\kern0pt}Star\ {\isacharquery}{\kern0pt}r{\isacharparenright}{\kern0pt}\ {\isacharquery}{\kern0pt}r{\isacharparenright}{\kern0pt}\ v\ {\isacharat}{\kern0pt}{\isacharat}{\kern0pt}\ eval\ sol\ v{\isacharparenright}{\kern0pt}{\isachardoublequoteclose}\isanewline
\ \ \ \ \isakeywordONE{using}\isamarkupfalse%
\ homogeneous{\isacharunderscore}{\kern0pt}app\ \isakeywordONE{by}\isamarkupfalse%
\ {\isacharparenleft}{\kern0pt}metis\ dual{\isacharunderscore}{\kern0pt}order{\isachardot}{\kern0pt}refl\ parikh{\isacharunderscore}{\kern0pt}conc{\isacharunderscore}{\kern0pt}right{\isacharunderscore}{\kern0pt}subset\ parikh{\isacharunderscore}{\kern0pt}img{\isacharunderscore}{\kern0pt}Un\ sup{\isachardot}{\kern0pt}mono{\isacharparenright}{\kern0pt}\isanewline
\ \ \isakeywordONE{also}\isamarkupfalse%
\ \isakeywordONE{have}\isamarkupfalse%
\ {\isachardoublequoteopen}{\isasymdots}\ {\isacharequal}{\kern0pt}\ {\isasymPsi}\ {\isacharparenleft}{\kern0pt}eval\ p\ v{\isacharparenright}{\kern0pt}\ {\isasymunion}\isanewline
\ \ \ \ \ \ {\isasymPsi}\ {\isacharparenleft}{\kern0pt}star\ {\isacharparenleft}{\kern0pt}eval\ {\isacharquery}{\kern0pt}r\ v{\isacharparenright}{\kern0pt}\ {\isacharat}{\kern0pt}{\isacharat}{\kern0pt}\ eval\ {\isacharquery}{\kern0pt}r\ v\ {\isacharat}{\kern0pt}{\isacharat}{\kern0pt}\ star\ {\isacharparenleft}{\kern0pt}eval\ {\isacharquery}{\kern0pt}r\ v{\isacharparenright}{\kern0pt}\ {\isacharat}{\kern0pt}{\isacharat}{\kern0pt}\ eval\ p\ v{\isacharparenright}{\kern0pt}{\isachardoublequoteclose}\isanewline
\ \ \ \ \isakeywordONE{by}\isamarkupfalse%
\ {\isacharparenleft}{\kern0pt}simp\ add{\isacharcolon}{\kern0pt}\ conc{\isacharunderscore}{\kern0pt}assoc{\isacharparenright}{\kern0pt}\isanewline
\ \ \isakeywordONE{also}\isamarkupfalse%
\ \isakeywordONE{have}\isamarkupfalse%
\ {\isachardoublequoteopen}{\isasymdots}\ {\isacharequal}{\kern0pt}\ {\isasymPsi}\ {\isacharparenleft}{\kern0pt}eval\ p\ v{\isacharparenright}{\kern0pt}\ {\isasymunion}\isanewline
\ \ \ \ \ \ {\isasymPsi}\ {\isacharparenleft}{\kern0pt}eval\ {\isacharquery}{\kern0pt}r\ v\ {\isacharat}{\kern0pt}{\isacharat}{\kern0pt}\ star\ {\isacharparenleft}{\kern0pt}eval\ {\isacharquery}{\kern0pt}r\ v{\isacharparenright}{\kern0pt}\ {\isacharat}{\kern0pt}{\isacharat}{\kern0pt}\ eval\ p\ v{\isacharparenright}{\kern0pt}{\isachardoublequoteclose}\isanewline
\ \ \ \ \isakeywordONE{using}\isamarkupfalse%
\ parikh{\isacharunderscore}{\kern0pt}img{\isacharunderscore}{\kern0pt}commut\ conc{\isacharunderscore}{\kern0pt}star{\isacharunderscore}{\kern0pt}star\ \isakeywordONE{by}\isamarkupfalse%
\ {\isacharparenleft}{\kern0pt}smt\ {\isacharparenleft}{\kern0pt}verit{\isacharcomma}{\kern0pt}\ best{\isacharparenright}{\kern0pt}\ conc{\isacharunderscore}{\kern0pt}assoc\ conc{\isacharunderscore}{\kern0pt}star{\isacharunderscore}{\kern0pt}comm{\isacharparenright}{\kern0pt}\isanewline
\ \ \isakeywordONE{also}\isamarkupfalse%
\ \isakeywordONE{have}\isamarkupfalse%
\ {\isachardoublequoteopen}{\isasymdots}\ {\isacharequal}{\kern0pt}\ {\isasymPsi}\ {\isacharparenleft}{\kern0pt}star\ {\isacharparenleft}{\kern0pt}eval\ {\isacharquery}{\kern0pt}r\ v{\isacharparenright}{\kern0pt}\ {\isacharat}{\kern0pt}{\isacharat}{\kern0pt}\ eval\ p\ v{\isacharparenright}{\kern0pt}{\isachardoublequoteclose}\isanewline
\ \ \ \ \isakeywordONE{using}\isamarkupfalse%
\ star{\isacharunderscore}{\kern0pt}unfold{\isacharunderscore}{\kern0pt}left\isanewline
\ \ \ \ \isakeywordONE{by}\isamarkupfalse%
\ {\isacharparenleft}{\kern0pt}smt\ {\isacharparenleft}{\kern0pt}verit{\isacharparenright}{\kern0pt}\ conc{\isacharunderscore}{\kern0pt}Un{\isacharunderscore}{\kern0pt}distrib{\isacharparenleft}{\kern0pt}{\isadigit{2}}{\isacharparenright}{\kern0pt}\ conc{\isacharunderscore}{\kern0pt}assoc\ conc{\isacharunderscore}{\kern0pt}epsilon{\isacharparenleft}{\kern0pt}{\isadigit{1}}{\isacharparenright}{\kern0pt}\ parikh{\isacharunderscore}{\kern0pt}img{\isacharunderscore}{\kern0pt}Un\ sup{\isacharunderscore}{\kern0pt}commute{\isacharparenright}{\kern0pt}\isanewline
\ \ \isakeywordONE{finally}\isamarkupfalse%
\ \isakeywordONE{have}\isamarkupfalse%
\ {\isacharasterisk}{\kern0pt}{\isacharcolon}{\kern0pt}\ {\isachardoublequoteopen}{\isasymPsi}\ {\isacharparenleft}{\kern0pt}eval\ {\isacharquery}{\kern0pt}eq{\isacharunderscore}{\kern0pt}subst\ v{\isacharparenright}{\kern0pt}\ {\isasymsubseteq}\ {\isasymPsi}\ {\isacharparenleft}{\kern0pt}v\ x{\isacharparenright}{\kern0pt}{\isachardoublequoteclose}\ \isakeywordONE{using}\isamarkupfalse%
\ x{\isacharunderscore}{\kern0pt}is{\isacharunderscore}{\kern0pt}sol\ \isakeywordONE{by}\isamarkupfalse%
\ simp\isanewline
\ \ \isakeywordONE{from}\isamarkupfalse%
\ x{\isacharunderscore}{\kern0pt}is{\isacharunderscore}{\kern0pt}sol\ \isakeywordONE{have}\isamarkupfalse%
\ {\isachardoublequoteopen}v\ {\isacharequal}{\kern0pt}\ v{\isacharparenleft}{\kern0pt}x\ {\isacharcolon}{\kern0pt}{\isacharequal}{\kern0pt}\ eval\ sol\ v{\isacharparenright}{\kern0pt}{\isachardoublequoteclose}\ \isakeywordONE{using}\isamarkupfalse%
\ fun{\isacharunderscore}{\kern0pt}upd{\isacharunderscore}{\kern0pt}triv\ \isakeywordONE{by}\isamarkupfalse%
\ metis\isanewline
\ \ \isakeywordONE{then}\isamarkupfalse%
\ \isakeywordONE{have}\isamarkupfalse%
\ {\isachardoublequoteopen}eval\ eq\ v\ {\isacharequal}{\kern0pt}\ eval\ {\isacharparenleft}{\kern0pt}subst\ {\isacharparenleft}{\kern0pt}Var{\isacharparenleft}{\kern0pt}x\ {\isacharcolon}{\kern0pt}{\isacharequal}{\kern0pt}\ sol{\isacharparenright}{\kern0pt}{\isacharparenright}{\kern0pt}\ eq{\isacharparenright}{\kern0pt}\ v{\isachardoublequoteclose}\isanewline
\ \ \ \ \isakeywordONE{using}\isamarkupfalse%
\ substitution{\isacharunderscore}{\kern0pt}lemma{\isacharunderscore}{\kern0pt}upd{\isacharbrackleft}{\kern0pt}\isakeywordTWO{where}\ f{\isacharequal}{\kern0pt}eq{\isacharbrackright}{\kern0pt}\ \isakeywordONE{by}\isamarkupfalse%
\ presburger\isanewline
\ \ \isakeywordONE{with}\isamarkupfalse%
\ {\isacharasterisk}{\kern0pt}\ \isakeywordTHREE{show}\isamarkupfalse%
\ {\isachardoublequoteopen}solves{\isacharunderscore}{\kern0pt}ineq{\isacharunderscore}{\kern0pt}comm\ x\ eq\ v{\isachardoublequoteclose}\ \isakeywordONE{unfolding}\isamarkupfalse%
\ solves{\isacharunderscore}{\kern0pt}ineq{\isacharunderscore}{\kern0pt}comm{\isacharunderscore}{\kern0pt}def\ \isakeywordONE{by}\isamarkupfalse%
\ argo\isanewline
\isakeywordONE{qed}\isamarkupfalse%
%
\endisatagproof
{\isafoldproof}%
%
\isadelimproof
\isanewline
%
\endisadelimproof
\isanewline
\isakeywordONE{lemma}\isamarkupfalse%
\ sol{\isacharunderscore}{\kern0pt}is{\isacharunderscore}{\kern0pt}minimal{\isacharcolon}{\kern0pt}\isanewline
\ \ \isakeywordTWO{assumes}\ is{\isacharunderscore}{\kern0pt}sol{\isacharcolon}{\kern0pt}\ \ \ \ {\isachardoublequoteopen}solves{\isacharunderscore}{\kern0pt}ineq{\isacharunderscore}{\kern0pt}comm\ x\ eq\ v{\isachardoublequoteclose}\isanewline
\ \ \ \ \ \ \isakeywordTWO{and}\ sol{\isacharprime}{\kern0pt}{\isacharunderscore}{\kern0pt}s{\isacharcolon}{\kern0pt}\ \ \ \ {\isachardoublequoteopen}v\ x\ {\isacharequal}{\kern0pt}\ eval\ sol{\isacharprime}{\kern0pt}\ v{\isachardoublequoteclose}\isanewline
\ \ \ \ \isakeywordTWO{shows}\ \ \ \ \ \ \ \ \ \ \ \ {\isachardoublequoteopen}{\isasymPsi}\ {\isacharparenleft}{\kern0pt}eval\ sol\ v{\isacharparenright}{\kern0pt}\ {\isasymsubseteq}\ {\isasymPsi}\ {\isacharparenleft}{\kern0pt}v\ x{\isacharparenright}{\kern0pt}{\isachardoublequoteclose}\isanewline
%
\isadelimproof
%
\endisadelimproof
%
\isatagproof
\isakeywordONE{proof}\isamarkupfalse%
\ {\isacharminus}{\kern0pt}\isanewline
\ \ \isakeywordONE{from}\isamarkupfalse%
\ is{\isacharunderscore}{\kern0pt}sol\ sol{\isacharprime}{\kern0pt}{\isacharunderscore}{\kern0pt}s\ \isakeywordONE{have}\isamarkupfalse%
\ is{\isacharunderscore}{\kern0pt}sol{\isacharprime}{\kern0pt}{\isacharcolon}{\kern0pt}\ {\isachardoublequoteopen}{\isasymPsi}\ {\isacharparenleft}{\kern0pt}eval\ q\ v\ {\isacharat}{\kern0pt}{\isacharat}{\kern0pt}\ v\ x\ {\isasymunion}\ eval\ p\ v{\isacharparenright}{\kern0pt}\ {\isasymsubseteq}\ {\isasymPsi}\ {\isacharparenleft}{\kern0pt}v\ x{\isacharparenright}{\kern0pt}{\isachardoublequoteclose}\isanewline
\ \ \ \ \isakeywordONE{unfolding}\isamarkupfalse%
\ solves{\isacharunderscore}{\kern0pt}ineq{\isacharunderscore}{\kern0pt}comm{\isacharunderscore}{\kern0pt}def\ \isakeywordONE{by}\isamarkupfalse%
\ simp\isanewline
\ \ \isakeywordONE{then}\isamarkupfalse%
\ \isakeywordONE{have}\isamarkupfalse%
\ {\isadigit{1}}{\isacharcolon}{\kern0pt}\ {\isachardoublequoteopen}{\isasymPsi}\ {\isacharparenleft}{\kern0pt}eval\ {\isacharparenleft}{\kern0pt}Concat\ {\isacharparenleft}{\kern0pt}Star\ q{\isacharparenright}{\kern0pt}\ p{\isacharparenright}{\kern0pt}\ v{\isacharparenright}{\kern0pt}\ {\isasymsubseteq}\ {\isasymPsi}\ {\isacharparenleft}{\kern0pt}v\ x{\isacharparenright}{\kern0pt}{\isachardoublequoteclose}\isanewline
\ \ \ \ \isakeywordONE{using}\isamarkupfalse%
\ parikh{\isacharunderscore}{\kern0pt}img{\isacharunderscore}{\kern0pt}arden\ \isakeywordONE{by}\isamarkupfalse%
\ auto\isanewline
\ \ \isakeywordONE{from}\isamarkupfalse%
\ is{\isacharunderscore}{\kern0pt}sol{\isacharprime}{\kern0pt}\ \isakeywordONE{have}\isamarkupfalse%
\ {\isachardoublequoteopen}{\isasymPsi}\ {\isacharparenleft}{\kern0pt}eval\ p\ v{\isacharparenright}{\kern0pt}\ {\isasymsubseteq}\ {\isasymPsi}\ {\isacharparenleft}{\kern0pt}eval\ {\isacharparenleft}{\kern0pt}Var\ x{\isacharparenright}{\kern0pt}\ v{\isacharparenright}{\kern0pt}{\isachardoublequoteclose}\ \isakeywordONE{by}\isamarkupfalse%
\ auto\isanewline
\ \ \isakeywordONE{then}\isamarkupfalse%
\ \isakeywordONE{have}\isamarkupfalse%
\ {\isachardoublequoteopen}{\isasymPsi}\ {\isacharparenleft}{\kern0pt}eval\ {\isacharparenleft}{\kern0pt}subst\ {\isacharparenleft}{\kern0pt}Var{\isacharparenleft}{\kern0pt}x\ {\isacharcolon}{\kern0pt}{\isacharequal}{\kern0pt}\ p{\isacharparenright}{\kern0pt}{\isacharparenright}{\kern0pt}\ q{\isacharparenright}{\kern0pt}\ v{\isacharparenright}{\kern0pt}\ {\isasymsubseteq}\ {\isasymPsi}\ {\isacharparenleft}{\kern0pt}eval\ q\ v{\isacharparenright}{\kern0pt}{\isachardoublequoteclose}\isanewline
\ \ \ \ \isakeywordONE{using}\isamarkupfalse%
\ parikh{\isacharunderscore}{\kern0pt}img{\isacharunderscore}{\kern0pt}subst{\isacharunderscore}{\kern0pt}mono{\isacharunderscore}{\kern0pt}upd\ \isakeywordONE{by}\isamarkupfalse%
\ {\isacharparenleft}{\kern0pt}metis\ fun{\isacharunderscore}{\kern0pt}upd{\isacharunderscore}{\kern0pt}triv\ subst{\isacharunderscore}{\kern0pt}id{\isacharparenright}{\kern0pt}\isanewline
\ \ \isakeywordONE{then}\isamarkupfalse%
\ \isakeywordONE{have}\isamarkupfalse%
\ {\isachardoublequoteopen}{\isasymPsi}\ {\isacharparenleft}{\kern0pt}eval\ {\isacharparenleft}{\kern0pt}Star\ {\isacharparenleft}{\kern0pt}subst\ {\isacharparenleft}{\kern0pt}Var{\isacharparenleft}{\kern0pt}x\ {\isacharcolon}{\kern0pt}{\isacharequal}{\kern0pt}\ p{\isacharparenright}{\kern0pt}{\isacharparenright}{\kern0pt}\ q{\isacharparenright}{\kern0pt}{\isacharparenright}{\kern0pt}\ v{\isacharparenright}{\kern0pt}\ {\isasymsubseteq}\ {\isasymPsi}\ {\isacharparenleft}{\kern0pt}eval\ {\isacharparenleft}{\kern0pt}Star\ q{\isacharparenright}{\kern0pt}\ v{\isacharparenright}{\kern0pt}{\isachardoublequoteclose}\isanewline
\ \ \ \ \isakeywordONE{using}\isamarkupfalse%
\ parikh{\isacharunderscore}{\kern0pt}star{\isacharunderscore}{\kern0pt}mono\ \isakeywordONE{by}\isamarkupfalse%
\ auto\isanewline
\ \ \isakeywordONE{then}\isamarkupfalse%
\ \isakeywordONE{have}\isamarkupfalse%
\ {\isachardoublequoteopen}{\isasymPsi}\ {\isacharparenleft}{\kern0pt}eval\ sol\ v{\isacharparenright}{\kern0pt}\ {\isasymsubseteq}\ {\isasymPsi}\ {\isacharparenleft}{\kern0pt}eval\ {\isacharparenleft}{\kern0pt}Concat\ {\isacharparenleft}{\kern0pt}Star\ q{\isacharparenright}{\kern0pt}\ p{\isacharparenright}{\kern0pt}\ v{\isacharparenright}{\kern0pt}{\isachardoublequoteclose}\isanewline
\ \ \ \ \isakeywordONE{using}\isamarkupfalse%
\ parikh{\isacharunderscore}{\kern0pt}conc{\isacharunderscore}{\kern0pt}right{\isacharunderscore}{\kern0pt}subset\ \isakeywordONE{by}\isamarkupfalse%
\ {\isacharparenleft}{\kern0pt}metis\ eval{\isachardot}{\kern0pt}simps{\isacharparenleft}{\kern0pt}{\isadigit{4}}{\isacharparenright}{\kern0pt}{\isacharparenright}{\kern0pt}\isanewline
\ \ \isakeywordONE{with}\isamarkupfalse%
\ {\isadigit{1}}\ \isakeywordTHREE{show}\isamarkupfalse%
\ {\isacharquery}{\kern0pt}thesis\ \isakeywordONE{by}\isamarkupfalse%
\ fast\isanewline
\isakeywordONE{qed}\isamarkupfalse%
%
\endisatagproof
{\isafoldproof}%
%
\isadelimproof
%
\endisadelimproof
%
\begin{isamarkuptext}%
In summary, \isa{sol} is a minimal partial solution and it is \isa{\isaconst{reg{\isacharunderscore}{\kern0pt}eval}}:%
\end{isamarkuptext}\isamarkuptrue%
\isakeywordONE{lemma}\isamarkupfalse%
\ sol{\isacharunderscore}{\kern0pt}is{\isacharunderscore}{\kern0pt}minimal{\isacharunderscore}{\kern0pt}reg{\isacharunderscore}{\kern0pt}sol{\isacharcolon}{\kern0pt}\isanewline
\ \ {\isachardoublequoteopen}reg{\isacharunderscore}{\kern0pt}eval\ sol\ {\isasymand}\ partial{\isacharunderscore}{\kern0pt}min{\isacharunderscore}{\kern0pt}sol{\isacharunderscore}{\kern0pt}one{\isacharunderscore}{\kern0pt}ineq\ x\ eq\ sol{\isachardoublequoteclose}\isanewline
%
\isadelimproof
\ \ %
\endisadelimproof
%
\isatagproof
\isakeywordONE{unfolding}\isamarkupfalse%
\ partial{\isacharunderscore}{\kern0pt}min{\isacharunderscore}{\kern0pt}sol{\isacharunderscore}{\kern0pt}one{\isacharunderscore}{\kern0pt}ineq{\isacharunderscore}{\kern0pt}def\isanewline
\ \ \isakeywordONE{using}\isamarkupfalse%
\ sol{\isacharunderscore}{\kern0pt}is{\isacharunderscore}{\kern0pt}reg\ sol{\isacharunderscore}{\kern0pt}vars\ sol{\isacharunderscore}{\kern0pt}is{\isacharunderscore}{\kern0pt}sol{\isacharunderscore}{\kern0pt}ineq\ sol{\isacharunderscore}{\kern0pt}is{\isacharunderscore}{\kern0pt}minimal\isanewline
\ \ \isakeywordONE{by}\isamarkupfalse%
\ blast%
\endisatagproof
{\isafoldproof}%
%
\isadelimproof
\isanewline
%
\endisadelimproof
\isanewline
\isakeywordTWO{end}\isamarkupfalse%
%
\begin{isamarkuptext}%
As announced at the beginning of this section, we now extend the previous result to arbitrary
equations, i.e.\ we show that each \isa{\isaconst{reg{\isacharunderscore}{\kern0pt}eval}} equation has some minimal partial solution which is
\isa{\isaconst{reg{\isacharunderscore}{\kern0pt}eval}}:%
\end{isamarkuptext}\isamarkuptrue%
\isakeywordONE{lemma}\isamarkupfalse%
\ exists{\isacharunderscore}{\kern0pt}minimal{\isacharunderscore}{\kern0pt}reg{\isacharunderscore}{\kern0pt}sol{\isacharcolon}{\kern0pt}\isanewline
\ \ \isakeywordTWO{assumes}\ eq{\isacharunderscore}{\kern0pt}reg{\isacharcolon}{\kern0pt}\ {\isachardoublequoteopen}reg{\isacharunderscore}{\kern0pt}eval\ eq{\isachardoublequoteclose}\isanewline
\ \ \isakeywordTWO{shows}\ {\isachardoublequoteopen}{\isasymexists}sol{\isachardot}{\kern0pt}\ reg{\isacharunderscore}{\kern0pt}eval\ sol\ {\isasymand}\ partial{\isacharunderscore}{\kern0pt}min{\isacharunderscore}{\kern0pt}sol{\isacharunderscore}{\kern0pt}one{\isacharunderscore}{\kern0pt}ineq\ x\ eq\ sol{\isachardoublequoteclose}\isanewline
%
\isadelimproof
%
\endisadelimproof
%
\isatagproof
\isakeywordONE{proof}\isamarkupfalse%
\ {\isacharminus}{\kern0pt}\isanewline
\ \ \isakeywordONE{from}\isamarkupfalse%
\ reg{\isacharunderscore}{\kern0pt}eval{\isacharunderscore}{\kern0pt}bipart{\isacharunderscore}{\kern0pt}rlexp{\isacharbrackleft}{\kern0pt}OF\ eq{\isacharunderscore}{\kern0pt}reg{\isacharbrackright}{\kern0pt}\ \isakeywordTHREE{obtain}\isamarkupfalse%
\ eq{\isacharprime}{\kern0pt}\isanewline
\ \ \ \ \isakeywordTWO{where}\ eq{\isacharprime}{\kern0pt}{\isacharunderscore}{\kern0pt}intro{\isacharcolon}{\kern0pt}\ {\isachardoublequoteopen}bipart{\isacharunderscore}{\kern0pt}rlexp\ x\ eq{\isacharprime}{\kern0pt}\ {\isasymand}\ vars\ eq{\isacharprime}{\kern0pt}\ {\isacharequal}{\kern0pt}\ vars\ eq\ {\isasymunion}\ {\isacharbraceleft}{\kern0pt}x{\isacharbraceright}{\kern0pt}\ {\isasymand}\isanewline
\ \ \ \ \ \ \ \ \ \ \ \ \ \ \ \ \ \ \ \ {\isacharparenleft}{\kern0pt}{\isasymforall}v{\isachardot}{\kern0pt}\ {\isasymPsi}\ {\isacharparenleft}{\kern0pt}eval\ eq\ v{\isacharparenright}{\kern0pt}\ {\isacharequal}{\kern0pt}\ {\isasymPsi}\ {\isacharparenleft}{\kern0pt}eval\ eq{\isacharprime}{\kern0pt}\ v{\isacharparenright}{\kern0pt}{\isacharparenright}{\kern0pt}{\isachardoublequoteclose}\ \isakeywordONE{by}\isamarkupfalse%
\ blast\isanewline
\ \ \isakeywordONE{then}\isamarkupfalse%
\ \isakeywordTHREE{obtain}\isamarkupfalse%
\ p\ q\isanewline
\ \ \ \ \isakeywordTWO{where}\ p{\isacharunderscore}{\kern0pt}q{\isacharunderscore}{\kern0pt}intro{\isacharcolon}{\kern0pt}\ {\isachardoublequoteopen}reg{\isacharunderscore}{\kern0pt}eval\ p\ {\isasymand}\ reg{\isacharunderscore}{\kern0pt}eval\ q\ {\isasymand}\ eq{\isacharprime}{\kern0pt}\ {\isacharequal}{\kern0pt}\ Union\ p\ {\isacharparenleft}{\kern0pt}Concat\ q\ {\isacharparenleft}{\kern0pt}Var\ x{\isacharparenright}{\kern0pt}{\isacharparenright}{\kern0pt}\ {\isasymand}\ x\ {\isasymnotin}\ vars\ p{\isachardoublequoteclose}\isanewline
\ \ \ \ \isakeywordONE{unfolding}\isamarkupfalse%
\ bipart{\isacharunderscore}{\kern0pt}rlexp{\isacharunderscore}{\kern0pt}def\ \isakeywordONE{by}\isamarkupfalse%
\ blast\isanewline
\ \ \isakeywordONE{let}\isamarkupfalse%
\ {\isacharquery}{\kern0pt}sol\ {\isacharequal}{\kern0pt}\ {\isachardoublequoteopen}Concat\ {\isacharparenleft}{\kern0pt}Star\ {\isacharparenleft}{\kern0pt}subst\ {\isacharparenleft}{\kern0pt}Var{\isacharparenleft}{\kern0pt}x\ {\isacharcolon}{\kern0pt}{\isacharequal}{\kern0pt}\ p{\isacharparenright}{\kern0pt}{\isacharparenright}{\kern0pt}\ q{\isacharparenright}{\kern0pt}{\isacharparenright}{\kern0pt}\ p{\isachardoublequoteclose}\isanewline
\ \ \isakeywordONE{from}\isamarkupfalse%
\ p{\isacharunderscore}{\kern0pt}q{\isacharunderscore}{\kern0pt}intro\ \isakeywordONE{have}\isamarkupfalse%
\ sol{\isacharunderscore}{\kern0pt}prop{\isacharcolon}{\kern0pt}\ {\isachardoublequoteopen}reg{\isacharunderscore}{\kern0pt}eval\ {\isacharquery}{\kern0pt}sol\ {\isasymand}\ partial{\isacharunderscore}{\kern0pt}min{\isacharunderscore}{\kern0pt}sol{\isacharunderscore}{\kern0pt}one{\isacharunderscore}{\kern0pt}ineq\ x\ eq{\isacharprime}{\kern0pt}\ {\isacharquery}{\kern0pt}sol{\isachardoublequoteclose}\isanewline
\ \ \ \ \isakeywordONE{using}\isamarkupfalse%
\ single{\isacharunderscore}{\kern0pt}bipartite{\isacharunderscore}{\kern0pt}eq{\isachardot}{\kern0pt}sol{\isacharunderscore}{\kern0pt}is{\isacharunderscore}{\kern0pt}minimal{\isacharunderscore}{\kern0pt}reg{\isacharunderscore}{\kern0pt}sol\ \isakeywordONE{unfolding}\isamarkupfalse%
\ single{\isacharunderscore}{\kern0pt}bipartite{\isacharunderscore}{\kern0pt}eq{\isacharunderscore}{\kern0pt}def\ \isakeywordONE{by}\isamarkupfalse%
\ blast\isanewline
\ \ \isakeywordONE{with}\isamarkupfalse%
\ eq{\isacharprime}{\kern0pt}{\isacharunderscore}{\kern0pt}intro\ \isakeywordONE{have}\isamarkupfalse%
\ {\isachardoublequoteopen}partial{\isacharunderscore}{\kern0pt}min{\isacharunderscore}{\kern0pt}sol{\isacharunderscore}{\kern0pt}one{\isacharunderscore}{\kern0pt}ineq\ x\ eq\ {\isacharquery}{\kern0pt}sol{\isachardoublequoteclose}\isanewline
\ \ \ \ \isakeywordONE{using}\isamarkupfalse%
\ same{\isacharunderscore}{\kern0pt}min{\isacharunderscore}{\kern0pt}sol{\isacharunderscore}{\kern0pt}if{\isacharunderscore}{\kern0pt}same{\isacharunderscore}{\kern0pt}parikh{\isacharunderscore}{\kern0pt}img\ \isakeywordONE{by}\isamarkupfalse%
\ blast\isanewline
\ \ \isakeywordONE{with}\isamarkupfalse%
\ sol{\isacharunderscore}{\kern0pt}prop\ \isakeywordTHREE{show}\isamarkupfalse%
\ {\isacharquery}{\kern0pt}thesis\ \isakeywordONE{by}\isamarkupfalse%
\ blast\isanewline
\isakeywordONE{qed}\isamarkupfalse%
%
\endisatagproof
{\isafoldproof}%
%
\isadelimproof
%
\endisadelimproof
%
\isadelimdocument
%
\endisadelimdocument
%
\isatagdocument
%
\isamarkupsubsection{Minimal solution of the whole system of equations%
}
\isamarkuptrue%
%
\endisatagdocument
{\isafolddocument}%
%
\isadelimdocument
%
\endisadelimdocument
%
\begin{isamarkuptext}%
In this section we will extend the last section's result to whole systems of \isa{\isaconst{reg{\isacharunderscore}{\kern0pt}eval}}
equations. For this purpose, we will show by induction on \isa{r} that the first \isa{r} equations have
some minimal partial solution which is \isa{\isaconst{reg{\isacharunderscore}{\kern0pt}eval}}.

We start with the centerpiece of the induction step: If a \isa{\isaconst{reg{\isacharunderscore}{\kern0pt}eval}} and minimal partial solution
\isa{sols} exists for the first \isa{r} equations and furthermore a \isa{\isaconst{reg{\isacharunderscore}{\kern0pt}eval}} and minimal partial solution
\isa{sol{\isacharunderscore}{\kern0pt}r} exists for the \isa{r}-th equation, then there exists a \isa{\isaconst{reg{\isacharunderscore}{\kern0pt}eval}} and minimal partial solution
for the first \isa{Suc\ r} equations as well.%
\end{isamarkuptext}\isamarkuptrue%
\isakeywordONE{locale}\isamarkupfalse%
\ min{\isacharunderscore}{\kern0pt}sol{\isacharunderscore}{\kern0pt}induction{\isacharunderscore}{\kern0pt}step\ {\isacharequal}{\kern0pt}\isanewline
\ \ \isakeywordTWO{fixes}\ r\ {\isacharcolon}{\kern0pt}{\isacharcolon}{\kern0pt}\ nat\isanewline
\ \ \ \ \isakeywordTWO{and}\ sys\ {\isacharcolon}{\kern0pt}{\isacharcolon}{\kern0pt}\ {\isachardoublequoteopen}{\isacharprime}{\kern0pt}a\ eq{\isacharunderscore}{\kern0pt}sys{\isachardoublequoteclose}\isanewline
\ \ \ \ \isakeywordTWO{and}\ sols\ {\isacharcolon}{\kern0pt}{\isacharcolon}{\kern0pt}\ {\isachardoublequoteopen}nat\ {\isasymRightarrow}\ {\isacharprime}{\kern0pt}a\ rlexp{\isachardoublequoteclose}\isanewline
\ \ \ \ \isakeywordTWO{and}\ sol{\isacharunderscore}{\kern0pt}r\ {\isacharcolon}{\kern0pt}{\isacharcolon}{\kern0pt}\ {\isachardoublequoteopen}{\isacharprime}{\kern0pt}a\ rlexp{\isachardoublequoteclose}\isanewline
\ \ \isakeywordTWO{assumes}\ eqs{\isacharunderscore}{\kern0pt}reg{\isacharcolon}{\kern0pt}\ \ \ \ \ \ {\isachardoublequoteopen}{\isasymforall}eq\ {\isasymin}\ set\ sys{\isachardot}{\kern0pt}\ reg{\isacharunderscore}{\kern0pt}eval\ eq{\isachardoublequoteclose}\isanewline
\ \ \ \ \ \ \isakeywordTWO{and}\ sys{\isacharunderscore}{\kern0pt}valid{\isacharcolon}{\kern0pt}\ \ \ \ {\isachardoublequoteopen}{\isasymforall}eq\ {\isasymin}\ set\ sys{\isachardot}{\kern0pt}\ {\isasymforall}x\ {\isasymin}\ vars\ eq{\isachardot}{\kern0pt}\ x\ {\isacharless}{\kern0pt}\ length\ sys{\isachardoublequoteclose}\isanewline
\ \ \ \ \ \ \isakeywordTWO{and}\ r{\isacharunderscore}{\kern0pt}valid{\isacharcolon}{\kern0pt}\ \ \ \ \ \ {\isachardoublequoteopen}r\ {\isacharless}{\kern0pt}\ length\ sys{\isachardoublequoteclose}\isanewline
\ \ \ \ \ \ \isakeywordTWO{and}\ sols{\isacharunderscore}{\kern0pt}is{\isacharunderscore}{\kern0pt}sol{\isacharcolon}{\kern0pt}\ \ {\isachardoublequoteopen}partial{\isacharunderscore}{\kern0pt}min{\isacharunderscore}{\kern0pt}sol{\isacharunderscore}{\kern0pt}ineq{\isacharunderscore}{\kern0pt}sys\ r\ sys\ sols{\isachardoublequoteclose}\isanewline
\ \ \ \ \ \ \isakeywordTWO{and}\ sols{\isacharunderscore}{\kern0pt}reg{\isacharcolon}{\kern0pt}\ \ \ \ \ {\isachardoublequoteopen}{\isasymforall}i{\isachardot}{\kern0pt}\ reg{\isacharunderscore}{\kern0pt}eval\ {\isacharparenleft}{\kern0pt}sols\ i{\isacharparenright}{\kern0pt}{\isachardoublequoteclose}\isanewline
\ \ \ \ \ \ \isakeywordTWO{and}\ sol{\isacharunderscore}{\kern0pt}r{\isacharunderscore}{\kern0pt}is{\isacharunderscore}{\kern0pt}sol{\isacharcolon}{\kern0pt}\ {\isachardoublequoteopen}partial{\isacharunderscore}{\kern0pt}min{\isacharunderscore}{\kern0pt}sol{\isacharunderscore}{\kern0pt}one{\isacharunderscore}{\kern0pt}ineq\ r\ {\isacharparenleft}{\kern0pt}subst{\isacharunderscore}{\kern0pt}sys\ sols\ sys\ {\isacharbang}{\kern0pt}\ r{\isacharparenright}{\kern0pt}\ sol{\isacharunderscore}{\kern0pt}r{\isachardoublequoteclose}\isanewline
\ \ \ \ \ \ \isakeywordTWO{and}\ sol{\isacharunderscore}{\kern0pt}r{\isacharunderscore}{\kern0pt}reg{\isacharcolon}{\kern0pt}\ \ \ \ {\isachardoublequoteopen}reg{\isacharunderscore}{\kern0pt}eval\ sol{\isacharunderscore}{\kern0pt}r{\isachardoublequoteclose}\isanewline
\isakeywordTWO{begin}%
\begin{isamarkuptext}%
Throughout the proof, a modified system of equations will be occasionally used to simplify
the proof; this modified system is obtained by substituting the partial solutions of
the first \isa{r} equations into the original system. Additionally
we retrieve a partial solution for the first \isa{Suc\ r} equations - named \isa{sols{\isacharprime}{\kern0pt}} - by substituting the partial
solution of the \isa{r}-th equation into the partial solutions of each of the first \isa{r} equations:%
\end{isamarkuptext}\isamarkuptrue%
\isakeywordONE{abbreviation}\isamarkupfalse%
\ {\isachardoublequoteopen}sys{\isacharprime}{\kern0pt}\ {\isasymequiv}\ subst{\isacharunderscore}{\kern0pt}sys\ sols\ sys{\isachardoublequoteclose}\isanewline
\isakeywordONE{abbreviation}\isamarkupfalse%
\ {\isachardoublequoteopen}sols{\isacharprime}{\kern0pt}\ {\isasymequiv}\ {\isasymlambda}i{\isachardot}{\kern0pt}\ subst\ {\isacharparenleft}{\kern0pt}Var{\isacharparenleft}{\kern0pt}r\ {\isacharcolon}{\kern0pt}{\isacharequal}{\kern0pt}\ sol{\isacharunderscore}{\kern0pt}r{\isacharparenright}{\kern0pt}{\isacharparenright}{\kern0pt}\ {\isacharparenleft}{\kern0pt}sols\ i{\isacharparenright}{\kern0pt}{\isachardoublequoteclose}\isanewline
\isanewline
\isakeywordONE{lemma}\isamarkupfalse%
\ sols{\isacharprime}{\kern0pt}{\isacharunderscore}{\kern0pt}r{\isacharcolon}{\kern0pt}\ {\isachardoublequoteopen}sols{\isacharprime}{\kern0pt}\ r\ {\isacharequal}{\kern0pt}\ sol{\isacharunderscore}{\kern0pt}r{\isachardoublequoteclose}\isanewline
%
\isadelimproof
\ \ %
\endisadelimproof
%
\isatagproof
\isakeywordONE{using}\isamarkupfalse%
\ sols{\isacharunderscore}{\kern0pt}is{\isacharunderscore}{\kern0pt}sol\ \isakeywordONE{unfolding}\isamarkupfalse%
\ partial{\isacharunderscore}{\kern0pt}min{\isacharunderscore}{\kern0pt}sol{\isacharunderscore}{\kern0pt}ineq{\isacharunderscore}{\kern0pt}sys{\isacharunderscore}{\kern0pt}def\ \isakeywordONE{by}\isamarkupfalse%
\ simp%
\endisatagproof
{\isafoldproof}%
%
\isadelimproof
%
\endisadelimproof
%
\begin{isamarkuptext}%
The next lemmas show that \isa{\isaconst{sols{\isacharprime}{\kern0pt}}} is still \isa{\isaconst{reg{\isacharunderscore}{\kern0pt}eval}} and that it complies with
each of the four conditions defined by the predicate \isa{\isaconst{partial{\isacharunderscore}{\kern0pt}min{\isacharunderscore}{\kern0pt}sol{\isacharunderscore}{\kern0pt}ineq{\isacharunderscore}{\kern0pt}sys}}:%
\end{isamarkuptext}\isamarkuptrue%
\isakeywordONE{lemma}\isamarkupfalse%
\ sols{\isacharprime}{\kern0pt}{\isacharunderscore}{\kern0pt}reg{\isacharcolon}{\kern0pt}\ {\isachardoublequoteopen}{\isasymforall}i{\isachardot}{\kern0pt}\ reg{\isacharunderscore}{\kern0pt}eval\ {\isacharparenleft}{\kern0pt}sols{\isacharprime}{\kern0pt}\ i{\isacharparenright}{\kern0pt}{\isachardoublequoteclose}\isanewline
%
\isadelimproof
\ \ %
\endisadelimproof
%
\isatagproof
\isakeywordONE{using}\isamarkupfalse%
\ sols{\isacharunderscore}{\kern0pt}reg\ sol{\isacharunderscore}{\kern0pt}r{\isacharunderscore}{\kern0pt}reg\ \isakeywordONE{using}\isamarkupfalse%
\ subst{\isacharunderscore}{\kern0pt}reg{\isacharunderscore}{\kern0pt}eval{\isacharunderscore}{\kern0pt}update\ \isakeywordONE{by}\isamarkupfalse%
\ blast%
\endisatagproof
{\isafoldproof}%
%
\isadelimproof
\isanewline
%
\endisadelimproof
\isanewline
\isakeywordONE{lemma}\isamarkupfalse%
\ sols{\isacharprime}{\kern0pt}{\isacharunderscore}{\kern0pt}is{\isacharunderscore}{\kern0pt}sol{\isacharcolon}{\kern0pt}\ {\isachardoublequoteopen}solution{\isacharunderscore}{\kern0pt}ineq{\isacharunderscore}{\kern0pt}sys\ {\isacharparenleft}{\kern0pt}take\ {\isacharparenleft}{\kern0pt}Suc\ r{\isacharparenright}{\kern0pt}\ sys{\isacharparenright}{\kern0pt}\ sols{\isacharprime}{\kern0pt}{\isachardoublequoteclose}\isanewline
%
\isadelimproof
%
\endisadelimproof
%
\isatagproof
\isakeywordONE{unfolding}\isamarkupfalse%
\ solution{\isacharunderscore}{\kern0pt}ineq{\isacharunderscore}{\kern0pt}sys{\isacharunderscore}{\kern0pt}def\ \isakeywordONE{proof}\isamarkupfalse%
\ {\isacharparenleft}{\kern0pt}rule\ allI{\isacharcomma}{\kern0pt}\ rule\ impI{\isacharparenright}{\kern0pt}\isanewline
\ \ \isakeywordTHREE{fix}\isamarkupfalse%
\ v\isanewline
\ \ \isakeywordTHREE{assume}\isamarkupfalse%
\ s{\isacharunderscore}{\kern0pt}sols{\isacharprime}{\kern0pt}{\isacharcolon}{\kern0pt}\ {\isachardoublequoteopen}{\isasymforall}x{\isachardot}{\kern0pt}\ v\ x\ {\isacharequal}{\kern0pt}\ eval\ {\isacharparenleft}{\kern0pt}sols{\isacharprime}{\kern0pt}\ x{\isacharparenright}{\kern0pt}\ v{\isachardoublequoteclose}\isanewline
\ \ \isakeywordONE{from}\isamarkupfalse%
\ sols{\isacharprime}{\kern0pt}{\isacharunderscore}{\kern0pt}r\ s{\isacharunderscore}{\kern0pt}sols{\isacharprime}{\kern0pt}\ \isakeywordONE{have}\isamarkupfalse%
\ s{\isacharunderscore}{\kern0pt}r{\isacharunderscore}{\kern0pt}sol{\isacharunderscore}{\kern0pt}r{\isacharcolon}{\kern0pt}\ {\isachardoublequoteopen}v\ r\ {\isacharequal}{\kern0pt}\ eval\ sol{\isacharunderscore}{\kern0pt}r\ v{\isachardoublequoteclose}\ \isakeywordONE{by}\isamarkupfalse%
\ simp\isanewline
\ \ \isakeywordONE{with}\isamarkupfalse%
\ s{\isacharunderscore}{\kern0pt}sols{\isacharprime}{\kern0pt}\ \isakeywordONE{have}\isamarkupfalse%
\ s{\isacharunderscore}{\kern0pt}sols{\isacharcolon}{\kern0pt}\ {\isachardoublequoteopen}v\ x\ {\isacharequal}{\kern0pt}\ eval\ {\isacharparenleft}{\kern0pt}sols\ x{\isacharparenright}{\kern0pt}\ v{\isachardoublequoteclose}\ \isakeywordTWO{for}\ x\isanewline
\ \ \ \ \isakeywordONE{using}\isamarkupfalse%
\ substitution{\isacharunderscore}{\kern0pt}lemma{\isacharunderscore}{\kern0pt}upd{\isacharbrackleft}{\kern0pt}\isakeywordTWO{where}\ f{\isacharequal}{\kern0pt}{\isachardoublequoteopen}sols\ x{\isachardoublequoteclose}{\isacharbrackright}{\kern0pt}\ \isakeywordONE{by}\isamarkupfalse%
\ {\isacharparenleft}{\kern0pt}auto\ simp\ add{\isacharcolon}{\kern0pt}\ fun{\isacharunderscore}{\kern0pt}upd{\isacharunderscore}{\kern0pt}idem{\isacharparenright}{\kern0pt}\isanewline
\ \ \isakeywordONE{with}\isamarkupfalse%
\ sols{\isacharunderscore}{\kern0pt}is{\isacharunderscore}{\kern0pt}sol\ \isakeywordONE{have}\isamarkupfalse%
\ solves{\isacharunderscore}{\kern0pt}r{\isacharunderscore}{\kern0pt}sys{\isacharcolon}{\kern0pt}\ {\isachardoublequoteopen}solves{\isacharunderscore}{\kern0pt}ineq{\isacharunderscore}{\kern0pt}sys{\isacharunderscore}{\kern0pt}comm\ {\isacharparenleft}{\kern0pt}take\ r\ sys{\isacharparenright}{\kern0pt}\ v{\isachardoublequoteclose}\isanewline
\ \ \ \ \isakeywordONE{unfolding}\isamarkupfalse%
\ partial{\isacharunderscore}{\kern0pt}min{\isacharunderscore}{\kern0pt}sol{\isacharunderscore}{\kern0pt}ineq{\isacharunderscore}{\kern0pt}sys{\isacharunderscore}{\kern0pt}def\ solution{\isacharunderscore}{\kern0pt}ineq{\isacharunderscore}{\kern0pt}sys{\isacharunderscore}{\kern0pt}def\ \isakeywordONE{by}\isamarkupfalse%
\ meson\isanewline
\ \ \isakeywordONE{have}\isamarkupfalse%
\ {\isachardoublequoteopen}eval\ {\isacharparenleft}{\kern0pt}sys\ {\isacharbang}{\kern0pt}\ r{\isacharparenright}{\kern0pt}\ {\isacharparenleft}{\kern0pt}{\isasymlambda}y{\isachardot}{\kern0pt}\ eval\ {\isacharparenleft}{\kern0pt}sols\ y{\isacharparenright}{\kern0pt}\ v{\isacharparenright}{\kern0pt}\ {\isacharequal}{\kern0pt}\ eval\ {\isacharparenleft}{\kern0pt}sys{\isacharprime}{\kern0pt}\ {\isacharbang}{\kern0pt}\ r{\isacharparenright}{\kern0pt}\ v{\isachardoublequoteclose}\isanewline
\ \ \ \ \isakeywordONE{using}\isamarkupfalse%
\ substitution{\isacharunderscore}{\kern0pt}lemma{\isacharbrackleft}{\kern0pt}of\ {\isachardoublequoteopen}{\isasymlambda}y{\isachardot}{\kern0pt}\ eval\ {\isacharparenleft}{\kern0pt}sols\ y{\isacharparenright}{\kern0pt}\ v{\isachardoublequoteclose}{\isacharbrackright}{\kern0pt}\isanewline
\ \ \ \ \isakeywordONE{by}\isamarkupfalse%
\ {\isacharparenleft}{\kern0pt}simp\ add{\isacharcolon}{\kern0pt}\ r{\isacharunderscore}{\kern0pt}valid\ Suc{\isacharunderscore}{\kern0pt}le{\isacharunderscore}{\kern0pt}lessD\ subst{\isacharunderscore}{\kern0pt}sys{\isacharunderscore}{\kern0pt}subst{\isacharparenright}{\kern0pt}\isanewline
\ \ \isakeywordONE{with}\isamarkupfalse%
\ s{\isacharunderscore}{\kern0pt}sols\ \isakeywordONE{have}\isamarkupfalse%
\ {\isachardoublequoteopen}eval\ {\isacharparenleft}{\kern0pt}sys\ {\isacharbang}{\kern0pt}\ r{\isacharparenright}{\kern0pt}\ v\ {\isacharequal}{\kern0pt}\ eval\ {\isacharparenleft}{\kern0pt}sys{\isacharprime}{\kern0pt}\ {\isacharbang}{\kern0pt}\ r{\isacharparenright}{\kern0pt}\ v{\isachardoublequoteclose}\isanewline
\ \ \ \ \isakeywordONE{by}\isamarkupfalse%
\ {\isacharparenleft}{\kern0pt}metis\ {\isacharparenleft}{\kern0pt}mono{\isacharunderscore}{\kern0pt}tags{\isacharcomma}{\kern0pt}\ lifting{\isacharparenright}{\kern0pt}\ eval{\isacharunderscore}{\kern0pt}vars{\isacharparenright}{\kern0pt}\isanewline
\ \ \isakeywordONE{with}\isamarkupfalse%
\ sol{\isacharunderscore}{\kern0pt}r{\isacharunderscore}{\kern0pt}is{\isacharunderscore}{\kern0pt}sol\ s{\isacharunderscore}{\kern0pt}r{\isacharunderscore}{\kern0pt}sol{\isacharunderscore}{\kern0pt}r\ \isakeywordONE{have}\isamarkupfalse%
\ {\isachardoublequoteopen}{\isasymPsi}\ {\isacharparenleft}{\kern0pt}eval\ {\isacharparenleft}{\kern0pt}sys\ {\isacharbang}{\kern0pt}\ r{\isacharparenright}{\kern0pt}\ v{\isacharparenright}{\kern0pt}\ {\isasymsubseteq}\ {\isasymPsi}\ {\isacharparenleft}{\kern0pt}v\ r{\isacharparenright}{\kern0pt}{\isachardoublequoteclose}\isanewline
\ \ \ \ \isakeywordONE{unfolding}\isamarkupfalse%
\ partial{\isacharunderscore}{\kern0pt}min{\isacharunderscore}{\kern0pt}sol{\isacharunderscore}{\kern0pt}one{\isacharunderscore}{\kern0pt}ineq{\isacharunderscore}{\kern0pt}def\ partial{\isacharunderscore}{\kern0pt}sol{\isacharunderscore}{\kern0pt}ineq{\isacharunderscore}{\kern0pt}def\ solves{\isacharunderscore}{\kern0pt}ineq{\isacharunderscore}{\kern0pt}comm{\isacharunderscore}{\kern0pt}def\ \isakeywordONE{by}\isamarkupfalse%
\ simp\isanewline
\ \ \isakeywordONE{with}\isamarkupfalse%
\ solves{\isacharunderscore}{\kern0pt}r{\isacharunderscore}{\kern0pt}sys\ \isakeywordTHREE{show}\isamarkupfalse%
\ {\isachardoublequoteopen}solves{\isacharunderscore}{\kern0pt}ineq{\isacharunderscore}{\kern0pt}sys{\isacharunderscore}{\kern0pt}comm\ {\isacharparenleft}{\kern0pt}take\ {\isacharparenleft}{\kern0pt}Suc\ r{\isacharparenright}{\kern0pt}\ sys{\isacharparenright}{\kern0pt}\ v{\isachardoublequoteclose}\isanewline
\ \ \ \ \isakeywordONE{unfolding}\isamarkupfalse%
\ solves{\isacharunderscore}{\kern0pt}ineq{\isacharunderscore}{\kern0pt}sys{\isacharunderscore}{\kern0pt}comm{\isacharunderscore}{\kern0pt}def\ solves{\isacharunderscore}{\kern0pt}ineq{\isacharunderscore}{\kern0pt}comm{\isacharunderscore}{\kern0pt}def\ \isakeywordONE{by}\isamarkupfalse%
\ {\isacharparenleft}{\kern0pt}auto\ simp\ add{\isacharcolon}{\kern0pt}\ less{\isacharunderscore}{\kern0pt}Suc{\isacharunderscore}{\kern0pt}eq{\isacharparenright}{\kern0pt}\isanewline
\isakeywordONE{qed}\isamarkupfalse%
%
\endisatagproof
{\isafoldproof}%
%
\isadelimproof
\isanewline
%
\endisadelimproof
\isanewline
\isakeywordONE{lemma}\isamarkupfalse%
\ sols{\isacharprime}{\kern0pt}{\isacharunderscore}{\kern0pt}min{\isacharcolon}{\kern0pt}\ {\isachardoublequoteopen}{\isasymforall}sols{\isadigit{2}}\ v{\isadigit{2}}{\isachardot}{\kern0pt}\ {\isacharparenleft}{\kern0pt}{\isasymforall}x{\isachardot}{\kern0pt}\ v{\isadigit{2}}\ x\ {\isacharequal}{\kern0pt}\ eval\ {\isacharparenleft}{\kern0pt}sols{\isadigit{2}}\ x{\isacharparenright}{\kern0pt}\ v{\isadigit{2}}{\isacharparenright}{\kern0pt}\isanewline
\ \ \ \ \ \ \ \ \ \ \ \ \ \ \ \ \ \ \ {\isasymand}\ solves{\isacharunderscore}{\kern0pt}ineq{\isacharunderscore}{\kern0pt}sys{\isacharunderscore}{\kern0pt}comm\ {\isacharparenleft}{\kern0pt}take\ {\isacharparenleft}{\kern0pt}Suc\ r{\isacharparenright}{\kern0pt}\ sys{\isacharparenright}{\kern0pt}\ v{\isadigit{2}}\isanewline
\ \ \ \ \ \ \ \ \ \ \ \ \ \ \ \ \ \ \ {\isasymlongrightarrow}\ {\isacharparenleft}{\kern0pt}{\isasymforall}i{\isachardot}{\kern0pt}\ {\isasymPsi}\ {\isacharparenleft}{\kern0pt}eval\ {\isacharparenleft}{\kern0pt}sols{\isacharprime}{\kern0pt}\ i{\isacharparenright}{\kern0pt}\ v{\isadigit{2}}{\isacharparenright}{\kern0pt}\ {\isasymsubseteq}\ {\isasymPsi}\ {\isacharparenleft}{\kern0pt}v{\isadigit{2}}\ i{\isacharparenright}{\kern0pt}{\isacharparenright}{\kern0pt}{\isachardoublequoteclose}\isanewline
%
\isadelimproof
%
\endisadelimproof
%
\isatagproof
\isakeywordONE{proof}\isamarkupfalse%
\ {\isacharparenleft}{\kern0pt}rule\ allI\ {\isacharbar}{\kern0pt}\ rule\ impI{\isacharparenright}{\kern0pt}{\isacharplus}{\kern0pt}\isanewline
\ \ \isakeywordTHREE{fix}\isamarkupfalse%
\ sols{\isadigit{2}}\ v{\isadigit{2}}\ i\isanewline
\ \ \isakeywordTHREE{assume}\isamarkupfalse%
\ as{\isacharcolon}{\kern0pt}\ {\isachardoublequoteopen}{\isacharparenleft}{\kern0pt}{\isasymforall}x{\isachardot}{\kern0pt}\ v{\isadigit{2}}\ x\ {\isacharequal}{\kern0pt}\ eval\ {\isacharparenleft}{\kern0pt}sols{\isadigit{2}}\ x{\isacharparenright}{\kern0pt}\ v{\isadigit{2}}{\isacharparenright}{\kern0pt}\ {\isasymand}\ solves{\isacharunderscore}{\kern0pt}ineq{\isacharunderscore}{\kern0pt}sys{\isacharunderscore}{\kern0pt}comm\ {\isacharparenleft}{\kern0pt}take\ {\isacharparenleft}{\kern0pt}Suc\ r{\isacharparenright}{\kern0pt}\ sys{\isacharparenright}{\kern0pt}\ v{\isadigit{2}}{\isachardoublequoteclose}\isanewline
\ \ \isakeywordONE{then}\isamarkupfalse%
\ \isakeywordONE{have}\isamarkupfalse%
\ {\isachardoublequoteopen}solves{\isacharunderscore}{\kern0pt}ineq{\isacharunderscore}{\kern0pt}sys{\isacharunderscore}{\kern0pt}comm\ {\isacharparenleft}{\kern0pt}take\ r\ sys{\isacharparenright}{\kern0pt}\ v{\isadigit{2}}{\isachardoublequoteclose}\ \isakeywordONE{unfolding}\isamarkupfalse%
\ solves{\isacharunderscore}{\kern0pt}ineq{\isacharunderscore}{\kern0pt}sys{\isacharunderscore}{\kern0pt}comm{\isacharunderscore}{\kern0pt}def\ \isakeywordONE{by}\isamarkupfalse%
\ fastforce\isanewline
\ \ \isakeywordONE{with}\isamarkupfalse%
\ as\ sols{\isacharunderscore}{\kern0pt}is{\isacharunderscore}{\kern0pt}sol\ \isakeywordONE{have}\isamarkupfalse%
\ sols{\isacharunderscore}{\kern0pt}s{\isadigit{2}}{\isacharcolon}{\kern0pt}\ {\isachardoublequoteopen}{\isasymPsi}\ {\isacharparenleft}{\kern0pt}eval\ {\isacharparenleft}{\kern0pt}sols\ i{\isacharparenright}{\kern0pt}\ v{\isadigit{2}}{\isacharparenright}{\kern0pt}\ {\isasymsubseteq}\ {\isasymPsi}\ {\isacharparenleft}{\kern0pt}v{\isadigit{2}}\ i{\isacharparenright}{\kern0pt}{\isachardoublequoteclose}\ \isakeywordTWO{for}\ i\isanewline
\ \ \ \ \isakeywordONE{unfolding}\isamarkupfalse%
\ partial{\isacharunderscore}{\kern0pt}min{\isacharunderscore}{\kern0pt}sol{\isacharunderscore}{\kern0pt}ineq{\isacharunderscore}{\kern0pt}sys{\isacharunderscore}{\kern0pt}def\ \isakeywordONE{by}\isamarkupfalse%
\ auto\isanewline
\ \ \isakeywordONE{have}\isamarkupfalse%
\ {\isachardoublequoteopen}eval\ {\isacharparenleft}{\kern0pt}sys{\isacharprime}{\kern0pt}\ {\isacharbang}{\kern0pt}\ r{\isacharparenright}{\kern0pt}\ v{\isadigit{2}}\ {\isacharequal}{\kern0pt}\ eval\ {\isacharparenleft}{\kern0pt}sys\ {\isacharbang}{\kern0pt}\ r{\isacharparenright}{\kern0pt}\ {\isacharparenleft}{\kern0pt}{\isasymlambda}i{\isachardot}{\kern0pt}\ eval\ {\isacharparenleft}{\kern0pt}sols\ i{\isacharparenright}{\kern0pt}\ v{\isadigit{2}}{\isacharparenright}{\kern0pt}{\isachardoublequoteclose}\isanewline
\ \ \ \ \isakeywordONE{unfolding}\isamarkupfalse%
\ subst{\isacharunderscore}{\kern0pt}sys{\isacharunderscore}{\kern0pt}def\ \isakeywordONE{using}\isamarkupfalse%
\ substitution{\isacharunderscore}{\kern0pt}lemma{\isacharbrackleft}{\kern0pt}\isakeywordTWO{where}\ f{\isacharequal}{\kern0pt}{\isachardoublequoteopen}sys\ {\isacharbang}{\kern0pt}\ r{\isachardoublequoteclose}{\isacharbrackright}{\kern0pt}\isanewline
\ \ \ \ \isakeywordONE{by}\isamarkupfalse%
\ {\isacharparenleft}{\kern0pt}simp\ add{\isacharcolon}{\kern0pt}\ r{\isacharunderscore}{\kern0pt}valid\ Suc{\isacharunderscore}{\kern0pt}le{\isacharunderscore}{\kern0pt}lessD{\isacharparenright}{\kern0pt}\isanewline
\ \ \isakeywordONE{with}\isamarkupfalse%
\ sols{\isacharunderscore}{\kern0pt}s{\isadigit{2}}\ \isakeywordONE{have}\isamarkupfalse%
\ {\isachardoublequoteopen}{\isasymPsi}\ {\isacharparenleft}{\kern0pt}eval\ {\isacharparenleft}{\kern0pt}sys{\isacharprime}{\kern0pt}\ {\isacharbang}{\kern0pt}\ r{\isacharparenright}{\kern0pt}\ v{\isadigit{2}}{\isacharparenright}{\kern0pt}\ {\isasymsubseteq}\ {\isasymPsi}\ {\isacharparenleft}{\kern0pt}eval\ {\isacharparenleft}{\kern0pt}sys\ {\isacharbang}{\kern0pt}\ r{\isacharparenright}{\kern0pt}\ v{\isadigit{2}}{\isacharparenright}{\kern0pt}{\isachardoublequoteclose}\isanewline
\ \ \ \ \isakeywordONE{using}\isamarkupfalse%
\ rlexp{\isacharunderscore}{\kern0pt}mono{\isacharunderscore}{\kern0pt}parikh{\isacharbrackleft}{\kern0pt}of\ {\isachardoublequoteopen}sys\ {\isacharbang}{\kern0pt}\ r{\isachardoublequoteclose}{\isacharbrackright}{\kern0pt}\ \isakeywordONE{by}\isamarkupfalse%
\ auto\isanewline
\ \ \isakeywordONE{with}\isamarkupfalse%
\ as\ \isakeywordONE{have}\isamarkupfalse%
\ {\isachardoublequoteopen}solves{\isacharunderscore}{\kern0pt}ineq{\isacharunderscore}{\kern0pt}comm\ r\ {\isacharparenleft}{\kern0pt}sys{\isacharprime}{\kern0pt}\ {\isacharbang}{\kern0pt}\ r{\isacharparenright}{\kern0pt}\ v{\isadigit{2}}{\isachardoublequoteclose}\isanewline
\ \ \ \ \isakeywordONE{unfolding}\isamarkupfalse%
\ solves{\isacharunderscore}{\kern0pt}ineq{\isacharunderscore}{\kern0pt}sys{\isacharunderscore}{\kern0pt}comm{\isacharunderscore}{\kern0pt}def\ solves{\isacharunderscore}{\kern0pt}ineq{\isacharunderscore}{\kern0pt}comm{\isacharunderscore}{\kern0pt}def\ \isakeywordONE{using}\isamarkupfalse%
\ r{\isacharunderscore}{\kern0pt}valid\ \isakeywordONE{by}\isamarkupfalse%
\ force\isanewline
\ \ \isakeywordONE{with}\isamarkupfalse%
\ as\ sol{\isacharunderscore}{\kern0pt}r{\isacharunderscore}{\kern0pt}is{\isacharunderscore}{\kern0pt}sol\ \isakeywordONE{have}\isamarkupfalse%
\ sol{\isacharunderscore}{\kern0pt}r{\isacharunderscore}{\kern0pt}min{\isacharcolon}{\kern0pt}\ {\isachardoublequoteopen}{\isasymPsi}\ {\isacharparenleft}{\kern0pt}eval\ sol{\isacharunderscore}{\kern0pt}r\ v{\isadigit{2}}{\isacharparenright}{\kern0pt}\ {\isasymsubseteq}\ {\isasymPsi}\ {\isacharparenleft}{\kern0pt}v{\isadigit{2}}\ r{\isacharparenright}{\kern0pt}{\isachardoublequoteclose}\isanewline
\ \ \ \ \isakeywordONE{unfolding}\isamarkupfalse%
\ partial{\isacharunderscore}{\kern0pt}min{\isacharunderscore}{\kern0pt}sol{\isacharunderscore}{\kern0pt}one{\isacharunderscore}{\kern0pt}ineq{\isacharunderscore}{\kern0pt}def\ \isakeywordONE{by}\isamarkupfalse%
\ blast\isanewline
\ \ \isakeywordONE{let}\isamarkupfalse%
\ {\isacharquery}{\kern0pt}v{\isacharprime}{\kern0pt}\ {\isacharequal}{\kern0pt}\ {\isachardoublequoteopen}v{\isadigit{2}}{\isacharparenleft}{\kern0pt}r\ {\isacharcolon}{\kern0pt}{\isacharequal}{\kern0pt}\ eval\ sol{\isacharunderscore}{\kern0pt}r\ v{\isadigit{2}}{\isacharparenright}{\kern0pt}{\isachardoublequoteclose}\isanewline
\ \ \isakeywordONE{from}\isamarkupfalse%
\ sol{\isacharunderscore}{\kern0pt}r{\isacharunderscore}{\kern0pt}min\ \isakeywordONE{have}\isamarkupfalse%
\ {\isachardoublequoteopen}{\isasymPsi}\ {\isacharparenleft}{\kern0pt}{\isacharquery}{\kern0pt}v{\isacharprime}{\kern0pt}\ i{\isacharparenright}{\kern0pt}\ {\isasymsubseteq}\ {\isasymPsi}\ {\isacharparenleft}{\kern0pt}v{\isadigit{2}}\ i{\isacharparenright}{\kern0pt}{\isachardoublequoteclose}\ \isakeywordTWO{for}\ i\ \isakeywordONE{by}\isamarkupfalse%
\ simp\isanewline
\ \ \isakeywordONE{with}\isamarkupfalse%
\ sols{\isacharunderscore}{\kern0pt}s{\isadigit{2}}\ \isakeywordTHREE{show}\isamarkupfalse%
\ {\isachardoublequoteopen}{\isasymPsi}\ {\isacharparenleft}{\kern0pt}eval\ {\isacharparenleft}{\kern0pt}sols{\isacharprime}{\kern0pt}\ i{\isacharparenright}{\kern0pt}\ v{\isadigit{2}}{\isacharparenright}{\kern0pt}\ {\isasymsubseteq}\ {\isasymPsi}\ {\isacharparenleft}{\kern0pt}v{\isadigit{2}}\ i{\isacharparenright}{\kern0pt}{\isachardoublequoteclose}\isanewline
\ \ \ \ \isakeywordONE{using}\isamarkupfalse%
\ substitution{\isacharunderscore}{\kern0pt}lemma{\isacharunderscore}{\kern0pt}upd{\isacharbrackleft}{\kern0pt}\isakeywordTWO{where}\ f{\isacharequal}{\kern0pt}{\isachardoublequoteopen}sols\ i{\isachardoublequoteclose}{\isacharbrackright}{\kern0pt}\ rlexp{\isacharunderscore}{\kern0pt}mono{\isacharunderscore}{\kern0pt}parikh{\isacharbrackleft}{\kern0pt}of\ {\isachardoublequoteopen}sols\ i{\isachardoublequoteclose}\ {\isacharquery}{\kern0pt}v{\isacharprime}{\kern0pt}\ v{\isadigit{2}}{\isacharbrackright}{\kern0pt}\ \isakeywordONE{by}\isamarkupfalse%
\ force\isanewline
\isakeywordONE{qed}\isamarkupfalse%
%
\endisatagproof
{\isafoldproof}%
%
\isadelimproof
\isanewline
%
\endisadelimproof
\isanewline
\isakeywordONE{lemma}\isamarkupfalse%
\ sols{\isacharprime}{\kern0pt}{\isacharunderscore}{\kern0pt}vars{\isacharunderscore}{\kern0pt}gt{\isacharunderscore}{\kern0pt}r{\isacharcolon}{\kern0pt}\ {\isachardoublequoteopen}{\isasymforall}i\ {\isasymge}\ Suc\ r{\isachardot}{\kern0pt}\ sols{\isacharprime}{\kern0pt}\ i\ {\isacharequal}{\kern0pt}\ Var\ i{\isachardoublequoteclose}\isanewline
%
\isadelimproof
\ \ %
\endisadelimproof
%
\isatagproof
\isakeywordONE{using}\isamarkupfalse%
\ sols{\isacharunderscore}{\kern0pt}is{\isacharunderscore}{\kern0pt}sol\ \isakeywordONE{unfolding}\isamarkupfalse%
\ partial{\isacharunderscore}{\kern0pt}min{\isacharunderscore}{\kern0pt}sol{\isacharunderscore}{\kern0pt}ineq{\isacharunderscore}{\kern0pt}sys{\isacharunderscore}{\kern0pt}def\ \isakeywordONE{by}\isamarkupfalse%
\ auto%
\endisatagproof
{\isafoldproof}%
%
\isadelimproof
\isanewline
%
\endisadelimproof
\isanewline
\isakeywordONE{lemma}\isamarkupfalse%
\ sols{\isacharprime}{\kern0pt}{\isacharunderscore}{\kern0pt}vars{\isacharunderscore}{\kern0pt}leq{\isacharunderscore}{\kern0pt}r{\isacharcolon}{\kern0pt}\ {\isachardoublequoteopen}{\isasymforall}i\ {\isacharless}{\kern0pt}\ Suc\ r{\isachardot}{\kern0pt}\ {\isasymforall}x\ {\isasymin}\ vars\ {\isacharparenleft}{\kern0pt}sols{\isacharprime}{\kern0pt}\ i{\isacharparenright}{\kern0pt}{\isachardot}{\kern0pt}\ x\ {\isasymge}\ Suc\ r\ {\isasymand}\ x\ {\isacharless}{\kern0pt}\ length\ sys{\isachardoublequoteclose}\isanewline
%
\isadelimproof
%
\endisadelimproof
%
\isatagproof
\isakeywordONE{proof}\isamarkupfalse%
\ {\isacharminus}{\kern0pt}\isanewline
\ \ \isakeywordONE{from}\isamarkupfalse%
\ sols{\isacharunderscore}{\kern0pt}is{\isacharunderscore}{\kern0pt}sol\ \isakeywordONE{have}\isamarkupfalse%
\ {\isachardoublequoteopen}{\isasymforall}i\ {\isacharless}{\kern0pt}\ r{\isachardot}{\kern0pt}\ {\isasymforall}x\ {\isasymin}\ vars\ {\isacharparenleft}{\kern0pt}sols\ i{\isacharparenright}{\kern0pt}{\isachardot}{\kern0pt}\ x\ {\isasymge}\ r\ {\isasymand}\ x\ {\isacharless}{\kern0pt}\ length\ sys{\isachardoublequoteclose}\isanewline
\ \ \ \ \isakeywordONE{unfolding}\isamarkupfalse%
\ partial{\isacharunderscore}{\kern0pt}min{\isacharunderscore}{\kern0pt}sol{\isacharunderscore}{\kern0pt}ineq{\isacharunderscore}{\kern0pt}sys{\isacharunderscore}{\kern0pt}def\ \isakeywordONE{by}\isamarkupfalse%
\ simp\isanewline
\ \ \isakeywordONE{with}\isamarkupfalse%
\ sols{\isacharunderscore}{\kern0pt}is{\isacharunderscore}{\kern0pt}sol\ \isakeywordONE{have}\isamarkupfalse%
\ vars{\isacharunderscore}{\kern0pt}sols{\isacharcolon}{\kern0pt}\ {\isachardoublequoteopen}{\isasymforall}i\ {\isacharless}{\kern0pt}\ length\ sys{\isachardot}{\kern0pt}\ {\isasymforall}x\ {\isasymin}\ vars\ {\isacharparenleft}{\kern0pt}sols\ i{\isacharparenright}{\kern0pt}{\isachardot}{\kern0pt}\ x\ {\isasymge}\ r\ {\isasymand}\ x\ {\isacharless}{\kern0pt}\ length\ sys{\isachardoublequoteclose}\isanewline
\ \ \ \ \isakeywordONE{unfolding}\isamarkupfalse%
\ partial{\isacharunderscore}{\kern0pt}min{\isacharunderscore}{\kern0pt}sol{\isacharunderscore}{\kern0pt}ineq{\isacharunderscore}{\kern0pt}sys{\isacharunderscore}{\kern0pt}def\ \isakeywordONE{by}\isamarkupfalse%
\ {\isacharparenleft}{\kern0pt}metis\ empty{\isacharunderscore}{\kern0pt}iff\ insert{\isacharunderscore}{\kern0pt}iff\ leI\ vars{\isachardot}{\kern0pt}simps{\isacharparenleft}{\kern0pt}{\isadigit{1}}{\isacharparenright}{\kern0pt}{\isacharparenright}{\kern0pt}\isanewline
\ \ \isakeywordONE{with}\isamarkupfalse%
\ sys{\isacharunderscore}{\kern0pt}valid\ \isakeywordONE{have}\isamarkupfalse%
\ {\isachardoublequoteopen}{\isasymforall}x\ {\isasymin}\ vars\ {\isacharparenleft}{\kern0pt}subst\ sols\ {\isacharparenleft}{\kern0pt}sys\ {\isacharbang}{\kern0pt}\ i{\isacharparenright}{\kern0pt}{\isacharparenright}{\kern0pt}{\isachardot}{\kern0pt}\ x\ {\isasymge}\ r\ {\isasymand}\ x\ {\isacharless}{\kern0pt}\ length\ sys{\isachardoublequoteclose}\ \isakeywordTWO{if}\ {\isachardoublequoteopen}i\ {\isacharless}{\kern0pt}\ length\ sys{\isachardoublequoteclose}\ \isakeywordTWO{for}\ i\isanewline
\ \ \ \ \isakeywordONE{using}\isamarkupfalse%
\ vars{\isacharunderscore}{\kern0pt}subst{\isacharbrackleft}{\kern0pt}of\ sols\ {\isachardoublequoteopen}sys\ {\isacharbang}{\kern0pt}\ i{\isachardoublequoteclose}{\isacharbrackright}{\kern0pt}\ that\ \isakeywordONE{by}\isamarkupfalse%
\ {\isacharparenleft}{\kern0pt}metis\ UN{\isacharunderscore}{\kern0pt}E\ nth{\isacharunderscore}{\kern0pt}mem{\isacharparenright}{\kern0pt}\isanewline
\ \ \isakeywordONE{then}\isamarkupfalse%
\ \isakeywordONE{have}\isamarkupfalse%
\ {\isachardoublequoteopen}{\isasymforall}x\ {\isasymin}\ vars\ {\isacharparenleft}{\kern0pt}sys{\isacharprime}{\kern0pt}\ {\isacharbang}{\kern0pt}\ i{\isacharparenright}{\kern0pt}{\isachardot}{\kern0pt}\ x\ {\isasymge}\ r\ {\isasymand}\ x\ {\isacharless}{\kern0pt}\ length\ sys{\isachardoublequoteclose}\ \isakeywordTWO{if}\ {\isachardoublequoteopen}i\ {\isacharless}{\kern0pt}\ length\ sys{\isachardoublequoteclose}\ \isakeywordTWO{for}\ i\isanewline
\ \ \ \ \isakeywordONE{unfolding}\isamarkupfalse%
\ subst{\isacharunderscore}{\kern0pt}sys{\isacharunderscore}{\kern0pt}def\ \isakeywordONE{using}\isamarkupfalse%
\ r{\isacharunderscore}{\kern0pt}valid\ that\ \isakeywordONE{by}\isamarkupfalse%
\ auto\isanewline
\ \ \isakeywordONE{moreover}\isamarkupfalse%
\ \isakeywordONE{from}\isamarkupfalse%
\ sol{\isacharunderscore}{\kern0pt}r{\isacharunderscore}{\kern0pt}is{\isacharunderscore}{\kern0pt}sol\ \isakeywordONE{have}\isamarkupfalse%
\ {\isachardoublequoteopen}vars\ {\isacharparenleft}{\kern0pt}sol{\isacharunderscore}{\kern0pt}r{\isacharparenright}{\kern0pt}\ {\isasymsubseteq}\ vars\ {\isacharparenleft}{\kern0pt}sys{\isacharprime}{\kern0pt}\ {\isacharbang}{\kern0pt}\ r{\isacharparenright}{\kern0pt}\ {\isacharminus}{\kern0pt}\ {\isacharbraceleft}{\kern0pt}r{\isacharbraceright}{\kern0pt}{\isachardoublequoteclose}\isanewline
\ \ \ \ \isakeywordONE{unfolding}\isamarkupfalse%
\ partial{\isacharunderscore}{\kern0pt}min{\isacharunderscore}{\kern0pt}sol{\isacharunderscore}{\kern0pt}one{\isacharunderscore}{\kern0pt}ineq{\isacharunderscore}{\kern0pt}def\ \isakeywordONE{by}\isamarkupfalse%
\ simp\isanewline
\ \ \isakeywordONE{ultimately}\isamarkupfalse%
\ \isakeywordONE{have}\isamarkupfalse%
\ vars{\isacharunderscore}{\kern0pt}sol{\isacharunderscore}{\kern0pt}r{\isacharcolon}{\kern0pt}\ {\isachardoublequoteopen}{\isasymforall}x\ {\isasymin}\ vars\ sol{\isacharunderscore}{\kern0pt}r{\isachardot}{\kern0pt}\ x\ {\isachargreater}{\kern0pt}\ r\ {\isasymand}\ x\ {\isacharless}{\kern0pt}\ length\ sys{\isachardoublequoteclose}\isanewline
\ \ \ \ \isakeywordONE{unfolding}\isamarkupfalse%
\ partial{\isacharunderscore}{\kern0pt}min{\isacharunderscore}{\kern0pt}sol{\isacharunderscore}{\kern0pt}one{\isacharunderscore}{\kern0pt}ineq{\isacharunderscore}{\kern0pt}def\ \isakeywordONE{using}\isamarkupfalse%
\ r{\isacharunderscore}{\kern0pt}valid\isanewline
\ \ \ \ \isakeywordONE{by}\isamarkupfalse%
\ {\isacharparenleft}{\kern0pt}metis\ DiffE\ insertCI\ nat{\isacharunderscore}{\kern0pt}less{\isacharunderscore}{\kern0pt}le\ subsetD{\isacharparenright}{\kern0pt}\isanewline
\ \ \isakeywordONE{moreover}\isamarkupfalse%
\ \isakeywordONE{have}\isamarkupfalse%
\ {\isachardoublequoteopen}vars\ {\isacharparenleft}{\kern0pt}sols{\isacharprime}{\kern0pt}\ i{\isacharparenright}{\kern0pt}\ {\isasymsubseteq}\ vars\ {\isacharparenleft}{\kern0pt}sols\ i{\isacharparenright}{\kern0pt}\ {\isacharminus}{\kern0pt}\ {\isacharbraceleft}{\kern0pt}r{\isacharbraceright}{\kern0pt}\ {\isasymunion}\ vars\ sol{\isacharunderscore}{\kern0pt}r{\isachardoublequoteclose}\ \isakeywordTWO{if}\ {\isachardoublequoteopen}i\ {\isacharless}{\kern0pt}\ length\ sys{\isachardoublequoteclose}\ \isakeywordTWO{for}\ i\isanewline
\ \ \ \ \isakeywordONE{using}\isamarkupfalse%
\ vars{\isacharunderscore}{\kern0pt}subst{\isacharunderscore}{\kern0pt}upd{\isacharunderscore}{\kern0pt}upper\ \isakeywordONE{by}\isamarkupfalse%
\ meson\isanewline
\ \ \isakeywordONE{ultimately}\isamarkupfalse%
\ \isakeywordONE{have}\isamarkupfalse%
\ {\isachardoublequoteopen}{\isasymforall}x\ {\isasymin}\ vars\ {\isacharparenleft}{\kern0pt}sols{\isacharprime}{\kern0pt}\ i{\isacharparenright}{\kern0pt}{\isachardot}{\kern0pt}\ x\ {\isachargreater}{\kern0pt}\ r\ {\isasymand}\ x\ {\isacharless}{\kern0pt}\ length\ sys{\isachardoublequoteclose}\ \isakeywordTWO{if}\ {\isachardoublequoteopen}i\ {\isacharless}{\kern0pt}\ length\ sys{\isachardoublequoteclose}\ \isakeywordTWO{for}\ i\isanewline
\ \ \ \ \isakeywordONE{using}\isamarkupfalse%
\ vars{\isacharunderscore}{\kern0pt}sols\ that\ \isakeywordONE{by}\isamarkupfalse%
\ fastforce\isanewline
\ \ \isakeywordONE{then}\isamarkupfalse%
\ \isakeywordTHREE{show}\isamarkupfalse%
\ {\isacharquery}{\kern0pt}thesis\ \isakeywordONE{by}\isamarkupfalse%
\ {\isacharparenleft}{\kern0pt}meson\ r{\isacharunderscore}{\kern0pt}valid\ Suc{\isacharunderscore}{\kern0pt}le{\isacharunderscore}{\kern0pt}eq\ dual{\isacharunderscore}{\kern0pt}order{\isachardot}{\kern0pt}strict{\isacharunderscore}{\kern0pt}trans{\isadigit{1}}{\isacharparenright}{\kern0pt}\isanewline
\isakeywordONE{qed}\isamarkupfalse%
%
\endisatagproof
{\isafoldproof}%
%
\isadelimproof
%
\endisadelimproof
%
\begin{isamarkuptext}%
In summary, \isa{\isaconst{sols{\isacharprime}{\kern0pt}}} is a minimal partial solution of the first \isa{Suc\ r} equations. This
allows us to prove the centerpiece of the induction step in the next lemma, namely that there exists
a \isa{\isaconst{reg{\isacharunderscore}{\kern0pt}eval}} and minimal partial solution for the first \isa{Suc\ r} equations:%
\end{isamarkuptext}\isamarkuptrue%
\isakeywordONE{lemma}\isamarkupfalse%
\ sols{\isacharprime}{\kern0pt}{\isacharunderscore}{\kern0pt}is{\isacharunderscore}{\kern0pt}min{\isacharunderscore}{\kern0pt}sol{\isacharcolon}{\kern0pt}\ {\isachardoublequoteopen}partial{\isacharunderscore}{\kern0pt}min{\isacharunderscore}{\kern0pt}sol{\isacharunderscore}{\kern0pt}ineq{\isacharunderscore}{\kern0pt}sys\ {\isacharparenleft}{\kern0pt}Suc\ r{\isacharparenright}{\kern0pt}\ sys\ sols{\isacharprime}{\kern0pt}{\isachardoublequoteclose}\isanewline
%
\isadelimproof
\ \ %
\endisadelimproof
%
\isatagproof
\isakeywordONE{unfolding}\isamarkupfalse%
\ partial{\isacharunderscore}{\kern0pt}min{\isacharunderscore}{\kern0pt}sol{\isacharunderscore}{\kern0pt}ineq{\isacharunderscore}{\kern0pt}sys{\isacharunderscore}{\kern0pt}def\isanewline
\ \ \isakeywordONE{using}\isamarkupfalse%
\ sols{\isacharprime}{\kern0pt}{\isacharunderscore}{\kern0pt}is{\isacharunderscore}{\kern0pt}sol\ sols{\isacharprime}{\kern0pt}{\isacharunderscore}{\kern0pt}min\ sols{\isacharprime}{\kern0pt}{\isacharunderscore}{\kern0pt}vars{\isacharunderscore}{\kern0pt}gt{\isacharunderscore}{\kern0pt}r\ sols{\isacharprime}{\kern0pt}{\isacharunderscore}{\kern0pt}vars{\isacharunderscore}{\kern0pt}leq{\isacharunderscore}{\kern0pt}r\isanewline
\ \ \isakeywordONE{by}\isamarkupfalse%
\ blast%
\endisatagproof
{\isafoldproof}%
%
\isadelimproof
\isanewline
%
\endisadelimproof
\isanewline
\isakeywordONE{lemma}\isamarkupfalse%
\ exists{\isacharunderscore}{\kern0pt}min{\isacharunderscore}{\kern0pt}sol{\isacharunderscore}{\kern0pt}Suc{\isacharunderscore}{\kern0pt}r{\isacharcolon}{\kern0pt}\isanewline
\ \ {\isachardoublequoteopen}{\isasymexists}sols{\isacharprime}{\kern0pt}{\isachardot}{\kern0pt}\ partial{\isacharunderscore}{\kern0pt}min{\isacharunderscore}{\kern0pt}sol{\isacharunderscore}{\kern0pt}ineq{\isacharunderscore}{\kern0pt}sys\ {\isacharparenleft}{\kern0pt}Suc\ r{\isacharparenright}{\kern0pt}\ sys\ sols{\isacharprime}{\kern0pt}\ {\isasymand}\ {\isacharparenleft}{\kern0pt}{\isasymforall}i{\isachardot}{\kern0pt}\ reg{\isacharunderscore}{\kern0pt}eval\ {\isacharparenleft}{\kern0pt}sols{\isacharprime}{\kern0pt}\ i{\isacharparenright}{\kern0pt}{\isacharparenright}{\kern0pt}{\isachardoublequoteclose}\isanewline
%
\isadelimproof
\ \ %
\endisadelimproof
%
\isatagproof
\isakeywordONE{using}\isamarkupfalse%
\ sols{\isacharprime}{\kern0pt}{\isacharunderscore}{\kern0pt}reg\ sols{\isacharprime}{\kern0pt}{\isacharunderscore}{\kern0pt}is{\isacharunderscore}{\kern0pt}min{\isacharunderscore}{\kern0pt}sol\ \isakeywordONE{by}\isamarkupfalse%
\ blast%
\endisatagproof
{\isafoldproof}%
%
\isadelimproof
\isanewline
%
\endisadelimproof
\isanewline
\isakeywordTWO{end}\isamarkupfalse%
%
\begin{isamarkuptext}%
Now follows the actual induction proof: For every \isa{r}, there exists a \isa{\isaconst{reg{\isacharunderscore}{\kern0pt}eval}} and minimal partial
solution of the first \isa{r} equations. This then implies that there exists a regular and minimal (non-partial)
solution of the whole system:%
\end{isamarkuptext}\isamarkuptrue%
\isakeywordONE{lemma}\isamarkupfalse%
\ exists{\isacharunderscore}{\kern0pt}minimal{\isacharunderscore}{\kern0pt}reg{\isacharunderscore}{\kern0pt}sol{\isacharunderscore}{\kern0pt}sys{\isacharunderscore}{\kern0pt}aux{\isacharcolon}{\kern0pt}\isanewline
\ \ \isakeywordTWO{assumes}\ eqs{\isacharunderscore}{\kern0pt}reg{\isacharcolon}{\kern0pt}\ \ \ {\isachardoublequoteopen}{\isasymforall}eq\ {\isasymin}\ set\ sys{\isachardot}{\kern0pt}\ reg{\isacharunderscore}{\kern0pt}eval\ eq{\isachardoublequoteclose}\isanewline
\ \ \ \ \ \ \isakeywordTWO{and}\ sys{\isacharunderscore}{\kern0pt}valid{\isacharcolon}{\kern0pt}\ {\isachardoublequoteopen}{\isasymforall}eq\ {\isasymin}\ set\ sys{\isachardot}{\kern0pt}\ {\isasymforall}x\ {\isasymin}\ vars\ eq{\isachardot}{\kern0pt}\ x\ {\isacharless}{\kern0pt}\ length\ sys{\isachardoublequoteclose}\isanewline
\ \ \ \ \ \ \isakeywordTWO{and}\ r{\isacharunderscore}{\kern0pt}valid{\isacharcolon}{\kern0pt}\ \ \ {\isachardoublequoteopen}r\ {\isasymle}\ length\ sys{\isachardoublequoteclose}\ \ \ \isanewline
\ \ \ \ \isakeywordTWO{shows}\ \ \ \ \ \ \ \ \ \ \ \ {\isachardoublequoteopen}{\isasymexists}sols{\isachardot}{\kern0pt}\ partial{\isacharunderscore}{\kern0pt}min{\isacharunderscore}{\kern0pt}sol{\isacharunderscore}{\kern0pt}ineq{\isacharunderscore}{\kern0pt}sys\ r\ sys\ sols\ {\isasymand}\ {\isacharparenleft}{\kern0pt}{\isasymforall}i{\isachardot}{\kern0pt}\ reg{\isacharunderscore}{\kern0pt}eval\ {\isacharparenleft}{\kern0pt}sols\ i{\isacharparenright}{\kern0pt}{\isacharparenright}{\kern0pt}{\isachardoublequoteclose}\isanewline
%
\isadelimproof
%
\endisadelimproof
%
\isatagproof
\isakeywordONE{using}\isamarkupfalse%
\ r{\isacharunderscore}{\kern0pt}valid\ \isakeywordONE{proof}\isamarkupfalse%
\ {\isacharparenleft}{\kern0pt}induction\ r{\isacharparenright}{\kern0pt}\isanewline
\ \ \isakeywordTHREE{case}\isamarkupfalse%
\ {\isadigit{0}}\isanewline
\ \ \isakeywordONE{have}\isamarkupfalse%
\ {\isachardoublequoteopen}solution{\isacharunderscore}{\kern0pt}ineq{\isacharunderscore}{\kern0pt}sys\ {\isacharparenleft}{\kern0pt}take\ {\isadigit{0}}\ sys{\isacharparenright}{\kern0pt}\ Var{\isachardoublequoteclose}\isanewline
\ \ \ \ \isakeywordONE{unfolding}\isamarkupfalse%
\ solution{\isacharunderscore}{\kern0pt}ineq{\isacharunderscore}{\kern0pt}sys{\isacharunderscore}{\kern0pt}def\ solves{\isacharunderscore}{\kern0pt}ineq{\isacharunderscore}{\kern0pt}sys{\isacharunderscore}{\kern0pt}comm{\isacharunderscore}{\kern0pt}def\ \isakeywordONE{by}\isamarkupfalse%
\ simp\isanewline
\ \ \isakeywordONE{then}\isamarkupfalse%
\ \isakeywordTHREE{show}\isamarkupfalse%
\ {\isacharquery}{\kern0pt}case\ \isakeywordONE{unfolding}\isamarkupfalse%
\ partial{\isacharunderscore}{\kern0pt}min{\isacharunderscore}{\kern0pt}sol{\isacharunderscore}{\kern0pt}ineq{\isacharunderscore}{\kern0pt}sys{\isacharunderscore}{\kern0pt}def\ \isakeywordONE{by}\isamarkupfalse%
\ auto\isanewline
\isakeywordONE{next}\isamarkupfalse%
\isanewline
\ \ \isakeywordTHREE{case}\isamarkupfalse%
\ {\isacharparenleft}{\kern0pt}Suc\ r{\isacharparenright}{\kern0pt}\isanewline
\ \ \isakeywordONE{then}\isamarkupfalse%
\ \isakeywordTHREE{obtain}\isamarkupfalse%
\ sols\ \isakeywordTWO{where}\ sols{\isacharunderscore}{\kern0pt}intro{\isacharcolon}{\kern0pt}\ {\isachardoublequoteopen}partial{\isacharunderscore}{\kern0pt}min{\isacharunderscore}{\kern0pt}sol{\isacharunderscore}{\kern0pt}ineq{\isacharunderscore}{\kern0pt}sys\ r\ sys\ sols\ {\isasymand}\ {\isacharparenleft}{\kern0pt}{\isasymforall}i{\isachardot}{\kern0pt}\ reg{\isacharunderscore}{\kern0pt}eval\ {\isacharparenleft}{\kern0pt}sols\ i{\isacharparenright}{\kern0pt}{\isacharparenright}{\kern0pt}{\isachardoublequoteclose}\isanewline
\ \ \ \ \isakeywordONE{by}\isamarkupfalse%
\ auto\isanewline
\ \ \isakeywordONE{let}\isamarkupfalse%
\ {\isacharquery}{\kern0pt}sys{\isacharprime}{\kern0pt}\ {\isacharequal}{\kern0pt}\ {\isachardoublequoteopen}subst{\isacharunderscore}{\kern0pt}sys\ sols\ sys{\isachardoublequoteclose}\isanewline
\ \ \isakeywordONE{from}\isamarkupfalse%
\ eqs{\isacharunderscore}{\kern0pt}reg\ Suc{\isachardot}{\kern0pt}prems\ \isakeywordONE{have}\isamarkupfalse%
\ {\isachardoublequoteopen}reg{\isacharunderscore}{\kern0pt}eval\ {\isacharparenleft}{\kern0pt}sys\ {\isacharbang}{\kern0pt}\ r{\isacharparenright}{\kern0pt}{\isachardoublequoteclose}\ \isakeywordONE{by}\isamarkupfalse%
\ simp\isanewline
\ \ \isakeywordONE{with}\isamarkupfalse%
\ sols{\isacharunderscore}{\kern0pt}intro\ Suc{\isachardot}{\kern0pt}prems\ \isakeywordONE{have}\isamarkupfalse%
\ sys{\isacharunderscore}{\kern0pt}r{\isacharunderscore}{\kern0pt}reg{\isacharcolon}{\kern0pt}\ {\isachardoublequoteopen}reg{\isacharunderscore}{\kern0pt}eval\ {\isacharparenleft}{\kern0pt}{\isacharquery}{\kern0pt}sys{\isacharprime}{\kern0pt}\ {\isacharbang}{\kern0pt}\ r{\isacharparenright}{\kern0pt}{\isachardoublequoteclose}\isanewline
\ \ \ \ \isakeywordONE{using}\isamarkupfalse%
\ subst{\isacharunderscore}{\kern0pt}reg{\isacharunderscore}{\kern0pt}eval{\isacharbrackleft}{\kern0pt}of\ {\isachardoublequoteopen}sys\ {\isacharbang}{\kern0pt}\ r{\isachardoublequoteclose}{\isacharbrackright}{\kern0pt}\ subst{\isacharunderscore}{\kern0pt}sys{\isacharunderscore}{\kern0pt}subst{\isacharbrackleft}{\kern0pt}of\ r\ sys{\isacharbrackright}{\kern0pt}\ \isakeywordONE{by}\isamarkupfalse%
\ simp\isanewline
\ \ \isakeywordONE{then}\isamarkupfalse%
\ \isakeywordTHREE{obtain}\isamarkupfalse%
\ sol{\isacharunderscore}{\kern0pt}r\ \isakeywordTWO{where}\ sol{\isacharunderscore}{\kern0pt}r{\isacharunderscore}{\kern0pt}intro{\isacharcolon}{\kern0pt}\isanewline
\ \ \ \ {\isachardoublequoteopen}reg{\isacharunderscore}{\kern0pt}eval\ sol{\isacharunderscore}{\kern0pt}r\ {\isasymand}\ partial{\isacharunderscore}{\kern0pt}min{\isacharunderscore}{\kern0pt}sol{\isacharunderscore}{\kern0pt}one{\isacharunderscore}{\kern0pt}ineq\ r\ {\isacharparenleft}{\kern0pt}{\isacharquery}{\kern0pt}sys{\isacharprime}{\kern0pt}\ {\isacharbang}{\kern0pt}\ r{\isacharparenright}{\kern0pt}\ sol{\isacharunderscore}{\kern0pt}r{\isachardoublequoteclose}\isanewline
\ \ \ \ \isakeywordONE{using}\isamarkupfalse%
\ exists{\isacharunderscore}{\kern0pt}minimal{\isacharunderscore}{\kern0pt}reg{\isacharunderscore}{\kern0pt}sol\ \isakeywordONE{by}\isamarkupfalse%
\ blast\isanewline
\ \ \isakeywordONE{with}\isamarkupfalse%
\ Suc\ sols{\isacharunderscore}{\kern0pt}intro\ sys{\isacharunderscore}{\kern0pt}valid\ eqs{\isacharunderscore}{\kern0pt}reg\ \isakeywordONE{have}\isamarkupfalse%
\ {\isachardoublequoteopen}min{\isacharunderscore}{\kern0pt}sol{\isacharunderscore}{\kern0pt}induction{\isacharunderscore}{\kern0pt}step\ r\ sys\ sols\ sol{\isacharunderscore}{\kern0pt}r{\isachardoublequoteclose}\isanewline
\ \ \ \ \isakeywordONE{unfolding}\isamarkupfalse%
\ min{\isacharunderscore}{\kern0pt}sol{\isacharunderscore}{\kern0pt}induction{\isacharunderscore}{\kern0pt}step{\isacharunderscore}{\kern0pt}def\ \isakeywordONE{by}\isamarkupfalse%
\ force\isanewline
\ \ \isakeywordONE{from}\isamarkupfalse%
\ min{\isacharunderscore}{\kern0pt}sol{\isacharunderscore}{\kern0pt}induction{\isacharunderscore}{\kern0pt}step{\isachardot}{\kern0pt}exists{\isacharunderscore}{\kern0pt}min{\isacharunderscore}{\kern0pt}sol{\isacharunderscore}{\kern0pt}Suc{\isacharunderscore}{\kern0pt}r{\isacharbrackleft}{\kern0pt}OF\ this{\isacharbrackright}{\kern0pt}\ \isakeywordTHREE{show}\isamarkupfalse%
\ {\isacharquery}{\kern0pt}case\ \isakeywordONE{by}\isamarkupfalse%
\ blast\isanewline
\isakeywordONE{qed}\isamarkupfalse%
%
\endisatagproof
{\isafoldproof}%
%
\isadelimproof
\isanewline
%
\endisadelimproof
\isanewline
\isakeywordONE{lemma}\isamarkupfalse%
\ exists{\isacharunderscore}{\kern0pt}minimal{\isacharunderscore}{\kern0pt}reg{\isacharunderscore}{\kern0pt}sol{\isacharunderscore}{\kern0pt}sys{\isacharcolon}{\kern0pt}\isanewline
\ \ \isakeywordTWO{assumes}\ eqs{\isacharunderscore}{\kern0pt}reg{\isacharcolon}{\kern0pt}\ \ \ {\isachardoublequoteopen}{\isasymforall}eq\ {\isasymin}\ set\ sys{\isachardot}{\kern0pt}\ reg{\isacharunderscore}{\kern0pt}eval\ eq{\isachardoublequoteclose}\isanewline
\ \ \ \ \ \ \isakeywordTWO{and}\ sys{\isacharunderscore}{\kern0pt}valid{\isacharcolon}{\kern0pt}\ {\isachardoublequoteopen}{\isasymforall}eq\ {\isasymin}\ set\ sys{\isachardot}{\kern0pt}\ {\isasymforall}x\ {\isasymin}\ vars\ eq{\isachardot}{\kern0pt}\ x\ {\isacharless}{\kern0pt}\ length\ sys{\isachardoublequoteclose}\isanewline
\ \ \ \ \isakeywordTWO{shows}\ \ \ \ \ \ \ \ \ \ \ \ {\isachardoublequoteopen}{\isasymexists}sols{\isachardot}{\kern0pt}\ min{\isacharunderscore}{\kern0pt}sol{\isacharunderscore}{\kern0pt}ineq{\isacharunderscore}{\kern0pt}sys{\isacharunderscore}{\kern0pt}comm\ sys\ sols\ {\isasymand}\ {\isacharparenleft}{\kern0pt}{\isasymforall}i{\isachardot}{\kern0pt}\ regular{\isacharunderscore}{\kern0pt}lang\ {\isacharparenleft}{\kern0pt}sols\ i{\isacharparenright}{\kern0pt}{\isacharparenright}{\kern0pt}{\isachardoublequoteclose}\isanewline
%
\isadelimproof
%
\endisadelimproof
%
\isatagproof
\isakeywordONE{proof}\isamarkupfalse%
\ {\isacharminus}{\kern0pt}\isanewline
\ \ \isakeywordONE{from}\isamarkupfalse%
\ eqs{\isacharunderscore}{\kern0pt}reg\ sys{\isacharunderscore}{\kern0pt}valid\ \isakeywordONE{have}\isamarkupfalse%
\isanewline
\ \ \ \ {\isachardoublequoteopen}{\isasymexists}sols{\isachardot}{\kern0pt}\ partial{\isacharunderscore}{\kern0pt}min{\isacharunderscore}{\kern0pt}sol{\isacharunderscore}{\kern0pt}ineq{\isacharunderscore}{\kern0pt}sys\ {\isacharparenleft}{\kern0pt}length\ sys{\isacharparenright}{\kern0pt}\ sys\ sols\ {\isasymand}\ {\isacharparenleft}{\kern0pt}{\isasymforall}i{\isachardot}{\kern0pt}\ reg{\isacharunderscore}{\kern0pt}eval\ {\isacharparenleft}{\kern0pt}sols\ i{\isacharparenright}{\kern0pt}{\isacharparenright}{\kern0pt}{\isachardoublequoteclose}\isanewline
\ \ \ \ \isakeywordONE{using}\isamarkupfalse%
\ exists{\isacharunderscore}{\kern0pt}minimal{\isacharunderscore}{\kern0pt}reg{\isacharunderscore}{\kern0pt}sol{\isacharunderscore}{\kern0pt}sys{\isacharunderscore}{\kern0pt}aux\ \isakeywordONE{by}\isamarkupfalse%
\ blast\isanewline
\ \ \isakeywordONE{then}\isamarkupfalse%
\ \isakeywordTHREE{obtain}\isamarkupfalse%
\ sols\ \isakeywordTWO{where}\isanewline
\ \ \ \ sols{\isacharunderscore}{\kern0pt}intro{\isacharcolon}{\kern0pt}\ {\isachardoublequoteopen}partial{\isacharunderscore}{\kern0pt}min{\isacharunderscore}{\kern0pt}sol{\isacharunderscore}{\kern0pt}ineq{\isacharunderscore}{\kern0pt}sys\ {\isacharparenleft}{\kern0pt}length\ sys{\isacharparenright}{\kern0pt}\ sys\ sols\ {\isasymand}\ {\isacharparenleft}{\kern0pt}{\isasymforall}i{\isachardot}{\kern0pt}\ reg{\isacharunderscore}{\kern0pt}eval\ {\isacharparenleft}{\kern0pt}sols\ i{\isacharparenright}{\kern0pt}{\isacharparenright}{\kern0pt}{\isachardoublequoteclose}\isanewline
\ \ \ \ \isakeywordONE{by}\isamarkupfalse%
\ blast\isanewline
\ \ \isakeywordONE{then}\isamarkupfalse%
\ \isakeywordONE{have}\isamarkupfalse%
\ {\isachardoublequoteopen}const{\isacharunderscore}{\kern0pt}rlexp\ {\isacharparenleft}{\kern0pt}sols\ i{\isacharparenright}{\kern0pt}{\isachardoublequoteclose}\ \isakeywordTWO{if}\ {\isachardoublequoteopen}i\ {\isacharless}{\kern0pt}\ length\ sys{\isachardoublequoteclose}\ \isakeywordTWO{for}\ i\isanewline
\ \ \ \ \isakeywordONE{using}\isamarkupfalse%
\ that\ \isakeywordONE{unfolding}\isamarkupfalse%
\ partial{\isacharunderscore}{\kern0pt}min{\isacharunderscore}{\kern0pt}sol{\isacharunderscore}{\kern0pt}ineq{\isacharunderscore}{\kern0pt}sys{\isacharunderscore}{\kern0pt}def\ \isakeywordONE{by}\isamarkupfalse%
\ {\isacharparenleft}{\kern0pt}meson\ equals{\isadigit{0}}I\ leD{\isacharparenright}{\kern0pt}\isanewline
\ \ \isakeywordONE{with}\isamarkupfalse%
\ sols{\isacharunderscore}{\kern0pt}intro\ \isakeywordONE{have}\isamarkupfalse%
\ {\isachardoublequoteopen}{\isasymexists}l{\isachardot}{\kern0pt}\ regular{\isacharunderscore}{\kern0pt}lang\ l\ {\isasymand}\ {\isacharparenleft}{\kern0pt}{\isasymforall}v{\isachardot}{\kern0pt}\ eval\ {\isacharparenleft}{\kern0pt}sols\ i{\isacharparenright}{\kern0pt}\ v\ {\isacharequal}{\kern0pt}\ l{\isacharparenright}{\kern0pt}{\isachardoublequoteclose}\ \isakeywordTWO{if}\ {\isachardoublequoteopen}i\ {\isacharless}{\kern0pt}\ length\ sys{\isachardoublequoteclose}\ \isakeywordTWO{for}\ i\isanewline
\ \ \ \ \isakeywordONE{using}\isamarkupfalse%
\ that\ const{\isacharunderscore}{\kern0pt}rlexp{\isacharunderscore}{\kern0pt}regular{\isacharunderscore}{\kern0pt}lang\ \isakeywordONE{by}\isamarkupfalse%
\ metis\isanewline
\ \ \isakeywordONE{then}\isamarkupfalse%
\ \isakeywordTHREE{obtain}\isamarkupfalse%
\ ls\ \isakeywordTWO{where}\ ls{\isacharunderscore}{\kern0pt}intro{\isacharcolon}{\kern0pt}\ {\isachardoublequoteopen}{\isasymforall}i\ {\isacharless}{\kern0pt}\ length\ sys{\isachardot}{\kern0pt}\ regular{\isacharunderscore}{\kern0pt}lang\ {\isacharparenleft}{\kern0pt}ls\ i{\isacharparenright}{\kern0pt}\ {\isasymand}\ {\isacharparenleft}{\kern0pt}{\isasymforall}v{\isachardot}{\kern0pt}\ eval\ {\isacharparenleft}{\kern0pt}sols\ i{\isacharparenright}{\kern0pt}\ v\ {\isacharequal}{\kern0pt}\ ls\ i{\isacharparenright}{\kern0pt}{\isachardoublequoteclose}\isanewline
\ \ \ \ \isakeywordONE{by}\isamarkupfalse%
\ metis\isanewline
\ \ \isakeywordONE{let}\isamarkupfalse%
\ {\isacharquery}{\kern0pt}ls{\isacharprime}{\kern0pt}\ {\isacharequal}{\kern0pt}\ {\isachardoublequoteopen}{\isasymlambda}i{\isachardot}{\kern0pt}\ if\ i\ {\isacharless}{\kern0pt}\ length\ sys\ then\ ls\ i\ else\ {\isacharbraceleft}{\kern0pt}{\isacharbraceright}{\kern0pt}{\isachardoublequoteclose}\isanewline
\ \ \isakeywordONE{from}\isamarkupfalse%
\ ls{\isacharunderscore}{\kern0pt}intro\ \isakeywordONE{have}\isamarkupfalse%
\ ls{\isacharprime}{\kern0pt}{\isacharunderscore}{\kern0pt}intro{\isacharcolon}{\kern0pt}\isanewline
\ \ \ \ {\isachardoublequoteopen}{\isacharparenleft}{\kern0pt}{\isasymforall}i\ {\isacharless}{\kern0pt}\ length\ sys{\isachardot}{\kern0pt}\ regular{\isacharunderscore}{\kern0pt}lang\ {\isacharparenleft}{\kern0pt}{\isacharquery}{\kern0pt}ls{\isacharprime}{\kern0pt}\ i{\isacharparenright}{\kern0pt}\ {\isasymand}\ {\isacharparenleft}{\kern0pt}{\isasymforall}v{\isachardot}{\kern0pt}\ eval\ {\isacharparenleft}{\kern0pt}sols\ i{\isacharparenright}{\kern0pt}\ v\ {\isacharequal}{\kern0pt}\ {\isacharquery}{\kern0pt}ls{\isacharprime}{\kern0pt}\ i{\isacharparenright}{\kern0pt}{\isacharparenright}{\kern0pt}\isanewline
\ \ \ \ \ {\isasymand}\ {\isacharparenleft}{\kern0pt}{\isasymforall}i\ {\isasymge}\ length\ sys{\isachardot}{\kern0pt}\ {\isacharquery}{\kern0pt}ls{\isacharprime}{\kern0pt}\ i\ {\isacharequal}{\kern0pt}\ {\isacharbraceleft}{\kern0pt}{\isacharbraceright}{\kern0pt}{\isacharparenright}{\kern0pt}{\isachardoublequoteclose}\ \isakeywordONE{by}\isamarkupfalse%
\ force\isanewline
\ \ \isakeywordONE{then}\isamarkupfalse%
\ \isakeywordONE{have}\isamarkupfalse%
\ ls{\isacharprime}{\kern0pt}{\isacharunderscore}{\kern0pt}regular{\isacharcolon}{\kern0pt}\ {\isachardoublequoteopen}regular{\isacharunderscore}{\kern0pt}lang\ {\isacharparenleft}{\kern0pt}{\isacharquery}{\kern0pt}ls{\isacharprime}{\kern0pt}\ i{\isacharparenright}{\kern0pt}{\isachardoublequoteclose}\ \isakeywordTWO{for}\ i\ \isakeywordONE{by}\isamarkupfalse%
\ {\isacharparenleft}{\kern0pt}meson\ lang{\isachardot}{\kern0pt}simps{\isacharparenleft}{\kern0pt}{\isadigit{1}}{\isacharparenright}{\kern0pt}{\isacharparenright}{\kern0pt}\isanewline
\ \ \isakeywordONE{from}\isamarkupfalse%
\ ls{\isacharprime}{\kern0pt}{\isacharunderscore}{\kern0pt}intro\ sols{\isacharunderscore}{\kern0pt}intro\ \isakeywordONE{have}\isamarkupfalse%
\ {\isachardoublequoteopen}solves{\isacharunderscore}{\kern0pt}ineq{\isacharunderscore}{\kern0pt}sys{\isacharunderscore}{\kern0pt}comm\ sys\ {\isacharquery}{\kern0pt}ls{\isacharprime}{\kern0pt}{\isachardoublequoteclose}\isanewline
\ \ \ \ \isakeywordONE{unfolding}\isamarkupfalse%
\ partial{\isacharunderscore}{\kern0pt}min{\isacharunderscore}{\kern0pt}sol{\isacharunderscore}{\kern0pt}ineq{\isacharunderscore}{\kern0pt}sys{\isacharunderscore}{\kern0pt}def\ solution{\isacharunderscore}{\kern0pt}ineq{\isacharunderscore}{\kern0pt}sys{\isacharunderscore}{\kern0pt}def\isanewline
\ \ \ \ \isakeywordONE{by}\isamarkupfalse%
\ {\isacharparenleft}{\kern0pt}smt\ {\isacharparenleft}{\kern0pt}verit{\isacharparenright}{\kern0pt}\ eval{\isachardot}{\kern0pt}simps{\isacharparenleft}{\kern0pt}{\isadigit{1}}{\isacharparenright}{\kern0pt}\ linorder{\isacharunderscore}{\kern0pt}not{\isacharunderscore}{\kern0pt}less\ nless{\isacharunderscore}{\kern0pt}le\ take{\isacharunderscore}{\kern0pt}all{\isacharunderscore}{\kern0pt}iff{\isacharparenright}{\kern0pt}\isanewline
\ \ \isakeywordONE{moreover}\isamarkupfalse%
\ \isakeywordONE{have}\isamarkupfalse%
\ {\isachardoublequoteopen}{\isasymforall}sol{\isacharprime}{\kern0pt}{\isachardot}{\kern0pt}\ solves{\isacharunderscore}{\kern0pt}ineq{\isacharunderscore}{\kern0pt}sys{\isacharunderscore}{\kern0pt}comm\ sys\ sol{\isacharprime}{\kern0pt}\ {\isasymlongrightarrow}\ {\isacharparenleft}{\kern0pt}{\isasymforall}x{\isachardot}{\kern0pt}\ {\isasymPsi}\ {\isacharparenleft}{\kern0pt}{\isacharquery}{\kern0pt}ls{\isacharprime}{\kern0pt}\ x{\isacharparenright}{\kern0pt}\ {\isasymsubseteq}\ {\isasymPsi}\ {\isacharparenleft}{\kern0pt}sol{\isacharprime}{\kern0pt}\ x{\isacharparenright}{\kern0pt}{\isacharparenright}{\kern0pt}{\isachardoublequoteclose}\isanewline
\ \ \isakeywordONE{proof}\isamarkupfalse%
\ {\isacharparenleft}{\kern0pt}rule\ allI{\isacharcomma}{\kern0pt}\ rule\ impI{\isacharparenright}{\kern0pt}\isanewline
\ \ \ \ \isakeywordTHREE{fix}\isamarkupfalse%
\ sol{\isacharprime}{\kern0pt}\ x\isanewline
\ \ \ \ \isakeywordTHREE{assume}\isamarkupfalse%
\ as{\isacharcolon}{\kern0pt}\ {\isachardoublequoteopen}solves{\isacharunderscore}{\kern0pt}ineq{\isacharunderscore}{\kern0pt}sys{\isacharunderscore}{\kern0pt}comm\ sys\ sol{\isacharprime}{\kern0pt}{\isachardoublequoteclose}\isanewline
\ \ \ \ \isakeywordONE{let}\isamarkupfalse%
\ {\isacharquery}{\kern0pt}sol{\isacharunderscore}{\kern0pt}rlexps\ {\isacharequal}{\kern0pt}\ {\isachardoublequoteopen}{\isasymlambda}i{\isachardot}{\kern0pt}\ Const\ {\isacharparenleft}{\kern0pt}sol{\isacharprime}{\kern0pt}\ i{\isacharparenright}{\kern0pt}{\isachardoublequoteclose}\isanewline
\ \ \ \ \isakeywordONE{from}\isamarkupfalse%
\ as\ \isakeywordONE{have}\isamarkupfalse%
\ {\isachardoublequoteopen}solves{\isacharunderscore}{\kern0pt}ineq{\isacharunderscore}{\kern0pt}sys{\isacharunderscore}{\kern0pt}comm\ {\isacharparenleft}{\kern0pt}take\ {\isacharparenleft}{\kern0pt}length\ sys{\isacharparenright}{\kern0pt}\ sys{\isacharparenright}{\kern0pt}\ sol{\isacharprime}{\kern0pt}{\isachardoublequoteclose}\ \isakeywordONE{by}\isamarkupfalse%
\ simp\isanewline
\ \ \ \ \isakeywordONE{moreover}\isamarkupfalse%
\ \isakeywordONE{have}\isamarkupfalse%
\ {\isachardoublequoteopen}sol{\isacharprime}{\kern0pt}\ x\ {\isacharequal}{\kern0pt}\ eval\ {\isacharparenleft}{\kern0pt}{\isacharquery}{\kern0pt}sol{\isacharunderscore}{\kern0pt}rlexps\ x{\isacharparenright}{\kern0pt}\ sol{\isacharprime}{\kern0pt}{\isachardoublequoteclose}\ \isakeywordTWO{for}\ x\ \isakeywordONE{by}\isamarkupfalse%
\ simp\isanewline
\ \ \ \ \isakeywordONE{ultimately}\isamarkupfalse%
\ \isakeywordTHREE{show}\isamarkupfalse%
\ {\isachardoublequoteopen}{\isasymforall}x{\isachardot}{\kern0pt}\ {\isasymPsi}\ {\isacharparenleft}{\kern0pt}{\isacharquery}{\kern0pt}ls{\isacharprime}{\kern0pt}\ x{\isacharparenright}{\kern0pt}\ {\isasymsubseteq}\ {\isasymPsi}\ {\isacharparenleft}{\kern0pt}sol{\isacharprime}{\kern0pt}\ x{\isacharparenright}{\kern0pt}{\isachardoublequoteclose}\isanewline
\ \ \ \ \ \ \isakeywordONE{using}\isamarkupfalse%
\ sols{\isacharunderscore}{\kern0pt}intro\ \isakeywordONE{unfolding}\isamarkupfalse%
\ partial{\isacharunderscore}{\kern0pt}min{\isacharunderscore}{\kern0pt}sol{\isacharunderscore}{\kern0pt}ineq{\isacharunderscore}{\kern0pt}sys{\isacharunderscore}{\kern0pt}def\isanewline
\ \ \ \ \ \ \isakeywordONE{by}\isamarkupfalse%
\ {\isacharparenleft}{\kern0pt}smt\ {\isacharparenleft}{\kern0pt}verit{\isacharparenright}{\kern0pt}\ empty{\isacharunderscore}{\kern0pt}subsetI\ eval{\isachardot}{\kern0pt}simps{\isacharparenleft}{\kern0pt}{\isadigit{1}}{\isacharparenright}{\kern0pt}\ ls{\isacharprime}{\kern0pt}{\isacharunderscore}{\kern0pt}intro\ parikh{\isacharunderscore}{\kern0pt}img{\isacharunderscore}{\kern0pt}mono{\isacharparenright}{\kern0pt}\isanewline
\ \ \isakeywordONE{qed}\isamarkupfalse%
\isanewline
\ \ \isakeywordONE{ultimately}\isamarkupfalse%
\ \isakeywordONE{have}\isamarkupfalse%
\ {\isachardoublequoteopen}min{\isacharunderscore}{\kern0pt}sol{\isacharunderscore}{\kern0pt}ineq{\isacharunderscore}{\kern0pt}sys{\isacharunderscore}{\kern0pt}comm\ sys\ {\isacharquery}{\kern0pt}ls{\isacharprime}{\kern0pt}{\isachardoublequoteclose}\ \isakeywordONE{unfolding}\isamarkupfalse%
\ min{\isacharunderscore}{\kern0pt}sol{\isacharunderscore}{\kern0pt}ineq{\isacharunderscore}{\kern0pt}sys{\isacharunderscore}{\kern0pt}comm{\isacharunderscore}{\kern0pt}def\ \isakeywordONE{by}\isamarkupfalse%
\ blast\isanewline
\ \ \isakeywordONE{with}\isamarkupfalse%
\ ls{\isacharprime}{\kern0pt}{\isacharunderscore}{\kern0pt}regular\ \isakeywordTHREE{show}\isamarkupfalse%
\ {\isacharquery}{\kern0pt}thesis\ \isakeywordONE{by}\isamarkupfalse%
\ blast\isanewline
\isakeywordONE{qed}\isamarkupfalse%
%
\endisatagproof
{\isafoldproof}%
%
\isadelimproof
%
\endisadelimproof
%
\isadelimdocument
%
\endisadelimdocument
%
\isatagdocument
%
\isamarkupsubsection{Parikh's theorem%
}
\isamarkuptrue%
%
\endisatagdocument
{\isafolddocument}%
%
\isadelimdocument
%
\endisadelimdocument
%
\begin{isamarkuptext}%
Finally we are able to prove Parikh's theorem, i.e.\ that to each context free language exists
a regular language with identical Parikh image:%
\end{isamarkuptext}\isamarkuptrue%
\isakeywordONE{theorem}\isamarkupfalse%
\ Parikh{\isacharcolon}{\kern0pt}\isanewline
\ \ \isakeywordTWO{assumes}\ {\isachardoublequoteopen}CFL\ {\isacharparenleft}{\kern0pt}TYPE{\isacharparenleft}{\kern0pt}{\isacharprime}{\kern0pt}n{\isacharparenright}{\kern0pt}{\isacharparenright}{\kern0pt}\ L{\isachardoublequoteclose}\isanewline
\ \ \isakeywordTWO{shows}\ \ \ {\isachardoublequoteopen}{\isasymexists}L{\isacharprime}{\kern0pt}{\isachardot}{\kern0pt}\ regular{\isacharunderscore}{\kern0pt}lang\ L{\isacharprime}{\kern0pt}\ {\isasymand}\ {\isasymPsi}\ L\ {\isacharequal}{\kern0pt}\ {\isasymPsi}\ L{\isacharprime}{\kern0pt}{\isachardoublequoteclose}\isanewline
%
\isadelimproof
%
\endisadelimproof
%
\isatagproof
\isakeywordONE{proof}\isamarkupfalse%
\ {\isacharminus}{\kern0pt}\isanewline
\ \ \isakeywordONE{from}\isamarkupfalse%
\ assms\ \isakeywordTHREE{obtain}\isamarkupfalse%
\ P\ \isakeywordTWO{and}\ S{\isacharcolon}{\kern0pt}{\isacharcolon}{\kern0pt}{\isacharprime}{\kern0pt}n\ \isakeywordTWO{where}\ {\isacharasterisk}{\kern0pt}{\isacharcolon}{\kern0pt}\ {\isachardoublequoteopen}L\ {\isacharequal}{\kern0pt}\ Lang\ P\ S\ {\isasymand}\ finite\ P{\isachardoublequoteclose}\ \isakeywordONE{unfolding}\isamarkupfalse%
\ CFL{\isacharunderscore}{\kern0pt}def\ \isakeywordONE{by}\isamarkupfalse%
\ blast\isanewline
\ \ \isakeywordTHREE{show}\isamarkupfalse%
\ {\isacharquery}{\kern0pt}thesis\isanewline
\ \ \isakeywordONE{proof}\isamarkupfalse%
\ {\isacharparenleft}{\kern0pt}cases\ {\isachardoublequoteopen}S\ {\isasymin}\ Nts\ P{\isachardoublequoteclose}{\isacharparenright}{\kern0pt}\isanewline
\ \ \ \ \isakeywordTHREE{case}\isamarkupfalse%
\ True\isanewline
\ \ \ \ \isakeywordONE{from}\isamarkupfalse%
\ {\isacharasterisk}{\kern0pt}\ finite{\isacharunderscore}{\kern0pt}Nts\ exists{\isacharunderscore}{\kern0pt}bij{\isacharunderscore}{\kern0pt}Nt{\isacharunderscore}{\kern0pt}Var\ \isakeywordTHREE{obtain}\isamarkupfalse%
\ {\isasymgamma}\ {\isasymgamma}{\isacharprime}{\kern0pt}\ \isakeywordTWO{where}\ {\isacharasterisk}{\kern0pt}{\isacharasterisk}{\kern0pt}{\isacharcolon}{\kern0pt}\ {\isachardoublequoteopen}bij{\isacharunderscore}{\kern0pt}Nt{\isacharunderscore}{\kern0pt}Var\ {\isacharparenleft}{\kern0pt}Nts\ P{\isacharparenright}{\kern0pt}\ {\isasymgamma}\ {\isasymgamma}{\isacharprime}{\kern0pt}{\isachardoublequoteclose}\ \isakeywordONE{by}\isamarkupfalse%
\ metis\isanewline
\ \ \ \ \isakeywordONE{let}\isamarkupfalse%
\ {\isacharquery}{\kern0pt}sol\ {\isacharequal}{\kern0pt}\ {\isachardoublequoteopen}{\isasymlambda}i{\isachardot}{\kern0pt}\ if\ i\ {\isacharless}{\kern0pt}\ card\ {\isacharparenleft}{\kern0pt}Nts\ P{\isacharparenright}{\kern0pt}\ then\ Lang{\isacharunderscore}{\kern0pt}lfp\ P\ {\isacharparenleft}{\kern0pt}{\isasymgamma}\ i{\isacharparenright}{\kern0pt}\ else\ {\isacharbraceleft}{\kern0pt}{\isacharbraceright}{\kern0pt}{\isachardoublequoteclose}\isanewline
\ \ \ \ \isakeywordONE{from}\isamarkupfalse%
\ {\isacharasterisk}{\kern0pt}{\isacharasterisk}{\kern0pt}\ True\ \isakeywordONE{have}\isamarkupfalse%
\ {\isachardoublequoteopen}{\isasymgamma}{\isacharprime}{\kern0pt}\ S\ {\isacharless}{\kern0pt}\ card\ {\isacharparenleft}{\kern0pt}Nts\ P{\isacharparenright}{\kern0pt}{\isachardoublequoteclose}\ {\isachardoublequoteopen}{\isasymgamma}\ {\isacharparenleft}{\kern0pt}{\isasymgamma}{\isacharprime}{\kern0pt}\ S{\isacharparenright}{\kern0pt}\ {\isacharequal}{\kern0pt}\ S{\isachardoublequoteclose}\isanewline
\ \ \ \ \ \ \isakeywordONE{unfolding}\isamarkupfalse%
\ bij{\isacharunderscore}{\kern0pt}Nt{\isacharunderscore}{\kern0pt}Var{\isacharunderscore}{\kern0pt}def\ bij{\isacharunderscore}{\kern0pt}betw{\isacharunderscore}{\kern0pt}def\ \isakeywordONE{by}\isamarkupfalse%
\ auto\isanewline
\ \ \ \ \isakeywordONE{with}\isamarkupfalse%
\ Lang{\isacharunderscore}{\kern0pt}lfp{\isacharunderscore}{\kern0pt}eq{\isacharunderscore}{\kern0pt}Lang\ \isakeywordONE{have}\isamarkupfalse%
\ {\isacharasterisk}{\kern0pt}{\isacharasterisk}{\kern0pt}{\isacharasterisk}{\kern0pt}{\isacharcolon}{\kern0pt}\ {\isachardoublequoteopen}Lang\ P\ S\ {\isacharequal}{\kern0pt}\ {\isacharquery}{\kern0pt}sol\ {\isacharparenleft}{\kern0pt}{\isasymgamma}{\isacharprime}{\kern0pt}\ S{\isacharparenright}{\kern0pt}{\isachardoublequoteclose}\ \isakeywordONE{by}\isamarkupfalse%
\ metis\isanewline
\ \ \ \ \isakeywordONE{from}\isamarkupfalse%
\ {\isacharasterisk}{\kern0pt}\ {\isacharasterisk}{\kern0pt}{\isacharasterisk}{\kern0pt}\ CFG{\isacharunderscore}{\kern0pt}eq{\isacharunderscore}{\kern0pt}sys{\isachardot}{\kern0pt}CFL{\isacharunderscore}{\kern0pt}is{\isacharunderscore}{\kern0pt}min{\isacharunderscore}{\kern0pt}sol\ \isakeywordTHREE{obtain}\isamarkupfalse%
\ sys\isanewline
\ \ \ \ \ \ \isakeywordTWO{where}\ sys{\isacharunderscore}{\kern0pt}intro{\isacharcolon}{\kern0pt}\ {\isachardoublequoteopen}{\isacharparenleft}{\kern0pt}{\isasymforall}eq\ {\isasymin}\ set\ sys{\isachardot}{\kern0pt}\ reg{\isacharunderscore}{\kern0pt}eval\ eq{\isacharparenright}{\kern0pt}\ {\isasymand}\ {\isacharparenleft}{\kern0pt}{\isasymforall}eq\ {\isasymin}\ set\ sys{\isachardot}{\kern0pt}\ {\isasymforall}x\ {\isasymin}\ vars\ eq{\isachardot}{\kern0pt}\ x\ {\isacharless}{\kern0pt}\ length\ sys{\isacharparenright}{\kern0pt}\isanewline
\ \ \ \ \ \ \ \ \ \ \ \ \ \ \ \ \ \ \ \ \ \ \ \ {\isasymand}\ min{\isacharunderscore}{\kern0pt}sol{\isacharunderscore}{\kern0pt}ineq{\isacharunderscore}{\kern0pt}sys\ sys\ {\isacharquery}{\kern0pt}sol{\isachardoublequoteclose}\isanewline
\ \ \ \ \ \ \isakeywordONE{unfolding}\isamarkupfalse%
\ CFG{\isacharunderscore}{\kern0pt}eq{\isacharunderscore}{\kern0pt}sys{\isacharunderscore}{\kern0pt}def\ \isakeywordONE{by}\isamarkupfalse%
\ blast\isanewline
\ \ \ \ \isakeywordONE{with}\isamarkupfalse%
\ min{\isacharunderscore}{\kern0pt}sol{\isacharunderscore}{\kern0pt}min{\isacharunderscore}{\kern0pt}sol{\isacharunderscore}{\kern0pt}comm\ \isakeywordONE{have}\isamarkupfalse%
\ sol{\isacharunderscore}{\kern0pt}is{\isacharunderscore}{\kern0pt}min{\isacharunderscore}{\kern0pt}sol{\isacharcolon}{\kern0pt}\ {\isachardoublequoteopen}min{\isacharunderscore}{\kern0pt}sol{\isacharunderscore}{\kern0pt}ineq{\isacharunderscore}{\kern0pt}sys{\isacharunderscore}{\kern0pt}comm\ sys\ {\isacharquery}{\kern0pt}sol{\isachardoublequoteclose}\ \isakeywordONE{by}\isamarkupfalse%
\ fast\isanewline
\ \ \ \ \isakeywordONE{from}\isamarkupfalse%
\ sys{\isacharunderscore}{\kern0pt}intro\ exists{\isacharunderscore}{\kern0pt}minimal{\isacharunderscore}{\kern0pt}reg{\isacharunderscore}{\kern0pt}sol{\isacharunderscore}{\kern0pt}sys\ \isakeywordTHREE{obtain}\isamarkupfalse%
\ sol{\isacharprime}{\kern0pt}\ \isakeywordTWO{where}\isanewline
\ \ \ \ \ \ sol{\isacharprime}{\kern0pt}{\isacharunderscore}{\kern0pt}intro{\isacharcolon}{\kern0pt}\ {\isachardoublequoteopen}min{\isacharunderscore}{\kern0pt}sol{\isacharunderscore}{\kern0pt}ineq{\isacharunderscore}{\kern0pt}sys{\isacharunderscore}{\kern0pt}comm\ sys\ sol{\isacharprime}{\kern0pt}\ {\isasymand}\ regular{\isacharunderscore}{\kern0pt}lang\ {\isacharparenleft}{\kern0pt}sol{\isacharprime}{\kern0pt}\ {\isacharparenleft}{\kern0pt}{\isasymgamma}{\isacharprime}{\kern0pt}\ S{\isacharparenright}{\kern0pt}{\isacharparenright}{\kern0pt}{\isachardoublequoteclose}\ \isakeywordONE{by}\isamarkupfalse%
\ fastforce\isanewline
\ \ \ \ \isakeywordONE{with}\isamarkupfalse%
\ sol{\isacharunderscore}{\kern0pt}is{\isacharunderscore}{\kern0pt}min{\isacharunderscore}{\kern0pt}sol\ min{\isacharunderscore}{\kern0pt}sol{\isacharunderscore}{\kern0pt}comm{\isacharunderscore}{\kern0pt}unique\ \isakeywordONE{have}\isamarkupfalse%
\ {\isachardoublequoteopen}{\isasymPsi}\ {\isacharparenleft}{\kern0pt}{\isacharquery}{\kern0pt}sol\ {\isacharparenleft}{\kern0pt}{\isasymgamma}{\isacharprime}{\kern0pt}\ S{\isacharparenright}{\kern0pt}{\isacharparenright}{\kern0pt}\ {\isacharequal}{\kern0pt}\ {\isasymPsi}\ {\isacharparenleft}{\kern0pt}sol{\isacharprime}{\kern0pt}\ {\isacharparenleft}{\kern0pt}{\isasymgamma}{\isacharprime}{\kern0pt}\ S{\isacharparenright}{\kern0pt}{\isacharparenright}{\kern0pt}{\isachardoublequoteclose}\isanewline
\ \ \ \ \ \ \isakeywordONE{by}\isamarkupfalse%
\ blast\isanewline
\ \ \ \ \isakeywordONE{with}\isamarkupfalse%
\ {\isacharasterisk}{\kern0pt}\ {\isacharasterisk}{\kern0pt}{\isacharasterisk}{\kern0pt}{\isacharasterisk}{\kern0pt}\ sol{\isacharprime}{\kern0pt}{\isacharunderscore}{\kern0pt}intro\ \isakeywordTHREE{show}\isamarkupfalse%
\ {\isacharquery}{\kern0pt}thesis\ \isakeywordONE{by}\isamarkupfalse%
\ auto\isanewline
\ \ \isakeywordONE{next}\isamarkupfalse%
\isanewline
\ \ \ \ \isakeywordTHREE{case}\isamarkupfalse%
\ False\isanewline
\ \ \ \ \isakeywordONE{with}\isamarkupfalse%
\ Nts{\isacharunderscore}{\kern0pt}Lhss{\isacharunderscore}{\kern0pt}Rhs{\isacharunderscore}{\kern0pt}Nts\ \isakeywordONE{have}\isamarkupfalse%
\ {\isachardoublequoteopen}S\ {\isasymnotin}\ Lhss\ P{\isachardoublequoteclose}\ \isakeywordONE{by}\isamarkupfalse%
\ fast\isanewline
\ \ \ \ \isakeywordONE{from}\isamarkupfalse%
\ Lang{\isacharunderscore}{\kern0pt}empty{\isacharunderscore}{\kern0pt}if{\isacharunderscore}{\kern0pt}notin{\isacharunderscore}{\kern0pt}Lhss{\isacharbrackleft}{\kern0pt}OF\ this{\isacharbrackright}{\kern0pt}\ {\isacharasterisk}{\kern0pt}\ \isakeywordTHREE{show}\isamarkupfalse%
\ {\isacharquery}{\kern0pt}thesis\ \isakeywordONE{by}\isamarkupfalse%
\ {\isacharparenleft}{\kern0pt}metis\ lang{\isachardot}{\kern0pt}simps{\isacharparenleft}{\kern0pt}{\isadigit{1}}{\isacharparenright}{\kern0pt}{\isacharparenright}{\kern0pt}\isanewline
\ \ \isakeywordONE{qed}\isamarkupfalse%
\isanewline
\isakeywordONE{qed}\isamarkupfalse%
%
\endisatagproof
{\isafoldproof}%
%
\isadelimproof
\isanewline
%
\endisadelimproof
%
\isadelimtheory
\isanewline
%
\endisadelimtheory
%
\isatagtheory
\isakeywordTWO{end}\isamarkupfalse%
%
\endisatagtheory
{\isafoldtheory}%
%
\isadelimtheory
%
\endisadelimtheory
%
\end{isabellebody}%
\endinput
%:%file=~/studium/semester_7/semantik/homeworks/AIST/Parikh/Pilling.thy%:%
%:%11=3%:%
%:%27=5%:%
%:%28=5%:%
%:%29=6%:%
%:%30=7%:%
%:%31=8%:%
%:%40=11%:%
%:%41=12%:%
%:%42=13%:%
%:%43=14%:%
%:%44=15%:%
%:%45=16%:%
%:%46=17%:%
%:%47=18%:%
%:%48=19%:%
%:%49=20%:%
%:%50=21%:%
%:%59=24%:%
%:%71=26%:%
%:%72=27%:%
%:%73=28%:%
%:%74=29%:%
%:%75=30%:%
%:%77=31%:%
%:%78=31%:%
%:%79=32%:%
%:%82=36%:%
%:%83=37%:%
%:%84=38%:%
%:%85=39%:%
%:%86=40%:%
%:%88=41%:%
%:%89=41%:%
%:%92=42%:%
%:%96=42%:%
%:%97=42%:%
%:%98=42%:%
%:%103=42%:%
%:%106=43%:%
%:%107=44%:%
%:%108=45%:%
%:%109=45%:%
%:%110=46%:%
%:%117=47%:%
%:%118=47%:%
%:%119=48%:%
%:%120=48%:%
%:%121=49%:%
%:%122=49%:%
%:%123=50%:%
%:%124=50%:%
%:%125=50%:%
%:%126=51%:%
%:%127=51%:%
%:%128=51%:%
%:%129=51%:%
%:%130=52%:%
%:%131=52%:%
%:%132=52%:%
%:%133=52%:%
%:%134=52%:%
%:%135=53%:%
%:%136=53%:%
%:%137=53%:%
%:%138=53%:%
%:%139=53%:%
%:%140=54%:%
%:%141=54%:%
%:%142=54%:%
%:%143=54%:%
%:%144=55%:%
%:%145=55%:%
%:%146=56%:%
%:%147=56%:%
%:%148=57%:%
%:%149=57%:%
%:%150=58%:%
%:%151=58%:%
%:%152=58%:%
%:%153=59%:%
%:%154=59%:%
%:%155=59%:%
%:%156=59%:%
%:%157=60%:%
%:%158=60%:%
%:%159=60%:%
%:%160=60%:%
%:%161=60%:%
%:%162=61%:%
%:%163=61%:%
%:%164=61%:%
%:%165=61%:%
%:%166=62%:%
%:%167=62%:%
%:%168=62%:%
%:%169=62%:%
%:%170=63%:%
%:%176=63%:%
%:%179=64%:%
%:%180=65%:%
%:%181=65%:%
%:%182=66%:%
%:%183=67%:%
%:%184=68%:%
%:%191=69%:%
%:%192=69%:%
%:%193=70%:%
%:%194=70%:%
%:%195=71%:%
%:%196=71%:%
%:%197=72%:%
%:%198=72%:%
%:%199=72%:%
%:%200=72%:%
%:%201=73%:%
%:%202=73%:%
%:%203=73%:%
%:%204=73%:%
%:%205=74%:%
%:%206=74%:%
%:%207=74%:%
%:%208=74%:%
%:%209=75%:%
%:%210=75%:%
%:%211=75%:%
%:%212=75%:%
%:%213=76%:%
%:%219=76%:%
%:%222=77%:%
%:%223=78%:%
%:%224=78%:%
%:%225=79%:%
%:%226=80%:%
%:%227=81%:%
%:%228=82%:%
%:%229=83%:%
%:%230=84%:%
%:%237=85%:%
%:%238=85%:%
%:%239=86%:%
%:%240=86%:%
%:%241=86%:%
%:%242=87%:%
%:%243=88%:%
%:%244=89%:%
%:%245=90%:%
%:%246=90%:%
%:%247=91%:%
%:%248=91%:%
%:%249=91%:%
%:%250=92%:%
%:%251=93%:%
%:%252=94%:%
%:%253=94%:%
%:%254=94%:%
%:%255=95%:%
%:%256=95%:%
%:%257=96%:%
%:%258=96%:%
%:%259=96%:%
%:%260=96%:%
%:%261=96%:%
%:%262=97%:%
%:%263=97%:%
%:%264=97%:%
%:%265=98%:%
%:%266=98%:%
%:%267=99%:%
%:%268=99%:%
%:%269=100%:%
%:%270=100%:%
%:%271=100%:%
%:%272=101%:%
%:%273=101%:%
%:%274=101%:%
%:%275=102%:%
%:%276=102%:%
%:%277=102%:%
%:%278=102%:%
%:%279=103%:%
%:%285=103%:%
%:%288=104%:%
%:%289=105%:%
%:%290=105%:%
%:%291=106%:%
%:%292=107%:%
%:%293=108%:%
%:%294=109%:%
%:%295=110%:%
%:%296=111%:%
%:%303=112%:%
%:%304=112%:%
%:%305=113%:%
%:%306=113%:%
%:%307=113%:%
%:%308=114%:%
%:%309=115%:%
%:%310=116%:%
%:%311=117%:%
%:%312=117%:%
%:%313=118%:%
%:%314=118%:%
%:%315=118%:%
%:%316=119%:%
%:%317=120%:%
%:%318=121%:%
%:%319=121%:%
%:%320=121%:%
%:%321=122%:%
%:%322=122%:%
%:%323=123%:%
%:%324=123%:%
%:%325=124%:%
%:%326=124%:%
%:%327=125%:%
%:%328=125%:%
%:%329=126%:%
%:%330=126%:%
%:%331=127%:%
%:%332=127%:%
%:%333=128%:%
%:%334=128%:%
%:%335=128%:%
%:%336=129%:%
%:%337=129%:%
%:%338=130%:%
%:%339=130%:%
%:%340=131%:%
%:%341=131%:%
%:%342=132%:%
%:%343=132%:%
%:%344=132%:%
%:%345=133%:%
%:%346=133%:%
%:%347=134%:%
%:%348=134%:%
%:%349=134%:%
%:%350=135%:%
%:%351=136%:%
%:%352=136%:%
%:%353=136%:%
%:%354=137%:%
%:%355=137%:%
%:%356=137%:%
%:%357=138%:%
%:%358=139%:%
%:%359=139%:%
%:%360=139%:%
%:%361=140%:%
%:%362=140%:%
%:%363=140%:%
%:%364=141%:%
%:%365=142%:%
%:%366=142%:%
%:%367=143%:%
%:%368=143%:%
%:%369=143%:%
%:%370=144%:%
%:%371=144%:%
%:%372=145%:%
%:%373=145%:%
%:%374=145%:%
%:%375=145%:%
%:%376=146%:%
%:%377=146%:%
%:%378=147%:%
%:%379=147%:%
%:%380=147%:%
%:%381=147%:%
%:%382=147%:%
%:%383=147%:%
%:%384=148%:%
%:%385=148%:%
%:%386=148%:%
%:%387=149%:%
%:%388=149%:%
%:%389=149%:%
%:%390=150%:%
%:%391=150%:%
%:%392=150%:%
%:%393=150%:%
%:%394=151%:%
%:%400=151%:%
%:%403=152%:%
%:%404=153%:%
%:%405=153%:%
%:%406=154%:%
%:%407=155%:%
%:%408=156%:%
%:%409=157%:%
%:%416=158%:%
%:%417=158%:%
%:%418=159%:%
%:%419=159%:%
%:%420=159%:%
%:%421=160%:%
%:%422=160%:%
%:%423=161%:%
%:%424=161%:%
%:%425=161%:%
%:%426=162%:%
%:%427=162%:%
%:%428=162%:%
%:%429=163%:%
%:%430=163%:%
%:%431=164%:%
%:%432=164%:%
%:%433=165%:%
%:%434=165%:%
%:%435=166%:%
%:%436=166%:%
%:%437=167%:%
%:%438=167%:%
%:%439=168%:%
%:%440=168%:%
%:%441=169%:%
%:%442=169%:%
%:%443=170%:%
%:%444=170%:%
%:%445=171%:%
%:%446=171%:%
%:%447=171%:%
%:%448=172%:%
%:%449=172%:%
%:%450=172%:%
%:%451=173%:%
%:%452=173%:%
%:%453=173%:%
%:%454=174%:%
%:%455=175%:%
%:%456=175%:%
%:%457=176%:%
%:%458=176%:%
%:%459=176%:%
%:%460=177%:%
%:%461=177%:%
%:%462=178%:%
%:%463=178%:%
%:%464=178%:%
%:%465=179%:%
%:%466=180%:%
%:%467=180%:%
%:%468=180%:%
%:%469=181%:%
%:%470=181%:%
%:%471=181%:%
%:%472=182%:%
%:%473=182%:%
%:%474=183%:%
%:%475=183%:%
%:%476=183%:%
%:%477=183%:%
%:%478=184%:%
%:%479=184%:%
%:%480=184%:%
%:%481=184%:%
%:%482=185%:%
%:%483=185%:%
%:%484=186%:%
%:%485=186%:%
%:%486=186%:%
%:%487=186%:%
%:%488=186%:%
%:%489=186%:%
%:%490=187%:%
%:%491=187%:%
%:%492=187%:%
%:%493=187%:%
%:%494=187%:%
%:%495=188%:%
%:%496=188%:%
%:%497=188%:%
%:%498=188%:%
%:%499=189%:%
%:%505=189%:%
%:%508=190%:%
%:%509=191%:%
%:%510=191%:%
%:%512=193%:%
%:%519=194%:%
%:%520=194%:%
%:%521=195%:%
%:%522=195%:%
%:%523=196%:%
%:%524=196%:%
%:%525=196%:%
%:%526=196%:%
%:%527=197%:%
%:%528=197%:%
%:%529=198%:%
%:%530=198%:%
%:%531=199%:%
%:%532=199%:%
%:%533=199%:%
%:%534=199%:%
%:%535=200%:%
%:%536=200%:%
%:%537=200%:%
%:%538=200%:%
%:%539=201%:%
%:%540=201%:%
%:%541=202%:%
%:%542=202%:%
%:%543=203%:%
%:%544=203%:%
%:%545=203%:%
%:%546=204%:%
%:%547=205%:%
%:%548=205%:%
%:%549=206%:%
%:%550=206%:%
%:%551=206%:%
%:%552=206%:%
%:%553=207%:%
%:%554=207%:%
%:%555=208%:%
%:%556=208%:%
%:%557=209%:%
%:%558=209%:%
%:%559=209%:%
%:%560=210%:%
%:%561=211%:%
%:%562=211%:%
%:%563=212%:%
%:%564=212%:%
%:%565=212%:%
%:%566=212%:%
%:%567=213%:%
%:%568=213%:%
%:%569=214%:%
%:%570=214%:%
%:%571=215%:%
%:%572=215%:%
%:%573=215%:%
%:%574=216%:%
%:%575=216%:%
%:%576=217%:%
%:%577=217%:%
%:%578=217%:%
%:%579=217%:%
%:%580=218%:%
%:%595=221%:%
%:%607=223%:%
%:%608=224%:%
%:%609=225%:%
%:%610=226%:%
%:%612=228%:%
%:%613=228%:%
%:%614=229%:%
%:%615=230%:%
%:%616=231%:%
%:%617=232%:%
%:%618=233%:%
%:%619=234%:%
%:%620=235%:%
%:%622=237%:%
%:%623=238%:%
%:%624=239%:%
%:%625=240%:%
%:%627=241%:%
%:%628=241%:%
%:%629=242%:%
%:%630=242%:%
%:%631=243%:%
%:%632=244%:%
%:%633=244%:%
%:%640=245%:%
%:%641=245%:%
%:%642=246%:%
%:%643=246%:%
%:%644=246%:%
%:%645=247%:%
%:%646=247%:%
%:%647=247%:%
%:%648=248%:%
%:%649=248%:%
%:%650=248%:%
%:%651=248%:%
%:%652=249%:%
%:%658=249%:%
%:%661=250%:%
%:%662=251%:%
%:%663=251%:%
%:%670=252%:%
%:%671=252%:%
%:%672=253%:%
%:%673=253%:%
%:%674=254%:%
%:%675=254%:%
%:%676=255%:%
%:%677=255%:%
%:%678=255%:%
%:%679=255%:%
%:%680=256%:%
%:%681=256%:%
%:%682=256%:%
%:%683=256%:%
%:%684=257%:%
%:%685=257%:%
%:%686=257%:%
%:%687=257%:%
%:%688=257%:%
%:%689=258%:%
%:%690=258%:%
%:%691=258%:%
%:%692=258%:%
%:%693=259%:%
%:%699=259%:%
%:%702=260%:%
%:%703=261%:%
%:%704=261%:%
%:%711=262%:%
%:%712=262%:%
%:%713=262%:%
%:%714=263%:%
%:%715=263%:%
%:%716=264%:%
%:%717=264%:%
%:%718=265%:%
%:%719=265%:%
%:%720=266%:%
%:%721=266%:%
%:%722=267%:%
%:%723=267%:%
%:%724=268%:%
%:%725=268%:%
%:%726=269%:%
%:%727=269%:%
%:%728=270%:%
%:%729=270%:%
%:%730=270%:%
%:%731=271%:%
%:%732=271%:%
%:%733=271%:%
%:%734=271%:%
%:%735=271%:%
%:%736=272%:%
%:%737=272%:%
%:%738=272%:%
%:%739=273%:%
%:%740=273%:%
%:%741=274%:%
%:%742=274%:%
%:%743=274%:%
%:%744=275%:%
%:%745=275%:%
%:%746=275%:%
%:%747=276%:%
%:%748=276%:%
%:%749=276%:%
%:%750=277%:%
%:%751=278%:%
%:%752=278%:%
%:%753=279%:%
%:%754=279%:%
%:%755=279%:%
%:%756=280%:%
%:%757=281%:%
%:%758=281%:%
%:%759=281%:%
%:%760=282%:%
%:%761=282%:%
%:%762=282%:%
%:%763=283%:%
%:%764=283%:%
%:%765=284%:%
%:%766=284%:%
%:%767=285%:%
%:%768=285%:%
%:%769=285%:%
%:%770=285%:%
%:%771=285%:%
%:%772=286%:%
%:%773=286%:%
%:%774=286%:%
%:%775=286%:%
%:%776=286%:%
%:%777=287%:%
%:%778=287%:%
%:%779=287%:%
%:%780=288%:%
%:%781=288%:%
%:%782=288%:%
%:%783=289%:%
%:%784=289%:%
%:%785=289%:%
%:%786=289%:%
%:%787=289%:%
%:%788=290%:%
%:%794=290%:%
%:%797=291%:%
%:%798=292%:%
%:%799=292%:%
%:%800=293%:%
%:%801=294%:%
%:%802=295%:%
%:%809=296%:%
%:%810=296%:%
%:%811=297%:%
%:%812=297%:%
%:%813=297%:%
%:%814=298%:%
%:%815=298%:%
%:%816=298%:%
%:%817=299%:%
%:%818=299%:%
%:%819=299%:%
%:%820=300%:%
%:%821=300%:%
%:%822=300%:%
%:%823=301%:%
%:%824=301%:%
%:%825=301%:%
%:%826=301%:%
%:%827=302%:%
%:%828=302%:%
%:%829=302%:%
%:%830=303%:%
%:%831=303%:%
%:%832=303%:%
%:%833=304%:%
%:%834=304%:%
%:%835=304%:%
%:%836=305%:%
%:%837=305%:%
%:%838=305%:%
%:%839=306%:%
%:%840=306%:%
%:%841=306%:%
%:%842=307%:%
%:%843=307%:%
%:%844=307%:%
%:%845=308%:%
%:%846=308%:%
%:%847=308%:%
%:%848=308%:%
%:%849=309%:%
%:%859=311%:%
%:%861=312%:%
%:%862=312%:%
%:%863=313%:%
%:%866=314%:%
%:%870=314%:%
%:%871=314%:%
%:%872=315%:%
%:%873=315%:%
%:%874=316%:%
%:%875=316%:%
%:%880=316%:%
%:%883=317%:%
%:%884=318%:%
%:%887=321%:%
%:%888=322%:%
%:%889=323%:%
%:%891=324%:%
%:%892=324%:%
%:%893=325%:%
%:%894=326%:%
%:%901=327%:%
%:%902=327%:%
%:%903=328%:%
%:%904=328%:%
%:%905=328%:%
%:%906=329%:%
%:%907=330%:%
%:%908=330%:%
%:%909=331%:%
%:%910=331%:%
%:%911=331%:%
%:%912=332%:%
%:%913=333%:%
%:%914=333%:%
%:%915=333%:%
%:%916=334%:%
%:%917=334%:%
%:%918=335%:%
%:%919=335%:%
%:%920=335%:%
%:%921=336%:%
%:%922=336%:%
%:%923=336%:%
%:%924=336%:%
%:%925=337%:%
%:%926=337%:%
%:%927=337%:%
%:%928=338%:%
%:%929=338%:%
%:%930=338%:%
%:%931=339%:%
%:%932=339%:%
%:%933=339%:%
%:%934=339%:%
%:%935=340%:%
%:%950=343%:%
%:%962=345%:%
%:%963=346%:%
%:%964=347%:%
%:%965=348%:%
%:%966=349%:%
%:%967=350%:%
%:%968=351%:%
%:%969=352%:%
%:%971=354%:%
%:%972=354%:%
%:%973=355%:%
%:%974=356%:%
%:%975=357%:%
%:%976=358%:%
%:%977=359%:%
%:%978=360%:%
%:%979=361%:%
%:%980=362%:%
%:%981=363%:%
%:%982=364%:%
%:%983=365%:%
%:%984=366%:%
%:%986=368%:%
%:%987=369%:%
%:%988=370%:%
%:%989=371%:%
%:%990=372%:%
%:%992=373%:%
%:%993=373%:%
%:%994=374%:%
%:%995=374%:%
%:%996=375%:%
%:%997=376%:%
%:%998=376%:%
%:%1001=377%:%
%:%1005=377%:%
%:%1006=377%:%
%:%1007=377%:%
%:%1008=377%:%
%:%1017=380%:%
%:%1018=381%:%
%:%1020=382%:%
%:%1021=382%:%
%:%1024=383%:%
%:%1028=383%:%
%:%1029=383%:%
%:%1030=383%:%
%:%1031=383%:%
%:%1036=383%:%
%:%1039=384%:%
%:%1040=385%:%
%:%1041=385%:%
%:%1048=386%:%
%:%1049=386%:%
%:%1050=386%:%
%:%1051=387%:%
%:%1052=387%:%
%:%1053=388%:%
%:%1054=388%:%
%:%1055=389%:%
%:%1056=389%:%
%:%1057=389%:%
%:%1058=389%:%
%:%1059=390%:%
%:%1060=390%:%
%:%1061=390%:%
%:%1062=391%:%
%:%1063=391%:%
%:%1064=391%:%
%:%1065=392%:%
%:%1066=392%:%
%:%1067=392%:%
%:%1068=393%:%
%:%1069=393%:%
%:%1070=393%:%
%:%1071=394%:%
%:%1072=394%:%
%:%1073=395%:%
%:%1074=395%:%
%:%1075=396%:%
%:%1076=396%:%
%:%1077=397%:%
%:%1078=397%:%
%:%1079=397%:%
%:%1080=398%:%
%:%1081=398%:%
%:%1082=399%:%
%:%1083=399%:%
%:%1084=399%:%
%:%1085=400%:%
%:%1086=400%:%
%:%1087=400%:%
%:%1088=401%:%
%:%1089=401%:%
%:%1090=401%:%
%:%1091=402%:%
%:%1092=402%:%
%:%1093=402%:%
%:%1094=403%:%
%:%1100=403%:%
%:%1103=404%:%
%:%1104=405%:%
%:%1105=405%:%
%:%1107=407%:%
%:%1114=408%:%
%:%1115=408%:%
%:%1116=409%:%
%:%1117=409%:%
%:%1118=410%:%
%:%1119=410%:%
%:%1120=411%:%
%:%1121=411%:%
%:%1122=411%:%
%:%1123=411%:%
%:%1124=411%:%
%:%1125=412%:%
%:%1126=412%:%
%:%1127=412%:%
%:%1128=413%:%
%:%1129=413%:%
%:%1130=413%:%
%:%1131=414%:%
%:%1132=414%:%
%:%1133=415%:%
%:%1134=415%:%
%:%1135=415%:%
%:%1136=416%:%
%:%1137=416%:%
%:%1138=417%:%
%:%1139=417%:%
%:%1140=417%:%
%:%1141=418%:%
%:%1142=418%:%
%:%1143=418%:%
%:%1144=419%:%
%:%1145=419%:%
%:%1146=419%:%
%:%1147=420%:%
%:%1148=420%:%
%:%1149=420%:%
%:%1150=420%:%
%:%1151=421%:%
%:%1152=421%:%
%:%1153=421%:%
%:%1154=422%:%
%:%1155=422%:%
%:%1156=422%:%
%:%1157=423%:%
%:%1158=423%:%
%:%1159=424%:%
%:%1160=424%:%
%:%1161=424%:%
%:%1162=424%:%
%:%1163=425%:%
%:%1164=425%:%
%:%1165=425%:%
%:%1166=426%:%
%:%1167=426%:%
%:%1168=426%:%
%:%1169=427%:%
%:%1175=427%:%
%:%1178=428%:%
%:%1179=429%:%
%:%1180=429%:%
%:%1183=430%:%
%:%1187=430%:%
%:%1188=430%:%
%:%1189=430%:%
%:%1190=430%:%
%:%1195=430%:%
%:%1198=431%:%
%:%1199=432%:%
%:%1200=432%:%
%:%1207=433%:%
%:%1208=433%:%
%:%1209=434%:%
%:%1210=434%:%
%:%1211=434%:%
%:%1212=435%:%
%:%1213=435%:%
%:%1214=435%:%
%:%1215=436%:%
%:%1216=436%:%
%:%1217=436%:%
%:%1218=437%:%
%:%1219=437%:%
%:%1220=437%:%
%:%1221=438%:%
%:%1222=438%:%
%:%1223=438%:%
%:%1224=439%:%
%:%1225=439%:%
%:%1226=439%:%
%:%1227=440%:%
%:%1228=440%:%
%:%1229=440%:%
%:%1230=441%:%
%:%1231=441%:%
%:%1232=441%:%
%:%1233=441%:%
%:%1234=442%:%
%:%1235=442%:%
%:%1236=442%:%
%:%1237=442%:%
%:%1238=443%:%
%:%1239=443%:%
%:%1240=443%:%
%:%1241=444%:%
%:%1242=444%:%
%:%1243=444%:%
%:%1244=445%:%
%:%1245=445%:%
%:%1246=445%:%
%:%1247=446%:%
%:%1248=446%:%
%:%1249=447%:%
%:%1250=447%:%
%:%1251=447%:%
%:%1252=448%:%
%:%1253=448%:%
%:%1254=448%:%
%:%1255=449%:%
%:%1256=449%:%
%:%1257=449%:%
%:%1258=450%:%
%:%1259=450%:%
%:%1260=450%:%
%:%1261=451%:%
%:%1262=451%:%
%:%1263=451%:%
%:%1264=451%:%
%:%1265=452%:%
%:%1275=455%:%
%:%1276=456%:%
%:%1277=457%:%
%:%1279=458%:%
%:%1280=458%:%
%:%1283=459%:%
%:%1287=459%:%
%:%1288=459%:%
%:%1289=460%:%
%:%1290=460%:%
%:%1291=461%:%
%:%1292=461%:%
%:%1297=461%:%
%:%1300=462%:%
%:%1301=463%:%
%:%1302=463%:%
%:%1303=464%:%
%:%1306=465%:%
%:%1310=465%:%
%:%1311=465%:%
%:%1312=465%:%
%:%1317=465%:%
%:%1320=466%:%
%:%1321=467%:%
%:%1324=470%:%
%:%1325=471%:%
%:%1326=472%:%
%:%1328=473%:%
%:%1329=473%:%
%:%1330=474%:%
%:%1331=475%:%
%:%1332=476%:%
%:%1333=477%:%
%:%1340=478%:%
%:%1341=478%:%
%:%1342=478%:%
%:%1343=479%:%
%:%1344=479%:%
%:%1345=480%:%
%:%1346=480%:%
%:%1347=481%:%
%:%1348=481%:%
%:%1349=481%:%
%:%1350=482%:%
%:%1351=482%:%
%:%1352=482%:%
%:%1353=482%:%
%:%1354=482%:%
%:%1355=483%:%
%:%1356=483%:%
%:%1357=484%:%
%:%1358=484%:%
%:%1359=485%:%
%:%1360=485%:%
%:%1361=485%:%
%:%1362=486%:%
%:%1363=486%:%
%:%1364=487%:%
%:%1365=487%:%
%:%1366=488%:%
%:%1367=488%:%
%:%1368=488%:%
%:%1369=488%:%
%:%1370=489%:%
%:%1371=489%:%
%:%1372=489%:%
%:%1373=490%:%
%:%1374=490%:%
%:%1375=490%:%
%:%1376=491%:%
%:%1377=491%:%
%:%1378=491%:%
%:%1379=492%:%
%:%1380=493%:%
%:%1381=493%:%
%:%1382=493%:%
%:%1383=494%:%
%:%1384=494%:%
%:%1385=494%:%
%:%1386=495%:%
%:%1387=495%:%
%:%1388=495%:%
%:%1389=496%:%
%:%1390=496%:%
%:%1391=496%:%
%:%1392=496%:%
%:%1393=497%:%
%:%1399=497%:%
%:%1402=498%:%
%:%1403=499%:%
%:%1404=499%:%
%:%1405=500%:%
%:%1406=501%:%
%:%1407=502%:%
%:%1414=503%:%
%:%1415=503%:%
%:%1416=504%:%
%:%1417=504%:%
%:%1418=504%:%
%:%1419=505%:%
%:%1420=506%:%
%:%1421=506%:%
%:%1422=506%:%
%:%1423=507%:%
%:%1424=507%:%
%:%1425=507%:%
%:%1426=508%:%
%:%1427=509%:%
%:%1428=509%:%
%:%1429=510%:%
%:%1430=510%:%
%:%1431=510%:%
%:%1432=511%:%
%:%1433=511%:%
%:%1434=511%:%
%:%1435=511%:%
%:%1436=512%:%
%:%1437=512%:%
%:%1438=512%:%
%:%1439=513%:%
%:%1440=513%:%
%:%1441=513%:%
%:%1442=514%:%
%:%1443=514%:%
%:%1444=514%:%
%:%1445=515%:%
%:%1446=515%:%
%:%1447=516%:%
%:%1448=516%:%
%:%1449=517%:%
%:%1450=517%:%
%:%1451=517%:%
%:%1452=518%:%
%:%1453=519%:%
%:%1454=519%:%
%:%1455=520%:%
%:%1456=520%:%
%:%1457=520%:%
%:%1458=520%:%
%:%1459=521%:%
%:%1460=521%:%
%:%1461=521%:%
%:%1462=522%:%
%:%1463=522%:%
%:%1464=523%:%
%:%1465=523%:%
%:%1466=524%:%
%:%1467=524%:%
%:%1468=524%:%
%:%1469=525%:%
%:%1470=525%:%
%:%1471=526%:%
%:%1472=526%:%
%:%1473=527%:%
%:%1474=527%:%
%:%1475=528%:%
%:%1476=528%:%
%:%1477=529%:%
%:%1478=529%:%
%:%1479=529%:%
%:%1480=529%:%
%:%1481=530%:%
%:%1482=530%:%
%:%1483=530%:%
%:%1484=530%:%
%:%1485=531%:%
%:%1486=531%:%
%:%1487=531%:%
%:%1488=532%:%
%:%1489=532%:%
%:%1490=532%:%
%:%1491=533%:%
%:%1492=533%:%
%:%1493=534%:%
%:%1494=534%:%
%:%1495=535%:%
%:%1496=535%:%
%:%1497=535%:%
%:%1498=535%:%
%:%1499=535%:%
%:%1500=536%:%
%:%1501=536%:%
%:%1502=536%:%
%:%1503=536%:%
%:%1504=537%:%
%:%1519=541%:%
%:%1531=543%:%
%:%1532=544%:%
%:%1534=545%:%
%:%1535=545%:%
%:%1536=546%:%
%:%1537=547%:%
%:%1544=548%:%
%:%1545=548%:%
%:%1546=549%:%
%:%1547=549%:%
%:%1548=549%:%
%:%1549=549%:%
%:%1550=549%:%
%:%1551=550%:%
%:%1552=550%:%
%:%1553=551%:%
%:%1554=551%:%
%:%1555=552%:%
%:%1556=552%:%
%:%1557=553%:%
%:%1558=553%:%
%:%1559=553%:%
%:%1560=553%:%
%:%1561=554%:%
%:%1562=554%:%
%:%1563=555%:%
%:%1564=555%:%
%:%1565=555%:%
%:%1566=556%:%
%:%1567=556%:%
%:%1568=556%:%
%:%1569=557%:%
%:%1570=557%:%
%:%1571=557%:%
%:%1572=557%:%
%:%1573=558%:%
%:%1574=558%:%
%:%1575=558%:%
%:%1576=559%:%
%:%1577=560%:%
%:%1578=561%:%
%:%1579=561%:%
%:%1580=561%:%
%:%1581=562%:%
%:%1582=562%:%
%:%1583=562%:%
%:%1584=562%:%
%:%1585=563%:%
%:%1586=563%:%
%:%1587=563%:%
%:%1588=564%:%
%:%1589=564%:%
%:%1590=565%:%
%:%1591=565%:%
%:%1592=565%:%
%:%1593=566%:%
%:%1594=566%:%
%:%1595=567%:%
%:%1596=567%:%
%:%1597=567%:%
%:%1598=567%:%
%:%1599=568%:%
%:%1600=568%:%
%:%1601=569%:%
%:%1602=569%:%
%:%1603=570%:%
%:%1604=570%:%
%:%1605=570%:%
%:%1606=570%:%
%:%1607=571%:%
%:%1608=571%:%
%:%1609=571%:%
%:%1610=571%:%
%:%1611=572%:%
%:%1612=572%:%
%:%1613=573%:%
%:%1619=573%:%
%:%1624=574%:%
%:%1629=575%:%



\bibliographystyle{abbrv}
\bibliography{root}

\end{document}

%%% Local Variables:
%%% mode: latex
%%% TeX-master: t
%%% End:
