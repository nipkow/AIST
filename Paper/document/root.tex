\documentclass[runningheads]{llncs}
\pagestyle{plain} % turn on page numbers
\usepackage[T1]{fontenc}
%\usepackage{graphicx}
% Used for displaying a sample figure. If possible, figure files should
% be included in EPS format.
%
% If you use the hyperref package, please uncomment the following two lines
% to display URLs in blue roman font according to Springer's eBook style:
%\usepackage{color}
%\renewcommand\UrlFont{\color{blue}\rmfamily}
%\urlstyle{rm}

\usepackage{mathpartir}
\usepackage{amsfonts}
\usepackage{amsmath,amssymb}

\usepackage{isabelle,isabellesym}
\usepackage{latexsym}
%because of orcidlink
\usepackage{xcolor}
% this should be the last package used
\usepackage{pdfsetup}
%clashes with pdfsetup
\usepackage{orcidlink}

\urlstyle{rm}
\isabellestyle{literal}

\renewcommand{\isadigit}[1]{{\rm #1}}

% isabelle keyword, adapted from isabelle.sty
\renewcommand{\isakeyword}[1]
{{\normalfont\sffamily\bfseries\def\isachardot{.}\def\isacharunderscore{\isacharunderscorekeyword}%
\def\isacharbraceleft{\{}\def\isacharbraceright{\}}#1}}

\newcommand{\noquotes}[1]{{\renewcommand{\isachardoublequote}{}\renewcommand{\isachardoublequoteopen}{}\renewcommand{\isachardoublequoteclose}{}#1}}

% no right margin in quote:
%\renewenvironment{quote}
%{\list{}{}%
%\item\relax}
%{\endlist}
% no margins in quote:
\renewenvironment{quote}%
{\list{}{\leftmargin=0mm\rightmargin=0mm}\item[]}%
{\endlist}

% for uniform font size
\renewcommand{\isastyle}{\isastyleminor}

\newcommand{\concept}[1]{\textbf{#1}}

\newcommand{\eqnum}[1]{{\upshape\refstepcounter{equation}\hfill(\theequation\label{#1})}}

\newlength{\funheadersep}
\setlength{\funheadersep}{\smallskipamount}

\hyphenation{Isa-belle}

\newcommand{\prestar}{$\mathit{pre}^*$}

\begin{document}
\title{A Unified Formalization of Context-Free Grammar Theory}
%\titlerunning{Abbreviated paper title}
% If the paper title is too long for the running head, you can set
% an abbreviated paper title here

\author{Tobias Nipkow\inst{1}\orcidlink{0000-0003-0730-515X} \and
Fabian Lehr\inst{2} \and
Moritz Roos\inst{3} \and
Akihisa Yamada\inst{4}\orcidlink{0000-0001-8872-2240}}
%
\authorrunning{T. Nipkow et al.}

\institute{Technical University of Munich, Germany
  \url{https://www.proof.cit.tum.de/~nipkow} \and
  Technical University of Munich, Germany
  \email{fabian.lehr@tum.de} \and
  Technical University of Munich, Germany
  \email{moritz.roos@tum.de} \and
AIST Tokyo, Japan \email{}
\url{https://akihisayamada.github.io/}}

\maketitle

\begin{abstract}
We present a formalized theory of context-free grammars and
their links to finite automata. In particular we focus on first-time
formalizations of an executable translation into Greibach Normal Form, the Chomsky-Sch\"utzenberger Representation Theorem and Parikh's Theorem.

\keywords{context-free grammar \and formal language  \and proof assistant \and Isabelle.}
\end{abstract}

%\begin{table}
%\caption{Table captions should be placed above the tables.}\label{tab1}
%\begin{tabular}{|l|l|l|}
%\end{tabular}
%\end{table}

%\begin{theorem}
%\end{theorem}
% the environments 'definition', 'lemma', 'proposition', 'corollary',
% 'remark', and 'example' are defined in the LLNCS documentclass as well.
%
%\begin{proof}
%\end{proof}

\input{Paper}

\begin{credits}
\subsubsection{\ackname}
We thank Alexander Haberl, Tassilo Lemke, August Martin Stimpfle, Kaan Taskin, Felipe Escallon
and Thomas Ammer for their proofs and Franz Regensburger for his comments on a draft version.
We are grateful to Hitoshi Ohsaki, AIST, and JST, CREST Grant Number JPMJCR22M1, Japan,
for supporting two research visits by the first author.
%\vspace{-1ex}
\subsubsection{\discintname}
%It is now necessary to declare any competing interests or to specifically
%state that the authors have no competing interests. Please place the
%statement with a bold run-in heading in small font size beneath the
%(optional) acknowledgments
%for example:
The authors have no competing interests to declare that are relevant to the content of this article.
\end{credits}

\bibliographystyle{splncs04}
\bibliography{root}

\end{document}
